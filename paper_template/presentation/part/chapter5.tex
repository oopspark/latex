\intro
{\chapterfive}

\slide
{\maintitle}
{\chapterfive}
{논콩 재배의 확장성}{
\vspace{10pt}

\begin{itemize}
    \item 쌀 이외의 주요 곡물을 논에서 재배하려면 \\배수시설 확충 등 생산 기반 확보가 필요함
    \item 농가가 논을 타작물 이용에 적합하게 개량할 유인이 필요함
    \item 콩은 국내 생산 및 유통 기반이 갖추어져 있음
    \item 따라서 논콩 생산 확대를 통해 논 체질 개선을 이룬다면 \\다른 곡물 생산 확대로 이어질 가능성이 있음
\end{itemize}

{\small ※ 일본은 23년부터  밭화(畑地化) 촉진조성사업을 진행하며}
\par
\vspace{-5pt}
\hspace{9pt}{\small '토지개량구 지구 제외 결제금' 지원의 항목을 넣음}
\par
\vspace{-5pt}
\hspace{9pt}{\small 이는 논 재배 대채작물 증산 뿐만 아니라 농지 개량 또한 목표에 두고 있음을 보여줌}
\par
\vspace{-5pt}
\hspace{9pt}{\small 참고 - 김기흥. 2023. 일본의 전략작물 관련 정책. 한국농촌경제연구원}
}


\slide
{\maintitle}
{\chapterfive}
{논콩 재배 농가의 경영안정}{
\begin{itemize}
    \item 논콩 생산 농가의 경우 재해 대비가 미흡할 경우 \\초기 수입 변동이 클 수 있음
    \item 안정적인 영농을 지원하기 위해 \\재해 예방 및 대응 지원이 필요함
    \item 현행 재해 보장 제도는 논콩 농가의 위험 회피에 미흡함 
    \item 재해 대응을 위해 논콩 농가, 보험회사, 정부 \\공동의 노력이 요구됨
\end{itemize}
}