% ─────────────────────────────────────────────
% 1) 글꼴 및 한글 설정 (XeLaTeX 전용)
% ─────────────────────────────────────────────
\usepackage{fontspec}
\usepackage{xeCJK}

% 기본 폰트 설정
\setmainfont{NanumMyeongjo}[
  BoldFont = NanumMyeongjoBold
]

\setCJKmainfont{Noto Serif CJK KR}

% ─────────────────────────────────────────────
% 2) 여백 및 문서 스타일
% ─────────────────────────────────────────────
\usepackage[a4paper,
  left=30mm,
  right=25mm,
  top=30mm,
  bottom=30mm,
  includeheadfoot]{geometry}

\usepackage{setspace}
\linespread{1.5}

\setlength{\parskip}{0pt}
\setlength{\parindent}{2em}

% ─────────────────────────────────────────────
% 3) 표 관련 패키지
% ─────────────────────────────────────────────
\usepackage{multirow}
\usepackage{hhline}
\usepackage{array}
\usepackage{booktabs}
\usepackage{makecell}
\usepackage[table]{xcolor}

\usepackage{tabularx}
\newcolumntype{Y}{>{\centering\arraybackslash}X} % tabularx 용 중앙 정렬 컬럼
\renewcommand{\arraystretch}{1.5}

% ─────────────────────────────────────────────
% 4) 그림/수식/헤더
% ─────────────────────────────────────────────
\usepackage{graphicx}
\usepackage{amsmath, amssymb}
\usepackage{fancyhdr}

% ─────────────────────────────────────────────
% 5) 참고문헌 (biblatex + biber)
% ─────────────────────────────────────────────
\usepackage[
  backend=biber,
  style=authoryear,
  language=korean
]{biblatex}

\addbibresource{data/references.bib}

% ─────────────────────────────────────────────
% 6) 챕터/섹션 제목 설정 (titlesec)
% ─────────────────────────────────────────────
\usepackage{titlesec}

\renewcommand{\chaptername}{제~\thechapter~장}

% Chapter 제목
\titleformat{\chapter}[hang]
  {\normalfont\Large\bfseries\centering}
  {\chaptername}
  {20pt}
  {}

% Section 제목
\renewcommand{\thesection}{\arabic{section}}
\titleformat{\section}
  {\normalfont\Large\bfseries}
  {제~\thesection~절}
  {1em}
  {}

% Subsection 제목
\renewcommand{\thesubsection}{\arabic{subsection}}
\titleformat{\subsection}
  {\normalfont\large\bfseries}
  {\thesubsection .}
  {1em}
  {}

% Chapter titlespacing
\titlespacing*{\chapter}{0pt}{0pt}{80pt}

% ─────────────────────────────────────────────
% 7) 캡션 설정 (figure/table)
% ─────────────────────────────────────────────
\usepackage{caption}

\DeclareCaptionLabelFormat{brackets}{[#1~#2]}

\captionsetup{
  labelformat=brackets,
  labelsep=space,
  format=hang,
  font=normalsize
}

\captionsetup[figure]{name=그림}
\captionsetup[table]{name=표}

% ─────────────────────────────────────────────
% 8) 목차(toC), 그림목차, 표목차 (tocloft만 사용)
% ─────────────────────────────────────────────
\usepackage{tocloft}

\renewcommand{\contentsname}{목 차}
\renewcommand{\listfigurename}{그림 목차}
\renewcommand{\listtablename}{표 목차}

% 목차 글꼴
\renewcommand{\cftchapfont}{\Large\bfseries}
\renewcommand{\cftsecfont}{\large}
\renewcommand{\cftsubsecfont}{\normalsize}

\renewcommand{\cftchappagefont}{\Large}
\renewcommand{\cftsecpagefont}{\large}
\renewcommand{\cftsubsecpagefont}{\normalsize}

% 목차 간격
\setlength{\cftbeforechapskip}{10pt}
\setlength{\cftbeforesecskip}{5pt}
\setlength{\cftbeforesubsecskip}{0pt}

% 목차 제목 커스텀
\usepackage{etoolbox}
\makeatletter
\patchcmd{\tableofcontents}
  {\chapter*{\contentsname}}
  {\chapter*{\fontsize{20}{24}\selectfont\bfseries\centering 목차}}
  {}{}
\patchcmd{\listoffigures}
  {\chapter*{\listfigurename}}
  {\chapter*{\fontsize{18}{22}\selectfont\bfseries\centering 그림 목록}}
  {}{}
\patchcmd{\listoftables}
  {\chapter*{\listtablename}}
  {\chapter*{\fontsize{18}{22}\selectfont\bfseries\centering 표 목록}}
  {}{}
\makeatother
