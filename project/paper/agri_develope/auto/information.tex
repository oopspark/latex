%%%%%%%%%%%%% Auto-generated LaTeX Macros


%%%%%%%%%%%% basic_info

\newcommand{\authorname}{박지민} % basic_info.authorname
\newcommand{\department}{농경제사회학과} % basic_info.department
\newcommand{\program}{석사과정} % basic_info.course
\newcommand{\course}{농업발전연구}
\newcommand{\publishyearmonth}{2025년도 2학기} % approval.publish_year_month

%%%%%%%%%%%% title

\newcommand{\maintitleline}{글로벌 콩 무역 네트워크의 구조적 변화와 \\[10pt] 한국 식량안보에 대한 함의} % title.main

\newcommand{\maintitle}{글로벌 콩 무역 네트워크의 구조적 변화와 한국 식량안보에 대한 함의} % title.main
\newcommand{\subtitle}{2023년 복잡계 네트워크 분석을 중심으로} % title.sub



\newcommand{\chapterone}{서 론}
\newcommand{\chaptertwo}{이론적 배경 및 선행연구 검토} % toc.chapterone
\newcommand{\chapterthree}{자료 및 연구방법} % toc.chaptertwo
\newcommand{\chapterfour}{2023년 글로벌 콩 무역 네트워크 분석 결과} % toc.chapterthree
\newcommand{\chapterfive}{시나리오 분석의 확장과 한국적 함의: 2020년과 2023년의 비교} 
\newcommand{\chaptersix}{정책적 시사점 및 결론: 네트워크 관점에서 본 한국의 선택지} 


\newcommand{\chapterfourtoc}{2023년 글로벌 콩 무역 네트워크 \par 분석 결과} % toc.chapterthree
\newcommand{\chapterfivetoc}{시나리오 분석의 확장과 한국적 \par 함의: 2020년과 2023년의 비교} 
\newcommand{\chaptersixtoc}{정책적 시사점 및 결론: 네트워크 \par 관점에서 본 한국의 선택지} 