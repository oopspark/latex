\section{\chaptersix}

\subsection{정책적 시사점}

본 연구는 2022년 글로벌 콩 무역 데이터를 바탕으로 국가 간 교역 관계를
방향성과 가중치를 가진 네트워크로 재구성하고,
그 구조적 특성과 변화 양상을 분석하였다.
특히 선행연구가 주로 다루어 온 2000--2020년 구간의 결과와 비교함으로써,
최근 글로벌 공급망 불안정성이 반영된 네트워크 구조의 연속성과 변화를 함께 점검하였다.

분석 결과, 글로벌 콩 무역은 여전히 소수의 주요 수출국에 의해 좌우되는 구조를 유지하고 있으며,
네트워크의 기본적인 연결성 또한 단기간에 급격히 붕괴되기보다는
경로의 재배치와 역할 분담을 통해 유지되는 성격을 보였다.
다만 이러한 구조적 ``유지''가 곧바로 개별 수입국의 안정성을 의미하지는 않는다.
특히 한국과 같이 네트워크의 중심부가 아닌 말단 수요국의 경우,
충격 발생 시 물량 확보와 비용 부담이 주변부로 이전될 가능성이 크며,
그 영향은 단순한 교역량이나 중심성 지표만으로 충분히 설명되기 어렵다.

이에 본 연구는 한국의 취약성을 평가하기 위해
수입 물량이 한국에 도달하는 \emph{경로의 직접성}과
직접 연결 단절 시 \emph{우회 의존의 확대}를 함께 고려하는
수입 경로 효율성 지표(IPEI)를 도입하였다.
그 결과, 평상시 한국의 수입 구조는 비교적 직접적이지만,
주요 공급국과의 직접 연결이 약화될 경우
수입 경로 효율성이 크게 저하될 수 있음을 확인하였다.
이는 한국의 식량안보가 단순히 ``수입선을 늘리는가'' 또는 ``자급률을 높이는가''의 문제가 아니라,
국제 네트워크 속에서 형성된 조달 구조 자체를 어떻게 관리할 것인가의 문제임을 시사한다.

이러한 결과는 다음과 같은 정책적 시사점을 제공한다.
첫째, 식량안보 평가는 자급률 지표에 더해
국제 조달 구조의 취약성을 반영할 수 있는 네트워크 관점의 평가를 병행할 필요가 있다.
둘째, 수입선 다변화는 공급국의 숫자 확대만으로 충분하지 않으며,
실제 충격 상황에서 조달 경로가 어떻게 재편되는지까지 고려하는 접근이 요구된다.
셋째, 국내 콩 생산은 가격 경쟁력의 관점만으로 평가되기보다,
외부 충격에 대한 완충 장치로서 식량안보 체계에서 수행할 역할을 재정의할 필요가 있다.
요컨대 콩 정책은 농업 부문 내부의 문제가 아니라,
공급망 안정성과 국가 리스크 관리의 관점에서 함께 다루어져야 한다.


\subsection{연구의 한계 및 향후 연구 방향}

본 연구는 수입 경로 효율성(IPEI)을 통해 한국의 조달 구조를
``직접성''과 ``단절 시 우회 의존 확대''의 관점에서 정량화하였다.
다만 본 지표는 네트워크가 표현하는 엣지를 \emph{실질적 공급 경로}로 해석하는 과정에서,
현행 무역 구조에서의 \emph{우회 수입(재수출)}이 제한적이라는 가정을
암묵적으로 포함할 수밖에 없다는 한계를 가진다.
즉, 특정 국가로부터 한국으로 향하는 수입이 관측되더라도,
그 물량이 해당 국가의 국내 생산에 기반한 ``직접 공급''인지,
아니면 제3국 물량이 유입된 뒤 재수출된 ``중개 공급''인지를
본 연구의 네트워크 구성만으로는 완전히 구분하기 어렵다.

실제로 무역 통계에는 가공·환적·재수출을 포함한 다양한 형태의 우회 거래가 존재할 수 있으며,
만약 한국이 이미 일정 수준의 우회 수입 구조에 편입되어 있다면,
평상시 IPEI가 높게 나타나는 결과는
``직접 연결 중심''을 의미하기보다
``표면상 직접 거래처럼 보이는 중개 경로''가 포함된 결과일 가능성도 배제할 수 없다.
따라서 본 연구의 IPEI는 한국의 조달 구조를 설명하는 유용한 지표이지만,
개별 엣지가 의미하는 공급의 실체를 완전히 식별한다는 점에서는 제한이 존재한다.

이 한계를 보완하기 위해 향후 연구에서는
무역 네트워크 지표를 생산·수급 구조와 결합한 확장이 필요하다.
구체적으로, 특정 국가가 \emph{중개국(re-export hub)}인지 여부를 판단하기 위해
해당 국가의 콩 \emph{생산량}과 \emph{순수출량(수출--수입)} 및 재고 변화량을 동시에 고려하는 접근이 요구된다.
예컨대 생산량이 크지 않음에도 순수출이 크게 나타나는 국가는
자국 생산 기반의 직접 공급국이라기보다,
수입 물량을 재분배하는 중개국으로 해석될 가능성이 높다.
참고할 만한 선행연구로는 \textcite{imSoybeanAdjustingPattern2010}이 있으며,
해당 연구는 세계의 콩 생산변동 조정유형을 분석하면서 무역전이효과를 살펴본 바 있다.
이와 같은 판별을 통해 네트워크 상의 엣지를
``생산 기반 공급''과 ``재수출 기반 공급''으로 구분할 수 있다면,
한국의 수입 경로 효율성(IPEI) 또한
단순한 최단 경로 길이 기반 측정을 넘어,
\emph{실질 공급 경로의 직접성}과 \emph{중개 의존도}를 더 정밀하게 반영하는 지표로 개선될 수 있다.

요컨대 향후 연구는 (i) 무역 네트워크와 생산·수급 데이터를 결합하고,
(ii) 중개국 식별 및 엣지 유형화를 수행함으로써,
(iii) 한국의 조달 경로 취약성을 보다 현실적인 공급망 구조로 확장하여 평가하는 방향으로 진행될 필요가 있다.

\subsection{결 론}

본 연구는 글로벌 콩 무역을 네트워크로 재구성하여
2022년 시점의 구조적 특성을 분석하고,
선행연구(2020년까지의 분석)와 비교함으로써
최근 변화가 반영된 글로벌 콩 무역 구조의 연속성과 변화를 제시하였다.
분석 결과, 글로벌 콩 무역은 소수 핵심 수출국에 대한 의존 구조가 지속되는 가운데,
중개 기능과 연결 경로의 일부 재편이 관찰되었다.
그러나 이러한 변화는 네트워크 전체의 구조적 유지와는 별개로,
한국과 같은 주변부 수입국의 구조적 취약성을 근본적으로 해소하기에는 제한적이었다.

특히 본 연구는 한국의 취약성을
네트워크 내 ``중심성''이 아니라
``조달 경로의 효율성과 회복력''이라는 관점에서 재정의하고,
수입 경로 효율성(IPEI)과 단절 시나리오를 통해 이를 정량적으로 제시하였다.
그 결과, 한국은 평상시 직접 연결에 기반한 조달 구조를 갖더라도,
주요 공급국과의 연결 약화가 곧바로 우회 경로 의존과 취약성 심화로 이어질 수 있음을 확인하였다.

종합하면, 한국의 식량안보 전략은
생산 확대와 수입 확대라는 이분법을 넘어,
국제 무역 네트워크 속에서 형성된 조달 구조의 위험을 인식하고
이를 관리하는 방향으로 전환될 필요가 있다.
본 연구는 이러한 관점 전환을 위한 분석틀로서
네트워크 분석과 경로 기반 지표를 결합한 접근을 제시했다는 점에서 의의를 가진다.
