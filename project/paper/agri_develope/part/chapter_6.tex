\section{\chaptersix}


\subsection{네트워크 관점에서 본 식량안보의 재정의}

본 연구의 분석 결과는 식량안보를 전통적인 자급률 중심 개념에서 벗어나, 
글로벌 무역 네트워크 내 구조적 위치의 문제로 재정의할 필요성을 제기한다. 
2023년 글로벌 콩 무역 네트워크는 낮은 밀도와 짧은 평균 경로 길이를 갖는 
고효율 구조를 유지하고 있으며, 이는 표면적으로는 안정적인 국제 공급 환경처럼 보인다. 
그러나 이러한 효율성은 소수 허브 국가에 대한 강한 의존을 전제로 성립하는 것이며, 
네트워크 외곽에 위치한 국가들에게는 위험이 이전되는 구조를 내포하고 있다.

네트워크 전체의 효율성과 개별 국가의 안전성은 동일한 개념이 아니다. 
Wang et al.(2023)의 2020년 분석과 본 연구의 2023년 분석을 종합하면, 
글로벌 콩 무역 네트워크는 허브 국가의 일부 역할 재조정을 통해 
단기적 회복력을 강화해 왔으나, 
이러한 변화가 주변부 수입국의 구조적 취약성을 완화시키지는 못하고 있음을 확인할 수 있다.

즉, 네트워크가 “버틴다”는 사실은 곧 “한국이 안전하다”는 의미로 해석될 수 없다.


\subsection{한국의 네트워크상 위치: 통제 불가능한 종속 노드}

네트워크 지표를 통해 볼 때, 
한국은 글로벌 콩 무역 네트워크에서 다음과 같은 특징적 위치를 가진다.
첫째, 가중 차수 측면에서 한국은 수출 허브도, 대규모 수입 허브도 아니다.
둘째, 매개 중심성 측면에서 한국은 교역 흐름을 경유하거나 조정하는 위치에 있지 않다.
셋째, 평균 경로 길이가 짧은 네트워크 구조 속에서, 
한국은 충격이 빠르게 전이되는 위치에 놓여 있다.

이러한 위치는 한국이 글로벌 콩 무역 네트워크에서 통제력을 행사할 수 없는 
종속적 수입 노드임을 의미한다. 
허브 국가의 정책 변화, 생산 차질, 물류 교란은 네트워크를 통해 빠르게 전달되며, 
한국은 이를 완화할 수 있는 구조적 수단을 거의 보유하지 못하고 있다. 
이는 네트워크 중심부에 위치한 중국과도 구별되는 특징으로, 
중국은 막대한 수입 규모를 통해 제한적이나마 협상력과 경로 선택권을 보유하고 있는 반면, 
한국은 그러한 여지가 매우 제한적이다.


\subsection{수입선 다변화 정책의 네트워크적 한계}

한국의 식량 정책 논의에서 반복적으로 제시되는 수단 중 하나는 수입선 다변화이다. 
그러나 네트워크 관점에서 볼 때, 수입선 다변화는 구조적 제약을 가진 전략이다. 
글로벌 콩 무역 네트워크에서 실질적인 공급 능력을 가진 국가는 소수에 불과하며, 
이들 국가는 상호 연결된 허브 집단을 형성하고 있다. 
2020년 대비 2023년 네트워크에서 미국의 상대적 영향력이 일부 약화된 것은 사실이나, 
이는 허브 구조의 해체라기보다는 브라질 중심 구조의 강화로 해석하는 것이 타당하다.

따라서 한국이 특정 국가로부터의 수입 비중을 조정하더라도, 
네트워크 전체 차원에서는 동일한 허브 집단에 대한 의존이 지속될 가능성이 높다. 
이는 수입선 다변화가 무의미하다는 뜻이 아니라, 
그것만으로는 네트워크 리스크를 근본적으로 해소할 수 없음을 의미한다. 
네트워크 외곽에 위치한 국가에게 다변화는 충격의 강도를 완화할 수는 있으나, 
충격 자체를 차단하는 수단은 아니다.


\subsection{국내 콩 생산의 네트워크적 의미: 효율이 아닌 완충}

네트워크 관점에서 볼 때, 국내 콩 생산의 정책적 의미는 
기존의 경제성 논리와는 다르게 해석될 필요가 있다. 
글로벌 콩 시장에서 한국의 생산은 가격 경쟁력 측면에서 비교우위를 갖기 어렵다. 
그러나 이는 국내 생산이 불필요하다는 결론으로 이어지지 않는다. 
오히려 국내 콩 생산은 글로벌 네트워크 충격에 대한 완충 장치(buffer)로서의 기능을 가진다.

네트워크가 고효율 구조를 유지할수록, 외부 충격은 빠르고 광범위하게 확산된다. 
이러한 구조 속에서 일정 수준의 국내 생산 기반은 수입 물량의 대체 수단이라기보다는, 
단기적 충격을 흡수하고 정책 대응 시간을 확보하는 전략적 자산으로 기능한다. 
이는 국내 콩 생산을 단순히 자급률 제고 수단이 아니라, 
위험 관리 인프라로 인식해야 함을 의미한다.


\subsection{한국 식량안보 정책을 위한 네트워크 기반 원칙}

본 연구의 분석을 바탕으로, 
한국의 식량안보 정책은 다음과 같은 네트워크 기반 원칙 위에서 재구성될 필요가 있다.

첫째, 식량안보 평가는 자급률 지표에 네트워크 취약성 지표를 
결합하는 방식으로 확장되어야 한다.
둘째, 수입선 다변화는 허브 회피 전략이 아니라, 
충격 분산 전략으로 이해되어야 한다.
셋째, 국내 생산 정책은 경제적 효율성만이 아니라, 
네트워크 안정성 기여도를 기준으로 평가되어야 한다.
넷째, 중장기적으로는 콩을 포함한 사료용 곡물에 대해 
전략 비축과 생산 기반을 연계한 복합적 대응 체계가 필요하다.

이러한 원칙은 글로벌 시장과의 단절을 의미하지 않는다. 
오히려 글로벌 네트워크에 편입된 상태에서 발생하는 구조적 위험을 인식하고, 
그 위험을 관리하는 방향으로 정책 목표를 재설정하자는 제안이다.


\subsection{결 론}

본 연구는 2023년 글로벌 콩 무역 네트워크를 재구성하고, 
이를 2020년 선행연구와 비교함으로써 글로벌 콩 무역 구조의 연속성과 변화를 분석하였다. 
분석 결과, 글로벌 콩 무역은 여전히 소수 허브 국가에 의해 지배되는 
고효율 네트워크 구조를 유지하고 있으며, 
허브 국가의 역할에는 점진적인 재편이 나타나고 있다. 
그러나 이러한 변화는 네트워크 전체의 안정성을 강화하는 방향으로 작용할 뿐, 
한국과 같은 주변부 수입국의 구조적 취약성을 근본적으로 완화시키지는 못하고 있다.

따라서 한국의 식량안보 전략은 생산 확대와 수입 확대라는 이분법을 넘어, 
글로벌 무역 네트워크 내에서의 구조적 위치를 인식하고 
이를 관리하는 방향으로 전환되어야 한다. 
콩 정책은 농업 정책이 아니라, 
공급망 안정성과 국가 리스크 관리 정책의 일부로 재정의될 필요가 있다. 
본 연구는 이러한 관점 전환을 위한 분석적 근거를 제공한다는 점에서 의의를 가진다.