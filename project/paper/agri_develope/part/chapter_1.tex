\section{\chapterone}

\subsection{연구 배경: 글로벌 곡물 무역 환경의 구조적 변화}


21세기 들어 글로벌 농산물 시장은 단순한 국가 간 교역의 집합이 아니라, 복잡한 상호의존 구조를 갖는 글로벌 공급망 네트워크로 빠르게 전환되었다. 특히 사료·식용·산업 원료로 동시에 활용되는 전략 작물의 경우, 특정 국가 또는 소수의 교역 경로에 대한 의존도가 심화되면서 식량안보 문제는 더 이상 생산량의 문제가 아닌 구조적 취약성의 문제로 인식되고 있다.

콩(soybean)은 이러한 변화가 가장 극명하게 드러나는 작물 중 하나이다. 콩은 직접 소비되는 식량 작물일 뿐 아니라, 축산업의 사료 원료, 식물성 유지 산업, 바이오 연료 등 다양한 산업 부문과 연계되어 있어, 그 공급 안정성은 국가 경제 전반과 직결된다. 그럼에도 불구하고 글로벌 콩 생산과 수출은 브라질, 미국, 아르헨티나 등 소수 국가에 집중되어 있으며, 다수의 국가들은 구조적으로 수입 의존적 위치에 놓여 있다.

최근의 국제 정세는 이러한 집중적 구조가 내포한 위험을 반복적으로 드러내고 있다. 미·중 무역 갈등, 코로나19 팬데믹, 우크라이나 전쟁, 기후변화로 인한 생산 불확실성은 모두 글로벌 곡물 공급망의 취약성을 노출시켰다. 이 과정에서 국제 곡물 가격은 단기간에 급등락을 반복하였고, 수입 의존도가 높은 국가일수록 충격의 강도는 더욱 증폭되었다. 이러한 상황은 기존의 “국제 시장에 의존하면 된다”는 전통적 비교우위 논리에 근본적인 재검토를 요구하고 있다.



\subsection{문제 제기: 한국 식량안보 논의의 구조적 공백}


한국은 경제 규모와 산업 경쟁력에 비해 농업 부문, 특히 사료용 곡물의 자급 기반이 매우 취약한 국가에 속한다. 쌀을 제외한 주요 곡물의 자급률은 10% 내외에 머물러 있으며, 콩 역시 식용을 제외하면 대부분을 수입에 의존하고 있다. 그럼에도 불구하고 한국의 식량안보 논의는 전통적으로 자급률 수치 또는 국내 생산 확대 여부에 집중되어 왔다.

그러나 글로벌 곡물 무역이 네트워크화된 오늘날, 단순한 자급률 지표만으로는 한 국가가 처한 실제 위험을 충분히 설명하기 어렵다. 동일한 자급률을 가진 국가라 하더라도, 어떤 국가들과 어떤 구조 속에서 교역하고 있는지에 따라 충격에 대한 노출 정도는 크게 달라질 수 있다. 즉, 식량안보는 더 이상 “얼마를 생산하느냐”의 문제가 아니라, “어떤 네트워크에 어떤 위치로 편입되어 있는가”의 문제로 확장되어야 한다.

이러한 관점에서 볼 때, 한국은 글로벌 곡물 무역 네트워크에서 비가시적이지만 취약한 말단 노드에 가까운 위치를 차지하고 있다. 한국은 글로벌 콩 무역에서 거래량 기준의 핵심 플레이어는 아니지만, 허브 국가들의 공급 결정과 가격 변동을 직접적으로 수용할 수밖에 없는 구조에 놓여 있다. 그럼에도 불구하고 한국을 대상으로 글로벌 콩 무역 네트워크의 구조적 특성과 그 함의를 분석한 연구는 상대적으로 부족한 실정이다.



\subsection{선행연구와 연구 공백}

최근 국제 곡물 무역을 복잡계 네트워크 관점에서 분석하려는 연구들이 점차 축적되고 있다. 특히 Wang et al.(2023)은 2000–2020년 글로벌 콩 무역 네트워크를 구축하여, 네트워크가 소수 허브 국가에 의해 지배되는 scale-free 구조를 가지며, 이러한 구조가 특정 국가(중국)의 식량안보 취약성을 심화시킨다는 점을 실증적으로 보여주었다 

. 이 연구는 글로벌 콩 무역을 단순한 양적 흐름이 아니라, 구조적 관계망으로 파악했다는 점에서 중요한 학술적 기여를 가진다.

그러나 기존 연구들은 대체로 중국, 미국, 브라질 등 글로벌 곡물 시장의 핵심 국가들을 중심으로 분석을 전개해 왔다. 이러한 접근은 네트워크의 중심부를 이해하는 데에는 효과적이지만, 한국과 같이 네트워크의 중심은 아니지만 충격에는 민감한 국가의 위치와 위험을 충분히 설명하는 데에는 한계가 있다. 다시 말해, “허브 국가가 사라지면 네트워크가 어떻게 붕괴되는가”에 대한 분석은 존재하지만, “그 붕괴가 한국과 같은 국가에 어떤 의미를 가지는가”에 대한 논의는 상대적으로 부족하다.

또한 기존 연구들은 주로 2020년 이전의 데이터를 활용하고 있어, 최근 몇 년간 가속화된 글로벌 공급망 불안정성과 구조 변화를 충분히 반영하지 못한다는 한계도 존재한다. 2023년을 전후한 시점에서 글로벌 콩 무역 네트워크를 재구성하고, 그 구조적 특성을 다시 점검하는 작업은 학문적·정책적으로 모두 의미 있는 과제라 할 수 있다.


\subsection{연구 목적 및 연구 질문}

이러한 문제의식을 바탕으로, 본 연구는 2023년 기준 글로벌 콩 무역 네트워크를 재구성하고, 그 구조적 특성과 취약성을 분석한 뒤, 이를 한국 식량안보의 관점에서 재해석하는 것을 목적으로 한다. 특히 본 연구는 네트워크의 중심 국가가 아닌 한국의 위치를 조명함으로써, 기존 식량안보 논의의 시야를 확장하고자 한다.

구체적인 연구 질문은 다음과 같다.

첫째, 2023년 글로벌 콩 무역 네트워크는 어떤 구조적 특성을 가지며, 주요 허브 국가는 누구인가?
둘째, 브라질과 미국 등 핵심 허브 국가의 이탈은 네트워크 효율성과 안정성에 어떤 영향을 미치는가?
셋째, 이러한 네트워크 구조는 한국과 같은 수입 의존 국가의 식량안보에 어떤 시사점을 제공하는가?


\subsection{연구의 의의}

본 연구의 의의는 다음 세 가지로 정리할 수 있다.
첫째, 글로벌 콩 무역 네트워크를 최신 시점에서 재구성함으로써 기존 연구를 시간적으로 확장한다.
둘째, 중국 중심의 분석 틀을 넘어, 한국의 구조적 위치와 취약성을 네트워크 관점에서 해석한다.
셋째, 콩 생산 및 수입 정책을 단순한 농업 문제가 아닌 식량안보·공급망 안정성 문제로 재정의하는 데 기여한다.

이러한 분석은 향후 한국의 곡물 정책, 논·밭 구조 개편, 수입선 다변화 전략을 평가하는 데 있어 중요한 이론적·정책적 기초를 제공할 것으로 기대된다.