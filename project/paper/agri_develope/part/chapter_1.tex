\section{\chapterone}

\subsection{연구 배경: 글로벌 곡물 무역 환경의 구조적 변화}

21세기 들어 글로벌 농산물 시장은 단순한 국가 간 교역의 합이 아니라,
다층적 상호의존이 얽힌 \emph{글로벌 공급망 네트워크}로 빠르게 전환되었다.
특히 사료·식용·산업 원료로 동시에 활용되는 전략 작물은
생산과 수출이 특정 국가에 집중되는 경향이 강하며,
그 결과 다수의 수입 의존 국가는 소수의 공급국 및 제한된 교역 경로에 구조적으로 종속되는 형태를 보인다.
이러한 조건에서는 식량안보를 ``생산량''이나 ``자급률''의 문제로만 다루기 어렵고,
국가가 어떤 관계망에 어떤 방식으로 편입되어 있는지, 즉 \emph{구조적 취약성}의 문제로 확장해 이해할 필요가 있다.

콩(soybean)은 이러한 구조 변화가 가장 뚜렷하게 나타나는 작물 중 하나이다.
콩은 직접 소비되는 식량 자원일 뿐 아니라,
축산업 사료, 식물성 유지 산업, 바이오 연료 등 다양한 산업 부문과 연계되어 있어
공급 안정성이 국가 경제 전반의 비용 구조와 물가, 산업 경쟁력에 직결된다.
그럼에도 글로벌 콩 생산과 수출은 브라질, 미국, 아르헨티나 등 소수 국가에 집중되어 있으며,
다수의 국가는 구조적으로 수입 의존적 위치에 놓여 있다.

최근 국제 정세 또한 이러한 집중 구조가 내포한 위험을 반복적으로 드러냈다.
미·중 무역 갈등, 코로나19 팬데믹, 우크라이나 전쟁, 기후변화에 따른 생산 불확실성은
공급망의 단절과 재편을 촉발하며 국제 곡물 가격의 변동성을 확대시켰다.
이 과정에서 수입 의존도가 높은 국가는 공급국의 정책 변화, 물류 차질, 가격 급등락을
직접적으로 흡수해야 했고, 결과적으로 ``국제 시장에 의존하면 된다''는 전통적 비교우위 논리에 대한
재검토 필요성이 제기되었다.
따라서 오늘날 식량안보는 생산 확대만으로 설명되기보다,
\emph{국제 무역 네트워크에서의 위치와 연결 구조}를 함께 고려하는 분석틀이 요구된다.


\subsection{문제 제기: 한국 식량안보 논의의 구조적 공백}

한국은 경제 규모와 산업 경쟁력에 비해 농업 부문,
특히 사료용 곡물의 자급 기반이 취약한 국가로 분류된다.
곡물 자급률이 낮은 수준에 머무는 가운데,
콩 역시 식용 목적을 제외하면 상당 부분을 수입에 의존하는 구조가 지속되고 있다.
그럼에도 한국의 식량안보 논의는 전통적으로 자급률 지표,
혹은 국내 생산 확대 여부에 정책적 관심이 집중되어 왔다.

그러나 글로벌 곡물 무역이 네트워크화된 현실에서
동일한 자급률을 가진 국가라도 어떤 국가와 어떤 경로로 연결되어 있는지에 따라
외부 충격에 대한 노출 정도와 회복력은 크게 달라질 수 있다.
즉 식량안보는 ``얼마를 생산하는가''의 문제를 넘어,
``어떤 관계망 속에서 어떤 방식으로 조달하는가''라는 구조적 질문을 포함해야 한다.

이 관점에서 볼 때 한국은 글로벌 콩 무역 네트워크에서
거래량 기준의 핵심 허브 국가는 아니지만,
허브 국가의 공급 결정과 국제 가격 변동을 직접적으로 수용할 수밖에 없는
\emph{말단 수요국(end-demand country)}의 성격이 강하다.
그럼에도 한국을 대상으로 글로벌 콩 무역 네트워크의 구조적 특성과
그 정책적 함의를 체계적으로 분석한 연구는 상대적으로 부족하다.
특히 한국의 수입 구조가 ``직접 연결''에 기반한 것인지,
아니면 제3국을 경유하는 ``우회 경로'' 의존이 큰지에 따라
충격 전파 방식과 취약성의 성격은 달라질 수 있으나,
이에 대한 정량적 논의는 제한적이다.

\begin{figure}[htbp]
    \centering
    \hspace*{-55pt}%
    \resizebox{1.6\textwidth}{!}{%% Creator: Matplotlib, PGF backend
%%
%% To include the figure in your LaTeX document, write
%%   \input{<filename>.pgf}
%%
%% Make sure the required packages are loaded in your preamble
%%   \usepackage{pgf}
%%
%% Also ensure that all the required font packages are loaded; for instance,
%% the lmodern package is sometimes necessary when using math font.
%%   \usepackage{lmodern}
%%
%% Figures using additional raster images can only be included by \input if
%% they are in the same directory as the main LaTeX file. For loading figures
%% from other directories you can use the `import` package
%%   \usepackage{import}
%%
%% and then include the figures with
%%   \import{<path to file>}{<filename>.pgf}
%%
%% Matplotlib used the following preamble
%%   \def\mathdefault#1{#1}
%%   \everymath=\expandafter{\the\everymath\displaystyle}
%%   \IfFileExists{scrextend.sty}{
%%     \usepackage[fontsize=5.000000pt]{scrextend}
%%   }{
%%     \renewcommand{\normalsize}{\fontsize{5.000000}{6.000000}\selectfont}
%%     \normalsize
%%   }
%%   
%%   \ifdefined\pdftexversion\else  % non-pdftex case.
%%     \usepackage{fontspec}
%%     \setmainfont{DejaVuSerif.ttf}[Path=\detokenize{/home/user/.cache/pypoetry/virtualenvs/graph-KASAOWVd-py3.12/lib/python3.12/site-packages/matplotlib/mpl-data/fonts/ttf/}]
%%     \setsansfont{DejaVuSans.ttf}[Path=\detokenize{/home/user/.cache/pypoetry/virtualenvs/graph-KASAOWVd-py3.12/lib/python3.12/site-packages/matplotlib/mpl-data/fonts/ttf/}]
%%     \setmonofont{DejaVuSansMono.ttf}[Path=\detokenize{/home/user/.cache/pypoetry/virtualenvs/graph-KASAOWVd-py3.12/lib/python3.12/site-packages/matplotlib/mpl-data/fonts/ttf/}]
%%   \fi
%%   \makeatletter\@ifpackageloaded{underscore}{}{\usepackage[strings]{underscore}}\makeatother
%%
\begingroup%
\makeatletter%
\begin{pgfpicture}%
\pgfpathrectangle{\pgfpointorigin}{\pgfqpoint{5.555556in}{2.083333in}}%
\pgfusepath{use as bounding box, clip}%
\begin{pgfscope}%
\pgfsetbuttcap%
\pgfsetmiterjoin%
\definecolor{currentfill}{rgb}{1.000000,1.000000,1.000000}%
\pgfsetfillcolor{currentfill}%
\pgfsetlinewidth{0.000000pt}%
\definecolor{currentstroke}{rgb}{1.000000,1.000000,1.000000}%
\pgfsetstrokecolor{currentstroke}%
\pgfsetdash{}{0pt}%
\pgfpathmoveto{\pgfqpoint{0.000000in}{0.000000in}}%
\pgfpathlineto{\pgfqpoint{5.555556in}{0.000000in}}%
\pgfpathlineto{\pgfqpoint{5.555556in}{2.083333in}}%
\pgfpathlineto{\pgfqpoint{0.000000in}{2.083333in}}%
\pgfpathlineto{\pgfqpoint{0.000000in}{0.000000in}}%
\pgfpathclose%
\pgfusepath{fill}%
\end{pgfscope}%
\begin{pgfscope}%
\pgfsetbuttcap%
\pgfsetmiterjoin%
\definecolor{currentfill}{rgb}{1.000000,1.000000,1.000000}%
\pgfsetfillcolor{currentfill}%
\pgfsetlinewidth{0.000000pt}%
\definecolor{currentstroke}{rgb}{0.000000,0.000000,0.000000}%
\pgfsetstrokecolor{currentstroke}%
\pgfsetstrokeopacity{0.000000}%
\pgfsetdash{}{0pt}%
\pgfpathmoveto{\pgfqpoint{0.694444in}{0.416667in}}%
\pgfpathlineto{\pgfqpoint{3.888889in}{0.416667in}}%
\pgfpathlineto{\pgfqpoint{3.888889in}{1.833333in}}%
\pgfpathlineto{\pgfqpoint{0.694444in}{1.833333in}}%
\pgfpathlineto{\pgfqpoint{0.694444in}{0.416667in}}%
\pgfpathclose%
\pgfusepath{fill}%
\end{pgfscope}%
\begin{pgfscope}%
\pgfsetbuttcap%
\pgfsetroundjoin%
\definecolor{currentfill}{rgb}{0.000000,0.000000,0.000000}%
\pgfsetfillcolor{currentfill}%
\pgfsetlinewidth{0.752812pt}%
\definecolor{currentstroke}{rgb}{0.000000,0.000000,0.000000}%
\pgfsetstrokecolor{currentstroke}%
\pgfsetdash{}{0pt}%
\pgfsys@defobject{currentmarker}{\pgfqpoint{0.000000in}{-0.013889in}}{\pgfqpoint{0.000000in}{0.000000in}}{%
\pgfpathmoveto{\pgfqpoint{0.000000in}{0.000000in}}%
\pgfpathlineto{\pgfqpoint{0.000000in}{-0.013889in}}%
\pgfusepath{stroke,fill}%
}%
\begin{pgfscope}%
\pgfsys@transformshift{0.926519in}{0.416667in}%
\pgfsys@useobject{currentmarker}{}%
\end{pgfscope}%
\end{pgfscope}%
\begin{pgfscope}%
\definecolor{textcolor}{rgb}{0.000000,0.000000,0.000000}%
\pgfsetstrokecolor{textcolor}%
\pgfsetfillcolor{textcolor}%
\pgftext[x=0.926519in,y=0.354167in,,top]{\color{textcolor}{\ifdefined\pdftexversion\else\setmainfont{NanumMyeongjo}\rmfamily\fi\fontsize{5.000000}{6.000000}\selectfont\catcode`\^=\active\def^{\ifmmode\sp\else\^{}\fi}\catcode`\%=\active\def%{\%}2011}}%
\end{pgfscope}%
\begin{pgfscope}%
\pgfsetbuttcap%
\pgfsetroundjoin%
\definecolor{currentfill}{rgb}{0.000000,0.000000,0.000000}%
\pgfsetfillcolor{currentfill}%
\pgfsetlinewidth{0.752812pt}%
\definecolor{currentstroke}{rgb}{0.000000,0.000000,0.000000}%
\pgfsetstrokecolor{currentstroke}%
\pgfsetdash{}{0pt}%
\pgfsys@defobject{currentmarker}{\pgfqpoint{0.000000in}{-0.013889in}}{\pgfqpoint{0.000000in}{0.000000in}}{%
\pgfpathmoveto{\pgfqpoint{0.000000in}{0.000000in}}%
\pgfpathlineto{\pgfqpoint{0.000000in}{-0.013889in}}%
\pgfusepath{stroke,fill}%
}%
\begin{pgfscope}%
\pgfsys@transformshift{1.174728in}{0.416667in}%
\pgfsys@useobject{currentmarker}{}%
\end{pgfscope}%
\end{pgfscope}%
\begin{pgfscope}%
\definecolor{textcolor}{rgb}{0.000000,0.000000,0.000000}%
\pgfsetstrokecolor{textcolor}%
\pgfsetfillcolor{textcolor}%
\pgftext[x=1.174728in,y=0.354167in,,top]{\color{textcolor}{\ifdefined\pdftexversion\else\setmainfont{NanumMyeongjo}\rmfamily\fi\fontsize{5.000000}{6.000000}\selectfont\catcode`\^=\active\def^{\ifmmode\sp\else\^{}\fi}\catcode`\%=\active\def%{\%}2012}}%
\end{pgfscope}%
\begin{pgfscope}%
\pgfsetbuttcap%
\pgfsetroundjoin%
\definecolor{currentfill}{rgb}{0.000000,0.000000,0.000000}%
\pgfsetfillcolor{currentfill}%
\pgfsetlinewidth{0.752812pt}%
\definecolor{currentstroke}{rgb}{0.000000,0.000000,0.000000}%
\pgfsetstrokecolor{currentstroke}%
\pgfsetdash{}{0pt}%
\pgfsys@defobject{currentmarker}{\pgfqpoint{0.000000in}{-0.013889in}}{\pgfqpoint{0.000000in}{0.000000in}}{%
\pgfpathmoveto{\pgfqpoint{0.000000in}{0.000000in}}%
\pgfpathlineto{\pgfqpoint{0.000000in}{-0.013889in}}%
\pgfusepath{stroke,fill}%
}%
\begin{pgfscope}%
\pgfsys@transformshift{1.422937in}{0.416667in}%
\pgfsys@useobject{currentmarker}{}%
\end{pgfscope}%
\end{pgfscope}%
\begin{pgfscope}%
\definecolor{textcolor}{rgb}{0.000000,0.000000,0.000000}%
\pgfsetstrokecolor{textcolor}%
\pgfsetfillcolor{textcolor}%
\pgftext[x=1.422937in,y=0.354167in,,top]{\color{textcolor}{\ifdefined\pdftexversion\else\setmainfont{NanumMyeongjo}\rmfamily\fi\fontsize{5.000000}{6.000000}\selectfont\catcode`\^=\active\def^{\ifmmode\sp\else\^{}\fi}\catcode`\%=\active\def%{\%}2013}}%
\end{pgfscope}%
\begin{pgfscope}%
\pgfsetbuttcap%
\pgfsetroundjoin%
\definecolor{currentfill}{rgb}{0.000000,0.000000,0.000000}%
\pgfsetfillcolor{currentfill}%
\pgfsetlinewidth{0.752812pt}%
\definecolor{currentstroke}{rgb}{0.000000,0.000000,0.000000}%
\pgfsetstrokecolor{currentstroke}%
\pgfsetdash{}{0pt}%
\pgfsys@defobject{currentmarker}{\pgfqpoint{0.000000in}{-0.013889in}}{\pgfqpoint{0.000000in}{0.000000in}}{%
\pgfpathmoveto{\pgfqpoint{0.000000in}{0.000000in}}%
\pgfpathlineto{\pgfqpoint{0.000000in}{-0.013889in}}%
\pgfusepath{stroke,fill}%
}%
\begin{pgfscope}%
\pgfsys@transformshift{1.671145in}{0.416667in}%
\pgfsys@useobject{currentmarker}{}%
\end{pgfscope}%
\end{pgfscope}%
\begin{pgfscope}%
\definecolor{textcolor}{rgb}{0.000000,0.000000,0.000000}%
\pgfsetstrokecolor{textcolor}%
\pgfsetfillcolor{textcolor}%
\pgftext[x=1.671145in,y=0.354167in,,top]{\color{textcolor}{\ifdefined\pdftexversion\else\setmainfont{NanumMyeongjo}\rmfamily\fi\fontsize{5.000000}{6.000000}\selectfont\catcode`\^=\active\def^{\ifmmode\sp\else\^{}\fi}\catcode`\%=\active\def%{\%}2014}}%
\end{pgfscope}%
\begin{pgfscope}%
\pgfsetbuttcap%
\pgfsetroundjoin%
\definecolor{currentfill}{rgb}{0.000000,0.000000,0.000000}%
\pgfsetfillcolor{currentfill}%
\pgfsetlinewidth{0.752812pt}%
\definecolor{currentstroke}{rgb}{0.000000,0.000000,0.000000}%
\pgfsetstrokecolor{currentstroke}%
\pgfsetdash{}{0pt}%
\pgfsys@defobject{currentmarker}{\pgfqpoint{0.000000in}{-0.013889in}}{\pgfqpoint{0.000000in}{0.000000in}}{%
\pgfpathmoveto{\pgfqpoint{0.000000in}{0.000000in}}%
\pgfpathlineto{\pgfqpoint{0.000000in}{-0.013889in}}%
\pgfusepath{stroke,fill}%
}%
\begin{pgfscope}%
\pgfsys@transformshift{1.919354in}{0.416667in}%
\pgfsys@useobject{currentmarker}{}%
\end{pgfscope}%
\end{pgfscope}%
\begin{pgfscope}%
\definecolor{textcolor}{rgb}{0.000000,0.000000,0.000000}%
\pgfsetstrokecolor{textcolor}%
\pgfsetfillcolor{textcolor}%
\pgftext[x=1.919354in,y=0.354167in,,top]{\color{textcolor}{\ifdefined\pdftexversion\else\setmainfont{NanumMyeongjo}\rmfamily\fi\fontsize{5.000000}{6.000000}\selectfont\catcode`\^=\active\def^{\ifmmode\sp\else\^{}\fi}\catcode`\%=\active\def%{\%}2015}}%
\end{pgfscope}%
\begin{pgfscope}%
\pgfsetbuttcap%
\pgfsetroundjoin%
\definecolor{currentfill}{rgb}{0.000000,0.000000,0.000000}%
\pgfsetfillcolor{currentfill}%
\pgfsetlinewidth{0.752812pt}%
\definecolor{currentstroke}{rgb}{0.000000,0.000000,0.000000}%
\pgfsetstrokecolor{currentstroke}%
\pgfsetdash{}{0pt}%
\pgfsys@defobject{currentmarker}{\pgfqpoint{0.000000in}{-0.013889in}}{\pgfqpoint{0.000000in}{0.000000in}}{%
\pgfpathmoveto{\pgfqpoint{0.000000in}{0.000000in}}%
\pgfpathlineto{\pgfqpoint{0.000000in}{-0.013889in}}%
\pgfusepath{stroke,fill}%
}%
\begin{pgfscope}%
\pgfsys@transformshift{2.167562in}{0.416667in}%
\pgfsys@useobject{currentmarker}{}%
\end{pgfscope}%
\end{pgfscope}%
\begin{pgfscope}%
\definecolor{textcolor}{rgb}{0.000000,0.000000,0.000000}%
\pgfsetstrokecolor{textcolor}%
\pgfsetfillcolor{textcolor}%
\pgftext[x=2.167562in,y=0.354167in,,top]{\color{textcolor}{\ifdefined\pdftexversion\else\setmainfont{NanumMyeongjo}\rmfamily\fi\fontsize{5.000000}{6.000000}\selectfont\catcode`\^=\active\def^{\ifmmode\sp\else\^{}\fi}\catcode`\%=\active\def%{\%}2016}}%
\end{pgfscope}%
\begin{pgfscope}%
\pgfsetbuttcap%
\pgfsetroundjoin%
\definecolor{currentfill}{rgb}{0.000000,0.000000,0.000000}%
\pgfsetfillcolor{currentfill}%
\pgfsetlinewidth{0.752812pt}%
\definecolor{currentstroke}{rgb}{0.000000,0.000000,0.000000}%
\pgfsetstrokecolor{currentstroke}%
\pgfsetdash{}{0pt}%
\pgfsys@defobject{currentmarker}{\pgfqpoint{0.000000in}{-0.013889in}}{\pgfqpoint{0.000000in}{0.000000in}}{%
\pgfpathmoveto{\pgfqpoint{0.000000in}{0.000000in}}%
\pgfpathlineto{\pgfqpoint{0.000000in}{-0.013889in}}%
\pgfusepath{stroke,fill}%
}%
\begin{pgfscope}%
\pgfsys@transformshift{2.415771in}{0.416667in}%
\pgfsys@useobject{currentmarker}{}%
\end{pgfscope}%
\end{pgfscope}%
\begin{pgfscope}%
\definecolor{textcolor}{rgb}{0.000000,0.000000,0.000000}%
\pgfsetstrokecolor{textcolor}%
\pgfsetfillcolor{textcolor}%
\pgftext[x=2.415771in,y=0.354167in,,top]{\color{textcolor}{\ifdefined\pdftexversion\else\setmainfont{NanumMyeongjo}\rmfamily\fi\fontsize{5.000000}{6.000000}\selectfont\catcode`\^=\active\def^{\ifmmode\sp\else\^{}\fi}\catcode`\%=\active\def%{\%}2017}}%
\end{pgfscope}%
\begin{pgfscope}%
\pgfsetbuttcap%
\pgfsetroundjoin%
\definecolor{currentfill}{rgb}{0.000000,0.000000,0.000000}%
\pgfsetfillcolor{currentfill}%
\pgfsetlinewidth{0.752812pt}%
\definecolor{currentstroke}{rgb}{0.000000,0.000000,0.000000}%
\pgfsetstrokecolor{currentstroke}%
\pgfsetdash{}{0pt}%
\pgfsys@defobject{currentmarker}{\pgfqpoint{0.000000in}{-0.013889in}}{\pgfqpoint{0.000000in}{0.000000in}}{%
\pgfpathmoveto{\pgfqpoint{0.000000in}{0.000000in}}%
\pgfpathlineto{\pgfqpoint{0.000000in}{-0.013889in}}%
\pgfusepath{stroke,fill}%
}%
\begin{pgfscope}%
\pgfsys@transformshift{2.663980in}{0.416667in}%
\pgfsys@useobject{currentmarker}{}%
\end{pgfscope}%
\end{pgfscope}%
\begin{pgfscope}%
\definecolor{textcolor}{rgb}{0.000000,0.000000,0.000000}%
\pgfsetstrokecolor{textcolor}%
\pgfsetfillcolor{textcolor}%
\pgftext[x=2.663980in,y=0.354167in,,top]{\color{textcolor}{\ifdefined\pdftexversion\else\setmainfont{NanumMyeongjo}\rmfamily\fi\fontsize{5.000000}{6.000000}\selectfont\catcode`\^=\active\def^{\ifmmode\sp\else\^{}\fi}\catcode`\%=\active\def%{\%}2018}}%
\end{pgfscope}%
\begin{pgfscope}%
\pgfsetbuttcap%
\pgfsetroundjoin%
\definecolor{currentfill}{rgb}{0.000000,0.000000,0.000000}%
\pgfsetfillcolor{currentfill}%
\pgfsetlinewidth{0.752812pt}%
\definecolor{currentstroke}{rgb}{0.000000,0.000000,0.000000}%
\pgfsetstrokecolor{currentstroke}%
\pgfsetdash{}{0pt}%
\pgfsys@defobject{currentmarker}{\pgfqpoint{0.000000in}{-0.013889in}}{\pgfqpoint{0.000000in}{0.000000in}}{%
\pgfpathmoveto{\pgfqpoint{0.000000in}{0.000000in}}%
\pgfpathlineto{\pgfqpoint{0.000000in}{-0.013889in}}%
\pgfusepath{stroke,fill}%
}%
\begin{pgfscope}%
\pgfsys@transformshift{2.912188in}{0.416667in}%
\pgfsys@useobject{currentmarker}{}%
\end{pgfscope}%
\end{pgfscope}%
\begin{pgfscope}%
\definecolor{textcolor}{rgb}{0.000000,0.000000,0.000000}%
\pgfsetstrokecolor{textcolor}%
\pgfsetfillcolor{textcolor}%
\pgftext[x=2.912188in,y=0.354167in,,top]{\color{textcolor}{\ifdefined\pdftexversion\else\setmainfont{NanumMyeongjo}\rmfamily\fi\fontsize{5.000000}{6.000000}\selectfont\catcode`\^=\active\def^{\ifmmode\sp\else\^{}\fi}\catcode`\%=\active\def%{\%}2019}}%
\end{pgfscope}%
\begin{pgfscope}%
\pgfsetbuttcap%
\pgfsetroundjoin%
\definecolor{currentfill}{rgb}{0.000000,0.000000,0.000000}%
\pgfsetfillcolor{currentfill}%
\pgfsetlinewidth{0.752812pt}%
\definecolor{currentstroke}{rgb}{0.000000,0.000000,0.000000}%
\pgfsetstrokecolor{currentstroke}%
\pgfsetdash{}{0pt}%
\pgfsys@defobject{currentmarker}{\pgfqpoint{0.000000in}{-0.013889in}}{\pgfqpoint{0.000000in}{0.000000in}}{%
\pgfpathmoveto{\pgfqpoint{0.000000in}{0.000000in}}%
\pgfpathlineto{\pgfqpoint{0.000000in}{-0.013889in}}%
\pgfusepath{stroke,fill}%
}%
\begin{pgfscope}%
\pgfsys@transformshift{3.160397in}{0.416667in}%
\pgfsys@useobject{currentmarker}{}%
\end{pgfscope}%
\end{pgfscope}%
\begin{pgfscope}%
\definecolor{textcolor}{rgb}{0.000000,0.000000,0.000000}%
\pgfsetstrokecolor{textcolor}%
\pgfsetfillcolor{textcolor}%
\pgftext[x=3.160397in,y=0.354167in,,top]{\color{textcolor}{\ifdefined\pdftexversion\else\setmainfont{NanumMyeongjo}\rmfamily\fi\fontsize{5.000000}{6.000000}\selectfont\catcode`\^=\active\def^{\ifmmode\sp\else\^{}\fi}\catcode`\%=\active\def%{\%}2020}}%
\end{pgfscope}%
\begin{pgfscope}%
\pgfsetbuttcap%
\pgfsetroundjoin%
\definecolor{currentfill}{rgb}{0.000000,0.000000,0.000000}%
\pgfsetfillcolor{currentfill}%
\pgfsetlinewidth{0.752812pt}%
\definecolor{currentstroke}{rgb}{0.000000,0.000000,0.000000}%
\pgfsetstrokecolor{currentstroke}%
\pgfsetdash{}{0pt}%
\pgfsys@defobject{currentmarker}{\pgfqpoint{0.000000in}{-0.013889in}}{\pgfqpoint{0.000000in}{0.000000in}}{%
\pgfpathmoveto{\pgfqpoint{0.000000in}{0.000000in}}%
\pgfpathlineto{\pgfqpoint{0.000000in}{-0.013889in}}%
\pgfusepath{stroke,fill}%
}%
\begin{pgfscope}%
\pgfsys@transformshift{3.408605in}{0.416667in}%
\pgfsys@useobject{currentmarker}{}%
\end{pgfscope}%
\end{pgfscope}%
\begin{pgfscope}%
\definecolor{textcolor}{rgb}{0.000000,0.000000,0.000000}%
\pgfsetstrokecolor{textcolor}%
\pgfsetfillcolor{textcolor}%
\pgftext[x=3.408605in,y=0.354167in,,top]{\color{textcolor}{\ifdefined\pdftexversion\else\setmainfont{NanumMyeongjo}\rmfamily\fi\fontsize{5.000000}{6.000000}\selectfont\catcode`\^=\active\def^{\ifmmode\sp\else\^{}\fi}\catcode`\%=\active\def%{\%}2021}}%
\end{pgfscope}%
\begin{pgfscope}%
\pgfsetbuttcap%
\pgfsetroundjoin%
\definecolor{currentfill}{rgb}{0.000000,0.000000,0.000000}%
\pgfsetfillcolor{currentfill}%
\pgfsetlinewidth{0.752812pt}%
\definecolor{currentstroke}{rgb}{0.000000,0.000000,0.000000}%
\pgfsetstrokecolor{currentstroke}%
\pgfsetdash{}{0pt}%
\pgfsys@defobject{currentmarker}{\pgfqpoint{0.000000in}{-0.013889in}}{\pgfqpoint{0.000000in}{0.000000in}}{%
\pgfpathmoveto{\pgfqpoint{0.000000in}{0.000000in}}%
\pgfpathlineto{\pgfqpoint{0.000000in}{-0.013889in}}%
\pgfusepath{stroke,fill}%
}%
\begin{pgfscope}%
\pgfsys@transformshift{3.656814in}{0.416667in}%
\pgfsys@useobject{currentmarker}{}%
\end{pgfscope}%
\end{pgfscope}%
\begin{pgfscope}%
\definecolor{textcolor}{rgb}{0.000000,0.000000,0.000000}%
\pgfsetstrokecolor{textcolor}%
\pgfsetfillcolor{textcolor}%
\pgftext[x=3.656814in,y=0.354167in,,top]{\color{textcolor}{\ifdefined\pdftexversion\else\setmainfont{NanumMyeongjo}\rmfamily\fi\fontsize{5.000000}{6.000000}\selectfont\catcode`\^=\active\def^{\ifmmode\sp\else\^{}\fi}\catcode`\%=\active\def%{\%}2022}}%
\end{pgfscope}%
\begin{pgfscope}%
\pgfpathrectangle{\pgfqpoint{0.694444in}{0.416667in}}{\pgfqpoint{3.194444in}{1.416667in}}%
\pgfusepath{clip}%
\pgfsetbuttcap%
\pgfsetroundjoin%
\pgfsetlinewidth{0.602250pt}%
\definecolor{currentstroke}{rgb}{0.690196,0.690196,0.690196}%
\pgfsetstrokecolor{currentstroke}%
\pgfsetstrokeopacity{0.400000}%
\pgfsetdash{{2.220000pt}{0.960000pt}}{0.000000pt}%
\pgfpathmoveto{\pgfqpoint{0.694444in}{0.416667in}}%
\pgfpathlineto{\pgfqpoint{3.888889in}{0.416667in}}%
\pgfusepath{stroke}%
\end{pgfscope}%
\begin{pgfscope}%
\pgfsetbuttcap%
\pgfsetroundjoin%
\definecolor{currentfill}{rgb}{0.000000,0.000000,0.000000}%
\pgfsetfillcolor{currentfill}%
\pgfsetlinewidth{0.752812pt}%
\definecolor{currentstroke}{rgb}{0.000000,0.000000,0.000000}%
\pgfsetstrokecolor{currentstroke}%
\pgfsetdash{}{0pt}%
\pgfsys@defobject{currentmarker}{\pgfqpoint{-0.013889in}{0.000000in}}{\pgfqpoint{-0.000000in}{0.000000in}}{%
\pgfpathmoveto{\pgfqpoint{-0.000000in}{0.000000in}}%
\pgfpathlineto{\pgfqpoint{-0.013889in}{0.000000in}}%
\pgfusepath{stroke,fill}%
}%
\begin{pgfscope}%
\pgfsys@transformshift{0.694444in}{0.416667in}%
\pgfsys@useobject{currentmarker}{}%
\end{pgfscope}%
\end{pgfscope}%
\begin{pgfscope}%
\definecolor{textcolor}{rgb}{0.000000,0.000000,0.000000}%
\pgfsetstrokecolor{textcolor}%
\pgfsetfillcolor{textcolor}%
\pgftext[x=0.594645in, y=0.388930in, left, base]{\color{textcolor}{\ifdefined\pdftexversion\else\setmainfont{NanumMyeongjo}\rmfamily\fi\fontsize{5.000000}{6.000000}\selectfont\catcode`\^=\active\def^{\ifmmode\sp\else\^{}\fi}\catcode`\%=\active\def%{\%}0}}%
\end{pgfscope}%
\begin{pgfscope}%
\pgfpathrectangle{\pgfqpoint{0.694444in}{0.416667in}}{\pgfqpoint{3.194444in}{1.416667in}}%
\pgfusepath{clip}%
\pgfsetbuttcap%
\pgfsetroundjoin%
\pgfsetlinewidth{0.602250pt}%
\definecolor{currentstroke}{rgb}{0.690196,0.690196,0.690196}%
\pgfsetstrokecolor{currentstroke}%
\pgfsetstrokeopacity{0.400000}%
\pgfsetdash{{2.220000pt}{0.960000pt}}{0.000000pt}%
\pgfpathmoveto{\pgfqpoint{0.694444in}{0.700000in}}%
\pgfpathlineto{\pgfqpoint{3.888889in}{0.700000in}}%
\pgfusepath{stroke}%
\end{pgfscope}%
\begin{pgfscope}%
\pgfsetbuttcap%
\pgfsetroundjoin%
\definecolor{currentfill}{rgb}{0.000000,0.000000,0.000000}%
\pgfsetfillcolor{currentfill}%
\pgfsetlinewidth{0.752812pt}%
\definecolor{currentstroke}{rgb}{0.000000,0.000000,0.000000}%
\pgfsetstrokecolor{currentstroke}%
\pgfsetdash{}{0pt}%
\pgfsys@defobject{currentmarker}{\pgfqpoint{-0.013889in}{0.000000in}}{\pgfqpoint{-0.000000in}{0.000000in}}{%
\pgfpathmoveto{\pgfqpoint{-0.000000in}{0.000000in}}%
\pgfpathlineto{\pgfqpoint{-0.013889in}{0.000000in}}%
\pgfusepath{stroke,fill}%
}%
\begin{pgfscope}%
\pgfsys@transformshift{0.694444in}{0.700000in}%
\pgfsys@useobject{currentmarker}{}%
\end{pgfscope}%
\end{pgfscope}%
\begin{pgfscope}%
\definecolor{textcolor}{rgb}{0.000000,0.000000,0.000000}%
\pgfsetstrokecolor{textcolor}%
\pgfsetfillcolor{textcolor}%
\pgftext[x=0.520047in, y=0.672263in, left, base]{\color{textcolor}{\ifdefined\pdftexversion\else\setmainfont{NanumMyeongjo}\rmfamily\fi\fontsize{5.000000}{6.000000}\selectfont\catcode`\^=\active\def^{\ifmmode\sp\else\^{}\fi}\catcode`\%=\active\def%{\%}300}}%
\end{pgfscope}%
\begin{pgfscope}%
\pgfpathrectangle{\pgfqpoint{0.694444in}{0.416667in}}{\pgfqpoint{3.194444in}{1.416667in}}%
\pgfusepath{clip}%
\pgfsetbuttcap%
\pgfsetroundjoin%
\pgfsetlinewidth{0.602250pt}%
\definecolor{currentstroke}{rgb}{0.690196,0.690196,0.690196}%
\pgfsetstrokecolor{currentstroke}%
\pgfsetstrokeopacity{0.400000}%
\pgfsetdash{{2.220000pt}{0.960000pt}}{0.000000pt}%
\pgfpathmoveto{\pgfqpoint{0.694444in}{0.983333in}}%
\pgfpathlineto{\pgfqpoint{3.888889in}{0.983333in}}%
\pgfusepath{stroke}%
\end{pgfscope}%
\begin{pgfscope}%
\pgfsetbuttcap%
\pgfsetroundjoin%
\definecolor{currentfill}{rgb}{0.000000,0.000000,0.000000}%
\pgfsetfillcolor{currentfill}%
\pgfsetlinewidth{0.752812pt}%
\definecolor{currentstroke}{rgb}{0.000000,0.000000,0.000000}%
\pgfsetstrokecolor{currentstroke}%
\pgfsetdash{}{0pt}%
\pgfsys@defobject{currentmarker}{\pgfqpoint{-0.013889in}{0.000000in}}{\pgfqpoint{-0.000000in}{0.000000in}}{%
\pgfpathmoveto{\pgfqpoint{-0.000000in}{0.000000in}}%
\pgfpathlineto{\pgfqpoint{-0.013889in}{0.000000in}}%
\pgfusepath{stroke,fill}%
}%
\begin{pgfscope}%
\pgfsys@transformshift{0.694444in}{0.983333in}%
\pgfsys@useobject{currentmarker}{}%
\end{pgfscope}%
\end{pgfscope}%
\begin{pgfscope}%
\definecolor{textcolor}{rgb}{0.000000,0.000000,0.000000}%
\pgfsetstrokecolor{textcolor}%
\pgfsetfillcolor{textcolor}%
\pgftext[x=0.520047in, y=0.955596in, left, base]{\color{textcolor}{\ifdefined\pdftexversion\else\setmainfont{NanumMyeongjo}\rmfamily\fi\fontsize{5.000000}{6.000000}\selectfont\catcode`\^=\active\def^{\ifmmode\sp\else\^{}\fi}\catcode`\%=\active\def%{\%}600}}%
\end{pgfscope}%
\begin{pgfscope}%
\pgfpathrectangle{\pgfqpoint{0.694444in}{0.416667in}}{\pgfqpoint{3.194444in}{1.416667in}}%
\pgfusepath{clip}%
\pgfsetbuttcap%
\pgfsetroundjoin%
\pgfsetlinewidth{0.602250pt}%
\definecolor{currentstroke}{rgb}{0.690196,0.690196,0.690196}%
\pgfsetstrokecolor{currentstroke}%
\pgfsetstrokeopacity{0.400000}%
\pgfsetdash{{2.220000pt}{0.960000pt}}{0.000000pt}%
\pgfpathmoveto{\pgfqpoint{0.694444in}{1.266667in}}%
\pgfpathlineto{\pgfqpoint{3.888889in}{1.266667in}}%
\pgfusepath{stroke}%
\end{pgfscope}%
\begin{pgfscope}%
\pgfsetbuttcap%
\pgfsetroundjoin%
\definecolor{currentfill}{rgb}{0.000000,0.000000,0.000000}%
\pgfsetfillcolor{currentfill}%
\pgfsetlinewidth{0.752812pt}%
\definecolor{currentstroke}{rgb}{0.000000,0.000000,0.000000}%
\pgfsetstrokecolor{currentstroke}%
\pgfsetdash{}{0pt}%
\pgfsys@defobject{currentmarker}{\pgfqpoint{-0.013889in}{0.000000in}}{\pgfqpoint{-0.000000in}{0.000000in}}{%
\pgfpathmoveto{\pgfqpoint{-0.000000in}{0.000000in}}%
\pgfpathlineto{\pgfqpoint{-0.013889in}{0.000000in}}%
\pgfusepath{stroke,fill}%
}%
\begin{pgfscope}%
\pgfsys@transformshift{0.694444in}{1.266667in}%
\pgfsys@useobject{currentmarker}{}%
\end{pgfscope}%
\end{pgfscope}%
\begin{pgfscope}%
\definecolor{textcolor}{rgb}{0.000000,0.000000,0.000000}%
\pgfsetstrokecolor{textcolor}%
\pgfsetfillcolor{textcolor}%
\pgftext[x=0.520047in, y=1.238930in, left, base]{\color{textcolor}{\ifdefined\pdftexversion\else\setmainfont{NanumMyeongjo}\rmfamily\fi\fontsize{5.000000}{6.000000}\selectfont\catcode`\^=\active\def^{\ifmmode\sp\else\^{}\fi}\catcode`\%=\active\def%{\%}900}}%
\end{pgfscope}%
\begin{pgfscope}%
\pgfpathrectangle{\pgfqpoint{0.694444in}{0.416667in}}{\pgfqpoint{3.194444in}{1.416667in}}%
\pgfusepath{clip}%
\pgfsetbuttcap%
\pgfsetroundjoin%
\pgfsetlinewidth{0.602250pt}%
\definecolor{currentstroke}{rgb}{0.690196,0.690196,0.690196}%
\pgfsetstrokecolor{currentstroke}%
\pgfsetstrokeopacity{0.400000}%
\pgfsetdash{{2.220000pt}{0.960000pt}}{0.000000pt}%
\pgfpathmoveto{\pgfqpoint{0.694444in}{1.550000in}}%
\pgfpathlineto{\pgfqpoint{3.888889in}{1.550000in}}%
\pgfusepath{stroke}%
\end{pgfscope}%
\begin{pgfscope}%
\pgfsetbuttcap%
\pgfsetroundjoin%
\definecolor{currentfill}{rgb}{0.000000,0.000000,0.000000}%
\pgfsetfillcolor{currentfill}%
\pgfsetlinewidth{0.752812pt}%
\definecolor{currentstroke}{rgb}{0.000000,0.000000,0.000000}%
\pgfsetstrokecolor{currentstroke}%
\pgfsetdash{}{0pt}%
\pgfsys@defobject{currentmarker}{\pgfqpoint{-0.013889in}{0.000000in}}{\pgfqpoint{-0.000000in}{0.000000in}}{%
\pgfpathmoveto{\pgfqpoint{-0.000000in}{0.000000in}}%
\pgfpathlineto{\pgfqpoint{-0.013889in}{0.000000in}}%
\pgfusepath{stroke,fill}%
}%
\begin{pgfscope}%
\pgfsys@transformshift{0.694444in}{1.550000in}%
\pgfsys@useobject{currentmarker}{}%
\end{pgfscope}%
\end{pgfscope}%
\begin{pgfscope}%
\definecolor{textcolor}{rgb}{0.000000,0.000000,0.000000}%
\pgfsetstrokecolor{textcolor}%
\pgfsetfillcolor{textcolor}%
\pgftext[x=0.463759in, y=1.522263in, left, base]{\color{textcolor}{\ifdefined\pdftexversion\else\setmainfont{NanumMyeongjo}\rmfamily\fi\fontsize{5.000000}{6.000000}\selectfont\catcode`\^=\active\def^{\ifmmode\sp\else\^{}\fi}\catcode`\%=\active\def%{\%}1,200}}%
\end{pgfscope}%
\begin{pgfscope}%
\pgfpathrectangle{\pgfqpoint{0.694444in}{0.416667in}}{\pgfqpoint{3.194444in}{1.416667in}}%
\pgfusepath{clip}%
\pgfsetbuttcap%
\pgfsetroundjoin%
\pgfsetlinewidth{0.602250pt}%
\definecolor{currentstroke}{rgb}{0.690196,0.690196,0.690196}%
\pgfsetstrokecolor{currentstroke}%
\pgfsetstrokeopacity{0.400000}%
\pgfsetdash{{2.220000pt}{0.960000pt}}{0.000000pt}%
\pgfpathmoveto{\pgfqpoint{0.694444in}{1.833333in}}%
\pgfpathlineto{\pgfqpoint{3.888889in}{1.833333in}}%
\pgfusepath{stroke}%
\end{pgfscope}%
\begin{pgfscope}%
\pgfsetbuttcap%
\pgfsetroundjoin%
\definecolor{currentfill}{rgb}{0.000000,0.000000,0.000000}%
\pgfsetfillcolor{currentfill}%
\pgfsetlinewidth{0.752812pt}%
\definecolor{currentstroke}{rgb}{0.000000,0.000000,0.000000}%
\pgfsetstrokecolor{currentstroke}%
\pgfsetdash{}{0pt}%
\pgfsys@defobject{currentmarker}{\pgfqpoint{-0.013889in}{0.000000in}}{\pgfqpoint{-0.000000in}{0.000000in}}{%
\pgfpathmoveto{\pgfqpoint{-0.000000in}{0.000000in}}%
\pgfpathlineto{\pgfqpoint{-0.013889in}{0.000000in}}%
\pgfusepath{stroke,fill}%
}%
\begin{pgfscope}%
\pgfsys@transformshift{0.694444in}{1.833333in}%
\pgfsys@useobject{currentmarker}{}%
\end{pgfscope}%
\end{pgfscope}%
\begin{pgfscope}%
\definecolor{textcolor}{rgb}{0.000000,0.000000,0.000000}%
\pgfsetstrokecolor{textcolor}%
\pgfsetfillcolor{textcolor}%
\pgftext[x=0.463759in, y=1.805596in, left, base]{\color{textcolor}{\ifdefined\pdftexversion\else\setmainfont{NanumMyeongjo}\rmfamily\fi\fontsize{5.000000}{6.000000}\selectfont\catcode`\^=\active\def^{\ifmmode\sp\else\^{}\fi}\catcode`\%=\active\def%{\%}1,500}}%
\end{pgfscope}%
\begin{pgfscope}%
\pgfsetrectcap%
\pgfsetmiterjoin%
\pgfsetlinewidth{0.752812pt}%
\definecolor{currentstroke}{rgb}{0.000000,0.000000,0.000000}%
\pgfsetstrokecolor{currentstroke}%
\pgfsetdash{}{0pt}%
\pgfpathmoveto{\pgfqpoint{0.694444in}{0.416667in}}%
\pgfpathlineto{\pgfqpoint{0.694444in}{1.833333in}}%
\pgfusepath{stroke}%
\end{pgfscope}%
\begin{pgfscope}%
\pgfsetrectcap%
\pgfsetmiterjoin%
\pgfsetlinewidth{0.752812pt}%
\definecolor{currentstroke}{rgb}{0.000000,0.000000,0.000000}%
\pgfsetstrokecolor{currentstroke}%
\pgfsetdash{}{0pt}%
\pgfpathmoveto{\pgfqpoint{0.694444in}{0.416667in}}%
\pgfpathlineto{\pgfqpoint{3.888889in}{0.416667in}}%
\pgfusepath{stroke}%
\end{pgfscope}%
\begin{pgfscope}%
\pgfpathrectangle{\pgfqpoint{0.694444in}{0.416667in}}{\pgfqpoint{3.194444in}{1.416667in}}%
\pgfusepath{clip}%
\pgfsetbuttcap%
\pgfsetmiterjoin%
\definecolor{currentfill}{rgb}{0.235294,0.490196,0.764706}%
\pgfsetfillcolor{currentfill}%
\pgfsetlinewidth{1.003750pt}%
\definecolor{currentstroke}{rgb}{0.266667,0.266667,0.266667}%
\pgfsetstrokecolor{currentstroke}%
\pgfsetdash{}{0pt}%
\pgfpathmoveto{\pgfqpoint{0.839646in}{0.416667in}}%
\pgfpathlineto{\pgfqpoint{1.013392in}{0.416667in}}%
\pgfpathlineto{\pgfqpoint{1.013392in}{0.938879in}}%
\pgfpathlineto{\pgfqpoint{0.839646in}{0.938879in}}%
\pgfpathlineto{\pgfqpoint{0.839646in}{0.416667in}}%
\pgfpathclose%
\pgfusepath{stroke,fill}%
\end{pgfscope}%
\begin{pgfscope}%
\pgfpathrectangle{\pgfqpoint{0.694444in}{0.416667in}}{\pgfqpoint{3.194444in}{1.416667in}}%
\pgfusepath{clip}%
\pgfsetbuttcap%
\pgfsetmiterjoin%
\definecolor{currentfill}{rgb}{0.235294,0.490196,0.764706}%
\pgfsetfillcolor{currentfill}%
\pgfsetlinewidth{1.003750pt}%
\definecolor{currentstroke}{rgb}{0.266667,0.266667,0.266667}%
\pgfsetstrokecolor{currentstroke}%
\pgfsetdash{}{0pt}%
\pgfpathmoveto{\pgfqpoint{1.087855in}{0.416667in}}%
\pgfpathlineto{\pgfqpoint{1.261601in}{0.416667in}}%
\pgfpathlineto{\pgfqpoint{1.261601in}{0.883661in}}%
\pgfpathlineto{\pgfqpoint{1.087855in}{0.883661in}}%
\pgfpathlineto{\pgfqpoint{1.087855in}{0.416667in}}%
\pgfpathclose%
\pgfusepath{stroke,fill}%
\end{pgfscope}%
\begin{pgfscope}%
\pgfpathrectangle{\pgfqpoint{0.694444in}{0.416667in}}{\pgfqpoint{3.194444in}{1.416667in}}%
\pgfusepath{clip}%
\pgfsetbuttcap%
\pgfsetmiterjoin%
\definecolor{currentfill}{rgb}{0.235294,0.490196,0.764706}%
\pgfsetfillcolor{currentfill}%
\pgfsetlinewidth{1.003750pt}%
\definecolor{currentstroke}{rgb}{0.266667,0.266667,0.266667}%
\pgfsetstrokecolor{currentstroke}%
\pgfsetdash{}{0pt}%
\pgfpathmoveto{\pgfqpoint{1.336064in}{0.416667in}}%
\pgfpathlineto{\pgfqpoint{1.509810in}{0.416667in}}%
\pgfpathlineto{\pgfqpoint{1.509810in}{0.939058in}}%
\pgfpathlineto{\pgfqpoint{1.336064in}{0.939058in}}%
\pgfpathlineto{\pgfqpoint{1.336064in}{0.416667in}}%
\pgfpathclose%
\pgfusepath{stroke,fill}%
\end{pgfscope}%
\begin{pgfscope}%
\pgfpathrectangle{\pgfqpoint{0.694444in}{0.416667in}}{\pgfqpoint{3.194444in}{1.416667in}}%
\pgfusepath{clip}%
\pgfsetbuttcap%
\pgfsetmiterjoin%
\definecolor{currentfill}{rgb}{0.235294,0.490196,0.764706}%
\pgfsetfillcolor{currentfill}%
\pgfsetlinewidth{1.003750pt}%
\definecolor{currentstroke}{rgb}{0.266667,0.266667,0.266667}%
\pgfsetstrokecolor{currentstroke}%
\pgfsetdash{}{0pt}%
\pgfpathmoveto{\pgfqpoint{1.584272in}{0.416667in}}%
\pgfpathlineto{\pgfqpoint{1.758018in}{0.416667in}}%
\pgfpathlineto{\pgfqpoint{1.758018in}{0.990999in}}%
\pgfpathlineto{\pgfqpoint{1.584272in}{0.990999in}}%
\pgfpathlineto{\pgfqpoint{1.584272in}{0.416667in}}%
\pgfpathclose%
\pgfusepath{stroke,fill}%
\end{pgfscope}%
\begin{pgfscope}%
\pgfpathrectangle{\pgfqpoint{0.694444in}{0.416667in}}{\pgfqpoint{3.194444in}{1.416667in}}%
\pgfusepath{clip}%
\pgfsetbuttcap%
\pgfsetmiterjoin%
\definecolor{currentfill}{rgb}{0.235294,0.490196,0.764706}%
\pgfsetfillcolor{currentfill}%
\pgfsetlinewidth{1.003750pt}%
\definecolor{currentstroke}{rgb}{0.266667,0.266667,0.266667}%
\pgfsetstrokecolor{currentstroke}%
\pgfsetdash{}{0pt}%
\pgfpathmoveto{\pgfqpoint{1.832481in}{0.416667in}}%
\pgfpathlineto{\pgfqpoint{2.006227in}{0.416667in}}%
\pgfpathlineto{\pgfqpoint{2.006227in}{0.916890in}}%
\pgfpathlineto{\pgfqpoint{1.832481in}{0.916890in}}%
\pgfpathlineto{\pgfqpoint{1.832481in}{0.416667in}}%
\pgfpathclose%
\pgfusepath{stroke,fill}%
\end{pgfscope}%
\begin{pgfscope}%
\pgfpathrectangle{\pgfqpoint{0.694444in}{0.416667in}}{\pgfqpoint{3.194444in}{1.416667in}}%
\pgfusepath{clip}%
\pgfsetbuttcap%
\pgfsetmiterjoin%
\definecolor{currentfill}{rgb}{0.235294,0.490196,0.764706}%
\pgfsetfillcolor{currentfill}%
\pgfsetlinewidth{1.003750pt}%
\definecolor{currentstroke}{rgb}{0.266667,0.266667,0.266667}%
\pgfsetstrokecolor{currentstroke}%
\pgfsetdash{}{0pt}%
\pgfpathmoveto{\pgfqpoint{2.080689in}{0.416667in}}%
\pgfpathlineto{\pgfqpoint{2.254435in}{0.416667in}}%
\pgfpathlineto{\pgfqpoint{2.254435in}{1.002199in}}%
\pgfpathlineto{\pgfqpoint{2.080689in}{1.002199in}}%
\pgfpathlineto{\pgfqpoint{2.080689in}{0.416667in}}%
\pgfpathclose%
\pgfusepath{stroke,fill}%
\end{pgfscope}%
\begin{pgfscope}%
\pgfpathrectangle{\pgfqpoint{0.694444in}{0.416667in}}{\pgfqpoint{3.194444in}{1.416667in}}%
\pgfusepath{clip}%
\pgfsetbuttcap%
\pgfsetmiterjoin%
\definecolor{currentfill}{rgb}{0.235294,0.490196,0.764706}%
\pgfsetfillcolor{currentfill}%
\pgfsetlinewidth{1.003750pt}%
\definecolor{currentstroke}{rgb}{0.266667,0.266667,0.266667}%
\pgfsetstrokecolor{currentstroke}%
\pgfsetdash{}{0pt}%
\pgfpathmoveto{\pgfqpoint{2.328898in}{0.416667in}}%
\pgfpathlineto{\pgfqpoint{2.502644in}{0.416667in}}%
\pgfpathlineto{\pgfqpoint{2.502644in}{0.968436in}}%
\pgfpathlineto{\pgfqpoint{2.328898in}{0.968436in}}%
\pgfpathlineto{\pgfqpoint{2.328898in}{0.416667in}}%
\pgfpathclose%
\pgfusepath{stroke,fill}%
\end{pgfscope}%
\begin{pgfscope}%
\pgfpathrectangle{\pgfqpoint{0.694444in}{0.416667in}}{\pgfqpoint{3.194444in}{1.416667in}}%
\pgfusepath{clip}%
\pgfsetbuttcap%
\pgfsetmiterjoin%
\definecolor{currentfill}{rgb}{0.235294,0.490196,0.764706}%
\pgfsetfillcolor{currentfill}%
\pgfsetlinewidth{1.003750pt}%
\definecolor{currentstroke}{rgb}{0.266667,0.266667,0.266667}%
\pgfsetstrokecolor{currentstroke}%
\pgfsetdash{}{0pt}%
\pgfpathmoveto{\pgfqpoint{2.577107in}{0.416667in}}%
\pgfpathlineto{\pgfqpoint{2.750853in}{0.416667in}}%
\pgfpathlineto{\pgfqpoint{2.750853in}{1.071590in}}%
\pgfpathlineto{\pgfqpoint{2.577107in}{1.071590in}}%
\pgfpathlineto{\pgfqpoint{2.577107in}{0.416667in}}%
\pgfpathclose%
\pgfusepath{stroke,fill}%
\end{pgfscope}%
\begin{pgfscope}%
\pgfpathrectangle{\pgfqpoint{0.694444in}{0.416667in}}{\pgfqpoint{3.194444in}{1.416667in}}%
\pgfusepath{clip}%
\pgfsetbuttcap%
\pgfsetmiterjoin%
\definecolor{currentfill}{rgb}{0.235294,0.490196,0.764706}%
\pgfsetfillcolor{currentfill}%
\pgfsetlinewidth{1.003750pt}%
\definecolor{currentstroke}{rgb}{0.266667,0.266667,0.266667}%
\pgfsetstrokecolor{currentstroke}%
\pgfsetdash{}{0pt}%
\pgfpathmoveto{\pgfqpoint{2.825315in}{0.416667in}}%
\pgfpathlineto{\pgfqpoint{2.999061in}{0.416667in}}%
\pgfpathlineto{\pgfqpoint{2.999061in}{1.431396in}}%
\pgfpathlineto{\pgfqpoint{2.825315in}{1.431396in}}%
\pgfpathlineto{\pgfqpoint{2.825315in}{0.416667in}}%
\pgfpathclose%
\pgfusepath{stroke,fill}%
\end{pgfscope}%
\begin{pgfscope}%
\pgfpathrectangle{\pgfqpoint{0.694444in}{0.416667in}}{\pgfqpoint{3.194444in}{1.416667in}}%
\pgfusepath{clip}%
\pgfsetbuttcap%
\pgfsetmiterjoin%
\definecolor{currentfill}{rgb}{0.235294,0.490196,0.764706}%
\pgfsetfillcolor{currentfill}%
\pgfsetlinewidth{1.003750pt}%
\definecolor{currentstroke}{rgb}{0.266667,0.266667,0.266667}%
\pgfsetstrokecolor{currentstroke}%
\pgfsetdash{}{0pt}%
\pgfpathmoveto{\pgfqpoint{3.073524in}{0.416667in}}%
\pgfpathlineto{\pgfqpoint{3.247270in}{0.416667in}}%
\pgfpathlineto{\pgfqpoint{3.247270in}{1.030431in}}%
\pgfpathlineto{\pgfqpoint{3.073524in}{1.030431in}}%
\pgfpathlineto{\pgfqpoint{3.073524in}{0.416667in}}%
\pgfpathclose%
\pgfusepath{stroke,fill}%
\end{pgfscope}%
\begin{pgfscope}%
\pgfpathrectangle{\pgfqpoint{0.694444in}{0.416667in}}{\pgfqpoint{3.194444in}{1.416667in}}%
\pgfusepath{clip}%
\pgfsetbuttcap%
\pgfsetmiterjoin%
\definecolor{currentfill}{rgb}{0.235294,0.490196,0.764706}%
\pgfsetfillcolor{currentfill}%
\pgfsetlinewidth{1.003750pt}%
\definecolor{currentstroke}{rgb}{0.266667,0.266667,0.266667}%
\pgfsetstrokecolor{currentstroke}%
\pgfsetdash{}{0pt}%
\pgfpathmoveto{\pgfqpoint{3.321732in}{0.416667in}}%
\pgfpathlineto{\pgfqpoint{3.495478in}{0.416667in}}%
\pgfpathlineto{\pgfqpoint{3.495478in}{0.932198in}}%
\pgfpathlineto{\pgfqpoint{3.321732in}{0.932198in}}%
\pgfpathlineto{\pgfqpoint{3.321732in}{0.416667in}}%
\pgfpathclose%
\pgfusepath{stroke,fill}%
\end{pgfscope}%
\begin{pgfscope}%
\pgfpathrectangle{\pgfqpoint{0.694444in}{0.416667in}}{\pgfqpoint{3.194444in}{1.416667in}}%
\pgfusepath{clip}%
\pgfsetbuttcap%
\pgfsetmiterjoin%
\definecolor{currentfill}{rgb}{0.235294,0.490196,0.764706}%
\pgfsetfillcolor{currentfill}%
\pgfsetlinewidth{1.003750pt}%
\definecolor{currentstroke}{rgb}{0.266667,0.266667,0.266667}%
\pgfsetstrokecolor{currentstroke}%
\pgfsetdash{}{0pt}%
\pgfpathmoveto{\pgfqpoint{3.569941in}{0.416667in}}%
\pgfpathlineto{\pgfqpoint{3.743687in}{0.416667in}}%
\pgfpathlineto{\pgfqpoint{3.743687in}{1.016839in}}%
\pgfpathlineto{\pgfqpoint{3.569941in}{1.016839in}}%
\pgfpathlineto{\pgfqpoint{3.569941in}{0.416667in}}%
\pgfpathclose%
\pgfusepath{stroke,fill}%
\end{pgfscope}%
\begin{pgfscope}%
\pgfpathrectangle{\pgfqpoint{0.694444in}{0.416667in}}{\pgfqpoint{3.194444in}{1.416667in}}%
\pgfusepath{clip}%
\pgfsetbuttcap%
\pgfsetmiterjoin%
\definecolor{currentfill}{rgb}{0.337255,0.713725,0.627451}%
\pgfsetfillcolor{currentfill}%
\pgfsetlinewidth{1.003750pt}%
\definecolor{currentstroke}{rgb}{0.266667,0.266667,0.266667}%
\pgfsetstrokecolor{currentstroke}%
\pgfsetdash{}{0pt}%
\pgfpathmoveto{\pgfqpoint{0.839646in}{0.938879in}}%
\pgfpathlineto{\pgfqpoint{1.013392in}{0.938879in}}%
\pgfpathlineto{\pgfqpoint{1.013392in}{1.303507in}}%
\pgfpathlineto{\pgfqpoint{0.839646in}{1.303507in}}%
\pgfpathlineto{\pgfqpoint{0.839646in}{0.938879in}}%
\pgfpathclose%
\pgfusepath{stroke,fill}%
\end{pgfscope}%
\begin{pgfscope}%
\pgfpathrectangle{\pgfqpoint{0.694444in}{0.416667in}}{\pgfqpoint{3.194444in}{1.416667in}}%
\pgfusepath{clip}%
\pgfsetbuttcap%
\pgfsetmiterjoin%
\definecolor{currentfill}{rgb}{0.337255,0.713725,0.627451}%
\pgfsetfillcolor{currentfill}%
\pgfsetlinewidth{1.003750pt}%
\definecolor{currentstroke}{rgb}{0.266667,0.266667,0.266667}%
\pgfsetstrokecolor{currentstroke}%
\pgfsetdash{}{0pt}%
\pgfpathmoveto{\pgfqpoint{1.087855in}{0.883661in}}%
\pgfpathlineto{\pgfqpoint{1.261601in}{0.883661in}}%
\pgfpathlineto{\pgfqpoint{1.261601in}{1.222495in}}%
\pgfpathlineto{\pgfqpoint{1.087855in}{1.222495in}}%
\pgfpathlineto{\pgfqpoint{1.087855in}{0.883661in}}%
\pgfpathclose%
\pgfusepath{stroke,fill}%
\end{pgfscope}%
\begin{pgfscope}%
\pgfpathrectangle{\pgfqpoint{0.694444in}{0.416667in}}{\pgfqpoint{3.194444in}{1.416667in}}%
\pgfusepath{clip}%
\pgfsetbuttcap%
\pgfsetmiterjoin%
\definecolor{currentfill}{rgb}{0.337255,0.713725,0.627451}%
\pgfsetfillcolor{currentfill}%
\pgfsetlinewidth{1.003750pt}%
\definecolor{currentstroke}{rgb}{0.266667,0.266667,0.266667}%
\pgfsetstrokecolor{currentstroke}%
\pgfsetdash{}{0pt}%
\pgfpathmoveto{\pgfqpoint{1.336064in}{0.939058in}}%
\pgfpathlineto{\pgfqpoint{1.509810in}{0.939058in}}%
\pgfpathlineto{\pgfqpoint{1.509810in}{1.313913in}}%
\pgfpathlineto{\pgfqpoint{1.336064in}{1.313913in}}%
\pgfpathlineto{\pgfqpoint{1.336064in}{0.939058in}}%
\pgfpathclose%
\pgfusepath{stroke,fill}%
\end{pgfscope}%
\begin{pgfscope}%
\pgfpathrectangle{\pgfqpoint{0.694444in}{0.416667in}}{\pgfqpoint{3.194444in}{1.416667in}}%
\pgfusepath{clip}%
\pgfsetbuttcap%
\pgfsetmiterjoin%
\definecolor{currentfill}{rgb}{0.337255,0.713725,0.627451}%
\pgfsetfillcolor{currentfill}%
\pgfsetlinewidth{1.003750pt}%
\definecolor{currentstroke}{rgb}{0.266667,0.266667,0.266667}%
\pgfsetstrokecolor{currentstroke}%
\pgfsetdash{}{0pt}%
\pgfpathmoveto{\pgfqpoint{1.584272in}{0.990999in}}%
\pgfpathlineto{\pgfqpoint{1.758018in}{0.990999in}}%
\pgfpathlineto{\pgfqpoint{1.758018in}{1.430648in}}%
\pgfpathlineto{\pgfqpoint{1.584272in}{1.430648in}}%
\pgfpathlineto{\pgfqpoint{1.584272in}{0.990999in}}%
\pgfpathclose%
\pgfusepath{stroke,fill}%
\end{pgfscope}%
\begin{pgfscope}%
\pgfpathrectangle{\pgfqpoint{0.694444in}{0.416667in}}{\pgfqpoint{3.194444in}{1.416667in}}%
\pgfusepath{clip}%
\pgfsetbuttcap%
\pgfsetmiterjoin%
\definecolor{currentfill}{rgb}{0.337255,0.713725,0.627451}%
\pgfsetfillcolor{currentfill}%
\pgfsetlinewidth{1.003750pt}%
\definecolor{currentstroke}{rgb}{0.266667,0.266667,0.266667}%
\pgfsetstrokecolor{currentstroke}%
\pgfsetdash{}{0pt}%
\pgfpathmoveto{\pgfqpoint{1.832481in}{0.916890in}}%
\pgfpathlineto{\pgfqpoint{2.006227in}{0.916890in}}%
\pgfpathlineto{\pgfqpoint{2.006227in}{1.621270in}}%
\pgfpathlineto{\pgfqpoint{1.832481in}{1.621270in}}%
\pgfpathlineto{\pgfqpoint{1.832481in}{0.916890in}}%
\pgfpathclose%
\pgfusepath{stroke,fill}%
\end{pgfscope}%
\begin{pgfscope}%
\pgfpathrectangle{\pgfqpoint{0.694444in}{0.416667in}}{\pgfqpoint{3.194444in}{1.416667in}}%
\pgfusepath{clip}%
\pgfsetbuttcap%
\pgfsetmiterjoin%
\definecolor{currentfill}{rgb}{0.337255,0.713725,0.627451}%
\pgfsetfillcolor{currentfill}%
\pgfsetlinewidth{1.003750pt}%
\definecolor{currentstroke}{rgb}{0.266667,0.266667,0.266667}%
\pgfsetstrokecolor{currentstroke}%
\pgfsetdash{}{0pt}%
\pgfpathmoveto{\pgfqpoint{2.080689in}{1.002199in}}%
\pgfpathlineto{\pgfqpoint{2.254435in}{1.002199in}}%
\pgfpathlineto{\pgfqpoint{2.254435in}{1.459020in}}%
\pgfpathlineto{\pgfqpoint{2.080689in}{1.459020in}}%
\pgfpathlineto{\pgfqpoint{2.080689in}{1.002199in}}%
\pgfpathclose%
\pgfusepath{stroke,fill}%
\end{pgfscope}%
\begin{pgfscope}%
\pgfpathrectangle{\pgfqpoint{0.694444in}{0.416667in}}{\pgfqpoint{3.194444in}{1.416667in}}%
\pgfusepath{clip}%
\pgfsetbuttcap%
\pgfsetmiterjoin%
\definecolor{currentfill}{rgb}{0.337255,0.713725,0.627451}%
\pgfsetfillcolor{currentfill}%
\pgfsetlinewidth{1.003750pt}%
\definecolor{currentstroke}{rgb}{0.266667,0.266667,0.266667}%
\pgfsetstrokecolor{currentstroke}%
\pgfsetdash{}{0pt}%
\pgfpathmoveto{\pgfqpoint{2.328898in}{0.968436in}}%
\pgfpathlineto{\pgfqpoint{2.502644in}{0.968436in}}%
\pgfpathlineto{\pgfqpoint{2.502644in}{1.436810in}}%
\pgfpathlineto{\pgfqpoint{2.328898in}{1.436810in}}%
\pgfpathlineto{\pgfqpoint{2.328898in}{0.968436in}}%
\pgfpathclose%
\pgfusepath{stroke,fill}%
\end{pgfscope}%
\begin{pgfscope}%
\pgfpathrectangle{\pgfqpoint{0.694444in}{0.416667in}}{\pgfqpoint{3.194444in}{1.416667in}}%
\pgfusepath{clip}%
\pgfsetbuttcap%
\pgfsetmiterjoin%
\definecolor{currentfill}{rgb}{0.337255,0.713725,0.627451}%
\pgfsetfillcolor{currentfill}%
\pgfsetlinewidth{1.003750pt}%
\definecolor{currentstroke}{rgb}{0.266667,0.266667,0.266667}%
\pgfsetstrokecolor{currentstroke}%
\pgfsetdash{}{0pt}%
\pgfpathmoveto{\pgfqpoint{2.577107in}{1.071590in}}%
\pgfpathlineto{\pgfqpoint{2.750853in}{1.071590in}}%
\pgfpathlineto{\pgfqpoint{2.750853in}{1.517632in}}%
\pgfpathlineto{\pgfqpoint{2.577107in}{1.517632in}}%
\pgfpathlineto{\pgfqpoint{2.577107in}{1.071590in}}%
\pgfpathclose%
\pgfusepath{stroke,fill}%
\end{pgfscope}%
\begin{pgfscope}%
\pgfpathrectangle{\pgfqpoint{0.694444in}{0.416667in}}{\pgfqpoint{3.194444in}{1.416667in}}%
\pgfusepath{clip}%
\pgfsetbuttcap%
\pgfsetmiterjoin%
\definecolor{currentfill}{rgb}{0.337255,0.713725,0.627451}%
\pgfsetfillcolor{currentfill}%
\pgfsetlinewidth{1.003750pt}%
\definecolor{currentstroke}{rgb}{0.266667,0.266667,0.266667}%
\pgfsetstrokecolor{currentstroke}%
\pgfsetdash{}{0pt}%
\pgfpathmoveto{\pgfqpoint{2.825315in}{1.431396in}}%
\pgfpathlineto{\pgfqpoint{2.999061in}{1.431396in}}%
\pgfpathlineto{\pgfqpoint{2.999061in}{1.543325in}}%
\pgfpathlineto{\pgfqpoint{2.825315in}{1.543325in}}%
\pgfpathlineto{\pgfqpoint{2.825315in}{1.431396in}}%
\pgfpathclose%
\pgfusepath{stroke,fill}%
\end{pgfscope}%
\begin{pgfscope}%
\pgfpathrectangle{\pgfqpoint{0.694444in}{0.416667in}}{\pgfqpoint{3.194444in}{1.416667in}}%
\pgfusepath{clip}%
\pgfsetbuttcap%
\pgfsetmiterjoin%
\definecolor{currentfill}{rgb}{0.337255,0.713725,0.627451}%
\pgfsetfillcolor{currentfill}%
\pgfsetlinewidth{1.003750pt}%
\definecolor{currentstroke}{rgb}{0.266667,0.266667,0.266667}%
\pgfsetstrokecolor{currentstroke}%
\pgfsetdash{}{0pt}%
\pgfpathmoveto{\pgfqpoint{3.073524in}{1.030431in}}%
\pgfpathlineto{\pgfqpoint{3.247270in}{1.030431in}}%
\pgfpathlineto{\pgfqpoint{3.247270in}{1.609003in}}%
\pgfpathlineto{\pgfqpoint{3.073524in}{1.609003in}}%
\pgfpathlineto{\pgfqpoint{3.073524in}{1.030431in}}%
\pgfpathclose%
\pgfusepath{stroke,fill}%
\end{pgfscope}%
\begin{pgfscope}%
\pgfpathrectangle{\pgfqpoint{0.694444in}{0.416667in}}{\pgfqpoint{3.194444in}{1.416667in}}%
\pgfusepath{clip}%
\pgfsetbuttcap%
\pgfsetmiterjoin%
\definecolor{currentfill}{rgb}{0.337255,0.713725,0.627451}%
\pgfsetfillcolor{currentfill}%
\pgfsetlinewidth{1.003750pt}%
\definecolor{currentstroke}{rgb}{0.266667,0.266667,0.266667}%
\pgfsetstrokecolor{currentstroke}%
\pgfsetdash{}{0pt}%
\pgfpathmoveto{\pgfqpoint{3.321732in}{0.932198in}}%
\pgfpathlineto{\pgfqpoint{3.495478in}{0.932198in}}%
\pgfpathlineto{\pgfqpoint{3.495478in}{1.550686in}}%
\pgfpathlineto{\pgfqpoint{3.321732in}{1.550686in}}%
\pgfpathlineto{\pgfqpoint{3.321732in}{0.932198in}}%
\pgfpathclose%
\pgfusepath{stroke,fill}%
\end{pgfscope}%
\begin{pgfscope}%
\pgfpathrectangle{\pgfqpoint{0.694444in}{0.416667in}}{\pgfqpoint{3.194444in}{1.416667in}}%
\pgfusepath{clip}%
\pgfsetbuttcap%
\pgfsetmiterjoin%
\definecolor{currentfill}{rgb}{0.337255,0.713725,0.627451}%
\pgfsetfillcolor{currentfill}%
\pgfsetlinewidth{1.003750pt}%
\definecolor{currentstroke}{rgb}{0.266667,0.266667,0.266667}%
\pgfsetstrokecolor{currentstroke}%
\pgfsetdash{}{0pt}%
\pgfpathmoveto{\pgfqpoint{3.569941in}{1.016839in}}%
\pgfpathlineto{\pgfqpoint{3.743687in}{1.016839in}}%
\pgfpathlineto{\pgfqpoint{3.743687in}{1.525633in}}%
\pgfpathlineto{\pgfqpoint{3.569941in}{1.525633in}}%
\pgfpathlineto{\pgfqpoint{3.569941in}{1.016839in}}%
\pgfpathclose%
\pgfusepath{stroke,fill}%
\end{pgfscope}%
\begin{pgfscope}%
\pgfpathrectangle{\pgfqpoint{0.694444in}{0.416667in}}{\pgfqpoint{3.194444in}{1.416667in}}%
\pgfusepath{clip}%
\pgfsetbuttcap%
\pgfsetmiterjoin%
\definecolor{currentfill}{rgb}{0.725490,0.486275,0.164706}%
\pgfsetfillcolor{currentfill}%
\pgfsetlinewidth{1.003750pt}%
\definecolor{currentstroke}{rgb}{0.266667,0.266667,0.266667}%
\pgfsetstrokecolor{currentstroke}%
\pgfsetdash{}{0pt}%
\pgfpathmoveto{\pgfqpoint{0.839646in}{1.303507in}}%
\pgfpathlineto{\pgfqpoint{1.013392in}{1.303507in}}%
\pgfpathlineto{\pgfqpoint{1.013392in}{1.403350in}}%
\pgfpathlineto{\pgfqpoint{0.839646in}{1.403350in}}%
\pgfpathlineto{\pgfqpoint{0.839646in}{1.303507in}}%
\pgfpathclose%
\pgfusepath{stroke,fill}%
\end{pgfscope}%
\begin{pgfscope}%
\pgfpathrectangle{\pgfqpoint{0.694444in}{0.416667in}}{\pgfqpoint{3.194444in}{1.416667in}}%
\pgfusepath{clip}%
\pgfsetbuttcap%
\pgfsetmiterjoin%
\definecolor{currentfill}{rgb}{0.725490,0.486275,0.164706}%
\pgfsetfillcolor{currentfill}%
\pgfsetlinewidth{1.003750pt}%
\definecolor{currentstroke}{rgb}{0.266667,0.266667,0.266667}%
\pgfsetstrokecolor{currentstroke}%
\pgfsetdash{}{0pt}%
\pgfpathmoveto{\pgfqpoint{1.087855in}{1.222495in}}%
\pgfpathlineto{\pgfqpoint{1.261601in}{1.222495in}}%
\pgfpathlineto{\pgfqpoint{1.261601in}{1.337215in}}%
\pgfpathlineto{\pgfqpoint{1.087855in}{1.337215in}}%
\pgfpathlineto{\pgfqpoint{1.087855in}{1.222495in}}%
\pgfpathclose%
\pgfusepath{stroke,fill}%
\end{pgfscope}%
\begin{pgfscope}%
\pgfpathrectangle{\pgfqpoint{0.694444in}{0.416667in}}{\pgfqpoint{3.194444in}{1.416667in}}%
\pgfusepath{clip}%
\pgfsetbuttcap%
\pgfsetmiterjoin%
\definecolor{currentfill}{rgb}{0.725490,0.486275,0.164706}%
\pgfsetfillcolor{currentfill}%
\pgfsetlinewidth{1.003750pt}%
\definecolor{currentstroke}{rgb}{0.266667,0.266667,0.266667}%
\pgfsetstrokecolor{currentstroke}%
\pgfsetdash{}{0pt}%
\pgfpathmoveto{\pgfqpoint{1.336064in}{1.313913in}}%
\pgfpathlineto{\pgfqpoint{1.509810in}{1.313913in}}%
\pgfpathlineto{\pgfqpoint{1.509810in}{1.363305in}}%
\pgfpathlineto{\pgfqpoint{1.336064in}{1.363305in}}%
\pgfpathlineto{\pgfqpoint{1.336064in}{1.313913in}}%
\pgfpathclose%
\pgfusepath{stroke,fill}%
\end{pgfscope}%
\begin{pgfscope}%
\pgfpathrectangle{\pgfqpoint{0.694444in}{0.416667in}}{\pgfqpoint{3.194444in}{1.416667in}}%
\pgfusepath{clip}%
\pgfsetbuttcap%
\pgfsetmiterjoin%
\definecolor{currentfill}{rgb}{0.725490,0.486275,0.164706}%
\pgfsetfillcolor{currentfill}%
\pgfsetlinewidth{1.003750pt}%
\definecolor{currentstroke}{rgb}{0.266667,0.266667,0.266667}%
\pgfsetstrokecolor{currentstroke}%
\pgfsetdash{}{0pt}%
\pgfpathmoveto{\pgfqpoint{1.584272in}{1.430648in}}%
\pgfpathlineto{\pgfqpoint{1.758018in}{1.430648in}}%
\pgfpathlineto{\pgfqpoint{1.758018in}{1.498981in}}%
\pgfpathlineto{\pgfqpoint{1.584272in}{1.498981in}}%
\pgfpathlineto{\pgfqpoint{1.584272in}{1.430648in}}%
\pgfpathclose%
\pgfusepath{stroke,fill}%
\end{pgfscope}%
\begin{pgfscope}%
\pgfpathrectangle{\pgfqpoint{0.694444in}{0.416667in}}{\pgfqpoint{3.194444in}{1.416667in}}%
\pgfusepath{clip}%
\pgfsetbuttcap%
\pgfsetmiterjoin%
\definecolor{currentfill}{rgb}{0.725490,0.486275,0.164706}%
\pgfsetfillcolor{currentfill}%
\pgfsetlinewidth{1.003750pt}%
\definecolor{currentstroke}{rgb}{0.266667,0.266667,0.266667}%
\pgfsetstrokecolor{currentstroke}%
\pgfsetdash{}{0pt}%
\pgfpathmoveto{\pgfqpoint{1.832481in}{1.621270in}}%
\pgfpathlineto{\pgfqpoint{2.006227in}{1.621270in}}%
\pgfpathlineto{\pgfqpoint{2.006227in}{1.644287in}}%
\pgfpathlineto{\pgfqpoint{1.832481in}{1.644287in}}%
\pgfpathlineto{\pgfqpoint{1.832481in}{1.621270in}}%
\pgfpathclose%
\pgfusepath{stroke,fill}%
\end{pgfscope}%
\begin{pgfscope}%
\pgfpathrectangle{\pgfqpoint{0.694444in}{0.416667in}}{\pgfqpoint{3.194444in}{1.416667in}}%
\pgfusepath{clip}%
\pgfsetbuttcap%
\pgfsetmiterjoin%
\definecolor{currentfill}{rgb}{0.725490,0.486275,0.164706}%
\pgfsetfillcolor{currentfill}%
\pgfsetlinewidth{1.003750pt}%
\definecolor{currentstroke}{rgb}{0.266667,0.266667,0.266667}%
\pgfsetstrokecolor{currentstroke}%
\pgfsetdash{}{0pt}%
\pgfpathmoveto{\pgfqpoint{2.080689in}{1.459020in}}%
\pgfpathlineto{\pgfqpoint{2.254435in}{1.459020in}}%
\pgfpathlineto{\pgfqpoint{2.254435in}{1.491256in}}%
\pgfpathlineto{\pgfqpoint{2.080689in}{1.491256in}}%
\pgfpathlineto{\pgfqpoint{2.080689in}{1.459020in}}%
\pgfpathclose%
\pgfusepath{stroke,fill}%
\end{pgfscope}%
\begin{pgfscope}%
\pgfpathrectangle{\pgfqpoint{0.694444in}{0.416667in}}{\pgfqpoint{3.194444in}{1.416667in}}%
\pgfusepath{clip}%
\pgfsetbuttcap%
\pgfsetmiterjoin%
\definecolor{currentfill}{rgb}{0.725490,0.486275,0.164706}%
\pgfsetfillcolor{currentfill}%
\pgfsetlinewidth{1.003750pt}%
\definecolor{currentstroke}{rgb}{0.266667,0.266667,0.266667}%
\pgfsetstrokecolor{currentstroke}%
\pgfsetdash{}{0pt}%
\pgfpathmoveto{\pgfqpoint{2.328898in}{1.436810in}}%
\pgfpathlineto{\pgfqpoint{2.502644in}{1.436810in}}%
\pgfpathlineto{\pgfqpoint{2.502644in}{1.473668in}}%
\pgfpathlineto{\pgfqpoint{2.328898in}{1.473668in}}%
\pgfpathlineto{\pgfqpoint{2.328898in}{1.436810in}}%
\pgfpathclose%
\pgfusepath{stroke,fill}%
\end{pgfscope}%
\begin{pgfscope}%
\pgfpathrectangle{\pgfqpoint{0.694444in}{0.416667in}}{\pgfqpoint{3.194444in}{1.416667in}}%
\pgfusepath{clip}%
\pgfsetbuttcap%
\pgfsetmiterjoin%
\definecolor{currentfill}{rgb}{0.725490,0.486275,0.164706}%
\pgfsetfillcolor{currentfill}%
\pgfsetlinewidth{1.003750pt}%
\definecolor{currentstroke}{rgb}{0.266667,0.266667,0.266667}%
\pgfsetstrokecolor{currentstroke}%
\pgfsetdash{}{0pt}%
\pgfpathmoveto{\pgfqpoint{2.577107in}{1.517632in}}%
\pgfpathlineto{\pgfqpoint{2.750853in}{1.517632in}}%
\pgfpathlineto{\pgfqpoint{2.750853in}{1.563632in}}%
\pgfpathlineto{\pgfqpoint{2.577107in}{1.563632in}}%
\pgfpathlineto{\pgfqpoint{2.577107in}{1.517632in}}%
\pgfpathclose%
\pgfusepath{stroke,fill}%
\end{pgfscope}%
\begin{pgfscope}%
\pgfpathrectangle{\pgfqpoint{0.694444in}{0.416667in}}{\pgfqpoint{3.194444in}{1.416667in}}%
\pgfusepath{clip}%
\pgfsetbuttcap%
\pgfsetmiterjoin%
\definecolor{currentfill}{rgb}{0.725490,0.486275,0.164706}%
\pgfsetfillcolor{currentfill}%
\pgfsetlinewidth{1.003750pt}%
\definecolor{currentstroke}{rgb}{0.266667,0.266667,0.266667}%
\pgfsetstrokecolor{currentstroke}%
\pgfsetdash{}{0pt}%
\pgfpathmoveto{\pgfqpoint{2.825315in}{1.543325in}}%
\pgfpathlineto{\pgfqpoint{2.999061in}{1.543325in}}%
\pgfpathlineto{\pgfqpoint{2.999061in}{1.584502in}}%
\pgfpathlineto{\pgfqpoint{2.825315in}{1.584502in}}%
\pgfpathlineto{\pgfqpoint{2.825315in}{1.543325in}}%
\pgfpathclose%
\pgfusepath{stroke,fill}%
\end{pgfscope}%
\begin{pgfscope}%
\pgfpathrectangle{\pgfqpoint{0.694444in}{0.416667in}}{\pgfqpoint{3.194444in}{1.416667in}}%
\pgfusepath{clip}%
\pgfsetbuttcap%
\pgfsetmiterjoin%
\definecolor{currentfill}{rgb}{0.725490,0.486275,0.164706}%
\pgfsetfillcolor{currentfill}%
\pgfsetlinewidth{1.003750pt}%
\definecolor{currentstroke}{rgb}{0.266667,0.266667,0.266667}%
\pgfsetstrokecolor{currentstroke}%
\pgfsetdash{}{0pt}%
\pgfpathmoveto{\pgfqpoint{3.073524in}{1.609003in}}%
\pgfpathlineto{\pgfqpoint{3.247270in}{1.609003in}}%
\pgfpathlineto{\pgfqpoint{3.247270in}{1.646162in}}%
\pgfpathlineto{\pgfqpoint{3.073524in}{1.646162in}}%
\pgfpathlineto{\pgfqpoint{3.073524in}{1.609003in}}%
\pgfpathclose%
\pgfusepath{stroke,fill}%
\end{pgfscope}%
\begin{pgfscope}%
\pgfpathrectangle{\pgfqpoint{0.694444in}{0.416667in}}{\pgfqpoint{3.194444in}{1.416667in}}%
\pgfusepath{clip}%
\pgfsetbuttcap%
\pgfsetmiterjoin%
\definecolor{currentfill}{rgb}{0.725490,0.486275,0.164706}%
\pgfsetfillcolor{currentfill}%
\pgfsetlinewidth{1.003750pt}%
\definecolor{currentstroke}{rgb}{0.266667,0.266667,0.266667}%
\pgfsetstrokecolor{currentstroke}%
\pgfsetdash{}{0pt}%
\pgfpathmoveto{\pgfqpoint{3.321732in}{1.550686in}}%
\pgfpathlineto{\pgfqpoint{3.495478in}{1.550686in}}%
\pgfpathlineto{\pgfqpoint{3.495478in}{1.586136in}}%
\pgfpathlineto{\pgfqpoint{3.321732in}{1.586136in}}%
\pgfpathlineto{\pgfqpoint{3.321732in}{1.550686in}}%
\pgfpathclose%
\pgfusepath{stroke,fill}%
\end{pgfscope}%
\begin{pgfscope}%
\pgfpathrectangle{\pgfqpoint{0.694444in}{0.416667in}}{\pgfqpoint{3.194444in}{1.416667in}}%
\pgfusepath{clip}%
\pgfsetbuttcap%
\pgfsetmiterjoin%
\definecolor{currentfill}{rgb}{0.725490,0.486275,0.164706}%
\pgfsetfillcolor{currentfill}%
\pgfsetlinewidth{1.003750pt}%
\definecolor{currentstroke}{rgb}{0.266667,0.266667,0.266667}%
\pgfsetstrokecolor{currentstroke}%
\pgfsetdash{}{0pt}%
\pgfpathmoveto{\pgfqpoint{3.569941in}{1.525633in}}%
\pgfpathlineto{\pgfqpoint{3.743687in}{1.525633in}}%
\pgfpathlineto{\pgfqpoint{3.743687in}{1.600486in}}%
\pgfpathlineto{\pgfqpoint{3.569941in}{1.600486in}}%
\pgfpathlineto{\pgfqpoint{3.569941in}{1.525633in}}%
\pgfpathclose%
\pgfusepath{stroke,fill}%
\end{pgfscope}%
\begin{pgfscope}%
\pgfpathrectangle{\pgfqpoint{0.694444in}{0.416667in}}{\pgfqpoint{3.194444in}{1.416667in}}%
\pgfusepath{clip}%
\pgfsetbuttcap%
\pgfsetmiterjoin%
\definecolor{currentfill}{rgb}{0.447059,0.447059,0.447059}%
\pgfsetfillcolor{currentfill}%
\pgfsetlinewidth{1.003750pt}%
\definecolor{currentstroke}{rgb}{0.266667,0.266667,0.266667}%
\pgfsetstrokecolor{currentstroke}%
\pgfsetdash{}{0pt}%
\pgfpathmoveto{\pgfqpoint{0.839646in}{1.403350in}}%
\pgfpathlineto{\pgfqpoint{1.013392in}{1.403350in}}%
\pgfpathlineto{\pgfqpoint{1.013392in}{1.416067in}}%
\pgfpathlineto{\pgfqpoint{0.839646in}{1.416067in}}%
\pgfpathlineto{\pgfqpoint{0.839646in}{1.403350in}}%
\pgfpathclose%
\pgfusepath{stroke,fill}%
\end{pgfscope}%
\begin{pgfscope}%
\pgfpathrectangle{\pgfqpoint{0.694444in}{0.416667in}}{\pgfqpoint{3.194444in}{1.416667in}}%
\pgfusepath{clip}%
\pgfsetbuttcap%
\pgfsetmiterjoin%
\definecolor{currentfill}{rgb}{0.447059,0.447059,0.447059}%
\pgfsetfillcolor{currentfill}%
\pgfsetlinewidth{1.003750pt}%
\definecolor{currentstroke}{rgb}{0.266667,0.266667,0.266667}%
\pgfsetstrokecolor{currentstroke}%
\pgfsetdash{}{0pt}%
\pgfpathmoveto{\pgfqpoint{1.087855in}{1.337215in}}%
\pgfpathlineto{\pgfqpoint{1.261601in}{1.337215in}}%
\pgfpathlineto{\pgfqpoint{1.261601in}{1.344581in}}%
\pgfpathlineto{\pgfqpoint{1.087855in}{1.344581in}}%
\pgfpathlineto{\pgfqpoint{1.087855in}{1.337215in}}%
\pgfpathclose%
\pgfusepath{stroke,fill}%
\end{pgfscope}%
\begin{pgfscope}%
\pgfpathrectangle{\pgfqpoint{0.694444in}{0.416667in}}{\pgfqpoint{3.194444in}{1.416667in}}%
\pgfusepath{clip}%
\pgfsetbuttcap%
\pgfsetmiterjoin%
\definecolor{currentfill}{rgb}{0.447059,0.447059,0.447059}%
\pgfsetfillcolor{currentfill}%
\pgfsetlinewidth{1.003750pt}%
\definecolor{currentstroke}{rgb}{0.266667,0.266667,0.266667}%
\pgfsetstrokecolor{currentstroke}%
\pgfsetdash{}{0pt}%
\pgfpathmoveto{\pgfqpoint{1.336064in}{1.363305in}}%
\pgfpathlineto{\pgfqpoint{1.509810in}{1.363305in}}%
\pgfpathlineto{\pgfqpoint{1.509810in}{1.380703in}}%
\pgfpathlineto{\pgfqpoint{1.336064in}{1.380703in}}%
\pgfpathlineto{\pgfqpoint{1.336064in}{1.363305in}}%
\pgfpathclose%
\pgfusepath{stroke,fill}%
\end{pgfscope}%
\begin{pgfscope}%
\pgfpathrectangle{\pgfqpoint{0.694444in}{0.416667in}}{\pgfqpoint{3.194444in}{1.416667in}}%
\pgfusepath{clip}%
\pgfsetbuttcap%
\pgfsetmiterjoin%
\definecolor{currentfill}{rgb}{0.447059,0.447059,0.447059}%
\pgfsetfillcolor{currentfill}%
\pgfsetlinewidth{1.003750pt}%
\definecolor{currentstroke}{rgb}{0.266667,0.266667,0.266667}%
\pgfsetstrokecolor{currentstroke}%
\pgfsetdash{}{0pt}%
\pgfpathmoveto{\pgfqpoint{1.584272in}{1.498981in}}%
\pgfpathlineto{\pgfqpoint{1.758018in}{1.498981in}}%
\pgfpathlineto{\pgfqpoint{1.758018in}{1.505116in}}%
\pgfpathlineto{\pgfqpoint{1.584272in}{1.505116in}}%
\pgfpathlineto{\pgfqpoint{1.584272in}{1.498981in}}%
\pgfpathclose%
\pgfusepath{stroke,fill}%
\end{pgfscope}%
\begin{pgfscope}%
\pgfpathrectangle{\pgfqpoint{0.694444in}{0.416667in}}{\pgfqpoint{3.194444in}{1.416667in}}%
\pgfusepath{clip}%
\pgfsetbuttcap%
\pgfsetmiterjoin%
\definecolor{currentfill}{rgb}{0.447059,0.447059,0.447059}%
\pgfsetfillcolor{currentfill}%
\pgfsetlinewidth{1.003750pt}%
\definecolor{currentstroke}{rgb}{0.266667,0.266667,0.266667}%
\pgfsetstrokecolor{currentstroke}%
\pgfsetdash{}{0pt}%
\pgfpathmoveto{\pgfqpoint{1.832481in}{1.644287in}}%
\pgfpathlineto{\pgfqpoint{2.006227in}{1.644287in}}%
\pgfpathlineto{\pgfqpoint{2.006227in}{1.656866in}}%
\pgfpathlineto{\pgfqpoint{1.832481in}{1.656866in}}%
\pgfpathlineto{\pgfqpoint{1.832481in}{1.644287in}}%
\pgfpathclose%
\pgfusepath{stroke,fill}%
\end{pgfscope}%
\begin{pgfscope}%
\pgfpathrectangle{\pgfqpoint{0.694444in}{0.416667in}}{\pgfqpoint{3.194444in}{1.416667in}}%
\pgfusepath{clip}%
\pgfsetbuttcap%
\pgfsetmiterjoin%
\definecolor{currentfill}{rgb}{0.447059,0.447059,0.447059}%
\pgfsetfillcolor{currentfill}%
\pgfsetlinewidth{1.003750pt}%
\definecolor{currentstroke}{rgb}{0.266667,0.266667,0.266667}%
\pgfsetstrokecolor{currentstroke}%
\pgfsetdash{}{0pt}%
\pgfpathmoveto{\pgfqpoint{2.080689in}{1.491256in}}%
\pgfpathlineto{\pgfqpoint{2.254435in}{1.491256in}}%
\pgfpathlineto{\pgfqpoint{2.254435in}{1.514106in}}%
\pgfpathlineto{\pgfqpoint{2.080689in}{1.514106in}}%
\pgfpathlineto{\pgfqpoint{2.080689in}{1.491256in}}%
\pgfpathclose%
\pgfusepath{stroke,fill}%
\end{pgfscope}%
\begin{pgfscope}%
\pgfpathrectangle{\pgfqpoint{0.694444in}{0.416667in}}{\pgfqpoint{3.194444in}{1.416667in}}%
\pgfusepath{clip}%
\pgfsetbuttcap%
\pgfsetmiterjoin%
\definecolor{currentfill}{rgb}{0.447059,0.447059,0.447059}%
\pgfsetfillcolor{currentfill}%
\pgfsetlinewidth{1.003750pt}%
\definecolor{currentstroke}{rgb}{0.266667,0.266667,0.266667}%
\pgfsetstrokecolor{currentstroke}%
\pgfsetdash{}{0pt}%
\pgfpathmoveto{\pgfqpoint{2.328898in}{1.473668in}}%
\pgfpathlineto{\pgfqpoint{2.502644in}{1.473668in}}%
\pgfpathlineto{\pgfqpoint{2.502644in}{1.488063in}}%
\pgfpathlineto{\pgfqpoint{2.328898in}{1.488063in}}%
\pgfpathlineto{\pgfqpoint{2.328898in}{1.473668in}}%
\pgfpathclose%
\pgfusepath{stroke,fill}%
\end{pgfscope}%
\begin{pgfscope}%
\pgfpathrectangle{\pgfqpoint{0.694444in}{0.416667in}}{\pgfqpoint{3.194444in}{1.416667in}}%
\pgfusepath{clip}%
\pgfsetbuttcap%
\pgfsetmiterjoin%
\definecolor{currentfill}{rgb}{0.447059,0.447059,0.447059}%
\pgfsetfillcolor{currentfill}%
\pgfsetlinewidth{1.003750pt}%
\definecolor{currentstroke}{rgb}{0.266667,0.266667,0.266667}%
\pgfsetstrokecolor{currentstroke}%
\pgfsetdash{}{0pt}%
\pgfpathmoveto{\pgfqpoint{2.577107in}{1.563632in}}%
\pgfpathlineto{\pgfqpoint{2.750853in}{1.563632in}}%
\pgfpathlineto{\pgfqpoint{2.750853in}{1.575433in}}%
\pgfpathlineto{\pgfqpoint{2.577107in}{1.575433in}}%
\pgfpathlineto{\pgfqpoint{2.577107in}{1.563632in}}%
\pgfpathclose%
\pgfusepath{stroke,fill}%
\end{pgfscope}%
\begin{pgfscope}%
\pgfpathrectangle{\pgfqpoint{0.694444in}{0.416667in}}{\pgfqpoint{3.194444in}{1.416667in}}%
\pgfusepath{clip}%
\pgfsetbuttcap%
\pgfsetmiterjoin%
\definecolor{currentfill}{rgb}{0.447059,0.447059,0.447059}%
\pgfsetfillcolor{currentfill}%
\pgfsetlinewidth{1.003750pt}%
\definecolor{currentstroke}{rgb}{0.266667,0.266667,0.266667}%
\pgfsetstrokecolor{currentstroke}%
\pgfsetdash{}{0pt}%
\pgfpathmoveto{\pgfqpoint{2.825315in}{1.584502in}}%
\pgfpathlineto{\pgfqpoint{2.999061in}{1.584502in}}%
\pgfpathlineto{\pgfqpoint{2.999061in}{1.602293in}}%
\pgfpathlineto{\pgfqpoint{2.825315in}{1.602293in}}%
\pgfpathlineto{\pgfqpoint{2.825315in}{1.584502in}}%
\pgfpathclose%
\pgfusepath{stroke,fill}%
\end{pgfscope}%
\begin{pgfscope}%
\pgfpathrectangle{\pgfqpoint{0.694444in}{0.416667in}}{\pgfqpoint{3.194444in}{1.416667in}}%
\pgfusepath{clip}%
\pgfsetbuttcap%
\pgfsetmiterjoin%
\definecolor{currentfill}{rgb}{0.447059,0.447059,0.447059}%
\pgfsetfillcolor{currentfill}%
\pgfsetlinewidth{1.003750pt}%
\definecolor{currentstroke}{rgb}{0.266667,0.266667,0.266667}%
\pgfsetstrokecolor{currentstroke}%
\pgfsetdash{}{0pt}%
\pgfpathmoveto{\pgfqpoint{3.073524in}{1.646162in}}%
\pgfpathlineto{\pgfqpoint{3.247270in}{1.646162in}}%
\pgfpathlineto{\pgfqpoint{3.247270in}{1.661413in}}%
\pgfpathlineto{\pgfqpoint{3.073524in}{1.661413in}}%
\pgfpathlineto{\pgfqpoint{3.073524in}{1.646162in}}%
\pgfpathclose%
\pgfusepath{stroke,fill}%
\end{pgfscope}%
\begin{pgfscope}%
\pgfpathrectangle{\pgfqpoint{0.694444in}{0.416667in}}{\pgfqpoint{3.194444in}{1.416667in}}%
\pgfusepath{clip}%
\pgfsetbuttcap%
\pgfsetmiterjoin%
\definecolor{currentfill}{rgb}{0.447059,0.447059,0.447059}%
\pgfsetfillcolor{currentfill}%
\pgfsetlinewidth{1.003750pt}%
\definecolor{currentstroke}{rgb}{0.266667,0.266667,0.266667}%
\pgfsetstrokecolor{currentstroke}%
\pgfsetdash{}{0pt}%
\pgfpathmoveto{\pgfqpoint{3.321732in}{1.586136in}}%
\pgfpathlineto{\pgfqpoint{3.495478in}{1.586136in}}%
\pgfpathlineto{\pgfqpoint{3.495478in}{1.599796in}}%
\pgfpathlineto{\pgfqpoint{3.321732in}{1.599796in}}%
\pgfpathlineto{\pgfqpoint{3.321732in}{1.586136in}}%
\pgfpathclose%
\pgfusepath{stroke,fill}%
\end{pgfscope}%
\begin{pgfscope}%
\pgfpathrectangle{\pgfqpoint{0.694444in}{0.416667in}}{\pgfqpoint{3.194444in}{1.416667in}}%
\pgfusepath{clip}%
\pgfsetbuttcap%
\pgfsetmiterjoin%
\definecolor{currentfill}{rgb}{0.447059,0.447059,0.447059}%
\pgfsetfillcolor{currentfill}%
\pgfsetlinewidth{1.003750pt}%
\definecolor{currentstroke}{rgb}{0.266667,0.266667,0.266667}%
\pgfsetstrokecolor{currentstroke}%
\pgfsetdash{}{0pt}%
\pgfpathmoveto{\pgfqpoint{3.569941in}{1.600486in}}%
\pgfpathlineto{\pgfqpoint{3.743687in}{1.600486in}}%
\pgfpathlineto{\pgfqpoint{3.743687in}{1.633651in}}%
\pgfpathlineto{\pgfqpoint{3.569941in}{1.633651in}}%
\pgfpathlineto{\pgfqpoint{3.569941in}{1.600486in}}%
\pgfpathclose%
\pgfusepath{stroke,fill}%
\end{pgfscope}%
\begin{pgfscope}%
\pgfpathrectangle{\pgfqpoint{0.694444in}{0.416667in}}{\pgfqpoint{3.194444in}{1.416667in}}%
\pgfusepath{clip}%
\pgfsetbuttcap%
\pgfsetmiterjoin%
\definecolor{currentfill}{rgb}{0.447059,0.447059,0.447059}%
\pgfsetfillcolor{currentfill}%
\pgfsetlinewidth{1.003750pt}%
\definecolor{currentstroke}{rgb}{0.266667,0.266667,0.266667}%
\pgfsetstrokecolor{currentstroke}%
\pgfsetdash{}{0pt}%
\pgfpathmoveto{\pgfqpoint{0.839646in}{1.416067in}}%
\pgfpathlineto{\pgfqpoint{1.013392in}{1.416067in}}%
\pgfpathlineto{\pgfqpoint{1.013392in}{1.416154in}}%
\pgfpathlineto{\pgfqpoint{0.839646in}{1.416154in}}%
\pgfpathlineto{\pgfqpoint{0.839646in}{1.416067in}}%
\pgfpathclose%
\pgfusepath{stroke,fill}%
\end{pgfscope}%
\begin{pgfscope}%
\pgfpathrectangle{\pgfqpoint{0.694444in}{0.416667in}}{\pgfqpoint{3.194444in}{1.416667in}}%
\pgfusepath{clip}%
\pgfsetbuttcap%
\pgfsetmiterjoin%
\definecolor{currentfill}{rgb}{0.447059,0.447059,0.447059}%
\pgfsetfillcolor{currentfill}%
\pgfsetlinewidth{1.003750pt}%
\definecolor{currentstroke}{rgb}{0.266667,0.266667,0.266667}%
\pgfsetstrokecolor{currentstroke}%
\pgfsetdash{}{0pt}%
\pgfpathmoveto{\pgfqpoint{1.087855in}{1.344581in}}%
\pgfpathlineto{\pgfqpoint{1.261601in}{1.344581in}}%
\pgfpathlineto{\pgfqpoint{1.261601in}{1.344649in}}%
\pgfpathlineto{\pgfqpoint{1.087855in}{1.344649in}}%
\pgfpathlineto{\pgfqpoint{1.087855in}{1.344581in}}%
\pgfpathclose%
\pgfusepath{stroke,fill}%
\end{pgfscope}%
\begin{pgfscope}%
\pgfpathrectangle{\pgfqpoint{0.694444in}{0.416667in}}{\pgfqpoint{3.194444in}{1.416667in}}%
\pgfusepath{clip}%
\pgfsetbuttcap%
\pgfsetmiterjoin%
\definecolor{currentfill}{rgb}{0.447059,0.447059,0.447059}%
\pgfsetfillcolor{currentfill}%
\pgfsetlinewidth{1.003750pt}%
\definecolor{currentstroke}{rgb}{0.266667,0.266667,0.266667}%
\pgfsetstrokecolor{currentstroke}%
\pgfsetdash{}{0pt}%
\pgfpathmoveto{\pgfqpoint{1.336064in}{1.380703in}}%
\pgfpathlineto{\pgfqpoint{1.509810in}{1.380703in}}%
\pgfpathlineto{\pgfqpoint{1.509810in}{1.380863in}}%
\pgfpathlineto{\pgfqpoint{1.336064in}{1.380863in}}%
\pgfpathlineto{\pgfqpoint{1.336064in}{1.380703in}}%
\pgfpathclose%
\pgfusepath{stroke,fill}%
\end{pgfscope}%
\begin{pgfscope}%
\pgfpathrectangle{\pgfqpoint{0.694444in}{0.416667in}}{\pgfqpoint{3.194444in}{1.416667in}}%
\pgfusepath{clip}%
\pgfsetbuttcap%
\pgfsetmiterjoin%
\definecolor{currentfill}{rgb}{0.447059,0.447059,0.447059}%
\pgfsetfillcolor{currentfill}%
\pgfsetlinewidth{1.003750pt}%
\definecolor{currentstroke}{rgb}{0.266667,0.266667,0.266667}%
\pgfsetstrokecolor{currentstroke}%
\pgfsetdash{}{0pt}%
\pgfpathmoveto{\pgfqpoint{1.584272in}{1.505116in}}%
\pgfpathlineto{\pgfqpoint{1.758018in}{1.505116in}}%
\pgfpathlineto{\pgfqpoint{1.758018in}{1.505218in}}%
\pgfpathlineto{\pgfqpoint{1.584272in}{1.505218in}}%
\pgfpathlineto{\pgfqpoint{1.584272in}{1.505116in}}%
\pgfpathclose%
\pgfusepath{stroke,fill}%
\end{pgfscope}%
\begin{pgfscope}%
\pgfpathrectangle{\pgfqpoint{0.694444in}{0.416667in}}{\pgfqpoint{3.194444in}{1.416667in}}%
\pgfusepath{clip}%
\pgfsetbuttcap%
\pgfsetmiterjoin%
\definecolor{currentfill}{rgb}{0.447059,0.447059,0.447059}%
\pgfsetfillcolor{currentfill}%
\pgfsetlinewidth{1.003750pt}%
\definecolor{currentstroke}{rgb}{0.266667,0.266667,0.266667}%
\pgfsetstrokecolor{currentstroke}%
\pgfsetdash{}{0pt}%
\pgfpathmoveto{\pgfqpoint{1.832481in}{1.656866in}}%
\pgfpathlineto{\pgfqpoint{2.006227in}{1.656866in}}%
\pgfpathlineto{\pgfqpoint{2.006227in}{1.657493in}}%
\pgfpathlineto{\pgfqpoint{1.832481in}{1.657493in}}%
\pgfpathlineto{\pgfqpoint{1.832481in}{1.656866in}}%
\pgfpathclose%
\pgfusepath{stroke,fill}%
\end{pgfscope}%
\begin{pgfscope}%
\pgfpathrectangle{\pgfqpoint{0.694444in}{0.416667in}}{\pgfqpoint{3.194444in}{1.416667in}}%
\pgfusepath{clip}%
\pgfsetbuttcap%
\pgfsetmiterjoin%
\definecolor{currentfill}{rgb}{0.447059,0.447059,0.447059}%
\pgfsetfillcolor{currentfill}%
\pgfsetlinewidth{1.003750pt}%
\definecolor{currentstroke}{rgb}{0.266667,0.266667,0.266667}%
\pgfsetstrokecolor{currentstroke}%
\pgfsetdash{}{0pt}%
\pgfpathmoveto{\pgfqpoint{2.080689in}{1.514106in}}%
\pgfpathlineto{\pgfqpoint{2.254435in}{1.514106in}}%
\pgfpathlineto{\pgfqpoint{2.254435in}{1.514722in}}%
\pgfpathlineto{\pgfqpoint{2.080689in}{1.514722in}}%
\pgfpathlineto{\pgfqpoint{2.080689in}{1.514106in}}%
\pgfpathclose%
\pgfusepath{stroke,fill}%
\end{pgfscope}%
\begin{pgfscope}%
\pgfpathrectangle{\pgfqpoint{0.694444in}{0.416667in}}{\pgfqpoint{3.194444in}{1.416667in}}%
\pgfusepath{clip}%
\pgfsetbuttcap%
\pgfsetmiterjoin%
\definecolor{currentfill}{rgb}{0.447059,0.447059,0.447059}%
\pgfsetfillcolor{currentfill}%
\pgfsetlinewidth{1.003750pt}%
\definecolor{currentstroke}{rgb}{0.266667,0.266667,0.266667}%
\pgfsetstrokecolor{currentstroke}%
\pgfsetdash{}{0pt}%
\pgfpathmoveto{\pgfqpoint{2.328898in}{1.488063in}}%
\pgfpathlineto{\pgfqpoint{2.502644in}{1.488063in}}%
\pgfpathlineto{\pgfqpoint{2.502644in}{1.489478in}}%
\pgfpathlineto{\pgfqpoint{2.328898in}{1.489478in}}%
\pgfpathlineto{\pgfqpoint{2.328898in}{1.488063in}}%
\pgfpathclose%
\pgfusepath{stroke,fill}%
\end{pgfscope}%
\begin{pgfscope}%
\pgfpathrectangle{\pgfqpoint{0.694444in}{0.416667in}}{\pgfqpoint{3.194444in}{1.416667in}}%
\pgfusepath{clip}%
\pgfsetbuttcap%
\pgfsetmiterjoin%
\definecolor{currentfill}{rgb}{0.447059,0.447059,0.447059}%
\pgfsetfillcolor{currentfill}%
\pgfsetlinewidth{1.003750pt}%
\definecolor{currentstroke}{rgb}{0.266667,0.266667,0.266667}%
\pgfsetstrokecolor{currentstroke}%
\pgfsetdash{}{0pt}%
\pgfpathmoveto{\pgfqpoint{2.577107in}{1.575433in}}%
\pgfpathlineto{\pgfqpoint{2.750853in}{1.575433in}}%
\pgfpathlineto{\pgfqpoint{2.750853in}{1.586464in}}%
\pgfpathlineto{\pgfqpoint{2.577107in}{1.586464in}}%
\pgfpathlineto{\pgfqpoint{2.577107in}{1.575433in}}%
\pgfpathclose%
\pgfusepath{stroke,fill}%
\end{pgfscope}%
\begin{pgfscope}%
\pgfpathrectangle{\pgfqpoint{0.694444in}{0.416667in}}{\pgfqpoint{3.194444in}{1.416667in}}%
\pgfusepath{clip}%
\pgfsetbuttcap%
\pgfsetmiterjoin%
\definecolor{currentfill}{rgb}{0.447059,0.447059,0.447059}%
\pgfsetfillcolor{currentfill}%
\pgfsetlinewidth{1.003750pt}%
\definecolor{currentstroke}{rgb}{0.266667,0.266667,0.266667}%
\pgfsetstrokecolor{currentstroke}%
\pgfsetdash{}{0pt}%
\pgfpathmoveto{\pgfqpoint{2.825315in}{1.602293in}}%
\pgfpathlineto{\pgfqpoint{2.999061in}{1.602293in}}%
\pgfpathlineto{\pgfqpoint{2.999061in}{1.610119in}}%
\pgfpathlineto{\pgfqpoint{2.825315in}{1.610119in}}%
\pgfpathlineto{\pgfqpoint{2.825315in}{1.602293in}}%
\pgfpathclose%
\pgfusepath{stroke,fill}%
\end{pgfscope}%
\begin{pgfscope}%
\pgfpathrectangle{\pgfqpoint{0.694444in}{0.416667in}}{\pgfqpoint{3.194444in}{1.416667in}}%
\pgfusepath{clip}%
\pgfsetbuttcap%
\pgfsetmiterjoin%
\definecolor{currentfill}{rgb}{0.447059,0.447059,0.447059}%
\pgfsetfillcolor{currentfill}%
\pgfsetlinewidth{1.003750pt}%
\definecolor{currentstroke}{rgb}{0.266667,0.266667,0.266667}%
\pgfsetstrokecolor{currentstroke}%
\pgfsetdash{}{0pt}%
\pgfpathmoveto{\pgfqpoint{3.073524in}{1.661413in}}%
\pgfpathlineto{\pgfqpoint{3.247270in}{1.661413in}}%
\pgfpathlineto{\pgfqpoint{3.247270in}{1.670406in}}%
\pgfpathlineto{\pgfqpoint{3.073524in}{1.670406in}}%
\pgfpathlineto{\pgfqpoint{3.073524in}{1.661413in}}%
\pgfpathclose%
\pgfusepath{stroke,fill}%
\end{pgfscope}%
\begin{pgfscope}%
\pgfpathrectangle{\pgfqpoint{0.694444in}{0.416667in}}{\pgfqpoint{3.194444in}{1.416667in}}%
\pgfusepath{clip}%
\pgfsetbuttcap%
\pgfsetmiterjoin%
\definecolor{currentfill}{rgb}{0.447059,0.447059,0.447059}%
\pgfsetfillcolor{currentfill}%
\pgfsetlinewidth{1.003750pt}%
\definecolor{currentstroke}{rgb}{0.266667,0.266667,0.266667}%
\pgfsetstrokecolor{currentstroke}%
\pgfsetdash{}{0pt}%
\pgfpathmoveto{\pgfqpoint{3.321732in}{1.599796in}}%
\pgfpathlineto{\pgfqpoint{3.495478in}{1.599796in}}%
\pgfpathlineto{\pgfqpoint{3.495478in}{1.612003in}}%
\pgfpathlineto{\pgfqpoint{3.321732in}{1.612003in}}%
\pgfpathlineto{\pgfqpoint{3.321732in}{1.599796in}}%
\pgfpathclose%
\pgfusepath{stroke,fill}%
\end{pgfscope}%
\begin{pgfscope}%
\pgfpathrectangle{\pgfqpoint{0.694444in}{0.416667in}}{\pgfqpoint{3.194444in}{1.416667in}}%
\pgfusepath{clip}%
\pgfsetbuttcap%
\pgfsetmiterjoin%
\definecolor{currentfill}{rgb}{0.447059,0.447059,0.447059}%
\pgfsetfillcolor{currentfill}%
\pgfsetlinewidth{1.003750pt}%
\definecolor{currentstroke}{rgb}{0.266667,0.266667,0.266667}%
\pgfsetstrokecolor{currentstroke}%
\pgfsetdash{}{0pt}%
\pgfpathmoveto{\pgfqpoint{3.569941in}{1.633651in}}%
\pgfpathlineto{\pgfqpoint{3.743687in}{1.633651in}}%
\pgfpathlineto{\pgfqpoint{3.743687in}{1.642799in}}%
\pgfpathlineto{\pgfqpoint{3.569941in}{1.642799in}}%
\pgfpathlineto{\pgfqpoint{3.569941in}{1.633651in}}%
\pgfpathclose%
\pgfusepath{stroke,fill}%
\end{pgfscope}%
\begin{pgfscope}%
\pgfpathrectangle{\pgfqpoint{0.694444in}{0.416667in}}{\pgfqpoint{3.194444in}{1.416667in}}%
\pgfusepath{clip}%
\pgfsetbuttcap%
\pgfsetmiterjoin%
\definecolor{currentfill}{rgb}{0.447059,0.447059,0.447059}%
\pgfsetfillcolor{currentfill}%
\pgfsetlinewidth{1.003750pt}%
\definecolor{currentstroke}{rgb}{0.266667,0.266667,0.266667}%
\pgfsetstrokecolor{currentstroke}%
\pgfsetdash{}{0pt}%
\pgfpathmoveto{\pgfqpoint{0.839646in}{1.416154in}}%
\pgfpathlineto{\pgfqpoint{1.013392in}{1.416154in}}%
\pgfpathlineto{\pgfqpoint{1.013392in}{1.416154in}}%
\pgfpathlineto{\pgfqpoint{0.839646in}{1.416154in}}%
\pgfpathlineto{\pgfqpoint{0.839646in}{1.416154in}}%
\pgfpathclose%
\pgfusepath{stroke,fill}%
\end{pgfscope}%
\begin{pgfscope}%
\pgfpathrectangle{\pgfqpoint{0.694444in}{0.416667in}}{\pgfqpoint{3.194444in}{1.416667in}}%
\pgfusepath{clip}%
\pgfsetbuttcap%
\pgfsetmiterjoin%
\definecolor{currentfill}{rgb}{0.447059,0.447059,0.447059}%
\pgfsetfillcolor{currentfill}%
\pgfsetlinewidth{1.003750pt}%
\definecolor{currentstroke}{rgb}{0.266667,0.266667,0.266667}%
\pgfsetstrokecolor{currentstroke}%
\pgfsetdash{}{0pt}%
\pgfpathmoveto{\pgfqpoint{1.087855in}{1.344649in}}%
\pgfpathlineto{\pgfqpoint{1.261601in}{1.344649in}}%
\pgfpathlineto{\pgfqpoint{1.261601in}{1.345649in}}%
\pgfpathlineto{\pgfqpoint{1.087855in}{1.345649in}}%
\pgfpathlineto{\pgfqpoint{1.087855in}{1.344649in}}%
\pgfpathclose%
\pgfusepath{stroke,fill}%
\end{pgfscope}%
\begin{pgfscope}%
\pgfpathrectangle{\pgfqpoint{0.694444in}{0.416667in}}{\pgfqpoint{3.194444in}{1.416667in}}%
\pgfusepath{clip}%
\pgfsetbuttcap%
\pgfsetmiterjoin%
\definecolor{currentfill}{rgb}{0.447059,0.447059,0.447059}%
\pgfsetfillcolor{currentfill}%
\pgfsetlinewidth{1.003750pt}%
\definecolor{currentstroke}{rgb}{0.266667,0.266667,0.266667}%
\pgfsetstrokecolor{currentstroke}%
\pgfsetdash{}{0pt}%
\pgfpathmoveto{\pgfqpoint{1.336064in}{1.380863in}}%
\pgfpathlineto{\pgfqpoint{1.509810in}{1.380863in}}%
\pgfpathlineto{\pgfqpoint{1.509810in}{1.385574in}}%
\pgfpathlineto{\pgfqpoint{1.336064in}{1.385574in}}%
\pgfpathlineto{\pgfqpoint{1.336064in}{1.380863in}}%
\pgfpathclose%
\pgfusepath{stroke,fill}%
\end{pgfscope}%
\begin{pgfscope}%
\pgfpathrectangle{\pgfqpoint{0.694444in}{0.416667in}}{\pgfqpoint{3.194444in}{1.416667in}}%
\pgfusepath{clip}%
\pgfsetbuttcap%
\pgfsetmiterjoin%
\definecolor{currentfill}{rgb}{0.447059,0.447059,0.447059}%
\pgfsetfillcolor{currentfill}%
\pgfsetlinewidth{1.003750pt}%
\definecolor{currentstroke}{rgb}{0.266667,0.266667,0.266667}%
\pgfsetstrokecolor{currentstroke}%
\pgfsetdash{}{0pt}%
\pgfpathmoveto{\pgfqpoint{1.584272in}{1.505218in}}%
\pgfpathlineto{\pgfqpoint{1.758018in}{1.505218in}}%
\pgfpathlineto{\pgfqpoint{1.758018in}{1.506263in}}%
\pgfpathlineto{\pgfqpoint{1.584272in}{1.506263in}}%
\pgfpathlineto{\pgfqpoint{1.584272in}{1.505218in}}%
\pgfpathclose%
\pgfusepath{stroke,fill}%
\end{pgfscope}%
\begin{pgfscope}%
\pgfpathrectangle{\pgfqpoint{0.694444in}{0.416667in}}{\pgfqpoint{3.194444in}{1.416667in}}%
\pgfusepath{clip}%
\pgfsetbuttcap%
\pgfsetmiterjoin%
\definecolor{currentfill}{rgb}{0.447059,0.447059,0.447059}%
\pgfsetfillcolor{currentfill}%
\pgfsetlinewidth{1.003750pt}%
\definecolor{currentstroke}{rgb}{0.266667,0.266667,0.266667}%
\pgfsetstrokecolor{currentstroke}%
\pgfsetdash{}{0pt}%
\pgfpathmoveto{\pgfqpoint{1.832481in}{1.657493in}}%
\pgfpathlineto{\pgfqpoint{2.006227in}{1.657493in}}%
\pgfpathlineto{\pgfqpoint{2.006227in}{1.658613in}}%
\pgfpathlineto{\pgfqpoint{1.832481in}{1.658613in}}%
\pgfpathlineto{\pgfqpoint{1.832481in}{1.657493in}}%
\pgfpathclose%
\pgfusepath{stroke,fill}%
\end{pgfscope}%
\begin{pgfscope}%
\pgfpathrectangle{\pgfqpoint{0.694444in}{0.416667in}}{\pgfqpoint{3.194444in}{1.416667in}}%
\pgfusepath{clip}%
\pgfsetbuttcap%
\pgfsetmiterjoin%
\definecolor{currentfill}{rgb}{0.447059,0.447059,0.447059}%
\pgfsetfillcolor{currentfill}%
\pgfsetlinewidth{1.003750pt}%
\definecolor{currentstroke}{rgb}{0.266667,0.266667,0.266667}%
\pgfsetstrokecolor{currentstroke}%
\pgfsetdash{}{0pt}%
\pgfpathmoveto{\pgfqpoint{2.080689in}{1.514722in}}%
\pgfpathlineto{\pgfqpoint{2.254435in}{1.514722in}}%
\pgfpathlineto{\pgfqpoint{2.254435in}{1.515946in}}%
\pgfpathlineto{\pgfqpoint{2.080689in}{1.515946in}}%
\pgfpathlineto{\pgfqpoint{2.080689in}{1.514722in}}%
\pgfpathclose%
\pgfusepath{stroke,fill}%
\end{pgfscope}%
\begin{pgfscope}%
\pgfpathrectangle{\pgfqpoint{0.694444in}{0.416667in}}{\pgfqpoint{3.194444in}{1.416667in}}%
\pgfusepath{clip}%
\pgfsetbuttcap%
\pgfsetmiterjoin%
\definecolor{currentfill}{rgb}{0.447059,0.447059,0.447059}%
\pgfsetfillcolor{currentfill}%
\pgfsetlinewidth{1.003750pt}%
\definecolor{currentstroke}{rgb}{0.266667,0.266667,0.266667}%
\pgfsetstrokecolor{currentstroke}%
\pgfsetdash{}{0pt}%
\pgfpathmoveto{\pgfqpoint{2.328898in}{1.489478in}}%
\pgfpathlineto{\pgfqpoint{2.502644in}{1.489478in}}%
\pgfpathlineto{\pgfqpoint{2.502644in}{1.490246in}}%
\pgfpathlineto{\pgfqpoint{2.328898in}{1.490246in}}%
\pgfpathlineto{\pgfqpoint{2.328898in}{1.489478in}}%
\pgfpathclose%
\pgfusepath{stroke,fill}%
\end{pgfscope}%
\begin{pgfscope}%
\pgfpathrectangle{\pgfqpoint{0.694444in}{0.416667in}}{\pgfqpoint{3.194444in}{1.416667in}}%
\pgfusepath{clip}%
\pgfsetbuttcap%
\pgfsetmiterjoin%
\definecolor{currentfill}{rgb}{0.447059,0.447059,0.447059}%
\pgfsetfillcolor{currentfill}%
\pgfsetlinewidth{1.003750pt}%
\definecolor{currentstroke}{rgb}{0.266667,0.266667,0.266667}%
\pgfsetstrokecolor{currentstroke}%
\pgfsetdash{}{0pt}%
\pgfpathmoveto{\pgfqpoint{2.577107in}{1.586464in}}%
\pgfpathlineto{\pgfqpoint{2.750853in}{1.586464in}}%
\pgfpathlineto{\pgfqpoint{2.750853in}{1.587320in}}%
\pgfpathlineto{\pgfqpoint{2.577107in}{1.587320in}}%
\pgfpathlineto{\pgfqpoint{2.577107in}{1.586464in}}%
\pgfpathclose%
\pgfusepath{stroke,fill}%
\end{pgfscope}%
\begin{pgfscope}%
\pgfpathrectangle{\pgfqpoint{0.694444in}{0.416667in}}{\pgfqpoint{3.194444in}{1.416667in}}%
\pgfusepath{clip}%
\pgfsetbuttcap%
\pgfsetmiterjoin%
\definecolor{currentfill}{rgb}{0.447059,0.447059,0.447059}%
\pgfsetfillcolor{currentfill}%
\pgfsetlinewidth{1.003750pt}%
\definecolor{currentstroke}{rgb}{0.266667,0.266667,0.266667}%
\pgfsetstrokecolor{currentstroke}%
\pgfsetdash{}{0pt}%
\pgfpathmoveto{\pgfqpoint{2.825315in}{1.610119in}}%
\pgfpathlineto{\pgfqpoint{2.999061in}{1.610119in}}%
\pgfpathlineto{\pgfqpoint{2.999061in}{1.610320in}}%
\pgfpathlineto{\pgfqpoint{2.825315in}{1.610320in}}%
\pgfpathlineto{\pgfqpoint{2.825315in}{1.610119in}}%
\pgfpathclose%
\pgfusepath{stroke,fill}%
\end{pgfscope}%
\begin{pgfscope}%
\pgfpathrectangle{\pgfqpoint{0.694444in}{0.416667in}}{\pgfqpoint{3.194444in}{1.416667in}}%
\pgfusepath{clip}%
\pgfsetbuttcap%
\pgfsetmiterjoin%
\definecolor{currentfill}{rgb}{0.447059,0.447059,0.447059}%
\pgfsetfillcolor{currentfill}%
\pgfsetlinewidth{1.003750pt}%
\definecolor{currentstroke}{rgb}{0.266667,0.266667,0.266667}%
\pgfsetstrokecolor{currentstroke}%
\pgfsetdash{}{0pt}%
\pgfpathmoveto{\pgfqpoint{3.073524in}{1.670406in}}%
\pgfpathlineto{\pgfqpoint{3.247270in}{1.670406in}}%
\pgfpathlineto{\pgfqpoint{3.247270in}{1.670646in}}%
\pgfpathlineto{\pgfqpoint{3.073524in}{1.670646in}}%
\pgfpathlineto{\pgfqpoint{3.073524in}{1.670406in}}%
\pgfpathclose%
\pgfusepath{stroke,fill}%
\end{pgfscope}%
\begin{pgfscope}%
\pgfpathrectangle{\pgfqpoint{0.694444in}{0.416667in}}{\pgfqpoint{3.194444in}{1.416667in}}%
\pgfusepath{clip}%
\pgfsetbuttcap%
\pgfsetmiterjoin%
\definecolor{currentfill}{rgb}{0.447059,0.447059,0.447059}%
\pgfsetfillcolor{currentfill}%
\pgfsetlinewidth{1.003750pt}%
\definecolor{currentstroke}{rgb}{0.266667,0.266667,0.266667}%
\pgfsetstrokecolor{currentstroke}%
\pgfsetdash{}{0pt}%
\pgfpathmoveto{\pgfqpoint{3.321732in}{1.612003in}}%
\pgfpathlineto{\pgfqpoint{3.495478in}{1.612003in}}%
\pgfpathlineto{\pgfqpoint{3.495478in}{1.612768in}}%
\pgfpathlineto{\pgfqpoint{3.321732in}{1.612768in}}%
\pgfpathlineto{\pgfqpoint{3.321732in}{1.612003in}}%
\pgfpathclose%
\pgfusepath{stroke,fill}%
\end{pgfscope}%
\begin{pgfscope}%
\pgfpathrectangle{\pgfqpoint{0.694444in}{0.416667in}}{\pgfqpoint{3.194444in}{1.416667in}}%
\pgfusepath{clip}%
\pgfsetbuttcap%
\pgfsetmiterjoin%
\definecolor{currentfill}{rgb}{0.447059,0.447059,0.447059}%
\pgfsetfillcolor{currentfill}%
\pgfsetlinewidth{1.003750pt}%
\definecolor{currentstroke}{rgb}{0.266667,0.266667,0.266667}%
\pgfsetstrokecolor{currentstroke}%
\pgfsetdash{}{0pt}%
\pgfpathmoveto{\pgfqpoint{3.569941in}{1.642799in}}%
\pgfpathlineto{\pgfqpoint{3.743687in}{1.642799in}}%
\pgfpathlineto{\pgfqpoint{3.743687in}{1.646719in}}%
\pgfpathlineto{\pgfqpoint{3.569941in}{1.646719in}}%
\pgfpathlineto{\pgfqpoint{3.569941in}{1.642799in}}%
\pgfpathclose%
\pgfusepath{stroke,fill}%
\end{pgfscope}%
\begin{pgfscope}%
\pgfpathrectangle{\pgfqpoint{0.694444in}{0.416667in}}{\pgfqpoint{3.194444in}{1.416667in}}%
\pgfusepath{clip}%
\pgfsetbuttcap%
\pgfsetmiterjoin%
\definecolor{currentfill}{rgb}{0.447059,0.447059,0.447059}%
\pgfsetfillcolor{currentfill}%
\pgfsetlinewidth{1.003750pt}%
\definecolor{currentstroke}{rgb}{0.266667,0.266667,0.266667}%
\pgfsetstrokecolor{currentstroke}%
\pgfsetdash{}{0pt}%
\pgfpathmoveto{\pgfqpoint{0.839646in}{1.416154in}}%
\pgfpathlineto{\pgfqpoint{1.013392in}{1.416154in}}%
\pgfpathlineto{\pgfqpoint{1.013392in}{1.416154in}}%
\pgfpathlineto{\pgfqpoint{0.839646in}{1.416154in}}%
\pgfpathlineto{\pgfqpoint{0.839646in}{1.416154in}}%
\pgfpathclose%
\pgfusepath{stroke,fill}%
\end{pgfscope}%
\begin{pgfscope}%
\pgfpathrectangle{\pgfqpoint{0.694444in}{0.416667in}}{\pgfqpoint{3.194444in}{1.416667in}}%
\pgfusepath{clip}%
\pgfsetbuttcap%
\pgfsetmiterjoin%
\definecolor{currentfill}{rgb}{0.447059,0.447059,0.447059}%
\pgfsetfillcolor{currentfill}%
\pgfsetlinewidth{1.003750pt}%
\definecolor{currentstroke}{rgb}{0.266667,0.266667,0.266667}%
\pgfsetstrokecolor{currentstroke}%
\pgfsetdash{}{0pt}%
\pgfpathmoveto{\pgfqpoint{1.087855in}{1.345649in}}%
\pgfpathlineto{\pgfqpoint{1.261601in}{1.345649in}}%
\pgfpathlineto{\pgfqpoint{1.261601in}{1.345649in}}%
\pgfpathlineto{\pgfqpoint{1.087855in}{1.345649in}}%
\pgfpathlineto{\pgfqpoint{1.087855in}{1.345649in}}%
\pgfpathclose%
\pgfusepath{stroke,fill}%
\end{pgfscope}%
\begin{pgfscope}%
\pgfpathrectangle{\pgfqpoint{0.694444in}{0.416667in}}{\pgfqpoint{3.194444in}{1.416667in}}%
\pgfusepath{clip}%
\pgfsetbuttcap%
\pgfsetmiterjoin%
\definecolor{currentfill}{rgb}{0.447059,0.447059,0.447059}%
\pgfsetfillcolor{currentfill}%
\pgfsetlinewidth{1.003750pt}%
\definecolor{currentstroke}{rgb}{0.266667,0.266667,0.266667}%
\pgfsetstrokecolor{currentstroke}%
\pgfsetdash{}{0pt}%
\pgfpathmoveto{\pgfqpoint{1.336064in}{1.385574in}}%
\pgfpathlineto{\pgfqpoint{1.509810in}{1.385574in}}%
\pgfpathlineto{\pgfqpoint{1.509810in}{1.385574in}}%
\pgfpathlineto{\pgfqpoint{1.336064in}{1.385574in}}%
\pgfpathlineto{\pgfqpoint{1.336064in}{1.385574in}}%
\pgfpathclose%
\pgfusepath{stroke,fill}%
\end{pgfscope}%
\begin{pgfscope}%
\pgfpathrectangle{\pgfqpoint{0.694444in}{0.416667in}}{\pgfqpoint{3.194444in}{1.416667in}}%
\pgfusepath{clip}%
\pgfsetbuttcap%
\pgfsetmiterjoin%
\definecolor{currentfill}{rgb}{0.447059,0.447059,0.447059}%
\pgfsetfillcolor{currentfill}%
\pgfsetlinewidth{1.003750pt}%
\definecolor{currentstroke}{rgb}{0.266667,0.266667,0.266667}%
\pgfsetstrokecolor{currentstroke}%
\pgfsetdash{}{0pt}%
\pgfpathmoveto{\pgfqpoint{1.584272in}{1.506263in}}%
\pgfpathlineto{\pgfqpoint{1.758018in}{1.506263in}}%
\pgfpathlineto{\pgfqpoint{1.758018in}{1.506263in}}%
\pgfpathlineto{\pgfqpoint{1.584272in}{1.506263in}}%
\pgfpathlineto{\pgfqpoint{1.584272in}{1.506263in}}%
\pgfpathclose%
\pgfusepath{stroke,fill}%
\end{pgfscope}%
\begin{pgfscope}%
\pgfpathrectangle{\pgfqpoint{0.694444in}{0.416667in}}{\pgfqpoint{3.194444in}{1.416667in}}%
\pgfusepath{clip}%
\pgfsetbuttcap%
\pgfsetmiterjoin%
\definecolor{currentfill}{rgb}{0.447059,0.447059,0.447059}%
\pgfsetfillcolor{currentfill}%
\pgfsetlinewidth{1.003750pt}%
\definecolor{currentstroke}{rgb}{0.266667,0.266667,0.266667}%
\pgfsetstrokecolor{currentstroke}%
\pgfsetdash{}{0pt}%
\pgfpathmoveto{\pgfqpoint{1.832481in}{1.658613in}}%
\pgfpathlineto{\pgfqpoint{2.006227in}{1.658613in}}%
\pgfpathlineto{\pgfqpoint{2.006227in}{1.658613in}}%
\pgfpathlineto{\pgfqpoint{1.832481in}{1.658613in}}%
\pgfpathlineto{\pgfqpoint{1.832481in}{1.658613in}}%
\pgfpathclose%
\pgfusepath{stroke,fill}%
\end{pgfscope}%
\begin{pgfscope}%
\pgfpathrectangle{\pgfqpoint{0.694444in}{0.416667in}}{\pgfqpoint{3.194444in}{1.416667in}}%
\pgfusepath{clip}%
\pgfsetbuttcap%
\pgfsetmiterjoin%
\definecolor{currentfill}{rgb}{0.447059,0.447059,0.447059}%
\pgfsetfillcolor{currentfill}%
\pgfsetlinewidth{1.003750pt}%
\definecolor{currentstroke}{rgb}{0.266667,0.266667,0.266667}%
\pgfsetstrokecolor{currentstroke}%
\pgfsetdash{}{0pt}%
\pgfpathmoveto{\pgfqpoint{2.080689in}{1.515946in}}%
\pgfpathlineto{\pgfqpoint{2.254435in}{1.515946in}}%
\pgfpathlineto{\pgfqpoint{2.254435in}{1.515946in}}%
\pgfpathlineto{\pgfqpoint{2.080689in}{1.515946in}}%
\pgfpathlineto{\pgfqpoint{2.080689in}{1.515946in}}%
\pgfpathclose%
\pgfusepath{stroke,fill}%
\end{pgfscope}%
\begin{pgfscope}%
\pgfpathrectangle{\pgfqpoint{0.694444in}{0.416667in}}{\pgfqpoint{3.194444in}{1.416667in}}%
\pgfusepath{clip}%
\pgfsetbuttcap%
\pgfsetmiterjoin%
\definecolor{currentfill}{rgb}{0.447059,0.447059,0.447059}%
\pgfsetfillcolor{currentfill}%
\pgfsetlinewidth{1.003750pt}%
\definecolor{currentstroke}{rgb}{0.266667,0.266667,0.266667}%
\pgfsetstrokecolor{currentstroke}%
\pgfsetdash{}{0pt}%
\pgfpathmoveto{\pgfqpoint{2.328898in}{1.490246in}}%
\pgfpathlineto{\pgfqpoint{2.502644in}{1.490246in}}%
\pgfpathlineto{\pgfqpoint{2.502644in}{1.490246in}}%
\pgfpathlineto{\pgfqpoint{2.328898in}{1.490246in}}%
\pgfpathlineto{\pgfqpoint{2.328898in}{1.490246in}}%
\pgfpathclose%
\pgfusepath{stroke,fill}%
\end{pgfscope}%
\begin{pgfscope}%
\pgfpathrectangle{\pgfqpoint{0.694444in}{0.416667in}}{\pgfqpoint{3.194444in}{1.416667in}}%
\pgfusepath{clip}%
\pgfsetbuttcap%
\pgfsetmiterjoin%
\definecolor{currentfill}{rgb}{0.447059,0.447059,0.447059}%
\pgfsetfillcolor{currentfill}%
\pgfsetlinewidth{1.003750pt}%
\definecolor{currentstroke}{rgb}{0.266667,0.266667,0.266667}%
\pgfsetstrokecolor{currentstroke}%
\pgfsetdash{}{0pt}%
\pgfpathmoveto{\pgfqpoint{2.577107in}{1.587320in}}%
\pgfpathlineto{\pgfqpoint{2.750853in}{1.587320in}}%
\pgfpathlineto{\pgfqpoint{2.750853in}{1.587320in}}%
\pgfpathlineto{\pgfqpoint{2.577107in}{1.587320in}}%
\pgfpathlineto{\pgfqpoint{2.577107in}{1.587320in}}%
\pgfpathclose%
\pgfusepath{stroke,fill}%
\end{pgfscope}%
\begin{pgfscope}%
\pgfpathrectangle{\pgfqpoint{0.694444in}{0.416667in}}{\pgfqpoint{3.194444in}{1.416667in}}%
\pgfusepath{clip}%
\pgfsetbuttcap%
\pgfsetmiterjoin%
\definecolor{currentfill}{rgb}{0.447059,0.447059,0.447059}%
\pgfsetfillcolor{currentfill}%
\pgfsetlinewidth{1.003750pt}%
\definecolor{currentstroke}{rgb}{0.266667,0.266667,0.266667}%
\pgfsetstrokecolor{currentstroke}%
\pgfsetdash{}{0pt}%
\pgfpathmoveto{\pgfqpoint{2.825315in}{1.610320in}}%
\pgfpathlineto{\pgfqpoint{2.999061in}{1.610320in}}%
\pgfpathlineto{\pgfqpoint{2.999061in}{1.610320in}}%
\pgfpathlineto{\pgfqpoint{2.825315in}{1.610320in}}%
\pgfpathlineto{\pgfqpoint{2.825315in}{1.610320in}}%
\pgfpathclose%
\pgfusepath{stroke,fill}%
\end{pgfscope}%
\begin{pgfscope}%
\pgfpathrectangle{\pgfqpoint{0.694444in}{0.416667in}}{\pgfqpoint{3.194444in}{1.416667in}}%
\pgfusepath{clip}%
\pgfsetbuttcap%
\pgfsetmiterjoin%
\definecolor{currentfill}{rgb}{0.447059,0.447059,0.447059}%
\pgfsetfillcolor{currentfill}%
\pgfsetlinewidth{1.003750pt}%
\definecolor{currentstroke}{rgb}{0.266667,0.266667,0.266667}%
\pgfsetstrokecolor{currentstroke}%
\pgfsetdash{}{0pt}%
\pgfpathmoveto{\pgfqpoint{3.073524in}{1.670646in}}%
\pgfpathlineto{\pgfqpoint{3.247270in}{1.670646in}}%
\pgfpathlineto{\pgfqpoint{3.247270in}{1.670646in}}%
\pgfpathlineto{\pgfqpoint{3.073524in}{1.670646in}}%
\pgfpathlineto{\pgfqpoint{3.073524in}{1.670646in}}%
\pgfpathclose%
\pgfusepath{stroke,fill}%
\end{pgfscope}%
\begin{pgfscope}%
\pgfpathrectangle{\pgfqpoint{0.694444in}{0.416667in}}{\pgfqpoint{3.194444in}{1.416667in}}%
\pgfusepath{clip}%
\pgfsetbuttcap%
\pgfsetmiterjoin%
\definecolor{currentfill}{rgb}{0.447059,0.447059,0.447059}%
\pgfsetfillcolor{currentfill}%
\pgfsetlinewidth{1.003750pt}%
\definecolor{currentstroke}{rgb}{0.266667,0.266667,0.266667}%
\pgfsetstrokecolor{currentstroke}%
\pgfsetdash{}{0pt}%
\pgfpathmoveto{\pgfqpoint{3.321732in}{1.612768in}}%
\pgfpathlineto{\pgfqpoint{3.495478in}{1.612768in}}%
\pgfpathlineto{\pgfqpoint{3.495478in}{1.612768in}}%
\pgfpathlineto{\pgfqpoint{3.321732in}{1.612768in}}%
\pgfpathlineto{\pgfqpoint{3.321732in}{1.612768in}}%
\pgfpathclose%
\pgfusepath{stroke,fill}%
\end{pgfscope}%
\begin{pgfscope}%
\pgfpathrectangle{\pgfqpoint{0.694444in}{0.416667in}}{\pgfqpoint{3.194444in}{1.416667in}}%
\pgfusepath{clip}%
\pgfsetbuttcap%
\pgfsetmiterjoin%
\definecolor{currentfill}{rgb}{0.447059,0.447059,0.447059}%
\pgfsetfillcolor{currentfill}%
\pgfsetlinewidth{1.003750pt}%
\definecolor{currentstroke}{rgb}{0.266667,0.266667,0.266667}%
\pgfsetstrokecolor{currentstroke}%
\pgfsetdash{}{0pt}%
\pgfpathmoveto{\pgfqpoint{3.569941in}{1.646719in}}%
\pgfpathlineto{\pgfqpoint{3.743687in}{1.646719in}}%
\pgfpathlineto{\pgfqpoint{3.743687in}{1.646907in}}%
\pgfpathlineto{\pgfqpoint{3.569941in}{1.646907in}}%
\pgfpathlineto{\pgfqpoint{3.569941in}{1.646719in}}%
\pgfpathclose%
\pgfusepath{stroke,fill}%
\end{pgfscope}%
\begin{pgfscope}%
\pgfpathrectangle{\pgfqpoint{0.694444in}{0.416667in}}{\pgfqpoint{3.194444in}{1.416667in}}%
\pgfusepath{clip}%
\pgfsetbuttcap%
\pgfsetmiterjoin%
\definecolor{currentfill}{rgb}{0.447059,0.447059,0.447059}%
\pgfsetfillcolor{currentfill}%
\pgfsetlinewidth{1.003750pt}%
\definecolor{currentstroke}{rgb}{0.266667,0.266667,0.266667}%
\pgfsetstrokecolor{currentstroke}%
\pgfsetdash{}{0pt}%
\pgfpathmoveto{\pgfqpoint{0.839646in}{1.416154in}}%
\pgfpathlineto{\pgfqpoint{1.013392in}{1.416154in}}%
\pgfpathlineto{\pgfqpoint{1.013392in}{1.416154in}}%
\pgfpathlineto{\pgfqpoint{0.839646in}{1.416154in}}%
\pgfpathlineto{\pgfqpoint{0.839646in}{1.416154in}}%
\pgfpathclose%
\pgfusepath{stroke,fill}%
\end{pgfscope}%
\begin{pgfscope}%
\pgfpathrectangle{\pgfqpoint{0.694444in}{0.416667in}}{\pgfqpoint{3.194444in}{1.416667in}}%
\pgfusepath{clip}%
\pgfsetbuttcap%
\pgfsetmiterjoin%
\definecolor{currentfill}{rgb}{0.447059,0.447059,0.447059}%
\pgfsetfillcolor{currentfill}%
\pgfsetlinewidth{1.003750pt}%
\definecolor{currentstroke}{rgb}{0.266667,0.266667,0.266667}%
\pgfsetstrokecolor{currentstroke}%
\pgfsetdash{}{0pt}%
\pgfpathmoveto{\pgfqpoint{1.087855in}{1.345649in}}%
\pgfpathlineto{\pgfqpoint{1.261601in}{1.345649in}}%
\pgfpathlineto{\pgfqpoint{1.261601in}{1.345649in}}%
\pgfpathlineto{\pgfqpoint{1.087855in}{1.345649in}}%
\pgfpathlineto{\pgfqpoint{1.087855in}{1.345649in}}%
\pgfpathclose%
\pgfusepath{stroke,fill}%
\end{pgfscope}%
\begin{pgfscope}%
\pgfpathrectangle{\pgfqpoint{0.694444in}{0.416667in}}{\pgfqpoint{3.194444in}{1.416667in}}%
\pgfusepath{clip}%
\pgfsetbuttcap%
\pgfsetmiterjoin%
\definecolor{currentfill}{rgb}{0.447059,0.447059,0.447059}%
\pgfsetfillcolor{currentfill}%
\pgfsetlinewidth{1.003750pt}%
\definecolor{currentstroke}{rgb}{0.266667,0.266667,0.266667}%
\pgfsetstrokecolor{currentstroke}%
\pgfsetdash{}{0pt}%
\pgfpathmoveto{\pgfqpoint{1.336064in}{1.385574in}}%
\pgfpathlineto{\pgfqpoint{1.509810in}{1.385574in}}%
\pgfpathlineto{\pgfqpoint{1.509810in}{1.385574in}}%
\pgfpathlineto{\pgfqpoint{1.336064in}{1.385574in}}%
\pgfpathlineto{\pgfqpoint{1.336064in}{1.385574in}}%
\pgfpathclose%
\pgfusepath{stroke,fill}%
\end{pgfscope}%
\begin{pgfscope}%
\pgfpathrectangle{\pgfqpoint{0.694444in}{0.416667in}}{\pgfqpoint{3.194444in}{1.416667in}}%
\pgfusepath{clip}%
\pgfsetbuttcap%
\pgfsetmiterjoin%
\definecolor{currentfill}{rgb}{0.447059,0.447059,0.447059}%
\pgfsetfillcolor{currentfill}%
\pgfsetlinewidth{1.003750pt}%
\definecolor{currentstroke}{rgb}{0.266667,0.266667,0.266667}%
\pgfsetstrokecolor{currentstroke}%
\pgfsetdash{}{0pt}%
\pgfpathmoveto{\pgfqpoint{1.584272in}{1.506263in}}%
\pgfpathlineto{\pgfqpoint{1.758018in}{1.506263in}}%
\pgfpathlineto{\pgfqpoint{1.758018in}{1.506263in}}%
\pgfpathlineto{\pgfqpoint{1.584272in}{1.506263in}}%
\pgfpathlineto{\pgfqpoint{1.584272in}{1.506263in}}%
\pgfpathclose%
\pgfusepath{stroke,fill}%
\end{pgfscope}%
\begin{pgfscope}%
\pgfpathrectangle{\pgfqpoint{0.694444in}{0.416667in}}{\pgfqpoint{3.194444in}{1.416667in}}%
\pgfusepath{clip}%
\pgfsetbuttcap%
\pgfsetmiterjoin%
\definecolor{currentfill}{rgb}{0.447059,0.447059,0.447059}%
\pgfsetfillcolor{currentfill}%
\pgfsetlinewidth{1.003750pt}%
\definecolor{currentstroke}{rgb}{0.266667,0.266667,0.266667}%
\pgfsetstrokecolor{currentstroke}%
\pgfsetdash{}{0pt}%
\pgfpathmoveto{\pgfqpoint{1.832481in}{1.658613in}}%
\pgfpathlineto{\pgfqpoint{2.006227in}{1.658613in}}%
\pgfpathlineto{\pgfqpoint{2.006227in}{1.658613in}}%
\pgfpathlineto{\pgfqpoint{1.832481in}{1.658613in}}%
\pgfpathlineto{\pgfqpoint{1.832481in}{1.658613in}}%
\pgfpathclose%
\pgfusepath{stroke,fill}%
\end{pgfscope}%
\begin{pgfscope}%
\pgfpathrectangle{\pgfqpoint{0.694444in}{0.416667in}}{\pgfqpoint{3.194444in}{1.416667in}}%
\pgfusepath{clip}%
\pgfsetbuttcap%
\pgfsetmiterjoin%
\definecolor{currentfill}{rgb}{0.447059,0.447059,0.447059}%
\pgfsetfillcolor{currentfill}%
\pgfsetlinewidth{1.003750pt}%
\definecolor{currentstroke}{rgb}{0.266667,0.266667,0.266667}%
\pgfsetstrokecolor{currentstroke}%
\pgfsetdash{}{0pt}%
\pgfpathmoveto{\pgfqpoint{2.080689in}{1.515946in}}%
\pgfpathlineto{\pgfqpoint{2.254435in}{1.515946in}}%
\pgfpathlineto{\pgfqpoint{2.254435in}{1.515946in}}%
\pgfpathlineto{\pgfqpoint{2.080689in}{1.515946in}}%
\pgfpathlineto{\pgfqpoint{2.080689in}{1.515946in}}%
\pgfpathclose%
\pgfusepath{stroke,fill}%
\end{pgfscope}%
\begin{pgfscope}%
\pgfpathrectangle{\pgfqpoint{0.694444in}{0.416667in}}{\pgfqpoint{3.194444in}{1.416667in}}%
\pgfusepath{clip}%
\pgfsetbuttcap%
\pgfsetmiterjoin%
\definecolor{currentfill}{rgb}{0.447059,0.447059,0.447059}%
\pgfsetfillcolor{currentfill}%
\pgfsetlinewidth{1.003750pt}%
\definecolor{currentstroke}{rgb}{0.266667,0.266667,0.266667}%
\pgfsetstrokecolor{currentstroke}%
\pgfsetdash{}{0pt}%
\pgfpathmoveto{\pgfqpoint{2.328898in}{1.490246in}}%
\pgfpathlineto{\pgfqpoint{2.502644in}{1.490246in}}%
\pgfpathlineto{\pgfqpoint{2.502644in}{1.490246in}}%
\pgfpathlineto{\pgfqpoint{2.328898in}{1.490246in}}%
\pgfpathlineto{\pgfqpoint{2.328898in}{1.490246in}}%
\pgfpathclose%
\pgfusepath{stroke,fill}%
\end{pgfscope}%
\begin{pgfscope}%
\pgfpathrectangle{\pgfqpoint{0.694444in}{0.416667in}}{\pgfqpoint{3.194444in}{1.416667in}}%
\pgfusepath{clip}%
\pgfsetbuttcap%
\pgfsetmiterjoin%
\definecolor{currentfill}{rgb}{0.447059,0.447059,0.447059}%
\pgfsetfillcolor{currentfill}%
\pgfsetlinewidth{1.003750pt}%
\definecolor{currentstroke}{rgb}{0.266667,0.266667,0.266667}%
\pgfsetstrokecolor{currentstroke}%
\pgfsetdash{}{0pt}%
\pgfpathmoveto{\pgfqpoint{2.577107in}{1.587320in}}%
\pgfpathlineto{\pgfqpoint{2.750853in}{1.587320in}}%
\pgfpathlineto{\pgfqpoint{2.750853in}{1.587320in}}%
\pgfpathlineto{\pgfqpoint{2.577107in}{1.587320in}}%
\pgfpathlineto{\pgfqpoint{2.577107in}{1.587320in}}%
\pgfpathclose%
\pgfusepath{stroke,fill}%
\end{pgfscope}%
\begin{pgfscope}%
\pgfpathrectangle{\pgfqpoint{0.694444in}{0.416667in}}{\pgfqpoint{3.194444in}{1.416667in}}%
\pgfusepath{clip}%
\pgfsetbuttcap%
\pgfsetmiterjoin%
\definecolor{currentfill}{rgb}{0.447059,0.447059,0.447059}%
\pgfsetfillcolor{currentfill}%
\pgfsetlinewidth{1.003750pt}%
\definecolor{currentstroke}{rgb}{0.266667,0.266667,0.266667}%
\pgfsetstrokecolor{currentstroke}%
\pgfsetdash{}{0pt}%
\pgfpathmoveto{\pgfqpoint{2.825315in}{1.610320in}}%
\pgfpathlineto{\pgfqpoint{2.999061in}{1.610320in}}%
\pgfpathlineto{\pgfqpoint{2.999061in}{1.610336in}}%
\pgfpathlineto{\pgfqpoint{2.825315in}{1.610336in}}%
\pgfpathlineto{\pgfqpoint{2.825315in}{1.610320in}}%
\pgfpathclose%
\pgfusepath{stroke,fill}%
\end{pgfscope}%
\begin{pgfscope}%
\pgfpathrectangle{\pgfqpoint{0.694444in}{0.416667in}}{\pgfqpoint{3.194444in}{1.416667in}}%
\pgfusepath{clip}%
\pgfsetbuttcap%
\pgfsetmiterjoin%
\definecolor{currentfill}{rgb}{0.447059,0.447059,0.447059}%
\pgfsetfillcolor{currentfill}%
\pgfsetlinewidth{1.003750pt}%
\definecolor{currentstroke}{rgb}{0.266667,0.266667,0.266667}%
\pgfsetstrokecolor{currentstroke}%
\pgfsetdash{}{0pt}%
\pgfpathmoveto{\pgfqpoint{3.073524in}{1.670646in}}%
\pgfpathlineto{\pgfqpoint{3.247270in}{1.670646in}}%
\pgfpathlineto{\pgfqpoint{3.247270in}{1.670885in}}%
\pgfpathlineto{\pgfqpoint{3.073524in}{1.670885in}}%
\pgfpathlineto{\pgfqpoint{3.073524in}{1.670646in}}%
\pgfpathclose%
\pgfusepath{stroke,fill}%
\end{pgfscope}%
\begin{pgfscope}%
\pgfpathrectangle{\pgfqpoint{0.694444in}{0.416667in}}{\pgfqpoint{3.194444in}{1.416667in}}%
\pgfusepath{clip}%
\pgfsetbuttcap%
\pgfsetmiterjoin%
\definecolor{currentfill}{rgb}{0.447059,0.447059,0.447059}%
\pgfsetfillcolor{currentfill}%
\pgfsetlinewidth{1.003750pt}%
\definecolor{currentstroke}{rgb}{0.266667,0.266667,0.266667}%
\pgfsetstrokecolor{currentstroke}%
\pgfsetdash{}{0pt}%
\pgfpathmoveto{\pgfqpoint{3.321732in}{1.612768in}}%
\pgfpathlineto{\pgfqpoint{3.495478in}{1.612768in}}%
\pgfpathlineto{\pgfqpoint{3.495478in}{1.612898in}}%
\pgfpathlineto{\pgfqpoint{3.321732in}{1.612898in}}%
\pgfpathlineto{\pgfqpoint{3.321732in}{1.612768in}}%
\pgfpathclose%
\pgfusepath{stroke,fill}%
\end{pgfscope}%
\begin{pgfscope}%
\pgfpathrectangle{\pgfqpoint{0.694444in}{0.416667in}}{\pgfqpoint{3.194444in}{1.416667in}}%
\pgfusepath{clip}%
\pgfsetbuttcap%
\pgfsetmiterjoin%
\definecolor{currentfill}{rgb}{0.447059,0.447059,0.447059}%
\pgfsetfillcolor{currentfill}%
\pgfsetlinewidth{1.003750pt}%
\definecolor{currentstroke}{rgb}{0.266667,0.266667,0.266667}%
\pgfsetstrokecolor{currentstroke}%
\pgfsetdash{}{0pt}%
\pgfpathmoveto{\pgfqpoint{3.569941in}{1.646907in}}%
\pgfpathlineto{\pgfqpoint{3.743687in}{1.646907in}}%
\pgfpathlineto{\pgfqpoint{3.743687in}{1.646995in}}%
\pgfpathlineto{\pgfqpoint{3.569941in}{1.646995in}}%
\pgfpathlineto{\pgfqpoint{3.569941in}{1.646907in}}%
\pgfpathclose%
\pgfusepath{stroke,fill}%
\end{pgfscope}%
\begin{pgfscope}%
\pgfpathrectangle{\pgfqpoint{0.694444in}{0.416667in}}{\pgfqpoint{3.194444in}{1.416667in}}%
\pgfusepath{clip}%
\pgfsetbuttcap%
\pgfsetmiterjoin%
\definecolor{currentfill}{rgb}{0.447059,0.447059,0.447059}%
\pgfsetfillcolor{currentfill}%
\pgfsetlinewidth{1.003750pt}%
\definecolor{currentstroke}{rgb}{0.266667,0.266667,0.266667}%
\pgfsetstrokecolor{currentstroke}%
\pgfsetdash{}{0pt}%
\pgfpathmoveto{\pgfqpoint{0.839646in}{1.416154in}}%
\pgfpathlineto{\pgfqpoint{1.013392in}{1.416154in}}%
\pgfpathlineto{\pgfqpoint{1.013392in}{1.500470in}}%
\pgfpathlineto{\pgfqpoint{0.839646in}{1.500470in}}%
\pgfpathlineto{\pgfqpoint{0.839646in}{1.416154in}}%
\pgfpathclose%
\pgfusepath{stroke,fill}%
\end{pgfscope}%
\begin{pgfscope}%
\pgfpathrectangle{\pgfqpoint{0.694444in}{0.416667in}}{\pgfqpoint{3.194444in}{1.416667in}}%
\pgfusepath{clip}%
\pgfsetbuttcap%
\pgfsetmiterjoin%
\definecolor{currentfill}{rgb}{0.447059,0.447059,0.447059}%
\pgfsetfillcolor{currentfill}%
\pgfsetlinewidth{1.003750pt}%
\definecolor{currentstroke}{rgb}{0.266667,0.266667,0.266667}%
\pgfsetstrokecolor{currentstroke}%
\pgfsetdash{}{0pt}%
\pgfpathmoveto{\pgfqpoint{1.087855in}{1.345649in}}%
\pgfpathlineto{\pgfqpoint{1.261601in}{1.345649in}}%
\pgfpathlineto{\pgfqpoint{1.261601in}{1.493113in}}%
\pgfpathlineto{\pgfqpoint{1.087855in}{1.493113in}}%
\pgfpathlineto{\pgfqpoint{1.087855in}{1.345649in}}%
\pgfpathclose%
\pgfusepath{stroke,fill}%
\end{pgfscope}%
\begin{pgfscope}%
\pgfpathrectangle{\pgfqpoint{0.694444in}{0.416667in}}{\pgfqpoint{3.194444in}{1.416667in}}%
\pgfusepath{clip}%
\pgfsetbuttcap%
\pgfsetmiterjoin%
\definecolor{currentfill}{rgb}{0.447059,0.447059,0.447059}%
\pgfsetfillcolor{currentfill}%
\pgfsetlinewidth{1.003750pt}%
\definecolor{currentstroke}{rgb}{0.266667,0.266667,0.266667}%
\pgfsetstrokecolor{currentstroke}%
\pgfsetdash{}{0pt}%
\pgfpathmoveto{\pgfqpoint{1.336064in}{1.385574in}}%
\pgfpathlineto{\pgfqpoint{1.509810in}{1.385574in}}%
\pgfpathlineto{\pgfqpoint{1.509810in}{1.471004in}}%
\pgfpathlineto{\pgfqpoint{1.336064in}{1.471004in}}%
\pgfpathlineto{\pgfqpoint{1.336064in}{1.385574in}}%
\pgfpathclose%
\pgfusepath{stroke,fill}%
\end{pgfscope}%
\begin{pgfscope}%
\pgfpathrectangle{\pgfqpoint{0.694444in}{0.416667in}}{\pgfqpoint{3.194444in}{1.416667in}}%
\pgfusepath{clip}%
\pgfsetbuttcap%
\pgfsetmiterjoin%
\definecolor{currentfill}{rgb}{0.447059,0.447059,0.447059}%
\pgfsetfillcolor{currentfill}%
\pgfsetlinewidth{1.003750pt}%
\definecolor{currentstroke}{rgb}{0.266667,0.266667,0.266667}%
\pgfsetstrokecolor{currentstroke}%
\pgfsetdash{}{0pt}%
\pgfpathmoveto{\pgfqpoint{1.584272in}{1.506263in}}%
\pgfpathlineto{\pgfqpoint{1.758018in}{1.506263in}}%
\pgfpathlineto{\pgfqpoint{1.758018in}{1.609389in}}%
\pgfpathlineto{\pgfqpoint{1.584272in}{1.609389in}}%
\pgfpathlineto{\pgfqpoint{1.584272in}{1.506263in}}%
\pgfpathclose%
\pgfusepath{stroke,fill}%
\end{pgfscope}%
\begin{pgfscope}%
\pgfpathrectangle{\pgfqpoint{0.694444in}{0.416667in}}{\pgfqpoint{3.194444in}{1.416667in}}%
\pgfusepath{clip}%
\pgfsetbuttcap%
\pgfsetmiterjoin%
\definecolor{currentfill}{rgb}{0.447059,0.447059,0.447059}%
\pgfsetfillcolor{currentfill}%
\pgfsetlinewidth{1.003750pt}%
\definecolor{currentstroke}{rgb}{0.266667,0.266667,0.266667}%
\pgfsetstrokecolor{currentstroke}%
\pgfsetdash{}{0pt}%
\pgfpathmoveto{\pgfqpoint{1.832481in}{1.658613in}}%
\pgfpathlineto{\pgfqpoint{2.006227in}{1.658613in}}%
\pgfpathlineto{\pgfqpoint{2.006227in}{1.659762in}}%
\pgfpathlineto{\pgfqpoint{1.832481in}{1.659762in}}%
\pgfpathlineto{\pgfqpoint{1.832481in}{1.658613in}}%
\pgfpathclose%
\pgfusepath{stroke,fill}%
\end{pgfscope}%
\begin{pgfscope}%
\pgfpathrectangle{\pgfqpoint{0.694444in}{0.416667in}}{\pgfqpoint{3.194444in}{1.416667in}}%
\pgfusepath{clip}%
\pgfsetbuttcap%
\pgfsetmiterjoin%
\definecolor{currentfill}{rgb}{0.447059,0.447059,0.447059}%
\pgfsetfillcolor{currentfill}%
\pgfsetlinewidth{1.003750pt}%
\definecolor{currentstroke}{rgb}{0.266667,0.266667,0.266667}%
\pgfsetstrokecolor{currentstroke}%
\pgfsetdash{}{0pt}%
\pgfpathmoveto{\pgfqpoint{2.080689in}{1.515946in}}%
\pgfpathlineto{\pgfqpoint{2.254435in}{1.515946in}}%
\pgfpathlineto{\pgfqpoint{2.254435in}{1.669671in}}%
\pgfpathlineto{\pgfqpoint{2.080689in}{1.669671in}}%
\pgfpathlineto{\pgfqpoint{2.080689in}{1.515946in}}%
\pgfpathclose%
\pgfusepath{stroke,fill}%
\end{pgfscope}%
\begin{pgfscope}%
\pgfpathrectangle{\pgfqpoint{0.694444in}{0.416667in}}{\pgfqpoint{3.194444in}{1.416667in}}%
\pgfusepath{clip}%
\pgfsetbuttcap%
\pgfsetmiterjoin%
\definecolor{currentfill}{rgb}{0.447059,0.447059,0.447059}%
\pgfsetfillcolor{currentfill}%
\pgfsetlinewidth{1.003750pt}%
\definecolor{currentstroke}{rgb}{0.266667,0.266667,0.266667}%
\pgfsetstrokecolor{currentstroke}%
\pgfsetdash{}{0pt}%
\pgfpathmoveto{\pgfqpoint{2.328898in}{1.490246in}}%
\pgfpathlineto{\pgfqpoint{2.502644in}{1.490246in}}%
\pgfpathlineto{\pgfqpoint{2.502644in}{1.631526in}}%
\pgfpathlineto{\pgfqpoint{2.328898in}{1.631526in}}%
\pgfpathlineto{\pgfqpoint{2.328898in}{1.490246in}}%
\pgfpathclose%
\pgfusepath{stroke,fill}%
\end{pgfscope}%
\begin{pgfscope}%
\pgfpathrectangle{\pgfqpoint{0.694444in}{0.416667in}}{\pgfqpoint{3.194444in}{1.416667in}}%
\pgfusepath{clip}%
\pgfsetbuttcap%
\pgfsetmiterjoin%
\definecolor{currentfill}{rgb}{0.447059,0.447059,0.447059}%
\pgfsetfillcolor{currentfill}%
\pgfsetlinewidth{1.003750pt}%
\definecolor{currentstroke}{rgb}{0.266667,0.266667,0.266667}%
\pgfsetstrokecolor{currentstroke}%
\pgfsetdash{}{0pt}%
\pgfpathmoveto{\pgfqpoint{2.577107in}{1.587320in}}%
\pgfpathlineto{\pgfqpoint{2.750853in}{1.587320in}}%
\pgfpathlineto{\pgfqpoint{2.750853in}{1.587320in}}%
\pgfpathlineto{\pgfqpoint{2.577107in}{1.587320in}}%
\pgfpathlineto{\pgfqpoint{2.577107in}{1.587320in}}%
\pgfpathclose%
\pgfusepath{stroke,fill}%
\end{pgfscope}%
\begin{pgfscope}%
\pgfpathrectangle{\pgfqpoint{0.694444in}{0.416667in}}{\pgfqpoint{3.194444in}{1.416667in}}%
\pgfusepath{clip}%
\pgfsetbuttcap%
\pgfsetmiterjoin%
\definecolor{currentfill}{rgb}{0.447059,0.447059,0.447059}%
\pgfsetfillcolor{currentfill}%
\pgfsetlinewidth{1.003750pt}%
\definecolor{currentstroke}{rgb}{0.266667,0.266667,0.266667}%
\pgfsetstrokecolor{currentstroke}%
\pgfsetdash{}{0pt}%
\pgfpathmoveto{\pgfqpoint{2.825315in}{1.610336in}}%
\pgfpathlineto{\pgfqpoint{2.999061in}{1.610336in}}%
\pgfpathlineto{\pgfqpoint{2.999061in}{1.610349in}}%
\pgfpathlineto{\pgfqpoint{2.825315in}{1.610349in}}%
\pgfpathlineto{\pgfqpoint{2.825315in}{1.610336in}}%
\pgfpathclose%
\pgfusepath{stroke,fill}%
\end{pgfscope}%
\begin{pgfscope}%
\pgfpathrectangle{\pgfqpoint{0.694444in}{0.416667in}}{\pgfqpoint{3.194444in}{1.416667in}}%
\pgfusepath{clip}%
\pgfsetbuttcap%
\pgfsetmiterjoin%
\definecolor{currentfill}{rgb}{0.447059,0.447059,0.447059}%
\pgfsetfillcolor{currentfill}%
\pgfsetlinewidth{1.003750pt}%
\definecolor{currentstroke}{rgb}{0.266667,0.266667,0.266667}%
\pgfsetstrokecolor{currentstroke}%
\pgfsetdash{}{0pt}%
\pgfpathmoveto{\pgfqpoint{3.073524in}{1.670885in}}%
\pgfpathlineto{\pgfqpoint{3.247270in}{1.670885in}}%
\pgfpathlineto{\pgfqpoint{3.247270in}{1.671027in}}%
\pgfpathlineto{\pgfqpoint{3.073524in}{1.671027in}}%
\pgfpathlineto{\pgfqpoint{3.073524in}{1.670885in}}%
\pgfpathclose%
\pgfusepath{stroke,fill}%
\end{pgfscope}%
\begin{pgfscope}%
\pgfpathrectangle{\pgfqpoint{0.694444in}{0.416667in}}{\pgfqpoint{3.194444in}{1.416667in}}%
\pgfusepath{clip}%
\pgfsetbuttcap%
\pgfsetmiterjoin%
\definecolor{currentfill}{rgb}{0.447059,0.447059,0.447059}%
\pgfsetfillcolor{currentfill}%
\pgfsetlinewidth{1.003750pt}%
\definecolor{currentstroke}{rgb}{0.266667,0.266667,0.266667}%
\pgfsetstrokecolor{currentstroke}%
\pgfsetdash{}{0pt}%
\pgfpathmoveto{\pgfqpoint{3.321732in}{1.612898in}}%
\pgfpathlineto{\pgfqpoint{3.495478in}{1.612898in}}%
\pgfpathlineto{\pgfqpoint{3.495478in}{1.612970in}}%
\pgfpathlineto{\pgfqpoint{3.321732in}{1.612970in}}%
\pgfpathlineto{\pgfqpoint{3.321732in}{1.612898in}}%
\pgfpathclose%
\pgfusepath{stroke,fill}%
\end{pgfscope}%
\begin{pgfscope}%
\pgfpathrectangle{\pgfqpoint{0.694444in}{0.416667in}}{\pgfqpoint{3.194444in}{1.416667in}}%
\pgfusepath{clip}%
\pgfsetbuttcap%
\pgfsetmiterjoin%
\definecolor{currentfill}{rgb}{0.447059,0.447059,0.447059}%
\pgfsetfillcolor{currentfill}%
\pgfsetlinewidth{1.003750pt}%
\definecolor{currentstroke}{rgb}{0.266667,0.266667,0.266667}%
\pgfsetstrokecolor{currentstroke}%
\pgfsetdash{}{0pt}%
\pgfpathmoveto{\pgfqpoint{3.569941in}{1.646995in}}%
\pgfpathlineto{\pgfqpoint{3.743687in}{1.646995in}}%
\pgfpathlineto{\pgfqpoint{3.743687in}{1.647066in}}%
\pgfpathlineto{\pgfqpoint{3.569941in}{1.647066in}}%
\pgfpathlineto{\pgfqpoint{3.569941in}{1.646995in}}%
\pgfpathclose%
\pgfusepath{stroke,fill}%
\end{pgfscope}%
\begin{pgfscope}%
\pgfpathrectangle{\pgfqpoint{0.694444in}{0.416667in}}{\pgfqpoint{3.194444in}{1.416667in}}%
\pgfusepath{clip}%
\pgfsetbuttcap%
\pgfsetmiterjoin%
\definecolor{currentfill}{rgb}{0.447059,0.447059,0.447059}%
\pgfsetfillcolor{currentfill}%
\pgfsetlinewidth{1.003750pt}%
\definecolor{currentstroke}{rgb}{0.266667,0.266667,0.266667}%
\pgfsetstrokecolor{currentstroke}%
\pgfsetdash{}{0pt}%
\pgfpathmoveto{\pgfqpoint{0.839646in}{1.500470in}}%
\pgfpathlineto{\pgfqpoint{1.013392in}{1.500470in}}%
\pgfpathlineto{\pgfqpoint{1.013392in}{1.500470in}}%
\pgfpathlineto{\pgfqpoint{0.839646in}{1.500470in}}%
\pgfpathlineto{\pgfqpoint{0.839646in}{1.500470in}}%
\pgfpathclose%
\pgfusepath{stroke,fill}%
\end{pgfscope}%
\begin{pgfscope}%
\pgfpathrectangle{\pgfqpoint{0.694444in}{0.416667in}}{\pgfqpoint{3.194444in}{1.416667in}}%
\pgfusepath{clip}%
\pgfsetbuttcap%
\pgfsetmiterjoin%
\definecolor{currentfill}{rgb}{0.447059,0.447059,0.447059}%
\pgfsetfillcolor{currentfill}%
\pgfsetlinewidth{1.003750pt}%
\definecolor{currentstroke}{rgb}{0.266667,0.266667,0.266667}%
\pgfsetstrokecolor{currentstroke}%
\pgfsetdash{}{0pt}%
\pgfpathmoveto{\pgfqpoint{1.087855in}{1.493113in}}%
\pgfpathlineto{\pgfqpoint{1.261601in}{1.493113in}}%
\pgfpathlineto{\pgfqpoint{1.261601in}{1.493113in}}%
\pgfpathlineto{\pgfqpoint{1.087855in}{1.493113in}}%
\pgfpathlineto{\pgfqpoint{1.087855in}{1.493113in}}%
\pgfpathclose%
\pgfusepath{stroke,fill}%
\end{pgfscope}%
\begin{pgfscope}%
\pgfpathrectangle{\pgfqpoint{0.694444in}{0.416667in}}{\pgfqpoint{3.194444in}{1.416667in}}%
\pgfusepath{clip}%
\pgfsetbuttcap%
\pgfsetmiterjoin%
\definecolor{currentfill}{rgb}{0.447059,0.447059,0.447059}%
\pgfsetfillcolor{currentfill}%
\pgfsetlinewidth{1.003750pt}%
\definecolor{currentstroke}{rgb}{0.266667,0.266667,0.266667}%
\pgfsetstrokecolor{currentstroke}%
\pgfsetdash{}{0pt}%
\pgfpathmoveto{\pgfqpoint{1.336064in}{1.471004in}}%
\pgfpathlineto{\pgfqpoint{1.509810in}{1.471004in}}%
\pgfpathlineto{\pgfqpoint{1.509810in}{1.471004in}}%
\pgfpathlineto{\pgfqpoint{1.336064in}{1.471004in}}%
\pgfpathlineto{\pgfqpoint{1.336064in}{1.471004in}}%
\pgfpathclose%
\pgfusepath{stroke,fill}%
\end{pgfscope}%
\begin{pgfscope}%
\pgfpathrectangle{\pgfqpoint{0.694444in}{0.416667in}}{\pgfqpoint{3.194444in}{1.416667in}}%
\pgfusepath{clip}%
\pgfsetbuttcap%
\pgfsetmiterjoin%
\definecolor{currentfill}{rgb}{0.447059,0.447059,0.447059}%
\pgfsetfillcolor{currentfill}%
\pgfsetlinewidth{1.003750pt}%
\definecolor{currentstroke}{rgb}{0.266667,0.266667,0.266667}%
\pgfsetstrokecolor{currentstroke}%
\pgfsetdash{}{0pt}%
\pgfpathmoveto{\pgfqpoint{1.584272in}{1.609389in}}%
\pgfpathlineto{\pgfqpoint{1.758018in}{1.609389in}}%
\pgfpathlineto{\pgfqpoint{1.758018in}{1.609389in}}%
\pgfpathlineto{\pgfqpoint{1.584272in}{1.609389in}}%
\pgfpathlineto{\pgfqpoint{1.584272in}{1.609389in}}%
\pgfpathclose%
\pgfusepath{stroke,fill}%
\end{pgfscope}%
\begin{pgfscope}%
\pgfpathrectangle{\pgfqpoint{0.694444in}{0.416667in}}{\pgfqpoint{3.194444in}{1.416667in}}%
\pgfusepath{clip}%
\pgfsetbuttcap%
\pgfsetmiterjoin%
\definecolor{currentfill}{rgb}{0.447059,0.447059,0.447059}%
\pgfsetfillcolor{currentfill}%
\pgfsetlinewidth{1.003750pt}%
\definecolor{currentstroke}{rgb}{0.266667,0.266667,0.266667}%
\pgfsetstrokecolor{currentstroke}%
\pgfsetdash{}{0pt}%
\pgfpathmoveto{\pgfqpoint{1.832481in}{1.659762in}}%
\pgfpathlineto{\pgfqpoint{2.006227in}{1.659762in}}%
\pgfpathlineto{\pgfqpoint{2.006227in}{1.659762in}}%
\pgfpathlineto{\pgfqpoint{1.832481in}{1.659762in}}%
\pgfpathlineto{\pgfqpoint{1.832481in}{1.659762in}}%
\pgfpathclose%
\pgfusepath{stroke,fill}%
\end{pgfscope}%
\begin{pgfscope}%
\pgfpathrectangle{\pgfqpoint{0.694444in}{0.416667in}}{\pgfqpoint{3.194444in}{1.416667in}}%
\pgfusepath{clip}%
\pgfsetbuttcap%
\pgfsetmiterjoin%
\definecolor{currentfill}{rgb}{0.447059,0.447059,0.447059}%
\pgfsetfillcolor{currentfill}%
\pgfsetlinewidth{1.003750pt}%
\definecolor{currentstroke}{rgb}{0.266667,0.266667,0.266667}%
\pgfsetstrokecolor{currentstroke}%
\pgfsetdash{}{0pt}%
\pgfpathmoveto{\pgfqpoint{2.080689in}{1.669671in}}%
\pgfpathlineto{\pgfqpoint{2.254435in}{1.669671in}}%
\pgfpathlineto{\pgfqpoint{2.254435in}{1.669671in}}%
\pgfpathlineto{\pgfqpoint{2.080689in}{1.669671in}}%
\pgfpathlineto{\pgfqpoint{2.080689in}{1.669671in}}%
\pgfpathclose%
\pgfusepath{stroke,fill}%
\end{pgfscope}%
\begin{pgfscope}%
\pgfpathrectangle{\pgfqpoint{0.694444in}{0.416667in}}{\pgfqpoint{3.194444in}{1.416667in}}%
\pgfusepath{clip}%
\pgfsetbuttcap%
\pgfsetmiterjoin%
\definecolor{currentfill}{rgb}{0.447059,0.447059,0.447059}%
\pgfsetfillcolor{currentfill}%
\pgfsetlinewidth{1.003750pt}%
\definecolor{currentstroke}{rgb}{0.266667,0.266667,0.266667}%
\pgfsetstrokecolor{currentstroke}%
\pgfsetdash{}{0pt}%
\pgfpathmoveto{\pgfqpoint{2.328898in}{1.631526in}}%
\pgfpathlineto{\pgfqpoint{2.502644in}{1.631526in}}%
\pgfpathlineto{\pgfqpoint{2.502644in}{1.631526in}}%
\pgfpathlineto{\pgfqpoint{2.328898in}{1.631526in}}%
\pgfpathlineto{\pgfqpoint{2.328898in}{1.631526in}}%
\pgfpathclose%
\pgfusepath{stroke,fill}%
\end{pgfscope}%
\begin{pgfscope}%
\pgfpathrectangle{\pgfqpoint{0.694444in}{0.416667in}}{\pgfqpoint{3.194444in}{1.416667in}}%
\pgfusepath{clip}%
\pgfsetbuttcap%
\pgfsetmiterjoin%
\definecolor{currentfill}{rgb}{0.447059,0.447059,0.447059}%
\pgfsetfillcolor{currentfill}%
\pgfsetlinewidth{1.003750pt}%
\definecolor{currentstroke}{rgb}{0.266667,0.266667,0.266667}%
\pgfsetstrokecolor{currentstroke}%
\pgfsetdash{}{0pt}%
\pgfpathmoveto{\pgfqpoint{2.577107in}{1.587320in}}%
\pgfpathlineto{\pgfqpoint{2.750853in}{1.587320in}}%
\pgfpathlineto{\pgfqpoint{2.750853in}{1.587320in}}%
\pgfpathlineto{\pgfqpoint{2.577107in}{1.587320in}}%
\pgfpathlineto{\pgfqpoint{2.577107in}{1.587320in}}%
\pgfpathclose%
\pgfusepath{stroke,fill}%
\end{pgfscope}%
\begin{pgfscope}%
\pgfpathrectangle{\pgfqpoint{0.694444in}{0.416667in}}{\pgfqpoint{3.194444in}{1.416667in}}%
\pgfusepath{clip}%
\pgfsetbuttcap%
\pgfsetmiterjoin%
\definecolor{currentfill}{rgb}{0.447059,0.447059,0.447059}%
\pgfsetfillcolor{currentfill}%
\pgfsetlinewidth{1.003750pt}%
\definecolor{currentstroke}{rgb}{0.266667,0.266667,0.266667}%
\pgfsetstrokecolor{currentstroke}%
\pgfsetdash{}{0pt}%
\pgfpathmoveto{\pgfqpoint{2.825315in}{1.610349in}}%
\pgfpathlineto{\pgfqpoint{2.999061in}{1.610349in}}%
\pgfpathlineto{\pgfqpoint{2.999061in}{1.610349in}}%
\pgfpathlineto{\pgfqpoint{2.825315in}{1.610349in}}%
\pgfpathlineto{\pgfqpoint{2.825315in}{1.610349in}}%
\pgfpathclose%
\pgfusepath{stroke,fill}%
\end{pgfscope}%
\begin{pgfscope}%
\pgfpathrectangle{\pgfqpoint{0.694444in}{0.416667in}}{\pgfqpoint{3.194444in}{1.416667in}}%
\pgfusepath{clip}%
\pgfsetbuttcap%
\pgfsetmiterjoin%
\definecolor{currentfill}{rgb}{0.447059,0.447059,0.447059}%
\pgfsetfillcolor{currentfill}%
\pgfsetlinewidth{1.003750pt}%
\definecolor{currentstroke}{rgb}{0.266667,0.266667,0.266667}%
\pgfsetstrokecolor{currentstroke}%
\pgfsetdash{}{0pt}%
\pgfpathmoveto{\pgfqpoint{3.073524in}{1.671027in}}%
\pgfpathlineto{\pgfqpoint{3.247270in}{1.671027in}}%
\pgfpathlineto{\pgfqpoint{3.247270in}{1.671027in}}%
\pgfpathlineto{\pgfqpoint{3.073524in}{1.671027in}}%
\pgfpathlineto{\pgfqpoint{3.073524in}{1.671027in}}%
\pgfpathclose%
\pgfusepath{stroke,fill}%
\end{pgfscope}%
\begin{pgfscope}%
\pgfpathrectangle{\pgfqpoint{0.694444in}{0.416667in}}{\pgfqpoint{3.194444in}{1.416667in}}%
\pgfusepath{clip}%
\pgfsetbuttcap%
\pgfsetmiterjoin%
\definecolor{currentfill}{rgb}{0.447059,0.447059,0.447059}%
\pgfsetfillcolor{currentfill}%
\pgfsetlinewidth{1.003750pt}%
\definecolor{currentstroke}{rgb}{0.266667,0.266667,0.266667}%
\pgfsetstrokecolor{currentstroke}%
\pgfsetdash{}{0pt}%
\pgfpathmoveto{\pgfqpoint{3.321732in}{1.612970in}}%
\pgfpathlineto{\pgfqpoint{3.495478in}{1.612970in}}%
\pgfpathlineto{\pgfqpoint{3.495478in}{1.612970in}}%
\pgfpathlineto{\pgfqpoint{3.321732in}{1.612970in}}%
\pgfpathlineto{\pgfqpoint{3.321732in}{1.612970in}}%
\pgfpathclose%
\pgfusepath{stroke,fill}%
\end{pgfscope}%
\begin{pgfscope}%
\pgfpathrectangle{\pgfqpoint{0.694444in}{0.416667in}}{\pgfqpoint{3.194444in}{1.416667in}}%
\pgfusepath{clip}%
\pgfsetbuttcap%
\pgfsetmiterjoin%
\definecolor{currentfill}{rgb}{0.447059,0.447059,0.447059}%
\pgfsetfillcolor{currentfill}%
\pgfsetlinewidth{1.003750pt}%
\definecolor{currentstroke}{rgb}{0.266667,0.266667,0.266667}%
\pgfsetstrokecolor{currentstroke}%
\pgfsetdash{}{0pt}%
\pgfpathmoveto{\pgfqpoint{3.569941in}{1.647066in}}%
\pgfpathlineto{\pgfqpoint{3.743687in}{1.647066in}}%
\pgfpathlineto{\pgfqpoint{3.743687in}{1.647067in}}%
\pgfpathlineto{\pgfqpoint{3.569941in}{1.647067in}}%
\pgfpathlineto{\pgfqpoint{3.569941in}{1.647066in}}%
\pgfpathclose%
\pgfusepath{stroke,fill}%
\end{pgfscope}%
\begin{pgfscope}%
\pgfpathrectangle{\pgfqpoint{0.694444in}{0.416667in}}{\pgfqpoint{3.194444in}{1.416667in}}%
\pgfusepath{clip}%
\pgfsetbuttcap%
\pgfsetmiterjoin%
\definecolor{currentfill}{rgb}{0.447059,0.447059,0.447059}%
\pgfsetfillcolor{currentfill}%
\pgfsetlinewidth{1.003750pt}%
\definecolor{currentstroke}{rgb}{0.266667,0.266667,0.266667}%
\pgfsetstrokecolor{currentstroke}%
\pgfsetdash{}{0pt}%
\pgfpathmoveto{\pgfqpoint{0.839646in}{1.500470in}}%
\pgfpathlineto{\pgfqpoint{1.013392in}{1.500470in}}%
\pgfpathlineto{\pgfqpoint{1.013392in}{1.500470in}}%
\pgfpathlineto{\pgfqpoint{0.839646in}{1.500470in}}%
\pgfpathlineto{\pgfqpoint{0.839646in}{1.500470in}}%
\pgfpathclose%
\pgfusepath{stroke,fill}%
\end{pgfscope}%
\begin{pgfscope}%
\pgfpathrectangle{\pgfqpoint{0.694444in}{0.416667in}}{\pgfqpoint{3.194444in}{1.416667in}}%
\pgfusepath{clip}%
\pgfsetbuttcap%
\pgfsetmiterjoin%
\definecolor{currentfill}{rgb}{0.447059,0.447059,0.447059}%
\pgfsetfillcolor{currentfill}%
\pgfsetlinewidth{1.003750pt}%
\definecolor{currentstroke}{rgb}{0.266667,0.266667,0.266667}%
\pgfsetstrokecolor{currentstroke}%
\pgfsetdash{}{0pt}%
\pgfpathmoveto{\pgfqpoint{1.087855in}{1.493113in}}%
\pgfpathlineto{\pgfqpoint{1.261601in}{1.493113in}}%
\pgfpathlineto{\pgfqpoint{1.261601in}{1.493113in}}%
\pgfpathlineto{\pgfqpoint{1.087855in}{1.493113in}}%
\pgfpathlineto{\pgfqpoint{1.087855in}{1.493113in}}%
\pgfpathclose%
\pgfusepath{stroke,fill}%
\end{pgfscope}%
\begin{pgfscope}%
\pgfpathrectangle{\pgfqpoint{0.694444in}{0.416667in}}{\pgfqpoint{3.194444in}{1.416667in}}%
\pgfusepath{clip}%
\pgfsetbuttcap%
\pgfsetmiterjoin%
\definecolor{currentfill}{rgb}{0.447059,0.447059,0.447059}%
\pgfsetfillcolor{currentfill}%
\pgfsetlinewidth{1.003750pt}%
\definecolor{currentstroke}{rgb}{0.266667,0.266667,0.266667}%
\pgfsetstrokecolor{currentstroke}%
\pgfsetdash{}{0pt}%
\pgfpathmoveto{\pgfqpoint{1.336064in}{1.471004in}}%
\pgfpathlineto{\pgfqpoint{1.509810in}{1.471004in}}%
\pgfpathlineto{\pgfqpoint{1.509810in}{1.471004in}}%
\pgfpathlineto{\pgfqpoint{1.336064in}{1.471004in}}%
\pgfpathlineto{\pgfqpoint{1.336064in}{1.471004in}}%
\pgfpathclose%
\pgfusepath{stroke,fill}%
\end{pgfscope}%
\begin{pgfscope}%
\pgfpathrectangle{\pgfqpoint{0.694444in}{0.416667in}}{\pgfqpoint{3.194444in}{1.416667in}}%
\pgfusepath{clip}%
\pgfsetbuttcap%
\pgfsetmiterjoin%
\definecolor{currentfill}{rgb}{0.447059,0.447059,0.447059}%
\pgfsetfillcolor{currentfill}%
\pgfsetlinewidth{1.003750pt}%
\definecolor{currentstroke}{rgb}{0.266667,0.266667,0.266667}%
\pgfsetstrokecolor{currentstroke}%
\pgfsetdash{}{0pt}%
\pgfpathmoveto{\pgfqpoint{1.584272in}{1.609389in}}%
\pgfpathlineto{\pgfqpoint{1.758018in}{1.609389in}}%
\pgfpathlineto{\pgfqpoint{1.758018in}{1.609389in}}%
\pgfpathlineto{\pgfqpoint{1.584272in}{1.609389in}}%
\pgfpathlineto{\pgfqpoint{1.584272in}{1.609389in}}%
\pgfpathclose%
\pgfusepath{stroke,fill}%
\end{pgfscope}%
\begin{pgfscope}%
\pgfpathrectangle{\pgfqpoint{0.694444in}{0.416667in}}{\pgfqpoint{3.194444in}{1.416667in}}%
\pgfusepath{clip}%
\pgfsetbuttcap%
\pgfsetmiterjoin%
\definecolor{currentfill}{rgb}{0.447059,0.447059,0.447059}%
\pgfsetfillcolor{currentfill}%
\pgfsetlinewidth{1.003750pt}%
\definecolor{currentstroke}{rgb}{0.266667,0.266667,0.266667}%
\pgfsetstrokecolor{currentstroke}%
\pgfsetdash{}{0pt}%
\pgfpathmoveto{\pgfqpoint{1.832481in}{1.659762in}}%
\pgfpathlineto{\pgfqpoint{2.006227in}{1.659762in}}%
\pgfpathlineto{\pgfqpoint{2.006227in}{1.659762in}}%
\pgfpathlineto{\pgfqpoint{1.832481in}{1.659762in}}%
\pgfpathlineto{\pgfqpoint{1.832481in}{1.659762in}}%
\pgfpathclose%
\pgfusepath{stroke,fill}%
\end{pgfscope}%
\begin{pgfscope}%
\pgfpathrectangle{\pgfqpoint{0.694444in}{0.416667in}}{\pgfqpoint{3.194444in}{1.416667in}}%
\pgfusepath{clip}%
\pgfsetbuttcap%
\pgfsetmiterjoin%
\definecolor{currentfill}{rgb}{0.447059,0.447059,0.447059}%
\pgfsetfillcolor{currentfill}%
\pgfsetlinewidth{1.003750pt}%
\definecolor{currentstroke}{rgb}{0.266667,0.266667,0.266667}%
\pgfsetstrokecolor{currentstroke}%
\pgfsetdash{}{0pt}%
\pgfpathmoveto{\pgfqpoint{2.080689in}{1.669671in}}%
\pgfpathlineto{\pgfqpoint{2.254435in}{1.669671in}}%
\pgfpathlineto{\pgfqpoint{2.254435in}{1.669671in}}%
\pgfpathlineto{\pgfqpoint{2.080689in}{1.669671in}}%
\pgfpathlineto{\pgfqpoint{2.080689in}{1.669671in}}%
\pgfpathclose%
\pgfusepath{stroke,fill}%
\end{pgfscope}%
\begin{pgfscope}%
\pgfpathrectangle{\pgfqpoint{0.694444in}{0.416667in}}{\pgfqpoint{3.194444in}{1.416667in}}%
\pgfusepath{clip}%
\pgfsetbuttcap%
\pgfsetmiterjoin%
\definecolor{currentfill}{rgb}{0.447059,0.447059,0.447059}%
\pgfsetfillcolor{currentfill}%
\pgfsetlinewidth{1.003750pt}%
\definecolor{currentstroke}{rgb}{0.266667,0.266667,0.266667}%
\pgfsetstrokecolor{currentstroke}%
\pgfsetdash{}{0pt}%
\pgfpathmoveto{\pgfqpoint{2.328898in}{1.631526in}}%
\pgfpathlineto{\pgfqpoint{2.502644in}{1.631526in}}%
\pgfpathlineto{\pgfqpoint{2.502644in}{1.631526in}}%
\pgfpathlineto{\pgfqpoint{2.328898in}{1.631526in}}%
\pgfpathlineto{\pgfqpoint{2.328898in}{1.631526in}}%
\pgfpathclose%
\pgfusepath{stroke,fill}%
\end{pgfscope}%
\begin{pgfscope}%
\pgfpathrectangle{\pgfqpoint{0.694444in}{0.416667in}}{\pgfqpoint{3.194444in}{1.416667in}}%
\pgfusepath{clip}%
\pgfsetbuttcap%
\pgfsetmiterjoin%
\definecolor{currentfill}{rgb}{0.447059,0.447059,0.447059}%
\pgfsetfillcolor{currentfill}%
\pgfsetlinewidth{1.003750pt}%
\definecolor{currentstroke}{rgb}{0.266667,0.266667,0.266667}%
\pgfsetstrokecolor{currentstroke}%
\pgfsetdash{}{0pt}%
\pgfpathmoveto{\pgfqpoint{2.577107in}{1.587320in}}%
\pgfpathlineto{\pgfqpoint{2.750853in}{1.587320in}}%
\pgfpathlineto{\pgfqpoint{2.750853in}{1.587321in}}%
\pgfpathlineto{\pgfqpoint{2.577107in}{1.587321in}}%
\pgfpathlineto{\pgfqpoint{2.577107in}{1.587320in}}%
\pgfpathclose%
\pgfusepath{stroke,fill}%
\end{pgfscope}%
\begin{pgfscope}%
\pgfpathrectangle{\pgfqpoint{0.694444in}{0.416667in}}{\pgfqpoint{3.194444in}{1.416667in}}%
\pgfusepath{clip}%
\pgfsetbuttcap%
\pgfsetmiterjoin%
\definecolor{currentfill}{rgb}{0.447059,0.447059,0.447059}%
\pgfsetfillcolor{currentfill}%
\pgfsetlinewidth{1.003750pt}%
\definecolor{currentstroke}{rgb}{0.266667,0.266667,0.266667}%
\pgfsetstrokecolor{currentstroke}%
\pgfsetdash{}{0pt}%
\pgfpathmoveto{\pgfqpoint{2.825315in}{1.610349in}}%
\pgfpathlineto{\pgfqpoint{2.999061in}{1.610349in}}%
\pgfpathlineto{\pgfqpoint{2.999061in}{1.610349in}}%
\pgfpathlineto{\pgfqpoint{2.825315in}{1.610349in}}%
\pgfpathlineto{\pgfqpoint{2.825315in}{1.610349in}}%
\pgfpathclose%
\pgfusepath{stroke,fill}%
\end{pgfscope}%
\begin{pgfscope}%
\pgfpathrectangle{\pgfqpoint{0.694444in}{0.416667in}}{\pgfqpoint{3.194444in}{1.416667in}}%
\pgfusepath{clip}%
\pgfsetbuttcap%
\pgfsetmiterjoin%
\definecolor{currentfill}{rgb}{0.447059,0.447059,0.447059}%
\pgfsetfillcolor{currentfill}%
\pgfsetlinewidth{1.003750pt}%
\definecolor{currentstroke}{rgb}{0.266667,0.266667,0.266667}%
\pgfsetstrokecolor{currentstroke}%
\pgfsetdash{}{0pt}%
\pgfpathmoveto{\pgfqpoint{3.073524in}{1.671027in}}%
\pgfpathlineto{\pgfqpoint{3.247270in}{1.671027in}}%
\pgfpathlineto{\pgfqpoint{3.247270in}{1.671027in}}%
\pgfpathlineto{\pgfqpoint{3.073524in}{1.671027in}}%
\pgfpathlineto{\pgfqpoint{3.073524in}{1.671027in}}%
\pgfpathclose%
\pgfusepath{stroke,fill}%
\end{pgfscope}%
\begin{pgfscope}%
\pgfpathrectangle{\pgfqpoint{0.694444in}{0.416667in}}{\pgfqpoint{3.194444in}{1.416667in}}%
\pgfusepath{clip}%
\pgfsetbuttcap%
\pgfsetmiterjoin%
\definecolor{currentfill}{rgb}{0.447059,0.447059,0.447059}%
\pgfsetfillcolor{currentfill}%
\pgfsetlinewidth{1.003750pt}%
\definecolor{currentstroke}{rgb}{0.266667,0.266667,0.266667}%
\pgfsetstrokecolor{currentstroke}%
\pgfsetdash{}{0pt}%
\pgfpathmoveto{\pgfqpoint{3.321732in}{1.612970in}}%
\pgfpathlineto{\pgfqpoint{3.495478in}{1.612970in}}%
\pgfpathlineto{\pgfqpoint{3.495478in}{1.612970in}}%
\pgfpathlineto{\pgfqpoint{3.321732in}{1.612970in}}%
\pgfpathlineto{\pgfqpoint{3.321732in}{1.612970in}}%
\pgfpathclose%
\pgfusepath{stroke,fill}%
\end{pgfscope}%
\begin{pgfscope}%
\pgfpathrectangle{\pgfqpoint{0.694444in}{0.416667in}}{\pgfqpoint{3.194444in}{1.416667in}}%
\pgfusepath{clip}%
\pgfsetbuttcap%
\pgfsetmiterjoin%
\definecolor{currentfill}{rgb}{0.447059,0.447059,0.447059}%
\pgfsetfillcolor{currentfill}%
\pgfsetlinewidth{1.003750pt}%
\definecolor{currentstroke}{rgb}{0.266667,0.266667,0.266667}%
\pgfsetstrokecolor{currentstroke}%
\pgfsetdash{}{0pt}%
\pgfpathmoveto{\pgfqpoint{3.569941in}{1.647067in}}%
\pgfpathlineto{\pgfqpoint{3.743687in}{1.647067in}}%
\pgfpathlineto{\pgfqpoint{3.743687in}{1.647067in}}%
\pgfpathlineto{\pgfqpoint{3.569941in}{1.647067in}}%
\pgfpathlineto{\pgfqpoint{3.569941in}{1.647067in}}%
\pgfpathclose%
\pgfusepath{stroke,fill}%
\end{pgfscope}%
\begin{pgfscope}%
\pgfpathrectangle{\pgfqpoint{0.694444in}{0.416667in}}{\pgfqpoint{3.194444in}{1.416667in}}%
\pgfusepath{clip}%
\pgfsetbuttcap%
\pgfsetmiterjoin%
\definecolor{currentfill}{rgb}{0.447059,0.447059,0.447059}%
\pgfsetfillcolor{currentfill}%
\pgfsetlinewidth{1.003750pt}%
\definecolor{currentstroke}{rgb}{0.266667,0.266667,0.266667}%
\pgfsetstrokecolor{currentstroke}%
\pgfsetdash{}{0pt}%
\pgfpathmoveto{\pgfqpoint{0.839646in}{1.500470in}}%
\pgfpathlineto{\pgfqpoint{1.013392in}{1.500470in}}%
\pgfpathlineto{\pgfqpoint{1.013392in}{1.500470in}}%
\pgfpathlineto{\pgfqpoint{0.839646in}{1.500470in}}%
\pgfpathlineto{\pgfqpoint{0.839646in}{1.500470in}}%
\pgfpathclose%
\pgfusepath{stroke,fill}%
\end{pgfscope}%
\begin{pgfscope}%
\pgfpathrectangle{\pgfqpoint{0.694444in}{0.416667in}}{\pgfqpoint{3.194444in}{1.416667in}}%
\pgfusepath{clip}%
\pgfsetbuttcap%
\pgfsetmiterjoin%
\definecolor{currentfill}{rgb}{0.447059,0.447059,0.447059}%
\pgfsetfillcolor{currentfill}%
\pgfsetlinewidth{1.003750pt}%
\definecolor{currentstroke}{rgb}{0.266667,0.266667,0.266667}%
\pgfsetstrokecolor{currentstroke}%
\pgfsetdash{}{0pt}%
\pgfpathmoveto{\pgfqpoint{1.087855in}{1.493113in}}%
\pgfpathlineto{\pgfqpoint{1.261601in}{1.493113in}}%
\pgfpathlineto{\pgfqpoint{1.261601in}{1.493113in}}%
\pgfpathlineto{\pgfqpoint{1.087855in}{1.493113in}}%
\pgfpathlineto{\pgfqpoint{1.087855in}{1.493113in}}%
\pgfpathclose%
\pgfusepath{stroke,fill}%
\end{pgfscope}%
\begin{pgfscope}%
\pgfpathrectangle{\pgfqpoint{0.694444in}{0.416667in}}{\pgfqpoint{3.194444in}{1.416667in}}%
\pgfusepath{clip}%
\pgfsetbuttcap%
\pgfsetmiterjoin%
\definecolor{currentfill}{rgb}{0.447059,0.447059,0.447059}%
\pgfsetfillcolor{currentfill}%
\pgfsetlinewidth{1.003750pt}%
\definecolor{currentstroke}{rgb}{0.266667,0.266667,0.266667}%
\pgfsetstrokecolor{currentstroke}%
\pgfsetdash{}{0pt}%
\pgfpathmoveto{\pgfqpoint{1.336064in}{1.471004in}}%
\pgfpathlineto{\pgfqpoint{1.509810in}{1.471004in}}%
\pgfpathlineto{\pgfqpoint{1.509810in}{1.471004in}}%
\pgfpathlineto{\pgfqpoint{1.336064in}{1.471004in}}%
\pgfpathlineto{\pgfqpoint{1.336064in}{1.471004in}}%
\pgfpathclose%
\pgfusepath{stroke,fill}%
\end{pgfscope}%
\begin{pgfscope}%
\pgfpathrectangle{\pgfqpoint{0.694444in}{0.416667in}}{\pgfqpoint{3.194444in}{1.416667in}}%
\pgfusepath{clip}%
\pgfsetbuttcap%
\pgfsetmiterjoin%
\definecolor{currentfill}{rgb}{0.447059,0.447059,0.447059}%
\pgfsetfillcolor{currentfill}%
\pgfsetlinewidth{1.003750pt}%
\definecolor{currentstroke}{rgb}{0.266667,0.266667,0.266667}%
\pgfsetstrokecolor{currentstroke}%
\pgfsetdash{}{0pt}%
\pgfpathmoveto{\pgfqpoint{1.584272in}{1.609389in}}%
\pgfpathlineto{\pgfqpoint{1.758018in}{1.609389in}}%
\pgfpathlineto{\pgfqpoint{1.758018in}{1.609389in}}%
\pgfpathlineto{\pgfqpoint{1.584272in}{1.609389in}}%
\pgfpathlineto{\pgfqpoint{1.584272in}{1.609389in}}%
\pgfpathclose%
\pgfusepath{stroke,fill}%
\end{pgfscope}%
\begin{pgfscope}%
\pgfpathrectangle{\pgfqpoint{0.694444in}{0.416667in}}{\pgfqpoint{3.194444in}{1.416667in}}%
\pgfusepath{clip}%
\pgfsetbuttcap%
\pgfsetmiterjoin%
\definecolor{currentfill}{rgb}{0.447059,0.447059,0.447059}%
\pgfsetfillcolor{currentfill}%
\pgfsetlinewidth{1.003750pt}%
\definecolor{currentstroke}{rgb}{0.266667,0.266667,0.266667}%
\pgfsetstrokecolor{currentstroke}%
\pgfsetdash{}{0pt}%
\pgfpathmoveto{\pgfqpoint{1.832481in}{1.659762in}}%
\pgfpathlineto{\pgfqpoint{2.006227in}{1.659762in}}%
\pgfpathlineto{\pgfqpoint{2.006227in}{1.659762in}}%
\pgfpathlineto{\pgfqpoint{1.832481in}{1.659762in}}%
\pgfpathlineto{\pgfqpoint{1.832481in}{1.659762in}}%
\pgfpathclose%
\pgfusepath{stroke,fill}%
\end{pgfscope}%
\begin{pgfscope}%
\pgfpathrectangle{\pgfqpoint{0.694444in}{0.416667in}}{\pgfqpoint{3.194444in}{1.416667in}}%
\pgfusepath{clip}%
\pgfsetbuttcap%
\pgfsetmiterjoin%
\definecolor{currentfill}{rgb}{0.447059,0.447059,0.447059}%
\pgfsetfillcolor{currentfill}%
\pgfsetlinewidth{1.003750pt}%
\definecolor{currentstroke}{rgb}{0.266667,0.266667,0.266667}%
\pgfsetstrokecolor{currentstroke}%
\pgfsetdash{}{0pt}%
\pgfpathmoveto{\pgfqpoint{2.080689in}{1.669671in}}%
\pgfpathlineto{\pgfqpoint{2.254435in}{1.669671in}}%
\pgfpathlineto{\pgfqpoint{2.254435in}{1.669671in}}%
\pgfpathlineto{\pgfqpoint{2.080689in}{1.669671in}}%
\pgfpathlineto{\pgfqpoint{2.080689in}{1.669671in}}%
\pgfpathclose%
\pgfusepath{stroke,fill}%
\end{pgfscope}%
\begin{pgfscope}%
\pgfpathrectangle{\pgfqpoint{0.694444in}{0.416667in}}{\pgfqpoint{3.194444in}{1.416667in}}%
\pgfusepath{clip}%
\pgfsetbuttcap%
\pgfsetmiterjoin%
\definecolor{currentfill}{rgb}{0.447059,0.447059,0.447059}%
\pgfsetfillcolor{currentfill}%
\pgfsetlinewidth{1.003750pt}%
\definecolor{currentstroke}{rgb}{0.266667,0.266667,0.266667}%
\pgfsetstrokecolor{currentstroke}%
\pgfsetdash{}{0pt}%
\pgfpathmoveto{\pgfqpoint{2.328898in}{1.631526in}}%
\pgfpathlineto{\pgfqpoint{2.502644in}{1.631526in}}%
\pgfpathlineto{\pgfqpoint{2.502644in}{1.631526in}}%
\pgfpathlineto{\pgfqpoint{2.328898in}{1.631526in}}%
\pgfpathlineto{\pgfqpoint{2.328898in}{1.631526in}}%
\pgfpathclose%
\pgfusepath{stroke,fill}%
\end{pgfscope}%
\begin{pgfscope}%
\pgfpathrectangle{\pgfqpoint{0.694444in}{0.416667in}}{\pgfqpoint{3.194444in}{1.416667in}}%
\pgfusepath{clip}%
\pgfsetbuttcap%
\pgfsetmiterjoin%
\definecolor{currentfill}{rgb}{0.447059,0.447059,0.447059}%
\pgfsetfillcolor{currentfill}%
\pgfsetlinewidth{1.003750pt}%
\definecolor{currentstroke}{rgb}{0.266667,0.266667,0.266667}%
\pgfsetstrokecolor{currentstroke}%
\pgfsetdash{}{0pt}%
\pgfpathmoveto{\pgfqpoint{2.577107in}{1.587321in}}%
\pgfpathlineto{\pgfqpoint{2.750853in}{1.587321in}}%
\pgfpathlineto{\pgfqpoint{2.750853in}{1.587321in}}%
\pgfpathlineto{\pgfqpoint{2.577107in}{1.587321in}}%
\pgfpathlineto{\pgfqpoint{2.577107in}{1.587321in}}%
\pgfpathclose%
\pgfusepath{stroke,fill}%
\end{pgfscope}%
\begin{pgfscope}%
\pgfpathrectangle{\pgfqpoint{0.694444in}{0.416667in}}{\pgfqpoint{3.194444in}{1.416667in}}%
\pgfusepath{clip}%
\pgfsetbuttcap%
\pgfsetmiterjoin%
\definecolor{currentfill}{rgb}{0.447059,0.447059,0.447059}%
\pgfsetfillcolor{currentfill}%
\pgfsetlinewidth{1.003750pt}%
\definecolor{currentstroke}{rgb}{0.266667,0.266667,0.266667}%
\pgfsetstrokecolor{currentstroke}%
\pgfsetdash{}{0pt}%
\pgfpathmoveto{\pgfqpoint{2.825315in}{1.610349in}}%
\pgfpathlineto{\pgfqpoint{2.999061in}{1.610349in}}%
\pgfpathlineto{\pgfqpoint{2.999061in}{1.610349in}}%
\pgfpathlineto{\pgfqpoint{2.825315in}{1.610349in}}%
\pgfpathlineto{\pgfqpoint{2.825315in}{1.610349in}}%
\pgfpathclose%
\pgfusepath{stroke,fill}%
\end{pgfscope}%
\begin{pgfscope}%
\pgfpathrectangle{\pgfqpoint{0.694444in}{0.416667in}}{\pgfqpoint{3.194444in}{1.416667in}}%
\pgfusepath{clip}%
\pgfsetbuttcap%
\pgfsetmiterjoin%
\definecolor{currentfill}{rgb}{0.447059,0.447059,0.447059}%
\pgfsetfillcolor{currentfill}%
\pgfsetlinewidth{1.003750pt}%
\definecolor{currentstroke}{rgb}{0.266667,0.266667,0.266667}%
\pgfsetstrokecolor{currentstroke}%
\pgfsetdash{}{0pt}%
\pgfpathmoveto{\pgfqpoint{3.073524in}{1.671027in}}%
\pgfpathlineto{\pgfqpoint{3.247270in}{1.671027in}}%
\pgfpathlineto{\pgfqpoint{3.247270in}{1.671027in}}%
\pgfpathlineto{\pgfqpoint{3.073524in}{1.671027in}}%
\pgfpathlineto{\pgfqpoint{3.073524in}{1.671027in}}%
\pgfpathclose%
\pgfusepath{stroke,fill}%
\end{pgfscope}%
\begin{pgfscope}%
\pgfpathrectangle{\pgfqpoint{0.694444in}{0.416667in}}{\pgfqpoint{3.194444in}{1.416667in}}%
\pgfusepath{clip}%
\pgfsetbuttcap%
\pgfsetmiterjoin%
\definecolor{currentfill}{rgb}{0.447059,0.447059,0.447059}%
\pgfsetfillcolor{currentfill}%
\pgfsetlinewidth{1.003750pt}%
\definecolor{currentstroke}{rgb}{0.266667,0.266667,0.266667}%
\pgfsetstrokecolor{currentstroke}%
\pgfsetdash{}{0pt}%
\pgfpathmoveto{\pgfqpoint{3.321732in}{1.612970in}}%
\pgfpathlineto{\pgfqpoint{3.495478in}{1.612970in}}%
\pgfpathlineto{\pgfqpoint{3.495478in}{1.612970in}}%
\pgfpathlineto{\pgfqpoint{3.321732in}{1.612970in}}%
\pgfpathlineto{\pgfqpoint{3.321732in}{1.612970in}}%
\pgfpathclose%
\pgfusepath{stroke,fill}%
\end{pgfscope}%
\begin{pgfscope}%
\pgfpathrectangle{\pgfqpoint{0.694444in}{0.416667in}}{\pgfqpoint{3.194444in}{1.416667in}}%
\pgfusepath{clip}%
\pgfsetbuttcap%
\pgfsetmiterjoin%
\definecolor{currentfill}{rgb}{0.447059,0.447059,0.447059}%
\pgfsetfillcolor{currentfill}%
\pgfsetlinewidth{1.003750pt}%
\definecolor{currentstroke}{rgb}{0.266667,0.266667,0.266667}%
\pgfsetstrokecolor{currentstroke}%
\pgfsetdash{}{0pt}%
\pgfpathmoveto{\pgfqpoint{3.569941in}{1.647067in}}%
\pgfpathlineto{\pgfqpoint{3.743687in}{1.647067in}}%
\pgfpathlineto{\pgfqpoint{3.743687in}{1.647067in}}%
\pgfpathlineto{\pgfqpoint{3.569941in}{1.647067in}}%
\pgfpathlineto{\pgfqpoint{3.569941in}{1.647067in}}%
\pgfpathclose%
\pgfusepath{stroke,fill}%
\end{pgfscope}%
\begin{pgfscope}%
\pgfpathrectangle{\pgfqpoint{0.694444in}{0.416667in}}{\pgfqpoint{3.194444in}{1.416667in}}%
\pgfusepath{clip}%
\pgfsetbuttcap%
\pgfsetmiterjoin%
\definecolor{currentfill}{rgb}{0.447059,0.447059,0.447059}%
\pgfsetfillcolor{currentfill}%
\pgfsetlinewidth{1.003750pt}%
\definecolor{currentstroke}{rgb}{0.266667,0.266667,0.266667}%
\pgfsetstrokecolor{currentstroke}%
\pgfsetdash{}{0pt}%
\pgfpathmoveto{\pgfqpoint{0.839646in}{1.500470in}}%
\pgfpathlineto{\pgfqpoint{1.013392in}{1.500470in}}%
\pgfpathlineto{\pgfqpoint{1.013392in}{1.500470in}}%
\pgfpathlineto{\pgfqpoint{0.839646in}{1.500470in}}%
\pgfpathlineto{\pgfqpoint{0.839646in}{1.500470in}}%
\pgfpathclose%
\pgfusepath{stroke,fill}%
\end{pgfscope}%
\begin{pgfscope}%
\pgfpathrectangle{\pgfqpoint{0.694444in}{0.416667in}}{\pgfqpoint{3.194444in}{1.416667in}}%
\pgfusepath{clip}%
\pgfsetbuttcap%
\pgfsetmiterjoin%
\definecolor{currentfill}{rgb}{0.447059,0.447059,0.447059}%
\pgfsetfillcolor{currentfill}%
\pgfsetlinewidth{1.003750pt}%
\definecolor{currentstroke}{rgb}{0.266667,0.266667,0.266667}%
\pgfsetstrokecolor{currentstroke}%
\pgfsetdash{}{0pt}%
\pgfpathmoveto{\pgfqpoint{1.087855in}{1.493113in}}%
\pgfpathlineto{\pgfqpoint{1.261601in}{1.493113in}}%
\pgfpathlineto{\pgfqpoint{1.261601in}{1.493113in}}%
\pgfpathlineto{\pgfqpoint{1.087855in}{1.493113in}}%
\pgfpathlineto{\pgfqpoint{1.087855in}{1.493113in}}%
\pgfpathclose%
\pgfusepath{stroke,fill}%
\end{pgfscope}%
\begin{pgfscope}%
\pgfpathrectangle{\pgfqpoint{0.694444in}{0.416667in}}{\pgfqpoint{3.194444in}{1.416667in}}%
\pgfusepath{clip}%
\pgfsetbuttcap%
\pgfsetmiterjoin%
\definecolor{currentfill}{rgb}{0.447059,0.447059,0.447059}%
\pgfsetfillcolor{currentfill}%
\pgfsetlinewidth{1.003750pt}%
\definecolor{currentstroke}{rgb}{0.266667,0.266667,0.266667}%
\pgfsetstrokecolor{currentstroke}%
\pgfsetdash{}{0pt}%
\pgfpathmoveto{\pgfqpoint{1.336064in}{1.471004in}}%
\pgfpathlineto{\pgfqpoint{1.509810in}{1.471004in}}%
\pgfpathlineto{\pgfqpoint{1.509810in}{1.471004in}}%
\pgfpathlineto{\pgfqpoint{1.336064in}{1.471004in}}%
\pgfpathlineto{\pgfqpoint{1.336064in}{1.471004in}}%
\pgfpathclose%
\pgfusepath{stroke,fill}%
\end{pgfscope}%
\begin{pgfscope}%
\pgfpathrectangle{\pgfqpoint{0.694444in}{0.416667in}}{\pgfqpoint{3.194444in}{1.416667in}}%
\pgfusepath{clip}%
\pgfsetbuttcap%
\pgfsetmiterjoin%
\definecolor{currentfill}{rgb}{0.447059,0.447059,0.447059}%
\pgfsetfillcolor{currentfill}%
\pgfsetlinewidth{1.003750pt}%
\definecolor{currentstroke}{rgb}{0.266667,0.266667,0.266667}%
\pgfsetstrokecolor{currentstroke}%
\pgfsetdash{}{0pt}%
\pgfpathmoveto{\pgfqpoint{1.584272in}{1.609389in}}%
\pgfpathlineto{\pgfqpoint{1.758018in}{1.609389in}}%
\pgfpathlineto{\pgfqpoint{1.758018in}{1.609389in}}%
\pgfpathlineto{\pgfqpoint{1.584272in}{1.609389in}}%
\pgfpathlineto{\pgfqpoint{1.584272in}{1.609389in}}%
\pgfpathclose%
\pgfusepath{stroke,fill}%
\end{pgfscope}%
\begin{pgfscope}%
\pgfpathrectangle{\pgfqpoint{0.694444in}{0.416667in}}{\pgfqpoint{3.194444in}{1.416667in}}%
\pgfusepath{clip}%
\pgfsetbuttcap%
\pgfsetmiterjoin%
\definecolor{currentfill}{rgb}{0.447059,0.447059,0.447059}%
\pgfsetfillcolor{currentfill}%
\pgfsetlinewidth{1.003750pt}%
\definecolor{currentstroke}{rgb}{0.266667,0.266667,0.266667}%
\pgfsetstrokecolor{currentstroke}%
\pgfsetdash{}{0pt}%
\pgfpathmoveto{\pgfqpoint{1.832481in}{1.659762in}}%
\pgfpathlineto{\pgfqpoint{2.006227in}{1.659762in}}%
\pgfpathlineto{\pgfqpoint{2.006227in}{1.659762in}}%
\pgfpathlineto{\pgfqpoint{1.832481in}{1.659762in}}%
\pgfpathlineto{\pgfqpoint{1.832481in}{1.659762in}}%
\pgfpathclose%
\pgfusepath{stroke,fill}%
\end{pgfscope}%
\begin{pgfscope}%
\pgfpathrectangle{\pgfqpoint{0.694444in}{0.416667in}}{\pgfqpoint{3.194444in}{1.416667in}}%
\pgfusepath{clip}%
\pgfsetbuttcap%
\pgfsetmiterjoin%
\definecolor{currentfill}{rgb}{0.447059,0.447059,0.447059}%
\pgfsetfillcolor{currentfill}%
\pgfsetlinewidth{1.003750pt}%
\definecolor{currentstroke}{rgb}{0.266667,0.266667,0.266667}%
\pgfsetstrokecolor{currentstroke}%
\pgfsetdash{}{0pt}%
\pgfpathmoveto{\pgfqpoint{2.080689in}{1.669671in}}%
\pgfpathlineto{\pgfqpoint{2.254435in}{1.669671in}}%
\pgfpathlineto{\pgfqpoint{2.254435in}{1.669993in}}%
\pgfpathlineto{\pgfqpoint{2.080689in}{1.669993in}}%
\pgfpathlineto{\pgfqpoint{2.080689in}{1.669671in}}%
\pgfpathclose%
\pgfusepath{stroke,fill}%
\end{pgfscope}%
\begin{pgfscope}%
\pgfpathrectangle{\pgfqpoint{0.694444in}{0.416667in}}{\pgfqpoint{3.194444in}{1.416667in}}%
\pgfusepath{clip}%
\pgfsetbuttcap%
\pgfsetmiterjoin%
\definecolor{currentfill}{rgb}{0.447059,0.447059,0.447059}%
\pgfsetfillcolor{currentfill}%
\pgfsetlinewidth{1.003750pt}%
\definecolor{currentstroke}{rgb}{0.266667,0.266667,0.266667}%
\pgfsetstrokecolor{currentstroke}%
\pgfsetdash{}{0pt}%
\pgfpathmoveto{\pgfqpoint{2.328898in}{1.631526in}}%
\pgfpathlineto{\pgfqpoint{2.502644in}{1.631526in}}%
\pgfpathlineto{\pgfqpoint{2.502644in}{1.631526in}}%
\pgfpathlineto{\pgfqpoint{2.328898in}{1.631526in}}%
\pgfpathlineto{\pgfqpoint{2.328898in}{1.631526in}}%
\pgfpathclose%
\pgfusepath{stroke,fill}%
\end{pgfscope}%
\begin{pgfscope}%
\pgfpathrectangle{\pgfqpoint{0.694444in}{0.416667in}}{\pgfqpoint{3.194444in}{1.416667in}}%
\pgfusepath{clip}%
\pgfsetbuttcap%
\pgfsetmiterjoin%
\definecolor{currentfill}{rgb}{0.447059,0.447059,0.447059}%
\pgfsetfillcolor{currentfill}%
\pgfsetlinewidth{1.003750pt}%
\definecolor{currentstroke}{rgb}{0.266667,0.266667,0.266667}%
\pgfsetstrokecolor{currentstroke}%
\pgfsetdash{}{0pt}%
\pgfpathmoveto{\pgfqpoint{2.577107in}{1.587321in}}%
\pgfpathlineto{\pgfqpoint{2.750853in}{1.587321in}}%
\pgfpathlineto{\pgfqpoint{2.750853in}{1.587585in}}%
\pgfpathlineto{\pgfqpoint{2.577107in}{1.587585in}}%
\pgfpathlineto{\pgfqpoint{2.577107in}{1.587321in}}%
\pgfpathclose%
\pgfusepath{stroke,fill}%
\end{pgfscope}%
\begin{pgfscope}%
\pgfpathrectangle{\pgfqpoint{0.694444in}{0.416667in}}{\pgfqpoint{3.194444in}{1.416667in}}%
\pgfusepath{clip}%
\pgfsetbuttcap%
\pgfsetmiterjoin%
\definecolor{currentfill}{rgb}{0.447059,0.447059,0.447059}%
\pgfsetfillcolor{currentfill}%
\pgfsetlinewidth{1.003750pt}%
\definecolor{currentstroke}{rgb}{0.266667,0.266667,0.266667}%
\pgfsetstrokecolor{currentstroke}%
\pgfsetdash{}{0pt}%
\pgfpathmoveto{\pgfqpoint{2.825315in}{1.610349in}}%
\pgfpathlineto{\pgfqpoint{2.999061in}{1.610349in}}%
\pgfpathlineto{\pgfqpoint{2.999061in}{1.610349in}}%
\pgfpathlineto{\pgfqpoint{2.825315in}{1.610349in}}%
\pgfpathlineto{\pgfqpoint{2.825315in}{1.610349in}}%
\pgfpathclose%
\pgfusepath{stroke,fill}%
\end{pgfscope}%
\begin{pgfscope}%
\pgfpathrectangle{\pgfqpoint{0.694444in}{0.416667in}}{\pgfqpoint{3.194444in}{1.416667in}}%
\pgfusepath{clip}%
\pgfsetbuttcap%
\pgfsetmiterjoin%
\definecolor{currentfill}{rgb}{0.447059,0.447059,0.447059}%
\pgfsetfillcolor{currentfill}%
\pgfsetlinewidth{1.003750pt}%
\definecolor{currentstroke}{rgb}{0.266667,0.266667,0.266667}%
\pgfsetstrokecolor{currentstroke}%
\pgfsetdash{}{0pt}%
\pgfpathmoveto{\pgfqpoint{3.073524in}{1.671027in}}%
\pgfpathlineto{\pgfqpoint{3.247270in}{1.671027in}}%
\pgfpathlineto{\pgfqpoint{3.247270in}{1.671027in}}%
\pgfpathlineto{\pgfqpoint{3.073524in}{1.671027in}}%
\pgfpathlineto{\pgfqpoint{3.073524in}{1.671027in}}%
\pgfpathclose%
\pgfusepath{stroke,fill}%
\end{pgfscope}%
\begin{pgfscope}%
\pgfpathrectangle{\pgfqpoint{0.694444in}{0.416667in}}{\pgfqpoint{3.194444in}{1.416667in}}%
\pgfusepath{clip}%
\pgfsetbuttcap%
\pgfsetmiterjoin%
\definecolor{currentfill}{rgb}{0.447059,0.447059,0.447059}%
\pgfsetfillcolor{currentfill}%
\pgfsetlinewidth{1.003750pt}%
\definecolor{currentstroke}{rgb}{0.266667,0.266667,0.266667}%
\pgfsetstrokecolor{currentstroke}%
\pgfsetdash{}{0pt}%
\pgfpathmoveto{\pgfqpoint{3.321732in}{1.612970in}}%
\pgfpathlineto{\pgfqpoint{3.495478in}{1.612970in}}%
\pgfpathlineto{\pgfqpoint{3.495478in}{1.612970in}}%
\pgfpathlineto{\pgfqpoint{3.321732in}{1.612970in}}%
\pgfpathlineto{\pgfqpoint{3.321732in}{1.612970in}}%
\pgfpathclose%
\pgfusepath{stroke,fill}%
\end{pgfscope}%
\begin{pgfscope}%
\pgfpathrectangle{\pgfqpoint{0.694444in}{0.416667in}}{\pgfqpoint{3.194444in}{1.416667in}}%
\pgfusepath{clip}%
\pgfsetbuttcap%
\pgfsetmiterjoin%
\definecolor{currentfill}{rgb}{0.447059,0.447059,0.447059}%
\pgfsetfillcolor{currentfill}%
\pgfsetlinewidth{1.003750pt}%
\definecolor{currentstroke}{rgb}{0.266667,0.266667,0.266667}%
\pgfsetstrokecolor{currentstroke}%
\pgfsetdash{}{0pt}%
\pgfpathmoveto{\pgfqpoint{3.569941in}{1.647067in}}%
\pgfpathlineto{\pgfqpoint{3.743687in}{1.647067in}}%
\pgfpathlineto{\pgfqpoint{3.743687in}{1.647067in}}%
\pgfpathlineto{\pgfqpoint{3.569941in}{1.647067in}}%
\pgfpathlineto{\pgfqpoint{3.569941in}{1.647067in}}%
\pgfpathclose%
\pgfusepath{stroke,fill}%
\end{pgfscope}%
\begin{pgfscope}%
\pgfpathrectangle{\pgfqpoint{0.694444in}{0.416667in}}{\pgfqpoint{3.194444in}{1.416667in}}%
\pgfusepath{clip}%
\pgfsetbuttcap%
\pgfsetmiterjoin%
\definecolor{currentfill}{rgb}{0.447059,0.447059,0.447059}%
\pgfsetfillcolor{currentfill}%
\pgfsetlinewidth{1.003750pt}%
\definecolor{currentstroke}{rgb}{0.266667,0.266667,0.266667}%
\pgfsetstrokecolor{currentstroke}%
\pgfsetdash{}{0pt}%
\pgfpathmoveto{\pgfqpoint{0.839646in}{1.500470in}}%
\pgfpathlineto{\pgfqpoint{1.013392in}{1.500470in}}%
\pgfpathlineto{\pgfqpoint{1.013392in}{1.500470in}}%
\pgfpathlineto{\pgfqpoint{0.839646in}{1.500470in}}%
\pgfpathlineto{\pgfqpoint{0.839646in}{1.500470in}}%
\pgfpathclose%
\pgfusepath{stroke,fill}%
\end{pgfscope}%
\begin{pgfscope}%
\pgfpathrectangle{\pgfqpoint{0.694444in}{0.416667in}}{\pgfqpoint{3.194444in}{1.416667in}}%
\pgfusepath{clip}%
\pgfsetbuttcap%
\pgfsetmiterjoin%
\definecolor{currentfill}{rgb}{0.447059,0.447059,0.447059}%
\pgfsetfillcolor{currentfill}%
\pgfsetlinewidth{1.003750pt}%
\definecolor{currentstroke}{rgb}{0.266667,0.266667,0.266667}%
\pgfsetstrokecolor{currentstroke}%
\pgfsetdash{}{0pt}%
\pgfpathmoveto{\pgfqpoint{1.087855in}{1.493113in}}%
\pgfpathlineto{\pgfqpoint{1.261601in}{1.493113in}}%
\pgfpathlineto{\pgfqpoint{1.261601in}{1.493113in}}%
\pgfpathlineto{\pgfqpoint{1.087855in}{1.493113in}}%
\pgfpathlineto{\pgfqpoint{1.087855in}{1.493113in}}%
\pgfpathclose%
\pgfusepath{stroke,fill}%
\end{pgfscope}%
\begin{pgfscope}%
\pgfpathrectangle{\pgfqpoint{0.694444in}{0.416667in}}{\pgfqpoint{3.194444in}{1.416667in}}%
\pgfusepath{clip}%
\pgfsetbuttcap%
\pgfsetmiterjoin%
\definecolor{currentfill}{rgb}{0.447059,0.447059,0.447059}%
\pgfsetfillcolor{currentfill}%
\pgfsetlinewidth{1.003750pt}%
\definecolor{currentstroke}{rgb}{0.266667,0.266667,0.266667}%
\pgfsetstrokecolor{currentstroke}%
\pgfsetdash{}{0pt}%
\pgfpathmoveto{\pgfqpoint{1.336064in}{1.471004in}}%
\pgfpathlineto{\pgfqpoint{1.509810in}{1.471004in}}%
\pgfpathlineto{\pgfqpoint{1.509810in}{1.471004in}}%
\pgfpathlineto{\pgfqpoint{1.336064in}{1.471004in}}%
\pgfpathlineto{\pgfqpoint{1.336064in}{1.471004in}}%
\pgfpathclose%
\pgfusepath{stroke,fill}%
\end{pgfscope}%
\begin{pgfscope}%
\pgfpathrectangle{\pgfqpoint{0.694444in}{0.416667in}}{\pgfqpoint{3.194444in}{1.416667in}}%
\pgfusepath{clip}%
\pgfsetbuttcap%
\pgfsetmiterjoin%
\definecolor{currentfill}{rgb}{0.447059,0.447059,0.447059}%
\pgfsetfillcolor{currentfill}%
\pgfsetlinewidth{1.003750pt}%
\definecolor{currentstroke}{rgb}{0.266667,0.266667,0.266667}%
\pgfsetstrokecolor{currentstroke}%
\pgfsetdash{}{0pt}%
\pgfpathmoveto{\pgfqpoint{1.584272in}{1.609389in}}%
\pgfpathlineto{\pgfqpoint{1.758018in}{1.609389in}}%
\pgfpathlineto{\pgfqpoint{1.758018in}{1.609389in}}%
\pgfpathlineto{\pgfqpoint{1.584272in}{1.609389in}}%
\pgfpathlineto{\pgfqpoint{1.584272in}{1.609389in}}%
\pgfpathclose%
\pgfusepath{stroke,fill}%
\end{pgfscope}%
\begin{pgfscope}%
\pgfpathrectangle{\pgfqpoint{0.694444in}{0.416667in}}{\pgfqpoint{3.194444in}{1.416667in}}%
\pgfusepath{clip}%
\pgfsetbuttcap%
\pgfsetmiterjoin%
\definecolor{currentfill}{rgb}{0.447059,0.447059,0.447059}%
\pgfsetfillcolor{currentfill}%
\pgfsetlinewidth{1.003750pt}%
\definecolor{currentstroke}{rgb}{0.266667,0.266667,0.266667}%
\pgfsetstrokecolor{currentstroke}%
\pgfsetdash{}{0pt}%
\pgfpathmoveto{\pgfqpoint{1.832481in}{1.659762in}}%
\pgfpathlineto{\pgfqpoint{2.006227in}{1.659762in}}%
\pgfpathlineto{\pgfqpoint{2.006227in}{1.659762in}}%
\pgfpathlineto{\pgfqpoint{1.832481in}{1.659762in}}%
\pgfpathlineto{\pgfqpoint{1.832481in}{1.659762in}}%
\pgfpathclose%
\pgfusepath{stroke,fill}%
\end{pgfscope}%
\begin{pgfscope}%
\pgfpathrectangle{\pgfqpoint{0.694444in}{0.416667in}}{\pgfqpoint{3.194444in}{1.416667in}}%
\pgfusepath{clip}%
\pgfsetbuttcap%
\pgfsetmiterjoin%
\definecolor{currentfill}{rgb}{0.447059,0.447059,0.447059}%
\pgfsetfillcolor{currentfill}%
\pgfsetlinewidth{1.003750pt}%
\definecolor{currentstroke}{rgb}{0.266667,0.266667,0.266667}%
\pgfsetstrokecolor{currentstroke}%
\pgfsetdash{}{0pt}%
\pgfpathmoveto{\pgfqpoint{2.080689in}{1.669993in}}%
\pgfpathlineto{\pgfqpoint{2.254435in}{1.669993in}}%
\pgfpathlineto{\pgfqpoint{2.254435in}{1.669993in}}%
\pgfpathlineto{\pgfqpoint{2.080689in}{1.669993in}}%
\pgfpathlineto{\pgfqpoint{2.080689in}{1.669993in}}%
\pgfpathclose%
\pgfusepath{stroke,fill}%
\end{pgfscope}%
\begin{pgfscope}%
\pgfpathrectangle{\pgfqpoint{0.694444in}{0.416667in}}{\pgfqpoint{3.194444in}{1.416667in}}%
\pgfusepath{clip}%
\pgfsetbuttcap%
\pgfsetmiterjoin%
\definecolor{currentfill}{rgb}{0.447059,0.447059,0.447059}%
\pgfsetfillcolor{currentfill}%
\pgfsetlinewidth{1.003750pt}%
\definecolor{currentstroke}{rgb}{0.266667,0.266667,0.266667}%
\pgfsetstrokecolor{currentstroke}%
\pgfsetdash{}{0pt}%
\pgfpathmoveto{\pgfqpoint{2.328898in}{1.631526in}}%
\pgfpathlineto{\pgfqpoint{2.502644in}{1.631526in}}%
\pgfpathlineto{\pgfqpoint{2.502644in}{1.631526in}}%
\pgfpathlineto{\pgfqpoint{2.328898in}{1.631526in}}%
\pgfpathlineto{\pgfqpoint{2.328898in}{1.631526in}}%
\pgfpathclose%
\pgfusepath{stroke,fill}%
\end{pgfscope}%
\begin{pgfscope}%
\pgfpathrectangle{\pgfqpoint{0.694444in}{0.416667in}}{\pgfqpoint{3.194444in}{1.416667in}}%
\pgfusepath{clip}%
\pgfsetbuttcap%
\pgfsetmiterjoin%
\definecolor{currentfill}{rgb}{0.447059,0.447059,0.447059}%
\pgfsetfillcolor{currentfill}%
\pgfsetlinewidth{1.003750pt}%
\definecolor{currentstroke}{rgb}{0.266667,0.266667,0.266667}%
\pgfsetstrokecolor{currentstroke}%
\pgfsetdash{}{0pt}%
\pgfpathmoveto{\pgfqpoint{2.577107in}{1.587585in}}%
\pgfpathlineto{\pgfqpoint{2.750853in}{1.587585in}}%
\pgfpathlineto{\pgfqpoint{2.750853in}{1.587585in}}%
\pgfpathlineto{\pgfqpoint{2.577107in}{1.587585in}}%
\pgfpathlineto{\pgfqpoint{2.577107in}{1.587585in}}%
\pgfpathclose%
\pgfusepath{stroke,fill}%
\end{pgfscope}%
\begin{pgfscope}%
\pgfpathrectangle{\pgfqpoint{0.694444in}{0.416667in}}{\pgfqpoint{3.194444in}{1.416667in}}%
\pgfusepath{clip}%
\pgfsetbuttcap%
\pgfsetmiterjoin%
\definecolor{currentfill}{rgb}{0.447059,0.447059,0.447059}%
\pgfsetfillcolor{currentfill}%
\pgfsetlinewidth{1.003750pt}%
\definecolor{currentstroke}{rgb}{0.266667,0.266667,0.266667}%
\pgfsetstrokecolor{currentstroke}%
\pgfsetdash{}{0pt}%
\pgfpathmoveto{\pgfqpoint{2.825315in}{1.610349in}}%
\pgfpathlineto{\pgfqpoint{2.999061in}{1.610349in}}%
\pgfpathlineto{\pgfqpoint{2.999061in}{1.610349in}}%
\pgfpathlineto{\pgfqpoint{2.825315in}{1.610349in}}%
\pgfpathlineto{\pgfqpoint{2.825315in}{1.610349in}}%
\pgfpathclose%
\pgfusepath{stroke,fill}%
\end{pgfscope}%
\begin{pgfscope}%
\pgfpathrectangle{\pgfqpoint{0.694444in}{0.416667in}}{\pgfqpoint{3.194444in}{1.416667in}}%
\pgfusepath{clip}%
\pgfsetbuttcap%
\pgfsetmiterjoin%
\definecolor{currentfill}{rgb}{0.447059,0.447059,0.447059}%
\pgfsetfillcolor{currentfill}%
\pgfsetlinewidth{1.003750pt}%
\definecolor{currentstroke}{rgb}{0.266667,0.266667,0.266667}%
\pgfsetstrokecolor{currentstroke}%
\pgfsetdash{}{0pt}%
\pgfpathmoveto{\pgfqpoint{3.073524in}{1.671027in}}%
\pgfpathlineto{\pgfqpoint{3.247270in}{1.671027in}}%
\pgfpathlineto{\pgfqpoint{3.247270in}{1.671027in}}%
\pgfpathlineto{\pgfqpoint{3.073524in}{1.671027in}}%
\pgfpathlineto{\pgfqpoint{3.073524in}{1.671027in}}%
\pgfpathclose%
\pgfusepath{stroke,fill}%
\end{pgfscope}%
\begin{pgfscope}%
\pgfpathrectangle{\pgfqpoint{0.694444in}{0.416667in}}{\pgfqpoint{3.194444in}{1.416667in}}%
\pgfusepath{clip}%
\pgfsetbuttcap%
\pgfsetmiterjoin%
\definecolor{currentfill}{rgb}{0.447059,0.447059,0.447059}%
\pgfsetfillcolor{currentfill}%
\pgfsetlinewidth{1.003750pt}%
\definecolor{currentstroke}{rgb}{0.266667,0.266667,0.266667}%
\pgfsetstrokecolor{currentstroke}%
\pgfsetdash{}{0pt}%
\pgfpathmoveto{\pgfqpoint{3.321732in}{1.612970in}}%
\pgfpathlineto{\pgfqpoint{3.495478in}{1.612970in}}%
\pgfpathlineto{\pgfqpoint{3.495478in}{1.612970in}}%
\pgfpathlineto{\pgfqpoint{3.321732in}{1.612970in}}%
\pgfpathlineto{\pgfqpoint{3.321732in}{1.612970in}}%
\pgfpathclose%
\pgfusepath{stroke,fill}%
\end{pgfscope}%
\begin{pgfscope}%
\pgfpathrectangle{\pgfqpoint{0.694444in}{0.416667in}}{\pgfqpoint{3.194444in}{1.416667in}}%
\pgfusepath{clip}%
\pgfsetbuttcap%
\pgfsetmiterjoin%
\definecolor{currentfill}{rgb}{0.447059,0.447059,0.447059}%
\pgfsetfillcolor{currentfill}%
\pgfsetlinewidth{1.003750pt}%
\definecolor{currentstroke}{rgb}{0.266667,0.266667,0.266667}%
\pgfsetstrokecolor{currentstroke}%
\pgfsetdash{}{0pt}%
\pgfpathmoveto{\pgfqpoint{3.569941in}{1.647067in}}%
\pgfpathlineto{\pgfqpoint{3.743687in}{1.647067in}}%
\pgfpathlineto{\pgfqpoint{3.743687in}{1.647067in}}%
\pgfpathlineto{\pgfqpoint{3.569941in}{1.647067in}}%
\pgfpathlineto{\pgfqpoint{3.569941in}{1.647067in}}%
\pgfpathclose%
\pgfusepath{stroke,fill}%
\end{pgfscope}%
\begin{pgfscope}%
\pgfpathrectangle{\pgfqpoint{0.694444in}{0.416667in}}{\pgfqpoint{3.194444in}{1.416667in}}%
\pgfusepath{clip}%
\pgfsetbuttcap%
\pgfsetmiterjoin%
\definecolor{currentfill}{rgb}{0.447059,0.447059,0.447059}%
\pgfsetfillcolor{currentfill}%
\pgfsetlinewidth{1.003750pt}%
\definecolor{currentstroke}{rgb}{0.266667,0.266667,0.266667}%
\pgfsetstrokecolor{currentstroke}%
\pgfsetdash{}{0pt}%
\pgfpathmoveto{\pgfqpoint{0.839646in}{1.500470in}}%
\pgfpathlineto{\pgfqpoint{1.013392in}{1.500470in}}%
\pgfpathlineto{\pgfqpoint{1.013392in}{1.500470in}}%
\pgfpathlineto{\pgfqpoint{0.839646in}{1.500470in}}%
\pgfpathlineto{\pgfqpoint{0.839646in}{1.500470in}}%
\pgfpathclose%
\pgfusepath{stroke,fill}%
\end{pgfscope}%
\begin{pgfscope}%
\pgfpathrectangle{\pgfqpoint{0.694444in}{0.416667in}}{\pgfqpoint{3.194444in}{1.416667in}}%
\pgfusepath{clip}%
\pgfsetbuttcap%
\pgfsetmiterjoin%
\definecolor{currentfill}{rgb}{0.447059,0.447059,0.447059}%
\pgfsetfillcolor{currentfill}%
\pgfsetlinewidth{1.003750pt}%
\definecolor{currentstroke}{rgb}{0.266667,0.266667,0.266667}%
\pgfsetstrokecolor{currentstroke}%
\pgfsetdash{}{0pt}%
\pgfpathmoveto{\pgfqpoint{1.087855in}{1.493113in}}%
\pgfpathlineto{\pgfqpoint{1.261601in}{1.493113in}}%
\pgfpathlineto{\pgfqpoint{1.261601in}{1.493113in}}%
\pgfpathlineto{\pgfqpoint{1.087855in}{1.493113in}}%
\pgfpathlineto{\pgfqpoint{1.087855in}{1.493113in}}%
\pgfpathclose%
\pgfusepath{stroke,fill}%
\end{pgfscope}%
\begin{pgfscope}%
\pgfpathrectangle{\pgfqpoint{0.694444in}{0.416667in}}{\pgfqpoint{3.194444in}{1.416667in}}%
\pgfusepath{clip}%
\pgfsetbuttcap%
\pgfsetmiterjoin%
\definecolor{currentfill}{rgb}{0.447059,0.447059,0.447059}%
\pgfsetfillcolor{currentfill}%
\pgfsetlinewidth{1.003750pt}%
\definecolor{currentstroke}{rgb}{0.266667,0.266667,0.266667}%
\pgfsetstrokecolor{currentstroke}%
\pgfsetdash{}{0pt}%
\pgfpathmoveto{\pgfqpoint{1.336064in}{1.471004in}}%
\pgfpathlineto{\pgfqpoint{1.509810in}{1.471004in}}%
\pgfpathlineto{\pgfqpoint{1.509810in}{1.471125in}}%
\pgfpathlineto{\pgfqpoint{1.336064in}{1.471125in}}%
\pgfpathlineto{\pgfqpoint{1.336064in}{1.471004in}}%
\pgfpathclose%
\pgfusepath{stroke,fill}%
\end{pgfscope}%
\begin{pgfscope}%
\pgfpathrectangle{\pgfqpoint{0.694444in}{0.416667in}}{\pgfqpoint{3.194444in}{1.416667in}}%
\pgfusepath{clip}%
\pgfsetbuttcap%
\pgfsetmiterjoin%
\definecolor{currentfill}{rgb}{0.447059,0.447059,0.447059}%
\pgfsetfillcolor{currentfill}%
\pgfsetlinewidth{1.003750pt}%
\definecolor{currentstroke}{rgb}{0.266667,0.266667,0.266667}%
\pgfsetstrokecolor{currentstroke}%
\pgfsetdash{}{0pt}%
\pgfpathmoveto{\pgfqpoint{1.584272in}{1.609389in}}%
\pgfpathlineto{\pgfqpoint{1.758018in}{1.609389in}}%
\pgfpathlineto{\pgfqpoint{1.758018in}{1.609389in}}%
\pgfpathlineto{\pgfqpoint{1.584272in}{1.609389in}}%
\pgfpathlineto{\pgfqpoint{1.584272in}{1.609389in}}%
\pgfpathclose%
\pgfusepath{stroke,fill}%
\end{pgfscope}%
\begin{pgfscope}%
\pgfpathrectangle{\pgfqpoint{0.694444in}{0.416667in}}{\pgfqpoint{3.194444in}{1.416667in}}%
\pgfusepath{clip}%
\pgfsetbuttcap%
\pgfsetmiterjoin%
\definecolor{currentfill}{rgb}{0.447059,0.447059,0.447059}%
\pgfsetfillcolor{currentfill}%
\pgfsetlinewidth{1.003750pt}%
\definecolor{currentstroke}{rgb}{0.266667,0.266667,0.266667}%
\pgfsetstrokecolor{currentstroke}%
\pgfsetdash{}{0pt}%
\pgfpathmoveto{\pgfqpoint{1.832481in}{1.659762in}}%
\pgfpathlineto{\pgfqpoint{2.006227in}{1.659762in}}%
\pgfpathlineto{\pgfqpoint{2.006227in}{1.659762in}}%
\pgfpathlineto{\pgfqpoint{1.832481in}{1.659762in}}%
\pgfpathlineto{\pgfqpoint{1.832481in}{1.659762in}}%
\pgfpathclose%
\pgfusepath{stroke,fill}%
\end{pgfscope}%
\begin{pgfscope}%
\pgfpathrectangle{\pgfqpoint{0.694444in}{0.416667in}}{\pgfqpoint{3.194444in}{1.416667in}}%
\pgfusepath{clip}%
\pgfsetbuttcap%
\pgfsetmiterjoin%
\definecolor{currentfill}{rgb}{0.447059,0.447059,0.447059}%
\pgfsetfillcolor{currentfill}%
\pgfsetlinewidth{1.003750pt}%
\definecolor{currentstroke}{rgb}{0.266667,0.266667,0.266667}%
\pgfsetstrokecolor{currentstroke}%
\pgfsetdash{}{0pt}%
\pgfpathmoveto{\pgfqpoint{2.080689in}{1.669993in}}%
\pgfpathlineto{\pgfqpoint{2.254435in}{1.669993in}}%
\pgfpathlineto{\pgfqpoint{2.254435in}{1.670074in}}%
\pgfpathlineto{\pgfqpoint{2.080689in}{1.670074in}}%
\pgfpathlineto{\pgfqpoint{2.080689in}{1.669993in}}%
\pgfpathclose%
\pgfusepath{stroke,fill}%
\end{pgfscope}%
\begin{pgfscope}%
\pgfpathrectangle{\pgfqpoint{0.694444in}{0.416667in}}{\pgfqpoint{3.194444in}{1.416667in}}%
\pgfusepath{clip}%
\pgfsetbuttcap%
\pgfsetmiterjoin%
\definecolor{currentfill}{rgb}{0.447059,0.447059,0.447059}%
\pgfsetfillcolor{currentfill}%
\pgfsetlinewidth{1.003750pt}%
\definecolor{currentstroke}{rgb}{0.266667,0.266667,0.266667}%
\pgfsetstrokecolor{currentstroke}%
\pgfsetdash{}{0pt}%
\pgfpathmoveto{\pgfqpoint{2.328898in}{1.631526in}}%
\pgfpathlineto{\pgfqpoint{2.502644in}{1.631526in}}%
\pgfpathlineto{\pgfqpoint{2.502644in}{1.631526in}}%
\pgfpathlineto{\pgfqpoint{2.328898in}{1.631526in}}%
\pgfpathlineto{\pgfqpoint{2.328898in}{1.631526in}}%
\pgfpathclose%
\pgfusepath{stroke,fill}%
\end{pgfscope}%
\begin{pgfscope}%
\pgfpathrectangle{\pgfqpoint{0.694444in}{0.416667in}}{\pgfqpoint{3.194444in}{1.416667in}}%
\pgfusepath{clip}%
\pgfsetbuttcap%
\pgfsetmiterjoin%
\definecolor{currentfill}{rgb}{0.447059,0.447059,0.447059}%
\pgfsetfillcolor{currentfill}%
\pgfsetlinewidth{1.003750pt}%
\definecolor{currentstroke}{rgb}{0.266667,0.266667,0.266667}%
\pgfsetstrokecolor{currentstroke}%
\pgfsetdash{}{0pt}%
\pgfpathmoveto{\pgfqpoint{2.577107in}{1.587585in}}%
\pgfpathlineto{\pgfqpoint{2.750853in}{1.587585in}}%
\pgfpathlineto{\pgfqpoint{2.750853in}{1.587585in}}%
\pgfpathlineto{\pgfqpoint{2.577107in}{1.587585in}}%
\pgfpathlineto{\pgfqpoint{2.577107in}{1.587585in}}%
\pgfpathclose%
\pgfusepath{stroke,fill}%
\end{pgfscope}%
\begin{pgfscope}%
\pgfpathrectangle{\pgfqpoint{0.694444in}{0.416667in}}{\pgfqpoint{3.194444in}{1.416667in}}%
\pgfusepath{clip}%
\pgfsetbuttcap%
\pgfsetmiterjoin%
\definecolor{currentfill}{rgb}{0.447059,0.447059,0.447059}%
\pgfsetfillcolor{currentfill}%
\pgfsetlinewidth{1.003750pt}%
\definecolor{currentstroke}{rgb}{0.266667,0.266667,0.266667}%
\pgfsetstrokecolor{currentstroke}%
\pgfsetdash{}{0pt}%
\pgfpathmoveto{\pgfqpoint{2.825315in}{1.610349in}}%
\pgfpathlineto{\pgfqpoint{2.999061in}{1.610349in}}%
\pgfpathlineto{\pgfqpoint{2.999061in}{1.610349in}}%
\pgfpathlineto{\pgfqpoint{2.825315in}{1.610349in}}%
\pgfpathlineto{\pgfqpoint{2.825315in}{1.610349in}}%
\pgfpathclose%
\pgfusepath{stroke,fill}%
\end{pgfscope}%
\begin{pgfscope}%
\pgfpathrectangle{\pgfqpoint{0.694444in}{0.416667in}}{\pgfqpoint{3.194444in}{1.416667in}}%
\pgfusepath{clip}%
\pgfsetbuttcap%
\pgfsetmiterjoin%
\definecolor{currentfill}{rgb}{0.447059,0.447059,0.447059}%
\pgfsetfillcolor{currentfill}%
\pgfsetlinewidth{1.003750pt}%
\definecolor{currentstroke}{rgb}{0.266667,0.266667,0.266667}%
\pgfsetstrokecolor{currentstroke}%
\pgfsetdash{}{0pt}%
\pgfpathmoveto{\pgfqpoint{3.073524in}{1.671027in}}%
\pgfpathlineto{\pgfqpoint{3.247270in}{1.671027in}}%
\pgfpathlineto{\pgfqpoint{3.247270in}{1.671027in}}%
\pgfpathlineto{\pgfqpoint{3.073524in}{1.671027in}}%
\pgfpathlineto{\pgfqpoint{3.073524in}{1.671027in}}%
\pgfpathclose%
\pgfusepath{stroke,fill}%
\end{pgfscope}%
\begin{pgfscope}%
\pgfpathrectangle{\pgfqpoint{0.694444in}{0.416667in}}{\pgfqpoint{3.194444in}{1.416667in}}%
\pgfusepath{clip}%
\pgfsetbuttcap%
\pgfsetmiterjoin%
\definecolor{currentfill}{rgb}{0.447059,0.447059,0.447059}%
\pgfsetfillcolor{currentfill}%
\pgfsetlinewidth{1.003750pt}%
\definecolor{currentstroke}{rgb}{0.266667,0.266667,0.266667}%
\pgfsetstrokecolor{currentstroke}%
\pgfsetdash{}{0pt}%
\pgfpathmoveto{\pgfqpoint{3.321732in}{1.612970in}}%
\pgfpathlineto{\pgfqpoint{3.495478in}{1.612970in}}%
\pgfpathlineto{\pgfqpoint{3.495478in}{1.612970in}}%
\pgfpathlineto{\pgfqpoint{3.321732in}{1.612970in}}%
\pgfpathlineto{\pgfqpoint{3.321732in}{1.612970in}}%
\pgfpathclose%
\pgfusepath{stroke,fill}%
\end{pgfscope}%
\begin{pgfscope}%
\pgfpathrectangle{\pgfqpoint{0.694444in}{0.416667in}}{\pgfqpoint{3.194444in}{1.416667in}}%
\pgfusepath{clip}%
\pgfsetbuttcap%
\pgfsetmiterjoin%
\definecolor{currentfill}{rgb}{0.447059,0.447059,0.447059}%
\pgfsetfillcolor{currentfill}%
\pgfsetlinewidth{1.003750pt}%
\definecolor{currentstroke}{rgb}{0.266667,0.266667,0.266667}%
\pgfsetstrokecolor{currentstroke}%
\pgfsetdash{}{0pt}%
\pgfpathmoveto{\pgfqpoint{3.569941in}{1.647067in}}%
\pgfpathlineto{\pgfqpoint{3.743687in}{1.647067in}}%
\pgfpathlineto{\pgfqpoint{3.743687in}{1.647067in}}%
\pgfpathlineto{\pgfqpoint{3.569941in}{1.647067in}}%
\pgfpathlineto{\pgfqpoint{3.569941in}{1.647067in}}%
\pgfpathclose%
\pgfusepath{stroke,fill}%
\end{pgfscope}%
\begin{pgfscope}%
\pgfpathrectangle{\pgfqpoint{0.694444in}{0.416667in}}{\pgfqpoint{3.194444in}{1.416667in}}%
\pgfusepath{clip}%
\pgfsetbuttcap%
\pgfsetmiterjoin%
\definecolor{currentfill}{rgb}{0.447059,0.447059,0.447059}%
\pgfsetfillcolor{currentfill}%
\pgfsetlinewidth{1.003750pt}%
\definecolor{currentstroke}{rgb}{0.266667,0.266667,0.266667}%
\pgfsetstrokecolor{currentstroke}%
\pgfsetdash{}{0pt}%
\pgfpathmoveto{\pgfqpoint{0.839646in}{1.500470in}}%
\pgfpathlineto{\pgfqpoint{1.013392in}{1.500470in}}%
\pgfpathlineto{\pgfqpoint{1.013392in}{1.500470in}}%
\pgfpathlineto{\pgfqpoint{0.839646in}{1.500470in}}%
\pgfpathlineto{\pgfqpoint{0.839646in}{1.500470in}}%
\pgfpathclose%
\pgfusepath{stroke,fill}%
\end{pgfscope}%
\begin{pgfscope}%
\pgfpathrectangle{\pgfqpoint{0.694444in}{0.416667in}}{\pgfqpoint{3.194444in}{1.416667in}}%
\pgfusepath{clip}%
\pgfsetbuttcap%
\pgfsetmiterjoin%
\definecolor{currentfill}{rgb}{0.447059,0.447059,0.447059}%
\pgfsetfillcolor{currentfill}%
\pgfsetlinewidth{1.003750pt}%
\definecolor{currentstroke}{rgb}{0.266667,0.266667,0.266667}%
\pgfsetstrokecolor{currentstroke}%
\pgfsetdash{}{0pt}%
\pgfpathmoveto{\pgfqpoint{1.087855in}{1.493113in}}%
\pgfpathlineto{\pgfqpoint{1.261601in}{1.493113in}}%
\pgfpathlineto{\pgfqpoint{1.261601in}{1.493113in}}%
\pgfpathlineto{\pgfqpoint{1.087855in}{1.493113in}}%
\pgfpathlineto{\pgfqpoint{1.087855in}{1.493113in}}%
\pgfpathclose%
\pgfusepath{stroke,fill}%
\end{pgfscope}%
\begin{pgfscope}%
\pgfpathrectangle{\pgfqpoint{0.694444in}{0.416667in}}{\pgfqpoint{3.194444in}{1.416667in}}%
\pgfusepath{clip}%
\pgfsetbuttcap%
\pgfsetmiterjoin%
\definecolor{currentfill}{rgb}{0.447059,0.447059,0.447059}%
\pgfsetfillcolor{currentfill}%
\pgfsetlinewidth{1.003750pt}%
\definecolor{currentstroke}{rgb}{0.266667,0.266667,0.266667}%
\pgfsetstrokecolor{currentstroke}%
\pgfsetdash{}{0pt}%
\pgfpathmoveto{\pgfqpoint{1.336064in}{1.471125in}}%
\pgfpathlineto{\pgfqpoint{1.509810in}{1.471125in}}%
\pgfpathlineto{\pgfqpoint{1.509810in}{1.471125in}}%
\pgfpathlineto{\pgfqpoint{1.336064in}{1.471125in}}%
\pgfpathlineto{\pgfqpoint{1.336064in}{1.471125in}}%
\pgfpathclose%
\pgfusepath{stroke,fill}%
\end{pgfscope}%
\begin{pgfscope}%
\pgfpathrectangle{\pgfqpoint{0.694444in}{0.416667in}}{\pgfqpoint{3.194444in}{1.416667in}}%
\pgfusepath{clip}%
\pgfsetbuttcap%
\pgfsetmiterjoin%
\definecolor{currentfill}{rgb}{0.447059,0.447059,0.447059}%
\pgfsetfillcolor{currentfill}%
\pgfsetlinewidth{1.003750pt}%
\definecolor{currentstroke}{rgb}{0.266667,0.266667,0.266667}%
\pgfsetstrokecolor{currentstroke}%
\pgfsetdash{}{0pt}%
\pgfpathmoveto{\pgfqpoint{1.584272in}{1.609389in}}%
\pgfpathlineto{\pgfqpoint{1.758018in}{1.609389in}}%
\pgfpathlineto{\pgfqpoint{1.758018in}{1.609389in}}%
\pgfpathlineto{\pgfqpoint{1.584272in}{1.609389in}}%
\pgfpathlineto{\pgfqpoint{1.584272in}{1.609389in}}%
\pgfpathclose%
\pgfusepath{stroke,fill}%
\end{pgfscope}%
\begin{pgfscope}%
\pgfpathrectangle{\pgfqpoint{0.694444in}{0.416667in}}{\pgfqpoint{3.194444in}{1.416667in}}%
\pgfusepath{clip}%
\pgfsetbuttcap%
\pgfsetmiterjoin%
\definecolor{currentfill}{rgb}{0.447059,0.447059,0.447059}%
\pgfsetfillcolor{currentfill}%
\pgfsetlinewidth{1.003750pt}%
\definecolor{currentstroke}{rgb}{0.266667,0.266667,0.266667}%
\pgfsetstrokecolor{currentstroke}%
\pgfsetdash{}{0pt}%
\pgfpathmoveto{\pgfqpoint{1.832481in}{1.659762in}}%
\pgfpathlineto{\pgfqpoint{2.006227in}{1.659762in}}%
\pgfpathlineto{\pgfqpoint{2.006227in}{1.659762in}}%
\pgfpathlineto{\pgfqpoint{1.832481in}{1.659762in}}%
\pgfpathlineto{\pgfqpoint{1.832481in}{1.659762in}}%
\pgfpathclose%
\pgfusepath{stroke,fill}%
\end{pgfscope}%
\begin{pgfscope}%
\pgfpathrectangle{\pgfqpoint{0.694444in}{0.416667in}}{\pgfqpoint{3.194444in}{1.416667in}}%
\pgfusepath{clip}%
\pgfsetbuttcap%
\pgfsetmiterjoin%
\definecolor{currentfill}{rgb}{0.447059,0.447059,0.447059}%
\pgfsetfillcolor{currentfill}%
\pgfsetlinewidth{1.003750pt}%
\definecolor{currentstroke}{rgb}{0.266667,0.266667,0.266667}%
\pgfsetstrokecolor{currentstroke}%
\pgfsetdash{}{0pt}%
\pgfpathmoveto{\pgfqpoint{2.080689in}{1.670074in}}%
\pgfpathlineto{\pgfqpoint{2.254435in}{1.670074in}}%
\pgfpathlineto{\pgfqpoint{2.254435in}{1.670074in}}%
\pgfpathlineto{\pgfqpoint{2.080689in}{1.670074in}}%
\pgfpathlineto{\pgfqpoint{2.080689in}{1.670074in}}%
\pgfpathclose%
\pgfusepath{stroke,fill}%
\end{pgfscope}%
\begin{pgfscope}%
\pgfpathrectangle{\pgfqpoint{0.694444in}{0.416667in}}{\pgfqpoint{3.194444in}{1.416667in}}%
\pgfusepath{clip}%
\pgfsetbuttcap%
\pgfsetmiterjoin%
\definecolor{currentfill}{rgb}{0.447059,0.447059,0.447059}%
\pgfsetfillcolor{currentfill}%
\pgfsetlinewidth{1.003750pt}%
\definecolor{currentstroke}{rgb}{0.266667,0.266667,0.266667}%
\pgfsetstrokecolor{currentstroke}%
\pgfsetdash{}{0pt}%
\pgfpathmoveto{\pgfqpoint{2.328898in}{1.631526in}}%
\pgfpathlineto{\pgfqpoint{2.502644in}{1.631526in}}%
\pgfpathlineto{\pgfqpoint{2.502644in}{1.631526in}}%
\pgfpathlineto{\pgfqpoint{2.328898in}{1.631526in}}%
\pgfpathlineto{\pgfqpoint{2.328898in}{1.631526in}}%
\pgfpathclose%
\pgfusepath{stroke,fill}%
\end{pgfscope}%
\begin{pgfscope}%
\pgfpathrectangle{\pgfqpoint{0.694444in}{0.416667in}}{\pgfqpoint{3.194444in}{1.416667in}}%
\pgfusepath{clip}%
\pgfsetbuttcap%
\pgfsetmiterjoin%
\definecolor{currentfill}{rgb}{0.447059,0.447059,0.447059}%
\pgfsetfillcolor{currentfill}%
\pgfsetlinewidth{1.003750pt}%
\definecolor{currentstroke}{rgb}{0.266667,0.266667,0.266667}%
\pgfsetstrokecolor{currentstroke}%
\pgfsetdash{}{0pt}%
\pgfpathmoveto{\pgfqpoint{2.577107in}{1.587585in}}%
\pgfpathlineto{\pgfqpoint{2.750853in}{1.587585in}}%
\pgfpathlineto{\pgfqpoint{2.750853in}{1.587585in}}%
\pgfpathlineto{\pgfqpoint{2.577107in}{1.587585in}}%
\pgfpathlineto{\pgfqpoint{2.577107in}{1.587585in}}%
\pgfpathclose%
\pgfusepath{stroke,fill}%
\end{pgfscope}%
\begin{pgfscope}%
\pgfpathrectangle{\pgfqpoint{0.694444in}{0.416667in}}{\pgfqpoint{3.194444in}{1.416667in}}%
\pgfusepath{clip}%
\pgfsetbuttcap%
\pgfsetmiterjoin%
\definecolor{currentfill}{rgb}{0.447059,0.447059,0.447059}%
\pgfsetfillcolor{currentfill}%
\pgfsetlinewidth{1.003750pt}%
\definecolor{currentstroke}{rgb}{0.266667,0.266667,0.266667}%
\pgfsetstrokecolor{currentstroke}%
\pgfsetdash{}{0pt}%
\pgfpathmoveto{\pgfqpoint{2.825315in}{1.610349in}}%
\pgfpathlineto{\pgfqpoint{2.999061in}{1.610349in}}%
\pgfpathlineto{\pgfqpoint{2.999061in}{1.610349in}}%
\pgfpathlineto{\pgfqpoint{2.825315in}{1.610349in}}%
\pgfpathlineto{\pgfqpoint{2.825315in}{1.610349in}}%
\pgfpathclose%
\pgfusepath{stroke,fill}%
\end{pgfscope}%
\begin{pgfscope}%
\pgfpathrectangle{\pgfqpoint{0.694444in}{0.416667in}}{\pgfqpoint{3.194444in}{1.416667in}}%
\pgfusepath{clip}%
\pgfsetbuttcap%
\pgfsetmiterjoin%
\definecolor{currentfill}{rgb}{0.447059,0.447059,0.447059}%
\pgfsetfillcolor{currentfill}%
\pgfsetlinewidth{1.003750pt}%
\definecolor{currentstroke}{rgb}{0.266667,0.266667,0.266667}%
\pgfsetstrokecolor{currentstroke}%
\pgfsetdash{}{0pt}%
\pgfpathmoveto{\pgfqpoint{3.073524in}{1.671027in}}%
\pgfpathlineto{\pgfqpoint{3.247270in}{1.671027in}}%
\pgfpathlineto{\pgfqpoint{3.247270in}{1.671027in}}%
\pgfpathlineto{\pgfqpoint{3.073524in}{1.671027in}}%
\pgfpathlineto{\pgfqpoint{3.073524in}{1.671027in}}%
\pgfpathclose%
\pgfusepath{stroke,fill}%
\end{pgfscope}%
\begin{pgfscope}%
\pgfpathrectangle{\pgfqpoint{0.694444in}{0.416667in}}{\pgfqpoint{3.194444in}{1.416667in}}%
\pgfusepath{clip}%
\pgfsetbuttcap%
\pgfsetmiterjoin%
\definecolor{currentfill}{rgb}{0.447059,0.447059,0.447059}%
\pgfsetfillcolor{currentfill}%
\pgfsetlinewidth{1.003750pt}%
\definecolor{currentstroke}{rgb}{0.266667,0.266667,0.266667}%
\pgfsetstrokecolor{currentstroke}%
\pgfsetdash{}{0pt}%
\pgfpathmoveto{\pgfqpoint{3.321732in}{1.612970in}}%
\pgfpathlineto{\pgfqpoint{3.495478in}{1.612970in}}%
\pgfpathlineto{\pgfqpoint{3.495478in}{1.612970in}}%
\pgfpathlineto{\pgfqpoint{3.321732in}{1.612970in}}%
\pgfpathlineto{\pgfqpoint{3.321732in}{1.612970in}}%
\pgfpathclose%
\pgfusepath{stroke,fill}%
\end{pgfscope}%
\begin{pgfscope}%
\pgfpathrectangle{\pgfqpoint{0.694444in}{0.416667in}}{\pgfqpoint{3.194444in}{1.416667in}}%
\pgfusepath{clip}%
\pgfsetbuttcap%
\pgfsetmiterjoin%
\definecolor{currentfill}{rgb}{0.447059,0.447059,0.447059}%
\pgfsetfillcolor{currentfill}%
\pgfsetlinewidth{1.003750pt}%
\definecolor{currentstroke}{rgb}{0.266667,0.266667,0.266667}%
\pgfsetstrokecolor{currentstroke}%
\pgfsetdash{}{0pt}%
\pgfpathmoveto{\pgfqpoint{3.569941in}{1.647067in}}%
\pgfpathlineto{\pgfqpoint{3.743687in}{1.647067in}}%
\pgfpathlineto{\pgfqpoint{3.743687in}{1.647067in}}%
\pgfpathlineto{\pgfqpoint{3.569941in}{1.647067in}}%
\pgfpathlineto{\pgfqpoint{3.569941in}{1.647067in}}%
\pgfpathclose%
\pgfusepath{stroke,fill}%
\end{pgfscope}%
\begin{pgfscope}%
\pgfpathrectangle{\pgfqpoint{0.694444in}{0.416667in}}{\pgfqpoint{3.194444in}{1.416667in}}%
\pgfusepath{clip}%
\pgfsetbuttcap%
\pgfsetmiterjoin%
\definecolor{currentfill}{rgb}{0.447059,0.447059,0.447059}%
\pgfsetfillcolor{currentfill}%
\pgfsetlinewidth{1.003750pt}%
\definecolor{currentstroke}{rgb}{0.266667,0.266667,0.266667}%
\pgfsetstrokecolor{currentstroke}%
\pgfsetdash{}{0pt}%
\pgfpathmoveto{\pgfqpoint{0.839646in}{1.500470in}}%
\pgfpathlineto{\pgfqpoint{1.013392in}{1.500470in}}%
\pgfpathlineto{\pgfqpoint{1.013392in}{1.500470in}}%
\pgfpathlineto{\pgfqpoint{0.839646in}{1.500470in}}%
\pgfpathlineto{\pgfqpoint{0.839646in}{1.500470in}}%
\pgfpathclose%
\pgfusepath{stroke,fill}%
\end{pgfscope}%
\begin{pgfscope}%
\pgfpathrectangle{\pgfqpoint{0.694444in}{0.416667in}}{\pgfqpoint{3.194444in}{1.416667in}}%
\pgfusepath{clip}%
\pgfsetbuttcap%
\pgfsetmiterjoin%
\definecolor{currentfill}{rgb}{0.447059,0.447059,0.447059}%
\pgfsetfillcolor{currentfill}%
\pgfsetlinewidth{1.003750pt}%
\definecolor{currentstroke}{rgb}{0.266667,0.266667,0.266667}%
\pgfsetstrokecolor{currentstroke}%
\pgfsetdash{}{0pt}%
\pgfpathmoveto{\pgfqpoint{1.087855in}{1.493113in}}%
\pgfpathlineto{\pgfqpoint{1.261601in}{1.493113in}}%
\pgfpathlineto{\pgfqpoint{1.261601in}{1.493113in}}%
\pgfpathlineto{\pgfqpoint{1.087855in}{1.493113in}}%
\pgfpathlineto{\pgfqpoint{1.087855in}{1.493113in}}%
\pgfpathclose%
\pgfusepath{stroke,fill}%
\end{pgfscope}%
\begin{pgfscope}%
\pgfpathrectangle{\pgfqpoint{0.694444in}{0.416667in}}{\pgfqpoint{3.194444in}{1.416667in}}%
\pgfusepath{clip}%
\pgfsetbuttcap%
\pgfsetmiterjoin%
\definecolor{currentfill}{rgb}{0.447059,0.447059,0.447059}%
\pgfsetfillcolor{currentfill}%
\pgfsetlinewidth{1.003750pt}%
\definecolor{currentstroke}{rgb}{0.266667,0.266667,0.266667}%
\pgfsetstrokecolor{currentstroke}%
\pgfsetdash{}{0pt}%
\pgfpathmoveto{\pgfqpoint{1.336064in}{1.471125in}}%
\pgfpathlineto{\pgfqpoint{1.509810in}{1.471125in}}%
\pgfpathlineto{\pgfqpoint{1.509810in}{1.471125in}}%
\pgfpathlineto{\pgfqpoint{1.336064in}{1.471125in}}%
\pgfpathlineto{\pgfqpoint{1.336064in}{1.471125in}}%
\pgfpathclose%
\pgfusepath{stroke,fill}%
\end{pgfscope}%
\begin{pgfscope}%
\pgfpathrectangle{\pgfqpoint{0.694444in}{0.416667in}}{\pgfqpoint{3.194444in}{1.416667in}}%
\pgfusepath{clip}%
\pgfsetbuttcap%
\pgfsetmiterjoin%
\definecolor{currentfill}{rgb}{0.447059,0.447059,0.447059}%
\pgfsetfillcolor{currentfill}%
\pgfsetlinewidth{1.003750pt}%
\definecolor{currentstroke}{rgb}{0.266667,0.266667,0.266667}%
\pgfsetstrokecolor{currentstroke}%
\pgfsetdash{}{0pt}%
\pgfpathmoveto{\pgfqpoint{1.584272in}{1.609389in}}%
\pgfpathlineto{\pgfqpoint{1.758018in}{1.609389in}}%
\pgfpathlineto{\pgfqpoint{1.758018in}{1.609389in}}%
\pgfpathlineto{\pgfqpoint{1.584272in}{1.609389in}}%
\pgfpathlineto{\pgfqpoint{1.584272in}{1.609389in}}%
\pgfpathclose%
\pgfusepath{stroke,fill}%
\end{pgfscope}%
\begin{pgfscope}%
\pgfpathrectangle{\pgfqpoint{0.694444in}{0.416667in}}{\pgfqpoint{3.194444in}{1.416667in}}%
\pgfusepath{clip}%
\pgfsetbuttcap%
\pgfsetmiterjoin%
\definecolor{currentfill}{rgb}{0.447059,0.447059,0.447059}%
\pgfsetfillcolor{currentfill}%
\pgfsetlinewidth{1.003750pt}%
\definecolor{currentstroke}{rgb}{0.266667,0.266667,0.266667}%
\pgfsetstrokecolor{currentstroke}%
\pgfsetdash{}{0pt}%
\pgfpathmoveto{\pgfqpoint{1.832481in}{1.659762in}}%
\pgfpathlineto{\pgfqpoint{2.006227in}{1.659762in}}%
\pgfpathlineto{\pgfqpoint{2.006227in}{1.659762in}}%
\pgfpathlineto{\pgfqpoint{1.832481in}{1.659762in}}%
\pgfpathlineto{\pgfqpoint{1.832481in}{1.659762in}}%
\pgfpathclose%
\pgfusepath{stroke,fill}%
\end{pgfscope}%
\begin{pgfscope}%
\pgfpathrectangle{\pgfqpoint{0.694444in}{0.416667in}}{\pgfqpoint{3.194444in}{1.416667in}}%
\pgfusepath{clip}%
\pgfsetbuttcap%
\pgfsetmiterjoin%
\definecolor{currentfill}{rgb}{0.447059,0.447059,0.447059}%
\pgfsetfillcolor{currentfill}%
\pgfsetlinewidth{1.003750pt}%
\definecolor{currentstroke}{rgb}{0.266667,0.266667,0.266667}%
\pgfsetstrokecolor{currentstroke}%
\pgfsetdash{}{0pt}%
\pgfpathmoveto{\pgfqpoint{2.080689in}{1.670074in}}%
\pgfpathlineto{\pgfqpoint{2.254435in}{1.670074in}}%
\pgfpathlineto{\pgfqpoint{2.254435in}{1.670074in}}%
\pgfpathlineto{\pgfqpoint{2.080689in}{1.670074in}}%
\pgfpathlineto{\pgfqpoint{2.080689in}{1.670074in}}%
\pgfpathclose%
\pgfusepath{stroke,fill}%
\end{pgfscope}%
\begin{pgfscope}%
\pgfpathrectangle{\pgfqpoint{0.694444in}{0.416667in}}{\pgfqpoint{3.194444in}{1.416667in}}%
\pgfusepath{clip}%
\pgfsetbuttcap%
\pgfsetmiterjoin%
\definecolor{currentfill}{rgb}{0.447059,0.447059,0.447059}%
\pgfsetfillcolor{currentfill}%
\pgfsetlinewidth{1.003750pt}%
\definecolor{currentstroke}{rgb}{0.266667,0.266667,0.266667}%
\pgfsetstrokecolor{currentstroke}%
\pgfsetdash{}{0pt}%
\pgfpathmoveto{\pgfqpoint{2.328898in}{1.631526in}}%
\pgfpathlineto{\pgfqpoint{2.502644in}{1.631526in}}%
\pgfpathlineto{\pgfqpoint{2.502644in}{1.631526in}}%
\pgfpathlineto{\pgfqpoint{2.328898in}{1.631526in}}%
\pgfpathlineto{\pgfqpoint{2.328898in}{1.631526in}}%
\pgfpathclose%
\pgfusepath{stroke,fill}%
\end{pgfscope}%
\begin{pgfscope}%
\pgfpathrectangle{\pgfqpoint{0.694444in}{0.416667in}}{\pgfqpoint{3.194444in}{1.416667in}}%
\pgfusepath{clip}%
\pgfsetbuttcap%
\pgfsetmiterjoin%
\definecolor{currentfill}{rgb}{0.447059,0.447059,0.447059}%
\pgfsetfillcolor{currentfill}%
\pgfsetlinewidth{1.003750pt}%
\definecolor{currentstroke}{rgb}{0.266667,0.266667,0.266667}%
\pgfsetstrokecolor{currentstroke}%
\pgfsetdash{}{0pt}%
\pgfpathmoveto{\pgfqpoint{2.577107in}{1.587585in}}%
\pgfpathlineto{\pgfqpoint{2.750853in}{1.587585in}}%
\pgfpathlineto{\pgfqpoint{2.750853in}{1.587585in}}%
\pgfpathlineto{\pgfqpoint{2.577107in}{1.587585in}}%
\pgfpathlineto{\pgfqpoint{2.577107in}{1.587585in}}%
\pgfpathclose%
\pgfusepath{stroke,fill}%
\end{pgfscope}%
\begin{pgfscope}%
\pgfpathrectangle{\pgfqpoint{0.694444in}{0.416667in}}{\pgfqpoint{3.194444in}{1.416667in}}%
\pgfusepath{clip}%
\pgfsetbuttcap%
\pgfsetmiterjoin%
\definecolor{currentfill}{rgb}{0.447059,0.447059,0.447059}%
\pgfsetfillcolor{currentfill}%
\pgfsetlinewidth{1.003750pt}%
\definecolor{currentstroke}{rgb}{0.266667,0.266667,0.266667}%
\pgfsetstrokecolor{currentstroke}%
\pgfsetdash{}{0pt}%
\pgfpathmoveto{\pgfqpoint{2.825315in}{1.610349in}}%
\pgfpathlineto{\pgfqpoint{2.999061in}{1.610349in}}%
\pgfpathlineto{\pgfqpoint{2.999061in}{1.610349in}}%
\pgfpathlineto{\pgfqpoint{2.825315in}{1.610349in}}%
\pgfpathlineto{\pgfqpoint{2.825315in}{1.610349in}}%
\pgfpathclose%
\pgfusepath{stroke,fill}%
\end{pgfscope}%
\begin{pgfscope}%
\pgfpathrectangle{\pgfqpoint{0.694444in}{0.416667in}}{\pgfqpoint{3.194444in}{1.416667in}}%
\pgfusepath{clip}%
\pgfsetbuttcap%
\pgfsetmiterjoin%
\definecolor{currentfill}{rgb}{0.447059,0.447059,0.447059}%
\pgfsetfillcolor{currentfill}%
\pgfsetlinewidth{1.003750pt}%
\definecolor{currentstroke}{rgb}{0.266667,0.266667,0.266667}%
\pgfsetstrokecolor{currentstroke}%
\pgfsetdash{}{0pt}%
\pgfpathmoveto{\pgfqpoint{3.073524in}{1.671027in}}%
\pgfpathlineto{\pgfqpoint{3.247270in}{1.671027in}}%
\pgfpathlineto{\pgfqpoint{3.247270in}{1.671027in}}%
\pgfpathlineto{\pgfqpoint{3.073524in}{1.671027in}}%
\pgfpathlineto{\pgfqpoint{3.073524in}{1.671027in}}%
\pgfpathclose%
\pgfusepath{stroke,fill}%
\end{pgfscope}%
\begin{pgfscope}%
\pgfpathrectangle{\pgfqpoint{0.694444in}{0.416667in}}{\pgfqpoint{3.194444in}{1.416667in}}%
\pgfusepath{clip}%
\pgfsetbuttcap%
\pgfsetmiterjoin%
\definecolor{currentfill}{rgb}{0.447059,0.447059,0.447059}%
\pgfsetfillcolor{currentfill}%
\pgfsetlinewidth{1.003750pt}%
\definecolor{currentstroke}{rgb}{0.266667,0.266667,0.266667}%
\pgfsetstrokecolor{currentstroke}%
\pgfsetdash{}{0pt}%
\pgfpathmoveto{\pgfqpoint{3.321732in}{1.612970in}}%
\pgfpathlineto{\pgfqpoint{3.495478in}{1.612970in}}%
\pgfpathlineto{\pgfqpoint{3.495478in}{1.612970in}}%
\pgfpathlineto{\pgfqpoint{3.321732in}{1.612970in}}%
\pgfpathlineto{\pgfqpoint{3.321732in}{1.612970in}}%
\pgfpathclose%
\pgfusepath{stroke,fill}%
\end{pgfscope}%
\begin{pgfscope}%
\pgfpathrectangle{\pgfqpoint{0.694444in}{0.416667in}}{\pgfqpoint{3.194444in}{1.416667in}}%
\pgfusepath{clip}%
\pgfsetbuttcap%
\pgfsetmiterjoin%
\definecolor{currentfill}{rgb}{0.447059,0.447059,0.447059}%
\pgfsetfillcolor{currentfill}%
\pgfsetlinewidth{1.003750pt}%
\definecolor{currentstroke}{rgb}{0.266667,0.266667,0.266667}%
\pgfsetstrokecolor{currentstroke}%
\pgfsetdash{}{0pt}%
\pgfpathmoveto{\pgfqpoint{3.569941in}{1.647067in}}%
\pgfpathlineto{\pgfqpoint{3.743687in}{1.647067in}}%
\pgfpathlineto{\pgfqpoint{3.743687in}{1.647067in}}%
\pgfpathlineto{\pgfqpoint{3.569941in}{1.647067in}}%
\pgfpathlineto{\pgfqpoint{3.569941in}{1.647067in}}%
\pgfpathclose%
\pgfusepath{stroke,fill}%
\end{pgfscope}%
\begin{pgfscope}%
\pgfpathrectangle{\pgfqpoint{0.694444in}{0.416667in}}{\pgfqpoint{3.194444in}{1.416667in}}%
\pgfusepath{clip}%
\pgfsetbuttcap%
\pgfsetmiterjoin%
\definecolor{currentfill}{rgb}{0.447059,0.447059,0.447059}%
\pgfsetfillcolor{currentfill}%
\pgfsetlinewidth{1.003750pt}%
\definecolor{currentstroke}{rgb}{0.266667,0.266667,0.266667}%
\pgfsetstrokecolor{currentstroke}%
\pgfsetdash{}{0pt}%
\pgfpathmoveto{\pgfqpoint{0.839646in}{1.500470in}}%
\pgfpathlineto{\pgfqpoint{1.013392in}{1.500470in}}%
\pgfpathlineto{\pgfqpoint{1.013392in}{1.500470in}}%
\pgfpathlineto{\pgfqpoint{0.839646in}{1.500470in}}%
\pgfpathlineto{\pgfqpoint{0.839646in}{1.500470in}}%
\pgfpathclose%
\pgfusepath{stroke,fill}%
\end{pgfscope}%
\begin{pgfscope}%
\pgfpathrectangle{\pgfqpoint{0.694444in}{0.416667in}}{\pgfqpoint{3.194444in}{1.416667in}}%
\pgfusepath{clip}%
\pgfsetbuttcap%
\pgfsetmiterjoin%
\definecolor{currentfill}{rgb}{0.447059,0.447059,0.447059}%
\pgfsetfillcolor{currentfill}%
\pgfsetlinewidth{1.003750pt}%
\definecolor{currentstroke}{rgb}{0.266667,0.266667,0.266667}%
\pgfsetstrokecolor{currentstroke}%
\pgfsetdash{}{0pt}%
\pgfpathmoveto{\pgfqpoint{1.087855in}{1.493113in}}%
\pgfpathlineto{\pgfqpoint{1.261601in}{1.493113in}}%
\pgfpathlineto{\pgfqpoint{1.261601in}{1.493113in}}%
\pgfpathlineto{\pgfqpoint{1.087855in}{1.493113in}}%
\pgfpathlineto{\pgfqpoint{1.087855in}{1.493113in}}%
\pgfpathclose%
\pgfusepath{stroke,fill}%
\end{pgfscope}%
\begin{pgfscope}%
\pgfpathrectangle{\pgfqpoint{0.694444in}{0.416667in}}{\pgfqpoint{3.194444in}{1.416667in}}%
\pgfusepath{clip}%
\pgfsetbuttcap%
\pgfsetmiterjoin%
\definecolor{currentfill}{rgb}{0.447059,0.447059,0.447059}%
\pgfsetfillcolor{currentfill}%
\pgfsetlinewidth{1.003750pt}%
\definecolor{currentstroke}{rgb}{0.266667,0.266667,0.266667}%
\pgfsetstrokecolor{currentstroke}%
\pgfsetdash{}{0pt}%
\pgfpathmoveto{\pgfqpoint{1.336064in}{1.471125in}}%
\pgfpathlineto{\pgfqpoint{1.509810in}{1.471125in}}%
\pgfpathlineto{\pgfqpoint{1.509810in}{1.471125in}}%
\pgfpathlineto{\pgfqpoint{1.336064in}{1.471125in}}%
\pgfpathlineto{\pgfqpoint{1.336064in}{1.471125in}}%
\pgfpathclose%
\pgfusepath{stroke,fill}%
\end{pgfscope}%
\begin{pgfscope}%
\pgfpathrectangle{\pgfqpoint{0.694444in}{0.416667in}}{\pgfqpoint{3.194444in}{1.416667in}}%
\pgfusepath{clip}%
\pgfsetbuttcap%
\pgfsetmiterjoin%
\definecolor{currentfill}{rgb}{0.447059,0.447059,0.447059}%
\pgfsetfillcolor{currentfill}%
\pgfsetlinewidth{1.003750pt}%
\definecolor{currentstroke}{rgb}{0.266667,0.266667,0.266667}%
\pgfsetstrokecolor{currentstroke}%
\pgfsetdash{}{0pt}%
\pgfpathmoveto{\pgfqpoint{1.584272in}{1.609389in}}%
\pgfpathlineto{\pgfqpoint{1.758018in}{1.609389in}}%
\pgfpathlineto{\pgfqpoint{1.758018in}{1.609389in}}%
\pgfpathlineto{\pgfqpoint{1.584272in}{1.609389in}}%
\pgfpathlineto{\pgfqpoint{1.584272in}{1.609389in}}%
\pgfpathclose%
\pgfusepath{stroke,fill}%
\end{pgfscope}%
\begin{pgfscope}%
\pgfpathrectangle{\pgfqpoint{0.694444in}{0.416667in}}{\pgfqpoint{3.194444in}{1.416667in}}%
\pgfusepath{clip}%
\pgfsetbuttcap%
\pgfsetmiterjoin%
\definecolor{currentfill}{rgb}{0.447059,0.447059,0.447059}%
\pgfsetfillcolor{currentfill}%
\pgfsetlinewidth{1.003750pt}%
\definecolor{currentstroke}{rgb}{0.266667,0.266667,0.266667}%
\pgfsetstrokecolor{currentstroke}%
\pgfsetdash{}{0pt}%
\pgfpathmoveto{\pgfqpoint{1.832481in}{1.659762in}}%
\pgfpathlineto{\pgfqpoint{2.006227in}{1.659762in}}%
\pgfpathlineto{\pgfqpoint{2.006227in}{1.659762in}}%
\pgfpathlineto{\pgfqpoint{1.832481in}{1.659762in}}%
\pgfpathlineto{\pgfqpoint{1.832481in}{1.659762in}}%
\pgfpathclose%
\pgfusepath{stroke,fill}%
\end{pgfscope}%
\begin{pgfscope}%
\pgfpathrectangle{\pgfqpoint{0.694444in}{0.416667in}}{\pgfqpoint{3.194444in}{1.416667in}}%
\pgfusepath{clip}%
\pgfsetbuttcap%
\pgfsetmiterjoin%
\definecolor{currentfill}{rgb}{0.447059,0.447059,0.447059}%
\pgfsetfillcolor{currentfill}%
\pgfsetlinewidth{1.003750pt}%
\definecolor{currentstroke}{rgb}{0.266667,0.266667,0.266667}%
\pgfsetstrokecolor{currentstroke}%
\pgfsetdash{}{0pt}%
\pgfpathmoveto{\pgfqpoint{2.080689in}{1.670074in}}%
\pgfpathlineto{\pgfqpoint{2.254435in}{1.670074in}}%
\pgfpathlineto{\pgfqpoint{2.254435in}{1.670074in}}%
\pgfpathlineto{\pgfqpoint{2.080689in}{1.670074in}}%
\pgfpathlineto{\pgfqpoint{2.080689in}{1.670074in}}%
\pgfpathclose%
\pgfusepath{stroke,fill}%
\end{pgfscope}%
\begin{pgfscope}%
\pgfpathrectangle{\pgfqpoint{0.694444in}{0.416667in}}{\pgfqpoint{3.194444in}{1.416667in}}%
\pgfusepath{clip}%
\pgfsetbuttcap%
\pgfsetmiterjoin%
\definecolor{currentfill}{rgb}{0.447059,0.447059,0.447059}%
\pgfsetfillcolor{currentfill}%
\pgfsetlinewidth{1.003750pt}%
\definecolor{currentstroke}{rgb}{0.266667,0.266667,0.266667}%
\pgfsetstrokecolor{currentstroke}%
\pgfsetdash{}{0pt}%
\pgfpathmoveto{\pgfqpoint{2.328898in}{1.631526in}}%
\pgfpathlineto{\pgfqpoint{2.502644in}{1.631526in}}%
\pgfpathlineto{\pgfqpoint{2.502644in}{1.631526in}}%
\pgfpathlineto{\pgfqpoint{2.328898in}{1.631526in}}%
\pgfpathlineto{\pgfqpoint{2.328898in}{1.631526in}}%
\pgfpathclose%
\pgfusepath{stroke,fill}%
\end{pgfscope}%
\begin{pgfscope}%
\pgfpathrectangle{\pgfqpoint{0.694444in}{0.416667in}}{\pgfqpoint{3.194444in}{1.416667in}}%
\pgfusepath{clip}%
\pgfsetbuttcap%
\pgfsetmiterjoin%
\definecolor{currentfill}{rgb}{0.447059,0.447059,0.447059}%
\pgfsetfillcolor{currentfill}%
\pgfsetlinewidth{1.003750pt}%
\definecolor{currentstroke}{rgb}{0.266667,0.266667,0.266667}%
\pgfsetstrokecolor{currentstroke}%
\pgfsetdash{}{0pt}%
\pgfpathmoveto{\pgfqpoint{2.577107in}{1.587585in}}%
\pgfpathlineto{\pgfqpoint{2.750853in}{1.587585in}}%
\pgfpathlineto{\pgfqpoint{2.750853in}{1.587585in}}%
\pgfpathlineto{\pgfqpoint{2.577107in}{1.587585in}}%
\pgfpathlineto{\pgfqpoint{2.577107in}{1.587585in}}%
\pgfpathclose%
\pgfusepath{stroke,fill}%
\end{pgfscope}%
\begin{pgfscope}%
\pgfpathrectangle{\pgfqpoint{0.694444in}{0.416667in}}{\pgfqpoint{3.194444in}{1.416667in}}%
\pgfusepath{clip}%
\pgfsetbuttcap%
\pgfsetmiterjoin%
\definecolor{currentfill}{rgb}{0.447059,0.447059,0.447059}%
\pgfsetfillcolor{currentfill}%
\pgfsetlinewidth{1.003750pt}%
\definecolor{currentstroke}{rgb}{0.266667,0.266667,0.266667}%
\pgfsetstrokecolor{currentstroke}%
\pgfsetdash{}{0pt}%
\pgfpathmoveto{\pgfqpoint{2.825315in}{1.610349in}}%
\pgfpathlineto{\pgfqpoint{2.999061in}{1.610349in}}%
\pgfpathlineto{\pgfqpoint{2.999061in}{1.610349in}}%
\pgfpathlineto{\pgfqpoint{2.825315in}{1.610349in}}%
\pgfpathlineto{\pgfqpoint{2.825315in}{1.610349in}}%
\pgfpathclose%
\pgfusepath{stroke,fill}%
\end{pgfscope}%
\begin{pgfscope}%
\pgfpathrectangle{\pgfqpoint{0.694444in}{0.416667in}}{\pgfqpoint{3.194444in}{1.416667in}}%
\pgfusepath{clip}%
\pgfsetbuttcap%
\pgfsetmiterjoin%
\definecolor{currentfill}{rgb}{0.447059,0.447059,0.447059}%
\pgfsetfillcolor{currentfill}%
\pgfsetlinewidth{1.003750pt}%
\definecolor{currentstroke}{rgb}{0.266667,0.266667,0.266667}%
\pgfsetstrokecolor{currentstroke}%
\pgfsetdash{}{0pt}%
\pgfpathmoveto{\pgfqpoint{3.073524in}{1.671027in}}%
\pgfpathlineto{\pgfqpoint{3.247270in}{1.671027in}}%
\pgfpathlineto{\pgfqpoint{3.247270in}{1.671027in}}%
\pgfpathlineto{\pgfqpoint{3.073524in}{1.671027in}}%
\pgfpathlineto{\pgfqpoint{3.073524in}{1.671027in}}%
\pgfpathclose%
\pgfusepath{stroke,fill}%
\end{pgfscope}%
\begin{pgfscope}%
\pgfpathrectangle{\pgfqpoint{0.694444in}{0.416667in}}{\pgfqpoint{3.194444in}{1.416667in}}%
\pgfusepath{clip}%
\pgfsetbuttcap%
\pgfsetmiterjoin%
\definecolor{currentfill}{rgb}{0.447059,0.447059,0.447059}%
\pgfsetfillcolor{currentfill}%
\pgfsetlinewidth{1.003750pt}%
\definecolor{currentstroke}{rgb}{0.266667,0.266667,0.266667}%
\pgfsetstrokecolor{currentstroke}%
\pgfsetdash{}{0pt}%
\pgfpathmoveto{\pgfqpoint{3.321732in}{1.612970in}}%
\pgfpathlineto{\pgfqpoint{3.495478in}{1.612970in}}%
\pgfpathlineto{\pgfqpoint{3.495478in}{1.612970in}}%
\pgfpathlineto{\pgfqpoint{3.321732in}{1.612970in}}%
\pgfpathlineto{\pgfqpoint{3.321732in}{1.612970in}}%
\pgfpathclose%
\pgfusepath{stroke,fill}%
\end{pgfscope}%
\begin{pgfscope}%
\pgfpathrectangle{\pgfqpoint{0.694444in}{0.416667in}}{\pgfqpoint{3.194444in}{1.416667in}}%
\pgfusepath{clip}%
\pgfsetbuttcap%
\pgfsetmiterjoin%
\definecolor{currentfill}{rgb}{0.447059,0.447059,0.447059}%
\pgfsetfillcolor{currentfill}%
\pgfsetlinewidth{1.003750pt}%
\definecolor{currentstroke}{rgb}{0.266667,0.266667,0.266667}%
\pgfsetstrokecolor{currentstroke}%
\pgfsetdash{}{0pt}%
\pgfpathmoveto{\pgfqpoint{3.569941in}{1.647067in}}%
\pgfpathlineto{\pgfqpoint{3.743687in}{1.647067in}}%
\pgfpathlineto{\pgfqpoint{3.743687in}{1.647067in}}%
\pgfpathlineto{\pgfqpoint{3.569941in}{1.647067in}}%
\pgfpathlineto{\pgfqpoint{3.569941in}{1.647067in}}%
\pgfpathclose%
\pgfusepath{stroke,fill}%
\end{pgfscope}%
\begin{pgfscope}%
\pgfpathrectangle{\pgfqpoint{0.694444in}{0.416667in}}{\pgfqpoint{3.194444in}{1.416667in}}%
\pgfusepath{clip}%
\pgfsetbuttcap%
\pgfsetmiterjoin%
\definecolor{currentfill}{rgb}{0.447059,0.447059,0.447059}%
\pgfsetfillcolor{currentfill}%
\pgfsetlinewidth{1.003750pt}%
\definecolor{currentstroke}{rgb}{0.266667,0.266667,0.266667}%
\pgfsetstrokecolor{currentstroke}%
\pgfsetdash{}{0pt}%
\pgfpathmoveto{\pgfqpoint{0.839646in}{1.500470in}}%
\pgfpathlineto{\pgfqpoint{1.013392in}{1.500470in}}%
\pgfpathlineto{\pgfqpoint{1.013392in}{1.500470in}}%
\pgfpathlineto{\pgfqpoint{0.839646in}{1.500470in}}%
\pgfpathlineto{\pgfqpoint{0.839646in}{1.500470in}}%
\pgfpathclose%
\pgfusepath{stroke,fill}%
\end{pgfscope}%
\begin{pgfscope}%
\pgfpathrectangle{\pgfqpoint{0.694444in}{0.416667in}}{\pgfqpoint{3.194444in}{1.416667in}}%
\pgfusepath{clip}%
\pgfsetbuttcap%
\pgfsetmiterjoin%
\definecolor{currentfill}{rgb}{0.447059,0.447059,0.447059}%
\pgfsetfillcolor{currentfill}%
\pgfsetlinewidth{1.003750pt}%
\definecolor{currentstroke}{rgb}{0.266667,0.266667,0.266667}%
\pgfsetstrokecolor{currentstroke}%
\pgfsetdash{}{0pt}%
\pgfpathmoveto{\pgfqpoint{1.087855in}{1.493113in}}%
\pgfpathlineto{\pgfqpoint{1.261601in}{1.493113in}}%
\pgfpathlineto{\pgfqpoint{1.261601in}{1.493113in}}%
\pgfpathlineto{\pgfqpoint{1.087855in}{1.493113in}}%
\pgfpathlineto{\pgfqpoint{1.087855in}{1.493113in}}%
\pgfpathclose%
\pgfusepath{stroke,fill}%
\end{pgfscope}%
\begin{pgfscope}%
\pgfpathrectangle{\pgfqpoint{0.694444in}{0.416667in}}{\pgfqpoint{3.194444in}{1.416667in}}%
\pgfusepath{clip}%
\pgfsetbuttcap%
\pgfsetmiterjoin%
\definecolor{currentfill}{rgb}{0.447059,0.447059,0.447059}%
\pgfsetfillcolor{currentfill}%
\pgfsetlinewidth{1.003750pt}%
\definecolor{currentstroke}{rgb}{0.266667,0.266667,0.266667}%
\pgfsetstrokecolor{currentstroke}%
\pgfsetdash{}{0pt}%
\pgfpathmoveto{\pgfqpoint{1.336064in}{1.471125in}}%
\pgfpathlineto{\pgfqpoint{1.509810in}{1.471125in}}%
\pgfpathlineto{\pgfqpoint{1.509810in}{1.471125in}}%
\pgfpathlineto{\pgfqpoint{1.336064in}{1.471125in}}%
\pgfpathlineto{\pgfqpoint{1.336064in}{1.471125in}}%
\pgfpathclose%
\pgfusepath{stroke,fill}%
\end{pgfscope}%
\begin{pgfscope}%
\pgfpathrectangle{\pgfqpoint{0.694444in}{0.416667in}}{\pgfqpoint{3.194444in}{1.416667in}}%
\pgfusepath{clip}%
\pgfsetbuttcap%
\pgfsetmiterjoin%
\definecolor{currentfill}{rgb}{0.447059,0.447059,0.447059}%
\pgfsetfillcolor{currentfill}%
\pgfsetlinewidth{1.003750pt}%
\definecolor{currentstroke}{rgb}{0.266667,0.266667,0.266667}%
\pgfsetstrokecolor{currentstroke}%
\pgfsetdash{}{0pt}%
\pgfpathmoveto{\pgfqpoint{1.584272in}{1.609389in}}%
\pgfpathlineto{\pgfqpoint{1.758018in}{1.609389in}}%
\pgfpathlineto{\pgfqpoint{1.758018in}{1.609389in}}%
\pgfpathlineto{\pgfqpoint{1.584272in}{1.609389in}}%
\pgfpathlineto{\pgfqpoint{1.584272in}{1.609389in}}%
\pgfpathclose%
\pgfusepath{stroke,fill}%
\end{pgfscope}%
\begin{pgfscope}%
\pgfpathrectangle{\pgfqpoint{0.694444in}{0.416667in}}{\pgfqpoint{3.194444in}{1.416667in}}%
\pgfusepath{clip}%
\pgfsetbuttcap%
\pgfsetmiterjoin%
\definecolor{currentfill}{rgb}{0.447059,0.447059,0.447059}%
\pgfsetfillcolor{currentfill}%
\pgfsetlinewidth{1.003750pt}%
\definecolor{currentstroke}{rgb}{0.266667,0.266667,0.266667}%
\pgfsetstrokecolor{currentstroke}%
\pgfsetdash{}{0pt}%
\pgfpathmoveto{\pgfqpoint{1.832481in}{1.659762in}}%
\pgfpathlineto{\pgfqpoint{2.006227in}{1.659762in}}%
\pgfpathlineto{\pgfqpoint{2.006227in}{1.659762in}}%
\pgfpathlineto{\pgfqpoint{1.832481in}{1.659762in}}%
\pgfpathlineto{\pgfqpoint{1.832481in}{1.659762in}}%
\pgfpathclose%
\pgfusepath{stroke,fill}%
\end{pgfscope}%
\begin{pgfscope}%
\pgfpathrectangle{\pgfqpoint{0.694444in}{0.416667in}}{\pgfqpoint{3.194444in}{1.416667in}}%
\pgfusepath{clip}%
\pgfsetbuttcap%
\pgfsetmiterjoin%
\definecolor{currentfill}{rgb}{0.447059,0.447059,0.447059}%
\pgfsetfillcolor{currentfill}%
\pgfsetlinewidth{1.003750pt}%
\definecolor{currentstroke}{rgb}{0.266667,0.266667,0.266667}%
\pgfsetstrokecolor{currentstroke}%
\pgfsetdash{}{0pt}%
\pgfpathmoveto{\pgfqpoint{2.080689in}{1.670074in}}%
\pgfpathlineto{\pgfqpoint{2.254435in}{1.670074in}}%
\pgfpathlineto{\pgfqpoint{2.254435in}{1.670074in}}%
\pgfpathlineto{\pgfqpoint{2.080689in}{1.670074in}}%
\pgfpathlineto{\pgfqpoint{2.080689in}{1.670074in}}%
\pgfpathclose%
\pgfusepath{stroke,fill}%
\end{pgfscope}%
\begin{pgfscope}%
\pgfpathrectangle{\pgfqpoint{0.694444in}{0.416667in}}{\pgfqpoint{3.194444in}{1.416667in}}%
\pgfusepath{clip}%
\pgfsetbuttcap%
\pgfsetmiterjoin%
\definecolor{currentfill}{rgb}{0.447059,0.447059,0.447059}%
\pgfsetfillcolor{currentfill}%
\pgfsetlinewidth{1.003750pt}%
\definecolor{currentstroke}{rgb}{0.266667,0.266667,0.266667}%
\pgfsetstrokecolor{currentstroke}%
\pgfsetdash{}{0pt}%
\pgfpathmoveto{\pgfqpoint{2.328898in}{1.631526in}}%
\pgfpathlineto{\pgfqpoint{2.502644in}{1.631526in}}%
\pgfpathlineto{\pgfqpoint{2.502644in}{1.631526in}}%
\pgfpathlineto{\pgfqpoint{2.328898in}{1.631526in}}%
\pgfpathlineto{\pgfqpoint{2.328898in}{1.631526in}}%
\pgfpathclose%
\pgfusepath{stroke,fill}%
\end{pgfscope}%
\begin{pgfscope}%
\pgfpathrectangle{\pgfqpoint{0.694444in}{0.416667in}}{\pgfqpoint{3.194444in}{1.416667in}}%
\pgfusepath{clip}%
\pgfsetbuttcap%
\pgfsetmiterjoin%
\definecolor{currentfill}{rgb}{0.447059,0.447059,0.447059}%
\pgfsetfillcolor{currentfill}%
\pgfsetlinewidth{1.003750pt}%
\definecolor{currentstroke}{rgb}{0.266667,0.266667,0.266667}%
\pgfsetstrokecolor{currentstroke}%
\pgfsetdash{}{0pt}%
\pgfpathmoveto{\pgfqpoint{2.577107in}{1.587585in}}%
\pgfpathlineto{\pgfqpoint{2.750853in}{1.587585in}}%
\pgfpathlineto{\pgfqpoint{2.750853in}{1.587585in}}%
\pgfpathlineto{\pgfqpoint{2.577107in}{1.587585in}}%
\pgfpathlineto{\pgfqpoint{2.577107in}{1.587585in}}%
\pgfpathclose%
\pgfusepath{stroke,fill}%
\end{pgfscope}%
\begin{pgfscope}%
\pgfpathrectangle{\pgfqpoint{0.694444in}{0.416667in}}{\pgfqpoint{3.194444in}{1.416667in}}%
\pgfusepath{clip}%
\pgfsetbuttcap%
\pgfsetmiterjoin%
\definecolor{currentfill}{rgb}{0.447059,0.447059,0.447059}%
\pgfsetfillcolor{currentfill}%
\pgfsetlinewidth{1.003750pt}%
\definecolor{currentstroke}{rgb}{0.266667,0.266667,0.266667}%
\pgfsetstrokecolor{currentstroke}%
\pgfsetdash{}{0pt}%
\pgfpathmoveto{\pgfqpoint{2.825315in}{1.610349in}}%
\pgfpathlineto{\pgfqpoint{2.999061in}{1.610349in}}%
\pgfpathlineto{\pgfqpoint{2.999061in}{1.610349in}}%
\pgfpathlineto{\pgfqpoint{2.825315in}{1.610349in}}%
\pgfpathlineto{\pgfqpoint{2.825315in}{1.610349in}}%
\pgfpathclose%
\pgfusepath{stroke,fill}%
\end{pgfscope}%
\begin{pgfscope}%
\pgfpathrectangle{\pgfqpoint{0.694444in}{0.416667in}}{\pgfqpoint{3.194444in}{1.416667in}}%
\pgfusepath{clip}%
\pgfsetbuttcap%
\pgfsetmiterjoin%
\definecolor{currentfill}{rgb}{0.447059,0.447059,0.447059}%
\pgfsetfillcolor{currentfill}%
\pgfsetlinewidth{1.003750pt}%
\definecolor{currentstroke}{rgb}{0.266667,0.266667,0.266667}%
\pgfsetstrokecolor{currentstroke}%
\pgfsetdash{}{0pt}%
\pgfpathmoveto{\pgfqpoint{3.073524in}{1.671027in}}%
\pgfpathlineto{\pgfqpoint{3.247270in}{1.671027in}}%
\pgfpathlineto{\pgfqpoint{3.247270in}{1.671027in}}%
\pgfpathlineto{\pgfqpoint{3.073524in}{1.671027in}}%
\pgfpathlineto{\pgfqpoint{3.073524in}{1.671027in}}%
\pgfpathclose%
\pgfusepath{stroke,fill}%
\end{pgfscope}%
\begin{pgfscope}%
\pgfpathrectangle{\pgfqpoint{0.694444in}{0.416667in}}{\pgfqpoint{3.194444in}{1.416667in}}%
\pgfusepath{clip}%
\pgfsetbuttcap%
\pgfsetmiterjoin%
\definecolor{currentfill}{rgb}{0.447059,0.447059,0.447059}%
\pgfsetfillcolor{currentfill}%
\pgfsetlinewidth{1.003750pt}%
\definecolor{currentstroke}{rgb}{0.266667,0.266667,0.266667}%
\pgfsetstrokecolor{currentstroke}%
\pgfsetdash{}{0pt}%
\pgfpathmoveto{\pgfqpoint{3.321732in}{1.612970in}}%
\pgfpathlineto{\pgfqpoint{3.495478in}{1.612970in}}%
\pgfpathlineto{\pgfqpoint{3.495478in}{1.612970in}}%
\pgfpathlineto{\pgfqpoint{3.321732in}{1.612970in}}%
\pgfpathlineto{\pgfqpoint{3.321732in}{1.612970in}}%
\pgfpathclose%
\pgfusepath{stroke,fill}%
\end{pgfscope}%
\begin{pgfscope}%
\pgfpathrectangle{\pgfqpoint{0.694444in}{0.416667in}}{\pgfqpoint{3.194444in}{1.416667in}}%
\pgfusepath{clip}%
\pgfsetbuttcap%
\pgfsetmiterjoin%
\definecolor{currentfill}{rgb}{0.447059,0.447059,0.447059}%
\pgfsetfillcolor{currentfill}%
\pgfsetlinewidth{1.003750pt}%
\definecolor{currentstroke}{rgb}{0.266667,0.266667,0.266667}%
\pgfsetstrokecolor{currentstroke}%
\pgfsetdash{}{0pt}%
\pgfpathmoveto{\pgfqpoint{3.569941in}{1.647067in}}%
\pgfpathlineto{\pgfqpoint{3.743687in}{1.647067in}}%
\pgfpathlineto{\pgfqpoint{3.743687in}{1.647067in}}%
\pgfpathlineto{\pgfqpoint{3.569941in}{1.647067in}}%
\pgfpathlineto{\pgfqpoint{3.569941in}{1.647067in}}%
\pgfpathclose%
\pgfusepath{stroke,fill}%
\end{pgfscope}%
\begin{pgfscope}%
\pgfpathrectangle{\pgfqpoint{0.694444in}{0.416667in}}{\pgfqpoint{3.194444in}{1.416667in}}%
\pgfusepath{clip}%
\pgfsetbuttcap%
\pgfsetmiterjoin%
\definecolor{currentfill}{rgb}{0.447059,0.447059,0.447059}%
\pgfsetfillcolor{currentfill}%
\pgfsetlinewidth{1.003750pt}%
\definecolor{currentstroke}{rgb}{0.266667,0.266667,0.266667}%
\pgfsetstrokecolor{currentstroke}%
\pgfsetdash{}{0pt}%
\pgfpathmoveto{\pgfqpoint{0.839646in}{1.500470in}}%
\pgfpathlineto{\pgfqpoint{1.013392in}{1.500470in}}%
\pgfpathlineto{\pgfqpoint{1.013392in}{1.500470in}}%
\pgfpathlineto{\pgfqpoint{0.839646in}{1.500470in}}%
\pgfpathlineto{\pgfqpoint{0.839646in}{1.500470in}}%
\pgfpathclose%
\pgfusepath{stroke,fill}%
\end{pgfscope}%
\begin{pgfscope}%
\pgfpathrectangle{\pgfqpoint{0.694444in}{0.416667in}}{\pgfqpoint{3.194444in}{1.416667in}}%
\pgfusepath{clip}%
\pgfsetbuttcap%
\pgfsetmiterjoin%
\definecolor{currentfill}{rgb}{0.447059,0.447059,0.447059}%
\pgfsetfillcolor{currentfill}%
\pgfsetlinewidth{1.003750pt}%
\definecolor{currentstroke}{rgb}{0.266667,0.266667,0.266667}%
\pgfsetstrokecolor{currentstroke}%
\pgfsetdash{}{0pt}%
\pgfpathmoveto{\pgfqpoint{1.087855in}{1.493113in}}%
\pgfpathlineto{\pgfqpoint{1.261601in}{1.493113in}}%
\pgfpathlineto{\pgfqpoint{1.261601in}{1.493113in}}%
\pgfpathlineto{\pgfqpoint{1.087855in}{1.493113in}}%
\pgfpathlineto{\pgfqpoint{1.087855in}{1.493113in}}%
\pgfpathclose%
\pgfusepath{stroke,fill}%
\end{pgfscope}%
\begin{pgfscope}%
\pgfpathrectangle{\pgfqpoint{0.694444in}{0.416667in}}{\pgfqpoint{3.194444in}{1.416667in}}%
\pgfusepath{clip}%
\pgfsetbuttcap%
\pgfsetmiterjoin%
\definecolor{currentfill}{rgb}{0.447059,0.447059,0.447059}%
\pgfsetfillcolor{currentfill}%
\pgfsetlinewidth{1.003750pt}%
\definecolor{currentstroke}{rgb}{0.266667,0.266667,0.266667}%
\pgfsetstrokecolor{currentstroke}%
\pgfsetdash{}{0pt}%
\pgfpathmoveto{\pgfqpoint{1.336064in}{1.471125in}}%
\pgfpathlineto{\pgfqpoint{1.509810in}{1.471125in}}%
\pgfpathlineto{\pgfqpoint{1.509810in}{1.471125in}}%
\pgfpathlineto{\pgfqpoint{1.336064in}{1.471125in}}%
\pgfpathlineto{\pgfqpoint{1.336064in}{1.471125in}}%
\pgfpathclose%
\pgfusepath{stroke,fill}%
\end{pgfscope}%
\begin{pgfscope}%
\pgfpathrectangle{\pgfqpoint{0.694444in}{0.416667in}}{\pgfqpoint{3.194444in}{1.416667in}}%
\pgfusepath{clip}%
\pgfsetbuttcap%
\pgfsetmiterjoin%
\definecolor{currentfill}{rgb}{0.447059,0.447059,0.447059}%
\pgfsetfillcolor{currentfill}%
\pgfsetlinewidth{1.003750pt}%
\definecolor{currentstroke}{rgb}{0.266667,0.266667,0.266667}%
\pgfsetstrokecolor{currentstroke}%
\pgfsetdash{}{0pt}%
\pgfpathmoveto{\pgfqpoint{1.584272in}{1.609389in}}%
\pgfpathlineto{\pgfqpoint{1.758018in}{1.609389in}}%
\pgfpathlineto{\pgfqpoint{1.758018in}{1.609389in}}%
\pgfpathlineto{\pgfqpoint{1.584272in}{1.609389in}}%
\pgfpathlineto{\pgfqpoint{1.584272in}{1.609389in}}%
\pgfpathclose%
\pgfusepath{stroke,fill}%
\end{pgfscope}%
\begin{pgfscope}%
\pgfpathrectangle{\pgfqpoint{0.694444in}{0.416667in}}{\pgfqpoint{3.194444in}{1.416667in}}%
\pgfusepath{clip}%
\pgfsetbuttcap%
\pgfsetmiterjoin%
\definecolor{currentfill}{rgb}{0.447059,0.447059,0.447059}%
\pgfsetfillcolor{currentfill}%
\pgfsetlinewidth{1.003750pt}%
\definecolor{currentstroke}{rgb}{0.266667,0.266667,0.266667}%
\pgfsetstrokecolor{currentstroke}%
\pgfsetdash{}{0pt}%
\pgfpathmoveto{\pgfqpoint{1.832481in}{1.659762in}}%
\pgfpathlineto{\pgfqpoint{2.006227in}{1.659762in}}%
\pgfpathlineto{\pgfqpoint{2.006227in}{1.659762in}}%
\pgfpathlineto{\pgfqpoint{1.832481in}{1.659762in}}%
\pgfpathlineto{\pgfqpoint{1.832481in}{1.659762in}}%
\pgfpathclose%
\pgfusepath{stroke,fill}%
\end{pgfscope}%
\begin{pgfscope}%
\pgfpathrectangle{\pgfqpoint{0.694444in}{0.416667in}}{\pgfqpoint{3.194444in}{1.416667in}}%
\pgfusepath{clip}%
\pgfsetbuttcap%
\pgfsetmiterjoin%
\definecolor{currentfill}{rgb}{0.447059,0.447059,0.447059}%
\pgfsetfillcolor{currentfill}%
\pgfsetlinewidth{1.003750pt}%
\definecolor{currentstroke}{rgb}{0.266667,0.266667,0.266667}%
\pgfsetstrokecolor{currentstroke}%
\pgfsetdash{}{0pt}%
\pgfpathmoveto{\pgfqpoint{2.080689in}{1.670074in}}%
\pgfpathlineto{\pgfqpoint{2.254435in}{1.670074in}}%
\pgfpathlineto{\pgfqpoint{2.254435in}{1.670074in}}%
\pgfpathlineto{\pgfqpoint{2.080689in}{1.670074in}}%
\pgfpathlineto{\pgfqpoint{2.080689in}{1.670074in}}%
\pgfpathclose%
\pgfusepath{stroke,fill}%
\end{pgfscope}%
\begin{pgfscope}%
\pgfpathrectangle{\pgfqpoint{0.694444in}{0.416667in}}{\pgfqpoint{3.194444in}{1.416667in}}%
\pgfusepath{clip}%
\pgfsetbuttcap%
\pgfsetmiterjoin%
\definecolor{currentfill}{rgb}{0.447059,0.447059,0.447059}%
\pgfsetfillcolor{currentfill}%
\pgfsetlinewidth{1.003750pt}%
\definecolor{currentstroke}{rgb}{0.266667,0.266667,0.266667}%
\pgfsetstrokecolor{currentstroke}%
\pgfsetdash{}{0pt}%
\pgfpathmoveto{\pgfqpoint{2.328898in}{1.631526in}}%
\pgfpathlineto{\pgfqpoint{2.502644in}{1.631526in}}%
\pgfpathlineto{\pgfqpoint{2.502644in}{1.631526in}}%
\pgfpathlineto{\pgfqpoint{2.328898in}{1.631526in}}%
\pgfpathlineto{\pgfqpoint{2.328898in}{1.631526in}}%
\pgfpathclose%
\pgfusepath{stroke,fill}%
\end{pgfscope}%
\begin{pgfscope}%
\pgfpathrectangle{\pgfqpoint{0.694444in}{0.416667in}}{\pgfqpoint{3.194444in}{1.416667in}}%
\pgfusepath{clip}%
\pgfsetbuttcap%
\pgfsetmiterjoin%
\definecolor{currentfill}{rgb}{0.447059,0.447059,0.447059}%
\pgfsetfillcolor{currentfill}%
\pgfsetlinewidth{1.003750pt}%
\definecolor{currentstroke}{rgb}{0.266667,0.266667,0.266667}%
\pgfsetstrokecolor{currentstroke}%
\pgfsetdash{}{0pt}%
\pgfpathmoveto{\pgfqpoint{2.577107in}{1.587585in}}%
\pgfpathlineto{\pgfqpoint{2.750853in}{1.587585in}}%
\pgfpathlineto{\pgfqpoint{2.750853in}{1.587585in}}%
\pgfpathlineto{\pgfqpoint{2.577107in}{1.587585in}}%
\pgfpathlineto{\pgfqpoint{2.577107in}{1.587585in}}%
\pgfpathclose%
\pgfusepath{stroke,fill}%
\end{pgfscope}%
\begin{pgfscope}%
\pgfpathrectangle{\pgfqpoint{0.694444in}{0.416667in}}{\pgfqpoint{3.194444in}{1.416667in}}%
\pgfusepath{clip}%
\pgfsetbuttcap%
\pgfsetmiterjoin%
\definecolor{currentfill}{rgb}{0.447059,0.447059,0.447059}%
\pgfsetfillcolor{currentfill}%
\pgfsetlinewidth{1.003750pt}%
\definecolor{currentstroke}{rgb}{0.266667,0.266667,0.266667}%
\pgfsetstrokecolor{currentstroke}%
\pgfsetdash{}{0pt}%
\pgfpathmoveto{\pgfqpoint{2.825315in}{1.610349in}}%
\pgfpathlineto{\pgfqpoint{2.999061in}{1.610349in}}%
\pgfpathlineto{\pgfqpoint{2.999061in}{1.610349in}}%
\pgfpathlineto{\pgfqpoint{2.825315in}{1.610349in}}%
\pgfpathlineto{\pgfqpoint{2.825315in}{1.610349in}}%
\pgfpathclose%
\pgfusepath{stroke,fill}%
\end{pgfscope}%
\begin{pgfscope}%
\pgfpathrectangle{\pgfqpoint{0.694444in}{0.416667in}}{\pgfqpoint{3.194444in}{1.416667in}}%
\pgfusepath{clip}%
\pgfsetbuttcap%
\pgfsetmiterjoin%
\definecolor{currentfill}{rgb}{0.447059,0.447059,0.447059}%
\pgfsetfillcolor{currentfill}%
\pgfsetlinewidth{1.003750pt}%
\definecolor{currentstroke}{rgb}{0.266667,0.266667,0.266667}%
\pgfsetstrokecolor{currentstroke}%
\pgfsetdash{}{0pt}%
\pgfpathmoveto{\pgfqpoint{3.073524in}{1.671027in}}%
\pgfpathlineto{\pgfqpoint{3.247270in}{1.671027in}}%
\pgfpathlineto{\pgfqpoint{3.247270in}{1.671027in}}%
\pgfpathlineto{\pgfqpoint{3.073524in}{1.671027in}}%
\pgfpathlineto{\pgfqpoint{3.073524in}{1.671027in}}%
\pgfpathclose%
\pgfusepath{stroke,fill}%
\end{pgfscope}%
\begin{pgfscope}%
\pgfpathrectangle{\pgfqpoint{0.694444in}{0.416667in}}{\pgfqpoint{3.194444in}{1.416667in}}%
\pgfusepath{clip}%
\pgfsetbuttcap%
\pgfsetmiterjoin%
\definecolor{currentfill}{rgb}{0.447059,0.447059,0.447059}%
\pgfsetfillcolor{currentfill}%
\pgfsetlinewidth{1.003750pt}%
\definecolor{currentstroke}{rgb}{0.266667,0.266667,0.266667}%
\pgfsetstrokecolor{currentstroke}%
\pgfsetdash{}{0pt}%
\pgfpathmoveto{\pgfqpoint{3.321732in}{1.612970in}}%
\pgfpathlineto{\pgfqpoint{3.495478in}{1.612970in}}%
\pgfpathlineto{\pgfqpoint{3.495478in}{1.612970in}}%
\pgfpathlineto{\pgfqpoint{3.321732in}{1.612970in}}%
\pgfpathlineto{\pgfqpoint{3.321732in}{1.612970in}}%
\pgfpathclose%
\pgfusepath{stroke,fill}%
\end{pgfscope}%
\begin{pgfscope}%
\pgfpathrectangle{\pgfqpoint{0.694444in}{0.416667in}}{\pgfqpoint{3.194444in}{1.416667in}}%
\pgfusepath{clip}%
\pgfsetbuttcap%
\pgfsetmiterjoin%
\definecolor{currentfill}{rgb}{0.447059,0.447059,0.447059}%
\pgfsetfillcolor{currentfill}%
\pgfsetlinewidth{1.003750pt}%
\definecolor{currentstroke}{rgb}{0.266667,0.266667,0.266667}%
\pgfsetstrokecolor{currentstroke}%
\pgfsetdash{}{0pt}%
\pgfpathmoveto{\pgfqpoint{3.569941in}{1.647067in}}%
\pgfpathlineto{\pgfqpoint{3.743687in}{1.647067in}}%
\pgfpathlineto{\pgfqpoint{3.743687in}{1.647067in}}%
\pgfpathlineto{\pgfqpoint{3.569941in}{1.647067in}}%
\pgfpathlineto{\pgfqpoint{3.569941in}{1.647067in}}%
\pgfpathclose%
\pgfusepath{stroke,fill}%
\end{pgfscope}%
\begin{pgfscope}%
\pgfpathrectangle{\pgfqpoint{0.694444in}{0.416667in}}{\pgfqpoint{3.194444in}{1.416667in}}%
\pgfusepath{clip}%
\pgfsetbuttcap%
\pgfsetmiterjoin%
\definecolor{currentfill}{rgb}{0.447059,0.447059,0.447059}%
\pgfsetfillcolor{currentfill}%
\pgfsetlinewidth{1.003750pt}%
\definecolor{currentstroke}{rgb}{0.266667,0.266667,0.266667}%
\pgfsetstrokecolor{currentstroke}%
\pgfsetdash{}{0pt}%
\pgfpathmoveto{\pgfqpoint{0.839646in}{1.500470in}}%
\pgfpathlineto{\pgfqpoint{1.013392in}{1.500470in}}%
\pgfpathlineto{\pgfqpoint{1.013392in}{1.500470in}}%
\pgfpathlineto{\pgfqpoint{0.839646in}{1.500470in}}%
\pgfpathlineto{\pgfqpoint{0.839646in}{1.500470in}}%
\pgfpathclose%
\pgfusepath{stroke,fill}%
\end{pgfscope}%
\begin{pgfscope}%
\pgfpathrectangle{\pgfqpoint{0.694444in}{0.416667in}}{\pgfqpoint{3.194444in}{1.416667in}}%
\pgfusepath{clip}%
\pgfsetbuttcap%
\pgfsetmiterjoin%
\definecolor{currentfill}{rgb}{0.447059,0.447059,0.447059}%
\pgfsetfillcolor{currentfill}%
\pgfsetlinewidth{1.003750pt}%
\definecolor{currentstroke}{rgb}{0.266667,0.266667,0.266667}%
\pgfsetstrokecolor{currentstroke}%
\pgfsetdash{}{0pt}%
\pgfpathmoveto{\pgfqpoint{1.087855in}{1.493113in}}%
\pgfpathlineto{\pgfqpoint{1.261601in}{1.493113in}}%
\pgfpathlineto{\pgfqpoint{1.261601in}{1.493113in}}%
\pgfpathlineto{\pgfqpoint{1.087855in}{1.493113in}}%
\pgfpathlineto{\pgfqpoint{1.087855in}{1.493113in}}%
\pgfpathclose%
\pgfusepath{stroke,fill}%
\end{pgfscope}%
\begin{pgfscope}%
\pgfpathrectangle{\pgfqpoint{0.694444in}{0.416667in}}{\pgfqpoint{3.194444in}{1.416667in}}%
\pgfusepath{clip}%
\pgfsetbuttcap%
\pgfsetmiterjoin%
\definecolor{currentfill}{rgb}{0.447059,0.447059,0.447059}%
\pgfsetfillcolor{currentfill}%
\pgfsetlinewidth{1.003750pt}%
\definecolor{currentstroke}{rgb}{0.266667,0.266667,0.266667}%
\pgfsetstrokecolor{currentstroke}%
\pgfsetdash{}{0pt}%
\pgfpathmoveto{\pgfqpoint{1.336064in}{1.471125in}}%
\pgfpathlineto{\pgfqpoint{1.509810in}{1.471125in}}%
\pgfpathlineto{\pgfqpoint{1.509810in}{1.471125in}}%
\pgfpathlineto{\pgfqpoint{1.336064in}{1.471125in}}%
\pgfpathlineto{\pgfqpoint{1.336064in}{1.471125in}}%
\pgfpathclose%
\pgfusepath{stroke,fill}%
\end{pgfscope}%
\begin{pgfscope}%
\pgfpathrectangle{\pgfqpoint{0.694444in}{0.416667in}}{\pgfqpoint{3.194444in}{1.416667in}}%
\pgfusepath{clip}%
\pgfsetbuttcap%
\pgfsetmiterjoin%
\definecolor{currentfill}{rgb}{0.447059,0.447059,0.447059}%
\pgfsetfillcolor{currentfill}%
\pgfsetlinewidth{1.003750pt}%
\definecolor{currentstroke}{rgb}{0.266667,0.266667,0.266667}%
\pgfsetstrokecolor{currentstroke}%
\pgfsetdash{}{0pt}%
\pgfpathmoveto{\pgfqpoint{1.584272in}{1.609389in}}%
\pgfpathlineto{\pgfqpoint{1.758018in}{1.609389in}}%
\pgfpathlineto{\pgfqpoint{1.758018in}{1.609389in}}%
\pgfpathlineto{\pgfqpoint{1.584272in}{1.609389in}}%
\pgfpathlineto{\pgfqpoint{1.584272in}{1.609389in}}%
\pgfpathclose%
\pgfusepath{stroke,fill}%
\end{pgfscope}%
\begin{pgfscope}%
\pgfpathrectangle{\pgfqpoint{0.694444in}{0.416667in}}{\pgfqpoint{3.194444in}{1.416667in}}%
\pgfusepath{clip}%
\pgfsetbuttcap%
\pgfsetmiterjoin%
\definecolor{currentfill}{rgb}{0.447059,0.447059,0.447059}%
\pgfsetfillcolor{currentfill}%
\pgfsetlinewidth{1.003750pt}%
\definecolor{currentstroke}{rgb}{0.266667,0.266667,0.266667}%
\pgfsetstrokecolor{currentstroke}%
\pgfsetdash{}{0pt}%
\pgfpathmoveto{\pgfqpoint{1.832481in}{1.659762in}}%
\pgfpathlineto{\pgfqpoint{2.006227in}{1.659762in}}%
\pgfpathlineto{\pgfqpoint{2.006227in}{1.659762in}}%
\pgfpathlineto{\pgfqpoint{1.832481in}{1.659762in}}%
\pgfpathlineto{\pgfqpoint{1.832481in}{1.659762in}}%
\pgfpathclose%
\pgfusepath{stroke,fill}%
\end{pgfscope}%
\begin{pgfscope}%
\pgfpathrectangle{\pgfqpoint{0.694444in}{0.416667in}}{\pgfqpoint{3.194444in}{1.416667in}}%
\pgfusepath{clip}%
\pgfsetbuttcap%
\pgfsetmiterjoin%
\definecolor{currentfill}{rgb}{0.447059,0.447059,0.447059}%
\pgfsetfillcolor{currentfill}%
\pgfsetlinewidth{1.003750pt}%
\definecolor{currentstroke}{rgb}{0.266667,0.266667,0.266667}%
\pgfsetstrokecolor{currentstroke}%
\pgfsetdash{}{0pt}%
\pgfpathmoveto{\pgfqpoint{2.080689in}{1.670074in}}%
\pgfpathlineto{\pgfqpoint{2.254435in}{1.670074in}}%
\pgfpathlineto{\pgfqpoint{2.254435in}{1.670074in}}%
\pgfpathlineto{\pgfqpoint{2.080689in}{1.670074in}}%
\pgfpathlineto{\pgfqpoint{2.080689in}{1.670074in}}%
\pgfpathclose%
\pgfusepath{stroke,fill}%
\end{pgfscope}%
\begin{pgfscope}%
\pgfpathrectangle{\pgfqpoint{0.694444in}{0.416667in}}{\pgfqpoint{3.194444in}{1.416667in}}%
\pgfusepath{clip}%
\pgfsetbuttcap%
\pgfsetmiterjoin%
\definecolor{currentfill}{rgb}{0.447059,0.447059,0.447059}%
\pgfsetfillcolor{currentfill}%
\pgfsetlinewidth{1.003750pt}%
\definecolor{currentstroke}{rgb}{0.266667,0.266667,0.266667}%
\pgfsetstrokecolor{currentstroke}%
\pgfsetdash{}{0pt}%
\pgfpathmoveto{\pgfqpoint{2.328898in}{1.631526in}}%
\pgfpathlineto{\pgfqpoint{2.502644in}{1.631526in}}%
\pgfpathlineto{\pgfqpoint{2.502644in}{1.631526in}}%
\pgfpathlineto{\pgfqpoint{2.328898in}{1.631526in}}%
\pgfpathlineto{\pgfqpoint{2.328898in}{1.631526in}}%
\pgfpathclose%
\pgfusepath{stroke,fill}%
\end{pgfscope}%
\begin{pgfscope}%
\pgfpathrectangle{\pgfqpoint{0.694444in}{0.416667in}}{\pgfqpoint{3.194444in}{1.416667in}}%
\pgfusepath{clip}%
\pgfsetbuttcap%
\pgfsetmiterjoin%
\definecolor{currentfill}{rgb}{0.447059,0.447059,0.447059}%
\pgfsetfillcolor{currentfill}%
\pgfsetlinewidth{1.003750pt}%
\definecolor{currentstroke}{rgb}{0.266667,0.266667,0.266667}%
\pgfsetstrokecolor{currentstroke}%
\pgfsetdash{}{0pt}%
\pgfpathmoveto{\pgfqpoint{2.577107in}{1.587585in}}%
\pgfpathlineto{\pgfqpoint{2.750853in}{1.587585in}}%
\pgfpathlineto{\pgfqpoint{2.750853in}{1.587585in}}%
\pgfpathlineto{\pgfqpoint{2.577107in}{1.587585in}}%
\pgfpathlineto{\pgfqpoint{2.577107in}{1.587585in}}%
\pgfpathclose%
\pgfusepath{stroke,fill}%
\end{pgfscope}%
\begin{pgfscope}%
\pgfpathrectangle{\pgfqpoint{0.694444in}{0.416667in}}{\pgfqpoint{3.194444in}{1.416667in}}%
\pgfusepath{clip}%
\pgfsetbuttcap%
\pgfsetmiterjoin%
\definecolor{currentfill}{rgb}{0.447059,0.447059,0.447059}%
\pgfsetfillcolor{currentfill}%
\pgfsetlinewidth{1.003750pt}%
\definecolor{currentstroke}{rgb}{0.266667,0.266667,0.266667}%
\pgfsetstrokecolor{currentstroke}%
\pgfsetdash{}{0pt}%
\pgfpathmoveto{\pgfqpoint{2.825315in}{1.610349in}}%
\pgfpathlineto{\pgfqpoint{2.999061in}{1.610349in}}%
\pgfpathlineto{\pgfqpoint{2.999061in}{1.610349in}}%
\pgfpathlineto{\pgfqpoint{2.825315in}{1.610349in}}%
\pgfpathlineto{\pgfqpoint{2.825315in}{1.610349in}}%
\pgfpathclose%
\pgfusepath{stroke,fill}%
\end{pgfscope}%
\begin{pgfscope}%
\pgfpathrectangle{\pgfqpoint{0.694444in}{0.416667in}}{\pgfqpoint{3.194444in}{1.416667in}}%
\pgfusepath{clip}%
\pgfsetbuttcap%
\pgfsetmiterjoin%
\definecolor{currentfill}{rgb}{0.447059,0.447059,0.447059}%
\pgfsetfillcolor{currentfill}%
\pgfsetlinewidth{1.003750pt}%
\definecolor{currentstroke}{rgb}{0.266667,0.266667,0.266667}%
\pgfsetstrokecolor{currentstroke}%
\pgfsetdash{}{0pt}%
\pgfpathmoveto{\pgfqpoint{3.073524in}{1.671027in}}%
\pgfpathlineto{\pgfqpoint{3.247270in}{1.671027in}}%
\pgfpathlineto{\pgfqpoint{3.247270in}{1.671027in}}%
\pgfpathlineto{\pgfqpoint{3.073524in}{1.671027in}}%
\pgfpathlineto{\pgfqpoint{3.073524in}{1.671027in}}%
\pgfpathclose%
\pgfusepath{stroke,fill}%
\end{pgfscope}%
\begin{pgfscope}%
\pgfpathrectangle{\pgfqpoint{0.694444in}{0.416667in}}{\pgfqpoint{3.194444in}{1.416667in}}%
\pgfusepath{clip}%
\pgfsetbuttcap%
\pgfsetmiterjoin%
\definecolor{currentfill}{rgb}{0.447059,0.447059,0.447059}%
\pgfsetfillcolor{currentfill}%
\pgfsetlinewidth{1.003750pt}%
\definecolor{currentstroke}{rgb}{0.266667,0.266667,0.266667}%
\pgfsetstrokecolor{currentstroke}%
\pgfsetdash{}{0pt}%
\pgfpathmoveto{\pgfqpoint{3.321732in}{1.612970in}}%
\pgfpathlineto{\pgfqpoint{3.495478in}{1.612970in}}%
\pgfpathlineto{\pgfqpoint{3.495478in}{1.612970in}}%
\pgfpathlineto{\pgfqpoint{3.321732in}{1.612970in}}%
\pgfpathlineto{\pgfqpoint{3.321732in}{1.612970in}}%
\pgfpathclose%
\pgfusepath{stroke,fill}%
\end{pgfscope}%
\begin{pgfscope}%
\pgfpathrectangle{\pgfqpoint{0.694444in}{0.416667in}}{\pgfqpoint{3.194444in}{1.416667in}}%
\pgfusepath{clip}%
\pgfsetbuttcap%
\pgfsetmiterjoin%
\definecolor{currentfill}{rgb}{0.447059,0.447059,0.447059}%
\pgfsetfillcolor{currentfill}%
\pgfsetlinewidth{1.003750pt}%
\definecolor{currentstroke}{rgb}{0.266667,0.266667,0.266667}%
\pgfsetstrokecolor{currentstroke}%
\pgfsetdash{}{0pt}%
\pgfpathmoveto{\pgfqpoint{3.569941in}{1.647067in}}%
\pgfpathlineto{\pgfqpoint{3.743687in}{1.647067in}}%
\pgfpathlineto{\pgfqpoint{3.743687in}{1.647067in}}%
\pgfpathlineto{\pgfqpoint{3.569941in}{1.647067in}}%
\pgfpathlineto{\pgfqpoint{3.569941in}{1.647067in}}%
\pgfpathclose%
\pgfusepath{stroke,fill}%
\end{pgfscope}%
\begin{pgfscope}%
\pgfpathrectangle{\pgfqpoint{0.694444in}{0.416667in}}{\pgfqpoint{3.194444in}{1.416667in}}%
\pgfusepath{clip}%
\pgfsetbuttcap%
\pgfsetmiterjoin%
\definecolor{currentfill}{rgb}{0.447059,0.447059,0.447059}%
\pgfsetfillcolor{currentfill}%
\pgfsetlinewidth{1.003750pt}%
\definecolor{currentstroke}{rgb}{0.266667,0.266667,0.266667}%
\pgfsetstrokecolor{currentstroke}%
\pgfsetdash{}{0pt}%
\pgfpathmoveto{\pgfqpoint{0.839646in}{1.500470in}}%
\pgfpathlineto{\pgfqpoint{1.013392in}{1.500470in}}%
\pgfpathlineto{\pgfqpoint{1.013392in}{1.500470in}}%
\pgfpathlineto{\pgfqpoint{0.839646in}{1.500470in}}%
\pgfpathlineto{\pgfqpoint{0.839646in}{1.500470in}}%
\pgfpathclose%
\pgfusepath{stroke,fill}%
\end{pgfscope}%
\begin{pgfscope}%
\pgfpathrectangle{\pgfqpoint{0.694444in}{0.416667in}}{\pgfqpoint{3.194444in}{1.416667in}}%
\pgfusepath{clip}%
\pgfsetbuttcap%
\pgfsetmiterjoin%
\definecolor{currentfill}{rgb}{0.447059,0.447059,0.447059}%
\pgfsetfillcolor{currentfill}%
\pgfsetlinewidth{1.003750pt}%
\definecolor{currentstroke}{rgb}{0.266667,0.266667,0.266667}%
\pgfsetstrokecolor{currentstroke}%
\pgfsetdash{}{0pt}%
\pgfpathmoveto{\pgfqpoint{1.087855in}{1.493113in}}%
\pgfpathlineto{\pgfqpoint{1.261601in}{1.493113in}}%
\pgfpathlineto{\pgfqpoint{1.261601in}{1.493113in}}%
\pgfpathlineto{\pgfqpoint{1.087855in}{1.493113in}}%
\pgfpathlineto{\pgfqpoint{1.087855in}{1.493113in}}%
\pgfpathclose%
\pgfusepath{stroke,fill}%
\end{pgfscope}%
\begin{pgfscope}%
\pgfpathrectangle{\pgfqpoint{0.694444in}{0.416667in}}{\pgfqpoint{3.194444in}{1.416667in}}%
\pgfusepath{clip}%
\pgfsetbuttcap%
\pgfsetmiterjoin%
\definecolor{currentfill}{rgb}{0.447059,0.447059,0.447059}%
\pgfsetfillcolor{currentfill}%
\pgfsetlinewidth{1.003750pt}%
\definecolor{currentstroke}{rgb}{0.266667,0.266667,0.266667}%
\pgfsetstrokecolor{currentstroke}%
\pgfsetdash{}{0pt}%
\pgfpathmoveto{\pgfqpoint{1.336064in}{1.471125in}}%
\pgfpathlineto{\pgfqpoint{1.509810in}{1.471125in}}%
\pgfpathlineto{\pgfqpoint{1.509810in}{1.471125in}}%
\pgfpathlineto{\pgfqpoint{1.336064in}{1.471125in}}%
\pgfpathlineto{\pgfqpoint{1.336064in}{1.471125in}}%
\pgfpathclose%
\pgfusepath{stroke,fill}%
\end{pgfscope}%
\begin{pgfscope}%
\pgfpathrectangle{\pgfqpoint{0.694444in}{0.416667in}}{\pgfqpoint{3.194444in}{1.416667in}}%
\pgfusepath{clip}%
\pgfsetbuttcap%
\pgfsetmiterjoin%
\definecolor{currentfill}{rgb}{0.447059,0.447059,0.447059}%
\pgfsetfillcolor{currentfill}%
\pgfsetlinewidth{1.003750pt}%
\definecolor{currentstroke}{rgb}{0.266667,0.266667,0.266667}%
\pgfsetstrokecolor{currentstroke}%
\pgfsetdash{}{0pt}%
\pgfpathmoveto{\pgfqpoint{1.584272in}{1.609389in}}%
\pgfpathlineto{\pgfqpoint{1.758018in}{1.609389in}}%
\pgfpathlineto{\pgfqpoint{1.758018in}{1.609389in}}%
\pgfpathlineto{\pgfqpoint{1.584272in}{1.609389in}}%
\pgfpathlineto{\pgfqpoint{1.584272in}{1.609389in}}%
\pgfpathclose%
\pgfusepath{stroke,fill}%
\end{pgfscope}%
\begin{pgfscope}%
\pgfpathrectangle{\pgfqpoint{0.694444in}{0.416667in}}{\pgfqpoint{3.194444in}{1.416667in}}%
\pgfusepath{clip}%
\pgfsetbuttcap%
\pgfsetmiterjoin%
\definecolor{currentfill}{rgb}{0.447059,0.447059,0.447059}%
\pgfsetfillcolor{currentfill}%
\pgfsetlinewidth{1.003750pt}%
\definecolor{currentstroke}{rgb}{0.266667,0.266667,0.266667}%
\pgfsetstrokecolor{currentstroke}%
\pgfsetdash{}{0pt}%
\pgfpathmoveto{\pgfqpoint{1.832481in}{1.659762in}}%
\pgfpathlineto{\pgfqpoint{2.006227in}{1.659762in}}%
\pgfpathlineto{\pgfqpoint{2.006227in}{1.659762in}}%
\pgfpathlineto{\pgfqpoint{1.832481in}{1.659762in}}%
\pgfpathlineto{\pgfqpoint{1.832481in}{1.659762in}}%
\pgfpathclose%
\pgfusepath{stroke,fill}%
\end{pgfscope}%
\begin{pgfscope}%
\pgfpathrectangle{\pgfqpoint{0.694444in}{0.416667in}}{\pgfqpoint{3.194444in}{1.416667in}}%
\pgfusepath{clip}%
\pgfsetbuttcap%
\pgfsetmiterjoin%
\definecolor{currentfill}{rgb}{0.447059,0.447059,0.447059}%
\pgfsetfillcolor{currentfill}%
\pgfsetlinewidth{1.003750pt}%
\definecolor{currentstroke}{rgb}{0.266667,0.266667,0.266667}%
\pgfsetstrokecolor{currentstroke}%
\pgfsetdash{}{0pt}%
\pgfpathmoveto{\pgfqpoint{2.080689in}{1.670074in}}%
\pgfpathlineto{\pgfqpoint{2.254435in}{1.670074in}}%
\pgfpathlineto{\pgfqpoint{2.254435in}{1.670074in}}%
\pgfpathlineto{\pgfqpoint{2.080689in}{1.670074in}}%
\pgfpathlineto{\pgfqpoint{2.080689in}{1.670074in}}%
\pgfpathclose%
\pgfusepath{stroke,fill}%
\end{pgfscope}%
\begin{pgfscope}%
\pgfpathrectangle{\pgfqpoint{0.694444in}{0.416667in}}{\pgfqpoint{3.194444in}{1.416667in}}%
\pgfusepath{clip}%
\pgfsetbuttcap%
\pgfsetmiterjoin%
\definecolor{currentfill}{rgb}{0.447059,0.447059,0.447059}%
\pgfsetfillcolor{currentfill}%
\pgfsetlinewidth{1.003750pt}%
\definecolor{currentstroke}{rgb}{0.266667,0.266667,0.266667}%
\pgfsetstrokecolor{currentstroke}%
\pgfsetdash{}{0pt}%
\pgfpathmoveto{\pgfqpoint{2.328898in}{1.631526in}}%
\pgfpathlineto{\pgfqpoint{2.502644in}{1.631526in}}%
\pgfpathlineto{\pgfqpoint{2.502644in}{1.631526in}}%
\pgfpathlineto{\pgfqpoint{2.328898in}{1.631526in}}%
\pgfpathlineto{\pgfqpoint{2.328898in}{1.631526in}}%
\pgfpathclose%
\pgfusepath{stroke,fill}%
\end{pgfscope}%
\begin{pgfscope}%
\pgfpathrectangle{\pgfqpoint{0.694444in}{0.416667in}}{\pgfqpoint{3.194444in}{1.416667in}}%
\pgfusepath{clip}%
\pgfsetbuttcap%
\pgfsetmiterjoin%
\definecolor{currentfill}{rgb}{0.447059,0.447059,0.447059}%
\pgfsetfillcolor{currentfill}%
\pgfsetlinewidth{1.003750pt}%
\definecolor{currentstroke}{rgb}{0.266667,0.266667,0.266667}%
\pgfsetstrokecolor{currentstroke}%
\pgfsetdash{}{0pt}%
\pgfpathmoveto{\pgfqpoint{2.577107in}{1.587585in}}%
\pgfpathlineto{\pgfqpoint{2.750853in}{1.587585in}}%
\pgfpathlineto{\pgfqpoint{2.750853in}{1.587585in}}%
\pgfpathlineto{\pgfqpoint{2.577107in}{1.587585in}}%
\pgfpathlineto{\pgfqpoint{2.577107in}{1.587585in}}%
\pgfpathclose%
\pgfusepath{stroke,fill}%
\end{pgfscope}%
\begin{pgfscope}%
\pgfpathrectangle{\pgfqpoint{0.694444in}{0.416667in}}{\pgfqpoint{3.194444in}{1.416667in}}%
\pgfusepath{clip}%
\pgfsetbuttcap%
\pgfsetmiterjoin%
\definecolor{currentfill}{rgb}{0.447059,0.447059,0.447059}%
\pgfsetfillcolor{currentfill}%
\pgfsetlinewidth{1.003750pt}%
\definecolor{currentstroke}{rgb}{0.266667,0.266667,0.266667}%
\pgfsetstrokecolor{currentstroke}%
\pgfsetdash{}{0pt}%
\pgfpathmoveto{\pgfqpoint{2.825315in}{1.610349in}}%
\pgfpathlineto{\pgfqpoint{2.999061in}{1.610349in}}%
\pgfpathlineto{\pgfqpoint{2.999061in}{1.610349in}}%
\pgfpathlineto{\pgfqpoint{2.825315in}{1.610349in}}%
\pgfpathlineto{\pgfqpoint{2.825315in}{1.610349in}}%
\pgfpathclose%
\pgfusepath{stroke,fill}%
\end{pgfscope}%
\begin{pgfscope}%
\pgfpathrectangle{\pgfqpoint{0.694444in}{0.416667in}}{\pgfqpoint{3.194444in}{1.416667in}}%
\pgfusepath{clip}%
\pgfsetbuttcap%
\pgfsetmiterjoin%
\definecolor{currentfill}{rgb}{0.447059,0.447059,0.447059}%
\pgfsetfillcolor{currentfill}%
\pgfsetlinewidth{1.003750pt}%
\definecolor{currentstroke}{rgb}{0.266667,0.266667,0.266667}%
\pgfsetstrokecolor{currentstroke}%
\pgfsetdash{}{0pt}%
\pgfpathmoveto{\pgfqpoint{3.073524in}{1.671027in}}%
\pgfpathlineto{\pgfqpoint{3.247270in}{1.671027in}}%
\pgfpathlineto{\pgfqpoint{3.247270in}{1.671027in}}%
\pgfpathlineto{\pgfqpoint{3.073524in}{1.671027in}}%
\pgfpathlineto{\pgfqpoint{3.073524in}{1.671027in}}%
\pgfpathclose%
\pgfusepath{stroke,fill}%
\end{pgfscope}%
\begin{pgfscope}%
\pgfpathrectangle{\pgfqpoint{0.694444in}{0.416667in}}{\pgfqpoint{3.194444in}{1.416667in}}%
\pgfusepath{clip}%
\pgfsetbuttcap%
\pgfsetmiterjoin%
\definecolor{currentfill}{rgb}{0.447059,0.447059,0.447059}%
\pgfsetfillcolor{currentfill}%
\pgfsetlinewidth{1.003750pt}%
\definecolor{currentstroke}{rgb}{0.266667,0.266667,0.266667}%
\pgfsetstrokecolor{currentstroke}%
\pgfsetdash{}{0pt}%
\pgfpathmoveto{\pgfqpoint{3.321732in}{1.612970in}}%
\pgfpathlineto{\pgfqpoint{3.495478in}{1.612970in}}%
\pgfpathlineto{\pgfqpoint{3.495478in}{1.612970in}}%
\pgfpathlineto{\pgfqpoint{3.321732in}{1.612970in}}%
\pgfpathlineto{\pgfqpoint{3.321732in}{1.612970in}}%
\pgfpathclose%
\pgfusepath{stroke,fill}%
\end{pgfscope}%
\begin{pgfscope}%
\pgfpathrectangle{\pgfqpoint{0.694444in}{0.416667in}}{\pgfqpoint{3.194444in}{1.416667in}}%
\pgfusepath{clip}%
\pgfsetbuttcap%
\pgfsetmiterjoin%
\definecolor{currentfill}{rgb}{0.447059,0.447059,0.447059}%
\pgfsetfillcolor{currentfill}%
\pgfsetlinewidth{1.003750pt}%
\definecolor{currentstroke}{rgb}{0.266667,0.266667,0.266667}%
\pgfsetstrokecolor{currentstroke}%
\pgfsetdash{}{0pt}%
\pgfpathmoveto{\pgfqpoint{3.569941in}{1.647067in}}%
\pgfpathlineto{\pgfqpoint{3.743687in}{1.647067in}}%
\pgfpathlineto{\pgfqpoint{3.743687in}{1.647067in}}%
\pgfpathlineto{\pgfqpoint{3.569941in}{1.647067in}}%
\pgfpathlineto{\pgfqpoint{3.569941in}{1.647067in}}%
\pgfpathclose%
\pgfusepath{stroke,fill}%
\end{pgfscope}%
\begin{pgfscope}%
\pgfpathrectangle{\pgfqpoint{0.694444in}{0.416667in}}{\pgfqpoint{3.194444in}{1.416667in}}%
\pgfusepath{clip}%
\pgfsetbuttcap%
\pgfsetmiterjoin%
\definecolor{currentfill}{rgb}{0.447059,0.447059,0.447059}%
\pgfsetfillcolor{currentfill}%
\pgfsetlinewidth{1.003750pt}%
\definecolor{currentstroke}{rgb}{0.266667,0.266667,0.266667}%
\pgfsetstrokecolor{currentstroke}%
\pgfsetdash{}{0pt}%
\pgfpathmoveto{\pgfqpoint{0.839646in}{1.500470in}}%
\pgfpathlineto{\pgfqpoint{1.013392in}{1.500470in}}%
\pgfpathlineto{\pgfqpoint{1.013392in}{1.500470in}}%
\pgfpathlineto{\pgfqpoint{0.839646in}{1.500470in}}%
\pgfpathlineto{\pgfqpoint{0.839646in}{1.500470in}}%
\pgfpathclose%
\pgfusepath{stroke,fill}%
\end{pgfscope}%
\begin{pgfscope}%
\pgfpathrectangle{\pgfqpoint{0.694444in}{0.416667in}}{\pgfqpoint{3.194444in}{1.416667in}}%
\pgfusepath{clip}%
\pgfsetbuttcap%
\pgfsetmiterjoin%
\definecolor{currentfill}{rgb}{0.447059,0.447059,0.447059}%
\pgfsetfillcolor{currentfill}%
\pgfsetlinewidth{1.003750pt}%
\definecolor{currentstroke}{rgb}{0.266667,0.266667,0.266667}%
\pgfsetstrokecolor{currentstroke}%
\pgfsetdash{}{0pt}%
\pgfpathmoveto{\pgfqpoint{1.087855in}{1.493113in}}%
\pgfpathlineto{\pgfqpoint{1.261601in}{1.493113in}}%
\pgfpathlineto{\pgfqpoint{1.261601in}{1.493113in}}%
\pgfpathlineto{\pgfqpoint{1.087855in}{1.493113in}}%
\pgfpathlineto{\pgfqpoint{1.087855in}{1.493113in}}%
\pgfpathclose%
\pgfusepath{stroke,fill}%
\end{pgfscope}%
\begin{pgfscope}%
\pgfpathrectangle{\pgfqpoint{0.694444in}{0.416667in}}{\pgfqpoint{3.194444in}{1.416667in}}%
\pgfusepath{clip}%
\pgfsetbuttcap%
\pgfsetmiterjoin%
\definecolor{currentfill}{rgb}{0.447059,0.447059,0.447059}%
\pgfsetfillcolor{currentfill}%
\pgfsetlinewidth{1.003750pt}%
\definecolor{currentstroke}{rgb}{0.266667,0.266667,0.266667}%
\pgfsetstrokecolor{currentstroke}%
\pgfsetdash{}{0pt}%
\pgfpathmoveto{\pgfqpoint{1.336064in}{1.471125in}}%
\pgfpathlineto{\pgfqpoint{1.509810in}{1.471125in}}%
\pgfpathlineto{\pgfqpoint{1.509810in}{1.471125in}}%
\pgfpathlineto{\pgfqpoint{1.336064in}{1.471125in}}%
\pgfpathlineto{\pgfqpoint{1.336064in}{1.471125in}}%
\pgfpathclose%
\pgfusepath{stroke,fill}%
\end{pgfscope}%
\begin{pgfscope}%
\pgfpathrectangle{\pgfqpoint{0.694444in}{0.416667in}}{\pgfqpoint{3.194444in}{1.416667in}}%
\pgfusepath{clip}%
\pgfsetbuttcap%
\pgfsetmiterjoin%
\definecolor{currentfill}{rgb}{0.447059,0.447059,0.447059}%
\pgfsetfillcolor{currentfill}%
\pgfsetlinewidth{1.003750pt}%
\definecolor{currentstroke}{rgb}{0.266667,0.266667,0.266667}%
\pgfsetstrokecolor{currentstroke}%
\pgfsetdash{}{0pt}%
\pgfpathmoveto{\pgfqpoint{1.584272in}{1.609389in}}%
\pgfpathlineto{\pgfqpoint{1.758018in}{1.609389in}}%
\pgfpathlineto{\pgfqpoint{1.758018in}{1.609389in}}%
\pgfpathlineto{\pgfqpoint{1.584272in}{1.609389in}}%
\pgfpathlineto{\pgfqpoint{1.584272in}{1.609389in}}%
\pgfpathclose%
\pgfusepath{stroke,fill}%
\end{pgfscope}%
\begin{pgfscope}%
\pgfpathrectangle{\pgfqpoint{0.694444in}{0.416667in}}{\pgfqpoint{3.194444in}{1.416667in}}%
\pgfusepath{clip}%
\pgfsetbuttcap%
\pgfsetmiterjoin%
\definecolor{currentfill}{rgb}{0.447059,0.447059,0.447059}%
\pgfsetfillcolor{currentfill}%
\pgfsetlinewidth{1.003750pt}%
\definecolor{currentstroke}{rgb}{0.266667,0.266667,0.266667}%
\pgfsetstrokecolor{currentstroke}%
\pgfsetdash{}{0pt}%
\pgfpathmoveto{\pgfqpoint{1.832481in}{1.659762in}}%
\pgfpathlineto{\pgfqpoint{2.006227in}{1.659762in}}%
\pgfpathlineto{\pgfqpoint{2.006227in}{1.659762in}}%
\pgfpathlineto{\pgfqpoint{1.832481in}{1.659762in}}%
\pgfpathlineto{\pgfqpoint{1.832481in}{1.659762in}}%
\pgfpathclose%
\pgfusepath{stroke,fill}%
\end{pgfscope}%
\begin{pgfscope}%
\pgfpathrectangle{\pgfqpoint{0.694444in}{0.416667in}}{\pgfqpoint{3.194444in}{1.416667in}}%
\pgfusepath{clip}%
\pgfsetbuttcap%
\pgfsetmiterjoin%
\definecolor{currentfill}{rgb}{0.447059,0.447059,0.447059}%
\pgfsetfillcolor{currentfill}%
\pgfsetlinewidth{1.003750pt}%
\definecolor{currentstroke}{rgb}{0.266667,0.266667,0.266667}%
\pgfsetstrokecolor{currentstroke}%
\pgfsetdash{}{0pt}%
\pgfpathmoveto{\pgfqpoint{2.080689in}{1.670074in}}%
\pgfpathlineto{\pgfqpoint{2.254435in}{1.670074in}}%
\pgfpathlineto{\pgfqpoint{2.254435in}{1.670074in}}%
\pgfpathlineto{\pgfqpoint{2.080689in}{1.670074in}}%
\pgfpathlineto{\pgfqpoint{2.080689in}{1.670074in}}%
\pgfpathclose%
\pgfusepath{stroke,fill}%
\end{pgfscope}%
\begin{pgfscope}%
\pgfpathrectangle{\pgfqpoint{0.694444in}{0.416667in}}{\pgfqpoint{3.194444in}{1.416667in}}%
\pgfusepath{clip}%
\pgfsetbuttcap%
\pgfsetmiterjoin%
\definecolor{currentfill}{rgb}{0.447059,0.447059,0.447059}%
\pgfsetfillcolor{currentfill}%
\pgfsetlinewidth{1.003750pt}%
\definecolor{currentstroke}{rgb}{0.266667,0.266667,0.266667}%
\pgfsetstrokecolor{currentstroke}%
\pgfsetdash{}{0pt}%
\pgfpathmoveto{\pgfqpoint{2.328898in}{1.631526in}}%
\pgfpathlineto{\pgfqpoint{2.502644in}{1.631526in}}%
\pgfpathlineto{\pgfqpoint{2.502644in}{1.631526in}}%
\pgfpathlineto{\pgfqpoint{2.328898in}{1.631526in}}%
\pgfpathlineto{\pgfqpoint{2.328898in}{1.631526in}}%
\pgfpathclose%
\pgfusepath{stroke,fill}%
\end{pgfscope}%
\begin{pgfscope}%
\pgfpathrectangle{\pgfqpoint{0.694444in}{0.416667in}}{\pgfqpoint{3.194444in}{1.416667in}}%
\pgfusepath{clip}%
\pgfsetbuttcap%
\pgfsetmiterjoin%
\definecolor{currentfill}{rgb}{0.447059,0.447059,0.447059}%
\pgfsetfillcolor{currentfill}%
\pgfsetlinewidth{1.003750pt}%
\definecolor{currentstroke}{rgb}{0.266667,0.266667,0.266667}%
\pgfsetstrokecolor{currentstroke}%
\pgfsetdash{}{0pt}%
\pgfpathmoveto{\pgfqpoint{2.577107in}{1.587585in}}%
\pgfpathlineto{\pgfqpoint{2.750853in}{1.587585in}}%
\pgfpathlineto{\pgfqpoint{2.750853in}{1.587585in}}%
\pgfpathlineto{\pgfqpoint{2.577107in}{1.587585in}}%
\pgfpathlineto{\pgfqpoint{2.577107in}{1.587585in}}%
\pgfpathclose%
\pgfusepath{stroke,fill}%
\end{pgfscope}%
\begin{pgfscope}%
\pgfpathrectangle{\pgfqpoint{0.694444in}{0.416667in}}{\pgfqpoint{3.194444in}{1.416667in}}%
\pgfusepath{clip}%
\pgfsetbuttcap%
\pgfsetmiterjoin%
\definecolor{currentfill}{rgb}{0.447059,0.447059,0.447059}%
\pgfsetfillcolor{currentfill}%
\pgfsetlinewidth{1.003750pt}%
\definecolor{currentstroke}{rgb}{0.266667,0.266667,0.266667}%
\pgfsetstrokecolor{currentstroke}%
\pgfsetdash{}{0pt}%
\pgfpathmoveto{\pgfqpoint{2.825315in}{1.610349in}}%
\pgfpathlineto{\pgfqpoint{2.999061in}{1.610349in}}%
\pgfpathlineto{\pgfqpoint{2.999061in}{1.610349in}}%
\pgfpathlineto{\pgfqpoint{2.825315in}{1.610349in}}%
\pgfpathlineto{\pgfqpoint{2.825315in}{1.610349in}}%
\pgfpathclose%
\pgfusepath{stroke,fill}%
\end{pgfscope}%
\begin{pgfscope}%
\pgfpathrectangle{\pgfqpoint{0.694444in}{0.416667in}}{\pgfqpoint{3.194444in}{1.416667in}}%
\pgfusepath{clip}%
\pgfsetbuttcap%
\pgfsetmiterjoin%
\definecolor{currentfill}{rgb}{0.447059,0.447059,0.447059}%
\pgfsetfillcolor{currentfill}%
\pgfsetlinewidth{1.003750pt}%
\definecolor{currentstroke}{rgb}{0.266667,0.266667,0.266667}%
\pgfsetstrokecolor{currentstroke}%
\pgfsetdash{}{0pt}%
\pgfpathmoveto{\pgfqpoint{3.073524in}{1.671027in}}%
\pgfpathlineto{\pgfqpoint{3.247270in}{1.671027in}}%
\pgfpathlineto{\pgfqpoint{3.247270in}{1.671027in}}%
\pgfpathlineto{\pgfqpoint{3.073524in}{1.671027in}}%
\pgfpathlineto{\pgfqpoint{3.073524in}{1.671027in}}%
\pgfpathclose%
\pgfusepath{stroke,fill}%
\end{pgfscope}%
\begin{pgfscope}%
\pgfpathrectangle{\pgfqpoint{0.694444in}{0.416667in}}{\pgfqpoint{3.194444in}{1.416667in}}%
\pgfusepath{clip}%
\pgfsetbuttcap%
\pgfsetmiterjoin%
\definecolor{currentfill}{rgb}{0.447059,0.447059,0.447059}%
\pgfsetfillcolor{currentfill}%
\pgfsetlinewidth{1.003750pt}%
\definecolor{currentstroke}{rgb}{0.266667,0.266667,0.266667}%
\pgfsetstrokecolor{currentstroke}%
\pgfsetdash{}{0pt}%
\pgfpathmoveto{\pgfqpoint{3.321732in}{1.612970in}}%
\pgfpathlineto{\pgfqpoint{3.495478in}{1.612970in}}%
\pgfpathlineto{\pgfqpoint{3.495478in}{1.612970in}}%
\pgfpathlineto{\pgfqpoint{3.321732in}{1.612970in}}%
\pgfpathlineto{\pgfqpoint{3.321732in}{1.612970in}}%
\pgfpathclose%
\pgfusepath{stroke,fill}%
\end{pgfscope}%
\begin{pgfscope}%
\pgfpathrectangle{\pgfqpoint{0.694444in}{0.416667in}}{\pgfqpoint{3.194444in}{1.416667in}}%
\pgfusepath{clip}%
\pgfsetbuttcap%
\pgfsetmiterjoin%
\definecolor{currentfill}{rgb}{0.447059,0.447059,0.447059}%
\pgfsetfillcolor{currentfill}%
\pgfsetlinewidth{1.003750pt}%
\definecolor{currentstroke}{rgb}{0.266667,0.266667,0.266667}%
\pgfsetstrokecolor{currentstroke}%
\pgfsetdash{}{0pt}%
\pgfpathmoveto{\pgfqpoint{3.569941in}{1.647067in}}%
\pgfpathlineto{\pgfqpoint{3.743687in}{1.647067in}}%
\pgfpathlineto{\pgfqpoint{3.743687in}{1.647067in}}%
\pgfpathlineto{\pgfqpoint{3.569941in}{1.647067in}}%
\pgfpathlineto{\pgfqpoint{3.569941in}{1.647067in}}%
\pgfpathclose%
\pgfusepath{stroke,fill}%
\end{pgfscope}%
\begin{pgfscope}%
\pgfpathrectangle{\pgfqpoint{0.694444in}{0.416667in}}{\pgfqpoint{3.194444in}{1.416667in}}%
\pgfusepath{clip}%
\pgfsetbuttcap%
\pgfsetmiterjoin%
\definecolor{currentfill}{rgb}{0.447059,0.447059,0.447059}%
\pgfsetfillcolor{currentfill}%
\pgfsetlinewidth{1.003750pt}%
\definecolor{currentstroke}{rgb}{0.266667,0.266667,0.266667}%
\pgfsetstrokecolor{currentstroke}%
\pgfsetdash{}{0pt}%
\pgfpathmoveto{\pgfqpoint{0.839646in}{1.500470in}}%
\pgfpathlineto{\pgfqpoint{1.013392in}{1.500470in}}%
\pgfpathlineto{\pgfqpoint{1.013392in}{1.500470in}}%
\pgfpathlineto{\pgfqpoint{0.839646in}{1.500470in}}%
\pgfpathlineto{\pgfqpoint{0.839646in}{1.500470in}}%
\pgfpathclose%
\pgfusepath{stroke,fill}%
\end{pgfscope}%
\begin{pgfscope}%
\pgfpathrectangle{\pgfqpoint{0.694444in}{0.416667in}}{\pgfqpoint{3.194444in}{1.416667in}}%
\pgfusepath{clip}%
\pgfsetbuttcap%
\pgfsetmiterjoin%
\definecolor{currentfill}{rgb}{0.447059,0.447059,0.447059}%
\pgfsetfillcolor{currentfill}%
\pgfsetlinewidth{1.003750pt}%
\definecolor{currentstroke}{rgb}{0.266667,0.266667,0.266667}%
\pgfsetstrokecolor{currentstroke}%
\pgfsetdash{}{0pt}%
\pgfpathmoveto{\pgfqpoint{1.087855in}{1.493113in}}%
\pgfpathlineto{\pgfqpoint{1.261601in}{1.493113in}}%
\pgfpathlineto{\pgfqpoint{1.261601in}{1.493113in}}%
\pgfpathlineto{\pgfqpoint{1.087855in}{1.493113in}}%
\pgfpathlineto{\pgfqpoint{1.087855in}{1.493113in}}%
\pgfpathclose%
\pgfusepath{stroke,fill}%
\end{pgfscope}%
\begin{pgfscope}%
\pgfpathrectangle{\pgfqpoint{0.694444in}{0.416667in}}{\pgfqpoint{3.194444in}{1.416667in}}%
\pgfusepath{clip}%
\pgfsetbuttcap%
\pgfsetmiterjoin%
\definecolor{currentfill}{rgb}{0.447059,0.447059,0.447059}%
\pgfsetfillcolor{currentfill}%
\pgfsetlinewidth{1.003750pt}%
\definecolor{currentstroke}{rgb}{0.266667,0.266667,0.266667}%
\pgfsetstrokecolor{currentstroke}%
\pgfsetdash{}{0pt}%
\pgfpathmoveto{\pgfqpoint{1.336064in}{1.471125in}}%
\pgfpathlineto{\pgfqpoint{1.509810in}{1.471125in}}%
\pgfpathlineto{\pgfqpoint{1.509810in}{1.471125in}}%
\pgfpathlineto{\pgfqpoint{1.336064in}{1.471125in}}%
\pgfpathlineto{\pgfqpoint{1.336064in}{1.471125in}}%
\pgfpathclose%
\pgfusepath{stroke,fill}%
\end{pgfscope}%
\begin{pgfscope}%
\pgfpathrectangle{\pgfqpoint{0.694444in}{0.416667in}}{\pgfqpoint{3.194444in}{1.416667in}}%
\pgfusepath{clip}%
\pgfsetbuttcap%
\pgfsetmiterjoin%
\definecolor{currentfill}{rgb}{0.447059,0.447059,0.447059}%
\pgfsetfillcolor{currentfill}%
\pgfsetlinewidth{1.003750pt}%
\definecolor{currentstroke}{rgb}{0.266667,0.266667,0.266667}%
\pgfsetstrokecolor{currentstroke}%
\pgfsetdash{}{0pt}%
\pgfpathmoveto{\pgfqpoint{1.584272in}{1.609389in}}%
\pgfpathlineto{\pgfqpoint{1.758018in}{1.609389in}}%
\pgfpathlineto{\pgfqpoint{1.758018in}{1.609389in}}%
\pgfpathlineto{\pgfqpoint{1.584272in}{1.609389in}}%
\pgfpathlineto{\pgfqpoint{1.584272in}{1.609389in}}%
\pgfpathclose%
\pgfusepath{stroke,fill}%
\end{pgfscope}%
\begin{pgfscope}%
\pgfpathrectangle{\pgfqpoint{0.694444in}{0.416667in}}{\pgfqpoint{3.194444in}{1.416667in}}%
\pgfusepath{clip}%
\pgfsetbuttcap%
\pgfsetmiterjoin%
\definecolor{currentfill}{rgb}{0.447059,0.447059,0.447059}%
\pgfsetfillcolor{currentfill}%
\pgfsetlinewidth{1.003750pt}%
\definecolor{currentstroke}{rgb}{0.266667,0.266667,0.266667}%
\pgfsetstrokecolor{currentstroke}%
\pgfsetdash{}{0pt}%
\pgfpathmoveto{\pgfqpoint{1.832481in}{1.659762in}}%
\pgfpathlineto{\pgfqpoint{2.006227in}{1.659762in}}%
\pgfpathlineto{\pgfqpoint{2.006227in}{1.659762in}}%
\pgfpathlineto{\pgfqpoint{1.832481in}{1.659762in}}%
\pgfpathlineto{\pgfqpoint{1.832481in}{1.659762in}}%
\pgfpathclose%
\pgfusepath{stroke,fill}%
\end{pgfscope}%
\begin{pgfscope}%
\pgfpathrectangle{\pgfqpoint{0.694444in}{0.416667in}}{\pgfqpoint{3.194444in}{1.416667in}}%
\pgfusepath{clip}%
\pgfsetbuttcap%
\pgfsetmiterjoin%
\definecolor{currentfill}{rgb}{0.447059,0.447059,0.447059}%
\pgfsetfillcolor{currentfill}%
\pgfsetlinewidth{1.003750pt}%
\definecolor{currentstroke}{rgb}{0.266667,0.266667,0.266667}%
\pgfsetstrokecolor{currentstroke}%
\pgfsetdash{}{0pt}%
\pgfpathmoveto{\pgfqpoint{2.080689in}{1.670074in}}%
\pgfpathlineto{\pgfqpoint{2.254435in}{1.670074in}}%
\pgfpathlineto{\pgfqpoint{2.254435in}{1.670074in}}%
\pgfpathlineto{\pgfqpoint{2.080689in}{1.670074in}}%
\pgfpathlineto{\pgfqpoint{2.080689in}{1.670074in}}%
\pgfpathclose%
\pgfusepath{stroke,fill}%
\end{pgfscope}%
\begin{pgfscope}%
\pgfpathrectangle{\pgfqpoint{0.694444in}{0.416667in}}{\pgfqpoint{3.194444in}{1.416667in}}%
\pgfusepath{clip}%
\pgfsetbuttcap%
\pgfsetmiterjoin%
\definecolor{currentfill}{rgb}{0.447059,0.447059,0.447059}%
\pgfsetfillcolor{currentfill}%
\pgfsetlinewidth{1.003750pt}%
\definecolor{currentstroke}{rgb}{0.266667,0.266667,0.266667}%
\pgfsetstrokecolor{currentstroke}%
\pgfsetdash{}{0pt}%
\pgfpathmoveto{\pgfqpoint{2.328898in}{1.631526in}}%
\pgfpathlineto{\pgfqpoint{2.502644in}{1.631526in}}%
\pgfpathlineto{\pgfqpoint{2.502644in}{1.631526in}}%
\pgfpathlineto{\pgfqpoint{2.328898in}{1.631526in}}%
\pgfpathlineto{\pgfqpoint{2.328898in}{1.631526in}}%
\pgfpathclose%
\pgfusepath{stroke,fill}%
\end{pgfscope}%
\begin{pgfscope}%
\pgfpathrectangle{\pgfqpoint{0.694444in}{0.416667in}}{\pgfqpoint{3.194444in}{1.416667in}}%
\pgfusepath{clip}%
\pgfsetbuttcap%
\pgfsetmiterjoin%
\definecolor{currentfill}{rgb}{0.447059,0.447059,0.447059}%
\pgfsetfillcolor{currentfill}%
\pgfsetlinewidth{1.003750pt}%
\definecolor{currentstroke}{rgb}{0.266667,0.266667,0.266667}%
\pgfsetstrokecolor{currentstroke}%
\pgfsetdash{}{0pt}%
\pgfpathmoveto{\pgfqpoint{2.577107in}{1.587585in}}%
\pgfpathlineto{\pgfqpoint{2.750853in}{1.587585in}}%
\pgfpathlineto{\pgfqpoint{2.750853in}{1.587585in}}%
\pgfpathlineto{\pgfqpoint{2.577107in}{1.587585in}}%
\pgfpathlineto{\pgfqpoint{2.577107in}{1.587585in}}%
\pgfpathclose%
\pgfusepath{stroke,fill}%
\end{pgfscope}%
\begin{pgfscope}%
\pgfpathrectangle{\pgfqpoint{0.694444in}{0.416667in}}{\pgfqpoint{3.194444in}{1.416667in}}%
\pgfusepath{clip}%
\pgfsetbuttcap%
\pgfsetmiterjoin%
\definecolor{currentfill}{rgb}{0.447059,0.447059,0.447059}%
\pgfsetfillcolor{currentfill}%
\pgfsetlinewidth{1.003750pt}%
\definecolor{currentstroke}{rgb}{0.266667,0.266667,0.266667}%
\pgfsetstrokecolor{currentstroke}%
\pgfsetdash{}{0pt}%
\pgfpathmoveto{\pgfqpoint{2.825315in}{1.610349in}}%
\pgfpathlineto{\pgfqpoint{2.999061in}{1.610349in}}%
\pgfpathlineto{\pgfqpoint{2.999061in}{1.610349in}}%
\pgfpathlineto{\pgfqpoint{2.825315in}{1.610349in}}%
\pgfpathlineto{\pgfqpoint{2.825315in}{1.610349in}}%
\pgfpathclose%
\pgfusepath{stroke,fill}%
\end{pgfscope}%
\begin{pgfscope}%
\pgfpathrectangle{\pgfqpoint{0.694444in}{0.416667in}}{\pgfqpoint{3.194444in}{1.416667in}}%
\pgfusepath{clip}%
\pgfsetbuttcap%
\pgfsetmiterjoin%
\definecolor{currentfill}{rgb}{0.447059,0.447059,0.447059}%
\pgfsetfillcolor{currentfill}%
\pgfsetlinewidth{1.003750pt}%
\definecolor{currentstroke}{rgb}{0.266667,0.266667,0.266667}%
\pgfsetstrokecolor{currentstroke}%
\pgfsetdash{}{0pt}%
\pgfpathmoveto{\pgfqpoint{3.073524in}{1.671027in}}%
\pgfpathlineto{\pgfqpoint{3.247270in}{1.671027in}}%
\pgfpathlineto{\pgfqpoint{3.247270in}{1.671027in}}%
\pgfpathlineto{\pgfqpoint{3.073524in}{1.671027in}}%
\pgfpathlineto{\pgfqpoint{3.073524in}{1.671027in}}%
\pgfpathclose%
\pgfusepath{stroke,fill}%
\end{pgfscope}%
\begin{pgfscope}%
\pgfpathrectangle{\pgfqpoint{0.694444in}{0.416667in}}{\pgfqpoint{3.194444in}{1.416667in}}%
\pgfusepath{clip}%
\pgfsetbuttcap%
\pgfsetmiterjoin%
\definecolor{currentfill}{rgb}{0.447059,0.447059,0.447059}%
\pgfsetfillcolor{currentfill}%
\pgfsetlinewidth{1.003750pt}%
\definecolor{currentstroke}{rgb}{0.266667,0.266667,0.266667}%
\pgfsetstrokecolor{currentstroke}%
\pgfsetdash{}{0pt}%
\pgfpathmoveto{\pgfqpoint{3.321732in}{1.612970in}}%
\pgfpathlineto{\pgfqpoint{3.495478in}{1.612970in}}%
\pgfpathlineto{\pgfqpoint{3.495478in}{1.612970in}}%
\pgfpathlineto{\pgfqpoint{3.321732in}{1.612970in}}%
\pgfpathlineto{\pgfqpoint{3.321732in}{1.612970in}}%
\pgfpathclose%
\pgfusepath{stroke,fill}%
\end{pgfscope}%
\begin{pgfscope}%
\pgfpathrectangle{\pgfqpoint{0.694444in}{0.416667in}}{\pgfqpoint{3.194444in}{1.416667in}}%
\pgfusepath{clip}%
\pgfsetbuttcap%
\pgfsetmiterjoin%
\definecolor{currentfill}{rgb}{0.447059,0.447059,0.447059}%
\pgfsetfillcolor{currentfill}%
\pgfsetlinewidth{1.003750pt}%
\definecolor{currentstroke}{rgb}{0.266667,0.266667,0.266667}%
\pgfsetstrokecolor{currentstroke}%
\pgfsetdash{}{0pt}%
\pgfpathmoveto{\pgfqpoint{3.569941in}{1.647067in}}%
\pgfpathlineto{\pgfqpoint{3.743687in}{1.647067in}}%
\pgfpathlineto{\pgfqpoint{3.743687in}{1.647067in}}%
\pgfpathlineto{\pgfqpoint{3.569941in}{1.647067in}}%
\pgfpathlineto{\pgfqpoint{3.569941in}{1.647067in}}%
\pgfpathclose%
\pgfusepath{stroke,fill}%
\end{pgfscope}%
\begin{pgfscope}%
\pgfpathrectangle{\pgfqpoint{0.694444in}{0.416667in}}{\pgfqpoint{3.194444in}{1.416667in}}%
\pgfusepath{clip}%
\pgfsetbuttcap%
\pgfsetmiterjoin%
\definecolor{currentfill}{rgb}{0.447059,0.447059,0.447059}%
\pgfsetfillcolor{currentfill}%
\pgfsetlinewidth{1.003750pt}%
\definecolor{currentstroke}{rgb}{0.266667,0.266667,0.266667}%
\pgfsetstrokecolor{currentstroke}%
\pgfsetdash{}{0pt}%
\pgfpathmoveto{\pgfqpoint{0.839646in}{1.500470in}}%
\pgfpathlineto{\pgfqpoint{1.013392in}{1.500470in}}%
\pgfpathlineto{\pgfqpoint{1.013392in}{1.500471in}}%
\pgfpathlineto{\pgfqpoint{0.839646in}{1.500471in}}%
\pgfpathlineto{\pgfqpoint{0.839646in}{1.500470in}}%
\pgfpathclose%
\pgfusepath{stroke,fill}%
\end{pgfscope}%
\begin{pgfscope}%
\pgfpathrectangle{\pgfqpoint{0.694444in}{0.416667in}}{\pgfqpoint{3.194444in}{1.416667in}}%
\pgfusepath{clip}%
\pgfsetbuttcap%
\pgfsetmiterjoin%
\definecolor{currentfill}{rgb}{0.447059,0.447059,0.447059}%
\pgfsetfillcolor{currentfill}%
\pgfsetlinewidth{1.003750pt}%
\definecolor{currentstroke}{rgb}{0.266667,0.266667,0.266667}%
\pgfsetstrokecolor{currentstroke}%
\pgfsetdash{}{0pt}%
\pgfpathmoveto{\pgfqpoint{1.087855in}{1.493113in}}%
\pgfpathlineto{\pgfqpoint{1.261601in}{1.493113in}}%
\pgfpathlineto{\pgfqpoint{1.261601in}{1.493114in}}%
\pgfpathlineto{\pgfqpoint{1.087855in}{1.493114in}}%
\pgfpathlineto{\pgfqpoint{1.087855in}{1.493113in}}%
\pgfpathclose%
\pgfusepath{stroke,fill}%
\end{pgfscope}%
\begin{pgfscope}%
\pgfpathrectangle{\pgfqpoint{0.694444in}{0.416667in}}{\pgfqpoint{3.194444in}{1.416667in}}%
\pgfusepath{clip}%
\pgfsetbuttcap%
\pgfsetmiterjoin%
\definecolor{currentfill}{rgb}{0.447059,0.447059,0.447059}%
\pgfsetfillcolor{currentfill}%
\pgfsetlinewidth{1.003750pt}%
\definecolor{currentstroke}{rgb}{0.266667,0.266667,0.266667}%
\pgfsetstrokecolor{currentstroke}%
\pgfsetdash{}{0pt}%
\pgfpathmoveto{\pgfqpoint{1.336064in}{1.471125in}}%
\pgfpathlineto{\pgfqpoint{1.509810in}{1.471125in}}%
\pgfpathlineto{\pgfqpoint{1.509810in}{1.471126in}}%
\pgfpathlineto{\pgfqpoint{1.336064in}{1.471126in}}%
\pgfpathlineto{\pgfqpoint{1.336064in}{1.471125in}}%
\pgfpathclose%
\pgfusepath{stroke,fill}%
\end{pgfscope}%
\begin{pgfscope}%
\pgfpathrectangle{\pgfqpoint{0.694444in}{0.416667in}}{\pgfqpoint{3.194444in}{1.416667in}}%
\pgfusepath{clip}%
\pgfsetbuttcap%
\pgfsetmiterjoin%
\definecolor{currentfill}{rgb}{0.447059,0.447059,0.447059}%
\pgfsetfillcolor{currentfill}%
\pgfsetlinewidth{1.003750pt}%
\definecolor{currentstroke}{rgb}{0.266667,0.266667,0.266667}%
\pgfsetstrokecolor{currentstroke}%
\pgfsetdash{}{0pt}%
\pgfpathmoveto{\pgfqpoint{1.584272in}{1.609389in}}%
\pgfpathlineto{\pgfqpoint{1.758018in}{1.609389in}}%
\pgfpathlineto{\pgfqpoint{1.758018in}{1.609389in}}%
\pgfpathlineto{\pgfqpoint{1.584272in}{1.609389in}}%
\pgfpathlineto{\pgfqpoint{1.584272in}{1.609389in}}%
\pgfpathclose%
\pgfusepath{stroke,fill}%
\end{pgfscope}%
\begin{pgfscope}%
\pgfpathrectangle{\pgfqpoint{0.694444in}{0.416667in}}{\pgfqpoint{3.194444in}{1.416667in}}%
\pgfusepath{clip}%
\pgfsetbuttcap%
\pgfsetmiterjoin%
\definecolor{currentfill}{rgb}{0.447059,0.447059,0.447059}%
\pgfsetfillcolor{currentfill}%
\pgfsetlinewidth{1.003750pt}%
\definecolor{currentstroke}{rgb}{0.266667,0.266667,0.266667}%
\pgfsetstrokecolor{currentstroke}%
\pgfsetdash{}{0pt}%
\pgfpathmoveto{\pgfqpoint{1.832481in}{1.659762in}}%
\pgfpathlineto{\pgfqpoint{2.006227in}{1.659762in}}%
\pgfpathlineto{\pgfqpoint{2.006227in}{1.659762in}}%
\pgfpathlineto{\pgfqpoint{1.832481in}{1.659762in}}%
\pgfpathlineto{\pgfqpoint{1.832481in}{1.659762in}}%
\pgfpathclose%
\pgfusepath{stroke,fill}%
\end{pgfscope}%
\begin{pgfscope}%
\pgfpathrectangle{\pgfqpoint{0.694444in}{0.416667in}}{\pgfqpoint{3.194444in}{1.416667in}}%
\pgfusepath{clip}%
\pgfsetbuttcap%
\pgfsetmiterjoin%
\definecolor{currentfill}{rgb}{0.447059,0.447059,0.447059}%
\pgfsetfillcolor{currentfill}%
\pgfsetlinewidth{1.003750pt}%
\definecolor{currentstroke}{rgb}{0.266667,0.266667,0.266667}%
\pgfsetstrokecolor{currentstroke}%
\pgfsetdash{}{0pt}%
\pgfpathmoveto{\pgfqpoint{2.080689in}{1.670074in}}%
\pgfpathlineto{\pgfqpoint{2.254435in}{1.670074in}}%
\pgfpathlineto{\pgfqpoint{2.254435in}{1.670074in}}%
\pgfpathlineto{\pgfqpoint{2.080689in}{1.670074in}}%
\pgfpathlineto{\pgfqpoint{2.080689in}{1.670074in}}%
\pgfpathclose%
\pgfusepath{stroke,fill}%
\end{pgfscope}%
\begin{pgfscope}%
\pgfpathrectangle{\pgfqpoint{0.694444in}{0.416667in}}{\pgfqpoint{3.194444in}{1.416667in}}%
\pgfusepath{clip}%
\pgfsetbuttcap%
\pgfsetmiterjoin%
\definecolor{currentfill}{rgb}{0.447059,0.447059,0.447059}%
\pgfsetfillcolor{currentfill}%
\pgfsetlinewidth{1.003750pt}%
\definecolor{currentstroke}{rgb}{0.266667,0.266667,0.266667}%
\pgfsetstrokecolor{currentstroke}%
\pgfsetdash{}{0pt}%
\pgfpathmoveto{\pgfqpoint{2.328898in}{1.631526in}}%
\pgfpathlineto{\pgfqpoint{2.502644in}{1.631526in}}%
\pgfpathlineto{\pgfqpoint{2.502644in}{1.631526in}}%
\pgfpathlineto{\pgfqpoint{2.328898in}{1.631526in}}%
\pgfpathlineto{\pgfqpoint{2.328898in}{1.631526in}}%
\pgfpathclose%
\pgfusepath{stroke,fill}%
\end{pgfscope}%
\begin{pgfscope}%
\pgfpathrectangle{\pgfqpoint{0.694444in}{0.416667in}}{\pgfqpoint{3.194444in}{1.416667in}}%
\pgfusepath{clip}%
\pgfsetbuttcap%
\pgfsetmiterjoin%
\definecolor{currentfill}{rgb}{0.447059,0.447059,0.447059}%
\pgfsetfillcolor{currentfill}%
\pgfsetlinewidth{1.003750pt}%
\definecolor{currentstroke}{rgb}{0.266667,0.266667,0.266667}%
\pgfsetstrokecolor{currentstroke}%
\pgfsetdash{}{0pt}%
\pgfpathmoveto{\pgfqpoint{2.577107in}{1.587585in}}%
\pgfpathlineto{\pgfqpoint{2.750853in}{1.587585in}}%
\pgfpathlineto{\pgfqpoint{2.750853in}{1.587585in}}%
\pgfpathlineto{\pgfqpoint{2.577107in}{1.587585in}}%
\pgfpathlineto{\pgfqpoint{2.577107in}{1.587585in}}%
\pgfpathclose%
\pgfusepath{stroke,fill}%
\end{pgfscope}%
\begin{pgfscope}%
\pgfpathrectangle{\pgfqpoint{0.694444in}{0.416667in}}{\pgfqpoint{3.194444in}{1.416667in}}%
\pgfusepath{clip}%
\pgfsetbuttcap%
\pgfsetmiterjoin%
\definecolor{currentfill}{rgb}{0.447059,0.447059,0.447059}%
\pgfsetfillcolor{currentfill}%
\pgfsetlinewidth{1.003750pt}%
\definecolor{currentstroke}{rgb}{0.266667,0.266667,0.266667}%
\pgfsetstrokecolor{currentstroke}%
\pgfsetdash{}{0pt}%
\pgfpathmoveto{\pgfqpoint{2.825315in}{1.610349in}}%
\pgfpathlineto{\pgfqpoint{2.999061in}{1.610349in}}%
\pgfpathlineto{\pgfqpoint{2.999061in}{1.610349in}}%
\pgfpathlineto{\pgfqpoint{2.825315in}{1.610349in}}%
\pgfpathlineto{\pgfqpoint{2.825315in}{1.610349in}}%
\pgfpathclose%
\pgfusepath{stroke,fill}%
\end{pgfscope}%
\begin{pgfscope}%
\pgfpathrectangle{\pgfqpoint{0.694444in}{0.416667in}}{\pgfqpoint{3.194444in}{1.416667in}}%
\pgfusepath{clip}%
\pgfsetbuttcap%
\pgfsetmiterjoin%
\definecolor{currentfill}{rgb}{0.447059,0.447059,0.447059}%
\pgfsetfillcolor{currentfill}%
\pgfsetlinewidth{1.003750pt}%
\definecolor{currentstroke}{rgb}{0.266667,0.266667,0.266667}%
\pgfsetstrokecolor{currentstroke}%
\pgfsetdash{}{0pt}%
\pgfpathmoveto{\pgfqpoint{3.073524in}{1.671027in}}%
\pgfpathlineto{\pgfqpoint{3.247270in}{1.671027in}}%
\pgfpathlineto{\pgfqpoint{3.247270in}{1.671027in}}%
\pgfpathlineto{\pgfqpoint{3.073524in}{1.671027in}}%
\pgfpathlineto{\pgfqpoint{3.073524in}{1.671027in}}%
\pgfpathclose%
\pgfusepath{stroke,fill}%
\end{pgfscope}%
\begin{pgfscope}%
\pgfpathrectangle{\pgfqpoint{0.694444in}{0.416667in}}{\pgfqpoint{3.194444in}{1.416667in}}%
\pgfusepath{clip}%
\pgfsetbuttcap%
\pgfsetmiterjoin%
\definecolor{currentfill}{rgb}{0.447059,0.447059,0.447059}%
\pgfsetfillcolor{currentfill}%
\pgfsetlinewidth{1.003750pt}%
\definecolor{currentstroke}{rgb}{0.266667,0.266667,0.266667}%
\pgfsetstrokecolor{currentstroke}%
\pgfsetdash{}{0pt}%
\pgfpathmoveto{\pgfqpoint{3.321732in}{1.612970in}}%
\pgfpathlineto{\pgfqpoint{3.495478in}{1.612970in}}%
\pgfpathlineto{\pgfqpoint{3.495478in}{1.612970in}}%
\pgfpathlineto{\pgfqpoint{3.321732in}{1.612970in}}%
\pgfpathlineto{\pgfqpoint{3.321732in}{1.612970in}}%
\pgfpathclose%
\pgfusepath{stroke,fill}%
\end{pgfscope}%
\begin{pgfscope}%
\pgfpathrectangle{\pgfqpoint{0.694444in}{0.416667in}}{\pgfqpoint{3.194444in}{1.416667in}}%
\pgfusepath{clip}%
\pgfsetbuttcap%
\pgfsetmiterjoin%
\definecolor{currentfill}{rgb}{0.447059,0.447059,0.447059}%
\pgfsetfillcolor{currentfill}%
\pgfsetlinewidth{1.003750pt}%
\definecolor{currentstroke}{rgb}{0.266667,0.266667,0.266667}%
\pgfsetstrokecolor{currentstroke}%
\pgfsetdash{}{0pt}%
\pgfpathmoveto{\pgfqpoint{3.569941in}{1.647067in}}%
\pgfpathlineto{\pgfqpoint{3.743687in}{1.647067in}}%
\pgfpathlineto{\pgfqpoint{3.743687in}{1.647067in}}%
\pgfpathlineto{\pgfqpoint{3.569941in}{1.647067in}}%
\pgfpathlineto{\pgfqpoint{3.569941in}{1.647067in}}%
\pgfpathclose%
\pgfusepath{stroke,fill}%
\end{pgfscope}%
\begin{pgfscope}%
\pgfpathrectangle{\pgfqpoint{0.694444in}{0.416667in}}{\pgfqpoint{3.194444in}{1.416667in}}%
\pgfusepath{clip}%
\pgfsetbuttcap%
\pgfsetmiterjoin%
\definecolor{currentfill}{rgb}{0.447059,0.447059,0.447059}%
\pgfsetfillcolor{currentfill}%
\pgfsetlinewidth{1.003750pt}%
\definecolor{currentstroke}{rgb}{0.266667,0.266667,0.266667}%
\pgfsetstrokecolor{currentstroke}%
\pgfsetdash{}{0pt}%
\pgfpathmoveto{\pgfqpoint{0.839646in}{1.500471in}}%
\pgfpathlineto{\pgfqpoint{1.013392in}{1.500471in}}%
\pgfpathlineto{\pgfqpoint{1.013392in}{1.500471in}}%
\pgfpathlineto{\pgfqpoint{0.839646in}{1.500471in}}%
\pgfpathlineto{\pgfqpoint{0.839646in}{1.500471in}}%
\pgfpathclose%
\pgfusepath{stroke,fill}%
\end{pgfscope}%
\begin{pgfscope}%
\pgfpathrectangle{\pgfqpoint{0.694444in}{0.416667in}}{\pgfqpoint{3.194444in}{1.416667in}}%
\pgfusepath{clip}%
\pgfsetbuttcap%
\pgfsetmiterjoin%
\definecolor{currentfill}{rgb}{0.447059,0.447059,0.447059}%
\pgfsetfillcolor{currentfill}%
\pgfsetlinewidth{1.003750pt}%
\definecolor{currentstroke}{rgb}{0.266667,0.266667,0.266667}%
\pgfsetstrokecolor{currentstroke}%
\pgfsetdash{}{0pt}%
\pgfpathmoveto{\pgfqpoint{1.087855in}{1.493114in}}%
\pgfpathlineto{\pgfqpoint{1.261601in}{1.493114in}}%
\pgfpathlineto{\pgfqpoint{1.261601in}{1.493114in}}%
\pgfpathlineto{\pgfqpoint{1.087855in}{1.493114in}}%
\pgfpathlineto{\pgfqpoint{1.087855in}{1.493114in}}%
\pgfpathclose%
\pgfusepath{stroke,fill}%
\end{pgfscope}%
\begin{pgfscope}%
\pgfpathrectangle{\pgfqpoint{0.694444in}{0.416667in}}{\pgfqpoint{3.194444in}{1.416667in}}%
\pgfusepath{clip}%
\pgfsetbuttcap%
\pgfsetmiterjoin%
\definecolor{currentfill}{rgb}{0.447059,0.447059,0.447059}%
\pgfsetfillcolor{currentfill}%
\pgfsetlinewidth{1.003750pt}%
\definecolor{currentstroke}{rgb}{0.266667,0.266667,0.266667}%
\pgfsetstrokecolor{currentstroke}%
\pgfsetdash{}{0pt}%
\pgfpathmoveto{\pgfqpoint{1.336064in}{1.471126in}}%
\pgfpathlineto{\pgfqpoint{1.509810in}{1.471126in}}%
\pgfpathlineto{\pgfqpoint{1.509810in}{1.471126in}}%
\pgfpathlineto{\pgfqpoint{1.336064in}{1.471126in}}%
\pgfpathlineto{\pgfqpoint{1.336064in}{1.471126in}}%
\pgfpathclose%
\pgfusepath{stroke,fill}%
\end{pgfscope}%
\begin{pgfscope}%
\pgfpathrectangle{\pgfqpoint{0.694444in}{0.416667in}}{\pgfqpoint{3.194444in}{1.416667in}}%
\pgfusepath{clip}%
\pgfsetbuttcap%
\pgfsetmiterjoin%
\definecolor{currentfill}{rgb}{0.447059,0.447059,0.447059}%
\pgfsetfillcolor{currentfill}%
\pgfsetlinewidth{1.003750pt}%
\definecolor{currentstroke}{rgb}{0.266667,0.266667,0.266667}%
\pgfsetstrokecolor{currentstroke}%
\pgfsetdash{}{0pt}%
\pgfpathmoveto{\pgfqpoint{1.584272in}{1.609389in}}%
\pgfpathlineto{\pgfqpoint{1.758018in}{1.609389in}}%
\pgfpathlineto{\pgfqpoint{1.758018in}{1.609389in}}%
\pgfpathlineto{\pgfqpoint{1.584272in}{1.609389in}}%
\pgfpathlineto{\pgfqpoint{1.584272in}{1.609389in}}%
\pgfpathclose%
\pgfusepath{stroke,fill}%
\end{pgfscope}%
\begin{pgfscope}%
\pgfpathrectangle{\pgfqpoint{0.694444in}{0.416667in}}{\pgfqpoint{3.194444in}{1.416667in}}%
\pgfusepath{clip}%
\pgfsetbuttcap%
\pgfsetmiterjoin%
\definecolor{currentfill}{rgb}{0.447059,0.447059,0.447059}%
\pgfsetfillcolor{currentfill}%
\pgfsetlinewidth{1.003750pt}%
\definecolor{currentstroke}{rgb}{0.266667,0.266667,0.266667}%
\pgfsetstrokecolor{currentstroke}%
\pgfsetdash{}{0pt}%
\pgfpathmoveto{\pgfqpoint{1.832481in}{1.659762in}}%
\pgfpathlineto{\pgfqpoint{2.006227in}{1.659762in}}%
\pgfpathlineto{\pgfqpoint{2.006227in}{1.659762in}}%
\pgfpathlineto{\pgfqpoint{1.832481in}{1.659762in}}%
\pgfpathlineto{\pgfqpoint{1.832481in}{1.659762in}}%
\pgfpathclose%
\pgfusepath{stroke,fill}%
\end{pgfscope}%
\begin{pgfscope}%
\pgfpathrectangle{\pgfqpoint{0.694444in}{0.416667in}}{\pgfqpoint{3.194444in}{1.416667in}}%
\pgfusepath{clip}%
\pgfsetbuttcap%
\pgfsetmiterjoin%
\definecolor{currentfill}{rgb}{0.447059,0.447059,0.447059}%
\pgfsetfillcolor{currentfill}%
\pgfsetlinewidth{1.003750pt}%
\definecolor{currentstroke}{rgb}{0.266667,0.266667,0.266667}%
\pgfsetstrokecolor{currentstroke}%
\pgfsetdash{}{0pt}%
\pgfpathmoveto{\pgfqpoint{2.080689in}{1.670074in}}%
\pgfpathlineto{\pgfqpoint{2.254435in}{1.670074in}}%
\pgfpathlineto{\pgfqpoint{2.254435in}{1.670074in}}%
\pgfpathlineto{\pgfqpoint{2.080689in}{1.670074in}}%
\pgfpathlineto{\pgfqpoint{2.080689in}{1.670074in}}%
\pgfpathclose%
\pgfusepath{stroke,fill}%
\end{pgfscope}%
\begin{pgfscope}%
\pgfpathrectangle{\pgfqpoint{0.694444in}{0.416667in}}{\pgfqpoint{3.194444in}{1.416667in}}%
\pgfusepath{clip}%
\pgfsetbuttcap%
\pgfsetmiterjoin%
\definecolor{currentfill}{rgb}{0.447059,0.447059,0.447059}%
\pgfsetfillcolor{currentfill}%
\pgfsetlinewidth{1.003750pt}%
\definecolor{currentstroke}{rgb}{0.266667,0.266667,0.266667}%
\pgfsetstrokecolor{currentstroke}%
\pgfsetdash{}{0pt}%
\pgfpathmoveto{\pgfqpoint{2.328898in}{1.631526in}}%
\pgfpathlineto{\pgfqpoint{2.502644in}{1.631526in}}%
\pgfpathlineto{\pgfqpoint{2.502644in}{1.631526in}}%
\pgfpathlineto{\pgfqpoint{2.328898in}{1.631526in}}%
\pgfpathlineto{\pgfqpoint{2.328898in}{1.631526in}}%
\pgfpathclose%
\pgfusepath{stroke,fill}%
\end{pgfscope}%
\begin{pgfscope}%
\pgfpathrectangle{\pgfqpoint{0.694444in}{0.416667in}}{\pgfqpoint{3.194444in}{1.416667in}}%
\pgfusepath{clip}%
\pgfsetbuttcap%
\pgfsetmiterjoin%
\definecolor{currentfill}{rgb}{0.447059,0.447059,0.447059}%
\pgfsetfillcolor{currentfill}%
\pgfsetlinewidth{1.003750pt}%
\definecolor{currentstroke}{rgb}{0.266667,0.266667,0.266667}%
\pgfsetstrokecolor{currentstroke}%
\pgfsetdash{}{0pt}%
\pgfpathmoveto{\pgfqpoint{2.577107in}{1.587585in}}%
\pgfpathlineto{\pgfqpoint{2.750853in}{1.587585in}}%
\pgfpathlineto{\pgfqpoint{2.750853in}{1.587585in}}%
\pgfpathlineto{\pgfqpoint{2.577107in}{1.587585in}}%
\pgfpathlineto{\pgfqpoint{2.577107in}{1.587585in}}%
\pgfpathclose%
\pgfusepath{stroke,fill}%
\end{pgfscope}%
\begin{pgfscope}%
\pgfpathrectangle{\pgfqpoint{0.694444in}{0.416667in}}{\pgfqpoint{3.194444in}{1.416667in}}%
\pgfusepath{clip}%
\pgfsetbuttcap%
\pgfsetmiterjoin%
\definecolor{currentfill}{rgb}{0.447059,0.447059,0.447059}%
\pgfsetfillcolor{currentfill}%
\pgfsetlinewidth{1.003750pt}%
\definecolor{currentstroke}{rgb}{0.266667,0.266667,0.266667}%
\pgfsetstrokecolor{currentstroke}%
\pgfsetdash{}{0pt}%
\pgfpathmoveto{\pgfqpoint{2.825315in}{1.610349in}}%
\pgfpathlineto{\pgfqpoint{2.999061in}{1.610349in}}%
\pgfpathlineto{\pgfqpoint{2.999061in}{1.610349in}}%
\pgfpathlineto{\pgfqpoint{2.825315in}{1.610349in}}%
\pgfpathlineto{\pgfqpoint{2.825315in}{1.610349in}}%
\pgfpathclose%
\pgfusepath{stroke,fill}%
\end{pgfscope}%
\begin{pgfscope}%
\pgfpathrectangle{\pgfqpoint{0.694444in}{0.416667in}}{\pgfqpoint{3.194444in}{1.416667in}}%
\pgfusepath{clip}%
\pgfsetbuttcap%
\pgfsetmiterjoin%
\definecolor{currentfill}{rgb}{0.447059,0.447059,0.447059}%
\pgfsetfillcolor{currentfill}%
\pgfsetlinewidth{1.003750pt}%
\definecolor{currentstroke}{rgb}{0.266667,0.266667,0.266667}%
\pgfsetstrokecolor{currentstroke}%
\pgfsetdash{}{0pt}%
\pgfpathmoveto{\pgfqpoint{3.073524in}{1.671027in}}%
\pgfpathlineto{\pgfqpoint{3.247270in}{1.671027in}}%
\pgfpathlineto{\pgfqpoint{3.247270in}{1.671027in}}%
\pgfpathlineto{\pgfqpoint{3.073524in}{1.671027in}}%
\pgfpathlineto{\pgfqpoint{3.073524in}{1.671027in}}%
\pgfpathclose%
\pgfusepath{stroke,fill}%
\end{pgfscope}%
\begin{pgfscope}%
\pgfpathrectangle{\pgfqpoint{0.694444in}{0.416667in}}{\pgfqpoint{3.194444in}{1.416667in}}%
\pgfusepath{clip}%
\pgfsetbuttcap%
\pgfsetmiterjoin%
\definecolor{currentfill}{rgb}{0.447059,0.447059,0.447059}%
\pgfsetfillcolor{currentfill}%
\pgfsetlinewidth{1.003750pt}%
\definecolor{currentstroke}{rgb}{0.266667,0.266667,0.266667}%
\pgfsetstrokecolor{currentstroke}%
\pgfsetdash{}{0pt}%
\pgfpathmoveto{\pgfqpoint{3.321732in}{1.612970in}}%
\pgfpathlineto{\pgfqpoint{3.495478in}{1.612970in}}%
\pgfpathlineto{\pgfqpoint{3.495478in}{1.612970in}}%
\pgfpathlineto{\pgfqpoint{3.321732in}{1.612970in}}%
\pgfpathlineto{\pgfqpoint{3.321732in}{1.612970in}}%
\pgfpathclose%
\pgfusepath{stroke,fill}%
\end{pgfscope}%
\begin{pgfscope}%
\pgfpathrectangle{\pgfqpoint{0.694444in}{0.416667in}}{\pgfqpoint{3.194444in}{1.416667in}}%
\pgfusepath{clip}%
\pgfsetbuttcap%
\pgfsetmiterjoin%
\definecolor{currentfill}{rgb}{0.447059,0.447059,0.447059}%
\pgfsetfillcolor{currentfill}%
\pgfsetlinewidth{1.003750pt}%
\definecolor{currentstroke}{rgb}{0.266667,0.266667,0.266667}%
\pgfsetstrokecolor{currentstroke}%
\pgfsetdash{}{0pt}%
\pgfpathmoveto{\pgfqpoint{3.569941in}{1.647067in}}%
\pgfpathlineto{\pgfqpoint{3.743687in}{1.647067in}}%
\pgfpathlineto{\pgfqpoint{3.743687in}{1.647067in}}%
\pgfpathlineto{\pgfqpoint{3.569941in}{1.647067in}}%
\pgfpathlineto{\pgfqpoint{3.569941in}{1.647067in}}%
\pgfpathclose%
\pgfusepath{stroke,fill}%
\end{pgfscope}%
\begin{pgfscope}%
\definecolor{textcolor}{rgb}{0.000000,0.000000,0.000000}%
\pgfsetstrokecolor{textcolor}%
\pgfsetfillcolor{textcolor}%
\pgftext[x=0.926519in,y=1.528249in,,bottom]{\color{textcolor}{\ifdefined\pdftexversion\else\setmainfont{NanumMyeongjo}\rmfamily\fi\fontsize{5.000000}{6.000000}\selectfont\catcode`\^=\active\def^{\ifmmode\sp\else\^{}\fi}\catcode`\%=\active\def%{\%}1,148}}%
\end{pgfscope}%
\begin{pgfscope}%
\definecolor{textcolor}{rgb}{0.000000,0.000000,0.000000}%
\pgfsetstrokecolor{textcolor}%
\pgfsetfillcolor{textcolor}%
\pgftext[x=1.174728in,y=1.520892in,,bottom]{\color{textcolor}{\ifdefined\pdftexversion\else\setmainfont{NanumMyeongjo}\rmfamily\fi\fontsize{5.000000}{6.000000}\selectfont\catcode`\^=\active\def^{\ifmmode\sp\else\^{}\fi}\catcode`\%=\active\def%{\%}1,140}}%
\end{pgfscope}%
\begin{pgfscope}%
\definecolor{textcolor}{rgb}{0.000000,0.000000,0.000000}%
\pgfsetstrokecolor{textcolor}%
\pgfsetfillcolor{textcolor}%
\pgftext[x=1.422937in,y=1.498903in,,bottom]{\color{textcolor}{\ifdefined\pdftexversion\else\setmainfont{NanumMyeongjo}\rmfamily\fi\fontsize{5.000000}{6.000000}\selectfont\catcode`\^=\active\def^{\ifmmode\sp\else\^{}\fi}\catcode`\%=\active\def%{\%}1,116}}%
\end{pgfscope}%
\begin{pgfscope}%
\definecolor{textcolor}{rgb}{0.000000,0.000000,0.000000}%
\pgfsetstrokecolor{textcolor}%
\pgfsetfillcolor{textcolor}%
\pgftext[x=1.671145in,y=1.637167in,,bottom]{\color{textcolor}{\ifdefined\pdftexversion\else\setmainfont{NanumMyeongjo}\rmfamily\fi\fontsize{5.000000}{6.000000}\selectfont\catcode`\^=\active\def^{\ifmmode\sp\else\^{}\fi}\catcode`\%=\active\def%{\%}1,263}}%
\end{pgfscope}%
\begin{pgfscope}%
\definecolor{textcolor}{rgb}{0.000000,0.000000,0.000000}%
\pgfsetstrokecolor{textcolor}%
\pgfsetfillcolor{textcolor}%
\pgftext[x=1.919354in,y=1.687540in,,bottom]{\color{textcolor}{\ifdefined\pdftexversion\else\setmainfont{NanumMyeongjo}\rmfamily\fi\fontsize{5.000000}{6.000000}\selectfont\catcode`\^=\active\def^{\ifmmode\sp\else\^{}\fi}\catcode`\%=\active\def%{\%}1,316}}%
\end{pgfscope}%
\begin{pgfscope}%
\definecolor{textcolor}{rgb}{0.000000,0.000000,0.000000}%
\pgfsetstrokecolor{textcolor}%
\pgfsetfillcolor{textcolor}%
\pgftext[x=2.167562in,y=1.697852in,,bottom]{\color{textcolor}{\ifdefined\pdftexversion\else\setmainfont{NanumMyeongjo}\rmfamily\fi\fontsize{5.000000}{6.000000}\selectfont\catcode`\^=\active\def^{\ifmmode\sp\else\^{}\fi}\catcode`\%=\active\def%{\%}1,327}}%
\end{pgfscope}%
\begin{pgfscope}%
\definecolor{textcolor}{rgb}{0.000000,0.000000,0.000000}%
\pgfsetstrokecolor{textcolor}%
\pgfsetfillcolor{textcolor}%
\pgftext[x=2.415771in,y=1.659303in,,bottom]{\color{textcolor}{\ifdefined\pdftexversion\else\setmainfont{NanumMyeongjo}\rmfamily\fi\fontsize{5.000000}{6.000000}\selectfont\catcode`\^=\active\def^{\ifmmode\sp\else\^{}\fi}\catcode`\%=\active\def%{\%}1,286}}%
\end{pgfscope}%
\begin{pgfscope}%
\definecolor{textcolor}{rgb}{0.000000,0.000000,0.000000}%
\pgfsetstrokecolor{textcolor}%
\pgfsetfillcolor{textcolor}%
\pgftext[x=2.663980in,y=1.615363in,,bottom]{\color{textcolor}{\ifdefined\pdftexversion\else\setmainfont{NanumMyeongjo}\rmfamily\fi\fontsize{5.000000}{6.000000}\selectfont\catcode`\^=\active\def^{\ifmmode\sp\else\^{}\fi}\catcode`\%=\active\def%{\%}1,240}}%
\end{pgfscope}%
\begin{pgfscope}%
\definecolor{textcolor}{rgb}{0.000000,0.000000,0.000000}%
\pgfsetstrokecolor{textcolor}%
\pgfsetfillcolor{textcolor}%
\pgftext[x=2.912188in,y=1.638127in,,bottom]{\color{textcolor}{\ifdefined\pdftexversion\else\setmainfont{NanumMyeongjo}\rmfamily\fi\fontsize{5.000000}{6.000000}\selectfont\catcode`\^=\active\def^{\ifmmode\sp\else\^{}\fi}\catcode`\%=\active\def%{\%}1,264}}%
\end{pgfscope}%
\begin{pgfscope}%
\definecolor{textcolor}{rgb}{0.000000,0.000000,0.000000}%
\pgfsetstrokecolor{textcolor}%
\pgfsetfillcolor{textcolor}%
\pgftext[x=3.160397in,y=1.698804in,,bottom]{\color{textcolor}{\ifdefined\pdftexversion\else\setmainfont{NanumMyeongjo}\rmfamily\fi\fontsize{5.000000}{6.000000}\selectfont\catcode`\^=\active\def^{\ifmmode\sp\else\^{}\fi}\catcode`\%=\active\def%{\%}1,328}}%
\end{pgfscope}%
\begin{pgfscope}%
\definecolor{textcolor}{rgb}{0.000000,0.000000,0.000000}%
\pgfsetstrokecolor{textcolor}%
\pgfsetfillcolor{textcolor}%
\pgftext[x=3.408605in,y=1.640748in,,bottom]{\color{textcolor}{\ifdefined\pdftexversion\else\setmainfont{NanumMyeongjo}\rmfamily\fi\fontsize{5.000000}{6.000000}\selectfont\catcode`\^=\active\def^{\ifmmode\sp\else\^{}\fi}\catcode`\%=\active\def%{\%}1,267}}%
\end{pgfscope}%
\begin{pgfscope}%
\definecolor{textcolor}{rgb}{0.000000,0.000000,0.000000}%
\pgfsetstrokecolor{textcolor}%
\pgfsetfillcolor{textcolor}%
\pgftext[x=3.656814in,y=1.674844in,,bottom]{\color{textcolor}{\ifdefined\pdftexversion\else\setmainfont{NanumMyeongjo}\rmfamily\fi\fontsize{5.000000}{6.000000}\selectfont\catcode`\^=\active\def^{\ifmmode\sp\else\^{}\fi}\catcode`\%=\active\def%{\%}1,303}}%
\end{pgfscope}%
\begin{pgfscope}%
\definecolor{textcolor}{rgb}{1.000000,1.000000,1.000000}%
\pgfsetstrokecolor{textcolor}%
\pgfsetfillcolor{textcolor}%
\pgftext[x=0.926519in,y=0.863324in,,]{\color{textcolor}{\ifdefined\pdftexversion\else\setmainfont{NanumMyeongjo}\rmfamily\fi\fontsize{5.000000}{6.000000}\selectfont\catcode`\^=\active\def^{\ifmmode\sp\else\^{}\fi}\catcode`\%=\active\def%{\%}553}}%
\end{pgfscope}%
\begin{pgfscope}%
\definecolor{textcolor}{rgb}{1.000000,1.000000,1.000000}%
\pgfsetstrokecolor{textcolor}%
\pgfsetfillcolor{textcolor}%
\pgftext[x=1.174728in,y=0.808106in,,]{\color{textcolor}{\ifdefined\pdftexversion\else\setmainfont{NanumMyeongjo}\rmfamily\fi\fontsize{5.000000}{6.000000}\selectfont\catcode`\^=\active\def^{\ifmmode\sp\else\^{}\fi}\catcode`\%=\active\def%{\%}494}}%
\end{pgfscope}%
\begin{pgfscope}%
\definecolor{textcolor}{rgb}{1.000000,1.000000,1.000000}%
\pgfsetstrokecolor{textcolor}%
\pgfsetfillcolor{textcolor}%
\pgftext[x=1.422937in,y=0.863502in,,]{\color{textcolor}{\ifdefined\pdftexversion\else\setmainfont{NanumMyeongjo}\rmfamily\fi\fontsize{5.000000}{6.000000}\selectfont\catcode`\^=\active\def^{\ifmmode\sp\else\^{}\fi}\catcode`\%=\active\def%{\%}553}}%
\end{pgfscope}%
\begin{pgfscope}%
\definecolor{textcolor}{rgb}{1.000000,1.000000,1.000000}%
\pgfsetstrokecolor{textcolor}%
\pgfsetfillcolor{textcolor}%
\pgftext[x=1.671145in,y=0.915444in,,]{\color{textcolor}{\ifdefined\pdftexversion\else\setmainfont{NanumMyeongjo}\rmfamily\fi\fontsize{5.000000}{6.000000}\selectfont\catcode`\^=\active\def^{\ifmmode\sp\else\^{}\fi}\catcode`\%=\active\def%{\%}608}}%
\end{pgfscope}%
\begin{pgfscope}%
\definecolor{textcolor}{rgb}{1.000000,1.000000,1.000000}%
\pgfsetstrokecolor{textcolor}%
\pgfsetfillcolor{textcolor}%
\pgftext[x=1.919354in,y=0.841334in,,]{\color{textcolor}{\ifdefined\pdftexversion\else\setmainfont{NanumMyeongjo}\rmfamily\fi\fontsize{5.000000}{6.000000}\selectfont\catcode`\^=\active\def^{\ifmmode\sp\else\^{}\fi}\catcode`\%=\active\def%{\%}530}}%
\end{pgfscope}%
\begin{pgfscope}%
\definecolor{textcolor}{rgb}{1.000000,1.000000,1.000000}%
\pgfsetstrokecolor{textcolor}%
\pgfsetfillcolor{textcolor}%
\pgftext[x=2.167562in,y=0.926644in,,]{\color{textcolor}{\ifdefined\pdftexversion\else\setmainfont{NanumMyeongjo}\rmfamily\fi\fontsize{5.000000}{6.000000}\selectfont\catcode`\^=\active\def^{\ifmmode\sp\else\^{}\fi}\catcode`\%=\active\def%{\%}620}}%
\end{pgfscope}%
\begin{pgfscope}%
\definecolor{textcolor}{rgb}{1.000000,1.000000,1.000000}%
\pgfsetstrokecolor{textcolor}%
\pgfsetfillcolor{textcolor}%
\pgftext[x=2.415771in,y=0.892880in,,]{\color{textcolor}{\ifdefined\pdftexversion\else\setmainfont{NanumMyeongjo}\rmfamily\fi\fontsize{5.000000}{6.000000}\selectfont\catcode`\^=\active\def^{\ifmmode\sp\else\^{}\fi}\catcode`\%=\active\def%{\%}584}}%
\end{pgfscope}%
\begin{pgfscope}%
\definecolor{textcolor}{rgb}{1.000000,1.000000,1.000000}%
\pgfsetstrokecolor{textcolor}%
\pgfsetfillcolor{textcolor}%
\pgftext[x=2.663980in,y=0.996034in,,]{\color{textcolor}{\ifdefined\pdftexversion\else\setmainfont{NanumMyeongjo}\rmfamily\fi\fontsize{5.000000}{6.000000}\selectfont\catcode`\^=\active\def^{\ifmmode\sp\else\^{}\fi}\catcode`\%=\active\def%{\%}693}}%
\end{pgfscope}%
\begin{pgfscope}%
\definecolor{textcolor}{rgb}{1.000000,1.000000,1.000000}%
\pgfsetstrokecolor{textcolor}%
\pgfsetfillcolor{textcolor}%
\pgftext[x=2.912188in,y=1.355840in,,]{\color{textcolor}{\ifdefined\pdftexversion\else\setmainfont{NanumMyeongjo}\rmfamily\fi\fontsize{5.000000}{6.000000}\selectfont\catcode`\^=\active\def^{\ifmmode\sp\else\^{}\fi}\catcode`\%=\active\def%{\%}1,074}}%
\end{pgfscope}%
\begin{pgfscope}%
\definecolor{textcolor}{rgb}{1.000000,1.000000,1.000000}%
\pgfsetstrokecolor{textcolor}%
\pgfsetfillcolor{textcolor}%
\pgftext[x=3.160397in,y=0.954875in,,]{\color{textcolor}{\ifdefined\pdftexversion\else\setmainfont{NanumMyeongjo}\rmfamily\fi\fontsize{5.000000}{6.000000}\selectfont\catcode`\^=\active\def^{\ifmmode\sp\else\^{}\fi}\catcode`\%=\active\def%{\%}650}}%
\end{pgfscope}%
\begin{pgfscope}%
\definecolor{textcolor}{rgb}{1.000000,1.000000,1.000000}%
\pgfsetstrokecolor{textcolor}%
\pgfsetfillcolor{textcolor}%
\pgftext[x=3.408605in,y=0.856643in,,]{\color{textcolor}{\ifdefined\pdftexversion\else\setmainfont{NanumMyeongjo}\rmfamily\fi\fontsize{5.000000}{6.000000}\selectfont\catcode`\^=\active\def^{\ifmmode\sp\else\^{}\fi}\catcode`\%=\active\def%{\%}546}}%
\end{pgfscope}%
\begin{pgfscope}%
\definecolor{textcolor}{rgb}{1.000000,1.000000,1.000000}%
\pgfsetstrokecolor{textcolor}%
\pgfsetfillcolor{textcolor}%
\pgftext[x=3.656814in,y=0.941284in,,]{\color{textcolor}{\ifdefined\pdftexversion\else\setmainfont{NanumMyeongjo}\rmfamily\fi\fontsize{5.000000}{6.000000}\selectfont\catcode`\^=\active\def^{\ifmmode\sp\else\^{}\fi}\catcode`\%=\active\def%{\%}635}}%
\end{pgfscope}%
\begin{pgfscope}%
\definecolor{textcolor}{rgb}{1.000000,1.000000,1.000000}%
\pgfsetstrokecolor{textcolor}%
\pgfsetfillcolor{textcolor}%
\pgftext[x=0.926519in,y=1.227951in,,]{\color{textcolor}{\ifdefined\pdftexversion\else\setmainfont{NanumMyeongjo}\rmfamily\fi\fontsize{5.000000}{6.000000}\selectfont\catcode`\^=\active\def^{\ifmmode\sp\else\^{}\fi}\catcode`\%=\active\def%{\%}386}}%
\end{pgfscope}%
\begin{pgfscope}%
\definecolor{textcolor}{rgb}{1.000000,1.000000,1.000000}%
\pgfsetstrokecolor{textcolor}%
\pgfsetfillcolor{textcolor}%
\pgftext[x=1.174728in,y=1.146939in,,]{\color{textcolor}{\ifdefined\pdftexversion\else\setmainfont{NanumMyeongjo}\rmfamily\fi\fontsize{5.000000}{6.000000}\selectfont\catcode`\^=\active\def^{\ifmmode\sp\else\^{}\fi}\catcode`\%=\active\def%{\%}359}}%
\end{pgfscope}%
\begin{pgfscope}%
\definecolor{textcolor}{rgb}{1.000000,1.000000,1.000000}%
\pgfsetstrokecolor{textcolor}%
\pgfsetfillcolor{textcolor}%
\pgftext[x=1.422937in,y=1.238358in,,]{\color{textcolor}{\ifdefined\pdftexversion\else\setmainfont{NanumMyeongjo}\rmfamily\fi\fontsize{5.000000}{6.000000}\selectfont\catcode`\^=\active\def^{\ifmmode\sp\else\^{}\fi}\catcode`\%=\active\def%{\%}397}}%
\end{pgfscope}%
\begin{pgfscope}%
\definecolor{textcolor}{rgb}{1.000000,1.000000,1.000000}%
\pgfsetstrokecolor{textcolor}%
\pgfsetfillcolor{textcolor}%
\pgftext[x=1.671145in,y=1.355092in,,]{\color{textcolor}{\ifdefined\pdftexversion\else\setmainfont{NanumMyeongjo}\rmfamily\fi\fontsize{5.000000}{6.000000}\selectfont\catcode`\^=\active\def^{\ifmmode\sp\else\^{}\fi}\catcode`\%=\active\def%{\%}466}}%
\end{pgfscope}%
\begin{pgfscope}%
\definecolor{textcolor}{rgb}{1.000000,1.000000,1.000000}%
\pgfsetstrokecolor{textcolor}%
\pgfsetfillcolor{textcolor}%
\pgftext[x=1.919354in,y=1.545714in,,]{\color{textcolor}{\ifdefined\pdftexversion\else\setmainfont{NanumMyeongjo}\rmfamily\fi\fontsize{5.000000}{6.000000}\selectfont\catcode`\^=\active\def^{\ifmmode\sp\else\^{}\fi}\catcode`\%=\active\def%{\%}746}}%
\end{pgfscope}%
\begin{pgfscope}%
\definecolor{textcolor}{rgb}{1.000000,1.000000,1.000000}%
\pgfsetstrokecolor{textcolor}%
\pgfsetfillcolor{textcolor}%
\pgftext[x=2.167562in,y=1.383464in,,]{\color{textcolor}{\ifdefined\pdftexversion\else\setmainfont{NanumMyeongjo}\rmfamily\fi\fontsize{5.000000}{6.000000}\selectfont\catcode`\^=\active\def^{\ifmmode\sp\else\^{}\fi}\catcode`\%=\active\def%{\%}484}}%
\end{pgfscope}%
\begin{pgfscope}%
\definecolor{textcolor}{rgb}{1.000000,1.000000,1.000000}%
\pgfsetstrokecolor{textcolor}%
\pgfsetfillcolor{textcolor}%
\pgftext[x=2.415771in,y=1.361254in,,]{\color{textcolor}{\ifdefined\pdftexversion\else\setmainfont{NanumMyeongjo}\rmfamily\fi\fontsize{5.000000}{6.000000}\selectfont\catcode`\^=\active\def^{\ifmmode\sp\else\^{}\fi}\catcode`\%=\active\def%{\%}496}}%
\end{pgfscope}%
\begin{pgfscope}%
\definecolor{textcolor}{rgb}{1.000000,1.000000,1.000000}%
\pgfsetstrokecolor{textcolor}%
\pgfsetfillcolor{textcolor}%
\pgftext[x=2.663980in,y=1.442076in,,]{\color{textcolor}{\ifdefined\pdftexversion\else\setmainfont{NanumMyeongjo}\rmfamily\fi\fontsize{5.000000}{6.000000}\selectfont\catcode`\^=\active\def^{\ifmmode\sp\else\^{}\fi}\catcode`\%=\active\def%{\%}472}}%
\end{pgfscope}%
\begin{pgfscope}%
\definecolor{textcolor}{rgb}{1.000000,1.000000,1.000000}%
\pgfsetstrokecolor{textcolor}%
\pgfsetfillcolor{textcolor}%
\pgftext[x=2.912188in,y=1.467770in,,]{\color{textcolor}{\ifdefined\pdftexversion\else\setmainfont{NanumMyeongjo}\rmfamily\fi\fontsize{5.000000}{6.000000}\selectfont\catcode`\^=\active\def^{\ifmmode\sp\else\^{}\fi}\catcode`\%=\active\def%{\%}119}}%
\end{pgfscope}%
\begin{pgfscope}%
\definecolor{textcolor}{rgb}{1.000000,1.000000,1.000000}%
\pgfsetstrokecolor{textcolor}%
\pgfsetfillcolor{textcolor}%
\pgftext[x=3.160397in,y=1.533448in,,]{\color{textcolor}{\ifdefined\pdftexversion\else\setmainfont{NanumMyeongjo}\rmfamily\fi\fontsize{5.000000}{6.000000}\selectfont\catcode`\^=\active\def^{\ifmmode\sp\else\^{}\fi}\catcode`\%=\active\def%{\%}613}}%
\end{pgfscope}%
\begin{pgfscope}%
\definecolor{textcolor}{rgb}{1.000000,1.000000,1.000000}%
\pgfsetstrokecolor{textcolor}%
\pgfsetfillcolor{textcolor}%
\pgftext[x=3.408605in,y=1.475130in,,]{\color{textcolor}{\ifdefined\pdftexversion\else\setmainfont{NanumMyeongjo}\rmfamily\fi\fontsize{5.000000}{6.000000}\selectfont\catcode`\^=\active\def^{\ifmmode\sp\else\^{}\fi}\catcode`\%=\active\def%{\%}655}}%
\end{pgfscope}%
\begin{pgfscope}%
\definecolor{textcolor}{rgb}{1.000000,1.000000,1.000000}%
\pgfsetstrokecolor{textcolor}%
\pgfsetfillcolor{textcolor}%
\pgftext[x=3.656814in,y=1.450077in,,]{\color{textcolor}{\ifdefined\pdftexversion\else\setmainfont{NanumMyeongjo}\rmfamily\fi\fontsize{5.000000}{6.000000}\selectfont\catcode`\^=\active\def^{\ifmmode\sp\else\^{}\fi}\catcode`\%=\active\def%{\%}539}}%
\end{pgfscope}%
\begin{pgfscope}%
\pgfsetbuttcap%
\pgfsetmiterjoin%
\definecolor{currentfill}{rgb}{0.235294,0.490196,0.764706}%
\pgfsetfillcolor{currentfill}%
\pgfsetlinewidth{1.003750pt}%
\definecolor{currentstroke}{rgb}{0.266667,0.266667,0.266667}%
\pgfsetstrokecolor{currentstroke}%
\pgfsetdash{}{0pt}%
\pgfpathmoveto{\pgfqpoint{1.073968in}{0.157026in}}%
\pgfpathlineto{\pgfqpoint{1.212857in}{0.157026in}}%
\pgfpathlineto{\pgfqpoint{1.212857in}{0.205637in}}%
\pgfpathlineto{\pgfqpoint{1.073968in}{0.205637in}}%
\pgfpathlineto{\pgfqpoint{1.073968in}{0.157026in}}%
\pgfpathclose%
\pgfusepath{stroke,fill}%
\end{pgfscope}%
\begin{pgfscope}%
\definecolor{textcolor}{rgb}{0.000000,0.000000,0.000000}%
\pgfsetstrokecolor{textcolor}%
\pgfsetfillcolor{textcolor}%
\pgftext[x=1.240634in,y=0.157026in,left,base]{\color{textcolor}{\ifdefined\pdftexversion\else\setmainfont{NanumMyeongjo}\rmfamily\fi\fontsize{5.000000}{6.000000}\selectfont\catcode`\^=\active\def^{\ifmmode\sp\else\^{}\fi}\catcode`\%=\active\def%{\%}United States of America}}%
\end{pgfscope}%
\begin{pgfscope}%
\pgfsetbuttcap%
\pgfsetmiterjoin%
\definecolor{currentfill}{rgb}{0.337255,0.713725,0.627451}%
\pgfsetfillcolor{currentfill}%
\pgfsetlinewidth{1.003750pt}%
\definecolor{currentstroke}{rgb}{0.266667,0.266667,0.266667}%
\pgfsetstrokecolor{currentstroke}%
\pgfsetdash{}{0pt}%
\pgfpathmoveto{\pgfqpoint{2.061150in}{0.157026in}}%
\pgfpathlineto{\pgfqpoint{2.200039in}{0.157026in}}%
\pgfpathlineto{\pgfqpoint{2.200039in}{0.205637in}}%
\pgfpathlineto{\pgfqpoint{2.061150in}{0.205637in}}%
\pgfpathlineto{\pgfqpoint{2.061150in}{0.157026in}}%
\pgfpathclose%
\pgfusepath{stroke,fill}%
\end{pgfscope}%
\begin{pgfscope}%
\definecolor{textcolor}{rgb}{0.000000,0.000000,0.000000}%
\pgfsetstrokecolor{textcolor}%
\pgfsetfillcolor{textcolor}%
\pgftext[x=2.227817in,y=0.157026in,left,base]{\color{textcolor}{\ifdefined\pdftexversion\else\setmainfont{NanumMyeongjo}\rmfamily\fi\fontsize{5.000000}{6.000000}\selectfont\catcode`\^=\active\def^{\ifmmode\sp\else\^{}\fi}\catcode`\%=\active\def%{\%}Brazil}}%
\end{pgfscope}%
\begin{pgfscope}%
\pgfsetbuttcap%
\pgfsetmiterjoin%
\definecolor{currentfill}{rgb}{0.725490,0.486275,0.164706}%
\pgfsetfillcolor{currentfill}%
\pgfsetlinewidth{1.003750pt}%
\definecolor{currentstroke}{rgb}{0.266667,0.266667,0.266667}%
\pgfsetstrokecolor{currentstroke}%
\pgfsetdash{}{0pt}%
\pgfpathmoveto{\pgfqpoint{2.469788in}{0.157026in}}%
\pgfpathlineto{\pgfqpoint{2.608677in}{0.157026in}}%
\pgfpathlineto{\pgfqpoint{2.608677in}{0.205637in}}%
\pgfpathlineto{\pgfqpoint{2.469788in}{0.205637in}}%
\pgfpathlineto{\pgfqpoint{2.469788in}{0.157026in}}%
\pgfpathclose%
\pgfusepath{stroke,fill}%
\end{pgfscope}%
\begin{pgfscope}%
\definecolor{textcolor}{rgb}{0.000000,0.000000,0.000000}%
\pgfsetstrokecolor{textcolor}%
\pgfsetfillcolor{textcolor}%
\pgftext[x=2.636454in,y=0.157026in,left,base]{\color{textcolor}{\ifdefined\pdftexversion\else\setmainfont{NanumMyeongjo}\rmfamily\fi\fontsize{5.000000}{6.000000}\selectfont\catcode`\^=\active\def^{\ifmmode\sp\else\^{}\fi}\catcode`\%=\active\def%{\%}China, mainland}}%
\end{pgfscope}%
\begin{pgfscope}%
\pgfsetbuttcap%
\pgfsetmiterjoin%
\definecolor{currentfill}{rgb}{0.447059,0.447059,0.447059}%
\pgfsetfillcolor{currentfill}%
\pgfsetlinewidth{1.003750pt}%
\definecolor{currentstroke}{rgb}{0.266667,0.266667,0.266667}%
\pgfsetstrokecolor{currentstroke}%
\pgfsetdash{}{0pt}%
\pgfpathmoveto{\pgfqpoint{3.210727in}{0.157026in}}%
\pgfpathlineto{\pgfqpoint{3.349616in}{0.157026in}}%
\pgfpathlineto{\pgfqpoint{3.349616in}{0.205637in}}%
\pgfpathlineto{\pgfqpoint{3.210727in}{0.205637in}}%
\pgfpathlineto{\pgfqpoint{3.210727in}{0.157026in}}%
\pgfpathclose%
\pgfusepath{stroke,fill}%
\end{pgfscope}%
\begin{pgfscope}%
\definecolor{textcolor}{rgb}{0.000000,0.000000,0.000000}%
\pgfsetstrokecolor{textcolor}%
\pgfsetfillcolor{textcolor}%
\pgftext[x=3.377394in,y=0.157026in,left,base]{\color{textcolor}{\ifdefined\pdftexversion\else\setmainfont{NanumMyeongjo}\rmfamily\fi\fontsize{5.000000}{6.000000}\selectfont\catcode`\^=\active\def^{\ifmmode\sp\else\^{}\fi}\catcode`\%=\active\def%{\%}기타}}%
\end{pgfscope}%
\begin{pgfscope}%
\definecolor{textcolor}{rgb}{0.333333,0.333333,0.333333}%
\pgfsetstrokecolor{textcolor}%
\pgfsetfillcolor{textcolor}%
\pgftext[x=0.833333in,y=0.104167in,,top]{\color{textcolor}{\ifdefined\pdftexversion\else\setmainfont{NanumMyeongjo}\rmfamily\fi\fontsize{5.000000}{6.000000}\selectfont\catcode`\^=\active\def^{\ifmmode\sp\else\^{}\fi}\catcode`\%=\active\def%{\%}출처: FAOSTAT}}%
\end{pgfscope}%
\begin{pgfscope}%
\definecolor{textcolor}{rgb}{0.333333,0.333333,0.333333}%
\pgfsetstrokecolor{textcolor}%
\pgfsetfillcolor{textcolor}%
\pgftext[x=3.611111in,y=1.937500in,,top]{\color{textcolor}{\ifdefined\pdftexversion\else\setmainfont{NanumMyeongjo}\rmfamily\fi\fontsize{5.000000}{6.000000}\selectfont\catcode`\^=\active\def^{\ifmmode\sp\else\^{}\fi}\catcode`\%=\active\def%{\%}(단위: 1,000톤)}}%
\end{pgfscope}%
\end{pgfpicture}%
\makeatother%
\endgroup%
}
    \caption{한국의 주요 콩 수입국 구성 변화}
    \label{fig:korea-soy-import}
\end{figure}

그림~\ref{fig:korea-soy-import}는 한국의 수입 구조가
특정 공급국에 대한 집중과 조정 과정을 겪어 왔음을 시사하며,
이러한 변화가 네트워크 구조 측면에서 어떤 의미를 갖는지
추가적인 구조 분석의 필요성을 뒷받침한다.


\subsection{선행연구와 연구 공백}

국제 곡물 무역을 복잡계 네트워크 관점에서 분석하려는 연구는 점차 축적되고 있다.
특히 \textcite{wangStructuralEvolutionGlobal2023}은 2000--2020년 글로벌 콩 무역 네트워크를 구축하여,
네트워크가 소수 허브 국가에 의해 지배되는 scale-free 구조를 가지며,
특정 국가(중국)의 식량안보 취약성이 네트워크 구조와 결합될 수 있음을 실증적으로 보였다.
이는 글로벌 콩 무역을 단순한 물량 흐름이 아니라
\emph{구조적 관계망}으로 이해했다는 점에서 중요한 학술적 기여를 갖는다.

다만 기존 연구는 대체로 중국, 미국, 브라질 등
네트워크 중심부 국가를 중심으로 분석을 전개해 왔다.
이러한 접근은 허브의 역할과 네트워크 전체의 안정성 이해에는 효과적이지만,
한국과 같이 네트워크의 중심은 아니되 충격에 민감한 말단 수요국의
취약성을 충분히 설명하는 데에는 한계가 있다.
즉 ``허브가 제거될 때 네트워크가 어떻게 변하는가''에 대한 논의는 존재하지만,
그 변화가 한국과 같은 수입 의존국의 조달 경로 효율성과 위험 노출에
어떻게 연결되는지에 대한 분석은 상대적으로 부족하다.

또한 많은 선행연구가 2020년 이전 자료를 사용하고 있어,
팬데믹 이후 공급망 재편, 지정학적 리스크 확대, 기후 리스크 심화 등
최근 구조 변화를 충분히 반영하지 못한다.
따라서 2022년 시점의 글로벌 콩 무역 네트워크를 재구성하고
구조적 특성을 재점검하는 작업은 학문적·정책적으로 의미 있는 과제이다.


\subsection{연구 목적 및 연구 질문}

본 연구는 2022년 기준 글로벌 콩 무역 네트워크를 재구성하고,
그 구조적 특성과 취약성을 분석한 뒤,
이를 한국 식량안보의 관점에서 재해석하는 것을 목적으로 한다.
특히 한국을 네트워크의 중심국이 아니라
\emph{수입 경로에 의해 취약성이 결정되는 말단 수요국}으로 설정하여,
기존 논의의 분석 범위를 확장하고자 한다.

이를 위해 본 연구는 FAOSTAT의 Detailed Trade Matrix 자료를 활용해
국가 간 교역을 방향성과 가중치를 가진 네트워크로 구성하고,
기본 위상 지표 및 차수 기반 지표와 함께,
한국의 취약성을 직접적으로 포착하기 위한
\emph{수입 경로 효율성 지표(Import Path Efficiency Index, IPEI)}를 도입한다.
또한 특정 공급국과 한국 간의 직접 연결이 단절되는 시나리오를 설정하여,
우회 경로의 길이 변화가 한국의 조달 효율성에 미치는 영향을 정량적으로 평가한다.

구체적 연구 질문은 다음과 같다.
첫째, 2022년 글로벌 콩 무역 네트워크는 어떤 구조적 특성을 가지며,
주요 허브 국가는 누구인가?
둘째, 핵심 허브 국가 또는 주요 공급국과의 연결 약화(단절) 시,
네트워크 효율성과 한국의 수입 경로 효율성은 어떻게 변화하는가?
셋째, 이러한 구조 분석 결과는
한국의 수입선 다변화, 공급망 리스크 관리, 국내 생산 기반 확충 등
식량안보 정책에 어떤 시사점을 제공하는가?


\subsection{연구의 의의}

본 연구의 의의는 다음과 같이 정리할 수 있다.
첫째, 글로벌 콩 무역 네트워크를 2022년 최신 시점에서 재구성함으로써
기존 연구의 시간적 범위를 확장한다.
둘째, 중국 중심의 분석 틀을 넘어 한국을 말단 수요국으로 설정하고,
네트워크 내 위치와 조달 구조를 취약성 관점에서 해석한다.
셋째, 중심성 중심의 해석을 보완하기 위해
수입 조달의 ``경로'' 자체를 평가하는 IPEI와 단절 시나리오 분석을 결합함으로써,
공급 충격이 한국의 조달 효율성에 미치는 영향을 보다 직접적으로 제시한다.

이러한 분석은 향후 한국의 곡물 정책,
수입선 다변화 전략, 국내 생산 기반 및 논·밭 구조 개편 논의 등을
공급망 안정성의 관점에서 평가하는 데 유용한 이론적·정책적 기초를 제공할 것으로 기대된다.
