\section{\chapterfour}


\subsection{2022년 글로벌 콩 무역 네트워크의 기본 구조: 고효율-고집중의 공존}

2022년 기준 글로벌 콩 무역 네트워크는 총 142개 국가(노드)와 1,000개의 교역 관계(엣지)로 구성된다.
네트워크 밀도는 0.0466으로, 이론적으로 가능한 모든 국가 간 연결 중 약 5\%만이 실제 교역 관계로 나타난다.
이는 콩 무역이 전 세계에 균등하게 분산된 구조가 아니라,
일부 핵심 국가를 중심으로 선택적으로 연결되는 ``희소하지만 집중된'' 네트워크임을 의미한다.

그러나 연결의 수가 제한적임에도 평균 경로 길이는 2.32(최대 연결 성분 기준)로 매우 짧게 나타났고,
네트워크 직경 또한 5 수준으로 관측된다.
즉, 글로벌 콩 무역은 소수의 핵심 노드를 거치면 대부분의 국가가 비교적 적은 단계로 연결되는
small-world적 특성을 보인다.
이 구조는 평상시에는 높은 효율성을 제공하지만,
동시에 특정 허브(또는 필수 경유지)에 대한 의존을 통해 취약성이 축적될 수 있음을 시사한다.

본 장은 이러한 글로벌 네트워크의 일반적 성격을 출발점으로 삼되,
분석의 초점을 ``허브 국가의 영향력'' 자체가 아니라
한국과 같은 고의존 수입국의 관점에서 위험이 어떻게 전이되는지에 두고 결과를 재구성한다.
특히 한국은 글로벌 네트워크에서 중개자 역할이 제한적인 국가이므로,
전통적 중심성 지표만으로는 취약성이 과소평가될 수 있다.
따라서 본 연구는 한국 취약성 평가의 핵심을
수입 경로의 효율성과 회복력을 직접 반영하는 경로 기반 지표(IPEI)로 확장한다.

\par

\begin{table}[htbp]
\centering
\small
\renewcommand{\arraystretch}{2}
\label{tab:network-summary}

\begin{tabularx}{\textwidth}{c|c|c|c|Y|Y}
\hline
\rowcolor{gray!30}
Year & Nodes & Edges & Graph Density & Average Path Length & Network Diameter \\
\hline\hline
2000 & 123 & 660 & 0.039 & 3.029 & 8\\
2005 & 135 & 758 & 0.035 & 2.890 & 8\\
2010 & 129 & 792 & 0.043 & 3.058 & 8\\
2015 & 158 & 998 & 0.036 & 2.822 & 7\\
2020\footnotemark & 161 & 1,010 & 0.039 & 2.806 & 7\\
\hline\hline
\csvreader[
  late after line=\\\hline
]{asset/out_network_summary.csv}{}{
  \csvcoli & \csvcolii & \num{\csvcoliii} & \csvcoliv & \csvcolv & \csvcolvi
}
\end{tabularx}
\caption{글로벌 콩 무역 네트워크 요약 (2000-2022년)}

\end{table}


\footnotetext{2020년 이전 데이터는 \textcite{wangStructuralEvolutionGlobal2023}에서 인용}


\subsection{공급과 수요의 분리: 글로벌 허브 구조와 한국의 ``수요 종착점'' 위치}

가중 차수 분석에 따르면,
2022년 글로벌 콩 무역은 공급(수출)과 수요(수입) 기능이 뚜렷하게 분리되어 있다.

가중 수출 차수(weighted outdegree) 기준으로 브라질이 가장 큰 수출 허브로 나타났으며,
그 다음으로 미국, 아르헨티나, 우르과이, 캐나다가 뒤를 잇는다.
예를 들어 브라질의 가중 수출 차수는 약 71,305,972 수준,
미국은 약 51,725,761 수준으로 관측된다.
이는 글로벌 공급이 남미(브라질)와 북미(미국)에 구조적으로 집중되어 있음을 의미한다.

\par

\begin{table}[htbp]
\small{
\begin{center}
\renewcommand{\arraystretch}{2} 
\begin{tabularx}{\textwidth}{Y|Y}
\hline
\rowcolor{gray!30}
국가 & 가중수입차수  \\
\hline\hline
% 여기부터 CSV 데이터 행 자동 삽입
\csvreader[
  late after line=\\\hline
]{asset/out_top_exporters.csv}{}{% head to column names 제거, 인덱스 기반 컬럼 사용
  \csvcoli & \num{\csvcolii}
}
\end{tabularx}
\end{center}
}
\caption{2022년 콩 가중수출차수 상위 국가}
\end{table}



반면 가중 수입 차수(weighted indegree) 측면에서는
중국이 91,081,347 수준으로 단일 최대 수입 허브를 형성하며,
그 뒤를 네덜란드, 멕시코, 일본, 독일 등(가공·물류·재수출 기능 포함)이 따르는 것으로 나타난다.

이러한 ``공급의 집중(브라질·미국)''과 ``수요의 집중(중국)''이라는 비대칭 구조는
글로벌 콩 무역이 거시적으로는 효율적으로 작동하는 것처럼 보이더라도,
개별 수입국의 입장에서는
(1) 소수 공급국에 대한 물량 의존,
(2) 특정 수요 허브(중국)의 정책·가격 충격이 시장 전체를 재배열하는 효과,
라는 이중의 위험 경로를 내포한다.
한국은 바로 이 구조에서 전형적인 ``최종 수요(end-demand) 종착점''으로 위치한다.

\par

\begin{table}[htbp]
\small{
\begin{center}
\renewcommand{\arraystretch}{2} 
\begin{tabularx}{\textwidth}{Y|Y}
\hline
\rowcolor{gray!30}
국가 & 가중수입차수\\
\hline\hline
% 여기부터 CSV 데이터 행 자동 삽입
\csvreader[
  late after line=\\\hline
]{asset/out_top_importers.csv}{}{% head to column names 제거, 인덱스 기반 컬럼 사용
  \csvcoli & \num{\csvcolii}
}
\end{tabularx}
\end{center}
}
\caption{2022년 콩 가중수입차수 상위 국가}
\end{table}

\newpage

\subsection{한국의 교역 구조: ``연결은 직접적이나, 공급원은 좁다''}

한국의 글로벌 네트워크 내 위치를 한국 관점에서 해석하면,
핵심은 ``한국이 얼마나 중앙에 있느냐''가 아니라
``한국의 수입이 얼마나 소수 경로와 소수 국가에 의해 좌우되느냐''이다.

2022년 한국의 콩 수입(가중 수입 차수)은 약 1,302,775 수준으로,
글로벌 최대 수입국(중국)에 비해 규모는 작지만,
국내 수급 구조상 대체 가능성이 제한된 핵심 전략 품목이라는 점에서
체감 위험도는 교역량만으로 판단하기 어렵다.

또한 한국의 수입 상대국 분포는 미국 48.8\%, 브라질 41.4\%로,
상위 2개국(미국·브라질)에 대한 집중도가 매우 높게 나타난다.
이를 단순 집중도(HHI)로 환산하면 약 0.41 수준으로,
중국(0.46), 일본(0.58), 독일(0.4) 등 주요 수입국의 HHI와 비교했을 때 평균적인 수준으로 나타난다.
주요 수입국들이 대부분 소수의 공급원에 집중된 점을 고려하면, 
공급원 집중은 글로벌 콩 무역의 일반적 특성으로 볼 수 있다.



\begin{landscape}
\thispagestyle{empty}

\begin{table}[htbp]
\centering
\small
\renewcommand{\arraystretch}{2}
\begin{tabularx}{\linewidth}{c|c|c|c|c|c|Y}
\hline
\rowcolor{gray!30}
country & total\_import(t) & num\_partners & HHI & top1\_share & top2\_share & top\_partners \\
\hline\hline
\csvreader[
  late after line=\\\hline
]{asset/out_hhi_compare_major_importers.csv}{}{
  \csvcoli & \csvcolii & \csvcoliii & \csvcoliv & \csvcolv & \csvcolvi & \csvcolvii
}
\end{tabularx}
\caption{2022년 주요 수입국의 수입경로 효율성}
\end{table}
\end{landscape}


\subsection{한국 취약성의 핵심: 수입 경로 효율성(IPEI) 기반 평가}

중심성 지표(예: 매개 중심성)는 교역 경로에서의 ``중개'' 빈도를 반영하지만,
한국처럼 중개보다 ``도착'' 성격이 강한 수입국에는 위험을 과소평가할 수 있다.
이에 본 연구는 한국 취약성 평가를 위해
수입이 한국에 도달하기까지의 경로 효율을 직접 반영하는
수입 경로 효율성 지표(Import Path Efficiency Index, IPEI)를 사용한다.

IPEI는 각 수입 상대국으로부터 한국까지의 최단 경로 길이에 수입 물량을 결합해,
한국의 수입 연결이 얼마나 ``직접적이고 짧은 경로''에 의해 구성되는지를 $[0, 1]$ 범위에서 요약한다.

2022년 기준 한국의 IPEI는 약 1.0 수준으로,
평상시에는 한국의 수입이 대부분이 직접 연결(짧은 경로)에 의해 이루어짐을 의미한다.
하지만 이 값은 ``안전함''이라기보다
``직접 연결이 끊기는 순간 효율이 급락할 수 있는 구조''를 동시에 내포한다.
즉, 평상시 고효율(높은 IPEI)은 위기 시 급락 가능성과 동전의 양면이다.


\subsection{표적 단절 시나리오: 한국 수입 경로의 회복력(Resilience) 점검}



본 연구는 특정 공급국과 한국 간의 직접 연결이 단절될 경우를 가정하여,
우회 최단 경로를 재계산하고 IPEI의 변화를 비교한다.

기준 시나리오에서 한국의 수입 경로 효율성은 $IPEI_K^{baseline}=1.0$로 나타났다.
이때 다음의 표적 단절을 가정하면 한국의 수입 구조는 다음과 같이 변화한다.

\begin{table}[htbp]
\small{
\begin{center}
\renewcommand{\arraystretch}{2} 
\begin{tabularx}{\textwidth}{c|c|Y}
\hline
\rowcolor{gray!30}
Scenario & Description & $IPEI_K$\\
\hline\hline
% 여기부터 CSV 데이터 행 자동 삽입
\csvreader[
  late after line=\\\hline
]{asset/out_korea_ipei_scenarios.csv}{}{% head to column names 제거, 인덱스 기반 컬럼 사용
  \csvcoli & \csvcolii & \csvcoliii 
}
\end{tabularx}
\end{center}
}
\caption{한국의 수입 경로 제거 시나리오별 IPEI}
\end{table}


\begin{itemize}
    \item \textbf{시나리오 1: 미국$\rightarrow$한국 직수입 단절}\\
    $IPEI_K^{scenario}=0.7561$\\
    상위 1위 공급국 단절은 우회 경로 의존을 급격히 증가시키며,
    단기적으로는 물량 공백뿐 아니라 물류 재배치 비용(시간·가격)을 크게 확대시킬 수 있다.

    \item \textbf{시나리오 2: 브라질$\rightarrow$한국 직수입 단절}\\
    $IPEI_K^{scenario}=0.7932$\\
    브라질 단절은 미국 단절보다는 완만하나,
    한국 수입의 2대 축이 흔들린다는 점에서 가격 충격이 동반될 가능성이 크다.

    \item \textbf{시나리오 3: 미국 \& 브라질 동시 단절}\\
    $IPEI_K^{scenario}=0.5493$\\
    핵심 2개국의 동시 충격은 한국 수입 경로 효율성을 절반 수준으로 떨어뜨리며,
    이는 네트워크 상에서 ``연결은 존재하더라도'' 실제 조달이
    우회·재수출·경유 국가에 강하게 의존하는 비효율 상태로 전환됨을 의미한다.
\end{itemize}

요약하면, 글로벌 네트워크 전체의 효율성은 허브 제거에도 단기 붕괴가 크지 않을 수 있지만,
한국의 관점에서 보면 ``소수 공급국과의 직접 연결''이 곧 취약성의 핵심 레버리지로 작동한다.
따라서 한국 식량안보 논의는 글로벌 네트워크의 평균적 안정성보다,
한국의 공급원 집중과 경로 회복력(대체 경로의 존재 및 비용)을 중심으로 재구성될 필요가 있다.


\subsection{소결: 2022년 구조가 한국에 주는 정책적 함의}

2022년 글로벌 콩 무역 네트워크는 ``희소하지만 짧은 경로''라는 small-world적 효율성을 보이지만,
한국의 입장에서는 공급원 집중과 경로 취약성이 동시에 관측된다.
특히 한국의 수입은 상위 2개국에 크게 의존하며,
이들 국가와의 직접 연결 단절 시 IPEI가 급락하는 것으로 나타났다.

따라서 한국의 대응은 단순히 물량을 늘리거나 가격을 안정시키는 접근을 넘어,
(1) 공급원 다변화(집중도 완화),
(2) 대체 경로의 제도적 확보(우회 경로의 비용 최소화),
(3) 특정 국가 동시 충격을 상정한 복합 시나리오 기반 비상 조달 체계,
로 확장되어야 한다.

다음 절에서는 위 결과를 바탕으로,
한국의 수입 구조가 어떤 메커니즘을 통해
글로벌 허브 국가의 충격(정책·기후·물류·지정학)을 국내로 전이받는지,
그리고 이를 완화하기 위한 정책 조합을 구체적으로 논의한다.
