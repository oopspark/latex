\section{2023년 글로벌 콩 무역 네트워크 분석 결과}


집중된 비대칭적 구조를 형성하고 있음을 의미한다.

이와 같은 구조는 글로벌 콩 무역이 단기적으로는 효율적으로 작동할 수 있으나, 특정 국가의 정책 변화나 외부 충격이 발생할 경우 전체 네트워크에 큰 영향을 미칠 가능성을 내포하고 있다.

3. 매개 중심성 분석: 구조적 영향력의 분포

매개 중심성(betweenness centrality) 분석 결과는 교역량 중심의 분석과는 다른 차원의 구조적 특징을 드러낸다. 2023년 기준 매개 중심성이 가장 높은 국가는 미국(0.1589)으로 나타났으며, 그 뒤를 네덜란드(0.0977), 프랑스(0.0893), 캐나다(0.0829), 중국(0.0636)이 이었다.

미국은 높은 수출량뿐 아니라, 글로벌 콩 무역 경로에서 중요한 중개 역할을 수행하고 있음을 보여준다. 이는 미국이 단순한 생산·수출국을 넘어, 글로벌 콩 무역 흐름의 조정자 역할을 수행하고 있음을 시사한다. 특히 네덜란드와 프랑스와 같은 유럽 국가들이 높은 매개 중심성을 보인 점은, 이들 국가가 대규모 생산국은 아님에도 불구하고 물류·가공·재수출 등의 기능을 통해 글로벌 콩 무역 네트워크에서 중요한 경유지로 작동하고 있음을 의미한다.

중국은 압도적인 수입 규모에도 불구하고 매개 중심성에서는 미국이나 일부 유럽 국가들보다 낮은 값을 보였다. 이는 중국이 글로벌 콩 무역 네트워크에서 주로 **종착점(end node)**으로 기능하고 있으며, 교역 경로를 통제하거나 중개하는 위치에는 상대적으로 덜 놓여 있음을 시사한다. 이러한 구조적 위치는 중국이 글로벌 콩 무역에서 높은 의존성과 취약성을 동시에 가지는 이유를 설명해 준다.

4. 네트워크 안정성 분석: 표적 제거 시나리오 결과

글로벌 콩 무역 네트워크의 구조적 취약성을 분석하기 위해, 본 연구는 네트워크 효율성을 기준으로 한 표적 제거 시나리오 분석을 수행하였다. 기준 시나리오(baseline)에서 네트워크 효율성은 0.4749로 나타났으며, 이는 2023년 글로벌 콩 무역 네트워크가 비교적 높은 기능적 효율성을 유지하고 있음을 의미한다.

첫 번째 시나리오에서는 글로벌 최대 수출국인 브라질을 네트워크에서 제거하였다. 그 결과 네트워크 효율성은 0.4723으로 소폭 감소하였다. 이는 브라질이 글로벌 콩 공급에서 차지하는 물량적 비중은 매우 크지만, 네트워크 전체의 연결성과 효율성 측면에서는 단일 국가의 제거만으로 급격한 붕괴가 발생하지는 않음을 시사한다.

두 번째 시나리오에서는 미국을 네트워크에서 제거하였다. 이 경우 네트워크 효율성은 0.4749로 기준 시나리오와 거의 동일한 수준을 유지하였다. 이는 미국이 높은 매개 중심성을 가지고 있음에도 불구하고, 글로벌 콩 무역 네트워크가 단기적으로는 일정 수준의 대체 경로를 확보하고 있음을 의미한다.

세 번째 시나리오에서는 브라질과 미국을 동시에 제거하였다. 이 경우 네트워크 효율성은 다시 0.4723으로 감소하였다. 단일 국가 제거와 비교할 때 추가적인 급락은 관찰되지 않았으나, 네트워크가 소수 허브 국가의 복합적 이탈에 점차 취약해질 가능성을 시사하는 결과라 할 수 있다.

5. 소결: 2023년 글로벌 콩 무역 구조의 함의

2023년 글로벌 콩 무역 네트워크 분석 결과는 다음과 같은 구조적 특징을 보여준다. 첫째, 글로벌 콩 무역은 낮은 밀도와 짧은 평균 경로 길이를 가지는 고효율 네트워크로 구성되어 있다. 둘째, 공급 측면에서는 브라질과 미국을 중심으로 한 강한 집중 구조가 형성되어 있으며, 수요 측면에서는 중국이 압도적인 비중을 차지하고 있다. 셋째, 매개 중심성 분석을 통해 볼 때, 글로벌 콩 무역의 구조적 통제력은 일부 국가에 편중되어 있다.

이러한 결과는 글로벌 콩 무역이 평상시에는 효율적으로 작동할 수 있으나, 특정 허브 국가의 정책 변화나 외부 충격이 발생할 경우 구조적 불안정성이 확대될 수 있음을 시사한다. 특히 이러한 구조는 네트워크 중심부에 위치하지 않은 국가들에게 더 큰 위험으로 작용할 가능성이 크다.

다음 장에서는 이러한 2023년 글로벌 콩 무역 네트워크의 구조적 특성이 한국과 같은 수입 의존 국가의 식량안보에 어떤 의미를 가지는지, 그리고 허브 국가 중심 구조가 한국에 어떤 형태의 위험을 이전하는지를 집중적으로 분석한다.