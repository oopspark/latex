\section{\chapterfour}

\subsection{2023년 글로벌 콩 무역 네트워크의 기본 구조}

2023년 기준 글로벌 콩 무역 네트워크는 총 142개의 국가(노드)와 1,000개의 교역 관계(엣지)로 구성되어 있다. 네트워크 밀도는 0.0499로 나타나, 이론적으로 가능한 모든 국가 간 교역 관계 중 약 5%만이 실제로 형성되어 있음을 보여준다. 이는 글로벌 콩 무역이 모든 국가 간에 균등하게 분포된 구조가 아니라, 특정 국가들을 중심으로 선택적으로 연결된 구조임을 의미한다.

이와 같은 낮은 밀도에도 불구하고, 네트워크의 평균 경로 길이는 2.3367(최대 연결 성분 기준)로 매우 짧게 나타났다. 이는 글로벌 콩 무역 네트워크가 소수의 중개 국가를 통해 효율적으로 연결되어 있음을 시사하며, 전형적인 small-world 네트워크 특성을 보여준다. 즉, 글로벌 시장에서 콩은 비교적 적은 단계를 거쳐 대부분의 국가에 도달할 수 있는 구조를 가지고 있다.

한편, 네트워크 직경은 5로 나타나, 가장 멀리 떨어진 두 국가 간에도 최대 다섯 단계의 교역 경로만이 필요함을 의미한다. 이는 글로벌 콩 무역 네트워크가 지리적·정치적 장벽에도 불구하고 구조적으로 밀접하게 연결되어 있음을 보여주는 결과라 할 수 있다. 이러한 결과는 글로벌 콩 무역이 효율성 측면에서는 높은 성과를 보이는 반면, 구조적으로는 특정 연결 고리에 대한 의존도가 높을 가능성을 내포하고 있음을 시사한다.


\subsection{수출·수입 구조: 가중 차수 분석 결과}

가중 차수 분석 결과, 2023년 글로벌 콩 무역 네트워크는 수출과 수입 기능이 뚜렷하게 분리된 구조를 보였다. 가중 수출 차수(weighted outdegree)를 기준으로 할 때, 브라질이 가장 압도적인 수출 허브로 나타났으며, 그 뒤를 미국, 파라과이, 아르헨티나 등이 이었다. 특히 브라질의 가중 수출 차수는 91,326,573.11로, 다른 국가들과 비교할 때 현저히 높은 수준을 보였다. 이는 글로벌 콩 공급이 브라질에 구조적으로 집중되어 있음을 의미한다.

미국 역시 44,215,092.42의 높은 가중 수출 차수를 기록하며 여전히 핵심적인 수출국임을 확인할 수 있다. 다만 브라질과 비교할 때 그 규모는 상대적으로 작아, 최근 글로벌 콩 시장에서 브라질의 영향력이 더욱 확대되었음을 시사한다. 파라과이와 아르헨티나 또한 주요 수출국으로 기능하고 있으나, 브라질과 미국에 비해서는 한 단계 낮은 위치에 머무르고 있다.

반면 가중 수입 차수(weighted indegree) 분석에서는 중국이 압도적인 수입 허브로 나타났다. 중국의 가중 수입 차수는 103,383,644.17로, 글로벌 콩 무역에서 단일 최대 수입국임을 분명히 보여준다. 그 뒤를 아르헨티나, 미국, 브라질 등이 잇고 있으나, 중국과의 격차는 매우 크다. 이러한 결과는 글로벌 콩 무역이 공급 측면에서는 남미와 북미 일부 국가에 집중되어 있는 반면, 수요 측면에서는 중국에 극단적으로 집중된 비대칭적 구조를 형성하고 있음을 의미한다.

이와 같은 구조는 글로벌 콩 무역이 단기적으로는 효율적으로 작동할 수 있으나, 특정 국가의 정책 변화나 외부 충격이 발생할 경우 전체 네트워크에 큰 영향을 미칠 가능성을 내포하고 있다.


\subsection{매개 중심성 분석: 구조적 영향력의 분포}

매개 중심성(betweenness centrality) 분석 결과는 교역량 중심의 분석과는 다른 차원의 구조적 특징을 드러낸다. 2023년 기준 매개 중심성이 가장 높은 국가는 미국(0.1589)으로 나타났으며, 그 뒤를 네덜란드(0.0977), 프랑스(0.0893), 캐나다(0.0829), 중국(0.0636)이 이었다.

미국은 높은 수출량뿐 아니라, 글로벌 콩 무역 경로에서 중요한 중개 역할을 수행하고 있음을 보여준다. 이는 미국이 단순한 생산·수출국을 넘어, 글로벌 콩 무역 흐름의 조정자 역할을 수행하고 있음을 시사한다. 특히 네덜란드와 프랑스와 같은 유럽 국가들이 높은 매개 중심성을 보인 점은, 이들 국가가 대규모 생산국은 아님에도 불구하고 물류·가공·재수출 등의 기능을 통해 글로벌 콩 무역 네트워크에서 중요한 경유지로 작동하고 있음을 의미한다.

중국은 압도적인 수입 규모에도 불구하고 매개 중심성에서는 미국이나 일부 유럽 국가들보다 낮은 값을 보였다. 이는 중국이 글로벌 콩 무역 네트워크에서 주로 **종착점(end node)**으로 기능하고 있으며, 교역 경로를 통제하거나 중개하는 위치에는 상대적으로 덜 놓여 있음을 시사한다. 이러한 구조적 위치는 중국이 글로벌 콩 무역에서 높은 의존성과 취약성을 동시에 가지는 이유를 설명해 준다.


\subsection{네트워크 안정성 분석: 표적 제거 시나리오 결과}

글로벌 콩 무역 네트워크의 구조적 취약성을 분석하기 위해, 본 연구는 네트워크 효율성을 기준으로 한 표적 제거 시나리오 분석을 수행하였다. 기준 시나리오(baseline)에서 네트워크 효율성은 0.4749로 나타났으며, 이는 2023년 글로벌 콩 무역 네트워크가 비교적 높은 기능적 효율성을 유지하고 있음을 의미한다.

첫 번째 시나리오에서는 글로벌 최대 수출국인 브라질을 네트워크에서 제거하였다. 그 결과 네트워크 효율성은 0.4723으로 소폭 감소하였다. 이는 브라질이 글로벌 콩 공급에서 차지하는 물량적 비중은 매우 크지만, 네트워크 전체의 연결성과 효율성 측면에서는 단일 국가의 제거만으로 급격한 붕괴가 발생하지는 않음을 시사한다.

두 번째 시나리오에서는 미국을 네트워크에서 제거하였다. 이 경우 네트워크 효율성은 0.4749로 기준 시나리오와 거의 동일한 수준을 유지하였다. 이는 미국이 높은 매개 중심성을 가지고 있음에도 불구하고, 글로벌 콩 무역 네트워크가 단기적으로는 일정 수준의 대체 경로를 확보하고 있음을 의미한다.

세 번째 시나리오에서는 브라질과 미국을 동시에 제거하였다. 이 경우 네트워크 효율성은 다시 0.4723으로 감소하였다. 단일 국가 제거와 비교할 때 추가적인 급락은 관찰되지 않았으나, 네트워크가 소수 허브 국가의 복합적 이탈에 점차 취약해질 가능성을 시사하는 결과라 할 수 있다.


\subsection{소결: 2023년 글로벌 콩 무역 구조의 함의}

2023년 글로벌 콩 무역 네트워크 분석 결과는 다음과 같은 구조적 특징을 보여준다. 첫째, 글로벌 콩 무역은 낮은 밀도와 짧은 평균 경로 길이를 가지는 고효율 네트워크로 구성되어 있다. 둘째, 공급 측면에서는 브라질과 미국을 중심으로 한 강한 집중 구조가 형성되어 있으며, 수요 측면에서는 중국이 압도적인 비중을 차지하고 있다. 셋째, 매개 중심성 분석을 통해 볼 때, 글로벌 콩 무역의 구조적 통제력은 일부 국가에 편중되어 있다.

이러한 결과는 글로벌 콩 무역이 평상시에는 효율적으로 작동할 수 있으나, 특정 허브 국가의 정책 변화나 외부 충격이 발생할 경우 구조적 불안정성이 확대될 수 있음을 시사한다. 특히 이러한 구조는 네트워크 중심부에 위치하지 않은 국가들에게 더 큰 위험으로 작용할 가능성이 크다.

다음 장에서는 이러한 2023년 글로벌 콩 무역 네트워크의 구조적 특성이 한국과 같은 수입 의존 국가의 식량안보에 어떤 의미를 가지는지, 그리고 허브 국가 중심 구조가 한국에 어떤 형태의 위험을 이전하는지를 집중적으로 분석한다.