\section{\chapterfive}

\subsection{선행연구(2023년) 결과의 핵심 요약}

\textcite{wangStructuralEvolutionGlobal2023}은 2000--2020년 글로벌 콩 무역 네트워크(GSTN)를 구축하여,
네트워크가 small-world 및 scale-free 특성을 동시에 가지며
소수의 허브 국가가 구조적 안정성과 무역 흐름을 좌우한다는 점을 제시하였다.
특히 미국과 브라질은 핵심 허브로 기능하지만 그 기여 방식에는 차이가 존재하며,
표적 제거(targeted disruption) 시나리오에서 허브의 이탈이
네트워크 구조 및 특정 수입국(중국)의 교역 여건에 유의미한 영향을 줄 수 있음을 보였다.

본 연구는 이 선행연구의 문제의식(허브 의존과 취약성)을 계승하되,
분석의 초점을 ``네트워크 전체의 평균적 안정성''이 아니라
\emph{한국과 같은 말단 수요국의 조달 경로 취약성}으로 이동시켜 확장한다.
즉 선행연구가 네트워크 중심부의 구조 변화를 설명하는 데 강점을 가졌다면,
본 연구는 그 구조가 주변부 수입국에 어떤 위험 형태로 전이되는지를 규명하는 데 초점을 둔다.


\subsection{선행연구의 미흡점: 허브 중심 지표로는 한국의 취약성을 설명하기 어려움}

선행연구는 글로벌 콩 무역을 구조적 관계망으로 복원하고
허브의 중요성을 실증적으로 검증했다는 점에서 중요한 기여가 있다.
그러나 한국과 같은 중소 규모 수입국 관점에서는 다음의 한계가 남는다.

첫째, 허브 제거 시 ``네트워크가 유지되는가''라는 질문은
한국이 실제로 겪는 위험(대체 조달 비용 상승, 경유국 의존, 물류·계약 리스크)을
직접 측정하지 못한다.
네트워크가 연결을 유지하더라도,
주변부 수입국의 조달 구조는 더 길고 불확실한 경로로 재편될 수 있다.
즉 네트워크의 평균적 유지와 개별 국가의 조달 안정은 동일한 차원의 문제가 아니다.

둘째, 중심성(특히 매개 중심성)은 ``중개''와 ``경유''의 빈도를 강조한다.
하지만 한국은 글로벌 콩 무역에서 중개자라기보다
최종 수요(end-demand) 성격이 강하므로,
중심성 기반 해석만으로는 한국의 위험 노출이 과소평가될 수 있다.

셋째, 한국의 취약성은 단순히 ``누구와 거래하는가''가 아니라
``그 거래가 얼마나 직접적이며, 단절 시 대체 경로가 존재하는가''에 의해 크게 좌우된다.
따라서 한국을 설명하는 핵심은 허브의 위상보다
\emph{수입 경로의 구조적 효율성과 회복력}이다.
이 관점은 중국처럼 대규모 수입국뿐 아니라,
규모는 작더라도 충격에 민감한 말단 수요국에도 적용 가능한 분석 틀이다.


\subsection{본 연구의 확장: 수입 경로 효율성을 통한 말단 수요국 취약성의 정량화}

본 연구는 한국 취약성 평가의 핵심 지표로
수입 경로 효율성 지표(Import Path Efficiency Index, IPEI)를 도입한다.
IPEI는 각 수입 상대국으로부터 한국까지의 최단 경로 길이에 수입 물량을 결합하여,
한국의 수입 연결이 얼마나 ``직접적이고 짧은 경로''로 구성되어 있는지를
$[0,1]$ 범위에서 요약한다.
값이 1에 가까울수록 직접 연결 비중이 크며,
값이 낮아질수록 우회·경유 경로 의존이 확대되어
조달 구조가 구조적으로 비효율화되었음을 의미한다.

이 지표는 중심성처럼 네트워크 내 ``통제''나 ``중개''를 측정하기보다,
한국의 입장에서 중요한 질문,
즉 ``수입이 얼마나 우회 경로에 의존하는가''를 직접 측정하도록 설계되었다.
따라서 본 연구는 허브 중심 분석을 보완하여,
주변부 수입국의 구조적 취약성을 경로 기반(path-based)으로 재정의한다.

또한 본 연구는 특정 공급국과 한국 간의 \emph{직접 엣지}를 제거한 뒤
대체 최단 경로 길이를 재산정하는 단절 시나리오를 결합하여,
직접 연결 단절이 한국의 경로 효율성을 어떻게 변화시키는지(=회복력)를 평가하였다.
이는 ``허브의 중요성''을 논하는 단계에서 나아가,
한국의 조달 구조가 실제 충격에 어떻게 재편되는지를 보여주기 위한 확장이다.


\subsection{한국의 구조적 취약성: ``직접 연결의 고효율''이 곧 ``단절 시 급락 위험''}

2022년 기준 한국의 IPEI는 평상시 약 1.0 수준으로 나타나,
대부분의 수입이 직접 연결(짧은 경로)에 의해 이루어지는 구조임을 시사한다.
그러나 이 결과는 한국이 안전하다는 뜻이 아니라,
\emph{직접 연결이 끊길 경우 효율이 급락할 수 있는 구조}를 동시에 내포한다.
즉, 평상시의 ``고효율''은 위기 시 ``급락 가능성''과 동전의 양면이다.
직접 연결이 약화될수록 수입 물량은 제3국을 경유하거나 재수출 경로에 의존하게 되고,
그 과정에서 물류 불확실성과 거래 비용이 확대될 가능성이 커진다.

이 취약성은 한국의 공급원 집중도와 결합될 때 더욱 강화된다.
한국의 수입은 상위 2개국(미국·브라질)에 크게 집중되어 있으며,
집중도 지표(예: 상위국 점유율, HHI 등)에서도 이러한 편중이 확인된다.
따라서 상위 공급국과의 연결 약화는 곧바로 경로 재편(우회, 경유, 재수출 의존)을 유발한다.
결국 한국의 식량안보 위험은
네트워크 전체의 평균적 안정성보다,
\emph{소수 공급국과의 직접 연결이 단절될 때의 경로 대체 가능성}에 의해 더 크게 결정된다.


\subsection{한국적 함의: 수입선 다변화의 재정의와 국내 완충 능력의 중요성}

위 결과는 한국의 식량안보 논의가
``허브 국가의 안정''이나 ``네트워크 전체의 평균적 안정성''이 아니라,
\emph{한국의 조달 경로 구조}를 중심으로 재구성되어야 함을 시사한다.

첫째, 한국에 중요한 것은 ``수입선 다변화''를 공급국의 숫자 확대가 아니라
\emph{단절 시 대체 경로의 존재와 비용}으로 재정의하는 것이다.
즉 정책 평가의 단위는 ``국가 목록''이 아니라 ``경로의 회복력''이어야 한다.
이는 수입선 다변화 자체를 부정하기보다,
다변화의 성과를 ``경로 전환 가능성'' 관점에서 재평가해야 함을 의미한다.

둘째, 한국은 구조적으로 말단 수요국이므로
네트워크 내 통제력을 확대하기 어렵다.
따라서 정책의 현실적 목표는 통제력 확장이 아니라,
충격 발생 시 피해를 흡수할 수 있는 \emph{완충 장치(buffer)}의 구축이다.

셋째, 국내 콩 생산(및 비축·대체 체계)은
글로벌 시장에서의 가격 경쟁 수단이라기보다,
외부 충격에 대한 보험적 장치로서 재정의될 필요가 있다.
이는 자급률 제고를 부정하기보다,
자급률 정책을 ``목표''가 아니라 ``위험 관리 수단''으로 위치시키는 관점 전환을 의미한다.


\subsection{소 결}

본 장은 선행연구의 허브 중심 분석을 출발점으로 삼되,
한국과 같은 말단 수요국의 취약성을 설명하기 위해
수입 경로 효율성(IPEI)과 표적 단절 시나리오를 결합한 경로 기반 분석으로 확장하였다.
그 결과, 평상시의 직접 연결 고효율은
위기 시 효율 급락과 대체 조달 비용 증가로 이어질 수 있으며,
한국의 정책적 과제는 ``네트워크 안정''보다
``조달 경로의 회복력''을 강화하는 방향으로 재정립되어야 함을 확인하였다.

다음 장에서는 이러한 결과를 바탕으로,
정책적 시사점과 결론을 중심으로 연구의 함의를 정리한다.
