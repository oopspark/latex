\section{\chapterfive}

\subsection{선행연구(2020년) 결과의 핵심 요약}

Wang et al.(2023)은 2000–2020년 글로벌 콩 무역 네트워크(GSTN)를 분석하여 다음과 같은 핵심 결론을 제시하였다 

.
첫째, 글로벌 콩 무역 네트워크는 small-world 및 scale-free 특성을 동시에 가지며, 소수의 허브 국가가 네트워크의 구조적 안정성을 지배한다.
둘째, 미국과 브라질은 글로벌 콩 무역에서 핵심 허브 국가로 기능하지만, 네트워크 안정성에 기여하는 방식에는 차이가 존재한다.
셋째, 2020년 기준 시나리오 분석 결과, 미국의 제거는 네트워크 효율성을 유의미하게 저하시킨 반면, 브라질의 제거는 네트워크 전체의 효율성에는 상대적으로 제한적인 영향을 미쳤다.
넷째, 중국의 경우 브라질의 존재는 콩 무역 흐름을 통제하는 데 긍정적으로 작용한 반면, 미국의 존재는 오히려 중국의 통제력을 약화시키는 방향으로 작용하였다.

이러한 결과는 2020년까지의 글로벌 콩 무역 구조가 미국 중심의 중개 허브 구조를 강하게 유지하고 있었으며, 브라질은 물량 측면에서는 중요하지만 네트워크 구조적 통제력에서는 상대적으로 보조적 역할을 수행하고 있었음을 시사한다.


\subsection{2023년 시나리오 분석 결과의 특징}

본 연구에서 분석한 2023년 글로벌 콩 무역 네트워크에서도 표적 제거 방식의 시나리오 분석을 수행하였다. 기준 시나리오에서 네트워크 효율성은 0.4749로 나타났으며, 이는 네트워크가 여전히 높은 수준의 기능적 효율성을 유지하고 있음을 보여준다.

브라질 제거 시 네트워크 효율성은 0.4723으로 소폭 감소하였으며, 미국 제거 시에는 0.4749로 거의 변화가 없었다. 또한 브라질과 미국을 동시에 제거한 경우에도 네트워크 효율성은 0.4723 수준을 유지하였다. 이러한 결과는 단일 허브 국가의 이탈이 즉각적인 네트워크 붕괴로 이어지지 않는다는 점에서는 2020년 연구 결과와 유사하다.

그러나 중요한 차이점은 미국 제거의 효과가 2020년에 비해 약화되었다는 점이다. Wang et al.(2023)의 2020년 분석에서는 미국 제거 시 네트워크 효율성이 유의미하게 감소하였으나, 2023년 분석에서는 그러한 급격한 변화가 관찰되지 않았다. 이는 글로벌 콩 무역 네트워크에서 미국의 구조적 지위가 상대적으로 약화되었거나, 브라질을 중심으로 한 대체 경로가 보다 안정적으로 형성되었음을 시사한다.


\subsection{2020년 대비 2023년 구조 변화의 해석}

2020년과 2023년 결과를 비교할 때, 글로벌 콩 무역 네트워크에서는 연속성과 변화가 동시에 관찰된다.

먼저 연속성 측면에서 보면, 글로벌 콩 무역이 소수 수출국에 집중된 구조는 여전히 유지되고 있다. 브라질과 미국은 2023년에도 가중 수출 차수 기준 최상위 국가로 남아 있으며, 네트워크의 중심부를 구성하고 있다. 평균 경로 길이가 짧고 네트워크 직경이 제한적인 점 역시 small-world 구조가 유지되고 있음을 보여준다.

반면 변화 측면에서는 네트워크의 중개 기능 분산이 관찰된다. 2020년에는 미국이 매개 중심성 측면에서 압도적인 위치를 차지하며 네트워크 안정성에 핵심적인 역할을 수행하였으나, 2023년에는 네덜란드, 프랑스, 캐나다 등 비생산국의 매개 중심성이 상대적으로 두드러지게 나타났다. 이는 글로벌 콩 무역이 단순한 생산–소비 축을 넘어, 물류·가공·재수출을 포함한 다층적 네트워크로 진화하고 있음을 의미한다.

또한 브라질의 역할 역시 변화하였다. 2020년 연구에서는 브라질이 중국의 무역 통제력에 긍정적으로 작용하였으나, 네트워크 전체의 안정성에는 제한적인 영향을 미치는 것으로 평가되었다. 2023년 분석에서도 브라질 제거가 네트워크 효율성을 급격히 저하시키지는 않았으나, 브라질이 글로벌 공급에서 차지하는 압도적인 물량 비중을 고려할 때, 잠재적 충격의 크기는 오히려 확대되었을 가능성이 있다.


\subsection{한국적 함의 ①: “안정적인 네트워크 ≠ 안전한 국가”}

2023년 분석 결과에서 네트워크 효율성이 허브 국가 제거 이후에도 비교적 안정적으로 유지된 점은, 글로벌 콩 무역 네트워크가 구조적으로 일정 수준의 회복력을 갖추고 있음을 보여준다. 그러나 이는 네트워크 전체의 관점에서의 안정성일 뿐, 개별 국가의 안전성을 보장하지는 않는다.

한국과 같은 중소 규모 수입국은 네트워크 효율성이 유지되는 상황에서도 가격 급등, 계약 불이행, 물류 지연과 같은 충격을 직접적으로 감내해야 한다. 2020년과 비교할 때 2023년의 네트워크는 외형적으로 더 분산된 구조를 보이는 듯하지만, 이는 허브 국가 간의 역할 재배치에 가까우며, 한국이 차지하는 구조적 위치는 근본적으로 변화하지 않았다.

즉, “네트워크가 무너지지 않는다”는 사실은 “한국이 안전하다”는 의미와 동일하지 않다. 오히려 허브 국가 중심 구조가 유지되는 한, 충격은 네트워크 주변부 국가로 이전될 가능성이 높다.


\subsection{한국적 함의 ②: 수입선 다변화 전략의 구조적 한계}

2020년 이후 한국의 식량 정책 논의에서도 수입선 다변화가 반복적으로 제시되어 왔다. 그러나 2023년 네트워크 분석 결과는 수입선 다변화 전략이 구조적으로 가지는 한계를 보여준다. 글로벌 콩 무역 네트워크에서 실질적인 공급 능력을 가진 국가는 여전히 제한적이며, 이들 국가는 서로 강하게 연결된 허브를 형성하고 있다.

이는 한국이 특정 국가로부터의 수입 비중을 조정하더라도, 전체 네트워크 차원에서는 동일한 허브 집단에 의존하게 될 가능성이 높음을 의미한다. 2020년 대비 2023년 네트워크에서 미국의 상대적 영향력이 일부 약화되었음에도 불구하고, 이는 브라질 중심 구조의 강화로 이어졌을 뿐, 허브 의존 자체가 완화된 것은 아니다.


\subsection{한국적 함의 ③: 국내 콩 생산의 재정의}

이러한 분석 결과는 국내 콩 생산의 역할을 다시 정의할 필요성을 제기한다. 국내 콩 생산은 글로벌 시장과 가격 경쟁에서 우위를 확보하기 위한 수단이라기보다는, **국제 네트워크 충격에 대한 완충 장치(buffer)**로 이해될 필요가 있다. 2020년 연구에서 중국의 사례가 보여주듯, 허브 국가와의 관계 변화는 수입국의 통제력과 취약성을 동시에 변화시킨다.

한국의 경우 네트워크 내 통제력을 확대하는 것은 현실적으로 어렵다. 따라서 정책적 선택지는 네트워크 의존도를 완화하는 방향, 즉 일정 수준의 국내 생산 기반을 유지·강화함으로써 외부 충격 시 대응 여력을 확보하는 데 초점을 맞추어야 한다. 이는 자급률 제고라는 전통적 목표를 넘어, 식량안보를 위험 관리의 문제로 인식하는 관점 전환을 요구한다.


\subsection{소 결}

2020년과 2023년 글로벌 콩 무역 네트워크를 비교한 결과, 글로벌 콩 무역의 기본 구조는 유지되고 있으나, 허브 국가의 역할과 중개 기능에는 점진적인 변화가 관찰된다. 미국 중심의 구조는 다소 완화되었으나, 브라질을 중심으로 한 공급 집중 구조는 오히려 강화되었다. 이러한 변화는 네트워크 전체의 안정성을 높이는 방향으로 해석될 수 있으나, 한국과 같은 주변부 수입국의 구조적 취약성을 해소하는 데에는 제한적인 영향을 미친다.

다음 장에서는 이러한 분석을 바탕으로, 한국의 식량안보 정책과 농업 구조 개편에 대한 구체적인 정책적 시사점을 정리한다.