\section{\chapterthree}

\subsection{연구 자료}

본 연구는 글로벌 콩 무역의 구조적 특성을 분석하기 위해 
국제 곡물 무역 데이터를 활용하여 국가 간 교역 관계를 네트워크 형태로 재구성한다. 
분석에 사용된 기본 자료는 FAOSTAT(FAO Statistics Database)의 'Detailed Trade Matrix' 데이터이며\footnotemark, 
이는 국제적으로 가장 널리 활용되는 농산물 교역 통계 중 하나이다. 
수출과 수입 데이터는 방향성만 다른 동일한 데이터일 것으로 
기대할 수 있지만, 실제로는 수치상의 차이가 존재하였다.
따라서 본 연구에서는 관세의 대상이 되어 보다 정확한 수치 기입을 기대할 수 있는 
수입 물량을 중심으로 국가 간 교역량을 수집하였다.

\footnotetext{www.fao.org/faostat, 2025년 12월 16일 접속}

본 연구는 기존 연구들이 주로 다루어 온 2000-2020년 구간을 확장하여, 
2022년 시점의 글로벌 콩 무역 네트워크를 재구성하는 데 초점을 둔다
이를 통해 최근 몇 년간 심화된 글로벌 공급망 불안정성과 지정학적 리스크가 반영된 
네트워크 구조를 분석하고자 한다. 
분석 대상에는 일정 수준 이상의 교역량을 가진 국가들만을 포함하여, 
실질적인 무역 관계가 형성된 네트워크를 구성하였다.



\subsection{네트워크 구성 방법}

본 연구에서는 글로벌 콩 무역을 방향성과 가중치를 가진 
네트워크(directed weighted network)로 설정한다. 
구체적으로, 각 국가는 하나의 노드(node)로 정의되며, 
콩의 수출은 수출국에서 수입국으로 향하는 방향성을 가진 엣지(edge)로 표현된다. 
엣지의 가중치(weight)는 해당 연도 국가 간 콩 교역량을 의미하며, 
이를 통해 단순한 연결 여부가 아닌 교역 규모의 차이를 반영한다.

이와 같이 구성된 네트워크는 다음과 같은 특징을 가진다.
첫째, 방향성을 통해 수출국과 수입국의 역할을 구분할 수 있다.
둘째, 가중치를 통해 국가 간 교역 관계의 강도를 반영할 수 있다.
셋째, 다수 국가가 동시에 연결된 복잡한 구조를 분석할 수 있다.

2022년 기준으로 구성된 글로벌 콩 무역 네트워크는 
총 142개의 노드와 1,000개의 엣지로 이루어져 있으며,
이는 전 세계 콩 무역의 주요 흐름을 포괄하는 규모라 할 수 있다.
이러한 네트워크 구성 방식은 글로벌 콩 무역의 구조적 특성을 분석한 
기존 연구들과 동일한 접근법을 따르되, 최신 시점을 반영한다는 점에서 차별성을 가진다.


\subsection{네트워크 분석 지표}

글로벌 콩 무역 네트워크의 구조적 특성을 파악하기 위해 본 연구는 
복잡계 네트워크 분석에서 널리 활용되는 핵심 지표들을 사용한다. 
각 지표는 단순한 수학적 의미를 넘어, 국제 곡물 무역 구조를 해석하는 데 
중요한 정책적 함의를 제공한다.

\subsubsection{기본 위상 지표}

먼저, 네트워크의 전반적인 구조를 파악하기 위해 다음과 같은 기본 위상 지표를 활용한다.

네트워크 밀도(density, $\rho$)는 실제로 형성된 교역 관계의 수를, 
이론적으로 가능한 최대 교역 관계 수로 나눈 값이다. 
이는 글로벌 콩 무역에서 국가 간 연결성이 얼마나 조밀하게 형성되어 있는지를 나타내며, 
값이 낮을수록 소수 국가 중심의 선택적 교역 구조가 형성되어 있음을 의미한다.
\[
\rho = \frac{M}{N(N - 1)}
\]
\par
평균 경로 길이(average path length, $L$)는 임의의 두 국가 사이를 연결하는 
최단 경로의 평균값으로, 무역 네트워크의 효율성을 나타내는 지표이다. 
평균 경로 길이가 짧을수록 글로벌 시장에서 콩이 상대적으로 빠르고 효율적으로 
이동할 수 있음을 의미한다.
\[
L = \frac{2}{N(N-1)} \sum_{i \ge j} d_{ij}
\]
\par


네트워크 직경(diameter, $\mathrm{Dia}$)은 네트워크 내에서 가장 먼 두 국가 사이의 
최단 거리로 정의되며, 네트워크의 공간적·구조적 확장성을 나타낸다.
\[
\mathrm{Dia} = \max_{i,j} d_{ij}
\]
\par

\subsubsection{차수 및 가중 차수 지표}

차수(degree)는 한 국가가 직접적으로 연결된 교역 상대국의 수를 의미한다. 
이는 해당 국가가 글로벌 콩 무역에서 얼마나 많은 국가와 
직접적인 거래 관계를 유지하고 있는지를 보여준다.

가중 차수(weighted degree, $C^{\mathrm{sum}}_{i}$)는 교역 상대국의 수가 아니라, 
교역량을 고려한 연결 강도를 나타낸다. 
본 연구에서는 이를 다시 가중 수출 차수(outdegree, $C^{\mathrm{out}}_{i}$)와 
가중 수입 차수(indegree, $C^{\mathrm{in}}_{i}$)로 구분한다. 
가중 수출 차수는 한 국가가 다른 국가로 수출한 콩의 총량을, 
가중 수입 차수는 다른 국가로부터 수입한 총량을 의미한다. 
이를 통해 글로벌 콩 무역에서 주요 수출국과 수입국을 명확히 구분할 수 있다.
\[
C^{\mathrm{sum}}_{i}
= C^{\mathrm{out}}_{i} + C^{\mathrm{in}}_{i}
\]
\[
C^{\mathrm{out}}_{i}
= \sum_{\substack{j=1 , i \ne j}}^{N} W_{ij}
\]
\[
C^{\mathrm{in}}_{i}
= \sum_{\substack{j=1 , i \ne j}}^{N} W_{ji}
\]


\subsubsection{수입 경로 효율성 지표}

중심성(centrality) 지표는 네트워크 분석에서
각 노드가 차지하는 구조적 위치와 영향력을 평가하는 데 널리 활용된다.
특히 매개 중심성(betweenness centrality)은
교역 흐름이 특정 국가를 얼마나 자주 경유하는지를 나타내며,
글로벌 공급망에서 중개자 또는 허브 국가를 식별하는 데 유용하다.

그러나 본 연구의 분석 대상인 한국과 같이
글로벌 콩 무역에서 주요 공급국이나 중개국이 아닌
고의존 수입국(end-demand country)의 경우,
중심성 지표는 구조적 취약성을 충분히 반영하지 못하는 한계를 가진다.
실제로 한국은 교역 규모에 비해
네트워크 상에서의 매개 빈도나 경유 역할이 제한적이기 때문에,
매개 중심성이나 근접 중심성과 같은 지표는
한국의 위험 노출 수준을 과소평가할 가능성이 있다.

이에 본 연구에서는 중심성 지표를
글로벌 콩 무역 네트워크의 전반적 구조를 설명하는
보조적 지표로 활용하되,
한국의 취약성을 평가하는 핵심 지표로는
수입 경로의 효율성과 회복력을 직접적으로 측정하는
경로 기반 지표를 새롭게 도입한다.

본 연구는 한국의 구조적 취약성을
단순한 교역 규모나 중심성이 아닌,
수입 물량이 한국에 도달하기까지 요구되는
네트워크 경로의 효율성 관점에서 평가한다.
이를 위해 한국의 각 수입 상대국으로부터의 교역 흐름이
네트워크 상에서 얼마나 직접적이고 효율적으로 연결되어 있는지를
정량화하는 수입 경로 효율성 지표
(Import Path Efficiency Index, IPEI)를 정의한다.

IPEI는 각 수입국 $j$로부터 한국으로 유입되는 수입 물량에
해당 국가와 한국 간의 최단 경로 수를 가중치로 부여하여 합산한 값이,
모든 수입이 단일 경로(직접 연결)를 통해 이루어진다고 가정한 경우의
이론적 최대값 대비 어느 수준에 있는지를 나타낸다.

\[
IPEI_K
=
\frac{
\sum_{j} \dfrac{w_{jK}}{d_{jK}}
}{
\sum_{j} w_{jK}
}
\]


$IPEI_K$ 값은 $0 < IPEI_K \le 1$의 범위를 가지며,
값이 1에 가까울수록
한국의 콩 수입이 대부분 직접적이고 짧은 경로를 통해 이루어지고 있음을 의미한다.
반대로 $IPEI_K$ 값이 낮을수록,
수입 물량이 다수의 중간 국가를 경유하거나
네트워크 상에서 우회적인 경로에 의존하고 있음을 나타내며,
이는 외부 충격 발생 시 수입 구조의 취약성이 높음을 시사한다.



\subsection{수입 경로 단절 시나리오와 경로 효율성 변화 분석}
본 연구는 글로벌 콩 무역 네트워크에서
특정 수입 상대국과 한국 간의 직접적인 교역 연결이 단절될 경우,
한국이 해당 국가와 다시 연결되기 위해
네트워크 상에서 거쳐야 하는 우회 경로의 길이가
수입 구조의 효율성에 어떠한 영향을 미치는지를 분석한다.

이를 위해 각 시나리오에서는
특정 국가 $b$와 한국 $K$ 간의 직접 엣지 $(b,K)$를 제거한 후,
네트워크 상에서 $b$로부터 한국까지 도달 가능한
최단 경로 길이 $d^{(-b)}_{bK}$를 재계산한다.
만약 해당 국가와 한국 간의 경로가 완전히 단절될 경우,
$d^{(-b)}_{bK}$는 무한대로 정의된다.
이렇게 도출된 우회 최단 경로 길이를
기존의 수입 경로 효율성 지표 계산에 반영하여,
시나리오 이후의 수입 경로 효율성
$(IPEI_K^{scenario})$를 다시 계산한다.

$IPEI_K^{scenario}$ 값이 $IPEI_K^{baseline}$에 가까울수록
특정 국가와의 직접 교역 단절에도 불구하고
한국의 수입 경로 효율성이 크게 훼손되지 않음을 의미한다.
반대로 해당 값이 크게 감소할수록,
우회 경로 의존으로 인해 수입 구조의 효율성이 급격히 저하되었음을 나타내며,
이는 한국의 구조적 수입 취약성이 심화되었음을 시사한다.



\subsection{분석 방법의 의의와 한계}

본 연구의 분석 방법은 글로벌 콩 무역을 단순한 교역량의 집합이 아닌, 
구조적 관계망으로 파악할 수 있게 한다는 점에서 의의를 가진다. 
특히 허브 국가의 역할과 네트워크 취약성을 동시에 분석함으로써, 
기존의 자급률 중심 식량안보 논의를 확장하는 데 기여한다.

다만 네트워크 분석은 구조적 관계를 강조하는 반면, 
개별 국가의 정책·가격·물류 비용 등 미시적 요인을 
직접적으로 반영하기 어렵다는 한계를 가진다. 
따라서 본 연구의 결과는 개별 정책 설계의 직접적 해답이라기보다는, 
글로벌 곡물 무역 구조를 이해하기 위한 분석 틀로 해석될 필요가 있다.