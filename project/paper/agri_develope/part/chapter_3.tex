\section{\chapterthree}

\subsection{연구 자료}

본 연구는 글로벌 콩 무역의 구조적 특성을 분석하기 위해 국제 곡물 무역 데이터를 활용하여 국가 간 교역 관계를 네트워크 형태로 재구성한다. 분석에 사용된 기본 자료는 FAOSTAT(FAO Statistics Database)의 양자 간 콩 무역 데이터이며, 이는 국제적으로 가장 널리 활용되는 농산물 교역 통계 중 하나이다. FAOSTAT는 국가별 수출·수입 물량을 연도별로 제공함으로써, 글로벌 차원의 무역 흐름을 비교적 일관되게 파악할 수 있다는 장점을 가진다.

본 연구는 기존 연구들이 주로 다루어 온 2000–2020년 구간을 확장하여, 2023년 시점의 글로벌 콩 무역 네트워크를 재구성하는 데 초점을 둔다. 이를 통해 최근 몇 년간 심화된 글로벌 공급망 불안정성과 지정학적 리스크가 반영된 네트워크 구조를 분석하고자 한다. 분석 대상에는 일정 수준 이상의 교역량을 가진 국가들만을 포함하여, 실질적인 무역 관계가 형성된 네트워크를 구성하였다.


\subsection{네트워크 구성 방법}

본 연구에서는 글로벌 콩 무역을 **방향성과 가중치를 가진 네트워크(directed weighted network)**로 설정한다. 구체적으로, 각 국가는 하나의 노드(node)로 정의되며, 콩의 수출은 수출국에서 수입국으로 향하는 방향성을 가진 엣지(edge)로 표현된다. 엣지의 가중치(weight)는 해당 연도 국가 간 콩 교역량을 의미하며, 이를 통해 단순한 연결 여부가 아닌 교역 규모의 차이를 반영한다.

이와 같이 구성된 네트워크는 다음과 같은 특징을 가진다.
첫째, 방향성을 통해 수출국과 수입국의 역할을 구분할 수 있다.
둘째, 가중치를 통해 국가 간 교역 관계의 강도를 반영할 수 있다.
셋째, 다수 국가가 동시에 연결된 복잡한 구조를 분석할 수 있다.

2023년 기준으로 구성된 글로벌 콩 무역 네트워크는 총 142개의 노드와 1,000개의 엣지로 이루어져 있으며, 이는 전 세계 콩 무역의 주요 흐름을 포괄하는 규모라 할 수 있다. 이러한 네트워크 구성 방식은 글로벌 콩 무역의 구조적 특성을 분석한 기존 연구들과 동일한 접근법을 따르되, 최신 시점을 반영한다는 점에서 차별성을 가진다 


\subsection{네트워크 분석 지표}

글로벌 콩 무역 네트워크의 구조적 특성을 파악하기 위해 본 연구는 복잡계 네트워크 분석에서 널리 활용되는 핵심 지표들을 사용한다. 각 지표는 단순한 수학적 의미를 넘어, 국제 곡물 무역 구조를 해석하는 데 중요한 정책적 함의를 제공한다.

3.1 기본 위상 지표

먼저, 네트워크의 전반적인 구조를 파악하기 위해 다음과 같은 기본 위상 지표를 활용한다.

네트워크 밀도(network density)는 실제로 형성된 교역 관계의 수를, 이론적으로 가능한 최대 교역 관계 수로 나눈 값이다. 이는 글로벌 콩 무역에서 국가 간 연결성이 얼마나 조밀하게 형성되어 있는지를 나타내며, 값이 낮을수록 소수 국가 중심의 선택적 교역 구조가 형성되어 있음을 의미한다.

평균 경로 길이(average path length)는 임의의 두 국가 사이를 연결하는 최단 경로의 평균값으로, 무역 네트워크의 효율성을 나타내는 지표이다. 평균 경로 길이가 짧을수록 글로벌 시장에서 콩이 상대적으로 빠르고 효율적으로 이동할 수 있음을 의미한다.

네트워크 직경(network diameter)은 네트워크 내에서 가장 먼 두 국가 사이의 최단 거리로 정의되며, 네트워크의 공간적·구조적 확장성을 나타낸다.

3.2 차수 및 가중 차수 지표

차수(degree)는 한 국가가 직접적으로 연결된 교역 상대국의 수를 의미한다. 이는 해당 국가가 글로벌 콩 무역에서 얼마나 많은 국가와 직접적인 거래 관계를 유지하고 있는지를 보여준다.

가중 차수(weighted degree)는 교역 상대국의 수가 아니라, 교역량을 고려한 연결 강도를 나타낸다. 본 연구에서는 이를 다시 가중 수출 차수(weighted outdegree)와 가중 수입 차수(weighted indegree)로 구분한다. 가중 수출 차수는 한 국가가 다른 국가로 수출한 콩의 총량을, 가중 수입 차수는 다른 국가로부터 수입한 총량을 의미한다. 이를 통해 글로벌 콩 무역에서 주요 수출국과 수입국을 명확히 구분할 수 있다.

3.3 중심성 지표

본 연구에서 특히 중요하게 다루는 지표는 중심성(centrality) 지표이다. 중심성 지표는 단순한 교역 규모를 넘어, 네트워크 내에서 국가가 차지하는 구조적 위치와 영향력을 평가할 수 있게 한다.

매개 중심성(betweenness centrality)은 한 국가가 다른 국가들 간 교역 경로에서 얼마나 자주 경유되는지를 나타내는 지표이다. 매개 중심성이 높은 국가는 교역 흐름을 중개하거나 조정할 수 있는 위치에 있으며, 글로벌 공급망에서 전략적 영향력을 행사할 가능성이 크다. 이는 해당 국가가 반드시 최대 교역국일 필요는 없음을 의미한다.

근접 중심성(closeness centrality)은 한 국가가 네트워크 내 다른 모든 국가들과 얼마나 가까운 위치에 있는지를 나타낸다. 이는 외부 충격이 발생했을 때 대체 경로를 얼마나 빠르게 확보할 수 있는지를 보여주는 지표로 해석될 수 있으며, 네트워크 관점에서의 위험 대응 능력을 평가하는 데 활용된다.


\subsection{네트워크 안정성 분석과 시나리오 설정}

글로벌 콩 무역 네트워크는 소수 허브 국가에 의해 구조적으로 지탱되는 특성을 가지기 때문에, 특정 국가의 이탈이 전체 네트워크 안정성에 미치는 영향을 분석하는 것이 중요하다. 이를 위해 본 연구는 표적 제거(targeted disruption) 방식의 시나리오 분석을 활용한다.

표적 제거 방식은 네트워크에서 특정 국가를 의도적으로 제거한 뒤, 네트워크 구조와 효율성이 어떻게 변화하는지를 분석하는 방법이다. 이는 현실 세계에서 특정 국가가 수출 제한, 무역 제재, 전쟁, 기후 재해 등으로 인해 정상적인 교역을 수행하지 못하는 상황을 가상적으로 모의하는 데 유용하다.

본 연구에서는 2023년 네트워크를 기준으로 다음과 같은 세 가지 시나리오를 설정한다.
첫째, 글로벌 최대 수출국 중 하나인 브라질을 제거하는 시나리오.
둘째, 높은 매개 중심성을 가지는 미국을 제거하는 시나리오.
셋째, 브라질과 미국을 동시에 제거하는 복합 충격 시나리오.

각 시나리오별로 네트워크 효율성(network efficiency)의 변화를 측정하여, 허브 국가의 이탈이 글로벌 콩 무역 네트워크의 안정성과 회복력에 미치는 영향을 비교·분석한다. 네트워크 효율성은 네트워크 내 모든 국가 간 최단 경로의 역수를 평균한 값으로 정의되며, 값이 감소할수록 네트워크의 기능적 성능이 저하되었음을 의미한다.


\subsection{분석 방법의 의의와 한계}

본 연구의 분석 방법은 글로벌 콩 무역을 단순한 교역량의 집합이 아닌, 구조적 관계망으로 파악할 수 있게 한다는 점에서 의의를 가진다. 특히 허브 국가의 역할과 네트워크 취약성을 동시에 분석함으로써, 기존의 자급률 중심 식량안보 논의를 확장하는 데 기여한다.

다만 네트워크 분석은 구조적 관계를 강조하는 반면, 개별 국가의 정책·가격·물류 비용 등 미시적 요인을 직접적으로 반영하기 어렵다는 한계를 가진다. 따라서 본 연구의 결과는 개별 정책 설계의 직접적 해답이라기보다는, 글로벌 곡물 무역 구조를 이해하기 위한 분석 틀로 해석될 필요가 있다.