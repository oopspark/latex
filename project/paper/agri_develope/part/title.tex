
\begin{titlepage}
    {\large {\course \, 기말과제}} \hfill {\large {\publishyearmonth}}

    % \vspace{1cm}
    \begin{center}

        {\huge \textbf{\maintitleline}}\\
        \vspace{0.5cm}

        {\LARGE {\subtitle}}\\

    \end{center}

    \hfill \makebox[2.5cm][s]{{\LARGE \textbf{\authorname \textsuperscript{*}}}}\\

    \vspace{5pt}
    
    \begin{tcolorbox}[
        enhanced,
        colback=white,
        colframe=black,
        coltitle=black,          % ← 제목 글씨색을 검정으로 강제
        boxrule=0.6pt,
        arc=3pt,
        left=10pt,
        right=10pt,
        bottom=10pt,
        top=10pt,
        title={\large \textbf{목 \quad 차}},
        fonttitle=\bfseries,
        attach boxed title to top center={yshift=-11pt},
        boxed title style={
            colframe=black,
            colback=white,
            interior style={fill=white},
            boxrule=0pt,
        }
    ]
    
        % 첫 번째 표 (왼쪽)
        \begin{minipage}[t]{0.48\textwidth} % 0.45에서 약간 키우고, 여유를 주기 위해 0.48로 설정
        \centering % minipage 안의 내용을 중앙 정렬
        \begin{tabular}{c l}
        I. & \chapterone \\
        II. & \chaptertwo \\
        III. & \chapterthree 
        \end{tabular}
        \end{minipage}
        % --- 두 minipage 사이에 충분한 공간을 확보 ---
        \hfill % 왼쪽 minipage와 오른쪽 minipage 사이에 최대한의 수평 공간을 채움
        % --- --- --- --- --- --- --- --- --- --- ---
        % 두 번째 표 (오른쪽)
        \begin{minipage}[t]{0.48\textwidth} % 왼쪽과 같은 너비로 설정
        \centering % minipage 안의 내용을 중앙 정렬
        \begin{tabular}{c l}
        IV. & \chapterfour \\
        V. & \chapterfive \\
        VI. & \chaptersix
        \end{tabular}
        \end{minipage}
    \end{tcolorbox}


    \vspace{5pt}

    \begin{tcolorbox}[
        enhanced,
        colback=white,
        colframe=black,
        coltitle=black,          % ← 제목 글씨색을 검정으로 강제
        boxrule=0.6pt,
        arc=3pt,
        left=10pt,
        right=10pt,
        bottom=10pt,
        top=10pt,
        title={\large \textbf{요 \quad 약}},
        fonttitle=\bfseries,
        attach boxed title to top center={yshift=-11pt},
        boxed title style={
            colframe=black,
            colback=white,
            interior style={fill=white},
            boxrule=0pt,
        }
    ]

    \quad 본 연구는 2023년 글로벌 콩 무역 데이터를 활용하여 국가 간 교역 관계를 
    복잡계 네트워크로 재구성하고, 
    네트워크의 구조적 특성과 한국의 조달 취약성을 함께 분석하였다. 
    특히 한국을 말단 수요국으로 설정하여, 수입 물량이 한국에 도달하는 
    경로의 직접성과 우회 의존도를 정량화하는 
    \emph{수입 경로 효율성 지표}(Import Path Efficiency Index, IPEI)를 도입하였다. 
    또한 주요 공급국과 한국 간 직접 교역 연결이 단절되는 상황을 가정한 
    시나리오 분석을 통해, 단절 이후 대체 최단 경로의 변화가 IPEI에 미치는 영향을 
    평가함으로써 한국 수입 구조의 회복력을 검토하였다. 
    분석 결과, 글로벌 콩 무역은 여전히 소수 허브 국가에 대한 의존 구조를 유지하며 
    일부 역할 재편이 관찰되었으나, 이러한 변화는 네트워크 전체의 연결 유지에 기여할 뿐 
    한국과 같은 수입 의존 국가의 구조적 취약성을 근본적으로 완화하지는 못하였다. 
    이에 본 연구는 한국의 식량안보 정책이 자급률 중심 논의를 넘어, 
    국제 무역 네트워크 내 조달 경로 구조와 위험 관리 관점에서 재정립될 필요가 있음을 제시한다.
    
    \par
    \vspace{5pt}
    \begin{tabular}{c l}
    \textbullet\ 주제어: & \multirow{2}{=}{글로벌 콩 무역, 복잡계 네트워크 분석, 무역 네트워크 안정성, 식량안보, \\네트워크 취약성, 허브 국가, 한국 농업 정책}\\
    &
    \end{tabular}
    \end{tcolorbox}


    \vfill
    \hrule
    \vspace{0.2cm}
    {\large {* \department \, \program}}\\

\end{titlepage}
