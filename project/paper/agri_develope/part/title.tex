
\begin{titlepage}
    {\large {\course \, 기말과제}} \hfill {\large {\publishyearmonth}}

    % \vspace{1cm}
    \begin{center}

        {\huge \textbf{\maintitle}}\\
        \vspace{0.5cm}

        {\LARGE {\subtitle}}\\

    \end{center}

    \hfill \makebox[2.5cm][s]{{\LARGE \textbf{\authorname \textsuperscript{*}}}}\\

    
    \begin{tcolorbox}[
        enhanced,
        colback=white,
        colframe=black,
        coltitle=black,          % ← 제목 글씨색을 검정으로 강제
        boxrule=0.6pt,
        arc=3pt,
        left=10pt,
        right=10pt,
        bottom=10pt,
        top=10pt,
        title={\large \textbf{목 \quad 차}},
        fonttitle=\bfseries,
        attach boxed title to top center={yshift=-11pt},
        boxed title style={
            colframe=black,
            colback=white,
            interior style={fill=white},
            boxrule=0pt,
        }
    ]
    \begin{minipage}{0.45\textwidth}
    I. 서론 \\
    II. 논콩의 지역별 생산 현황 \\
    III. 농가경영안정 방안 비교
    \end{minipage}
    \hfill
    \begin{minipage}{0.45\textwidth}
    글로벌 대두 가치사슬의 재편\\
    IV. 중국 국내 대두 생산 및 가공 \\
    \quad 산업의 자립화\\
    V. 결론
    \end{minipage}
    \end{tcolorbox}



    \begin{tcolorbox}[
        enhanced,
        colback=white,
        colframe=black,
        coltitle=black,          % ← 제목 글씨색을 검정으로 강제
        boxrule=0.6pt,
        arc=3pt,
        left=10pt,
        right=10pt,
        bottom=10pt,
        top=10pt,
        title={\large \textbf{요 \quad 약}},
        fonttitle=\bfseries,
        attach boxed title to top center={yshift=-11pt},
        boxed title style={
            colframe=black,
            colback=white,
            interior style={fill=white},
            boxrule=0pt,
        }
    ]

    \quad 본 연구는 2023년 글로벌 콩 무역 네트워크를 복잡계 네트워크 분석으로 재구성하여 구조적 특성과 안정성을 분석하였다. 분석 결과, 글로벌 콩 무역은 소수 허브 국가에 집중된 고효율 구조를 유지하고 있으며, 허브 국가의 역할에는 일부 재편이 나타났다. 그러나 이러한 변화는 네트워크 전체의 안정성을 강화할 뿐, 한국과 같은 수입 의존 국가의 구조적 취약성을 완화하지는 못했다. 본 연구는 한국의 식량안보 정책이 자급률 중심 논의를 넘어, 국제 무역 네트워크 내 구조적 위치와 위험 관리 관점으로 전환되어야 함을 제시한다.
    \par
    \textbullet\ 주제어: 글로벌 콩 무역, 복잡계 네트워크 분석, 무역 네트워크 안정성, 식량안보, 네트워크 취약성, 허브 국가, 한국 농업 정책
    \end{tcolorbox}


    \vfill
    \hrule
    \vspace{0.2cm}
    {\large {* \department \, \program}}\\

\end{titlepage}
