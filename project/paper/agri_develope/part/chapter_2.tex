\section{\chaptertwo}


\subsection{식량안보 개념의 확장과 글로벌 곡물 무역}

전통적으로 식량안보(food security)는 한 국가가 자국민에게 
충분한 식량을 안정적으로 공급할 수 있는 능력을 의미해 왔다. 
FAO(Food and Agriculture Organization)에서는 식량안보를 
▲식량의 가용성(availability), ▲접근성(accessibility), 
▲이용성(utilization), ▲안정성(stability)의 네 가지 요소로 정의하고 있다. 
과거에는 이 중에서도 주로 국내 생산량과 자급률이 핵심 지표로 활용되었으며, 
식량 부족 문제는 생산성 향상이나 농지 확대로 해결 가능한 문제로 인식되었다.

그러나 글로벌화가 진전되고 국제 곡물 무역이 확대되면서, 
식량안보의 초점은 점차 국내 생산 중심의 논의에서 
국제 교역 구조를 포함한 문제로 이동하고 있다. 
특히 사료용 곡물과 유지 작물의 경우, 생산과 소비의 공간적 분리가 심화되면서 
다수 국가들이 구조적으로 수입에 의존하는 형태가 고착화되었다. 
이 과정에서 식량안보는 단순한 생산 능력의 문제가 아니라, 
국제 시장의 변동성, 교역 상대국의 정책 변화, 
지정학적 리스크에 노출되는 문제로 재정의되고 있다.

콩은 이러한 구조적 변화의 대표적인 사례라 할 수 있다. 
콩은 인간의 직접 소비뿐 아니라 축산 사료, 식용유, 산업 원료 등 다양한 용도로 활용되며, 
국가 간 교역 비중이 매우 높은 작물이다. 
그 결과, 콩 공급 안정성은 개별 국가의 농업 문제를 넘어, 
글로벌 공급망 차원의 위험 관리 문제로 인식되고 있다. 
최근 기후변화와 국제 분쟁, 무역 제재 등의 요인은 
콩 생산과 유통의 불확실성을 더욱 확대시키고 있으며, 
이는 수입 의존도가 높은 국가에 특히 큰 부담으로 작용하고 있다.


\subsection{글로벌 곡물 무역의 네트워크화와 구조적 취약성}

국제 곡물 무역은 단순한 양자 간 거래의 집합이 아니라, 
다수 국가가 상호 연결된 복잡한 네트워크(complex network)의 형태를 띠고 있다. 
이 네트워크에서는 일부 국가가 다수의 교역 관계를 집중적으로 보유하며, 
다른 국가들은 이러한 핵심 국가에 의존하는 구조가 형성된다. 
이러한 특징은 글로벌 곡물 무역이 높은 효율성을 가지는 동시에, 
특정 충격에 취약한 구조를 내포하고 있음을 의미한다.

특히 다수의 연구에서 글로벌 곡물 무역 네트워크는 
small-world 특성과 scale-free 특성을 동시에 지니는 것으로 보고되고 있다. 
small-world 네트워크는 평균 경로 길이가 짧아 자원의 이동이 효율적이라는 장점이 있으나, 
scale-free 네트워크는 소수의 허브(hub)에 연결이 집중되는 구조로 인해 
허브가 붕괴될 경우 전체 시스템의 안정성이 급격히 저하될 수 있다는 단점을 가진다. 
즉, 글로벌 곡물 무역은 평상시에는 효율적으로 작동하지만, 
특정 국가의 수출 제한이나 정치·경제적 충격이 발생할 경우 
연쇄적인 불안정을 초래할 가능성이 크다.

이러한 구조적 취약성은 최근의 국제적 사건들을 통해 현실적으로 확인되고 있다. 
팬데믹 기간 동안 일부 곡물 수출국들이 자국 식량 안정을 이유로 수출을 제한하였고, 
이는 국제 가격 급등과 수입국의 불안을 야기하였다. 
이처럼 글로벌 곡물 무역 네트워크의 구조는 
효율성과 안정성 사이의 긴장 관계를 내포하고 있으며, 
이를 이해하기 위해서는 단순한 교역량 분석을 넘어선 구조적 분석 틀이 요구된다.


\subsection{복잡계 네트워크 분석과 국제 곡물 무역 연구}

복잡계 네트워크 분석은 이러한 국제 곡물 무역 구조를 분석하는 데 있어 
유용한 도구로 활용되고 있다. 
네트워크 분석은 국가를 노드(node), 교역 관계를 엣지(edge)로 설정하여, 
교역의 양적 흐름뿐 아니라 구조적 위치와 상호의존 관계를 분석할 수 있게 한다. 
이를 통해 각 국가가 네트워크에서 차지하는 역할, 영향력, 그리고 취약성을 
정량적으로 평가할 수 있다.

국제 곡물 무역을 대상으로 한 네트워크 연구들은 주로 네트워크 밀도, 평균 경로 길이, 
중심성 지표(차수 중심성, 매개 중심성, 근접 중심성 등)를 활용하여 
무역 구조의 특성을 분석해 왔다. 
이러한 연구들은 글로벌 곡물 무역이 시간이 지남에 따라 점차 복잡해지고, 
소수의 핵심 국가가 무역 흐름을 주도하는 방향으로 진화해 왔음을 공통적으로 보고하고 있다.

특히 매개 중심성(betweenness centrality)은 
단순한 교역량과는 다른 차원의 중요성을 드러낸다. 
매개 중심성이 높은 국가는 교역 흐름의 경유지로서 기능하며, 
직접적인 생산·소비 규모와 무관하게 네트워크 전체에 대한 통제력을 가질 수 있다. 
이는 특정 국가가 수출국이 아니더라도, 글로벌 공급망에서 
중요한 전략적 위치를 차지할 수 있음을 의미한다.


\subsection{글로벌 콩 무역 네트워크 연구와 중국 중심 분석}

글로벌 콩 무역을 대상으로 한 대표적인 연구 중 하나는 Wang et al.(2023)이다. 
이 연구는 2000년부터 2020년까지의 데이터를 활용하여 글로벌 콩 무역 네트워크를 구축하고, 
네트워크의 구조적 진화와 중국의 식량안보 취약성을 분석하였다. 
연구 결과에 따르면, 글로벌 콩 무역 네트워크는 시간이 지남에 따라 점차 복잡해졌으며, 
미국과 브라질 등 소수 국가가 네트워크의 핵심 허브로 기능하고 있다.

해당 연구는 특히 표적 제거(targeted disruption) 시나리오를 통해 
허브 국가의 이탈이 네트워크 안정성과 중국의 교역 능력에 미치는 영향을 분석하였다. 
분석 결과, 미국과 브라질은 글로벌 콩 무역 네트워크에서 중요한 역할을 수행하며, 
이들 국가의 이탈은 네트워크 효율성과 중국의 대응 능력을 약화시키는 것으로 나타났다. 
이러한 결과는 글로벌 콩 무역 구조가 
특정 국가에 과도하게 의존하고 있음을 실증적으로 보여준다.

그러나 이러한 연구는 분석 대상과 정책적 함의가 주로 중국에 집중되어 있다는 한계를 가진다. 
중국은 세계 최대의 콩 수입국으로서 네트워크의 중심부에 위치한 국가이며, 
허브 국가들과 직접적인 대규모 교역 관계를 유지하고 있다. 
반면, 한국과 같은 중소 규모 수입국은 네트워크의 중심부보다는 주변부에 위치하며, 
구조적으로 다른 유형의 취약성을 가진다.


\subsection{연구 공백: 한국 관점에서의 구조적 해석 필요성}

한국은 글로벌 콩 무역 네트워크에서 거래량 기준으로는 상대적으로 작은 비중을 차지하지만, 
식량안보 측면에서는 매우 민감한 위치에 놓여 있다. 
한국은 허브 국가들과의 교섭력이나 대체 공급선 확보 능력이 제한적이며, 
국제 가격 변동과 공급 차질을 직접적으로 수용해야 하는 구조를 가진다. 
그럼에도 불구하고 기존 네트워크 분석 연구들은 
이러한 비중은 작지만 취약성이 큰 국가에 대한 논의를 충분히 다루지 못하고 있다.

또한 한국의 식량 정책 논의는 여전히 자급률이나 국내 생산 확대 여부에 
초점을 맞추는 경향이 강하며, 
글로벌 네트워크 내에서의 구조적 위치와 위험 노출을 체계적으로 분석한 연구는 드문 실정이다. 
이는 정책 설계 과정에서 국제 공급망 리스크를 과소평가할 가능성을 내포한다.

따라서 한국의 식량안보를 보다 현실적으로 이해하기 위해서는 
글로벌 콩 무역 네트워크의 구조를 분석하고, 
그 안에서 한국이 어떤 위치에 있으며 어떤 위험에 노출되어 있는지를 
구조적으로 해석할 필요가 있다. 
본 연구는 이러한 문제의식에 기반하여, 
기존 중국 중심 연구를 보완하고 한국적 함의를 도출하는 데 목적을 둔다.