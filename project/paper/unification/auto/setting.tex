\usepackage[a4paper, margin=3cm]{geometry}

\usepackage{xetexko}

\usepackage{fontspec}

% 기본 글꼴: 나눔명조


\setmainhangulfont{NanumMyeongjo}  % 한글용

\setmainhanjafont{Noto Serif CJK KR}

\setmainfont{TeX Gyre Pagella} % 영문/숫자


\usepackage[most]{tcolorbox}



\usepackage{setspace}
\linespread{1.5}

\usepackage{indentfirst}
\setlength{\parindent}{10pt}   % 15pt는 일반적인 들여쓰기




\usepackage{tabularx}
\usepackage{multirow}
\usepackage{colortbl}
\usepackage[table]{xcolor}
\usepackage{csvsimple}

% Y column type 정의
\newcolumntype{Y}{>{\centering\arraybackslash}X}




\usepackage{titlesec}

% Section 제목
\renewcommand{\thesection}{\Roman{section}}

\titleformat{\section}[hang]
  {\normalfont\Large\bfseries\centering}
  {\thesection.}
  {0.3cm}
  {}


% Chapter titlespacing
\titlespacing*{\section}{0pt}{40pt}{20pt}


\usepackage{amsmath} % 수식 환경을 위해 일반적으로 사용

\renewcommand{\thesubsection}{\hspace{4pt} \arabic{subsection}.}

% 서브섹션 제목 형식을 재정의합니다.
\titleformat{\subsection}
  % [스타일] [번호 표시 방식] [여백] [폰트 명령]
  {\normalfont\large\bfseries} % <--- 폰트 명령 (여기서 \bfseries를 사용)
  {\thesubsection}            % 번호 표시 방식 (예: 1.1)
  {4pt}                        % 번호와 제목 사이의 여백
  {}                           % 제목 앞에 추가할 코드 (비워둠)


\titlespacing{\subsection}
  {20pt}  % <---- 첫 번째 인자: 왼쪽 들여쓰기/공백 (Left indent/space)
  {30pt} % 두 번째 인자: 제목 위 간격
  {1.5ex plus .2ex}

\usepackage{multirow}



\setcounter{secnumdepth}{3}  % subsubsection까지 번호

\renewcommand{\thesubsubsection}{\hspace{10pt} (\arabic{subsubsection})}

\titleformat{\subsubsection}
  {\normalfont\normalsize\bfseries}
  {\thesubsubsection}{4pt}{}



\usepackage[
backend=biber,
style=apa,   % numeric, authoryear, chicago 등
sorting=nyt
]{biblatex}

\addbibresource{auto/refer.bib}

\setlength{\bibhang}{40pt}




\renewcommand{\thefootnote}{\arabic{footnote})}

\makeatletter
\renewcommand\@makefntext[1]{%
  \setlength{\parindent}{0pt}%
  \setlength{\hangindent}{1.5em}%
  \setlength{\hangafter}{1}%
  \footnotesize
  \makebox[1.5em][l]{\thefootnote}#1%
}
\makeatother

% \usepackage{pdflscape}  % 또는 lscape
\usepackage{lscape}



\usepackage{siunitx}

\sisetup{
  group-separator = {,},
  group-minimum-digits = 4,
  round-mode = places,
  round-precision = 0
}




% 날짜가 있어도 "연도만" 보이게 (bibliography + citation 모두)
\AtEveryBibitem{%
  \clearfield{month}%
  \clearfield{day}%
}
\AtEveryCitekey{%
  \clearfield{month}%
  \clearfield{day}%
}
