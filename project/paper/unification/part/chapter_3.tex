\section{\chapterthree}

경제제재의 성패는 제재의 존재 여부나 강도만으로 설명되기 어렵다. 
제재가 반복될수록 목표 달성과의 연결이 약화될 수 있고, 
정밀한 제재 또한 운영 조건이 뒷받침되지 않으면 범위 팽창과 부작용으로 전환될 수 있다. 
따라서 대북제재의 작동을 평가할 때는 제재를 단일 변수로 취급하기보다, 
작동 조건을 구분하여 살펴보는 분석틀이 요구된다. 
여기에서는 효과성, 정당성, 지속가능성이라는 세 축을 설정하고, 
각 축이 제재의 성과를 어떻게 규정하는지와 축 간 연쇄 효과를 체계적으로 정리한다.

효과성은 제재가 유발한 경제적 비용 그 자체가 아니라, 
그 비용이 정책 목표—즉 상대의 행동 변화—로 연결되는지 여부를 문제 삼는다. 
행동 변화는 협상 복귀, 특정 프로그램의 중단 또는 지연, 조달·확산 네트워크의 차단처럼 
여러 형태로 나타날 수 있으며, 각 목표는 서로 다른 관찰 지표와 시간축을 요구한다. 
이 때문에 효과성 평가는 목표의 명료화에서 출발한다. 
그 다음에는 제재가 목표에 도달하는 경로가 무엇인지, 그리고 그 경로가 작동하기 위한 조건이 
현실에서 충족되는지 점검해야 한다. 
핵심 조건은 크게 세 가지로 정리될 수 있다. 
첫째, 요구와 조건이 분명해 상대가 어떤 행동을 해야 하는지 인지 가능한가. 
둘째, 요구가 충족될 경우 완화·해제가 가능하다는 신뢰가 존재하는가. 
셋째, 회피·적응을 억제할 집행 역량과 정보 기반이 구축되어 있는가. 
요컨대 효과성은 “강한 제재냐 약한 제재냐”의 문제가 아니라, 
행동 변화를 유도하는 설계·집행·보상 구조가 실질적으로 작동하는가의 문제로 환원된다.

정당성은 제재가 국제규범과 법적·윤리적 기준에 부합하는지, 
그리고 민간·인도주의 영역에 전가되는 비용을 통제할 수 있는지를 묻는다. 
안보리 결의에 근거한 제재라 하더라도 인권·인도주의·비례성의 논점에서 면책될 수 없으며, 
특히 금융·물류·조달 경로에서 발생하는 병목은 제재의 의도와 무관하게 민간 피해를 확대할 수 있다. 
정당성은 선언적 기준이 아니라 운영의 문제로 측정된다. 
첫째, 목적과 수단이 비례하는가(최소침해 원칙을 충족하는가). 
둘째, 인도주의 예외가 규정 차원을 넘어 실제로 작동하는가(송금·보험·운송·통관의 장애를 포함). 
셋째, 준수 과정의 과잉 회피가 합법적 거래와 인도주의 지원까지 위축시키는지 여부다. 
정당성이 약화되면 국제적 지지와 협력 기반이 약해지고, 이는 집행력 저하를 통해 효과성에도 영향을 미친다.

지속가능성은 제재 체제가 장기간 유지될 수 있는지뿐 아니라, 
환경 변화에 맞춰 조정·교정될 수 있는 제도적 기반을 갖추었는지를 포괄한다. 
제재는 일회적 조치가 아니라 회피 방식의 진화에 따라 지속적으로 업데이트되어야 하고, 
부작용이 확인되면 교정되어야 하며, 목표 달성 가능성과 연동된 완화·해제 설계도 필요하다. 
지속가능성을 구성하는 핵심 요소는 다음과 같다. 
첫째, 국제적 합의와 공조가 안정적으로 유지되어 집행의 일관성이 담보되는가. 
둘째, 위반·회피를 탐지하고 공유하는 감시·정보 기반이 확보되어 있는가. 
셋째, 집행 비용과 행정 역량을 감당할 수 있는가(정밀한 제재일수록 요구되는 역량이 커진다). 
넷째, 정기적 평가와 수정·종료가 가능한 절차적 장치가 존재하는가. 
지속가능성이 약화되면 회피 억제가 느슨해져 효과성이 떨어지고, 
인도주의 예외의 실행도 어려워져 정당성 논란이 증폭되는 연쇄 효과가 발생할 수 있다.

세 축은 상호 독립적이지 않다. 
완화·해제의 신뢰가 약하면 효과성은 낮아지고, 낮은 효과성은 제재의 누적과 상징정치화를 촉진할 수 있다. 
제재가 누적될수록 민간 피해와 과잉 준수가 확대되어 정당성이 약화되고, 
정당성이 약화될수록 국제공조가 흔들리며 지속가능성이 떨어질 수 있다. 
또한 감시·정보 기반이 약해지면 회피가 쉬워져 효과성이 약화되고, 
필요한 예외를 조정·교정하는 능력도 떨어져 정당성 문제가 악화될 수 있다. 
따라서 대북제재의 평가에서는 세 축을 동시에 놓고 
취약 지점이 어디에서 발생하며 그것이 다른 축으로 어떻게 전이되는지를 함께 분석할 필요가 있다.
