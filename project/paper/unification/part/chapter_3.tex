\section{\chapterone}
3장. 분석틀: 효과성·정당성·지속가능성

앞선 2장은 제재를 둘러싼 논쟁이 단순한 “제재 찬반”이 아니라, 제재가 실제로 작동하는 조건—목표 달성 가능성, 부작용과 규범적 한계, 그리고 제재 체제의 운영·조정 능력—을 둘러싼 경쟁적 주장임을 보여주었다. 따라서 본 논문은 대북제재를 평가할 때 (1) 제재가 목표 달성에 기여하는가, (2) 그 과정에서 허용 가능한 비용과 한계를 넘지 않는가, (3) 제재 체제가 장기적으로 유지·조정될 수 있는가라는 세 가지 질문을 동시에 다루는 분석틀이 필요하다고 본다. 이 장은 이를 위해 효과성(effectiveness), 정당성(legitimacy/legality), **지속가능성(institutional sustainability/robustness)**의 3축을 제시하고, 각 축에서 무엇을 관찰하고 어떤 판단 기준을 적용할지를 정리한다. 세 축은 서로 독립적이라기보다 상호의존적이며, 한 축의 취약성은 다른 축의 성과를 잠식할 수 있다는 점에서 함께 검토되어야 한다.

3.1 효과성: ‘위력’이 아니라 ‘목표 달성’의 관점

효과성 축은 제재가 실제로 목표한 정책 결과를 만들어내는지에 대한 평가이다. 여기서 중요한 것은 제재가 대상국 경제에 “손해”를 주었는지(위력, potency)가 아니라, 그 손해가 정치적 행동 변화로 연결되었는지(목표 달성, effectiveness)이다. 대북제재의 목표는 일반적으로 비핵화 협상 진전, 도발 억제, 자금·물자 조달 차단 등으로 제시되지만, 이 목표들 사이에도 우선순위와 시간축이 다를 수 있다. 따라서 본 논문은 효과성을 평가할 때 먼저 “어떤 목표를 기준으로 할 것인가”를 분명히 하고, 다음으로 그 목표에 대한 제재의 기여 경로를 검토한다.

효과성을 구성하는 핵심 관찰 포인트는 세 가지다. 첫째, 명확한 요구와 조건이 존재하는가이다. 제재는 상대가 무엇을 해야 하는지, 무엇을 하면 제재가 완화·해제되는지를 이해할 때 협상 유인으로 기능한다. 요구가 과도하게 포괄적이거나, 조건이 모호하거나, 정치적으로 해제가 불가능해 보이면 제재는 협상 수단이 아니라 장기 봉쇄로 인식될 가능성이 커진다. 둘째, **해제 신뢰성(off-ramp)**이 확보되는가이다. 제재가 행동 변화를 유도하려면 “양보하면 보상이 있다”는 신호가 신뢰 가능해야 한다. 해제 신뢰성이 약하면 대상국은 비용을 감수하고 버티거나 회피 전략을 강화할 동기가 커진다. 셋째, 대상국의 적응과 회피 가능성이다. 제재는 시간이 지날수록 회피 네트워크가 학습되고 정교해지는 경향이 있으며, 이는 제재 효과를 감소시키는 대표적 요인이다. 따라서 효과성 평가는 제재의 설계뿐 아니라 집행·감시·정보 기반이 얼마나 회피를 억제하는지도 함께 포함한다.

정리하면, 본 논문에서 효과성은 “제재로 경제적 비용이 발생했는가”가 아니라, “제재가 행동 변화를 만들어내는 유인을 제공했고, 그 유인이 실제로 작동했는가”라는 질문으로 이해된다. 이때 해제 신뢰성과 집행 능력은 효과성의 하위 조건으로 포함되며, 단순히 부수적 요소로 취급되지 않는다.

3.2 정당성: 법적·규범적 한계와 인도주의 비용

정당성 축은 제재가 국제규범과 법적·윤리적 기준에 부합하는지, 그리고 민간·인도주의 영역에 대한 부작용을 통제할 수 있는지를 평가한다. 코논도치의 논지가 보여주듯, UN 안보리 제재는 강력한 권한 아래 시행되더라도 국제인권법, 국제인도법, 비례성 원칙 등에서 완전히 자유로울 수 없으며, 특히 장기·포괄적 제재가 민간의 생존권과 인권에 미치는 영향은 정당성 논쟁의 핵심이 된다. 따라서 정당성 축은 제재의 목적과 수단 사이의 비례성, 민간 피해 최소화, 인도주의 지원의 접근성과 예외 규정의 실효성을 중심으로 구성된다.

정당성의 관찰 포인트 역시 세 가지로 정리할 수 있다. 첫째, 비례성과 최소침해이다. 제재가 추구하는 안보 목표에 비해 민간이 감수해야 하는 피해가 과도한지, 혹은 피해를 줄일 수 있는 대안적 설계가 존재했는지를 검토한다. 둘째, 인도주의 예외의 실효성이다. 많은 제재 체제는 인도주의 목적의 예외를 규정하지만, 실제로는 금융·운송·보험·통관 등의 병목 때문에 인도주의 지원이 지연되거나 위축될 수 있다. 특히 은행이 준법 리스크를 이유로 거래 자체를 회피하면(과잉 준수), 법적으로 허용된 인도주의 거래도 사실상 불가능해질 수 있다. 셋째, **제재가 낳는 간접효과(2차적 피해)**이다. 표적 제재를 표방하더라도 금융·물류·제재 준수 비용의 상승은 시장과 생활 전반에 광범위한 영향을 미치며, 이는 정치적 정당성뿐 아니라 국제사회의 지지와 협력에도 영향을 줄 수 있다.

정당성 축은 단지 도덕적 평가를 위한 장치가 아니다. 민간 피해가 누적되면 제재에 대한 국제적 합의가 약해지고, 인도주의 채널이 막히면 국제기구와 NGO의 활동이 위축되며, 이는 결과적으로 제재의 집행력과 지속가능성을 약화시킬 수 있다. 즉 정당성은 효과성과 분리된 “부차적 고려”가 아니라, 제재의 기능 자체를 뒷받침하는 조건으로 이해될 필요가 있다.

3.3 지속가능성: 감시·집행·조정이 가능한 제도적 기반

지속가능성 축은 제재 체제가 장기간에 걸쳐 유지되고, 환경 변화에 따라 조정되며, 국제적 협력을 통해 집행될 수 있는 제도적 기반을 갖추고 있는지를 평가한다. 제재는 일회성 조치가 아니라 반복적으로 업데이트되고, 회피에 대응해 보완되며, 인도주의 예외·완화·해제 같은 조정이 이루어져야 하는 “운영체계”다. 따라서 지속가능성은 제재를 둘러싼 다자협력의 안정성, 정보 생산과 모니터링의 신뢰성, 집행 비용을 감당할 역량, 그리고 제재의 정기 평가·재설계 메커니즘을 포함한다.

지속가능성의 핵심 관찰 포인트는 네 가지다. 첫째, 다자협력의 결속과 정치적 안정성이다. 제재는 국제공조에 기반할 때 회피 비용이 올라가며, 제재 목표와 집행 방식에 대한 합의가 약해질수록 제재는 상징적 조치로 전락하거나 지역별로 파편화될 수 있다. 둘째, 감시·정보 기반의 신뢰성이다. 제재 위반과 회피를 탐지하고 공유할 수 있어야 집행이 가능하고, 제재가 목표를 향해 조정될 수 있다. 감시·보고 기능이 약화되면 무엇이 위반인지, 어떤 보완이 필요한지에 대한 공통 인식이 약해지고, 집행은 국가별 재량과 정치적 판단에 더 의존하게 된다. 셋째, 집행 역량과 비용이다. 금융·해운·무역 네트워크를 겨냥하는 제재일수록 기술적·행정적 집행 역량이 필요하며, 역량이 부족하면 “정교한 제재”는 선언에 그칠 수 있다. 넷째, 검토·일몰·조정 메커니즘이다. 제재가 적절한지, 부작용이 과도하지 않은지, 목표 달성에 기여하는지 정기적으로 검토하고 필요하면 수정·종료할 수 있어야 한다. 그렇지 않으면 제재는 누적되며 ‘풀 수 없는 정책’이 되고, 이는 해제 신뢰성을 약화시켜 효과성에도 부정적 영향을 준다.

지속가능성은 특히 ‘스마트 제재’ 논쟁과 직접 연결된다. 스마트 제재가 현실에서 의미를 가지려면, 단순히 표적을 좁히는 설계만으로는 부족하고, 그 표적성을 유지할 감시·정보 체계와 국제공조, 그리고 운영 과정에서 발생하는 부작용을 교정할 제도적 장치가 함께 작동해야 한다. 다시 말해, 지속가능성은 스마트 제재를 가능하게 하는 “바탕 조건”이다.

3.4 세 축의 상호작용과 분석틀의 적용 방식

본 논문이 제시하는 3축 분석틀은 각 축을 따로 점수화하려는 것이 아니라, 세 축이 서로 어떻게 영향을 주는지를 설명하는 데 목적이 있다. 효과성이 약하면 제재는 상징정치로 전락하여 남용될 수 있고, 남용된 제재는 정당성 문제와 국제적 반발을 낳아 지속가능성을 훼손한다. 반대로 정당성 문제가 누적되면 국제공조가 약해지고, 이는 집행력 약화로 이어져 효과성도 떨어진다. 또한 감시·집행 체계가 흔들리면 회피가 쉬워져 효과성이 감소하고, 동시에 인도주의 예외의 실행도 어렵게 만들어 정당성 논란이 커질 수 있다. 즉, 제재의 성패는 세 축이 동시에 일정 수준을 충족하는지에 달려 있다.

이 분석틀은 다음 장에서 대북제재 사례에 적용된다. 구체적으로는 (1) 제재가 목표 달성(비핵화 협상 유인, 조달 차단, 억지 등)에 어떤 경로로 기여하는지와 해제 신뢰성의 조건을 검토하고(효과성), (2) 인도주의·민간 피해와 법적·규범적 한계를 어떻게 통제할 수 있는지를 살피며(정당성), (3) 감시·집행·조정 체계가 장기적으로 작동 가능한지, 그리고 그 변화가 제재 운영 조건을 어떻게 바꾸는지를 검토할 것이다(지속가능성). 이를 통해 본 논문은 대북제재를 “강화냐 완화냐”의 단순한 선택이 아니라, 효과성·정당성·지속가능성을 함께 만족시키는 운영 재설계의 문제로 재정의하고자 한다.