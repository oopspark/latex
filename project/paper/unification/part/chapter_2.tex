\section{\chapterone}

2장. 이론적 논쟁: 제재의 과잉 사용 vs ‘스마트 제재’

이 장의 목적은 대북제재를 곧바로 평가하거나 검증하기보다, 세 편의 논문이 제기하는 쟁점을 하나의 논쟁 구조로 정리하는 데 있다. 핵심 질문은 세 가지로 요약된다. 첫째, 제재는 왜 반복적으로 선택되는 “쉬운 도구”가 되었는가. 둘째, 그 대안으로 제시되는 ‘스마트(표적) 제재’는 무엇을 의미하며, 어떤 조건에서 비용을 줄이면서 성과를 낼 수 있다고 주장되는가. 셋째, 표적 제재는 실제 정책 환경에서 왜 다시 “덜 표적화된” 조치로 팽창할 위험이 있는가. 이 논쟁을 정리하면, 이후 장에서 제재를 평가할 기준(효과성·정당성·지속가능성)을 도출할 수 있다.

2.1 제재의 과잉 사용: 왜 ‘쉽게 선택되는 도구’가 되었나(드레즈너)

드레즈너는 최근 미국 외교에서 경제제재가 일종의 기본값(default tool)처럼 사용되는 현상을 비판적으로 진단한다. 그의 문제의식은 “제재가 강력해졌다”는 사실 자체가 “제재가 효과적이다”라는 결론으로 곧장 이어지지 않는다는 데 있다. 그는 제재가 과잉 사용되는 배경을 크게 두 방향에서 설명한다.

첫째, 다른 외교정책 수단들이 정치적으로 어려워지면서 제재가 상대적으로 선택하기 쉬운 수단이 되었다는 점이다. 군사개입은 인명과 재정 부담이 크고, 국내 여론의 피로감도 누적되어 있다. 반면 원조나 무역 확대 같은 포지티브 인센티브는 정치적 비호감을 받기 쉽다. 이런 환경에서 제재는 “강경하게 보이면서도” 즉각적인 행동을 할 수 있고, 정치적으로도 ‘대응했다’는 메시지를 주기 때문에 정책결정자에게 매력적인 선택지가 된다는 것이다.

둘째, 세계화와 달러 중심 금융 네트워크가 제재의 잠재적 위력을 키웠다는 점이다. 국제결제와 금융거래에서 미국이 차지하는 중심성은 제재를 “경제적 고립”에 가깝게 만들 수 있고, 특히 금융부문 제재는 민간 금융기관이 규제·평판 리스크를 우려해 스스로 위험을 회피하는 ‘과잉 준수(overcompliance)’를 유발하기도 한다. 그러나 드레즈너의 요지는 이 잠재적 위력이 곧 정책 성과를 보장하지 않는다는 데 있다. 그는 “최대 압박”처럼 2차 제재까지 동원하는 강력한 조치가 대상국의 양보를 끌어내지 못한 사례들이 누적되고 있음을 상기시킨다.

따라서 드레즈너의 처방은 “제재를 더 정교하게 하자”라기보다, 먼저 “제재를 덜 자주, 더 전략적으로 사용해야 한다”는 방향에 가깝다. 그는 제재가 남발될수록 제재 체제 전체의 효율성이 약화될 수 있고, 무엇보다 행동 변화를 유도하려면 제재가 어떤 요구를 담고 있는지, 무엇을 하면 해제되는지(완화·해제 경로)가 명확해야 한다고 강조한다. 즉, 제재는 위협–부과–해제의 논리가 분명할 때 협상 유인으로 작동할 수 있다.

2.2 ‘스마트 제재’ 옹호: 표적화·테마화·네트워크 타격(구즈도프스카·프렌더가스트)

구즈도프스카와 프렌더가스트는 드레즈너의 진단(제재의 남용, 비용의 존재)에 상당 부분 동의하지만, 결론은 더 낙관적이다. 그들은 드레즈너가 주로 비판하는 대상이 포괄적 국가제재이며, 오늘날에는 훨씬 더 정밀한 제재 도구들이 발전해 왔다고 본다. 예컨대 대테러, 비확산, 마약, 인권, 반부패 등 특정 테마를 중심으로 설계된 제재 프로그램은 국가 전체를 봉쇄하기보다 특정 행위자와 행위를 겨냥하는 방식으로 발전해 왔다는 것이다.

그들이 말하는 ‘스마트’의 요점은 두 가지다. 첫째, 표적 제재는 개인·기업·조직 등 특정 행위자에게 초점을 맞추므로, 포괄 제재보다 민간에 대한 부수적 피해를 줄일 수 있다는 주장이다. 둘째, 표적성은 개인 수준에 머물러서는 안 되고, 불법 금융·조달·운송·중개를 가능하게 하는 “네트워크”를 함께 겨냥해야 실효성이 생긴다는 주장이다. 제재 대상국의 핵심 인물이나 기관을 지정하더라도, 그들이 제3국의 중개자나 페이퍼컴퍼니를 통해 우회하면 제재는 무력화되기 때문이다.

다만 이들은 제재를 만능 도구로 보지 않는다. 오히려 제재가 효과를 내려면 외교적 협상, 기술적 지원(금융기관의 준수 능력 강화 등), 국제공조, 민관 협력 같은 더 넓은 전략 패키지의 일부로 운용되어야 한다고 강조한다. 즉 스마트 제재는 “덜 쓰는 제재”만큼이나 “잘 쓰는 제재”가 가능하며, 이미 국제정치에서 그 방향으로의 진화가 진행되어 왔다고 보는 셈이다.

2.3 스마트 제재의 역설: 표적화는 왜 ‘덜 표적화’되는가(드레즈너의 답변)

이에 대해 드레즈너는 스마트 제재의 이상 자체를 부정하지 않지만, 현실의 정책 환경에서 그 정교함이 유지되기 어렵다는 점을 강조한다. 그의 반론은 크게 세 가지로 정리할 수 있다.

첫째, 제재가 다른 정책수단을 대체하면서 외교적 역량과 보조수단이 오히려 약화되었다는 점이다. 스마트 제재가 제대로 작동하려면 정보 역량, 외교 협상력, 동맹 및 파트너와의 조율, 금융·법 집행 능력 같은 복합 역량이 요구된다. 그러나 현실에서는 제재가 손쉬운 대체재가 되면서, 그런 역량을 지속적으로 투자·축적하기보다는 “제재 카드”에 더 의존하는 방향으로 정책이 기울었다는 것이다.

둘째, 제재의 과잉 사용은 ‘해제의 신뢰성’을 갉아먹는다. 제재가 국내정치적으로 풀기 어려운 구조로 쌓이고, 목표가 과도하게 확장되며, 완화의 조건이 모호해지면 상대는 “어차피 풀리지 않을 제재”라고 판단할 가능성이 커진다. 그 결과 제재는 행동 변화를 유도하는 협상 수단이 아니라, 장기적 대치의 상징이 되며 협상 유인을 약화시킬 수 있다.

셋째이자 핵심으로, 표적 제재는 시간이 지나면서 덜 표적화될 위험이 있다는 점이다. 네트워크를 겨냥해야 한다는 논리는 설득력 있어 보이지만, 그 논리를 확장할수록 점점 더 많은 중개자·서비스 제공자·관련 부문이 제재의 범위에 들어오게 된다. 결국 표적 제재와 포괄 제재의 경계가 흐려지고, 포괄 제재에 준하는 경제적 충격이 발생할 수 있다는 것이다. 드레즈너는 이런 과정이 인도주의적 비용을 확대할 수 있다고 경고한다. 또한 “상대의 삶에 마찰을 주는 것만으로도 의미가 있다”는 주장에 대해서도, 행동 변화가 없다면 그것은 목표 없는 압박, 즉 ‘제재를 위한 제재’로 전락할 위험이 있다고 비판한다.

2.4 소결: 본 논문의 분석 초점으로 연결

세 논문이 공유하는 출발점은 제재가 비용을 가진 도구이며 남용될 수 있다는 점이다. 그러나 구즈도프스카·프렌더가스트는 제재가 이미 테마형·네트워크형으로 진화하면서 더 정밀하게 운용될 여지가 크다고 보고, 드레즈너는 그 정교함을 유지할 역량과 제도적 조건이 취약해 표적 제재 역시 시간이 지나면 팽창하고 해제 신뢰성이 약화될 수 있다고 본다. 코논도치의 논의는 여기에 법적·규범적 차원의 제한과 인도주의적 비용 문제를 결합시킨다. 즉, 제재는 안보리 권한 아래 시행되더라도 인권·인도주의·비례성의 한계를 고려해야 하며, 실제 운영 과정에서 인도주의 지원과 민간 삶에 미치는 간접적 영향을 통제하지 못하면 정당성과 지속가능성이 훼손될 수 있다는 것이다.

이 장의 정리는 이후 분석틀 설정으로 이어진다. 첫째, 제재는 ‘위력’이 아니라 ‘목표 달성’의 관점에서 평가되어야 한다. 둘째, 정당성 문제(특히 인도주의적 비용)는 부차적 이슈가 아니라 제재의 지속가능성과 국제공조에 직결되는 핵심 변수다. 셋째, ‘스마트 제재’는 단지 라벨이 아니라 정보·집행 역량, 국제공조, 인도주의 안전장치, 해제 경로 설계 같은 운영 조건이 갖춰질 때에만 실질적 의미를 가진다. 다음 장에서는 이러한 논쟁을 바탕으로, 대북제재를 평가하기 위한 분석틀로서 효과성·정당성·지속가능성의 3축을 제시한다.