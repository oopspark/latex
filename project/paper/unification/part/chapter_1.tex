\section{\chapterone}

\subsection{연구 배경: 글로벌 곡물 무역 환경의 구조적 변화}
1. 서론: 문제제기와 연구질문

대북제재는 북한의 핵·미사일 개발을 억제하고 비핵화 협상을 견인하기 위한 대표적 정책수단으로 자리해 왔다. 그러나 제재가 강화될수록 목표(행동 변화)와 결과(지속되는 개발, 회피·적응, 인도주의적 부작용) 사이의 간극이 반복적으로 확인되면서, “제재는 무엇을 할 수 있고 무엇을 할 수 없는가”라는 질문이 다시 중요해지고 있다. 코논도치가 지적하듯 대북제재는 정치·윤리·법적 논쟁이 중층적으로 얽혀 있는 가장 포괄적이고 복잡한 제재 체제 중 하나이며, 실제로 주민의 인권·생계·의료 접근에 부정적 영향을 줄 수 있다는 경고도 꾸준히 제기되어 왔다. 

이 논문이 주목하는 지점은 제재의 “강도” 자체보다, 제재가 작동하는 운영 조건이다. 제재는 부과 순간에 완결되는 조치가 아니라, (1) 목표를 명확히 정의하고, (2) 집행·감시를 통해 위반과 회피를 줄이며, (3) 조건 충족 시 해제될 수 있다는 신뢰를 제공할 때 비로소 ‘행동 변화’로 이어질 가능성이 생긴다. 드레즈너는 경제적 강압이 가장 잘 작동하는 조건으로 제재의 위협·부과·해제 조건이 모호하지 않아야 하며, 요구가 충족되면 신속히 해제되어 교차국경 교류가 재개되어야 한다는 점을 강조한다.  또한 행동 변화를 유도하는(강요형, compellence) 제재는 “완화가 실제로 가능하다”는 신호를 제공할 때 순응 가능성이 커진다고 설명한다. 반면 현실에서 제재는 종종 “쉽게 선택되는 도구”가 되면서, 목표·수단·해제 경로가 정교하게 설계되지 못한 채 축적되는 경향이 있다. 드레즈너는 제재가 과잉 사용될수록 전체 효과가 약화될 수 있으며, 제재의 인도주의적 효과를 포함한 정기적 검토와 일몰(sunset) 같은 제도적 장치가 필요하다고 주장한다. 

즉, 제재의 문제는 ‘제재냐 아니냐’가 아니라, 목표에 맞는 제재 설계·운영·종료의 거버넌스가 갖춰져 있는지에 달려 있다.

이 논점은 ‘스마트 제재’ 논쟁에서 더욱 분명해진다. 구즈도프스카·프렌더가스트는 드레즈너의 회의론을 일정 부분 인정하면서도, 포괄적 국가제재만이 아니라 테마형(대테러·비확산·인권·반부패 등) 표적 제재처럼 더 정밀한 도구들이 발전해 왔고, 이는 특정 행위자에 초점을 맞추어 부수적 피해를 줄일 수 있다고 본다. 

나아가 “표적 개인만이 아니라 네트워크 전체”를 제재하는 접근이 중요하며, 이런 제재는 단독으로가 아니라 지속적 외교·기술지원·민관협력 등 더 넓은 전략 속에서 운용되어야 한다고 주장한다.  그러나 드레즈너의 재반론은 ‘스마트’가 현실에서 쉽게 유지되지 않는다는 점을 지적한다. 미국이 제재에 과도하게 의존하는 가운데 대체 정책수단이 약화되었고, 무엇보다 표적 제재는 시간이 지나며 덜 표적화되어 포괄적 조치로 “변형”되면서 인도주의적 고통을 유발할 수 있다는 것이다.  이 비판은 대북제재처럼 장기·다층 제재가 누적된 사례에서 특히 설득력을 갖는다. 실제로 코논도치는 UN 제재가 안보리 제7장에 근거하더라도 법적 한계가 존재하며, 국제인권법·국제인도법·비례성 원칙 등으로부터 완전히 자유로울 수 없음을 강조한다.  또한 국제금융 통제는 UN 현장 활동을 둔화시켜 식량·보건 키트 등 인도주의 지원 전달에 악영향을 줄 수 있으며, 면제 지연·은행 채널 붕괴·통관 지연 등 운영상의 병목이 반복될 수 있다고 지적한다.  이처럼 제재의 효과성과 정당성, 그리고 제도적 지속가능성은 서로 분리된 문제가 아니다. 효과가 약한 제재는 “무언가를 했다”는 상징정치로 전락할 위험이 있고(드레즈너의 ‘제재를 위한 제재’ 우려),  인도주의적 비용이 누적될수록 정당성이 흔들리며, 정당성이 약해질수록 국제공조와 집행력 또한 약화될 수 있다. 더구나 감시·보고 체계가 약화되면 위반과 회피를 둘러싼 정보의 신뢰성이 떨어지고, 이는 제재 효과에 대한 판단과 조정(완화·강화·예외 설계)을 더욱 어렵게 만든다. 예를 들어 2024년 3월 러시아의 거부권 행사로 UN 대북제재 전문가패널(PoE) 임기 연장이 무산되었다는 사실은, 제재의 집행·감시 거버넌스가 정치적 변동에 취약할 수 있음을 보여준다. 

(PoE 공백 이후 일부 국가들이 MSMT 같은 대체 감시 틀을 논의·추진한 것은, 본 논문의 핵심 분석 대상은 아니지만, ‘감시 구조 변화’가 제재 운영 조건을 바꿀 수 있음을 시사하는 배경 사례로만 간략히 언급한다.)

이 문제의식에서 본 논문은 다음의 연구질문을 제시한다.
첫째, 대북제재를 둘러싼 ‘스마트 제재’ 논쟁은 무엇을 쟁점화하며, 그 쟁점은 효과성·정당성·지속가능성의 관점에서 어떻게 정리될 수 있는가? (드레즈너 vs 구즈도프스카·프렌더가스트의 논쟁을 중심으로) 

 
둘째, UN 대북제재의 법적 한계와 인도주의 문제를 고려할 때, 대북제재의 운영 조건(해제 경로, 인도주의 안전장치, 감시·집행 체계)은 어떤 취약점을 갖는가?  본 논문의 기본 주장은 다음과 같다. 대북제재의 성패는 제재의 강도보다 운영 조건에 의해 좌우되며, 그 운영 조건은 (1) 행동 변화 유인을 강화하는 해제 신뢰성(off-ramp),  (2) 국제인권법·인도주의 원칙·비례성에 부합하는 인도주의 안전장치,  (3) 감시·집행·조정이 가능하도록 하는 제도적 지속가능성으로 요약될 수 있다. 그런데 제재가 장기화되고, 표적 제재조차 시간이 흐르며 덜 표적화되는 경향이 나타나며,  감시·보고 기반이 흔들릴수록(예: PoE 공백과 같은 사건) 이 세 조건은 동시에 취약해질 가능성이 크다.  따라서 향후 과제는 “제재 강화/완화”의 이분법이 아니라, 효과성·정당성·지속가능성을 함께 만족시키는 운영 재설계(정기 검토·일몰·해제 절차 명료화·인도주의 면제의 상시화 등)를 구체화하는 데 있다. 이후 논문 구성은 다음과 같다. 2장에서는 제재 과잉 사용과 해제 신뢰성(드레즈너), 스마트/표적·테마형 제재 및 네트워크 접근(구즈도프스카·프렌더가스트), 그리고 표적 제재의 팽창·역량 한계·인도주의 피해 가능성(드레즈너의 재반론)을 정리한다.   3장에서는 이를 바탕으로 효과성·정당성·지속가능성의 분석틀을 제시한다. 4장에서는 이 분석틀을 대북제재에 적용하여 법적 한계와 인도주의 문제를 포함한 핵심 쟁점과 정책적 과제를 도출한다. 