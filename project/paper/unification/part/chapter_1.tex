\section{\chapterone}

대북제재는 북한의 핵·미사일 개발을 억제하고 협상을 유도하기 위한 대표적 정책수단으로 기능해 왔다. 
그러나 제재가 반복적으로 강화되는 동안에도 핵·미사일 개발은 지속되었고, 
제재가 만들어낸 경제적 압박이 곧바로 정책적 양보로 이어지지 않는다는 점이 누적적으로 드러났다. 
이 때문에 대북제재 논의는 “제재를 더 강하게 해야 한다/완화해야 한다”라는 단순한 처방을 넘어서, 
제재가 실제로 어떤 조건에서 작동하고 어떤 경로에서 한계에 부딪히는지를 점검하는 방향으로 이동할 필요가 있다. 
특히 대북제재는 결의가 누적되면서 제재 범위가 확장되고, 
집행과 준수 비용이 금융·물류·조달 전반으로 파급될 수 있는 구조를 갖는다. 
이 점은 제재를 단일 조치라기보다 시간이 흐르며 성격이 변할 수 있는 운영체계로 보아야 함을 시사한다.

경제제재가 외교정책에서 과잉 사용될수록 ‘압박의 크기’가 ‘성과’로 오인되고, 
제재는 행동 변화를 유도하는 수단이라기보다 정치적 신호로 소비될 위험이 커진다. 
이러한 문제의식은 \textcite{dreznerUnitedStatesSanctions2021}의 논의와 맞닿아 있다. 
핵심은 제재의 강도 자체가 아니라, 행동 변화를 유도할 수 있는 설계와 운영이 가능한가에 있다. 
제재가 협상 유인으로 기능하려면 요구 조건과 완화·해제의 경로가 명확해야 하고, 
회피와 적응을 억제할 감시·집행 역량이 뒷받침되어야 하며, 
제재 준수 과정에서 발생하는 인도주의적 비용과 민간 피해를 통제할 안전장치가 실제로 작동해야 한다. 
해제 가능성이 신뢰되지 않으면 제재는 협상 유인이 아니라 장기 봉쇄로 인식될 수 있고, 
회피 억제가 약하면 제재는 상징적 압박으로 전락할 수 있다. 
또한 인도주의 비용이 누적되면 제재의 정당성과 국제적 지지 기반이 흔들려 집행력 자체가 약화될 수 있다.

한편 포괄적 제재의 부작용을 줄이기 위해 표적화·테마화·네트워크 차단과 같은 정밀한 접근이 대안으로 제시되어 왔다. 
예컨대 제재 대상을 국가 전체가 아니라 특정 행위자와 그들의 조달·자금·운송 네트워크로 좁히는 구상은 
부수적 피해를 줄이면서도 효과를 높일 수 있다는 기대를 낳았다\parencite{gudzowskajustynaCanSanctionsBe2022}. 
다만 이러한 정밀성은 별도의 운영 역량과 규칙이 없다면 시간이 지나며 범위 팽창과 민간 피해로 전환될 위험이 있다. 
정밀한 설계가 실제 집행으로 이어지지 못하거나, 준수 리스크가 과도하게 부풀려지면서 
광범위한 거래 위축과 과잉 준수가 나타나면 제재의 정당성과 지속성이 동시에 흔들릴 수 있다.

UN 차원의 대북제재는 법적 구속력이 있다는 강점을 갖지만, 
그럼에도 인도주의나 비례성 원칙의 측면에서 한계를 지니고 있으며, 
특히 금융·물류 경로에서 발생하는 병목은 인도주의 예외 규정을 형식화시킬 수 있다. 
대북제재의 인도주의적 비용과 규범적 쟁점을 둘러싼 이러한 논점은 
\textcite{borisLimitsSanctionsInternational2024}가 강조하는 문제의식과 연결된다. 
결국 대북제재의 평가 기준은 “얼마나 아프게 했는가”가 아니라 
“행동 변화를 유도할 수 있는 운영 조건을 갖추었는가”로 이동할 필요가 있다.

또한 제재의 실질적 작동은 규범과 문구만으로 결정되지 않고, 
감시·보고·집행 환경의 변화에 따라 달라질 수 있다. 
예컨대 최근 제재 감시 체계에 변동이 발생하면서 일부 국가들이 대체적 감시·보고 틀을 모색한 흐름은, 
같은 제재 문구 아래에서도 집행과 준수의 조건이 달라질 수 있음을 보여준다.
