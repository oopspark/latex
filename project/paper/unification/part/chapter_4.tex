\section{\chapterfour}

대북제재는 결의 조항의 나열만으로 파악하기 어렵다. 
제재는 특정 사건을 계기로 강화되거나 조정되며, 그 과정에서 협상 유인으로 작동하기도 하고 
회피·적응을 둘러싼 통제 경쟁의 성격을 띠기도 한다. 
또한 금융·물류 경로에서 병목이 발생할 경우 인도주의 영역과 민간경제에 간접 비용이 전가될 수 있다. 
따라서 대북제재의 평가는 제재의 유무나 형식적 강도보다, 사건의 전개 속에서 제재가 어떤 경로로 작동했고 
어떤 제약이 노출되었는지에 초점을 둘 필요가 있다.

2017년은 고강도 결의가 집중적으로 채택되며 제재의 물질적 강도가 상승한 시기였다. 
2017년 말 안보리 결의 2397호는 원유 공급 상한을 도입하고, 정제유 공급을 연간 50만 배럴로 제한하는 등 
에너지 제약을 제도화하였다\footnotemark. 이러한 조치는 제재의 압박 능력을 확대했으나, 
압박의 확대가 곧바로 목표한 행동 변화로 귀결되는지는 별도의 문제로 남는다. 
즉 강도 증대는 관찰 가능한 정책 변화이지만, 정책적 양보나 협상 진전과의 인과 연결은 자동적으로 성립하지 않는다.
\footnotetext{\fullcite{unitednationssecuritycouncilResolution239720172017}}


이와 관련하여 2019년 하노이 북미 정상회담 결렬은 제재가 
협상 수단으로 작동할 때의 취약성을 드러낸 사건으로 해석될 수 있다\footnotemark.
당시 북한은 영변 시설 폐기 등 특정 조치와 맞교환으로 유엔 제재 일부의 해제 또는 완화를 요구한 것으로 알려졌고, 
이는 제재가 협상 유인으로 기능하기 위해 완화·해제 경로의 신뢰성이 중요함을 시사한다. 
상대가 어느 정도의 조치가 어느 수준의 완화로 연결되는지 예측하기 어렵다면, 
제재는 협상 촉진 장치라기보다 불신을 증폭시키는 요인으로 전환될 가능성이 커진다. 
또한 동일 사건을 두고 요구 범위에 대한 서사가 엇갈릴 경우, 
완화의 조건과 범위를 둘러싼 인식 차이가 협상 구조를 취약하게 만들 수 있다.
\footnotetext{\fullcite{ijeongceolBugmiDaeribgwaCinseooegyo2025}}


제재가 장기화될수록 회피·적응의 확산은 핵심 변수로 부상한다. 
대북제재는 시간이 지날수록 ‘무엇을 금지했는가’만큼이나 ‘어떻게 회피했는가’가 중요해지며, 
대표적으로 해상 환적(Ship-to-ship transfer)을 통한 석유·석탄 거래 의혹은 
감시·집행 역량의 중요성을 부각시켜 왔다\footnotemark.
회피를 억제하기 위해 네트워크를 폭넓게 겨냥할수록 제재의 표적성이 강화되기보다 
거래 전반의 위축으로 이어질 위험도 존재한다. 
회피 억제 강화는 효과성을 높일 수 있는 한편, 정당성과 지속가능성의 비용을 증대시킬 수 있는 딜레마를 내포한다.
\footnotetext{\fullcite{caesuranGugjesahoeyiDaebugjejaeDonghyanggwa2024}}


정당성 문제는 선언적 규범 논쟁을 넘어 금융·물류 병목이라는 운영 차원에서 구체화된다. 
예컨대 2019년 안보리 문서(S/2019/691)는 인도주의 면제 처리 시간이 개선되는 경향을 언급하면서도, 
은행 채널이 복구되지 않아 유엔 및 인도주의 단체의 지속적 운영이 방해받을 수 있음을 지적하였다\footnotemark. 
이는 인도주의 예외 규정의 존재만으로는 충분하지 않으며, 
결제·송금·보험·운송 등 실무 인프라가 뒷받침되어야 예외가 현실에서 작동함을 보여준다. 
또한 규정 위반 위험이 과도하게 인식될 경우, 법적으로 허용된 거래조차 ‘확실하지 않으면 하지 않는’ 방향으로 수렴하며 
과잉 준수가 구조화될 수 있다. 그 결과 정당성의 쟁점은 목적의 정당성 여부를 넘어, 
예외를 실행 가능하게 만드는 운영 장치의 존재와 성능으로 이동한다.
\footnotetext{\fullcite{unitednationssecuritycouncilReportPanelExperts}}


지속가능성은 제재 체제의 감시·보고 기반이 변화할 때 가장 직접적으로 흔들리는 차원이다. 
2024년 3월 안보리 대북제재 감시를 담당해 오던 전문가패널(PoE)의 임기 연장이 무산\footnotemark 되면서
감시·보고 기반이 약화되었다는 평가가 제기되었다.
이후 일부 국가들은 대체적 감시·보고 체계를 모색하였고, 
유사입장국 중심의 다자 제재 모니터링 팀(Multilateral Sanctions Monitoring Team, MSMT)이 출범하여 
제재 위반·회피 관련 정보를 정리한 보고를 발간하는 흐름이 나타났다. 
이러한 변화는 제재 문구가 유지되더라도 공통 사실기반이 약화될 경우 
집행의 일관성과 신뢰가 흔들릴 수 있음을 시사한다. 
감시·보고 기반의 약화는 회피 억제의 곤란과 인도주의 예외의 조정·교정 능력 저하로 연쇄 전이될 수 있다.
\footnotetext{\fullcite{unitednationsSecurityCouncilFails2024}}


종합하면, 위 사례들은 대북제재의 핵심이 강도 경쟁이 아니라 운영 조건의 결합이라는 점을 보여준다. 
2017년의 제재 강화는 압박 능력의 확대를 보여주었으나 행동 변화를 보장하지는 않았고, 
2019년 하노이 결렬은 완화·해제 경로의 신뢰성이 중요하다는 점을 부각시켰다. 
해상 환적 등 회피 사례는 감시·집행 역량이 결여될 경우 제재가 조항의 촘촘함만으로는 충분히 작동하기 어렵다는 점을 드러냈다. 
또한 금융 채널과 과잉 준수 문제는 인도주의 예외가 실제로 작동하지 않을 때 정당성이 훼손될 수 있음을 시사한다. 
PoE 종료와 MSMT 출범은 감시 거버넌스 변화가 제재 체제의 지속가능성을 좌우할 수 있음을 보여준다.
