\section{\chapterone}

4장. 대북제재 사례 적용: 효과성·정당성·지속가능성의 긴장

3장에서 제시한 3축 분석틀(효과성·정당성·지속가능성)을 대북제재에 적용하면, 대북제재 논쟁의 핵심은 “제재의 강도”가 아니라 제재 체제가 어떤 운영 조건에서 돌아가고 있는가로 수렴한다. 대북제재는 2006년 안보리 결의 1718호를 기점으로 형성된 뒤, 핵실험과 미사일 발사에 대응해 점진적으로 확장되어 왔고, 무기·사치품·특정 품목 수출입뿐 아니라 에너지·노동력·운송 등 광범위한 영역으로 제재 범위가 누적되었다. 

이 장은 (1) 제재가 목표 달성에 기여하는 경로가 무엇인지(효과성), (2) 그 과정에서 법·규범·인도주의적 한계를 어떻게 다뤄야 하는지(정당성), (3) 제재 체제가 장기간 유지·조정 가능한 제도적 기반을 갖추고 있는지(지속가능성)를 차례로 살펴본다.

4.1 대북제재 체제의 기본 구조: 누적·다층·복합 제재의 형성

대북제재는 안보리가 북한의 핵실험을 “국제평화와 안보에 대한 위협”으로 판단하고, 유엔 헌장 제7장(특히 제41조)에 근거한 법적 구속력 있는 제재를 부과하면서 출발했다. 

이후 결의 1718, 1874, 2087, 2094, 2270, 2321, 2356, 2371, 2375, 2397 등 일련의 결의가 축적되며, 무기·이중용도 품목 통제, 자산 동결·여행 금지, 특정 사치품 통제, 주요 수출 품목(석탄·광물·해산물·섬유 등) 제한, 유류 상한, 해외 노동자 송출 제한 등 다층적 제재 패키지가 형성되었다. 중요한 점은 이 체제가 “표적 제재”의 요소(개인·기관 지정)와 “부문 제재”의 요소(에너지·무역·운송 등)를 함께 포함하는 혼합형 제재라는 사실이다. 설계상으로는 “주민을 겨냥하지 않는다”는 점이 강조되었지만, 

시간이 흐르며 제재의 집행과 준수(compliance)가 민간경제 전반의 거래 비용을 끌어올리는 방향으로 작동할 수 있다는 우려가 반복적으로 제기되어 왔다.

4.2 효과성: ‘핵·미사일 행동 변화’에 대한 제재의 한계와 조건

효과성 축에서 가장 먼저 확인되는 사실은, 대북제재의 목표가 통상 “비핵화/도발 억제/조달 차단/협상 유인” 등 여러 층위로 제시되지만, 그중 핵심 목표인 핵·미사일 행동 변화는 매우 제한적으로만 관찰된다는 점이다. 코논도치도 “UN 제재가 북한의 핵정책에 얼마나 영향을 미쳤는지 신뢰성 있게 평가하기는 매우 어렵다”고 전제하면서, 북한이 미사일·핵 개발 태도를 바꾸지 않았다고 지적한다. 

이는 드레즈너가 일반론에서 제기한 문제(고통의 크기=성과가 아니라는 문제)와 맞닿아 있다. 즉, 제재가 “얼마나 아프게 했는가”가 아니라 “행동 변화를 만들었는가”가 효과성의 기준이 되어야 한다는 관점이다. 

대북제재의 효과성을 제한하는 요인은 크게 세 가지로 정리할 수 있다.

첫째, 해제 신뢰성(off-ramp)의 약화다. 드레즈너가 지적하듯 강요형(compellence) 제재는 “구체적이고 달성 가능한 요구”와 “완화가 واقعی로 가능하다는 신호”가 결합될 때 순응 가능성이 커진다. 

그런데 대북제재는 시간이 흐르며 (1) 목표가 비핵화라는 ‘최종 상태’에 고정되고, (2) 제재가 누적되면서 정치적으로 “풀기 어려운” 구조가 될수록, 대상 입장에서는 완화·해제가 신뢰하기 어려운 약속으로 보일 수 있다. 이 경우 제재는 협상 유인이 아니라 장기적 봉쇄로 인식될 가능성이 커지고, 결과적으로 행동 변화의 동기가 약해진다.

둘째, 회피·적응 메커니즘의 발달이다. 제재가 장기화될수록 대상국은 불법 조달, 제3국 중개, 프런트 컴퍼니, 해상 환적 등 다양한 회피 전략을 학습하며, 이는 표적 지정과 품목 통제를 반복적으로 무력화한다. 이 지점에서 ‘스마트 제재’ 논자들이 강조하는 네트워크 접근(중개자·금융·운송·법인 네트워크를 함께 겨냥)이 중요해지지만, 

동시에 드레즈너가 경고하듯 네트워크를 넓게 겨냥할수록 제재 범위가 팽창해 “덜 표적화”될 위험도 커진다. 

대북제재는 바로 이 딜레마(정밀성 강화 vs 범위 팽창)를 지속적으로 안고 있다.

셋째, 목표-수단 불일치의 지속이다. 대북제재는 핵·미사일 개발을 막는다는 목표를 갖지만, 현실에서 제재는 종종 “압박 그 자체”가 성과로 제시되며(고통을 성과로 오인), 

제재 강화가 곧 ‘정책 진전’처럼 취급되기도 한다. 이때 제재는 행동 변화를 유도하는 정책도구라기보다, ‘대응했다’는 신호를 주는 상징정치로 전락할 위험이 커진다(드레즈너의 “sanctioning for sanctioning’s sake” 비판). 

요컨대 대북제재의 효과성은 “제재 강도”보다 해제 신뢰성, 회피 억제 역량, 목표-수단의 정합성이라는 운영 조건에 의해 좌우된다. 이 운영 조건이 취약할수록, 제재는 고통을 주더라도 행동 변화를 만들지 못하는 상태가 장기화될 수 있다.

4.3 정당성: 국제법적 한계와 인도주의 비용의 구조적 발생

정당성 축에서는 코논도치의 논의가 특히 직접적이다. 그는 안보리가 제7장 권한으로 제재를 부과하더라도 법적 한계가 존재하며, 그 한계는 강행규범(jus cogens), 국제인권법, 국제인도법, 비례성 원칙 등에서 찾을 수 있다는 점을 강조한다. 

즉 “안보리 결의=무제한 권한”이 아니며, 제재의 설계와 집행은 인권·인도주의·비례성의 제약을 받는다.

정당성 논쟁이 대북제재에서 반복되는 이유는, 제재가 민간경제에 미치는 영향이 단지 “의도치 않은 부작용” 수준을 넘어서 운영 구조 속에서 구조적으로 발생할 수 있기 때문이다. 코논도치는 유엔 인권최고대표가 국제은행 송금 통제가 UN 현장 활동을 둔화시켜 식량 배급, 보건 키트 등 인도주의 지원 전달에 영향을 준다고 지적했음을 소개한다. 

또한 2019년 전문가패널(PoE)이 제시한 “인도주의 프로그램 수행을 저해하는 6가지 요인”—면제 지연, 은행 채널 붕괴, 통관 지연, 해외 공급자 감소, 비용 증가, 운영 자금 감소—을 인용하며, 제재의 실무적 병목이 어떻게 인도주의 영역을 압박하는지 구체화한다. 특히 주목할 부분은 과잉 준수(overcompliance) 문제다. 코논도치는 제재 대상이 아닌 소비재·인도주의 물품의 배송조차, 행위자들이 제재 리스크를 과도하게 회피하면서 영향을 받을 수 있다고 지적한다.  이는 드레즈너가 금융기관의 “디리스킹(de-risking)”이 제재의 실제 지속기간과 파급을 늘릴 수 있다고 말하는 대목과도 연결된다. 

결과적으로 대북제재의 정당성은 “예외 조항이 존재한다”는 사실만으로 담보되지 않으며, **예외가 실제로 작동하는가(operational effectiveness of exemptions)**가 핵심 평가 항목이 된다.

한편, 코논도치는 최근 PoE 최종보고서가 제재의 부정적 함의로서 연료·농업 장비 부족, 물·위생 및 관개 관련 장비 조달 영향, 에너지·운송 능력 제약, 의료 공급망 교란, 실업과 생계 악화 등을 열거하고 있음을 소개한다. 

이 대목은 정당성 논쟁을 단지 “도덕적 문제”가 아니라, 제재가 민간의 생활 조건을 통해 인권·인도주의 영역을 압박할 수 있는 현실적 경로로 제시한다는 점에서 중요하다.

정리하면, 대북제재의 정당성 과제는 (1) 비례성·최소침해 원칙을 충족하는 설계와 (2) 인도주의 예외의 실효성 확보, (3) 과잉 준수로 인한 간접 피해를 줄이는 금융·물류 운영 장치로 귀결된다. 이것이 실패할 경우 제재는 국제적 지지 기반을 잃고, 이는 곧 집행력과 지속가능성의 약화를 통해 효과성까지 잠식할 수 있다.

4.4 지속가능성: 감시·집행 거버넌스의 균열과 ‘파편화’의 위험

지속가능성 축은 대북제재에서 최근 가장 큰 변동이 일어난 영역이다. 대북제재는 제재의 내용만큼이나 감시·보고·조정의 제도적 기반에 의해 유지되어 왔다. 그 핵심 축 중 하나가 1718 제재위원회를 지원해 위반·회피를 조사·보고해 온 PoE 체제였다. 그러나 2024년 3월 28일 안보리는 PoE 임기를 2025년 4월 30일까지 연장하는 결의안 채택에 실패했고, 러시아의 거부권 행사로 인해 연장이 무산되었다. 
CSIS
+3
UN Press
+3
Security Council Report
+3
 그 결과 PoE의 권한은 2024년 4월 말 종료되는 구조가 되었고, 이는 감시·보고 기반의 약화를 불러왔다. 
Security Council Report
+1

이 변화는 지속가능성의 관점에서 두 가지 함의를 갖는다.

첫째, **정보의 공통 기반(common factual baseline)**이 약해질 위험이다. PoE 체제는 단순히 “보고서”가 아니라, 위반과 회피에 대한 비교적 공인된 사실관계를 축적해 회원국 집행을 조율하는 기능을 수행해 왔다. 그 기반이 약해지면, 무엇이 위반인지, 어떤 회피가 새로운지에 대한 합의가 약화되고, 제재 집행은 국가별 재량과 정치적 판단에 더 의존하게 된다. 이는 곧 제재의 예측 가능성과 일관성을 떨어뜨려 효과성을 약화시키는 경로로 이어질 수 있다.

둘째, 다자 제재 체제의 ‘파편화(fragmentation)’ 위험이다. PoE 공백 이후 일부 국가들은 대체 감시 메커니즘을 추진했는데, 예컨대 일본 외무성은 “2024년 10월, 유사입장국들이 MSMT(다자 제재 모니터링 팀)를 설립했다”고 설명하며, 이는 2024년 4월 PoE 종료에 대응한 조치라고 밝힌다. 
외교부
+1
 한국 외교부 역시 MSMT가 UN 안보리 대북제재 이행을 지원하기 위해 보고서·정보를 발간한다고 소개한다. 
외교부
 (이 논문에서는 MSMT 자체를 본격 분석 대상으로 삼기보다는, PoE 종료가 “감시 구조 변화”를 촉발해 제재 거버넌스가 분절될 수 있다는 점을 보여주는 배경 사례로만 1회 언급하는 데 그친다.)

지속가능성 측면에서 더 큰 문제는, 이런 파편화가 제재의 ‘정밀성’과 ‘정당성’ 관리 능력까지 약화시킬 수 있다는 점이다. 감시·집행 기반이 약해지면 회피 억제(효과성)가 어려워질 뿐 아니라, 인도주의 예외의 실효성과 과잉 준수 문제를 교정하는 피드백 루프도 약해질 수 있다. 다시 말해, 대북제재는 “제재 내용의 지속”보다 “운영체계의 지속”이 더 어려운 단계로 진입할 가능성이 있다.

4장 소결

대북제재 사례에 3축 분석틀을 적용하면, 다음과 같은 결론이 도출된다. 첫째, 효과성은 제재의 강도보다 해제 신뢰성·회피 억제 역량·목표-수단 정합성이라는 운영 조건에 달려 있으며, 이 조건이 약할수록 제재는 장기화되어도 행동 변화를 만들지 못할 수 있다. 

 둘째, 정당성은 “예외 조항의 존재”가 아니라 “예외의 실제 작동”과 과잉 준수의 통제를 포함하며, 대북제재는 금융·통관·공급망 병목을 통해 인도주의 비용이 구조적으로 발생할 수 있다는 점에서 지속적으로 논쟁적이다.   셋째, 지속가능성은 최근 PoE 종료(연장 결의안 부결)로 인해 감시·보고 기반이 약화되면서 새롭게 부상한 쟁점이며, 이는 제재의 집행력뿐 아니라 정당성·효과성의 피드백 구조까지 흔들 수 있다.