\section{\chapterfive}

앞선 사례들은 대북제재의 성과와 한계가 제재 강도의 단순한 증감으로 설명되기보다, 
제재가 실제로 작동하도록 만드는 운영 조건에 의해 좌우됨을 시사한다. 
2017년 고강도 제재는 제재의 물질적 압박 능력을 확대했으나, 
그 자체로 행동 변화를 보장하지는 않았다. 
2019년 하노이 회담 결렬은 제재가 협상 수단으로 기능하기 위해 
완화·해제 경로가 신뢰 가능해야 한다는 점을 부각시켰다. 
또한 해상 환적 등 회피 사례는 제재 조항의 촘촘함보다 
감시·집행·정보 공유 역량이 효과성을 좌우할 수 있음을 보여준다. 
정당성 측면에서는 인도주의 예외가 규정되어 있더라도, 
송금·보험·운송·통관에서 병목이 발생하면 예외가 현실에서 작동하기 어렵다는 문제가 반복적으로 제기되어 왔다. 
2019년 유엔 문서(S/2019/691)가 지적하듯, 면제 처리 절차의 개선과 별개로 
은행 채널 문제는 현장 운영을 제약하는 요인으로 남아 왔다. 
아울러 2024년 전문가패널(PoE) 종료와 이후의 대체 감시 노력은, 
제재 내용이 동일하더라도 감시 거버넌스가 흔들릴 경우 집행의 일관성과 공통 사실기반이 약화될 수 있음을 시사한다.

이러한 진단은 대북제재의 개선 과제를 세 가지 방향으로 압축한다. 
우선 단계적 목표 설정과 단계적 완화(부분 해제) 경로의 명료화가 요구된다. 
제재가 행동 변화를 유도하려면 상대가 요구 조건과 보상의 범위를 예측할 수 있어야 하며, 
완화·해제 가능성이 신뢰되지 않을 경우 제재는 협상 유인보다 장기 봉쇄로 인식될 가능성이 커진다. 
다음으로 회피 억제를 위해 네트워크를 겨냥하더라도, 
그 과정에서 제재가 광범위한 거래 위축으로 전환되지 않도록 
집행의 우선순위와 표적성 유지 규칙을 병행할 필요가 있다. 
집행 강화가 과잉 준수와 민간 피해 확대로 연결되면 정당성 논란이 증폭되고, 
국제 공조가 약화되며, 결과적으로 효과성도 저하될 수 있다. 
마지막으로 인도주의 예외를 실행 가능한 체계로 만드는 조치가 필요하다. 
면제 조항의 존재 자체보다 송금·조달·운송이 실제로 가능한지, 
은행 채널의 차단과 통관 병목이 어떤 지점에서 발생하는지 등 운영 장애를 줄여야 
정당성의 훼손을 완화하고 제재 체제의 지속가능성도 확보할 수 있다.

종합하면 대북제재의 핵심 과제는 제재 강도의 조정이 아니라, 
완화·해제 경로의 신뢰성, 인도주의 예외의 실행가능성, 감시·집행 기반을 함께 충족시키는 운영체계의 구축에 있다. 
이 조건들이 동시에 충족되지 않는다면 제재의 목표 달성은 불확실해지고, 
부작용은 누적되며, 공조와 집행력은 약화되는 방향으로 구조가 고착될 가능성이 있다.
