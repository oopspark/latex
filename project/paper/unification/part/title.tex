
\begin{titlepage}
    {\large {\course \, 기말과제}} \hfill {\large {\publishyearmonth}}

    % \vspace{1cm}
    \begin{center}

        {\huge \textbf{\maintitleline}}\\
        \vspace{0.5cm}

        {\LARGE {\subtitle}}\\

    \end{center}

    \hfill \makebox[2.5cm][s]{{\LARGE \textbf{\authorname \textsuperscript{*}}}}\\

    \vspace{5pt}
    
    \begin{tcolorbox}[
        enhanced,
        colback=white,
        colframe=black,
        coltitle=black,          % ← 제목 글씨색을 검정으로 강제
        boxrule=0.6pt,
        arc=3pt,
        left=10pt,
        right=10pt,
        bottom=10pt,
        top=10pt,
        title={\large \textbf{목 \quad 차}},
        fonttitle=\bfseries,
        attach boxed title to top center={yshift=-11pt},
        boxed title style={
            colframe=black,
            colback=white,
            interior style={fill=white},
            boxrule=0pt,
        }
    ]
    
        % 첫 번째 표 (왼쪽)
        \begin{minipage}[t]{0.48\textwidth} % 0.45에서 약간 키우고, 여유를 주기 위해 0.48로 설정
        \centering % minipage 안의 내용을 중앙 정렬
        \begin{tabular}{c l}
        I. & \chapterone \\
        II. & 제재 논쟁의 핵심: \\
        &‘과잉 사용’과 ‘정밀성’ 사이 
        \end{tabular}
        \end{minipage}
        % --- 두 minipage 사이에 충분한 공간을 확보 ---
        \hfill % 왼쪽 minipage와 오른쪽 minipage 사이에 최대한의 수평 공간을 채움
        % --- --- --- --- --- --- --- --- --- --- ---
        % 두 번째 표 (오른쪽)
        \begin{minipage}[t]{0.48\textwidth} % 왼쪽과 같은 너비로 설정
        \centering % minipage 안의 내용을 중앙 정렬
        \begin{tabular}{c l}
        III. & \chapterthree \\
        IV. & 대북제재의 실제 사례 분석 \\
        V. & \chapterfive
        \end{tabular}
        \end{minipage}
    \end{tcolorbox}


    \vspace{5pt}

    \begin{tcolorbox}[
        enhanced,
        colback=white,
        colframe=black,
        coltitle=black,          % ← 제목 글씨색을 검정으로 강제
        boxrule=0.6pt,
        arc=3pt,
        left=10pt,
        right=10pt,
        bottom=10pt,
        top=10pt,
        title={\large \textbf{요 \quad 약}},
        fonttitle=\bfseries,
        attach boxed title to top center={yshift=-11pt},
        boxed title style={
            colframe=black,
            colback=white,
            interior style={fill=white},
            boxrule=0pt,
        }
    ]

    \quad 본 연구는 대북제재를 “강화/완화”의 선택 문제가 아닌, 
    제재가 실제로 작동하도록 만드는 운영 조건의 문제로 재정의한다. 
    이를 위해 효과성(행동 변화 유인), 정당성(인도주의·규범 비용 통제), 
    지속가능성(감시·집행·조정 기반)의 3축 분석틀을 제시하고 대북제재 사례에 적용한다. 
    2017년 고강도 결의 채택은 제재의 위력을 확대했으나 목표 달성을 보장하지는 않았고, 
    2019년 하노이 회담 결렬은 협상 수단으로서 제재가 기능하기 위해 해제(완화) 
    경로의 신뢰성이 핵심임을 보여준다. 
    해상 환적 등 회피 사례는 제재 조항의 촘촘함보다 감시·집행·정보 공유 역량이 
    효과성을 좌우함을 시사한다. 
    또한 인도주의 예외 규정이 존재하더라도 금융·물류 병목과 과잉 준수로 
    예외가 현실에서 작동하지 않을 수 있음을 지적한다. 
    결론적으로 대북제재의 핵심 과제는 제재 강도의 조정보다 단계적 완화 설계, 
    정밀성 유지 규칙, 인도주의 예외의 실행가능성 확보를 통해 
    효과성·정당성·지속가능성을 동시에 충족하는 운영체계를 구축하는 데 있다.
    
    \par
    \vspace{5pt}
    \begin{tabular}{c l}
    \textbullet\ 주제어: & 대북제재, 유엔 안보리, 표적 제재, 제재 회피, 인도주의, 감시 거버넌스
    \end{tabular}
    \end{tcolorbox}


    \vfill
    \hrule
    \vspace{0.2cm}
    {\large {* \department \, \program}}\\

\end{titlepage}
