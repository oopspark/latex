\section{\chaptertwo}

\subsection{식량안보 개념의 확장과 글로벌 곡물 무역}

전통적으로 식량안보(food security)는 한 국가가 자국민에게
충분한 식량을 \emph{지속적으로} 공급할 수 있는 능력을 의미해 왔다.
FAO는 식량안보를 ▲식량의 가용성(availability), ▲접근성(accessibility),
▲이용성(utilization), ▲안정성(stability)의 네 요소로 설명하며,
과거 정책 논의에서는 이 가운데 특히 가용성(국내 생산량)과 이를 대표하는 자급률이
핵심 지표로 활용되는 경향이 강했다.
즉 식량 부족은 생산성 제고, 농지 확충, 재고 관리 등
국내 공급 능력의 문제로 비교적 단순화되어 인식되었다.

그러나 글로벌화와 국제 곡물 무역의 확대는 식량안보의 논의를
국내 생산 중심에서 \emph{국제 교역 구조}를 포함하는 방향으로 이동시켰다.
사료용 곡물과 유지 작물은 생산과 소비의 공간적 분리가 심화되며,
다수 국가가 구조적으로 수입 의존적 위치에 고착화되었다.
이 과정에서 식량안보의 안정성(stability)은
국내 생산 변동뿐 아니라 국제 가격 변동성,
교역 상대국의 정책 변화(수출 제한, 검역, 보조금 등),
물류 차질, 지정학적 리스크에 의해 좌우되는 성격이 강해졌다.
따라서 현대의 식량안보는 ``얼마를 생산할 수 있는가''에 더해
``국제 시장에서 \emph{어떻게} 조달하는가''라는 구조적 질문을 포함해야 한다.

콩(soybean)은 이러한 구조 변화가 집약적으로 나타나는 대표 품목이다.
콩은 직접 소비뿐 아니라 축산 사료, 식용유, 산업 원료 등 다양한 용도로 활용되며,
국제 교역 비중이 높고 공급 충격의 파급 범위가 넓다.
특히 기후변화와 국제 분쟁, 무역 제재, 운송 병목 등은
생산과 유통의 불확실성을 확대시키며,
수입 의존도가 높은 국가일수록 충격을 흡수할 여력이 제한적이다.
결과적으로 콩의 공급 안정성은 개별 국가의 농업 문제를 넘어
\emph{글로벌 공급망 리스크 관리}의 문제로 재정의되고 있다.


\subsection{글로벌 곡물 무역의 네트워크화와 구조적 취약성}

국제 곡물 무역은 양자 간 거래의 단순 합이 아니라,
다수 국가가 동시에 연결된 \emph{복잡한 네트워크(complex network)}의 형태로 관찰된다.
이 네트워크에서는 일부 국가가 다수의 연결을 보유하거나
막대한 교역량을 집중적으로 담당하며,
다른 국가들은 이러한 핵심 국가의 공급 결정과 가격 신호에 의존하는 구조가 형성된다.
이는 국제 무역 시스템이 평상시에는 높은 효율성을 보이는 동시에,
특정 충격에 취약할 수 있음을 의미한다.

선행연구들은 글로벌 곡물 무역 네트워크가
small-world 및 scale-free 특성을 동시에 보일 수 있음을 보고해 왔다.
Small-world 구조는 평균 경로 길이가 짧아
상대적으로 적은 중개 단계를 통해 광범위한 연결이 가능하다는 점에서,
필수 자원의 국제적 이동을 빠르고 효율적으로 만드는 기반이 된다.
반면 scale-free 구조는 연결(또는 거래량)이 소수의 허브에 집중되는 특성을 가지며,
허브 국가의 수출 제한, 정치·경제적 충격, 기후 재해와 같은 사건이 발생할 경우
충격이 네트워크 전반으로 확산될 수 있는 잠재적 취약성을 내포한다.
즉 국제 곡물 무역은 ``효율성''과 ``안정성'' 사이의 긴장 관계 속에서 작동하며,
이를 이해하기 위해서는 단순한 교역량 비교를 넘어선
구조적 분석 틀이 요구된다.


\subsection{복잡계 네트워크 분석과 국제 곡물 무역 연구}

복잡계 네트워크 분석은 국제 곡물 무역을
국가 간 상호의존 구조로 해석하는 데 유용한 방법론이다.
네트워크 분석은 국가를 노드(node), 교역 관계를 엣지(edge)로 설정함으로써,
교역의 양적 흐름뿐 아니라 국가의 구조적 위치, 연결 패턴, 취약성의 성격을
정량적으로 평가할 수 있게 한다.

국제 곡물 무역을 대상으로 한 기존 연구는
네트워크 밀도, 평균 경로 길이, 직경 등 전반적 위상 지표와 함께,
차수 중심성, 매개 중심성, 근접 중심성 등 중심성 지표를 활용하여
무역 구조의 특성과 핵심 행위자를 규명해 왔다.
특히 매개 중심성(betweenness centrality)은
단순 교역량이 크지 않더라도 경유지로서 중요한 국가를 식별할 수 있다는 점에서,
글로벌 공급망의 ``중개자''와 ``병목''을 이해하는 데 기여해 왔다.

다만 중심성 지표는 기본적으로 네트워크 내 \emph{경유·통제 가능성}을 강조하기 때문에,
한국과 같이 생산·수출 허브가 아닌 \emph{말단 수요국(end-demand country)}의 취약성을
직접적으로 설명하는 데에는 한계가 있을 수 있다.
말단 수요국의 위험은 ``얼마나 중개하는가''보다는
``핵심 공급국과 얼마나 직접적·효율적으로 연결되어 있는가'',
그리고 ``직접 연결이 약화될 때 대체 경로가 존재하는가''에 의해 크게 좌우되기 때문이다.
따라서 말단 수요국의 식량안보를 논하기 위해서는
중심성 지표와 더불어 \emph{경로(path)} 자체의 효율성과 회복력을 평가하는 접근이 필요하다.


\subsection{글로벌 콩 무역 네트워크 연구와 중국 중심 분석}

글로벌 콩 무역 네트워크를 분석한 대표 연구로 \textcite{wangStructuralEvolutionGlobal2023}를 들 수 있다.
해당 연구는 2000--2020년 기간의 글로벌 콩 무역 네트워크를 구축하고,
네트워크의 구조적 진화와 특정 국가(중국)의 취약성을 분석하였다.
연구는 미국과 브라질 등 소수 국가가 네트워크의 핵심 허브로 기능하며,
허브의 변화가 네트워크 효율성과 특정 국가의 조달 능력에 영향을 미칠 수 있음을 제시하였다.
특히 표적 제거(targeted disruption) 시나리오를 통해
허브 국가 이탈이 네트워크 안정성에 미치는 효과를 검토함으로써,
글로벌 콩 무역이 특정 국가에 과도하게 의존하는 구조임을 실증적으로 보여주었다.

그러나 이러한 흐름의 연구는 분석 초점과 정책적 함의가
세계 최대 수입국인 중국에 집중되는 경향이 있다.
중국은 네트워크의 중심부에서 허브와 대규모 직접 교역을 수행하는 국가인 반면,
한국과 같은 중소 규모 수입국은 네트워크의 주변부에 가까우며
충격을 흡수하는 방식과 취약성의 형태가 다를 수 있다.
즉 ``허브의 이탈''이 네트워크 전체에 주는 영향뿐 아니라,
그 변화가 \emph{주변부 수입국의 조달 경로}에 어떤 영향을 미치는지에 대한
후속 분석이 요구된다.


\subsection{연구 공백: 한국 관점에서의 구조적 해석 필요성}

한국은 글로벌 콩 무역에서 거래량 기준으로는 상대적으로 작은 비중을 차지하지만,
식량안보 측면에서는 외부 충격에 민감한 위치에 놓여 있다.
허브 국가들과의 교섭력, 대체 공급선 확보 능력,
가격 충격에 대한 완충 장치가 제한적인 조건에서
국제 가격 변동과 공급 차질을 직접적으로 수용해야 하는 구조가 형성되어 있기 때문이다.
그럼에도 기존 네트워크 연구는 이러한 ``비중은 작지만 취약성이 큰'' 국가에 대한
구조적 논의를 충분히 축적하지 못했다.

또한 한국의 식량 정책 논의는 여전히 자급률 및 국내 생산 확대에 초점이 강해,
글로벌 네트워크 내 구조적 위치, 연결 방식, 경로 대체 가능성 등
공급망 리스크의 핵심 요인을 체계적으로 반영하기 어렵다는 한계를 가진다.
이는 정책 설계 과정에서 국제 공급망 리스크를 과소평가할 가능성을 내포한다.

따라서 한국의 식량안보를 보다 현실적으로 이해하기 위해서는
(1) 글로벌 콩 무역 네트워크의 구조를 최신 시점에서 재구성하고,
(2) 그 안에서 한국이 어떤 방식으로 연결되어 있는지를 구조적으로 해석하며,
(3) 특정 공급국과의 직접 연결이 약화되거나 단절될 때
대체 경로가 조달 효율성을 얼마나 훼손하는지를 정량적으로 평가할 필요가 있다.
본 연구는 이러한 문제의식에 기반하여,
기존 중국 중심 연구를 보완하고 한국적 함의를 도출하는 것을 목표로 한다.
