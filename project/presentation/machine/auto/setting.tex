% =========================  SLIDE DOCUMENT SETTINGS  =========================
% 16:9 슬라이드 비율 (A5 가로)
\usepackage[
  a6paper,
  landscape,
  top=10pt,
  bottom=10pt,
  left=20pt,
  right=20pt
]{geometry}

\usepackage{unicode-math}
\def\mathdefault#1{\displaystyle #1}

% 그래픽 · 티크즈
\usepackage{graphicx}
\usepackage{eso-pic}
\usepackage{tikz}
\usetikzlibrary{calc}

% ===============================  FONTS & KOREAN  =============================
\usepackage{xetexko}
\usepackage{fontspec}

\setmainhangulfont{NanumMyeongjo}
\setmainfont{TeX Gyre Pagella}
\setmainhanjafont{Noto Serif CJK KR}

\setlength{\parindent}{0pt} % 문단 들여쓰기 제거

% ================================  COLORS   ===================================
\usepackage{xcolor}
\definecolor{slidebg}{rgb}{0.1, 0.19, 0.1}


% ============================ TITLE BACKGROUND IMAGE ==========================
\newcommand{\TitleBackground}{%
  \AddToShipoutPicture*{%
    \put(332pt,0){%
      \includegraphics[width=90pt,height=117pt]{샤_벡터.png}%
    }%
  }%
}

% ===============================  SLIDE BORDER  ===============================
% 왼쪽 세로 바 + 자동 중앙 세로선
\newcommand{\SlideBorderVer}[1]{%
  \AddToShipoutPicture*{%
    % 왼쪽 세로 직사각형 바
    \begin{tikzpicture}[remember picture,overlay]
      \fill[slidebg]
        (current page.south west)
        rectangle ($(current page.north west) + (#1pt,0)$);
    \end{tikzpicture}

    % 중앙 세로 하이라이트 선 (바 폭에 따라 이동)
    \begin{tikzpicture}[remember picture,overlay]
      \draw[line width=1pt, white, line cap=round]
        ($(current page.center) + (#1pt - 215pt,  0.5\textheight)$)
        --
        ($(current page.center) + (#1pt - 215pt, -0.5\textheight)$);
    \end{tikzpicture}
  }%
}

\newcommand{\SlideBorderHor}[1]{%
  \AddToShipoutPicture*{%
    % 왼쪽 세로 직사각형 바
    \begin{tikzpicture}[remember picture,overlay]
      \fill[slidebg]
        ($(current page.north west) + (0,0)$)
        rectangle ($(current page.south east) + (0,#1pt)$);
    \end{tikzpicture}

    % 중앙 가로 하이라이트 선 (바 폭에 따라 이동)
    \begin{tikzpicture}[remember picture,overlay]
      \draw[line width=1pt, white, line cap=round]
        ($(current page.center) + (0.72\textheight, #1pt - 145pt)$)
        --
        ($(current page.center) + (-0.72\textheight, #1pt - 145pt)$);
    \end{tikzpicture}
  }%
}

% ================================  TABLE  =====================================
\usepackage{tabularx}
% X 계열: 비율 + 정렬 지정
\newcolumntype{O}{>{\hsize=0.1\hsize}X}      % 아주 좁은 열
\newcolumntype{A}{>{\centering\hsize=0.16\hsize}X} % 가운데, 중간 폭
\newcolumntype{B}{>{\hsize=1.0\hsize}X}      % 넓은 열

% ============================  SLIDE MACROS ==================================
% #1 = 제목, #2 = 본문

\newcommand{\firstgap}{26}


\newcommand{\slideone}[4]{%
\newpage
\SlideBorderHor{250}

\hfill {\small \textcolor{white}{#1}}%
\\[-\dimexpr \firstgap pt + 4pt \relax]

\LARGE \textcolor{white}{\textbf{#3}} \hfill \raisebox{-4pt}{{\large \textcolor{white}{#2}}}

\vfill
\Large #4
\vfill
}
\newcommand{\slide}[4]{%
\newpage
\SlideBorderHor{250}

\hfill {\small  \textcolor{white}{#1}}%
\\[-\dimexpr \firstgap pt \relax]

\LARGE \textcolor{white}{\textbf{#3}} \hfill \raisebox{-4pt}{{\large \textcolor{white}{#2}}}

\vfill
\Large #4
\vfill
}



