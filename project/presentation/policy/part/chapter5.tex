\intro
{\chapterfive}


\slide
{\maintitle}
{\chapterfive}
{적지선정의 중요성}{
\centering {\includegraphics[width=0.5\textwidth]{asset/논콩재배관리.png}
}
\begin{itemize}
    \item 일반적으로 논은 낮은 지대의 평야지에 위치함
    \item 대부분 지하수위가 높고 물빠짐이 좋지 않음
    \item 따라서 부적지에 콩을 재배할 경우 습해와 병해로 \\목표수량을 기대하기 어려움
\end{itemize}
\vspace{10pt}
\small ※ 유용환. 2005. 논 콩재배기술 및 재해관리 방법. 물만 먹고 자라요.
}

\slide
{\maintitle}
{\chapterfive}
{배수관리의 중요성}{
\vspace{10pt}
\centering {\includegraphics[width=0.7\textwidth]{asset/문윤만.png}
}
\vspace{10pt}
\begin{itemize}
    \item 밭작물인 콩을 논에 재배하기 위해서 \\가장 먼저 고려해야 할 사항이 배수 관리임
    \item 늦여름 시기는 콩이 영양생장기에서 \\생식생장기로 전환되는 중요한 시점
    \item 우리나라는 대부분의 강수량이 늦여름 시기에 몰려있음
    \item 배수시설이 완비되지 못한 논에 콩을 재배하면 \\수량감소와 생육장해를 초래
\end{itemize}
\vspace{10pt}
\small ※ 문윤만. 2018. 논 콩 재배 확대를 위한 새로운 배수개선 기술. 콩산업정보지, Vol.1 No.- [2018]
}


\slide
{\maintitle}
{\chapterfive}
{부안군 25년도 강수량}{
\begin{center}
    \hspace*{-40pt}{%% Creator: Matplotlib, PGF backend
%%
%% To include the figure in your LaTeX document, write
%%   \input{<filename>.pgf}
%%
%% Make sure the required packages are loaded in your preamble
%%   \usepackage{pgf}
%%
%% Also ensure that all the required font packages are loaded; for instance,
%% the lmodern package is sometimes necessary when using math font.
%%   \usepackage{lmodern}
%%
%% Figures using additional raster images can only be included by \input if
%% they are in the same directory as the main LaTeX file. For loading figures
%% from other directories you can use the `import` package
%%   \usepackage{import}
%%
%% and then include the figures with
%%   \import{<path to file>}{<filename>.pgf}
%%
%% Matplotlib used the following preamble
%%   \def\mathdefault#1{#1}
%%   \everymath=\expandafter{\the\everymath\displaystyle}
%%   \IfFileExists{scrextend.sty}{
%%     \usepackage[fontsize=9.000000pt]{scrextend}
%%   }{
%%     \renewcommand{\normalsize}{\fontsize{9.000000}{10.800000}\selectfont}
%%     \normalsize
%%   }
%%   
%%   \ifdefined\pdftexversion\else  % non-pdftex case.
%%     \usepackage{fontspec}
%%     \setmainfont{DejaVuSerif.ttf}[Path=\detokenize{/home/user/.cache/pypoetry/virtualenvs/graph-KASAOWVd-py3.12/lib/python3.12/site-packages/matplotlib/mpl-data/fonts/ttf/}]
%%     \setsansfont{DejaVuSans.ttf}[Path=\detokenize{/home/user/.cache/pypoetry/virtualenvs/graph-KASAOWVd-py3.12/lib/python3.12/site-packages/matplotlib/mpl-data/fonts/ttf/}]
%%     \setmonofont{DejaVuSansMono.ttf}[Path=\detokenize{/home/user/.cache/pypoetry/virtualenvs/graph-KASAOWVd-py3.12/lib/python3.12/site-packages/matplotlib/mpl-data/fonts/ttf/}]
%%   \fi
%%   \makeatletter\@ifpackageloaded{underscore}{}{\usepackage[strings]{underscore}}\makeatother
%%
\begingroup%
\makeatletter%
\begin{pgfpicture}%
\pgfpathrectangle{\pgfpointorigin}{\pgfqpoint{6.250000in}{3.194444in}}%
\pgfusepath{use as bounding box, clip}%
\begin{pgfscope}%
\pgfsetbuttcap%
\pgfsetmiterjoin%
\definecolor{currentfill}{rgb}{1.000000,1.000000,1.000000}%
\pgfsetfillcolor{currentfill}%
\pgfsetlinewidth{0.000000pt}%
\definecolor{currentstroke}{rgb}{1.000000,1.000000,1.000000}%
\pgfsetstrokecolor{currentstroke}%
\pgfsetdash{}{0pt}%
\pgfpathmoveto{\pgfqpoint{0.000000in}{0.000000in}}%
\pgfpathlineto{\pgfqpoint{6.250000in}{0.000000in}}%
\pgfpathlineto{\pgfqpoint{6.250000in}{3.194444in}}%
\pgfpathlineto{\pgfqpoint{0.000000in}{3.194444in}}%
\pgfpathlineto{\pgfqpoint{0.000000in}{0.000000in}}%
\pgfpathclose%
\pgfusepath{fill}%
\end{pgfscope}%
\begin{pgfscope}%
\pgfsetbuttcap%
\pgfsetmiterjoin%
\definecolor{currentfill}{rgb}{1.000000,1.000000,1.000000}%
\pgfsetfillcolor{currentfill}%
\pgfsetlinewidth{0.000000pt}%
\definecolor{currentstroke}{rgb}{0.000000,0.000000,0.000000}%
\pgfsetstrokecolor{currentstroke}%
\pgfsetstrokeopacity{0.000000}%
\pgfsetdash{}{0pt}%
\pgfpathmoveto{\pgfqpoint{0.781250in}{0.638889in}}%
\pgfpathlineto{\pgfqpoint{5.000000in}{0.638889in}}%
\pgfpathlineto{\pgfqpoint{5.000000in}{2.811111in}}%
\pgfpathlineto{\pgfqpoint{0.781250in}{2.811111in}}%
\pgfpathlineto{\pgfqpoint{0.781250in}{0.638889in}}%
\pgfpathclose%
\pgfusepath{fill}%
\end{pgfscope}%
\begin{pgfscope}%
\pgfsetbuttcap%
\pgfsetroundjoin%
\definecolor{currentfill}{rgb}{0.000000,0.000000,0.000000}%
\pgfsetfillcolor{currentfill}%
\pgfsetlinewidth{0.752812pt}%
\definecolor{currentstroke}{rgb}{0.000000,0.000000,0.000000}%
\pgfsetstrokecolor{currentstroke}%
\pgfsetdash{}{0pt}%
\pgfsys@defobject{currentmarker}{\pgfqpoint{0.000000in}{-0.013889in}}{\pgfqpoint{0.000000in}{0.000000in}}{%
\pgfpathmoveto{\pgfqpoint{0.000000in}{0.000000in}}%
\pgfpathlineto{\pgfqpoint{0.000000in}{-0.013889in}}%
\pgfusepath{stroke,fill}%
}%
\begin{pgfscope}%
\pgfsys@transformshift{1.087740in}{0.638889in}%
\pgfsys@useobject{currentmarker}{}%
\end{pgfscope}%
\end{pgfscope}%
\begin{pgfscope}%
\definecolor{textcolor}{rgb}{0.000000,0.000000,0.000000}%
\pgfsetstrokecolor{textcolor}%
\pgfsetfillcolor{textcolor}%
\pgftext[x=1.087740in,y=0.576389in,,top]{\color{textcolor}{\ifdefined\pdftexversion\else\setmainfont{NanumMyeongjo}\rmfamily\fi\fontsize{9.000000}{10.800000}\selectfont\catcode`\^=\active\def^{\ifmmode\sp\else\^{}\fi}\catcode`\%=\active\def%{\%}1월}}%
\end{pgfscope}%
\begin{pgfscope}%
\pgfsetbuttcap%
\pgfsetroundjoin%
\definecolor{currentfill}{rgb}{0.000000,0.000000,0.000000}%
\pgfsetfillcolor{currentfill}%
\pgfsetlinewidth{0.752812pt}%
\definecolor{currentstroke}{rgb}{0.000000,0.000000,0.000000}%
\pgfsetstrokecolor{currentstroke}%
\pgfsetdash{}{0pt}%
\pgfsys@defobject{currentmarker}{\pgfqpoint{0.000000in}{-0.013889in}}{\pgfqpoint{0.000000in}{0.000000in}}{%
\pgfpathmoveto{\pgfqpoint{0.000000in}{0.000000in}}%
\pgfpathlineto{\pgfqpoint{0.000000in}{-0.013889in}}%
\pgfusepath{stroke,fill}%
}%
\begin{pgfscope}%
\pgfsys@transformshift{1.415538in}{0.638889in}%
\pgfsys@useobject{currentmarker}{}%
\end{pgfscope}%
\end{pgfscope}%
\begin{pgfscope}%
\definecolor{textcolor}{rgb}{0.000000,0.000000,0.000000}%
\pgfsetstrokecolor{textcolor}%
\pgfsetfillcolor{textcolor}%
\pgftext[x=1.415538in,y=0.576389in,,top]{\color{textcolor}{\ifdefined\pdftexversion\else\setmainfont{NanumMyeongjo}\rmfamily\fi\fontsize{9.000000}{10.800000}\selectfont\catcode`\^=\active\def^{\ifmmode\sp\else\^{}\fi}\catcode`\%=\active\def%{\%}2월}}%
\end{pgfscope}%
\begin{pgfscope}%
\pgfsetbuttcap%
\pgfsetroundjoin%
\definecolor{currentfill}{rgb}{0.000000,0.000000,0.000000}%
\pgfsetfillcolor{currentfill}%
\pgfsetlinewidth{0.752812pt}%
\definecolor{currentstroke}{rgb}{0.000000,0.000000,0.000000}%
\pgfsetstrokecolor{currentstroke}%
\pgfsetdash{}{0pt}%
\pgfsys@defobject{currentmarker}{\pgfqpoint{0.000000in}{-0.013889in}}{\pgfqpoint{0.000000in}{0.000000in}}{%
\pgfpathmoveto{\pgfqpoint{0.000000in}{0.000000in}}%
\pgfpathlineto{\pgfqpoint{0.000000in}{-0.013889in}}%
\pgfusepath{stroke,fill}%
}%
\begin{pgfscope}%
\pgfsys@transformshift{1.743335in}{0.638889in}%
\pgfsys@useobject{currentmarker}{}%
\end{pgfscope}%
\end{pgfscope}%
\begin{pgfscope}%
\definecolor{textcolor}{rgb}{0.000000,0.000000,0.000000}%
\pgfsetstrokecolor{textcolor}%
\pgfsetfillcolor{textcolor}%
\pgftext[x=1.743335in,y=0.576389in,,top]{\color{textcolor}{\ifdefined\pdftexversion\else\setmainfont{NanumMyeongjo}\rmfamily\fi\fontsize{9.000000}{10.800000}\selectfont\catcode`\^=\active\def^{\ifmmode\sp\else\^{}\fi}\catcode`\%=\active\def%{\%}3월}}%
\end{pgfscope}%
\begin{pgfscope}%
\pgfsetbuttcap%
\pgfsetroundjoin%
\definecolor{currentfill}{rgb}{0.000000,0.000000,0.000000}%
\pgfsetfillcolor{currentfill}%
\pgfsetlinewidth{0.752812pt}%
\definecolor{currentstroke}{rgb}{0.000000,0.000000,0.000000}%
\pgfsetstrokecolor{currentstroke}%
\pgfsetdash{}{0pt}%
\pgfsys@defobject{currentmarker}{\pgfqpoint{0.000000in}{-0.013889in}}{\pgfqpoint{0.000000in}{0.000000in}}{%
\pgfpathmoveto{\pgfqpoint{0.000000in}{0.000000in}}%
\pgfpathlineto{\pgfqpoint{0.000000in}{-0.013889in}}%
\pgfusepath{stroke,fill}%
}%
\begin{pgfscope}%
\pgfsys@transformshift{2.071132in}{0.638889in}%
\pgfsys@useobject{currentmarker}{}%
\end{pgfscope}%
\end{pgfscope}%
\begin{pgfscope}%
\definecolor{textcolor}{rgb}{0.000000,0.000000,0.000000}%
\pgfsetstrokecolor{textcolor}%
\pgfsetfillcolor{textcolor}%
\pgftext[x=2.071132in,y=0.576389in,,top]{\color{textcolor}{\ifdefined\pdftexversion\else\setmainfont{NanumMyeongjo}\rmfamily\fi\fontsize{9.000000}{10.800000}\selectfont\catcode`\^=\active\def^{\ifmmode\sp\else\^{}\fi}\catcode`\%=\active\def%{\%}4월}}%
\end{pgfscope}%
\begin{pgfscope}%
\pgfsetbuttcap%
\pgfsetroundjoin%
\definecolor{currentfill}{rgb}{0.000000,0.000000,0.000000}%
\pgfsetfillcolor{currentfill}%
\pgfsetlinewidth{0.752812pt}%
\definecolor{currentstroke}{rgb}{0.000000,0.000000,0.000000}%
\pgfsetstrokecolor{currentstroke}%
\pgfsetdash{}{0pt}%
\pgfsys@defobject{currentmarker}{\pgfqpoint{0.000000in}{-0.013889in}}{\pgfqpoint{0.000000in}{0.000000in}}{%
\pgfpathmoveto{\pgfqpoint{0.000000in}{0.000000in}}%
\pgfpathlineto{\pgfqpoint{0.000000in}{-0.013889in}}%
\pgfusepath{stroke,fill}%
}%
\begin{pgfscope}%
\pgfsys@transformshift{2.398929in}{0.638889in}%
\pgfsys@useobject{currentmarker}{}%
\end{pgfscope}%
\end{pgfscope}%
\begin{pgfscope}%
\definecolor{textcolor}{rgb}{0.000000,0.000000,0.000000}%
\pgfsetstrokecolor{textcolor}%
\pgfsetfillcolor{textcolor}%
\pgftext[x=2.398929in,y=0.576389in,,top]{\color{textcolor}{\ifdefined\pdftexversion\else\setmainfont{NanumMyeongjo}\rmfamily\fi\fontsize{9.000000}{10.800000}\selectfont\catcode`\^=\active\def^{\ifmmode\sp\else\^{}\fi}\catcode`\%=\active\def%{\%}5월}}%
\end{pgfscope}%
\begin{pgfscope}%
\pgfsetbuttcap%
\pgfsetroundjoin%
\definecolor{currentfill}{rgb}{0.000000,0.000000,0.000000}%
\pgfsetfillcolor{currentfill}%
\pgfsetlinewidth{0.752812pt}%
\definecolor{currentstroke}{rgb}{0.000000,0.000000,0.000000}%
\pgfsetstrokecolor{currentstroke}%
\pgfsetdash{}{0pt}%
\pgfsys@defobject{currentmarker}{\pgfqpoint{0.000000in}{-0.013889in}}{\pgfqpoint{0.000000in}{0.000000in}}{%
\pgfpathmoveto{\pgfqpoint{0.000000in}{0.000000in}}%
\pgfpathlineto{\pgfqpoint{0.000000in}{-0.013889in}}%
\pgfusepath{stroke,fill}%
}%
\begin{pgfscope}%
\pgfsys@transformshift{2.726726in}{0.638889in}%
\pgfsys@useobject{currentmarker}{}%
\end{pgfscope}%
\end{pgfscope}%
\begin{pgfscope}%
\definecolor{textcolor}{rgb}{0.000000,0.000000,0.000000}%
\pgfsetstrokecolor{textcolor}%
\pgfsetfillcolor{textcolor}%
\pgftext[x=2.726726in,y=0.576389in,,top]{\color{textcolor}{\ifdefined\pdftexversion\else\setmainfont{NanumMyeongjo}\rmfamily\fi\fontsize{9.000000}{10.800000}\selectfont\catcode`\^=\active\def^{\ifmmode\sp\else\^{}\fi}\catcode`\%=\active\def%{\%}6월}}%
\end{pgfscope}%
\begin{pgfscope}%
\pgfsetbuttcap%
\pgfsetroundjoin%
\definecolor{currentfill}{rgb}{0.000000,0.000000,0.000000}%
\pgfsetfillcolor{currentfill}%
\pgfsetlinewidth{0.752812pt}%
\definecolor{currentstroke}{rgb}{0.000000,0.000000,0.000000}%
\pgfsetstrokecolor{currentstroke}%
\pgfsetdash{}{0pt}%
\pgfsys@defobject{currentmarker}{\pgfqpoint{0.000000in}{-0.013889in}}{\pgfqpoint{0.000000in}{0.000000in}}{%
\pgfpathmoveto{\pgfqpoint{0.000000in}{0.000000in}}%
\pgfpathlineto{\pgfqpoint{0.000000in}{-0.013889in}}%
\pgfusepath{stroke,fill}%
}%
\begin{pgfscope}%
\pgfsys@transformshift{3.054524in}{0.638889in}%
\pgfsys@useobject{currentmarker}{}%
\end{pgfscope}%
\end{pgfscope}%
\begin{pgfscope}%
\definecolor{textcolor}{rgb}{0.000000,0.000000,0.000000}%
\pgfsetstrokecolor{textcolor}%
\pgfsetfillcolor{textcolor}%
\pgftext[x=3.054524in,y=0.576389in,,top]{\color{textcolor}{\ifdefined\pdftexversion\else\setmainfont{NanumMyeongjo}\rmfamily\fi\fontsize{9.000000}{10.800000}\selectfont\catcode`\^=\active\def^{\ifmmode\sp\else\^{}\fi}\catcode`\%=\active\def%{\%}7월}}%
\end{pgfscope}%
\begin{pgfscope}%
\pgfsetbuttcap%
\pgfsetroundjoin%
\definecolor{currentfill}{rgb}{0.000000,0.000000,0.000000}%
\pgfsetfillcolor{currentfill}%
\pgfsetlinewidth{0.752812pt}%
\definecolor{currentstroke}{rgb}{0.000000,0.000000,0.000000}%
\pgfsetstrokecolor{currentstroke}%
\pgfsetdash{}{0pt}%
\pgfsys@defobject{currentmarker}{\pgfqpoint{0.000000in}{-0.013889in}}{\pgfqpoint{0.000000in}{0.000000in}}{%
\pgfpathmoveto{\pgfqpoint{0.000000in}{0.000000in}}%
\pgfpathlineto{\pgfqpoint{0.000000in}{-0.013889in}}%
\pgfusepath{stroke,fill}%
}%
\begin{pgfscope}%
\pgfsys@transformshift{3.382321in}{0.638889in}%
\pgfsys@useobject{currentmarker}{}%
\end{pgfscope}%
\end{pgfscope}%
\begin{pgfscope}%
\definecolor{textcolor}{rgb}{0.000000,0.000000,0.000000}%
\pgfsetstrokecolor{textcolor}%
\pgfsetfillcolor{textcolor}%
\pgftext[x=3.382321in,y=0.576389in,,top]{\color{textcolor}{\ifdefined\pdftexversion\else\setmainfont{NanumMyeongjo}\rmfamily\fi\fontsize{9.000000}{10.800000}\selectfont\catcode`\^=\active\def^{\ifmmode\sp\else\^{}\fi}\catcode`\%=\active\def%{\%}8월}}%
\end{pgfscope}%
\begin{pgfscope}%
\pgfsetbuttcap%
\pgfsetroundjoin%
\definecolor{currentfill}{rgb}{0.000000,0.000000,0.000000}%
\pgfsetfillcolor{currentfill}%
\pgfsetlinewidth{0.752812pt}%
\definecolor{currentstroke}{rgb}{0.000000,0.000000,0.000000}%
\pgfsetstrokecolor{currentstroke}%
\pgfsetdash{}{0pt}%
\pgfsys@defobject{currentmarker}{\pgfqpoint{0.000000in}{-0.013889in}}{\pgfqpoint{0.000000in}{0.000000in}}{%
\pgfpathmoveto{\pgfqpoint{0.000000in}{0.000000in}}%
\pgfpathlineto{\pgfqpoint{0.000000in}{-0.013889in}}%
\pgfusepath{stroke,fill}%
}%
\begin{pgfscope}%
\pgfsys@transformshift{3.710118in}{0.638889in}%
\pgfsys@useobject{currentmarker}{}%
\end{pgfscope}%
\end{pgfscope}%
\begin{pgfscope}%
\definecolor{textcolor}{rgb}{0.000000,0.000000,0.000000}%
\pgfsetstrokecolor{textcolor}%
\pgfsetfillcolor{textcolor}%
\pgftext[x=3.710118in,y=0.576389in,,top]{\color{textcolor}{\ifdefined\pdftexversion\else\setmainfont{NanumMyeongjo}\rmfamily\fi\fontsize{9.000000}{10.800000}\selectfont\catcode`\^=\active\def^{\ifmmode\sp\else\^{}\fi}\catcode`\%=\active\def%{\%}9월}}%
\end{pgfscope}%
\begin{pgfscope}%
\pgfsetbuttcap%
\pgfsetroundjoin%
\definecolor{currentfill}{rgb}{0.000000,0.000000,0.000000}%
\pgfsetfillcolor{currentfill}%
\pgfsetlinewidth{0.752812pt}%
\definecolor{currentstroke}{rgb}{0.000000,0.000000,0.000000}%
\pgfsetstrokecolor{currentstroke}%
\pgfsetdash{}{0pt}%
\pgfsys@defobject{currentmarker}{\pgfqpoint{0.000000in}{-0.013889in}}{\pgfqpoint{0.000000in}{0.000000in}}{%
\pgfpathmoveto{\pgfqpoint{0.000000in}{0.000000in}}%
\pgfpathlineto{\pgfqpoint{0.000000in}{-0.013889in}}%
\pgfusepath{stroke,fill}%
}%
\begin{pgfscope}%
\pgfsys@transformshift{4.037915in}{0.638889in}%
\pgfsys@useobject{currentmarker}{}%
\end{pgfscope}%
\end{pgfscope}%
\begin{pgfscope}%
\definecolor{textcolor}{rgb}{0.000000,0.000000,0.000000}%
\pgfsetstrokecolor{textcolor}%
\pgfsetfillcolor{textcolor}%
\pgftext[x=4.037915in,y=0.576389in,,top]{\color{textcolor}{\ifdefined\pdftexversion\else\setmainfont{NanumMyeongjo}\rmfamily\fi\fontsize{9.000000}{10.800000}\selectfont\catcode`\^=\active\def^{\ifmmode\sp\else\^{}\fi}\catcode`\%=\active\def%{\%}10월}}%
\end{pgfscope}%
\begin{pgfscope}%
\pgfsetbuttcap%
\pgfsetroundjoin%
\definecolor{currentfill}{rgb}{0.000000,0.000000,0.000000}%
\pgfsetfillcolor{currentfill}%
\pgfsetlinewidth{0.752812pt}%
\definecolor{currentstroke}{rgb}{0.000000,0.000000,0.000000}%
\pgfsetstrokecolor{currentstroke}%
\pgfsetdash{}{0pt}%
\pgfsys@defobject{currentmarker}{\pgfqpoint{0.000000in}{-0.013889in}}{\pgfqpoint{0.000000in}{0.000000in}}{%
\pgfpathmoveto{\pgfqpoint{0.000000in}{0.000000in}}%
\pgfpathlineto{\pgfqpoint{0.000000in}{-0.013889in}}%
\pgfusepath{stroke,fill}%
}%
\begin{pgfscope}%
\pgfsys@transformshift{4.365712in}{0.638889in}%
\pgfsys@useobject{currentmarker}{}%
\end{pgfscope}%
\end{pgfscope}%
\begin{pgfscope}%
\definecolor{textcolor}{rgb}{0.000000,0.000000,0.000000}%
\pgfsetstrokecolor{textcolor}%
\pgfsetfillcolor{textcolor}%
\pgftext[x=4.365712in,y=0.576389in,,top]{\color{textcolor}{\ifdefined\pdftexversion\else\setmainfont{NanumMyeongjo}\rmfamily\fi\fontsize{9.000000}{10.800000}\selectfont\catcode`\^=\active\def^{\ifmmode\sp\else\^{}\fi}\catcode`\%=\active\def%{\%}11월}}%
\end{pgfscope}%
\begin{pgfscope}%
\pgfsetbuttcap%
\pgfsetroundjoin%
\definecolor{currentfill}{rgb}{0.000000,0.000000,0.000000}%
\pgfsetfillcolor{currentfill}%
\pgfsetlinewidth{0.752812pt}%
\definecolor{currentstroke}{rgb}{0.000000,0.000000,0.000000}%
\pgfsetstrokecolor{currentstroke}%
\pgfsetdash{}{0pt}%
\pgfsys@defobject{currentmarker}{\pgfqpoint{0.000000in}{-0.013889in}}{\pgfqpoint{0.000000in}{0.000000in}}{%
\pgfpathmoveto{\pgfqpoint{0.000000in}{0.000000in}}%
\pgfpathlineto{\pgfqpoint{0.000000in}{-0.013889in}}%
\pgfusepath{stroke,fill}%
}%
\begin{pgfscope}%
\pgfsys@transformshift{4.693510in}{0.638889in}%
\pgfsys@useobject{currentmarker}{}%
\end{pgfscope}%
\end{pgfscope}%
\begin{pgfscope}%
\definecolor{textcolor}{rgb}{0.000000,0.000000,0.000000}%
\pgfsetstrokecolor{textcolor}%
\pgfsetfillcolor{textcolor}%
\pgftext[x=4.693510in,y=0.576389in,,top]{\color{textcolor}{\ifdefined\pdftexversion\else\setmainfont{NanumMyeongjo}\rmfamily\fi\fontsize{9.000000}{10.800000}\selectfont\catcode`\^=\active\def^{\ifmmode\sp\else\^{}\fi}\catcode`\%=\active\def%{\%}12월}}%
\end{pgfscope}%
\begin{pgfscope}%
\pgfpathrectangle{\pgfqpoint{0.781250in}{0.638889in}}{\pgfqpoint{4.218750in}{2.172222in}}%
\pgfusepath{clip}%
\pgfsetbuttcap%
\pgfsetroundjoin%
\pgfsetlinewidth{0.602250pt}%
\definecolor{currentstroke}{rgb}{0.690196,0.690196,0.690196}%
\pgfsetstrokecolor{currentstroke}%
\pgfsetstrokeopacity{0.450000}%
\pgfsetdash{{2.220000pt}{0.960000pt}}{0.000000pt}%
\pgfpathmoveto{\pgfqpoint{0.781250in}{0.638889in}}%
\pgfpathlineto{\pgfqpoint{5.000000in}{0.638889in}}%
\pgfusepath{stroke}%
\end{pgfscope}%
\begin{pgfscope}%
\pgfsetbuttcap%
\pgfsetroundjoin%
\definecolor{currentfill}{rgb}{0.000000,0.000000,0.000000}%
\pgfsetfillcolor{currentfill}%
\pgfsetlinewidth{0.752812pt}%
\definecolor{currentstroke}{rgb}{0.000000,0.000000,0.000000}%
\pgfsetstrokecolor{currentstroke}%
\pgfsetdash{}{0pt}%
\pgfsys@defobject{currentmarker}{\pgfqpoint{-0.013889in}{0.000000in}}{\pgfqpoint{-0.000000in}{0.000000in}}{%
\pgfpathmoveto{\pgfqpoint{-0.000000in}{0.000000in}}%
\pgfpathlineto{\pgfqpoint{-0.013889in}{0.000000in}}%
\pgfusepath{stroke,fill}%
}%
\begin{pgfscope}%
\pgfsys@transformshift{0.781250in}{0.638889in}%
\pgfsys@useobject{currentmarker}{}%
\end{pgfscope}%
\end{pgfscope}%
\begin{pgfscope}%
\definecolor{textcolor}{rgb}{0.000000,0.000000,0.000000}%
\pgfsetstrokecolor{textcolor}%
\pgfsetfillcolor{textcolor}%
\pgftext[x=0.651611in, y=0.588962in, left, base]{\color{textcolor}{\ifdefined\pdftexversion\else\setmainfont{NanumMyeongjo}\rmfamily\fi\fontsize{9.000000}{10.800000}\selectfont\catcode`\^=\active\def^{\ifmmode\sp\else\^{}\fi}\catcode`\%=\active\def%{\%}0}}%
\end{pgfscope}%
\begin{pgfscope}%
\pgfpathrectangle{\pgfqpoint{0.781250in}{0.638889in}}{\pgfqpoint{4.218750in}{2.172222in}}%
\pgfusepath{clip}%
\pgfsetbuttcap%
\pgfsetroundjoin%
\pgfsetlinewidth{0.602250pt}%
\definecolor{currentstroke}{rgb}{0.690196,0.690196,0.690196}%
\pgfsetstrokecolor{currentstroke}%
\pgfsetstrokeopacity{0.450000}%
\pgfsetdash{{2.220000pt}{0.960000pt}}{0.000000pt}%
\pgfpathmoveto{\pgfqpoint{0.781250in}{0.910417in}}%
\pgfpathlineto{\pgfqpoint{5.000000in}{0.910417in}}%
\pgfusepath{stroke}%
\end{pgfscope}%
\begin{pgfscope}%
\pgfsetbuttcap%
\pgfsetroundjoin%
\definecolor{currentfill}{rgb}{0.000000,0.000000,0.000000}%
\pgfsetfillcolor{currentfill}%
\pgfsetlinewidth{0.752812pt}%
\definecolor{currentstroke}{rgb}{0.000000,0.000000,0.000000}%
\pgfsetstrokecolor{currentstroke}%
\pgfsetdash{}{0pt}%
\pgfsys@defobject{currentmarker}{\pgfqpoint{-0.013889in}{0.000000in}}{\pgfqpoint{-0.000000in}{0.000000in}}{%
\pgfpathmoveto{\pgfqpoint{-0.000000in}{0.000000in}}%
\pgfpathlineto{\pgfqpoint{-0.013889in}{0.000000in}}%
\pgfusepath{stroke,fill}%
}%
\begin{pgfscope}%
\pgfsys@transformshift{0.781250in}{0.910417in}%
\pgfsys@useobject{currentmarker}{}%
\end{pgfscope}%
\end{pgfscope}%
\begin{pgfscope}%
\definecolor{textcolor}{rgb}{0.000000,0.000000,0.000000}%
\pgfsetstrokecolor{textcolor}%
\pgfsetfillcolor{textcolor}%
\pgftext[x=0.584473in, y=0.860490in, left, base]{\color{textcolor}{\ifdefined\pdftexversion\else\setmainfont{NanumMyeongjo}\rmfamily\fi\fontsize{9.000000}{10.800000}\selectfont\catcode`\^=\active\def^{\ifmmode\sp\else\^{}\fi}\catcode`\%=\active\def%{\%}50}}%
\end{pgfscope}%
\begin{pgfscope}%
\pgfpathrectangle{\pgfqpoint{0.781250in}{0.638889in}}{\pgfqpoint{4.218750in}{2.172222in}}%
\pgfusepath{clip}%
\pgfsetbuttcap%
\pgfsetroundjoin%
\pgfsetlinewidth{0.602250pt}%
\definecolor{currentstroke}{rgb}{0.690196,0.690196,0.690196}%
\pgfsetstrokecolor{currentstroke}%
\pgfsetstrokeopacity{0.450000}%
\pgfsetdash{{2.220000pt}{0.960000pt}}{0.000000pt}%
\pgfpathmoveto{\pgfqpoint{0.781250in}{1.181944in}}%
\pgfpathlineto{\pgfqpoint{5.000000in}{1.181944in}}%
\pgfusepath{stroke}%
\end{pgfscope}%
\begin{pgfscope}%
\pgfsetbuttcap%
\pgfsetroundjoin%
\definecolor{currentfill}{rgb}{0.000000,0.000000,0.000000}%
\pgfsetfillcolor{currentfill}%
\pgfsetlinewidth{0.752812pt}%
\definecolor{currentstroke}{rgb}{0.000000,0.000000,0.000000}%
\pgfsetstrokecolor{currentstroke}%
\pgfsetdash{}{0pt}%
\pgfsys@defobject{currentmarker}{\pgfqpoint{-0.013889in}{0.000000in}}{\pgfqpoint{-0.000000in}{0.000000in}}{%
\pgfpathmoveto{\pgfqpoint{-0.000000in}{0.000000in}}%
\pgfpathlineto{\pgfqpoint{-0.013889in}{0.000000in}}%
\pgfusepath{stroke,fill}%
}%
\begin{pgfscope}%
\pgfsys@transformshift{0.781250in}{1.181944in}%
\pgfsys@useobject{currentmarker}{}%
\end{pgfscope}%
\end{pgfscope}%
\begin{pgfscope}%
\definecolor{textcolor}{rgb}{0.000000,0.000000,0.000000}%
\pgfsetstrokecolor{textcolor}%
\pgfsetfillcolor{textcolor}%
\pgftext[x=0.517334in, y=1.132018in, left, base]{\color{textcolor}{\ifdefined\pdftexversion\else\setmainfont{NanumMyeongjo}\rmfamily\fi\fontsize{9.000000}{10.800000}\selectfont\catcode`\^=\active\def^{\ifmmode\sp\else\^{}\fi}\catcode`\%=\active\def%{\%}100}}%
\end{pgfscope}%
\begin{pgfscope}%
\pgfpathrectangle{\pgfqpoint{0.781250in}{0.638889in}}{\pgfqpoint{4.218750in}{2.172222in}}%
\pgfusepath{clip}%
\pgfsetbuttcap%
\pgfsetroundjoin%
\pgfsetlinewidth{0.602250pt}%
\definecolor{currentstroke}{rgb}{0.690196,0.690196,0.690196}%
\pgfsetstrokecolor{currentstroke}%
\pgfsetstrokeopacity{0.450000}%
\pgfsetdash{{2.220000pt}{0.960000pt}}{0.000000pt}%
\pgfpathmoveto{\pgfqpoint{0.781250in}{1.453472in}}%
\pgfpathlineto{\pgfqpoint{5.000000in}{1.453472in}}%
\pgfusepath{stroke}%
\end{pgfscope}%
\begin{pgfscope}%
\pgfsetbuttcap%
\pgfsetroundjoin%
\definecolor{currentfill}{rgb}{0.000000,0.000000,0.000000}%
\pgfsetfillcolor{currentfill}%
\pgfsetlinewidth{0.752812pt}%
\definecolor{currentstroke}{rgb}{0.000000,0.000000,0.000000}%
\pgfsetstrokecolor{currentstroke}%
\pgfsetdash{}{0pt}%
\pgfsys@defobject{currentmarker}{\pgfqpoint{-0.013889in}{0.000000in}}{\pgfqpoint{-0.000000in}{0.000000in}}{%
\pgfpathmoveto{\pgfqpoint{-0.000000in}{0.000000in}}%
\pgfpathlineto{\pgfqpoint{-0.013889in}{0.000000in}}%
\pgfusepath{stroke,fill}%
}%
\begin{pgfscope}%
\pgfsys@transformshift{0.781250in}{1.453472in}%
\pgfsys@useobject{currentmarker}{}%
\end{pgfscope}%
\end{pgfscope}%
\begin{pgfscope}%
\definecolor{textcolor}{rgb}{0.000000,0.000000,0.000000}%
\pgfsetstrokecolor{textcolor}%
\pgfsetfillcolor{textcolor}%
\pgftext[x=0.517334in, y=1.403545in, left, base]{\color{textcolor}{\ifdefined\pdftexversion\else\setmainfont{NanumMyeongjo}\rmfamily\fi\fontsize{9.000000}{10.800000}\selectfont\catcode`\^=\active\def^{\ifmmode\sp\else\^{}\fi}\catcode`\%=\active\def%{\%}150}}%
\end{pgfscope}%
\begin{pgfscope}%
\pgfpathrectangle{\pgfqpoint{0.781250in}{0.638889in}}{\pgfqpoint{4.218750in}{2.172222in}}%
\pgfusepath{clip}%
\pgfsetbuttcap%
\pgfsetroundjoin%
\pgfsetlinewidth{0.602250pt}%
\definecolor{currentstroke}{rgb}{0.690196,0.690196,0.690196}%
\pgfsetstrokecolor{currentstroke}%
\pgfsetstrokeopacity{0.450000}%
\pgfsetdash{{2.220000pt}{0.960000pt}}{0.000000pt}%
\pgfpathmoveto{\pgfqpoint{0.781250in}{1.725000in}}%
\pgfpathlineto{\pgfqpoint{5.000000in}{1.725000in}}%
\pgfusepath{stroke}%
\end{pgfscope}%
\begin{pgfscope}%
\pgfsetbuttcap%
\pgfsetroundjoin%
\definecolor{currentfill}{rgb}{0.000000,0.000000,0.000000}%
\pgfsetfillcolor{currentfill}%
\pgfsetlinewidth{0.752812pt}%
\definecolor{currentstroke}{rgb}{0.000000,0.000000,0.000000}%
\pgfsetstrokecolor{currentstroke}%
\pgfsetdash{}{0pt}%
\pgfsys@defobject{currentmarker}{\pgfqpoint{-0.013889in}{0.000000in}}{\pgfqpoint{-0.000000in}{0.000000in}}{%
\pgfpathmoveto{\pgfqpoint{-0.000000in}{0.000000in}}%
\pgfpathlineto{\pgfqpoint{-0.013889in}{0.000000in}}%
\pgfusepath{stroke,fill}%
}%
\begin{pgfscope}%
\pgfsys@transformshift{0.781250in}{1.725000in}%
\pgfsys@useobject{currentmarker}{}%
\end{pgfscope}%
\end{pgfscope}%
\begin{pgfscope}%
\definecolor{textcolor}{rgb}{0.000000,0.000000,0.000000}%
\pgfsetstrokecolor{textcolor}%
\pgfsetfillcolor{textcolor}%
\pgftext[x=0.517334in, y=1.675073in, left, base]{\color{textcolor}{\ifdefined\pdftexversion\else\setmainfont{NanumMyeongjo}\rmfamily\fi\fontsize{9.000000}{10.800000}\selectfont\catcode`\^=\active\def^{\ifmmode\sp\else\^{}\fi}\catcode`\%=\active\def%{\%}200}}%
\end{pgfscope}%
\begin{pgfscope}%
\pgfpathrectangle{\pgfqpoint{0.781250in}{0.638889in}}{\pgfqpoint{4.218750in}{2.172222in}}%
\pgfusepath{clip}%
\pgfsetbuttcap%
\pgfsetroundjoin%
\pgfsetlinewidth{0.602250pt}%
\definecolor{currentstroke}{rgb}{0.690196,0.690196,0.690196}%
\pgfsetstrokecolor{currentstroke}%
\pgfsetstrokeopacity{0.450000}%
\pgfsetdash{{2.220000pt}{0.960000pt}}{0.000000pt}%
\pgfpathmoveto{\pgfqpoint{0.781250in}{1.996528in}}%
\pgfpathlineto{\pgfqpoint{5.000000in}{1.996528in}}%
\pgfusepath{stroke}%
\end{pgfscope}%
\begin{pgfscope}%
\pgfsetbuttcap%
\pgfsetroundjoin%
\definecolor{currentfill}{rgb}{0.000000,0.000000,0.000000}%
\pgfsetfillcolor{currentfill}%
\pgfsetlinewidth{0.752812pt}%
\definecolor{currentstroke}{rgb}{0.000000,0.000000,0.000000}%
\pgfsetstrokecolor{currentstroke}%
\pgfsetdash{}{0pt}%
\pgfsys@defobject{currentmarker}{\pgfqpoint{-0.013889in}{0.000000in}}{\pgfqpoint{-0.000000in}{0.000000in}}{%
\pgfpathmoveto{\pgfqpoint{-0.000000in}{0.000000in}}%
\pgfpathlineto{\pgfqpoint{-0.013889in}{0.000000in}}%
\pgfusepath{stroke,fill}%
}%
\begin{pgfscope}%
\pgfsys@transformshift{0.781250in}{1.996528in}%
\pgfsys@useobject{currentmarker}{}%
\end{pgfscope}%
\end{pgfscope}%
\begin{pgfscope}%
\definecolor{textcolor}{rgb}{0.000000,0.000000,0.000000}%
\pgfsetstrokecolor{textcolor}%
\pgfsetfillcolor{textcolor}%
\pgftext[x=0.517334in, y=1.946601in, left, base]{\color{textcolor}{\ifdefined\pdftexversion\else\setmainfont{NanumMyeongjo}\rmfamily\fi\fontsize{9.000000}{10.800000}\selectfont\catcode`\^=\active\def^{\ifmmode\sp\else\^{}\fi}\catcode`\%=\active\def%{\%}250}}%
\end{pgfscope}%
\begin{pgfscope}%
\pgfpathrectangle{\pgfqpoint{0.781250in}{0.638889in}}{\pgfqpoint{4.218750in}{2.172222in}}%
\pgfusepath{clip}%
\pgfsetbuttcap%
\pgfsetroundjoin%
\pgfsetlinewidth{0.602250pt}%
\definecolor{currentstroke}{rgb}{0.690196,0.690196,0.690196}%
\pgfsetstrokecolor{currentstroke}%
\pgfsetstrokeopacity{0.450000}%
\pgfsetdash{{2.220000pt}{0.960000pt}}{0.000000pt}%
\pgfpathmoveto{\pgfqpoint{0.781250in}{2.268056in}}%
\pgfpathlineto{\pgfqpoint{5.000000in}{2.268056in}}%
\pgfusepath{stroke}%
\end{pgfscope}%
\begin{pgfscope}%
\pgfsetbuttcap%
\pgfsetroundjoin%
\definecolor{currentfill}{rgb}{0.000000,0.000000,0.000000}%
\pgfsetfillcolor{currentfill}%
\pgfsetlinewidth{0.752812pt}%
\definecolor{currentstroke}{rgb}{0.000000,0.000000,0.000000}%
\pgfsetstrokecolor{currentstroke}%
\pgfsetdash{}{0pt}%
\pgfsys@defobject{currentmarker}{\pgfqpoint{-0.013889in}{0.000000in}}{\pgfqpoint{-0.000000in}{0.000000in}}{%
\pgfpathmoveto{\pgfqpoint{-0.000000in}{0.000000in}}%
\pgfpathlineto{\pgfqpoint{-0.013889in}{0.000000in}}%
\pgfusepath{stroke,fill}%
}%
\begin{pgfscope}%
\pgfsys@transformshift{0.781250in}{2.268056in}%
\pgfsys@useobject{currentmarker}{}%
\end{pgfscope}%
\end{pgfscope}%
\begin{pgfscope}%
\definecolor{textcolor}{rgb}{0.000000,0.000000,0.000000}%
\pgfsetstrokecolor{textcolor}%
\pgfsetfillcolor{textcolor}%
\pgftext[x=0.517334in, y=2.218129in, left, base]{\color{textcolor}{\ifdefined\pdftexversion\else\setmainfont{NanumMyeongjo}\rmfamily\fi\fontsize{9.000000}{10.800000}\selectfont\catcode`\^=\active\def^{\ifmmode\sp\else\^{}\fi}\catcode`\%=\active\def%{\%}300}}%
\end{pgfscope}%
\begin{pgfscope}%
\pgfpathrectangle{\pgfqpoint{0.781250in}{0.638889in}}{\pgfqpoint{4.218750in}{2.172222in}}%
\pgfusepath{clip}%
\pgfsetbuttcap%
\pgfsetroundjoin%
\pgfsetlinewidth{0.602250pt}%
\definecolor{currentstroke}{rgb}{0.690196,0.690196,0.690196}%
\pgfsetstrokecolor{currentstroke}%
\pgfsetstrokeopacity{0.450000}%
\pgfsetdash{{2.220000pt}{0.960000pt}}{0.000000pt}%
\pgfpathmoveto{\pgfqpoint{0.781250in}{2.539583in}}%
\pgfpathlineto{\pgfqpoint{5.000000in}{2.539583in}}%
\pgfusepath{stroke}%
\end{pgfscope}%
\begin{pgfscope}%
\pgfsetbuttcap%
\pgfsetroundjoin%
\definecolor{currentfill}{rgb}{0.000000,0.000000,0.000000}%
\pgfsetfillcolor{currentfill}%
\pgfsetlinewidth{0.752812pt}%
\definecolor{currentstroke}{rgb}{0.000000,0.000000,0.000000}%
\pgfsetstrokecolor{currentstroke}%
\pgfsetdash{}{0pt}%
\pgfsys@defobject{currentmarker}{\pgfqpoint{-0.013889in}{0.000000in}}{\pgfqpoint{-0.000000in}{0.000000in}}{%
\pgfpathmoveto{\pgfqpoint{-0.000000in}{0.000000in}}%
\pgfpathlineto{\pgfqpoint{-0.013889in}{0.000000in}}%
\pgfusepath{stroke,fill}%
}%
\begin{pgfscope}%
\pgfsys@transformshift{0.781250in}{2.539583in}%
\pgfsys@useobject{currentmarker}{}%
\end{pgfscope}%
\end{pgfscope}%
\begin{pgfscope}%
\definecolor{textcolor}{rgb}{0.000000,0.000000,0.000000}%
\pgfsetstrokecolor{textcolor}%
\pgfsetfillcolor{textcolor}%
\pgftext[x=0.517334in, y=2.489657in, left, base]{\color{textcolor}{\ifdefined\pdftexversion\else\setmainfont{NanumMyeongjo}\rmfamily\fi\fontsize{9.000000}{10.800000}\selectfont\catcode`\^=\active\def^{\ifmmode\sp\else\^{}\fi}\catcode`\%=\active\def%{\%}350}}%
\end{pgfscope}%
\begin{pgfscope}%
\pgfpathrectangle{\pgfqpoint{0.781250in}{0.638889in}}{\pgfqpoint{4.218750in}{2.172222in}}%
\pgfusepath{clip}%
\pgfsetbuttcap%
\pgfsetroundjoin%
\pgfsetlinewidth{0.602250pt}%
\definecolor{currentstroke}{rgb}{0.690196,0.690196,0.690196}%
\pgfsetstrokecolor{currentstroke}%
\pgfsetstrokeopacity{0.450000}%
\pgfsetdash{{2.220000pt}{0.960000pt}}{0.000000pt}%
\pgfpathmoveto{\pgfqpoint{0.781250in}{2.811111in}}%
\pgfpathlineto{\pgfqpoint{5.000000in}{2.811111in}}%
\pgfusepath{stroke}%
\end{pgfscope}%
\begin{pgfscope}%
\pgfsetbuttcap%
\pgfsetroundjoin%
\definecolor{currentfill}{rgb}{0.000000,0.000000,0.000000}%
\pgfsetfillcolor{currentfill}%
\pgfsetlinewidth{0.752812pt}%
\definecolor{currentstroke}{rgb}{0.000000,0.000000,0.000000}%
\pgfsetstrokecolor{currentstroke}%
\pgfsetdash{}{0pt}%
\pgfsys@defobject{currentmarker}{\pgfqpoint{-0.013889in}{0.000000in}}{\pgfqpoint{-0.000000in}{0.000000in}}{%
\pgfpathmoveto{\pgfqpoint{-0.000000in}{0.000000in}}%
\pgfpathlineto{\pgfqpoint{-0.013889in}{0.000000in}}%
\pgfusepath{stroke,fill}%
}%
\begin{pgfscope}%
\pgfsys@transformshift{0.781250in}{2.811111in}%
\pgfsys@useobject{currentmarker}{}%
\end{pgfscope}%
\end{pgfscope}%
\begin{pgfscope}%
\definecolor{textcolor}{rgb}{0.000000,0.000000,0.000000}%
\pgfsetstrokecolor{textcolor}%
\pgfsetfillcolor{textcolor}%
\pgftext[x=0.517334in, y=2.761184in, left, base]{\color{textcolor}{\ifdefined\pdftexversion\else\setmainfont{NanumMyeongjo}\rmfamily\fi\fontsize{9.000000}{10.800000}\selectfont\catcode`\^=\active\def^{\ifmmode\sp\else\^{}\fi}\catcode`\%=\active\def%{\%}400}}%
\end{pgfscope}%
\begin{pgfscope}%
\pgfsetrectcap%
\pgfsetmiterjoin%
\pgfsetlinewidth{0.752812pt}%
\definecolor{currentstroke}{rgb}{0.000000,0.000000,0.000000}%
\pgfsetstrokecolor{currentstroke}%
\pgfsetdash{}{0pt}%
\pgfpathmoveto{\pgfqpoint{0.781250in}{0.638889in}}%
\pgfpathlineto{\pgfqpoint{0.781250in}{2.811111in}}%
\pgfusepath{stroke}%
\end{pgfscope}%
\begin{pgfscope}%
\pgfsetrectcap%
\pgfsetmiterjoin%
\pgfsetlinewidth{0.752812pt}%
\definecolor{currentstroke}{rgb}{0.000000,0.000000,0.000000}%
\pgfsetstrokecolor{currentstroke}%
\pgfsetdash{}{0pt}%
\pgfpathmoveto{\pgfqpoint{0.781250in}{0.638889in}}%
\pgfpathlineto{\pgfqpoint{5.000000in}{0.638889in}}%
\pgfusepath{stroke}%
\end{pgfscope}%
\begin{pgfscope}%
\pgfpathrectangle{\pgfqpoint{0.781250in}{0.638889in}}{\pgfqpoint{4.218750in}{2.172222in}}%
\pgfusepath{clip}%
\pgfsetbuttcap%
\pgfsetmiterjoin%
\definecolor{currentfill}{rgb}{0.227451,0.192157,0.427451}%
\pgfsetfillcolor{currentfill}%
\pgfsetlinewidth{1.003750pt}%
\definecolor{currentstroke}{rgb}{0.266667,0.266667,0.266667}%
\pgfsetstrokecolor{currentstroke}%
\pgfsetdash{}{0pt}%
\pgfpathmoveto{\pgfqpoint{0.973011in}{0.638889in}}%
\pgfpathlineto{\pgfqpoint{1.202469in}{0.638889in}}%
\pgfpathlineto{\pgfqpoint{1.202469in}{0.882178in}}%
\pgfpathlineto{\pgfqpoint{0.973011in}{0.882178in}}%
\pgfpathlineto{\pgfqpoint{0.973011in}{0.638889in}}%
\pgfpathclose%
\pgfusepath{stroke,fill}%
\end{pgfscope}%
\begin{pgfscope}%
\pgfpathrectangle{\pgfqpoint{0.781250in}{0.638889in}}{\pgfqpoint{4.218750in}{2.172222in}}%
\pgfusepath{clip}%
\pgfsetbuttcap%
\pgfsetmiterjoin%
\definecolor{currentfill}{rgb}{0.227451,0.192157,0.427451}%
\pgfsetfillcolor{currentfill}%
\pgfsetlinewidth{1.003750pt}%
\definecolor{currentstroke}{rgb}{0.266667,0.266667,0.266667}%
\pgfsetstrokecolor{currentstroke}%
\pgfsetdash{}{0pt}%
\pgfpathmoveto{\pgfqpoint{1.300809in}{0.638889in}}%
\pgfpathlineto{\pgfqpoint{1.530267in}{0.638889in}}%
\pgfpathlineto{\pgfqpoint{1.530267in}{0.832217in}}%
\pgfpathlineto{\pgfqpoint{1.300809in}{0.832217in}}%
\pgfpathlineto{\pgfqpoint{1.300809in}{0.638889in}}%
\pgfpathclose%
\pgfusepath{stroke,fill}%
\end{pgfscope}%
\begin{pgfscope}%
\pgfpathrectangle{\pgfqpoint{0.781250in}{0.638889in}}{\pgfqpoint{4.218750in}{2.172222in}}%
\pgfusepath{clip}%
\pgfsetbuttcap%
\pgfsetmiterjoin%
\definecolor{currentfill}{rgb}{0.227451,0.192157,0.427451}%
\pgfsetfillcolor{currentfill}%
\pgfsetlinewidth{1.003750pt}%
\definecolor{currentstroke}{rgb}{0.266667,0.266667,0.266667}%
\pgfsetstrokecolor{currentstroke}%
\pgfsetdash{}{0pt}%
\pgfpathmoveto{\pgfqpoint{1.628606in}{0.638889in}}%
\pgfpathlineto{\pgfqpoint{1.858064in}{0.638889in}}%
\pgfpathlineto{\pgfqpoint{1.858064in}{0.872403in}}%
\pgfpathlineto{\pgfqpoint{1.628606in}{0.872403in}}%
\pgfpathlineto{\pgfqpoint{1.628606in}{0.638889in}}%
\pgfpathclose%
\pgfusepath{stroke,fill}%
\end{pgfscope}%
\begin{pgfscope}%
\pgfpathrectangle{\pgfqpoint{0.781250in}{0.638889in}}{\pgfqpoint{4.218750in}{2.172222in}}%
\pgfusepath{clip}%
\pgfsetbuttcap%
\pgfsetmiterjoin%
\definecolor{currentfill}{rgb}{0.227451,0.192157,0.427451}%
\pgfsetfillcolor{currentfill}%
\pgfsetlinewidth{1.003750pt}%
\definecolor{currentstroke}{rgb}{0.266667,0.266667,0.266667}%
\pgfsetstrokecolor{currentstroke}%
\pgfsetdash{}{0pt}%
\pgfpathmoveto{\pgfqpoint{1.956403in}{0.638889in}}%
\pgfpathlineto{\pgfqpoint{2.185861in}{0.638889in}}%
\pgfpathlineto{\pgfqpoint{2.185861in}{0.978842in}}%
\pgfpathlineto{\pgfqpoint{1.956403in}{0.978842in}}%
\pgfpathlineto{\pgfqpoint{1.956403in}{0.638889in}}%
\pgfpathclose%
\pgfusepath{stroke,fill}%
\end{pgfscope}%
\begin{pgfscope}%
\pgfpathrectangle{\pgfqpoint{0.781250in}{0.638889in}}{\pgfqpoint{4.218750in}{2.172222in}}%
\pgfusepath{clip}%
\pgfsetbuttcap%
\pgfsetmiterjoin%
\definecolor{currentfill}{rgb}{0.227451,0.192157,0.427451}%
\pgfsetfillcolor{currentfill}%
\pgfsetlinewidth{1.003750pt}%
\definecolor{currentstroke}{rgb}{0.266667,0.266667,0.266667}%
\pgfsetstrokecolor{currentstroke}%
\pgfsetdash{}{0pt}%
\pgfpathmoveto{\pgfqpoint{2.284200in}{0.638889in}}%
\pgfpathlineto{\pgfqpoint{2.513658in}{0.638889in}}%
\pgfpathlineto{\pgfqpoint{2.513658in}{1.089082in}}%
\pgfpathlineto{\pgfqpoint{2.284200in}{1.089082in}}%
\pgfpathlineto{\pgfqpoint{2.284200in}{0.638889in}}%
\pgfpathclose%
\pgfusepath{stroke,fill}%
\end{pgfscope}%
\begin{pgfscope}%
\pgfpathrectangle{\pgfqpoint{0.781250in}{0.638889in}}{\pgfqpoint{4.218750in}{2.172222in}}%
\pgfusepath{clip}%
\pgfsetbuttcap%
\pgfsetmiterjoin%
\definecolor{currentfill}{rgb}{0.227451,0.192157,0.427451}%
\pgfsetfillcolor{currentfill}%
\pgfsetlinewidth{1.003750pt}%
\definecolor{currentstroke}{rgb}{0.266667,0.266667,0.266667}%
\pgfsetstrokecolor{currentstroke}%
\pgfsetdash{}{0pt}%
\pgfpathmoveto{\pgfqpoint{2.611997in}{0.638889in}}%
\pgfpathlineto{\pgfqpoint{2.841455in}{0.638889in}}%
\pgfpathlineto{\pgfqpoint{2.841455in}{2.363633in}}%
\pgfpathlineto{\pgfqpoint{2.611997in}{2.363633in}}%
\pgfpathlineto{\pgfqpoint{2.611997in}{0.638889in}}%
\pgfpathclose%
\pgfusepath{stroke,fill}%
\end{pgfscope}%
\begin{pgfscope}%
\pgfpathrectangle{\pgfqpoint{0.781250in}{0.638889in}}{\pgfqpoint{4.218750in}{2.172222in}}%
\pgfusepath{clip}%
\pgfsetbuttcap%
\pgfsetmiterjoin%
\definecolor{currentfill}{rgb}{0.227451,0.192157,0.427451}%
\pgfsetfillcolor{currentfill}%
\pgfsetlinewidth{1.003750pt}%
\definecolor{currentstroke}{rgb}{0.266667,0.266667,0.266667}%
\pgfsetstrokecolor{currentstroke}%
\pgfsetdash{}{0pt}%
\pgfpathmoveto{\pgfqpoint{2.939795in}{0.638889in}}%
\pgfpathlineto{\pgfqpoint{3.169253in}{0.638889in}}%
\pgfpathlineto{\pgfqpoint{3.169253in}{1.773875in}}%
\pgfpathlineto{\pgfqpoint{2.939795in}{1.773875in}}%
\pgfpathlineto{\pgfqpoint{2.939795in}{0.638889in}}%
\pgfpathclose%
\pgfusepath{stroke,fill}%
\end{pgfscope}%
\begin{pgfscope}%
\pgfpathrectangle{\pgfqpoint{0.781250in}{0.638889in}}{\pgfqpoint{4.218750in}{2.172222in}}%
\pgfusepath{clip}%
\pgfsetbuttcap%
\pgfsetmiterjoin%
\definecolor{currentfill}{rgb}{0.227451,0.192157,0.427451}%
\pgfsetfillcolor{currentfill}%
\pgfsetlinewidth{1.003750pt}%
\definecolor{currentstroke}{rgb}{0.266667,0.266667,0.266667}%
\pgfsetstrokecolor{currentstroke}%
\pgfsetdash{}{0pt}%
\pgfpathmoveto{\pgfqpoint{3.267592in}{0.638889in}}%
\pgfpathlineto{\pgfqpoint{3.497050in}{0.638889in}}%
\pgfpathlineto{\pgfqpoint{3.497050in}{1.632681in}}%
\pgfpathlineto{\pgfqpoint{3.267592in}{1.632681in}}%
\pgfpathlineto{\pgfqpoint{3.267592in}{0.638889in}}%
\pgfpathclose%
\pgfusepath{stroke,fill}%
\end{pgfscope}%
\begin{pgfscope}%
\pgfpathrectangle{\pgfqpoint{0.781250in}{0.638889in}}{\pgfqpoint{4.218750in}{2.172222in}}%
\pgfusepath{clip}%
\pgfsetbuttcap%
\pgfsetmiterjoin%
\definecolor{currentfill}{rgb}{0.227451,0.192157,0.427451}%
\pgfsetfillcolor{currentfill}%
\pgfsetlinewidth{1.003750pt}%
\definecolor{currentstroke}{rgb}{0.266667,0.266667,0.266667}%
\pgfsetstrokecolor{currentstroke}%
\pgfsetdash{}{0pt}%
\pgfpathmoveto{\pgfqpoint{3.595389in}{0.638889in}}%
\pgfpathlineto{\pgfqpoint{3.824847in}{0.638889in}}%
\pgfpathlineto{\pgfqpoint{3.824847in}{2.529265in}}%
\pgfpathlineto{\pgfqpoint{3.595389in}{2.529265in}}%
\pgfpathlineto{\pgfqpoint{3.595389in}{0.638889in}}%
\pgfpathclose%
\pgfusepath{stroke,fill}%
\end{pgfscope}%
\begin{pgfscope}%
\pgfpathrectangle{\pgfqpoint{0.781250in}{0.638889in}}{\pgfqpoint{4.218750in}{2.172222in}}%
\pgfusepath{clip}%
\pgfsetbuttcap%
\pgfsetmiterjoin%
\definecolor{currentfill}{rgb}{0.227451,0.192157,0.427451}%
\pgfsetfillcolor{currentfill}%
\pgfsetlinewidth{1.003750pt}%
\definecolor{currentstroke}{rgb}{0.266667,0.266667,0.266667}%
\pgfsetstrokecolor{currentstroke}%
\pgfsetdash{}{0pt}%
\pgfpathmoveto{\pgfqpoint{3.923186in}{0.638889in}}%
\pgfpathlineto{\pgfqpoint{4.152644in}{0.638889in}}%
\pgfpathlineto{\pgfqpoint{4.152644in}{1.131440in}}%
\pgfpathlineto{\pgfqpoint{3.923186in}{1.131440in}}%
\pgfpathlineto{\pgfqpoint{3.923186in}{0.638889in}}%
\pgfpathclose%
\pgfusepath{stroke,fill}%
\end{pgfscope}%
\begin{pgfscope}%
\pgfpathrectangle{\pgfqpoint{0.781250in}{0.638889in}}{\pgfqpoint{4.218750in}{2.172222in}}%
\pgfusepath{clip}%
\pgfsetbuttcap%
\pgfsetmiterjoin%
\definecolor{currentfill}{rgb}{0.227451,0.192157,0.427451}%
\pgfsetfillcolor{currentfill}%
\pgfsetlinewidth{1.003750pt}%
\definecolor{currentstroke}{rgb}{0.266667,0.266667,0.266667}%
\pgfsetstrokecolor{currentstroke}%
\pgfsetdash{}{0pt}%
\pgfpathmoveto{\pgfqpoint{4.250983in}{0.638889in}}%
\pgfpathlineto{\pgfqpoint{4.480441in}{0.638889in}}%
\pgfpathlineto{\pgfqpoint{4.480441in}{0.755646in}}%
\pgfpathlineto{\pgfqpoint{4.250983in}{0.755646in}}%
\pgfpathlineto{\pgfqpoint{4.250983in}{0.638889in}}%
\pgfpathclose%
\pgfusepath{stroke,fill}%
\end{pgfscope}%
\begin{pgfscope}%
\pgfpathrectangle{\pgfqpoint{0.781250in}{0.638889in}}{\pgfqpoint{4.218750in}{2.172222in}}%
\pgfusepath{clip}%
\pgfsetbuttcap%
\pgfsetmiterjoin%
\definecolor{currentfill}{rgb}{0.227451,0.192157,0.427451}%
\pgfsetfillcolor{currentfill}%
\pgfsetlinewidth{1.003750pt}%
\definecolor{currentstroke}{rgb}{0.266667,0.266667,0.266667}%
\pgfsetstrokecolor{currentstroke}%
\pgfsetdash{}{0pt}%
\pgfpathmoveto{\pgfqpoint{4.578781in}{0.638889in}}%
\pgfpathlineto{\pgfqpoint{4.808239in}{0.638889in}}%
\pgfpathlineto{\pgfqpoint{4.808239in}{0.650293in}}%
\pgfpathlineto{\pgfqpoint{4.578781in}{0.650293in}}%
\pgfpathlineto{\pgfqpoint{4.578781in}{0.638889in}}%
\pgfpathclose%
\pgfusepath{stroke,fill}%
\end{pgfscope}%
\begin{pgfscope}%
\definecolor{textcolor}{rgb}{0.000000,0.000000,0.000000}%
\pgfsetstrokecolor{textcolor}%
\pgfsetfillcolor{textcolor}%
\pgftext[x=1.087740in,y=0.909956in,,bottom]{\color{textcolor}{\ifdefined\pdftexversion\else\setmainfont{NanumMyeongjo}\rmfamily\fi\fontsize{7.000000}{8.400000}\bfseries\selectfont\catcode`\^=\active\def^{\ifmmode\sp\else\^{}\fi}\catcode`\%=\active\def%{\%}45}}%
\end{pgfscope}%
\begin{pgfscope}%
\definecolor{textcolor}{rgb}{0.000000,0.000000,0.000000}%
\pgfsetstrokecolor{textcolor}%
\pgfsetfillcolor{textcolor}%
\pgftext[x=1.415538in,y=0.859994in,,bottom]{\color{textcolor}{\ifdefined\pdftexversion\else\setmainfont{NanumMyeongjo}\rmfamily\fi\fontsize{7.000000}{8.400000}\bfseries\selectfont\catcode`\^=\active\def^{\ifmmode\sp\else\^{}\fi}\catcode`\%=\active\def%{\%}36}}%
\end{pgfscope}%
\begin{pgfscope}%
\definecolor{textcolor}{rgb}{0.000000,0.000000,0.000000}%
\pgfsetstrokecolor{textcolor}%
\pgfsetfillcolor{textcolor}%
\pgftext[x=1.743335in,y=0.900181in,,bottom]{\color{textcolor}{\ifdefined\pdftexversion\else\setmainfont{NanumMyeongjo}\rmfamily\fi\fontsize{7.000000}{8.400000}\bfseries\selectfont\catcode`\^=\active\def^{\ifmmode\sp\else\^{}\fi}\catcode`\%=\active\def%{\%}43}}%
\end{pgfscope}%
\begin{pgfscope}%
\definecolor{textcolor}{rgb}{0.000000,0.000000,0.000000}%
\pgfsetstrokecolor{textcolor}%
\pgfsetfillcolor{textcolor}%
\pgftext[x=2.071132in,y=1.006619in,,bottom]{\color{textcolor}{\ifdefined\pdftexversion\else\setmainfont{NanumMyeongjo}\rmfamily\fi\fontsize{7.000000}{8.400000}\bfseries\selectfont\catcode`\^=\active\def^{\ifmmode\sp\else\^{}\fi}\catcode`\%=\active\def%{\%}63}}%
\end{pgfscope}%
\begin{pgfscope}%
\definecolor{textcolor}{rgb}{0.000000,0.000000,0.000000}%
\pgfsetstrokecolor{textcolor}%
\pgfsetfillcolor{textcolor}%
\pgftext[x=2.398929in,y=1.116860in,,bottom]{\color{textcolor}{\ifdefined\pdftexversion\else\setmainfont{NanumMyeongjo}\rmfamily\fi\fontsize{7.000000}{8.400000}\bfseries\selectfont\catcode`\^=\active\def^{\ifmmode\sp\else\^{}\fi}\catcode`\%=\active\def%{\%}83}}%
\end{pgfscope}%
\begin{pgfscope}%
\definecolor{textcolor}{rgb}{0.000000,0.000000,0.000000}%
\pgfsetstrokecolor{textcolor}%
\pgfsetfillcolor{textcolor}%
\pgftext[x=2.726726in,y=2.391411in,,bottom]{\color{textcolor}{\ifdefined\pdftexversion\else\setmainfont{NanumMyeongjo}\rmfamily\fi\fontsize{7.000000}{8.400000}\bfseries\selectfont\catcode`\^=\active\def^{\ifmmode\sp\else\^{}\fi}\catcode`\%=\active\def%{\%}318}}%
\end{pgfscope}%
\begin{pgfscope}%
\definecolor{textcolor}{rgb}{0.000000,0.000000,0.000000}%
\pgfsetstrokecolor{textcolor}%
\pgfsetfillcolor{textcolor}%
\pgftext[x=3.054524in,y=1.801653in,,bottom]{\color{textcolor}{\ifdefined\pdftexversion\else\setmainfont{NanumMyeongjo}\rmfamily\fi\fontsize{7.000000}{8.400000}\bfseries\selectfont\catcode`\^=\active\def^{\ifmmode\sp\else\^{}\fi}\catcode`\%=\active\def%{\%}209}}%
\end{pgfscope}%
\begin{pgfscope}%
\definecolor{textcolor}{rgb}{0.000000,0.000000,0.000000}%
\pgfsetstrokecolor{textcolor}%
\pgfsetfillcolor{textcolor}%
\pgftext[x=3.382321in,y=1.660458in,,bottom]{\color{textcolor}{\ifdefined\pdftexversion\else\setmainfont{NanumMyeongjo}\rmfamily\fi\fontsize{7.000000}{8.400000}\bfseries\selectfont\catcode`\^=\active\def^{\ifmmode\sp\else\^{}\fi}\catcode`\%=\active\def%{\%}183}}%
\end{pgfscope}%
\begin{pgfscope}%
\definecolor{textcolor}{rgb}{0.000000,0.000000,0.000000}%
\pgfsetstrokecolor{textcolor}%
\pgfsetfillcolor{textcolor}%
\pgftext[x=3.710118in,y=2.557043in,,bottom]{\color{textcolor}{\ifdefined\pdftexversion\else\setmainfont{NanumMyeongjo}\rmfamily\fi\fontsize{7.000000}{8.400000}\bfseries\selectfont\catcode`\^=\active\def^{\ifmmode\sp\else\^{}\fi}\catcode`\%=\active\def%{\%}348}}%
\end{pgfscope}%
\begin{pgfscope}%
\definecolor{textcolor}{rgb}{0.000000,0.000000,0.000000}%
\pgfsetstrokecolor{textcolor}%
\pgfsetfillcolor{textcolor}%
\pgftext[x=4.037915in,y=1.159218in,,bottom]{\color{textcolor}{\ifdefined\pdftexversion\else\setmainfont{NanumMyeongjo}\rmfamily\fi\fontsize{7.000000}{8.400000}\bfseries\selectfont\catcode`\^=\active\def^{\ifmmode\sp\else\^{}\fi}\catcode`\%=\active\def%{\%}91}}%
\end{pgfscope}%
\begin{pgfscope}%
\definecolor{textcolor}{rgb}{0.000000,0.000000,0.000000}%
\pgfsetstrokecolor{textcolor}%
\pgfsetfillcolor{textcolor}%
\pgftext[x=4.365712in,y=0.783424in,,bottom]{\color{textcolor}{\ifdefined\pdftexversion\else\setmainfont{NanumMyeongjo}\rmfamily\fi\fontsize{7.000000}{8.400000}\bfseries\selectfont\catcode`\^=\active\def^{\ifmmode\sp\else\^{}\fi}\catcode`\%=\active\def%{\%}22}}%
\end{pgfscope}%
\begin{pgfscope}%
\definecolor{textcolor}{rgb}{0.000000,0.000000,0.000000}%
\pgfsetstrokecolor{textcolor}%
\pgfsetfillcolor{textcolor}%
\pgftext[x=4.693510in,y=0.678071in,,bottom]{\color{textcolor}{\ifdefined\pdftexversion\else\setmainfont{NanumMyeongjo}\rmfamily\fi\fontsize{7.000000}{8.400000}\bfseries\selectfont\catcode`\^=\active\def^{\ifmmode\sp\else\^{}\fi}\catcode`\%=\active\def%{\%}2}}%
\end{pgfscope}%
\begin{pgfscope}%
\pgfpathrectangle{\pgfqpoint{0.781250in}{0.638889in}}{\pgfqpoint{4.218750in}{2.172222in}}%
\pgfusepath{clip}%
\pgfsetrectcap%
\pgfsetroundjoin%
\pgfsetlinewidth{0.903375pt}%
\definecolor{currentstroke}{rgb}{0.164706,0.615686,0.560784}%
\pgfsetstrokecolor{currentstroke}%
\pgfsetdash{}{0pt}%
\pgfpathmoveto{\pgfqpoint{1.087740in}{0.811038in}}%
\pgfpathlineto{\pgfqpoint{1.415538in}{0.836561in}}%
\pgfpathlineto{\pgfqpoint{1.743335in}{0.895754in}}%
\pgfpathlineto{\pgfqpoint{2.071132in}{1.073333in}}%
\pgfpathlineto{\pgfqpoint{2.398929in}{1.099943in}}%
\pgfpathlineto{\pgfqpoint{2.726726in}{1.354636in}}%
\pgfpathlineto{\pgfqpoint{3.054524in}{2.135007in}}%
\pgfpathlineto{\pgfqpoint{3.382321in}{2.052462in}}%
\pgfpathlineto{\pgfqpoint{3.710118in}{1.395908in}}%
\pgfpathlineto{\pgfqpoint{4.037915in}{0.943000in}}%
\pgfpathlineto{\pgfqpoint{4.365712in}{0.922907in}}%
\pgfpathlineto{\pgfqpoint{4.693510in}{0.869144in}}%
\pgfusepath{stroke}%
\end{pgfscope}%
\begin{pgfscope}%
\pgfpathrectangle{\pgfqpoint{0.781250in}{0.638889in}}{\pgfqpoint{4.218750in}{2.172222in}}%
\pgfusepath{clip}%
\pgfsetbuttcap%
\pgfsetroundjoin%
\definecolor{currentfill}{rgb}{0.164706,0.615686,0.560784}%
\pgfsetfillcolor{currentfill}%
\pgfsetlinewidth{1.003750pt}%
\definecolor{currentstroke}{rgb}{0.164706,0.615686,0.560784}%
\pgfsetstrokecolor{currentstroke}%
\pgfsetdash{}{0pt}%
\pgfsys@defobject{currentmarker}{\pgfqpoint{-0.018750in}{-0.018750in}}{\pgfqpoint{0.018750in}{0.018750in}}{%
\pgfpathmoveto{\pgfqpoint{0.000000in}{-0.018750in}}%
\pgfpathcurveto{\pgfqpoint{0.004973in}{-0.018750in}}{\pgfqpoint{0.009742in}{-0.016774in}}{\pgfqpoint{0.013258in}{-0.013258in}}%
\pgfpathcurveto{\pgfqpoint{0.016774in}{-0.009742in}}{\pgfqpoint{0.018750in}{-0.004973in}}{\pgfqpoint{0.018750in}{0.000000in}}%
\pgfpathcurveto{\pgfqpoint{0.018750in}{0.004973in}}{\pgfqpoint{0.016774in}{0.009742in}}{\pgfqpoint{0.013258in}{0.013258in}}%
\pgfpathcurveto{\pgfqpoint{0.009742in}{0.016774in}}{\pgfqpoint{0.004973in}{0.018750in}}{\pgfqpoint{0.000000in}{0.018750in}}%
\pgfpathcurveto{\pgfqpoint{-0.004973in}{0.018750in}}{\pgfqpoint{-0.009742in}{0.016774in}}{\pgfqpoint{-0.013258in}{0.013258in}}%
\pgfpathcurveto{\pgfqpoint{-0.016774in}{0.009742in}}{\pgfqpoint{-0.018750in}{0.004973in}}{\pgfqpoint{-0.018750in}{0.000000in}}%
\pgfpathcurveto{\pgfqpoint{-0.018750in}{-0.004973in}}{\pgfqpoint{-0.016774in}{-0.009742in}}{\pgfqpoint{-0.013258in}{-0.013258in}}%
\pgfpathcurveto{\pgfqpoint{-0.009742in}{-0.016774in}}{\pgfqpoint{-0.004973in}{-0.018750in}}{\pgfqpoint{0.000000in}{-0.018750in}}%
\pgfpathlineto{\pgfqpoint{0.000000in}{-0.018750in}}%
\pgfpathclose%
\pgfusepath{stroke,fill}%
}%
\begin{pgfscope}%
\pgfsys@transformshift{1.087740in}{0.811038in}%
\pgfsys@useobject{currentmarker}{}%
\end{pgfscope}%
\begin{pgfscope}%
\pgfsys@transformshift{1.415538in}{0.836561in}%
\pgfsys@useobject{currentmarker}{}%
\end{pgfscope}%
\begin{pgfscope}%
\pgfsys@transformshift{1.743335in}{0.895754in}%
\pgfsys@useobject{currentmarker}{}%
\end{pgfscope}%
\begin{pgfscope}%
\pgfsys@transformshift{2.071132in}{1.073333in}%
\pgfsys@useobject{currentmarker}{}%
\end{pgfscope}%
\begin{pgfscope}%
\pgfsys@transformshift{2.398929in}{1.099943in}%
\pgfsys@useobject{currentmarker}{}%
\end{pgfscope}%
\begin{pgfscope}%
\pgfsys@transformshift{2.726726in}{1.354636in}%
\pgfsys@useobject{currentmarker}{}%
\end{pgfscope}%
\begin{pgfscope}%
\pgfsys@transformshift{3.054524in}{2.135007in}%
\pgfsys@useobject{currentmarker}{}%
\end{pgfscope}%
\begin{pgfscope}%
\pgfsys@transformshift{3.382321in}{2.052462in}%
\pgfsys@useobject{currentmarker}{}%
\end{pgfscope}%
\begin{pgfscope}%
\pgfsys@transformshift{3.710118in}{1.395908in}%
\pgfsys@useobject{currentmarker}{}%
\end{pgfscope}%
\begin{pgfscope}%
\pgfsys@transformshift{4.037915in}{0.943000in}%
\pgfsys@useobject{currentmarker}{}%
\end{pgfscope}%
\begin{pgfscope}%
\pgfsys@transformshift{4.365712in}{0.922907in}%
\pgfsys@useobject{currentmarker}{}%
\end{pgfscope}%
\begin{pgfscope}%
\pgfsys@transformshift{4.693510in}{0.869144in}%
\pgfsys@useobject{currentmarker}{}%
\end{pgfscope}%
\end{pgfscope}%
\begin{pgfscope}%
\pgfsetrectcap%
\pgfsetroundjoin%
\pgfsetlinewidth{0.903375pt}%
\definecolor{currentstroke}{rgb}{0.164706,0.615686,0.560784}%
\pgfsetstrokecolor{currentstroke}%
\pgfsetdash{}{0pt}%
\pgfpathmoveto{\pgfqpoint{5.112500in}{2.762986in}}%
\pgfpathlineto{\pgfqpoint{5.237500in}{2.762986in}}%
\pgfpathlineto{\pgfqpoint{5.362500in}{2.762986in}}%
\pgfusepath{stroke}%
\end{pgfscope}%
\begin{pgfscope}%
\pgfsetbuttcap%
\pgfsetroundjoin%
\definecolor{currentfill}{rgb}{0.164706,0.615686,0.560784}%
\pgfsetfillcolor{currentfill}%
\pgfsetlinewidth{1.003750pt}%
\definecolor{currentstroke}{rgb}{0.164706,0.615686,0.560784}%
\pgfsetstrokecolor{currentstroke}%
\pgfsetdash{}{0pt}%
\pgfsys@defobject{currentmarker}{\pgfqpoint{-0.018750in}{-0.018750in}}{\pgfqpoint{0.018750in}{0.018750in}}{%
\pgfpathmoveto{\pgfqpoint{0.000000in}{-0.018750in}}%
\pgfpathcurveto{\pgfqpoint{0.004973in}{-0.018750in}}{\pgfqpoint{0.009742in}{-0.016774in}}{\pgfqpoint{0.013258in}{-0.013258in}}%
\pgfpathcurveto{\pgfqpoint{0.016774in}{-0.009742in}}{\pgfqpoint{0.018750in}{-0.004973in}}{\pgfqpoint{0.018750in}{0.000000in}}%
\pgfpathcurveto{\pgfqpoint{0.018750in}{0.004973in}}{\pgfqpoint{0.016774in}{0.009742in}}{\pgfqpoint{0.013258in}{0.013258in}}%
\pgfpathcurveto{\pgfqpoint{0.009742in}{0.016774in}}{\pgfqpoint{0.004973in}{0.018750in}}{\pgfqpoint{0.000000in}{0.018750in}}%
\pgfpathcurveto{\pgfqpoint{-0.004973in}{0.018750in}}{\pgfqpoint{-0.009742in}{0.016774in}}{\pgfqpoint{-0.013258in}{0.013258in}}%
\pgfpathcurveto{\pgfqpoint{-0.016774in}{0.009742in}}{\pgfqpoint{-0.018750in}{0.004973in}}{\pgfqpoint{-0.018750in}{0.000000in}}%
\pgfpathcurveto{\pgfqpoint{-0.018750in}{-0.004973in}}{\pgfqpoint{-0.016774in}{-0.009742in}}{\pgfqpoint{-0.013258in}{-0.013258in}}%
\pgfpathcurveto{\pgfqpoint{-0.009742in}{-0.016774in}}{\pgfqpoint{-0.004973in}{-0.018750in}}{\pgfqpoint{0.000000in}{-0.018750in}}%
\pgfpathlineto{\pgfqpoint{0.000000in}{-0.018750in}}%
\pgfpathclose%
\pgfusepath{stroke,fill}%
}%
\begin{pgfscope}%
\pgfsys@transformshift{5.237500in}{2.762986in}%
\pgfsys@useobject{currentmarker}{}%
\end{pgfscope}%
\end{pgfscope}%
\begin{pgfscope}%
\definecolor{textcolor}{rgb}{0.000000,0.000000,0.000000}%
\pgfsetstrokecolor{textcolor}%
\pgfsetfillcolor{textcolor}%
\pgftext[x=5.462500in,y=2.719236in,left,base]{\color{textcolor}{\ifdefined\pdftexversion\else\setmainfont{NanumMyeongjo}\rmfamily\fi\fontsize{9.000000}{10.800000}\selectfont\catcode`\^=\active\def^{\ifmmode\sp\else\^{}\fi}\catcode`\%=\active\def%{\%}평년}}%
\end{pgfscope}%
\begin{pgfscope}%
\pgfsetbuttcap%
\pgfsetmiterjoin%
\definecolor{currentfill}{rgb}{0.227451,0.192157,0.427451}%
\pgfsetfillcolor{currentfill}%
\pgfsetlinewidth{1.003750pt}%
\definecolor{currentstroke}{rgb}{0.266667,0.266667,0.266667}%
\pgfsetstrokecolor{currentstroke}%
\pgfsetdash{}{0pt}%
\pgfpathmoveto{\pgfqpoint{5.112500in}{2.527952in}}%
\pgfpathlineto{\pgfqpoint{5.362500in}{2.527952in}}%
\pgfpathlineto{\pgfqpoint{5.362500in}{2.615452in}}%
\pgfpathlineto{\pgfqpoint{5.112500in}{2.615452in}}%
\pgfpathlineto{\pgfqpoint{5.112500in}{2.527952in}}%
\pgfpathclose%
\pgfusepath{stroke,fill}%
\end{pgfscope}%
\begin{pgfscope}%
\definecolor{textcolor}{rgb}{0.000000,0.000000,0.000000}%
\pgfsetstrokecolor{textcolor}%
\pgfsetfillcolor{textcolor}%
\pgftext[x=5.462500in,y=2.527952in,left,base]{\color{textcolor}{\ifdefined\pdftexversion\else\setmainfont{NanumMyeongjo}\rmfamily\fi\fontsize{9.000000}{10.800000}\selectfont\catcode`\^=\active\def^{\ifmmode\sp\else\^{}\fi}\catcode`\%=\active\def%{\%}25년도}}%
\end{pgfscope}%
\begin{pgfscope}%
\definecolor{textcolor}{rgb}{0.333333,0.333333,0.333333}%
\pgfsetstrokecolor{textcolor}%
\pgfsetfillcolor{textcolor}%
\pgftext[x=1.875000in,y=0.319444in,,top]{\color{textcolor}{\ifdefined\pdftexversion\else\setmainfont{NanumMyeongjo}\rmfamily\fi\fontsize{9.000000}{10.800000}\selectfont\catcode`\^=\active\def^{\ifmmode\sp\else\^{}\fi}\catcode`\%=\active\def%{\%}출처: 기상자료개방포털 자료 기반 저자 작성}}%
\end{pgfscope}%
\begin{pgfscope}%
\definecolor{textcolor}{rgb}{0.333333,0.333333,0.333333}%
\pgfsetstrokecolor{textcolor}%
\pgfsetfillcolor{textcolor}%
\pgftext[x=4.687500in,y=2.970833in,,top]{\color{textcolor}{\ifdefined\pdftexversion\else\setmainfont{NanumMyeongjo}\rmfamily\fi\fontsize{9.000000}{10.800000}\selectfont\catcode`\^=\active\def^{\ifmmode\sp\else\^{}\fi}\catcode`\%=\active\def%{\%}(단위: mm)}}%
\end{pgfscope}%
\end{pgfpicture}%
\makeatother%
\endgroup%
}
\end{center}
}


\slide
{\maintitle}
{\chapterfive}
{여름 장마}{

\begin{center}
\includegraphics[width=0.7\textwidth]{asset/여름장마.png}
\end{center}
\vspace{-10pt}
\small ※ 전업농신문

\begin{center}
\includegraphics[width=0.7\textwidth]{asset/여름장마_차관.png}
\end{center}
\vspace{-20pt}
\small ※ 농림축산식품부 보도자료
}


\slide
{\maintitle}
{\chapterfive}
{가을 장마}{
\vspace{10pt}

\begin{center}
\includegraphics[width=0.7\textwidth]{asset/가을장마.png}
\end{center}
\vspace{-10pt}
\small ※ 한국농정

\begin{center}
\includegraphics[width=0.7\textwidth]{asset/가을장마_김제.png}
\end{center}
\vspace{-20pt}
\small ※ 전북일보
}



\slide
{\maintitle}
{\chapterfive}
{부안군 과거 강수량}{
\begin{center}
    \hspace*{-40pt}{%% Creator: Matplotlib, PGF backend
%%
%% To include the figure in your LaTeX document, write
%%   \input{<filename>.pgf}
%%
%% Make sure the required packages are loaded in your preamble
%%   \usepackage{pgf}
%%
%% Also ensure that all the required font packages are loaded; for instance,
%% the lmodern package is sometimes necessary when using math font.
%%   \usepackage{lmodern}
%%
%% Figures using additional raster images can only be included by \input if
%% they are in the same directory as the main LaTeX file. For loading figures
%% from other directories you can use the `import` package
%%   \usepackage{import}
%%
%% and then include the figures with
%%   \import{<path to file>}{<filename>.pgf}
%%
%% Matplotlib used the following preamble
%%   \def\mathdefault#1{#1}
%%   \everymath=\expandafter{\the\everymath\displaystyle}
%%   \IfFileExists{scrextend.sty}{
%%     \usepackage[fontsize=9.000000pt]{scrextend}
%%   }{
%%     \renewcommand{\normalsize}{\fontsize{9.000000}{10.800000}\selectfont}
%%     \normalsize
%%   }
%%   
%%   \ifdefined\pdftexversion\else  % non-pdftex case.
%%     \usepackage{fontspec}
%%     \setmainfont{DejaVuSerif.ttf}[Path=\detokenize{/home/user/.cache/pypoetry/virtualenvs/graph-KASAOWVd-py3.12/lib/python3.12/site-packages/matplotlib/mpl-data/fonts/ttf/}]
%%     \setsansfont{DejaVuSans.ttf}[Path=\detokenize{/home/user/.cache/pypoetry/virtualenvs/graph-KASAOWVd-py3.12/lib/python3.12/site-packages/matplotlib/mpl-data/fonts/ttf/}]
%%     \setmonofont{DejaVuSansMono.ttf}[Path=\detokenize{/home/user/.cache/pypoetry/virtualenvs/graph-KASAOWVd-py3.12/lib/python3.12/site-packages/matplotlib/mpl-data/fonts/ttf/}]
%%   \fi
%%   \makeatletter\@ifpackageloaded{underscore}{}{\usepackage[strings]{underscore}}\makeatother
%%
\begingroup%
\makeatletter%
\begin{pgfpicture}%
\pgfpathrectangle{\pgfpointorigin}{\pgfqpoint{6.250000in}{3.194444in}}%
\pgfusepath{use as bounding box, clip}%
\begin{pgfscope}%
\pgfsetbuttcap%
\pgfsetmiterjoin%
\definecolor{currentfill}{rgb}{1.000000,1.000000,1.000000}%
\pgfsetfillcolor{currentfill}%
\pgfsetlinewidth{0.000000pt}%
\definecolor{currentstroke}{rgb}{1.000000,1.000000,1.000000}%
\pgfsetstrokecolor{currentstroke}%
\pgfsetdash{}{0pt}%
\pgfpathmoveto{\pgfqpoint{0.000000in}{0.000000in}}%
\pgfpathlineto{\pgfqpoint{6.250000in}{0.000000in}}%
\pgfpathlineto{\pgfqpoint{6.250000in}{3.194444in}}%
\pgfpathlineto{\pgfqpoint{0.000000in}{3.194444in}}%
\pgfpathlineto{\pgfqpoint{0.000000in}{0.000000in}}%
\pgfpathclose%
\pgfusepath{fill}%
\end{pgfscope}%
\begin{pgfscope}%
\pgfsetbuttcap%
\pgfsetmiterjoin%
\definecolor{currentfill}{rgb}{1.000000,1.000000,1.000000}%
\pgfsetfillcolor{currentfill}%
\pgfsetlinewidth{0.000000pt}%
\definecolor{currentstroke}{rgb}{0.000000,0.000000,0.000000}%
\pgfsetstrokecolor{currentstroke}%
\pgfsetstrokeopacity{0.000000}%
\pgfsetdash{}{0pt}%
\pgfpathmoveto{\pgfqpoint{0.781250in}{0.638889in}}%
\pgfpathlineto{\pgfqpoint{5.625000in}{0.638889in}}%
\pgfpathlineto{\pgfqpoint{5.625000in}{2.811111in}}%
\pgfpathlineto{\pgfqpoint{0.781250in}{2.811111in}}%
\pgfpathlineto{\pgfqpoint{0.781250in}{0.638889in}}%
\pgfpathclose%
\pgfusepath{fill}%
\end{pgfscope}%
\begin{pgfscope}%
\pgfsetbuttcap%
\pgfsetroundjoin%
\definecolor{currentfill}{rgb}{0.000000,0.000000,0.000000}%
\pgfsetfillcolor{currentfill}%
\pgfsetlinewidth{0.752812pt}%
\definecolor{currentstroke}{rgb}{0.000000,0.000000,0.000000}%
\pgfsetstrokecolor{currentstroke}%
\pgfsetdash{}{0pt}%
\pgfsys@defobject{currentmarker}{\pgfqpoint{0.000000in}{-0.013889in}}{\pgfqpoint{0.000000in}{0.000000in}}{%
\pgfpathmoveto{\pgfqpoint{0.000000in}{0.000000in}}%
\pgfpathlineto{\pgfqpoint{0.000000in}{-0.013889in}}%
\pgfusepath{stroke,fill}%
}%
\begin{pgfscope}%
\pgfsys@transformshift{1.014296in}{0.638889in}%
\pgfsys@useobject{currentmarker}{}%
\end{pgfscope}%
\end{pgfscope}%
\begin{pgfscope}%
\definecolor{textcolor}{rgb}{0.000000,0.000000,0.000000}%
\pgfsetstrokecolor{textcolor}%
\pgfsetfillcolor{textcolor}%
\pgftext[x=1.014296in,y=0.576389in,,top]{\color{textcolor}{\ifdefined\pdftexversion\else\setmainfont{NanumMyeongjo}\rmfamily\fi\fontsize{9.000000}{10.800000}\selectfont\catcode`\^=\active\def^{\ifmmode\sp\else\^{}\fi}\catcode`\%=\active\def%{\%}2016}}%
\end{pgfscope}%
\begin{pgfscope}%
\pgfsetbuttcap%
\pgfsetroundjoin%
\definecolor{currentfill}{rgb}{0.000000,0.000000,0.000000}%
\pgfsetfillcolor{currentfill}%
\pgfsetlinewidth{0.752812pt}%
\definecolor{currentstroke}{rgb}{0.000000,0.000000,0.000000}%
\pgfsetstrokecolor{currentstroke}%
\pgfsetdash{}{0pt}%
\pgfsys@defobject{currentmarker}{\pgfqpoint{0.000000in}{-0.013889in}}{\pgfqpoint{0.000000in}{0.000000in}}{%
\pgfpathmoveto{\pgfqpoint{0.000000in}{0.000000in}}%
\pgfpathlineto{\pgfqpoint{0.000000in}{-0.013889in}}%
\pgfusepath{stroke,fill}%
}%
\begin{pgfscope}%
\pgfsys@transformshift{1.455740in}{0.638889in}%
\pgfsys@useobject{currentmarker}{}%
\end{pgfscope}%
\end{pgfscope}%
\begin{pgfscope}%
\definecolor{textcolor}{rgb}{0.000000,0.000000,0.000000}%
\pgfsetstrokecolor{textcolor}%
\pgfsetfillcolor{textcolor}%
\pgftext[x=1.455740in,y=0.576389in,,top]{\color{textcolor}{\ifdefined\pdftexversion\else\setmainfont{NanumMyeongjo}\rmfamily\fi\fontsize{9.000000}{10.800000}\selectfont\catcode`\^=\active\def^{\ifmmode\sp\else\^{}\fi}\catcode`\%=\active\def%{\%}2017}}%
\end{pgfscope}%
\begin{pgfscope}%
\pgfsetbuttcap%
\pgfsetroundjoin%
\definecolor{currentfill}{rgb}{0.000000,0.000000,0.000000}%
\pgfsetfillcolor{currentfill}%
\pgfsetlinewidth{0.752812pt}%
\definecolor{currentstroke}{rgb}{0.000000,0.000000,0.000000}%
\pgfsetstrokecolor{currentstroke}%
\pgfsetdash{}{0pt}%
\pgfsys@defobject{currentmarker}{\pgfqpoint{0.000000in}{-0.013889in}}{\pgfqpoint{0.000000in}{0.000000in}}{%
\pgfpathmoveto{\pgfqpoint{0.000000in}{0.000000in}}%
\pgfpathlineto{\pgfqpoint{0.000000in}{-0.013889in}}%
\pgfusepath{stroke,fill}%
}%
\begin{pgfscope}%
\pgfsys@transformshift{1.897185in}{0.638889in}%
\pgfsys@useobject{currentmarker}{}%
\end{pgfscope}%
\end{pgfscope}%
\begin{pgfscope}%
\definecolor{textcolor}{rgb}{0.000000,0.000000,0.000000}%
\pgfsetstrokecolor{textcolor}%
\pgfsetfillcolor{textcolor}%
\pgftext[x=1.897185in,y=0.576389in,,top]{\color{textcolor}{\ifdefined\pdftexversion\else\setmainfont{NanumMyeongjo}\rmfamily\fi\fontsize{9.000000}{10.800000}\selectfont\catcode`\^=\active\def^{\ifmmode\sp\else\^{}\fi}\catcode`\%=\active\def%{\%}2018}}%
\end{pgfscope}%
\begin{pgfscope}%
\pgfsetbuttcap%
\pgfsetroundjoin%
\definecolor{currentfill}{rgb}{0.000000,0.000000,0.000000}%
\pgfsetfillcolor{currentfill}%
\pgfsetlinewidth{0.752812pt}%
\definecolor{currentstroke}{rgb}{0.000000,0.000000,0.000000}%
\pgfsetstrokecolor{currentstroke}%
\pgfsetdash{}{0pt}%
\pgfsys@defobject{currentmarker}{\pgfqpoint{0.000000in}{-0.013889in}}{\pgfqpoint{0.000000in}{0.000000in}}{%
\pgfpathmoveto{\pgfqpoint{0.000000in}{0.000000in}}%
\pgfpathlineto{\pgfqpoint{0.000000in}{-0.013889in}}%
\pgfusepath{stroke,fill}%
}%
\begin{pgfscope}%
\pgfsys@transformshift{2.338629in}{0.638889in}%
\pgfsys@useobject{currentmarker}{}%
\end{pgfscope}%
\end{pgfscope}%
\begin{pgfscope}%
\definecolor{textcolor}{rgb}{0.000000,0.000000,0.000000}%
\pgfsetstrokecolor{textcolor}%
\pgfsetfillcolor{textcolor}%
\pgftext[x=2.338629in,y=0.576389in,,top]{\color{textcolor}{\ifdefined\pdftexversion\else\setmainfont{NanumMyeongjo}\rmfamily\fi\fontsize{9.000000}{10.800000}\selectfont\catcode`\^=\active\def^{\ifmmode\sp\else\^{}\fi}\catcode`\%=\active\def%{\%}2019}}%
\end{pgfscope}%
\begin{pgfscope}%
\pgfsetbuttcap%
\pgfsetroundjoin%
\definecolor{currentfill}{rgb}{0.000000,0.000000,0.000000}%
\pgfsetfillcolor{currentfill}%
\pgfsetlinewidth{0.752812pt}%
\definecolor{currentstroke}{rgb}{0.000000,0.000000,0.000000}%
\pgfsetstrokecolor{currentstroke}%
\pgfsetdash{}{0pt}%
\pgfsys@defobject{currentmarker}{\pgfqpoint{0.000000in}{-0.013889in}}{\pgfqpoint{0.000000in}{0.000000in}}{%
\pgfpathmoveto{\pgfqpoint{0.000000in}{0.000000in}}%
\pgfpathlineto{\pgfqpoint{0.000000in}{-0.013889in}}%
\pgfusepath{stroke,fill}%
}%
\begin{pgfscope}%
\pgfsys@transformshift{2.780074in}{0.638889in}%
\pgfsys@useobject{currentmarker}{}%
\end{pgfscope}%
\end{pgfscope}%
\begin{pgfscope}%
\definecolor{textcolor}{rgb}{0.000000,0.000000,0.000000}%
\pgfsetstrokecolor{textcolor}%
\pgfsetfillcolor{textcolor}%
\pgftext[x=2.780074in,y=0.576389in,,top]{\color{textcolor}{\ifdefined\pdftexversion\else\setmainfont{NanumMyeongjo}\rmfamily\fi\fontsize{9.000000}{10.800000}\selectfont\catcode`\^=\active\def^{\ifmmode\sp\else\^{}\fi}\catcode`\%=\active\def%{\%}2020}}%
\end{pgfscope}%
\begin{pgfscope}%
\pgfsetbuttcap%
\pgfsetroundjoin%
\definecolor{currentfill}{rgb}{0.000000,0.000000,0.000000}%
\pgfsetfillcolor{currentfill}%
\pgfsetlinewidth{0.752812pt}%
\definecolor{currentstroke}{rgb}{0.000000,0.000000,0.000000}%
\pgfsetstrokecolor{currentstroke}%
\pgfsetdash{}{0pt}%
\pgfsys@defobject{currentmarker}{\pgfqpoint{0.000000in}{-0.013889in}}{\pgfqpoint{0.000000in}{0.000000in}}{%
\pgfpathmoveto{\pgfqpoint{0.000000in}{0.000000in}}%
\pgfpathlineto{\pgfqpoint{0.000000in}{-0.013889in}}%
\pgfusepath{stroke,fill}%
}%
\begin{pgfscope}%
\pgfsys@transformshift{3.221519in}{0.638889in}%
\pgfsys@useobject{currentmarker}{}%
\end{pgfscope}%
\end{pgfscope}%
\begin{pgfscope}%
\definecolor{textcolor}{rgb}{0.000000,0.000000,0.000000}%
\pgfsetstrokecolor{textcolor}%
\pgfsetfillcolor{textcolor}%
\pgftext[x=3.221519in,y=0.576389in,,top]{\color{textcolor}{\ifdefined\pdftexversion\else\setmainfont{NanumMyeongjo}\rmfamily\fi\fontsize{9.000000}{10.800000}\selectfont\catcode`\^=\active\def^{\ifmmode\sp\else\^{}\fi}\catcode`\%=\active\def%{\%}2021}}%
\end{pgfscope}%
\begin{pgfscope}%
\pgfsetbuttcap%
\pgfsetroundjoin%
\definecolor{currentfill}{rgb}{0.000000,0.000000,0.000000}%
\pgfsetfillcolor{currentfill}%
\pgfsetlinewidth{0.752812pt}%
\definecolor{currentstroke}{rgb}{0.000000,0.000000,0.000000}%
\pgfsetstrokecolor{currentstroke}%
\pgfsetdash{}{0pt}%
\pgfsys@defobject{currentmarker}{\pgfqpoint{0.000000in}{-0.013889in}}{\pgfqpoint{0.000000in}{0.000000in}}{%
\pgfpathmoveto{\pgfqpoint{0.000000in}{0.000000in}}%
\pgfpathlineto{\pgfqpoint{0.000000in}{-0.013889in}}%
\pgfusepath{stroke,fill}%
}%
\begin{pgfscope}%
\pgfsys@transformshift{3.662963in}{0.638889in}%
\pgfsys@useobject{currentmarker}{}%
\end{pgfscope}%
\end{pgfscope}%
\begin{pgfscope}%
\definecolor{textcolor}{rgb}{0.000000,0.000000,0.000000}%
\pgfsetstrokecolor{textcolor}%
\pgfsetfillcolor{textcolor}%
\pgftext[x=3.662963in,y=0.576389in,,top]{\color{textcolor}{\ifdefined\pdftexversion\else\setmainfont{NanumMyeongjo}\rmfamily\fi\fontsize{9.000000}{10.800000}\selectfont\catcode`\^=\active\def^{\ifmmode\sp\else\^{}\fi}\catcode`\%=\active\def%{\%}2022}}%
\end{pgfscope}%
\begin{pgfscope}%
\pgfsetbuttcap%
\pgfsetroundjoin%
\definecolor{currentfill}{rgb}{0.000000,0.000000,0.000000}%
\pgfsetfillcolor{currentfill}%
\pgfsetlinewidth{0.752812pt}%
\definecolor{currentstroke}{rgb}{0.000000,0.000000,0.000000}%
\pgfsetstrokecolor{currentstroke}%
\pgfsetdash{}{0pt}%
\pgfsys@defobject{currentmarker}{\pgfqpoint{0.000000in}{-0.013889in}}{\pgfqpoint{0.000000in}{0.000000in}}{%
\pgfpathmoveto{\pgfqpoint{0.000000in}{0.000000in}}%
\pgfpathlineto{\pgfqpoint{0.000000in}{-0.013889in}}%
\pgfusepath{stroke,fill}%
}%
\begin{pgfscope}%
\pgfsys@transformshift{4.104408in}{0.638889in}%
\pgfsys@useobject{currentmarker}{}%
\end{pgfscope}%
\end{pgfscope}%
\begin{pgfscope}%
\definecolor{textcolor}{rgb}{0.000000,0.000000,0.000000}%
\pgfsetstrokecolor{textcolor}%
\pgfsetfillcolor{textcolor}%
\pgftext[x=4.104408in,y=0.576389in,,top]{\color{textcolor}{\ifdefined\pdftexversion\else\setmainfont{NanumMyeongjo}\rmfamily\fi\fontsize{9.000000}{10.800000}\selectfont\catcode`\^=\active\def^{\ifmmode\sp\else\^{}\fi}\catcode`\%=\active\def%{\%}2023}}%
\end{pgfscope}%
\begin{pgfscope}%
\pgfsetbuttcap%
\pgfsetroundjoin%
\definecolor{currentfill}{rgb}{0.000000,0.000000,0.000000}%
\pgfsetfillcolor{currentfill}%
\pgfsetlinewidth{0.752812pt}%
\definecolor{currentstroke}{rgb}{0.000000,0.000000,0.000000}%
\pgfsetstrokecolor{currentstroke}%
\pgfsetdash{}{0pt}%
\pgfsys@defobject{currentmarker}{\pgfqpoint{0.000000in}{-0.013889in}}{\pgfqpoint{0.000000in}{0.000000in}}{%
\pgfpathmoveto{\pgfqpoint{0.000000in}{0.000000in}}%
\pgfpathlineto{\pgfqpoint{0.000000in}{-0.013889in}}%
\pgfusepath{stroke,fill}%
}%
\begin{pgfscope}%
\pgfsys@transformshift{4.545852in}{0.638889in}%
\pgfsys@useobject{currentmarker}{}%
\end{pgfscope}%
\end{pgfscope}%
\begin{pgfscope}%
\definecolor{textcolor}{rgb}{0.000000,0.000000,0.000000}%
\pgfsetstrokecolor{textcolor}%
\pgfsetfillcolor{textcolor}%
\pgftext[x=4.545852in,y=0.576389in,,top]{\color{textcolor}{\ifdefined\pdftexversion\else\setmainfont{NanumMyeongjo}\rmfamily\fi\fontsize{9.000000}{10.800000}\selectfont\catcode`\^=\active\def^{\ifmmode\sp\else\^{}\fi}\catcode`\%=\active\def%{\%}2024}}%
\end{pgfscope}%
\begin{pgfscope}%
\pgfsetbuttcap%
\pgfsetroundjoin%
\definecolor{currentfill}{rgb}{0.000000,0.000000,0.000000}%
\pgfsetfillcolor{currentfill}%
\pgfsetlinewidth{0.752812pt}%
\definecolor{currentstroke}{rgb}{0.000000,0.000000,0.000000}%
\pgfsetstrokecolor{currentstroke}%
\pgfsetdash{}{0pt}%
\pgfsys@defobject{currentmarker}{\pgfqpoint{0.000000in}{-0.013889in}}{\pgfqpoint{0.000000in}{0.000000in}}{%
\pgfpathmoveto{\pgfqpoint{0.000000in}{0.000000in}}%
\pgfpathlineto{\pgfqpoint{0.000000in}{-0.013889in}}%
\pgfusepath{stroke,fill}%
}%
\begin{pgfscope}%
\pgfsys@transformshift{4.987297in}{0.638889in}%
\pgfsys@useobject{currentmarker}{}%
\end{pgfscope}%
\end{pgfscope}%
\begin{pgfscope}%
\definecolor{textcolor}{rgb}{0.000000,0.000000,0.000000}%
\pgfsetstrokecolor{textcolor}%
\pgfsetfillcolor{textcolor}%
\pgftext[x=4.987297in,y=0.576389in,,top]{\color{textcolor}{\ifdefined\pdftexversion\else\setmainfont{NanumMyeongjo}\rmfamily\fi\fontsize{9.000000}{10.800000}\selectfont\catcode`\^=\active\def^{\ifmmode\sp\else\^{}\fi}\catcode`\%=\active\def%{\%}2025}}%
\end{pgfscope}%
\begin{pgfscope}%
\pgfpathrectangle{\pgfqpoint{0.781250in}{0.638889in}}{\pgfqpoint{4.843750in}{2.172222in}}%
\pgfusepath{clip}%
\pgfsetbuttcap%
\pgfsetroundjoin%
\pgfsetlinewidth{0.602250pt}%
\definecolor{currentstroke}{rgb}{0.690196,0.690196,0.690196}%
\pgfsetstrokecolor{currentstroke}%
\pgfsetstrokeopacity{0.450000}%
\pgfsetdash{{2.220000pt}{0.960000pt}}{0.000000pt}%
\pgfpathmoveto{\pgfqpoint{0.781250in}{0.638889in}}%
\pgfpathlineto{\pgfqpoint{5.625000in}{0.638889in}}%
\pgfusepath{stroke}%
\end{pgfscope}%
\begin{pgfscope}%
\pgfsetbuttcap%
\pgfsetroundjoin%
\definecolor{currentfill}{rgb}{0.000000,0.000000,0.000000}%
\pgfsetfillcolor{currentfill}%
\pgfsetlinewidth{0.752812pt}%
\definecolor{currentstroke}{rgb}{0.000000,0.000000,0.000000}%
\pgfsetstrokecolor{currentstroke}%
\pgfsetdash{}{0pt}%
\pgfsys@defobject{currentmarker}{\pgfqpoint{-0.013889in}{0.000000in}}{\pgfqpoint{-0.000000in}{0.000000in}}{%
\pgfpathmoveto{\pgfqpoint{-0.000000in}{0.000000in}}%
\pgfpathlineto{\pgfqpoint{-0.013889in}{0.000000in}}%
\pgfusepath{stroke,fill}%
}%
\begin{pgfscope}%
\pgfsys@transformshift{0.781250in}{0.638889in}%
\pgfsys@useobject{currentmarker}{}%
\end{pgfscope}%
\end{pgfscope}%
\begin{pgfscope}%
\definecolor{textcolor}{rgb}{0.000000,0.000000,0.000000}%
\pgfsetstrokecolor{textcolor}%
\pgfsetfillcolor{textcolor}%
\pgftext[x=0.651611in, y=0.588962in, left, base]{\color{textcolor}{\ifdefined\pdftexversion\else\setmainfont{NanumMyeongjo}\rmfamily\fi\fontsize{9.000000}{10.800000}\selectfont\catcode`\^=\active\def^{\ifmmode\sp\else\^{}\fi}\catcode`\%=\active\def%{\%}0}}%
\end{pgfscope}%
\begin{pgfscope}%
\pgfpathrectangle{\pgfqpoint{0.781250in}{0.638889in}}{\pgfqpoint{4.843750in}{2.172222in}}%
\pgfusepath{clip}%
\pgfsetbuttcap%
\pgfsetroundjoin%
\pgfsetlinewidth{0.602250pt}%
\definecolor{currentstroke}{rgb}{0.690196,0.690196,0.690196}%
\pgfsetstrokecolor{currentstroke}%
\pgfsetstrokeopacity{0.450000}%
\pgfsetdash{{2.220000pt}{0.960000pt}}{0.000000pt}%
\pgfpathmoveto{\pgfqpoint{0.781250in}{0.949206in}}%
\pgfpathlineto{\pgfqpoint{5.625000in}{0.949206in}}%
\pgfusepath{stroke}%
\end{pgfscope}%
\begin{pgfscope}%
\pgfsetbuttcap%
\pgfsetroundjoin%
\definecolor{currentfill}{rgb}{0.000000,0.000000,0.000000}%
\pgfsetfillcolor{currentfill}%
\pgfsetlinewidth{0.752812pt}%
\definecolor{currentstroke}{rgb}{0.000000,0.000000,0.000000}%
\pgfsetstrokecolor{currentstroke}%
\pgfsetdash{}{0pt}%
\pgfsys@defobject{currentmarker}{\pgfqpoint{-0.013889in}{0.000000in}}{\pgfqpoint{-0.000000in}{0.000000in}}{%
\pgfpathmoveto{\pgfqpoint{-0.000000in}{0.000000in}}%
\pgfpathlineto{\pgfqpoint{-0.013889in}{0.000000in}}%
\pgfusepath{stroke,fill}%
}%
\begin{pgfscope}%
\pgfsys@transformshift{0.781250in}{0.949206in}%
\pgfsys@useobject{currentmarker}{}%
\end{pgfscope}%
\end{pgfscope}%
\begin{pgfscope}%
\definecolor{textcolor}{rgb}{0.000000,0.000000,0.000000}%
\pgfsetstrokecolor{textcolor}%
\pgfsetfillcolor{textcolor}%
\pgftext[x=0.517334in, y=0.899280in, left, base]{\color{textcolor}{\ifdefined\pdftexversion\else\setmainfont{NanumMyeongjo}\rmfamily\fi\fontsize{9.000000}{10.800000}\selectfont\catcode`\^=\active\def^{\ifmmode\sp\else\^{}\fi}\catcode`\%=\active\def%{\%}100}}%
\end{pgfscope}%
\begin{pgfscope}%
\pgfpathrectangle{\pgfqpoint{0.781250in}{0.638889in}}{\pgfqpoint{4.843750in}{2.172222in}}%
\pgfusepath{clip}%
\pgfsetbuttcap%
\pgfsetroundjoin%
\pgfsetlinewidth{0.602250pt}%
\definecolor{currentstroke}{rgb}{0.690196,0.690196,0.690196}%
\pgfsetstrokecolor{currentstroke}%
\pgfsetstrokeopacity{0.450000}%
\pgfsetdash{{2.220000pt}{0.960000pt}}{0.000000pt}%
\pgfpathmoveto{\pgfqpoint{0.781250in}{1.259524in}}%
\pgfpathlineto{\pgfqpoint{5.625000in}{1.259524in}}%
\pgfusepath{stroke}%
\end{pgfscope}%
\begin{pgfscope}%
\pgfsetbuttcap%
\pgfsetroundjoin%
\definecolor{currentfill}{rgb}{0.000000,0.000000,0.000000}%
\pgfsetfillcolor{currentfill}%
\pgfsetlinewidth{0.752812pt}%
\definecolor{currentstroke}{rgb}{0.000000,0.000000,0.000000}%
\pgfsetstrokecolor{currentstroke}%
\pgfsetdash{}{0pt}%
\pgfsys@defobject{currentmarker}{\pgfqpoint{-0.013889in}{0.000000in}}{\pgfqpoint{-0.000000in}{0.000000in}}{%
\pgfpathmoveto{\pgfqpoint{-0.000000in}{0.000000in}}%
\pgfpathlineto{\pgfqpoint{-0.013889in}{0.000000in}}%
\pgfusepath{stroke,fill}%
}%
\begin{pgfscope}%
\pgfsys@transformshift{0.781250in}{1.259524in}%
\pgfsys@useobject{currentmarker}{}%
\end{pgfscope}%
\end{pgfscope}%
\begin{pgfscope}%
\definecolor{textcolor}{rgb}{0.000000,0.000000,0.000000}%
\pgfsetstrokecolor{textcolor}%
\pgfsetfillcolor{textcolor}%
\pgftext[x=0.517334in, y=1.209597in, left, base]{\color{textcolor}{\ifdefined\pdftexversion\else\setmainfont{NanumMyeongjo}\rmfamily\fi\fontsize{9.000000}{10.800000}\selectfont\catcode`\^=\active\def^{\ifmmode\sp\else\^{}\fi}\catcode`\%=\active\def%{\%}200}}%
\end{pgfscope}%
\begin{pgfscope}%
\pgfpathrectangle{\pgfqpoint{0.781250in}{0.638889in}}{\pgfqpoint{4.843750in}{2.172222in}}%
\pgfusepath{clip}%
\pgfsetbuttcap%
\pgfsetroundjoin%
\pgfsetlinewidth{0.602250pt}%
\definecolor{currentstroke}{rgb}{0.690196,0.690196,0.690196}%
\pgfsetstrokecolor{currentstroke}%
\pgfsetstrokeopacity{0.450000}%
\pgfsetdash{{2.220000pt}{0.960000pt}}{0.000000pt}%
\pgfpathmoveto{\pgfqpoint{0.781250in}{1.569841in}}%
\pgfpathlineto{\pgfqpoint{5.625000in}{1.569841in}}%
\pgfusepath{stroke}%
\end{pgfscope}%
\begin{pgfscope}%
\pgfsetbuttcap%
\pgfsetroundjoin%
\definecolor{currentfill}{rgb}{0.000000,0.000000,0.000000}%
\pgfsetfillcolor{currentfill}%
\pgfsetlinewidth{0.752812pt}%
\definecolor{currentstroke}{rgb}{0.000000,0.000000,0.000000}%
\pgfsetstrokecolor{currentstroke}%
\pgfsetdash{}{0pt}%
\pgfsys@defobject{currentmarker}{\pgfqpoint{-0.013889in}{0.000000in}}{\pgfqpoint{-0.000000in}{0.000000in}}{%
\pgfpathmoveto{\pgfqpoint{-0.000000in}{0.000000in}}%
\pgfpathlineto{\pgfqpoint{-0.013889in}{0.000000in}}%
\pgfusepath{stroke,fill}%
}%
\begin{pgfscope}%
\pgfsys@transformshift{0.781250in}{1.569841in}%
\pgfsys@useobject{currentmarker}{}%
\end{pgfscope}%
\end{pgfscope}%
\begin{pgfscope}%
\definecolor{textcolor}{rgb}{0.000000,0.000000,0.000000}%
\pgfsetstrokecolor{textcolor}%
\pgfsetfillcolor{textcolor}%
\pgftext[x=0.517334in, y=1.519915in, left, base]{\color{textcolor}{\ifdefined\pdftexversion\else\setmainfont{NanumMyeongjo}\rmfamily\fi\fontsize{9.000000}{10.800000}\selectfont\catcode`\^=\active\def^{\ifmmode\sp\else\^{}\fi}\catcode`\%=\active\def%{\%}300}}%
\end{pgfscope}%
\begin{pgfscope}%
\pgfpathrectangle{\pgfqpoint{0.781250in}{0.638889in}}{\pgfqpoint{4.843750in}{2.172222in}}%
\pgfusepath{clip}%
\pgfsetbuttcap%
\pgfsetroundjoin%
\pgfsetlinewidth{0.602250pt}%
\definecolor{currentstroke}{rgb}{0.690196,0.690196,0.690196}%
\pgfsetstrokecolor{currentstroke}%
\pgfsetstrokeopacity{0.450000}%
\pgfsetdash{{2.220000pt}{0.960000pt}}{0.000000pt}%
\pgfpathmoveto{\pgfqpoint{0.781250in}{1.880159in}}%
\pgfpathlineto{\pgfqpoint{5.625000in}{1.880159in}}%
\pgfusepath{stroke}%
\end{pgfscope}%
\begin{pgfscope}%
\pgfsetbuttcap%
\pgfsetroundjoin%
\definecolor{currentfill}{rgb}{0.000000,0.000000,0.000000}%
\pgfsetfillcolor{currentfill}%
\pgfsetlinewidth{0.752812pt}%
\definecolor{currentstroke}{rgb}{0.000000,0.000000,0.000000}%
\pgfsetstrokecolor{currentstroke}%
\pgfsetdash{}{0pt}%
\pgfsys@defobject{currentmarker}{\pgfqpoint{-0.013889in}{0.000000in}}{\pgfqpoint{-0.000000in}{0.000000in}}{%
\pgfpathmoveto{\pgfqpoint{-0.000000in}{0.000000in}}%
\pgfpathlineto{\pgfqpoint{-0.013889in}{0.000000in}}%
\pgfusepath{stroke,fill}%
}%
\begin{pgfscope}%
\pgfsys@transformshift{0.781250in}{1.880159in}%
\pgfsys@useobject{currentmarker}{}%
\end{pgfscope}%
\end{pgfscope}%
\begin{pgfscope}%
\definecolor{textcolor}{rgb}{0.000000,0.000000,0.000000}%
\pgfsetstrokecolor{textcolor}%
\pgfsetfillcolor{textcolor}%
\pgftext[x=0.517334in, y=1.830232in, left, base]{\color{textcolor}{\ifdefined\pdftexversion\else\setmainfont{NanumMyeongjo}\rmfamily\fi\fontsize{9.000000}{10.800000}\selectfont\catcode`\^=\active\def^{\ifmmode\sp\else\^{}\fi}\catcode`\%=\active\def%{\%}400}}%
\end{pgfscope}%
\begin{pgfscope}%
\pgfpathrectangle{\pgfqpoint{0.781250in}{0.638889in}}{\pgfqpoint{4.843750in}{2.172222in}}%
\pgfusepath{clip}%
\pgfsetbuttcap%
\pgfsetroundjoin%
\pgfsetlinewidth{0.602250pt}%
\definecolor{currentstroke}{rgb}{0.690196,0.690196,0.690196}%
\pgfsetstrokecolor{currentstroke}%
\pgfsetstrokeopacity{0.450000}%
\pgfsetdash{{2.220000pt}{0.960000pt}}{0.000000pt}%
\pgfpathmoveto{\pgfqpoint{0.781250in}{2.190476in}}%
\pgfpathlineto{\pgfqpoint{5.625000in}{2.190476in}}%
\pgfusepath{stroke}%
\end{pgfscope}%
\begin{pgfscope}%
\pgfsetbuttcap%
\pgfsetroundjoin%
\definecolor{currentfill}{rgb}{0.000000,0.000000,0.000000}%
\pgfsetfillcolor{currentfill}%
\pgfsetlinewidth{0.752812pt}%
\definecolor{currentstroke}{rgb}{0.000000,0.000000,0.000000}%
\pgfsetstrokecolor{currentstroke}%
\pgfsetdash{}{0pt}%
\pgfsys@defobject{currentmarker}{\pgfqpoint{-0.013889in}{0.000000in}}{\pgfqpoint{-0.000000in}{0.000000in}}{%
\pgfpathmoveto{\pgfqpoint{-0.000000in}{0.000000in}}%
\pgfpathlineto{\pgfqpoint{-0.013889in}{0.000000in}}%
\pgfusepath{stroke,fill}%
}%
\begin{pgfscope}%
\pgfsys@transformshift{0.781250in}{2.190476in}%
\pgfsys@useobject{currentmarker}{}%
\end{pgfscope}%
\end{pgfscope}%
\begin{pgfscope}%
\definecolor{textcolor}{rgb}{0.000000,0.000000,0.000000}%
\pgfsetstrokecolor{textcolor}%
\pgfsetfillcolor{textcolor}%
\pgftext[x=0.517334in, y=2.140549in, left, base]{\color{textcolor}{\ifdefined\pdftexversion\else\setmainfont{NanumMyeongjo}\rmfamily\fi\fontsize{9.000000}{10.800000}\selectfont\catcode`\^=\active\def^{\ifmmode\sp\else\^{}\fi}\catcode`\%=\active\def%{\%}500}}%
\end{pgfscope}%
\begin{pgfscope}%
\pgfpathrectangle{\pgfqpoint{0.781250in}{0.638889in}}{\pgfqpoint{4.843750in}{2.172222in}}%
\pgfusepath{clip}%
\pgfsetbuttcap%
\pgfsetroundjoin%
\pgfsetlinewidth{0.602250pt}%
\definecolor{currentstroke}{rgb}{0.690196,0.690196,0.690196}%
\pgfsetstrokecolor{currentstroke}%
\pgfsetstrokeopacity{0.450000}%
\pgfsetdash{{2.220000pt}{0.960000pt}}{0.000000pt}%
\pgfpathmoveto{\pgfqpoint{0.781250in}{2.500794in}}%
\pgfpathlineto{\pgfqpoint{5.625000in}{2.500794in}}%
\pgfusepath{stroke}%
\end{pgfscope}%
\begin{pgfscope}%
\pgfsetbuttcap%
\pgfsetroundjoin%
\definecolor{currentfill}{rgb}{0.000000,0.000000,0.000000}%
\pgfsetfillcolor{currentfill}%
\pgfsetlinewidth{0.752812pt}%
\definecolor{currentstroke}{rgb}{0.000000,0.000000,0.000000}%
\pgfsetstrokecolor{currentstroke}%
\pgfsetdash{}{0pt}%
\pgfsys@defobject{currentmarker}{\pgfqpoint{-0.013889in}{0.000000in}}{\pgfqpoint{-0.000000in}{0.000000in}}{%
\pgfpathmoveto{\pgfqpoint{-0.000000in}{0.000000in}}%
\pgfpathlineto{\pgfqpoint{-0.013889in}{0.000000in}}%
\pgfusepath{stroke,fill}%
}%
\begin{pgfscope}%
\pgfsys@transformshift{0.781250in}{2.500794in}%
\pgfsys@useobject{currentmarker}{}%
\end{pgfscope}%
\end{pgfscope}%
\begin{pgfscope}%
\definecolor{textcolor}{rgb}{0.000000,0.000000,0.000000}%
\pgfsetstrokecolor{textcolor}%
\pgfsetfillcolor{textcolor}%
\pgftext[x=0.517334in, y=2.450867in, left, base]{\color{textcolor}{\ifdefined\pdftexversion\else\setmainfont{NanumMyeongjo}\rmfamily\fi\fontsize{9.000000}{10.800000}\selectfont\catcode`\^=\active\def^{\ifmmode\sp\else\^{}\fi}\catcode`\%=\active\def%{\%}600}}%
\end{pgfscope}%
\begin{pgfscope}%
\pgfpathrectangle{\pgfqpoint{0.781250in}{0.638889in}}{\pgfqpoint{4.843750in}{2.172222in}}%
\pgfusepath{clip}%
\pgfsetbuttcap%
\pgfsetroundjoin%
\pgfsetlinewidth{0.602250pt}%
\definecolor{currentstroke}{rgb}{0.690196,0.690196,0.690196}%
\pgfsetstrokecolor{currentstroke}%
\pgfsetstrokeopacity{0.450000}%
\pgfsetdash{{2.220000pt}{0.960000pt}}{0.000000pt}%
\pgfpathmoveto{\pgfqpoint{0.781250in}{2.811111in}}%
\pgfpathlineto{\pgfqpoint{5.625000in}{2.811111in}}%
\pgfusepath{stroke}%
\end{pgfscope}%
\begin{pgfscope}%
\pgfsetbuttcap%
\pgfsetroundjoin%
\definecolor{currentfill}{rgb}{0.000000,0.000000,0.000000}%
\pgfsetfillcolor{currentfill}%
\pgfsetlinewidth{0.752812pt}%
\definecolor{currentstroke}{rgb}{0.000000,0.000000,0.000000}%
\pgfsetstrokecolor{currentstroke}%
\pgfsetdash{}{0pt}%
\pgfsys@defobject{currentmarker}{\pgfqpoint{-0.013889in}{0.000000in}}{\pgfqpoint{-0.000000in}{0.000000in}}{%
\pgfpathmoveto{\pgfqpoint{-0.000000in}{0.000000in}}%
\pgfpathlineto{\pgfqpoint{-0.013889in}{0.000000in}}%
\pgfusepath{stroke,fill}%
}%
\begin{pgfscope}%
\pgfsys@transformshift{0.781250in}{2.811111in}%
\pgfsys@useobject{currentmarker}{}%
\end{pgfscope}%
\end{pgfscope}%
\begin{pgfscope}%
\definecolor{textcolor}{rgb}{0.000000,0.000000,0.000000}%
\pgfsetstrokecolor{textcolor}%
\pgfsetfillcolor{textcolor}%
\pgftext[x=0.517334in, y=2.761184in, left, base]{\color{textcolor}{\ifdefined\pdftexversion\else\setmainfont{NanumMyeongjo}\rmfamily\fi\fontsize{9.000000}{10.800000}\selectfont\catcode`\^=\active\def^{\ifmmode\sp\else\^{}\fi}\catcode`\%=\active\def%{\%}700}}%
\end{pgfscope}%
\begin{pgfscope}%
\pgfsetrectcap%
\pgfsetmiterjoin%
\pgfsetlinewidth{0.752812pt}%
\definecolor{currentstroke}{rgb}{0.000000,0.000000,0.000000}%
\pgfsetstrokecolor{currentstroke}%
\pgfsetdash{}{0pt}%
\pgfpathmoveto{\pgfqpoint{0.781250in}{0.638889in}}%
\pgfpathlineto{\pgfqpoint{0.781250in}{2.811111in}}%
\pgfusepath{stroke}%
\end{pgfscope}%
\begin{pgfscope}%
\pgfsetrectcap%
\pgfsetmiterjoin%
\pgfsetlinewidth{0.752812pt}%
\definecolor{currentstroke}{rgb}{0.000000,0.000000,0.000000}%
\pgfsetstrokecolor{currentstroke}%
\pgfsetdash{}{0pt}%
\pgfpathmoveto{\pgfqpoint{0.781250in}{0.638889in}}%
\pgfpathlineto{\pgfqpoint{5.625000in}{0.638889in}}%
\pgfusepath{stroke}%
\end{pgfscope}%
\begin{pgfscope}%
\pgfpathrectangle{\pgfqpoint{0.781250in}{0.638889in}}{\pgfqpoint{4.843750in}{2.172222in}}%
\pgfusepath{clip}%
\pgfsetbuttcap%
\pgfsetmiterjoin%
\definecolor{currentfill}{rgb}{0.227451,0.192157,0.427451}%
\pgfsetfillcolor{currentfill}%
\pgfsetlinewidth{0.100375pt}%
\definecolor{currentstroke}{rgb}{0.266667,0.266667,0.266667}%
\pgfsetstrokecolor{currentstroke}%
\pgfsetdash{}{0pt}%
\pgfpathmoveto{\pgfqpoint{1.001420in}{0.638889in}}%
\pgfpathlineto{\pgfqpoint{1.027171in}{0.638889in}}%
\pgfpathlineto{\pgfqpoint{1.027171in}{0.797461in}}%
\pgfpathlineto{\pgfqpoint{1.001420in}{0.797461in}}%
\pgfpathlineto{\pgfqpoint{1.001420in}{0.638889in}}%
\pgfpathclose%
\pgfusepath{stroke,fill}%
\end{pgfscope}%
\begin{pgfscope}%
\pgfpathrectangle{\pgfqpoint{0.781250in}{0.638889in}}{\pgfqpoint{4.843750in}{2.172222in}}%
\pgfusepath{clip}%
\pgfsetbuttcap%
\pgfsetmiterjoin%
\definecolor{currentfill}{rgb}{0.227451,0.192157,0.427451}%
\pgfsetfillcolor{currentfill}%
\pgfsetlinewidth{0.100375pt}%
\definecolor{currentstroke}{rgb}{0.266667,0.266667,0.266667}%
\pgfsetstrokecolor{currentstroke}%
\pgfsetdash{}{0pt}%
\pgfpathmoveto{\pgfqpoint{1.038207in}{0.638889in}}%
\pgfpathlineto{\pgfqpoint{1.063958in}{0.638889in}}%
\pgfpathlineto{\pgfqpoint{1.063958in}{0.764878in}}%
\pgfpathlineto{\pgfqpoint{1.038207in}{0.764878in}}%
\pgfpathlineto{\pgfqpoint{1.038207in}{0.638889in}}%
\pgfpathclose%
\pgfusepath{stroke,fill}%
\end{pgfscope}%
\begin{pgfscope}%
\pgfpathrectangle{\pgfqpoint{0.781250in}{0.638889in}}{\pgfqpoint{4.843750in}{2.172222in}}%
\pgfusepath{clip}%
\pgfsetbuttcap%
\pgfsetmiterjoin%
\definecolor{currentfill}{rgb}{0.227451,0.192157,0.427451}%
\pgfsetfillcolor{currentfill}%
\pgfsetlinewidth{0.100375pt}%
\definecolor{currentstroke}{rgb}{0.266667,0.266667,0.266667}%
\pgfsetstrokecolor{currentstroke}%
\pgfsetdash{}{0pt}%
\pgfpathmoveto{\pgfqpoint{1.074995in}{0.638889in}}%
\pgfpathlineto{\pgfqpoint{1.100745in}{0.638889in}}%
\pgfpathlineto{\pgfqpoint{1.100745in}{0.800254in}}%
\pgfpathlineto{\pgfqpoint{1.074995in}{0.800254in}}%
\pgfpathlineto{\pgfqpoint{1.074995in}{0.638889in}}%
\pgfpathclose%
\pgfusepath{stroke,fill}%
\end{pgfscope}%
\begin{pgfscope}%
\pgfpathrectangle{\pgfqpoint{0.781250in}{0.638889in}}{\pgfqpoint{4.843750in}{2.172222in}}%
\pgfusepath{clip}%
\pgfsetbuttcap%
\pgfsetmiterjoin%
\definecolor{currentfill}{rgb}{0.227451,0.192157,0.427451}%
\pgfsetfillcolor{currentfill}%
\pgfsetlinewidth{0.100375pt}%
\definecolor{currentstroke}{rgb}{0.266667,0.266667,0.266667}%
\pgfsetstrokecolor{currentstroke}%
\pgfsetdash{}{0pt}%
\pgfpathmoveto{\pgfqpoint{1.111782in}{0.638889in}}%
\pgfpathlineto{\pgfqpoint{1.137533in}{0.638889in}}%
\pgfpathlineto{\pgfqpoint{1.137533in}{1.195909in}}%
\pgfpathlineto{\pgfqpoint{1.111782in}{1.195909in}}%
\pgfpathlineto{\pgfqpoint{1.111782in}{0.638889in}}%
\pgfpathclose%
\pgfusepath{stroke,fill}%
\end{pgfscope}%
\begin{pgfscope}%
\pgfpathrectangle{\pgfqpoint{0.781250in}{0.638889in}}{\pgfqpoint{4.843750in}{2.172222in}}%
\pgfusepath{clip}%
\pgfsetbuttcap%
\pgfsetmiterjoin%
\definecolor{currentfill}{rgb}{0.227451,0.192157,0.427451}%
\pgfsetfillcolor{currentfill}%
\pgfsetlinewidth{0.100375pt}%
\definecolor{currentstroke}{rgb}{0.266667,0.266667,0.266667}%
\pgfsetstrokecolor{currentstroke}%
\pgfsetdash{}{0pt}%
\pgfpathmoveto{\pgfqpoint{1.148569in}{0.638889in}}%
\pgfpathlineto{\pgfqpoint{1.174320in}{0.638889in}}%
\pgfpathlineto{\pgfqpoint{1.174320in}{0.948896in}}%
\pgfpathlineto{\pgfqpoint{1.148569in}{0.948896in}}%
\pgfpathlineto{\pgfqpoint{1.148569in}{0.638889in}}%
\pgfpathclose%
\pgfusepath{stroke,fill}%
\end{pgfscope}%
\begin{pgfscope}%
\pgfpathrectangle{\pgfqpoint{0.781250in}{0.638889in}}{\pgfqpoint{4.843750in}{2.172222in}}%
\pgfusepath{clip}%
\pgfsetbuttcap%
\pgfsetmiterjoin%
\definecolor{currentfill}{rgb}{0.164706,0.615686,0.560784}%
\pgfsetfillcolor{currentfill}%
\pgfsetlinewidth{0.100375pt}%
\definecolor{currentstroke}{rgb}{0.266667,0.266667,0.266667}%
\pgfsetstrokecolor{currentstroke}%
\pgfsetdash{}{0pt}%
\pgfpathmoveto{\pgfqpoint{1.185356in}{0.638889in}}%
\pgfpathlineto{\pgfqpoint{1.211107in}{0.638889in}}%
\pgfpathlineto{\pgfqpoint{1.211107in}{0.769222in}}%
\pgfpathlineto{\pgfqpoint{1.185356in}{0.769222in}}%
\pgfpathlineto{\pgfqpoint{1.185356in}{0.638889in}}%
\pgfpathclose%
\pgfusepath{stroke,fill}%
\end{pgfscope}%
\begin{pgfscope}%
\pgfpathrectangle{\pgfqpoint{0.781250in}{0.638889in}}{\pgfqpoint{4.843750in}{2.172222in}}%
\pgfusepath{clip}%
\pgfsetbuttcap%
\pgfsetmiterjoin%
\definecolor{currentfill}{rgb}{0.164706,0.615686,0.560784}%
\pgfsetfillcolor{currentfill}%
\pgfsetlinewidth{0.100375pt}%
\definecolor{currentstroke}{rgb}{0.266667,0.266667,0.266667}%
\pgfsetstrokecolor{currentstroke}%
\pgfsetdash{}{0pt}%
\pgfpathmoveto{\pgfqpoint{1.222143in}{0.638889in}}%
\pgfpathlineto{\pgfqpoint{1.247894in}{0.638889in}}%
\pgfpathlineto{\pgfqpoint{1.247894in}{1.323760in}}%
\pgfpathlineto{\pgfqpoint{1.222143in}{1.323760in}}%
\pgfpathlineto{\pgfqpoint{1.222143in}{0.638889in}}%
\pgfpathclose%
\pgfusepath{stroke,fill}%
\end{pgfscope}%
\begin{pgfscope}%
\pgfpathrectangle{\pgfqpoint{0.781250in}{0.638889in}}{\pgfqpoint{4.843750in}{2.172222in}}%
\pgfusepath{clip}%
\pgfsetbuttcap%
\pgfsetmiterjoin%
\definecolor{currentfill}{rgb}{0.164706,0.615686,0.560784}%
\pgfsetfillcolor{currentfill}%
\pgfsetlinewidth{0.100375pt}%
\definecolor{currentstroke}{rgb}{0.266667,0.266667,0.266667}%
\pgfsetstrokecolor{currentstroke}%
\pgfsetdash{}{0pt}%
\pgfpathmoveto{\pgfqpoint{1.258930in}{0.638889in}}%
\pgfpathlineto{\pgfqpoint{1.284681in}{0.638889in}}%
\pgfpathlineto{\pgfqpoint{1.284681in}{0.732605in}}%
\pgfpathlineto{\pgfqpoint{1.258930in}{0.732605in}}%
\pgfpathlineto{\pgfqpoint{1.258930in}{0.638889in}}%
\pgfpathclose%
\pgfusepath{stroke,fill}%
\end{pgfscope}%
\begin{pgfscope}%
\pgfpathrectangle{\pgfqpoint{0.781250in}{0.638889in}}{\pgfqpoint{4.843750in}{2.172222in}}%
\pgfusepath{clip}%
\pgfsetbuttcap%
\pgfsetmiterjoin%
\definecolor{currentfill}{rgb}{0.164706,0.615686,0.560784}%
\pgfsetfillcolor{currentfill}%
\pgfsetlinewidth{0.100375pt}%
\definecolor{currentstroke}{rgb}{0.266667,0.266667,0.266667}%
\pgfsetstrokecolor{currentstroke}%
\pgfsetdash{}{0pt}%
\pgfpathmoveto{\pgfqpoint{1.295717in}{0.638889in}}%
\pgfpathlineto{\pgfqpoint{1.321468in}{0.638889in}}%
\pgfpathlineto{\pgfqpoint{1.321468in}{1.071161in}}%
\pgfpathlineto{\pgfqpoint{1.295717in}{1.071161in}}%
\pgfpathlineto{\pgfqpoint{1.295717in}{0.638889in}}%
\pgfpathclose%
\pgfusepath{stroke,fill}%
\end{pgfscope}%
\begin{pgfscope}%
\pgfpathrectangle{\pgfqpoint{0.781250in}{0.638889in}}{\pgfqpoint{4.843750in}{2.172222in}}%
\pgfusepath{clip}%
\pgfsetbuttcap%
\pgfsetmiterjoin%
\definecolor{currentfill}{rgb}{0.227451,0.192157,0.427451}%
\pgfsetfillcolor{currentfill}%
\pgfsetlinewidth{0.100375pt}%
\definecolor{currentstroke}{rgb}{0.266667,0.266667,0.266667}%
\pgfsetstrokecolor{currentstroke}%
\pgfsetdash{}{0pt}%
\pgfpathmoveto{\pgfqpoint{1.332504in}{0.638889in}}%
\pgfpathlineto{\pgfqpoint{1.358255in}{0.638889in}}%
\pgfpathlineto{\pgfqpoint{1.358255in}{1.038267in}}%
\pgfpathlineto{\pgfqpoint{1.332504in}{1.038267in}}%
\pgfpathlineto{\pgfqpoint{1.332504in}{0.638889in}}%
\pgfpathclose%
\pgfusepath{stroke,fill}%
\end{pgfscope}%
\begin{pgfscope}%
\pgfpathrectangle{\pgfqpoint{0.781250in}{0.638889in}}{\pgfqpoint{4.843750in}{2.172222in}}%
\pgfusepath{clip}%
\pgfsetbuttcap%
\pgfsetmiterjoin%
\definecolor{currentfill}{rgb}{0.227451,0.192157,0.427451}%
\pgfsetfillcolor{currentfill}%
\pgfsetlinewidth{0.100375pt}%
\definecolor{currentstroke}{rgb}{0.266667,0.266667,0.266667}%
\pgfsetstrokecolor{currentstroke}%
\pgfsetdash{}{0pt}%
\pgfpathmoveto{\pgfqpoint{1.369291in}{0.638889in}}%
\pgfpathlineto{\pgfqpoint{1.395042in}{0.638889in}}%
\pgfpathlineto{\pgfqpoint{1.395042in}{0.746259in}}%
\pgfpathlineto{\pgfqpoint{1.369291in}{0.746259in}}%
\pgfpathlineto{\pgfqpoint{1.369291in}{0.638889in}}%
\pgfpathclose%
\pgfusepath{stroke,fill}%
\end{pgfscope}%
\begin{pgfscope}%
\pgfpathrectangle{\pgfqpoint{0.781250in}{0.638889in}}{\pgfqpoint{4.843750in}{2.172222in}}%
\pgfusepath{clip}%
\pgfsetbuttcap%
\pgfsetmiterjoin%
\definecolor{currentfill}{rgb}{0.227451,0.192157,0.427451}%
\pgfsetfillcolor{currentfill}%
\pgfsetlinewidth{0.100375pt}%
\definecolor{currentstroke}{rgb}{0.266667,0.266667,0.266667}%
\pgfsetstrokecolor{currentstroke}%
\pgfsetdash{}{0pt}%
\pgfpathmoveto{\pgfqpoint{1.406078in}{0.638889in}}%
\pgfpathlineto{\pgfqpoint{1.431829in}{0.638889in}}%
\pgfpathlineto{\pgfqpoint{1.431829in}{0.815460in}}%
\pgfpathlineto{\pgfqpoint{1.406078in}{0.815460in}}%
\pgfpathlineto{\pgfqpoint{1.406078in}{0.638889in}}%
\pgfpathclose%
\pgfusepath{stroke,fill}%
\end{pgfscope}%
\begin{pgfscope}%
\pgfpathrectangle{\pgfqpoint{0.781250in}{0.638889in}}{\pgfqpoint{4.843750in}{2.172222in}}%
\pgfusepath{clip}%
\pgfsetbuttcap%
\pgfsetmiterjoin%
\definecolor{currentfill}{rgb}{0.227451,0.192157,0.427451}%
\pgfsetfillcolor{currentfill}%
\pgfsetlinewidth{0.100375pt}%
\definecolor{currentstroke}{rgb}{0.266667,0.266667,0.266667}%
\pgfsetstrokecolor{currentstroke}%
\pgfsetdash{}{0pt}%
\pgfpathmoveto{\pgfqpoint{1.442865in}{0.638889in}}%
\pgfpathlineto{\pgfqpoint{1.468616in}{0.638889in}}%
\pgfpathlineto{\pgfqpoint{1.468616in}{0.696918in}}%
\pgfpathlineto{\pgfqpoint{1.442865in}{0.696918in}}%
\pgfpathlineto{\pgfqpoint{1.442865in}{0.638889in}}%
\pgfpathclose%
\pgfusepath{stroke,fill}%
\end{pgfscope}%
\begin{pgfscope}%
\pgfpathrectangle{\pgfqpoint{0.781250in}{0.638889in}}{\pgfqpoint{4.843750in}{2.172222in}}%
\pgfusepath{clip}%
\pgfsetbuttcap%
\pgfsetmiterjoin%
\definecolor{currentfill}{rgb}{0.227451,0.192157,0.427451}%
\pgfsetfillcolor{currentfill}%
\pgfsetlinewidth{0.100375pt}%
\definecolor{currentstroke}{rgb}{0.266667,0.266667,0.266667}%
\pgfsetstrokecolor{currentstroke}%
\pgfsetdash{}{0pt}%
\pgfpathmoveto{\pgfqpoint{1.479652in}{0.638889in}}%
\pgfpathlineto{\pgfqpoint{1.505403in}{0.638889in}}%
\pgfpathlineto{\pgfqpoint{1.505403in}{0.756189in}}%
\pgfpathlineto{\pgfqpoint{1.479652in}{0.756189in}}%
\pgfpathlineto{\pgfqpoint{1.479652in}{0.638889in}}%
\pgfpathclose%
\pgfusepath{stroke,fill}%
\end{pgfscope}%
\begin{pgfscope}%
\pgfpathrectangle{\pgfqpoint{0.781250in}{0.638889in}}{\pgfqpoint{4.843750in}{2.172222in}}%
\pgfusepath{clip}%
\pgfsetbuttcap%
\pgfsetmiterjoin%
\definecolor{currentfill}{rgb}{0.227451,0.192157,0.427451}%
\pgfsetfillcolor{currentfill}%
\pgfsetlinewidth{0.100375pt}%
\definecolor{currentstroke}{rgb}{0.266667,0.266667,0.266667}%
\pgfsetstrokecolor{currentstroke}%
\pgfsetdash{}{0pt}%
\pgfpathmoveto{\pgfqpoint{1.516439in}{0.638889in}}%
\pgfpathlineto{\pgfqpoint{1.542190in}{0.638889in}}%
\pgfpathlineto{\pgfqpoint{1.542190in}{0.706848in}}%
\pgfpathlineto{\pgfqpoint{1.516439in}{0.706848in}}%
\pgfpathlineto{\pgfqpoint{1.516439in}{0.638889in}}%
\pgfpathclose%
\pgfusepath{stroke,fill}%
\end{pgfscope}%
\begin{pgfscope}%
\pgfpathrectangle{\pgfqpoint{0.781250in}{0.638889in}}{\pgfqpoint{4.843750in}{2.172222in}}%
\pgfusepath{clip}%
\pgfsetbuttcap%
\pgfsetmiterjoin%
\definecolor{currentfill}{rgb}{0.227451,0.192157,0.427451}%
\pgfsetfillcolor{currentfill}%
\pgfsetlinewidth{0.100375pt}%
\definecolor{currentstroke}{rgb}{0.266667,0.266667,0.266667}%
\pgfsetstrokecolor{currentstroke}%
\pgfsetdash{}{0pt}%
\pgfpathmoveto{\pgfqpoint{1.553226in}{0.638889in}}%
\pgfpathlineto{\pgfqpoint{1.578977in}{0.638889in}}%
\pgfpathlineto{\pgfqpoint{1.578977in}{0.828493in}}%
\pgfpathlineto{\pgfqpoint{1.553226in}{0.828493in}}%
\pgfpathlineto{\pgfqpoint{1.553226in}{0.638889in}}%
\pgfpathclose%
\pgfusepath{stroke,fill}%
\end{pgfscope}%
\begin{pgfscope}%
\pgfpathrectangle{\pgfqpoint{0.781250in}{0.638889in}}{\pgfqpoint{4.843750in}{2.172222in}}%
\pgfusepath{clip}%
\pgfsetbuttcap%
\pgfsetmiterjoin%
\definecolor{currentfill}{rgb}{0.227451,0.192157,0.427451}%
\pgfsetfillcolor{currentfill}%
\pgfsetlinewidth{0.100375pt}%
\definecolor{currentstroke}{rgb}{0.266667,0.266667,0.266667}%
\pgfsetstrokecolor{currentstroke}%
\pgfsetdash{}{0pt}%
\pgfpathmoveto{\pgfqpoint{1.590013in}{0.638889in}}%
\pgfpathlineto{\pgfqpoint{1.615764in}{0.638889in}}%
\pgfpathlineto{\pgfqpoint{1.615764in}{0.757740in}}%
\pgfpathlineto{\pgfqpoint{1.590013in}{0.757740in}}%
\pgfpathlineto{\pgfqpoint{1.590013in}{0.638889in}}%
\pgfpathclose%
\pgfusepath{stroke,fill}%
\end{pgfscope}%
\begin{pgfscope}%
\pgfpathrectangle{\pgfqpoint{0.781250in}{0.638889in}}{\pgfqpoint{4.843750in}{2.172222in}}%
\pgfusepath{clip}%
\pgfsetbuttcap%
\pgfsetmiterjoin%
\definecolor{currentfill}{rgb}{0.164706,0.615686,0.560784}%
\pgfsetfillcolor{currentfill}%
\pgfsetlinewidth{0.100375pt}%
\definecolor{currentstroke}{rgb}{0.266667,0.266667,0.266667}%
\pgfsetstrokecolor{currentstroke}%
\pgfsetdash{}{0pt}%
\pgfpathmoveto{\pgfqpoint{1.626800in}{0.638889in}}%
\pgfpathlineto{\pgfqpoint{1.652551in}{0.638889in}}%
\pgfpathlineto{\pgfqpoint{1.652551in}{0.717089in}}%
\pgfpathlineto{\pgfqpoint{1.626800in}{0.717089in}}%
\pgfpathlineto{\pgfqpoint{1.626800in}{0.638889in}}%
\pgfpathclose%
\pgfusepath{stroke,fill}%
\end{pgfscope}%
\begin{pgfscope}%
\pgfpathrectangle{\pgfqpoint{0.781250in}{0.638889in}}{\pgfqpoint{4.843750in}{2.172222in}}%
\pgfusepath{clip}%
\pgfsetbuttcap%
\pgfsetmiterjoin%
\definecolor{currentfill}{rgb}{0.164706,0.615686,0.560784}%
\pgfsetfillcolor{currentfill}%
\pgfsetlinewidth{0.100375pt}%
\definecolor{currentstroke}{rgb}{0.266667,0.266667,0.266667}%
\pgfsetstrokecolor{currentstroke}%
\pgfsetdash{}{0pt}%
\pgfpathmoveto{\pgfqpoint{1.663587in}{0.638889in}}%
\pgfpathlineto{\pgfqpoint{1.689338in}{0.638889in}}%
\pgfpathlineto{\pgfqpoint{1.689338in}{1.795132in}}%
\pgfpathlineto{\pgfqpoint{1.663587in}{1.795132in}}%
\pgfpathlineto{\pgfqpoint{1.663587in}{0.638889in}}%
\pgfpathclose%
\pgfusepath{stroke,fill}%
\end{pgfscope}%
\begin{pgfscope}%
\pgfpathrectangle{\pgfqpoint{0.781250in}{0.638889in}}{\pgfqpoint{4.843750in}{2.172222in}}%
\pgfusepath{clip}%
\pgfsetbuttcap%
\pgfsetmiterjoin%
\definecolor{currentfill}{rgb}{0.164706,0.615686,0.560784}%
\pgfsetfillcolor{currentfill}%
\pgfsetlinewidth{0.100375pt}%
\definecolor{currentstroke}{rgb}{0.266667,0.266667,0.266667}%
\pgfsetstrokecolor{currentstroke}%
\pgfsetdash{}{0pt}%
\pgfpathmoveto{\pgfqpoint{1.700374in}{0.638889in}}%
\pgfpathlineto{\pgfqpoint{1.726125in}{0.638889in}}%
\pgfpathlineto{\pgfqpoint{1.726125in}{1.066506in}}%
\pgfpathlineto{\pgfqpoint{1.700374in}{1.066506in}}%
\pgfpathlineto{\pgfqpoint{1.700374in}{0.638889in}}%
\pgfpathclose%
\pgfusepath{stroke,fill}%
\end{pgfscope}%
\begin{pgfscope}%
\pgfpathrectangle{\pgfqpoint{0.781250in}{0.638889in}}{\pgfqpoint{4.843750in}{2.172222in}}%
\pgfusepath{clip}%
\pgfsetbuttcap%
\pgfsetmiterjoin%
\definecolor{currentfill}{rgb}{0.164706,0.615686,0.560784}%
\pgfsetfillcolor{currentfill}%
\pgfsetlinewidth{0.100375pt}%
\definecolor{currentstroke}{rgb}{0.266667,0.266667,0.266667}%
\pgfsetstrokecolor{currentstroke}%
\pgfsetdash{}{0pt}%
\pgfpathmoveto{\pgfqpoint{1.737161in}{0.638889in}}%
\pgfpathlineto{\pgfqpoint{1.762912in}{0.638889in}}%
\pgfpathlineto{\pgfqpoint{1.762912in}{0.964722in}}%
\pgfpathlineto{\pgfqpoint{1.737161in}{0.964722in}}%
\pgfpathlineto{\pgfqpoint{1.737161in}{0.638889in}}%
\pgfpathclose%
\pgfusepath{stroke,fill}%
\end{pgfscope}%
\begin{pgfscope}%
\pgfpathrectangle{\pgfqpoint{0.781250in}{0.638889in}}{\pgfqpoint{4.843750in}{2.172222in}}%
\pgfusepath{clip}%
\pgfsetbuttcap%
\pgfsetmiterjoin%
\definecolor{currentfill}{rgb}{0.227451,0.192157,0.427451}%
\pgfsetfillcolor{currentfill}%
\pgfsetlinewidth{0.100375pt}%
\definecolor{currentstroke}{rgb}{0.266667,0.266667,0.266667}%
\pgfsetstrokecolor{currentstroke}%
\pgfsetdash{}{0pt}%
\pgfpathmoveto{\pgfqpoint{1.773948in}{0.638889in}}%
\pgfpathlineto{\pgfqpoint{1.799699in}{0.638889in}}%
\pgfpathlineto{\pgfqpoint{1.799699in}{0.749052in}}%
\pgfpathlineto{\pgfqpoint{1.773948in}{0.749052in}}%
\pgfpathlineto{\pgfqpoint{1.773948in}{0.638889in}}%
\pgfpathclose%
\pgfusepath{stroke,fill}%
\end{pgfscope}%
\begin{pgfscope}%
\pgfpathrectangle{\pgfqpoint{0.781250in}{0.638889in}}{\pgfqpoint{4.843750in}{2.172222in}}%
\pgfusepath{clip}%
\pgfsetbuttcap%
\pgfsetmiterjoin%
\definecolor{currentfill}{rgb}{0.227451,0.192157,0.427451}%
\pgfsetfillcolor{currentfill}%
\pgfsetlinewidth{0.100375pt}%
\definecolor{currentstroke}{rgb}{0.266667,0.266667,0.266667}%
\pgfsetstrokecolor{currentstroke}%
\pgfsetdash{}{0pt}%
\pgfpathmoveto{\pgfqpoint{1.810735in}{0.638889in}}%
\pgfpathlineto{\pgfqpoint{1.836486in}{0.638889in}}%
\pgfpathlineto{\pgfqpoint{1.836486in}{0.671783in}}%
\pgfpathlineto{\pgfqpoint{1.810735in}{0.671783in}}%
\pgfpathlineto{\pgfqpoint{1.810735in}{0.638889in}}%
\pgfpathclose%
\pgfusepath{stroke,fill}%
\end{pgfscope}%
\begin{pgfscope}%
\pgfpathrectangle{\pgfqpoint{0.781250in}{0.638889in}}{\pgfqpoint{4.843750in}{2.172222in}}%
\pgfusepath{clip}%
\pgfsetbuttcap%
\pgfsetmiterjoin%
\definecolor{currentfill}{rgb}{0.227451,0.192157,0.427451}%
\pgfsetfillcolor{currentfill}%
\pgfsetlinewidth{0.100375pt}%
\definecolor{currentstroke}{rgb}{0.266667,0.266667,0.266667}%
\pgfsetstrokecolor{currentstroke}%
\pgfsetdash{}{0pt}%
\pgfpathmoveto{\pgfqpoint{1.847522in}{0.638889in}}%
\pgfpathlineto{\pgfqpoint{1.873273in}{0.638889in}}%
\pgfpathlineto{\pgfqpoint{1.873273in}{0.769222in}}%
\pgfpathlineto{\pgfqpoint{1.847522in}{0.769222in}}%
\pgfpathlineto{\pgfqpoint{1.847522in}{0.638889in}}%
\pgfpathclose%
\pgfusepath{stroke,fill}%
\end{pgfscope}%
\begin{pgfscope}%
\pgfpathrectangle{\pgfqpoint{0.781250in}{0.638889in}}{\pgfqpoint{4.843750in}{2.172222in}}%
\pgfusepath{clip}%
\pgfsetbuttcap%
\pgfsetmiterjoin%
\definecolor{currentfill}{rgb}{0.227451,0.192157,0.427451}%
\pgfsetfillcolor{currentfill}%
\pgfsetlinewidth{0.100375pt}%
\definecolor{currentstroke}{rgb}{0.266667,0.266667,0.266667}%
\pgfsetstrokecolor{currentstroke}%
\pgfsetdash{}{0pt}%
\pgfpathmoveto{\pgfqpoint{1.884309in}{0.638889in}}%
\pgfpathlineto{\pgfqpoint{1.910060in}{0.638889in}}%
\pgfpathlineto{\pgfqpoint{1.910060in}{0.746259in}}%
\pgfpathlineto{\pgfqpoint{1.884309in}{0.746259in}}%
\pgfpathlineto{\pgfqpoint{1.884309in}{0.638889in}}%
\pgfpathclose%
\pgfusepath{stroke,fill}%
\end{pgfscope}%
\begin{pgfscope}%
\pgfpathrectangle{\pgfqpoint{0.781250in}{0.638889in}}{\pgfqpoint{4.843750in}{2.172222in}}%
\pgfusepath{clip}%
\pgfsetbuttcap%
\pgfsetmiterjoin%
\definecolor{currentfill}{rgb}{0.227451,0.192157,0.427451}%
\pgfsetfillcolor{currentfill}%
\pgfsetlinewidth{0.100375pt}%
\definecolor{currentstroke}{rgb}{0.266667,0.266667,0.266667}%
\pgfsetstrokecolor{currentstroke}%
\pgfsetdash{}{0pt}%
\pgfpathmoveto{\pgfqpoint{1.921097in}{0.638889in}}%
\pgfpathlineto{\pgfqpoint{1.946847in}{0.638889in}}%
\pgfpathlineto{\pgfqpoint{1.946847in}{0.736639in}}%
\pgfpathlineto{\pgfqpoint{1.921097in}{0.736639in}}%
\pgfpathlineto{\pgfqpoint{1.921097in}{0.638889in}}%
\pgfpathclose%
\pgfusepath{stroke,fill}%
\end{pgfscope}%
\begin{pgfscope}%
\pgfpathrectangle{\pgfqpoint{0.781250in}{0.638889in}}{\pgfqpoint{4.843750in}{2.172222in}}%
\pgfusepath{clip}%
\pgfsetbuttcap%
\pgfsetmiterjoin%
\definecolor{currentfill}{rgb}{0.227451,0.192157,0.427451}%
\pgfsetfillcolor{currentfill}%
\pgfsetlinewidth{0.100375pt}%
\definecolor{currentstroke}{rgb}{0.266667,0.266667,0.266667}%
\pgfsetstrokecolor{currentstroke}%
\pgfsetdash{}{0pt}%
\pgfpathmoveto{\pgfqpoint{1.957884in}{0.638889in}}%
\pgfpathlineto{\pgfqpoint{1.983635in}{0.638889in}}%
\pgfpathlineto{\pgfqpoint{1.983635in}{0.944552in}}%
\pgfpathlineto{\pgfqpoint{1.957884in}{0.944552in}}%
\pgfpathlineto{\pgfqpoint{1.957884in}{0.638889in}}%
\pgfpathclose%
\pgfusepath{stroke,fill}%
\end{pgfscope}%
\begin{pgfscope}%
\pgfpathrectangle{\pgfqpoint{0.781250in}{0.638889in}}{\pgfqpoint{4.843750in}{2.172222in}}%
\pgfusepath{clip}%
\pgfsetbuttcap%
\pgfsetmiterjoin%
\definecolor{currentfill}{rgb}{0.227451,0.192157,0.427451}%
\pgfsetfillcolor{currentfill}%
\pgfsetlinewidth{0.100375pt}%
\definecolor{currentstroke}{rgb}{0.266667,0.266667,0.266667}%
\pgfsetstrokecolor{currentstroke}%
\pgfsetdash{}{0pt}%
\pgfpathmoveto{\pgfqpoint{1.994671in}{0.638889in}}%
\pgfpathlineto{\pgfqpoint{2.020422in}{0.638889in}}%
\pgfpathlineto{\pgfqpoint{2.020422in}{1.025544in}}%
\pgfpathlineto{\pgfqpoint{1.994671in}{1.025544in}}%
\pgfpathlineto{\pgfqpoint{1.994671in}{0.638889in}}%
\pgfpathclose%
\pgfusepath{stroke,fill}%
\end{pgfscope}%
\begin{pgfscope}%
\pgfpathrectangle{\pgfqpoint{0.781250in}{0.638889in}}{\pgfqpoint{4.843750in}{2.172222in}}%
\pgfusepath{clip}%
\pgfsetbuttcap%
\pgfsetmiterjoin%
\definecolor{currentfill}{rgb}{0.227451,0.192157,0.427451}%
\pgfsetfillcolor{currentfill}%
\pgfsetlinewidth{0.100375pt}%
\definecolor{currentstroke}{rgb}{0.266667,0.266667,0.266667}%
\pgfsetstrokecolor{currentstroke}%
\pgfsetdash{}{0pt}%
\pgfpathmoveto{\pgfqpoint{2.031458in}{0.638889in}}%
\pgfpathlineto{\pgfqpoint{2.057209in}{0.638889in}}%
\pgfpathlineto{\pgfqpoint{2.057209in}{0.987996in}}%
\pgfpathlineto{\pgfqpoint{2.031458in}{0.987996in}}%
\pgfpathlineto{\pgfqpoint{2.031458in}{0.638889in}}%
\pgfpathclose%
\pgfusepath{stroke,fill}%
\end{pgfscope}%
\begin{pgfscope}%
\pgfpathrectangle{\pgfqpoint{0.781250in}{0.638889in}}{\pgfqpoint{4.843750in}{2.172222in}}%
\pgfusepath{clip}%
\pgfsetbuttcap%
\pgfsetmiterjoin%
\definecolor{currentfill}{rgb}{0.164706,0.615686,0.560784}%
\pgfsetfillcolor{currentfill}%
\pgfsetlinewidth{0.100375pt}%
\definecolor{currentstroke}{rgb}{0.266667,0.266667,0.266667}%
\pgfsetstrokecolor{currentstroke}%
\pgfsetdash{}{0pt}%
\pgfpathmoveto{\pgfqpoint{2.068245in}{0.638889in}}%
\pgfpathlineto{\pgfqpoint{2.093996in}{0.638889in}}%
\pgfpathlineto{\pgfqpoint{2.093996in}{0.964722in}}%
\pgfpathlineto{\pgfqpoint{2.068245in}{0.964722in}}%
\pgfpathlineto{\pgfqpoint{2.068245in}{0.638889in}}%
\pgfpathclose%
\pgfusepath{stroke,fill}%
\end{pgfscope}%
\begin{pgfscope}%
\pgfpathrectangle{\pgfqpoint{0.781250in}{0.638889in}}{\pgfqpoint{4.843750in}{2.172222in}}%
\pgfusepath{clip}%
\pgfsetbuttcap%
\pgfsetmiterjoin%
\definecolor{currentfill}{rgb}{0.164706,0.615686,0.560784}%
\pgfsetfillcolor{currentfill}%
\pgfsetlinewidth{0.100375pt}%
\definecolor{currentstroke}{rgb}{0.266667,0.266667,0.266667}%
\pgfsetstrokecolor{currentstroke}%
\pgfsetdash{}{0pt}%
\pgfpathmoveto{\pgfqpoint{2.105032in}{0.638889in}}%
\pgfpathlineto{\pgfqpoint{2.130783in}{0.638889in}}%
\pgfpathlineto{\pgfqpoint{2.130783in}{1.440129in}}%
\pgfpathlineto{\pgfqpoint{2.105032in}{1.440129in}}%
\pgfpathlineto{\pgfqpoint{2.105032in}{0.638889in}}%
\pgfpathclose%
\pgfusepath{stroke,fill}%
\end{pgfscope}%
\begin{pgfscope}%
\pgfpathrectangle{\pgfqpoint{0.781250in}{0.638889in}}{\pgfqpoint{4.843750in}{2.172222in}}%
\pgfusepath{clip}%
\pgfsetbuttcap%
\pgfsetmiterjoin%
\definecolor{currentfill}{rgb}{0.164706,0.615686,0.560784}%
\pgfsetfillcolor{currentfill}%
\pgfsetlinewidth{0.100375pt}%
\definecolor{currentstroke}{rgb}{0.266667,0.266667,0.266667}%
\pgfsetstrokecolor{currentstroke}%
\pgfsetdash{}{0pt}%
\pgfpathmoveto{\pgfqpoint{2.141819in}{0.638889in}}%
\pgfpathlineto{\pgfqpoint{2.167570in}{0.638889in}}%
\pgfpathlineto{\pgfqpoint{2.167570in}{1.568290in}}%
\pgfpathlineto{\pgfqpoint{2.141819in}{1.568290in}}%
\pgfpathlineto{\pgfqpoint{2.141819in}{0.638889in}}%
\pgfpathclose%
\pgfusepath{stroke,fill}%
\end{pgfscope}%
\begin{pgfscope}%
\pgfpathrectangle{\pgfqpoint{0.781250in}{0.638889in}}{\pgfqpoint{4.843750in}{2.172222in}}%
\pgfusepath{clip}%
\pgfsetbuttcap%
\pgfsetmiterjoin%
\definecolor{currentfill}{rgb}{0.164706,0.615686,0.560784}%
\pgfsetfillcolor{currentfill}%
\pgfsetlinewidth{0.100375pt}%
\definecolor{currentstroke}{rgb}{0.266667,0.266667,0.266667}%
\pgfsetstrokecolor{currentstroke}%
\pgfsetdash{}{0pt}%
\pgfpathmoveto{\pgfqpoint{2.178606in}{0.638889in}}%
\pgfpathlineto{\pgfqpoint{2.204357in}{0.638889in}}%
\pgfpathlineto{\pgfqpoint{2.204357in}{0.846802in}}%
\pgfpathlineto{\pgfqpoint{2.178606in}{0.846802in}}%
\pgfpathlineto{\pgfqpoint{2.178606in}{0.638889in}}%
\pgfpathclose%
\pgfusepath{stroke,fill}%
\end{pgfscope}%
\begin{pgfscope}%
\pgfpathrectangle{\pgfqpoint{0.781250in}{0.638889in}}{\pgfqpoint{4.843750in}{2.172222in}}%
\pgfusepath{clip}%
\pgfsetbuttcap%
\pgfsetmiterjoin%
\definecolor{currentfill}{rgb}{0.227451,0.192157,0.427451}%
\pgfsetfillcolor{currentfill}%
\pgfsetlinewidth{0.100375pt}%
\definecolor{currentstroke}{rgb}{0.266667,0.266667,0.266667}%
\pgfsetstrokecolor{currentstroke}%
\pgfsetdash{}{0pt}%
\pgfpathmoveto{\pgfqpoint{2.215393in}{0.638889in}}%
\pgfpathlineto{\pgfqpoint{2.241144in}{0.638889in}}%
\pgfpathlineto{\pgfqpoint{2.241144in}{1.140052in}}%
\pgfpathlineto{\pgfqpoint{2.215393in}{1.140052in}}%
\pgfpathlineto{\pgfqpoint{2.215393in}{0.638889in}}%
\pgfpathclose%
\pgfusepath{stroke,fill}%
\end{pgfscope}%
\begin{pgfscope}%
\pgfpathrectangle{\pgfqpoint{0.781250in}{0.638889in}}{\pgfqpoint{4.843750in}{2.172222in}}%
\pgfusepath{clip}%
\pgfsetbuttcap%
\pgfsetmiterjoin%
\definecolor{currentfill}{rgb}{0.227451,0.192157,0.427451}%
\pgfsetfillcolor{currentfill}%
\pgfsetlinewidth{0.100375pt}%
\definecolor{currentstroke}{rgb}{0.266667,0.266667,0.266667}%
\pgfsetstrokecolor{currentstroke}%
\pgfsetdash{}{0pt}%
\pgfpathmoveto{\pgfqpoint{2.252180in}{0.638889in}}%
\pgfpathlineto{\pgfqpoint{2.277931in}{0.638889in}}%
\pgfpathlineto{\pgfqpoint{2.277931in}{0.762085in}}%
\pgfpathlineto{\pgfqpoint{2.252180in}{0.762085in}}%
\pgfpathlineto{\pgfqpoint{2.252180in}{0.638889in}}%
\pgfpathclose%
\pgfusepath{stroke,fill}%
\end{pgfscope}%
\begin{pgfscope}%
\pgfpathrectangle{\pgfqpoint{0.781250in}{0.638889in}}{\pgfqpoint{4.843750in}{2.172222in}}%
\pgfusepath{clip}%
\pgfsetbuttcap%
\pgfsetmiterjoin%
\definecolor{currentfill}{rgb}{0.227451,0.192157,0.427451}%
\pgfsetfillcolor{currentfill}%
\pgfsetlinewidth{0.100375pt}%
\definecolor{currentstroke}{rgb}{0.266667,0.266667,0.266667}%
\pgfsetstrokecolor{currentstroke}%
\pgfsetdash{}{0pt}%
\pgfpathmoveto{\pgfqpoint{2.288967in}{0.638889in}}%
\pgfpathlineto{\pgfqpoint{2.314718in}{0.638889in}}%
\pgfpathlineto{\pgfqpoint{2.314718in}{0.720813in}}%
\pgfpathlineto{\pgfqpoint{2.288967in}{0.720813in}}%
\pgfpathlineto{\pgfqpoint{2.288967in}{0.638889in}}%
\pgfpathclose%
\pgfusepath{stroke,fill}%
\end{pgfscope}%
\begin{pgfscope}%
\pgfpathrectangle{\pgfqpoint{0.781250in}{0.638889in}}{\pgfqpoint{4.843750in}{2.172222in}}%
\pgfusepath{clip}%
\pgfsetbuttcap%
\pgfsetmiterjoin%
\definecolor{currentfill}{rgb}{0.227451,0.192157,0.427451}%
\pgfsetfillcolor{currentfill}%
\pgfsetlinewidth{0.100375pt}%
\definecolor{currentstroke}{rgb}{0.266667,0.266667,0.266667}%
\pgfsetstrokecolor{currentstroke}%
\pgfsetdash{}{0pt}%
\pgfpathmoveto{\pgfqpoint{2.325754in}{0.638889in}}%
\pgfpathlineto{\pgfqpoint{2.351505in}{0.638889in}}%
\pgfpathlineto{\pgfqpoint{2.351505in}{0.666197in}}%
\pgfpathlineto{\pgfqpoint{2.325754in}{0.666197in}}%
\pgfpathlineto{\pgfqpoint{2.325754in}{0.638889in}}%
\pgfpathclose%
\pgfusepath{stroke,fill}%
\end{pgfscope}%
\begin{pgfscope}%
\pgfpathrectangle{\pgfqpoint{0.781250in}{0.638889in}}{\pgfqpoint{4.843750in}{2.172222in}}%
\pgfusepath{clip}%
\pgfsetbuttcap%
\pgfsetmiterjoin%
\definecolor{currentfill}{rgb}{0.227451,0.192157,0.427451}%
\pgfsetfillcolor{currentfill}%
\pgfsetlinewidth{0.100375pt}%
\definecolor{currentstroke}{rgb}{0.266667,0.266667,0.266667}%
\pgfsetstrokecolor{currentstroke}%
\pgfsetdash{}{0pt}%
\pgfpathmoveto{\pgfqpoint{2.362541in}{0.638889in}}%
\pgfpathlineto{\pgfqpoint{2.388292in}{0.638889in}}%
\pgfpathlineto{\pgfqpoint{2.388292in}{0.716779in}}%
\pgfpathlineto{\pgfqpoint{2.362541in}{0.716779in}}%
\pgfpathlineto{\pgfqpoint{2.362541in}{0.638889in}}%
\pgfpathclose%
\pgfusepath{stroke,fill}%
\end{pgfscope}%
\begin{pgfscope}%
\pgfpathrectangle{\pgfqpoint{0.781250in}{0.638889in}}{\pgfqpoint{4.843750in}{2.172222in}}%
\pgfusepath{clip}%
\pgfsetbuttcap%
\pgfsetmiterjoin%
\definecolor{currentfill}{rgb}{0.227451,0.192157,0.427451}%
\pgfsetfillcolor{currentfill}%
\pgfsetlinewidth{0.100375pt}%
\definecolor{currentstroke}{rgb}{0.266667,0.266667,0.266667}%
\pgfsetstrokecolor{currentstroke}%
\pgfsetdash{}{0pt}%
\pgfpathmoveto{\pgfqpoint{2.399328in}{0.638889in}}%
\pgfpathlineto{\pgfqpoint{2.425079in}{0.638889in}}%
\pgfpathlineto{\pgfqpoint{2.425079in}{0.733225in}}%
\pgfpathlineto{\pgfqpoint{2.399328in}{0.733225in}}%
\pgfpathlineto{\pgfqpoint{2.399328in}{0.638889in}}%
\pgfpathclose%
\pgfusepath{stroke,fill}%
\end{pgfscope}%
\begin{pgfscope}%
\pgfpathrectangle{\pgfqpoint{0.781250in}{0.638889in}}{\pgfqpoint{4.843750in}{2.172222in}}%
\pgfusepath{clip}%
\pgfsetbuttcap%
\pgfsetmiterjoin%
\definecolor{currentfill}{rgb}{0.227451,0.192157,0.427451}%
\pgfsetfillcolor{currentfill}%
\pgfsetlinewidth{0.100375pt}%
\definecolor{currentstroke}{rgb}{0.266667,0.266667,0.266667}%
\pgfsetstrokecolor{currentstroke}%
\pgfsetdash{}{0pt}%
\pgfpathmoveto{\pgfqpoint{2.436115in}{0.638889in}}%
\pgfpathlineto{\pgfqpoint{2.461866in}{0.638889in}}%
\pgfpathlineto{\pgfqpoint{2.461866in}{0.963171in}}%
\pgfpathlineto{\pgfqpoint{2.436115in}{0.963171in}}%
\pgfpathlineto{\pgfqpoint{2.436115in}{0.638889in}}%
\pgfpathclose%
\pgfusepath{stroke,fill}%
\end{pgfscope}%
\begin{pgfscope}%
\pgfpathrectangle{\pgfqpoint{0.781250in}{0.638889in}}{\pgfqpoint{4.843750in}{2.172222in}}%
\pgfusepath{clip}%
\pgfsetbuttcap%
\pgfsetmiterjoin%
\definecolor{currentfill}{rgb}{0.227451,0.192157,0.427451}%
\pgfsetfillcolor{currentfill}%
\pgfsetlinewidth{0.100375pt}%
\definecolor{currentstroke}{rgb}{0.266667,0.266667,0.266667}%
\pgfsetstrokecolor{currentstroke}%
\pgfsetdash{}{0pt}%
\pgfpathmoveto{\pgfqpoint{2.472902in}{0.638889in}}%
\pgfpathlineto{\pgfqpoint{2.498653in}{0.638889in}}%
\pgfpathlineto{\pgfqpoint{2.498653in}{0.827252in}}%
\pgfpathlineto{\pgfqpoint{2.472902in}{0.827252in}}%
\pgfpathlineto{\pgfqpoint{2.472902in}{0.638889in}}%
\pgfpathclose%
\pgfusepath{stroke,fill}%
\end{pgfscope}%
\begin{pgfscope}%
\pgfpathrectangle{\pgfqpoint{0.781250in}{0.638889in}}{\pgfqpoint{4.843750in}{2.172222in}}%
\pgfusepath{clip}%
\pgfsetbuttcap%
\pgfsetmiterjoin%
\definecolor{currentfill}{rgb}{0.164706,0.615686,0.560784}%
\pgfsetfillcolor{currentfill}%
\pgfsetlinewidth{0.100375pt}%
\definecolor{currentstroke}{rgb}{0.266667,0.266667,0.266667}%
\pgfsetstrokecolor{currentstroke}%
\pgfsetdash{}{0pt}%
\pgfpathmoveto{\pgfqpoint{2.509689in}{0.638889in}}%
\pgfpathlineto{\pgfqpoint{2.535440in}{0.638889in}}%
\pgfpathlineto{\pgfqpoint{2.535440in}{0.915071in}}%
\pgfpathlineto{\pgfqpoint{2.509689in}{0.915071in}}%
\pgfpathlineto{\pgfqpoint{2.509689in}{0.638889in}}%
\pgfpathclose%
\pgfusepath{stroke,fill}%
\end{pgfscope}%
\begin{pgfscope}%
\pgfpathrectangle{\pgfqpoint{0.781250in}{0.638889in}}{\pgfqpoint{4.843750in}{2.172222in}}%
\pgfusepath{clip}%
\pgfsetbuttcap%
\pgfsetmiterjoin%
\definecolor{currentfill}{rgb}{0.164706,0.615686,0.560784}%
\pgfsetfillcolor{currentfill}%
\pgfsetlinewidth{0.100375pt}%
\definecolor{currentstroke}{rgb}{0.266667,0.266667,0.266667}%
\pgfsetstrokecolor{currentstroke}%
\pgfsetdash{}{0pt}%
\pgfpathmoveto{\pgfqpoint{2.546476in}{0.638889in}}%
\pgfpathlineto{\pgfqpoint{2.572227in}{0.638889in}}%
\pgfpathlineto{\pgfqpoint{2.572227in}{1.338655in}}%
\pgfpathlineto{\pgfqpoint{2.546476in}{1.338655in}}%
\pgfpathlineto{\pgfqpoint{2.546476in}{0.638889in}}%
\pgfpathclose%
\pgfusepath{stroke,fill}%
\end{pgfscope}%
\begin{pgfscope}%
\pgfpathrectangle{\pgfqpoint{0.781250in}{0.638889in}}{\pgfqpoint{4.843750in}{2.172222in}}%
\pgfusepath{clip}%
\pgfsetbuttcap%
\pgfsetmiterjoin%
\definecolor{currentfill}{rgb}{0.164706,0.615686,0.560784}%
\pgfsetfillcolor{currentfill}%
\pgfsetlinewidth{0.100375pt}%
\definecolor{currentstroke}{rgb}{0.266667,0.266667,0.266667}%
\pgfsetstrokecolor{currentstroke}%
\pgfsetdash{}{0pt}%
\pgfpathmoveto{\pgfqpoint{2.583263in}{0.638889in}}%
\pgfpathlineto{\pgfqpoint{2.609014in}{0.638889in}}%
\pgfpathlineto{\pgfqpoint{2.609014in}{1.034233in}}%
\pgfpathlineto{\pgfqpoint{2.583263in}{1.034233in}}%
\pgfpathlineto{\pgfqpoint{2.583263in}{0.638889in}}%
\pgfpathclose%
\pgfusepath{stroke,fill}%
\end{pgfscope}%
\begin{pgfscope}%
\pgfpathrectangle{\pgfqpoint{0.781250in}{0.638889in}}{\pgfqpoint{4.843750in}{2.172222in}}%
\pgfusepath{clip}%
\pgfsetbuttcap%
\pgfsetmiterjoin%
\definecolor{currentfill}{rgb}{0.164706,0.615686,0.560784}%
\pgfsetfillcolor{currentfill}%
\pgfsetlinewidth{0.100375pt}%
\definecolor{currentstroke}{rgb}{0.266667,0.266667,0.266667}%
\pgfsetstrokecolor{currentstroke}%
\pgfsetdash{}{0pt}%
\pgfpathmoveto{\pgfqpoint{2.620050in}{0.638889in}}%
\pgfpathlineto{\pgfqpoint{2.645801in}{0.638889in}}%
\pgfpathlineto{\pgfqpoint{2.645801in}{1.392960in}}%
\pgfpathlineto{\pgfqpoint{2.620050in}{1.392960in}}%
\pgfpathlineto{\pgfqpoint{2.620050in}{0.638889in}}%
\pgfpathclose%
\pgfusepath{stroke,fill}%
\end{pgfscope}%
\begin{pgfscope}%
\pgfpathrectangle{\pgfqpoint{0.781250in}{0.638889in}}{\pgfqpoint{4.843750in}{2.172222in}}%
\pgfusepath{clip}%
\pgfsetbuttcap%
\pgfsetmiterjoin%
\definecolor{currentfill}{rgb}{0.227451,0.192157,0.427451}%
\pgfsetfillcolor{currentfill}%
\pgfsetlinewidth{0.100375pt}%
\definecolor{currentstroke}{rgb}{0.266667,0.266667,0.266667}%
\pgfsetstrokecolor{currentstroke}%
\pgfsetdash{}{0pt}%
\pgfpathmoveto{\pgfqpoint{2.656837in}{0.638889in}}%
\pgfpathlineto{\pgfqpoint{2.682588in}{0.638889in}}%
\pgfpathlineto{\pgfqpoint{2.682588in}{0.907624in}}%
\pgfpathlineto{\pgfqpoint{2.656837in}{0.907624in}}%
\pgfpathlineto{\pgfqpoint{2.656837in}{0.638889in}}%
\pgfpathclose%
\pgfusepath{stroke,fill}%
\end{pgfscope}%
\begin{pgfscope}%
\pgfpathrectangle{\pgfqpoint{0.781250in}{0.638889in}}{\pgfqpoint{4.843750in}{2.172222in}}%
\pgfusepath{clip}%
\pgfsetbuttcap%
\pgfsetmiterjoin%
\definecolor{currentfill}{rgb}{0.227451,0.192157,0.427451}%
\pgfsetfillcolor{currentfill}%
\pgfsetlinewidth{0.100375pt}%
\definecolor{currentstroke}{rgb}{0.266667,0.266667,0.266667}%
\pgfsetstrokecolor{currentstroke}%
\pgfsetdash{}{0pt}%
\pgfpathmoveto{\pgfqpoint{2.693624in}{0.638889in}}%
\pgfpathlineto{\pgfqpoint{2.719375in}{0.638889in}}%
\pgfpathlineto{\pgfqpoint{2.719375in}{0.811425in}}%
\pgfpathlineto{\pgfqpoint{2.693624in}{0.811425in}}%
\pgfpathlineto{\pgfqpoint{2.693624in}{0.638889in}}%
\pgfpathclose%
\pgfusepath{stroke,fill}%
\end{pgfscope}%
\begin{pgfscope}%
\pgfpathrectangle{\pgfqpoint{0.781250in}{0.638889in}}{\pgfqpoint{4.843750in}{2.172222in}}%
\pgfusepath{clip}%
\pgfsetbuttcap%
\pgfsetmiterjoin%
\definecolor{currentfill}{rgb}{0.227451,0.192157,0.427451}%
\pgfsetfillcolor{currentfill}%
\pgfsetlinewidth{0.100375pt}%
\definecolor{currentstroke}{rgb}{0.266667,0.266667,0.266667}%
\pgfsetstrokecolor{currentstroke}%
\pgfsetdash{}{0pt}%
\pgfpathmoveto{\pgfqpoint{2.730411in}{0.638889in}}%
\pgfpathlineto{\pgfqpoint{2.756162in}{0.638889in}}%
\pgfpathlineto{\pgfqpoint{2.756162in}{0.734777in}}%
\pgfpathlineto{\pgfqpoint{2.730411in}{0.734777in}}%
\pgfpathlineto{\pgfqpoint{2.730411in}{0.638889in}}%
\pgfpathclose%
\pgfusepath{stroke,fill}%
\end{pgfscope}%
\begin{pgfscope}%
\pgfpathrectangle{\pgfqpoint{0.781250in}{0.638889in}}{\pgfqpoint{4.843750in}{2.172222in}}%
\pgfusepath{clip}%
\pgfsetbuttcap%
\pgfsetmiterjoin%
\definecolor{currentfill}{rgb}{0.227451,0.192157,0.427451}%
\pgfsetfillcolor{currentfill}%
\pgfsetlinewidth{0.100375pt}%
\definecolor{currentstroke}{rgb}{0.266667,0.266667,0.266667}%
\pgfsetstrokecolor{currentstroke}%
\pgfsetdash{}{0pt}%
\pgfpathmoveto{\pgfqpoint{2.767199in}{0.638889in}}%
\pgfpathlineto{\pgfqpoint{2.792949in}{0.638889in}}%
\pgfpathlineto{\pgfqpoint{2.792949in}{0.884350in}}%
\pgfpathlineto{\pgfqpoint{2.767199in}{0.884350in}}%
\pgfpathlineto{\pgfqpoint{2.767199in}{0.638889in}}%
\pgfpathclose%
\pgfusepath{stroke,fill}%
\end{pgfscope}%
\begin{pgfscope}%
\pgfpathrectangle{\pgfqpoint{0.781250in}{0.638889in}}{\pgfqpoint{4.843750in}{2.172222in}}%
\pgfusepath{clip}%
\pgfsetbuttcap%
\pgfsetmiterjoin%
\definecolor{currentfill}{rgb}{0.227451,0.192157,0.427451}%
\pgfsetfillcolor{currentfill}%
\pgfsetlinewidth{0.100375pt}%
\definecolor{currentstroke}{rgb}{0.266667,0.266667,0.266667}%
\pgfsetstrokecolor{currentstroke}%
\pgfsetdash{}{0pt}%
\pgfpathmoveto{\pgfqpoint{2.803986in}{0.638889in}}%
\pgfpathlineto{\pgfqpoint{2.829737in}{0.638889in}}%
\pgfpathlineto{\pgfqpoint{2.829737in}{0.821976in}}%
\pgfpathlineto{\pgfqpoint{2.803986in}{0.821976in}}%
\pgfpathlineto{\pgfqpoint{2.803986in}{0.638889in}}%
\pgfpathclose%
\pgfusepath{stroke,fill}%
\end{pgfscope}%
\begin{pgfscope}%
\pgfpathrectangle{\pgfqpoint{0.781250in}{0.638889in}}{\pgfqpoint{4.843750in}{2.172222in}}%
\pgfusepath{clip}%
\pgfsetbuttcap%
\pgfsetmiterjoin%
\definecolor{currentfill}{rgb}{0.227451,0.192157,0.427451}%
\pgfsetfillcolor{currentfill}%
\pgfsetlinewidth{0.100375pt}%
\definecolor{currentstroke}{rgb}{0.266667,0.266667,0.266667}%
\pgfsetstrokecolor{currentstroke}%
\pgfsetdash{}{0pt}%
\pgfpathmoveto{\pgfqpoint{2.840773in}{0.638889in}}%
\pgfpathlineto{\pgfqpoint{2.866524in}{0.638889in}}%
\pgfpathlineto{\pgfqpoint{2.866524in}{0.707469in}}%
\pgfpathlineto{\pgfqpoint{2.840773in}{0.707469in}}%
\pgfpathlineto{\pgfqpoint{2.840773in}{0.638889in}}%
\pgfpathclose%
\pgfusepath{stroke,fill}%
\end{pgfscope}%
\begin{pgfscope}%
\pgfpathrectangle{\pgfqpoint{0.781250in}{0.638889in}}{\pgfqpoint{4.843750in}{2.172222in}}%
\pgfusepath{clip}%
\pgfsetbuttcap%
\pgfsetmiterjoin%
\definecolor{currentfill}{rgb}{0.227451,0.192157,0.427451}%
\pgfsetfillcolor{currentfill}%
\pgfsetlinewidth{0.100375pt}%
\definecolor{currentstroke}{rgb}{0.266667,0.266667,0.266667}%
\pgfsetstrokecolor{currentstroke}%
\pgfsetdash{}{0pt}%
\pgfpathmoveto{\pgfqpoint{2.877560in}{0.638889in}}%
\pgfpathlineto{\pgfqpoint{2.903311in}{0.638889in}}%
\pgfpathlineto{\pgfqpoint{2.903311in}{0.707779in}}%
\pgfpathlineto{\pgfqpoint{2.877560in}{0.707779in}}%
\pgfpathlineto{\pgfqpoint{2.877560in}{0.638889in}}%
\pgfpathclose%
\pgfusepath{stroke,fill}%
\end{pgfscope}%
\begin{pgfscope}%
\pgfpathrectangle{\pgfqpoint{0.781250in}{0.638889in}}{\pgfqpoint{4.843750in}{2.172222in}}%
\pgfusepath{clip}%
\pgfsetbuttcap%
\pgfsetmiterjoin%
\definecolor{currentfill}{rgb}{0.227451,0.192157,0.427451}%
\pgfsetfillcolor{currentfill}%
\pgfsetlinewidth{0.100375pt}%
\definecolor{currentstroke}{rgb}{0.266667,0.266667,0.266667}%
\pgfsetstrokecolor{currentstroke}%
\pgfsetdash{}{0pt}%
\pgfpathmoveto{\pgfqpoint{2.914347in}{0.638889in}}%
\pgfpathlineto{\pgfqpoint{2.940098in}{0.638889in}}%
\pgfpathlineto{\pgfqpoint{2.940098in}{0.986134in}}%
\pgfpathlineto{\pgfqpoint{2.914347in}{0.986134in}}%
\pgfpathlineto{\pgfqpoint{2.914347in}{0.638889in}}%
\pgfpathclose%
\pgfusepath{stroke,fill}%
\end{pgfscope}%
\begin{pgfscope}%
\pgfpathrectangle{\pgfqpoint{0.781250in}{0.638889in}}{\pgfqpoint{4.843750in}{2.172222in}}%
\pgfusepath{clip}%
\pgfsetbuttcap%
\pgfsetmiterjoin%
\definecolor{currentfill}{rgb}{0.164706,0.615686,0.560784}%
\pgfsetfillcolor{currentfill}%
\pgfsetlinewidth{0.100375pt}%
\definecolor{currentstroke}{rgb}{0.266667,0.266667,0.266667}%
\pgfsetstrokecolor{currentstroke}%
\pgfsetdash{}{0pt}%
\pgfpathmoveto{\pgfqpoint{2.951134in}{0.638889in}}%
\pgfpathlineto{\pgfqpoint{2.976885in}{0.638889in}}%
\pgfpathlineto{\pgfqpoint{2.976885in}{1.127949in}}%
\pgfpathlineto{\pgfqpoint{2.951134in}{1.127949in}}%
\pgfpathlineto{\pgfqpoint{2.951134in}{0.638889in}}%
\pgfpathclose%
\pgfusepath{stroke,fill}%
\end{pgfscope}%
\begin{pgfscope}%
\pgfpathrectangle{\pgfqpoint{0.781250in}{0.638889in}}{\pgfqpoint{4.843750in}{2.172222in}}%
\pgfusepath{clip}%
\pgfsetbuttcap%
\pgfsetmiterjoin%
\definecolor{currentfill}{rgb}{0.164706,0.615686,0.560784}%
\pgfsetfillcolor{currentfill}%
\pgfsetlinewidth{0.100375pt}%
\definecolor{currentstroke}{rgb}{0.266667,0.266667,0.266667}%
\pgfsetstrokecolor{currentstroke}%
\pgfsetdash{}{0pt}%
\pgfpathmoveto{\pgfqpoint{2.987921in}{0.638889in}}%
\pgfpathlineto{\pgfqpoint{3.013672in}{0.638889in}}%
\pgfpathlineto{\pgfqpoint{3.013672in}{2.220887in}}%
\pgfpathlineto{\pgfqpoint{2.987921in}{2.220887in}}%
\pgfpathlineto{\pgfqpoint{2.987921in}{0.638889in}}%
\pgfpathclose%
\pgfusepath{stroke,fill}%
\end{pgfscope}%
\begin{pgfscope}%
\pgfpathrectangle{\pgfqpoint{0.781250in}{0.638889in}}{\pgfqpoint{4.843750in}{2.172222in}}%
\pgfusepath{clip}%
\pgfsetbuttcap%
\pgfsetmiterjoin%
\definecolor{currentfill}{rgb}{0.164706,0.615686,0.560784}%
\pgfsetfillcolor{currentfill}%
\pgfsetlinewidth{0.100375pt}%
\definecolor{currentstroke}{rgb}{0.266667,0.266667,0.266667}%
\pgfsetstrokecolor{currentstroke}%
\pgfsetdash{}{0pt}%
\pgfpathmoveto{\pgfqpoint{3.024708in}{0.638889in}}%
\pgfpathlineto{\pgfqpoint{3.050459in}{0.638889in}}%
\pgfpathlineto{\pgfqpoint{3.050459in}{1.947187in}}%
\pgfpathlineto{\pgfqpoint{3.024708in}{1.947187in}}%
\pgfpathlineto{\pgfqpoint{3.024708in}{0.638889in}}%
\pgfpathclose%
\pgfusepath{stroke,fill}%
\end{pgfscope}%
\begin{pgfscope}%
\pgfpathrectangle{\pgfqpoint{0.781250in}{0.638889in}}{\pgfqpoint{4.843750in}{2.172222in}}%
\pgfusepath{clip}%
\pgfsetbuttcap%
\pgfsetmiterjoin%
\definecolor{currentfill}{rgb}{0.164706,0.615686,0.560784}%
\pgfsetfillcolor{currentfill}%
\pgfsetlinewidth{0.100375pt}%
\definecolor{currentstroke}{rgb}{0.266667,0.266667,0.266667}%
\pgfsetstrokecolor{currentstroke}%
\pgfsetdash{}{0pt}%
\pgfpathmoveto{\pgfqpoint{3.061495in}{0.638889in}}%
\pgfpathlineto{\pgfqpoint{3.087246in}{0.638889in}}%
\pgfpathlineto{\pgfqpoint{3.087246in}{1.334931in}}%
\pgfpathlineto{\pgfqpoint{3.061495in}{1.334931in}}%
\pgfpathlineto{\pgfqpoint{3.061495in}{0.638889in}}%
\pgfpathclose%
\pgfusepath{stroke,fill}%
\end{pgfscope}%
\begin{pgfscope}%
\pgfpathrectangle{\pgfqpoint{0.781250in}{0.638889in}}{\pgfqpoint{4.843750in}{2.172222in}}%
\pgfusepath{clip}%
\pgfsetbuttcap%
\pgfsetmiterjoin%
\definecolor{currentfill}{rgb}{0.227451,0.192157,0.427451}%
\pgfsetfillcolor{currentfill}%
\pgfsetlinewidth{0.100375pt}%
\definecolor{currentstroke}{rgb}{0.266667,0.266667,0.266667}%
\pgfsetstrokecolor{currentstroke}%
\pgfsetdash{}{0pt}%
\pgfpathmoveto{\pgfqpoint{3.098282in}{0.638889in}}%
\pgfpathlineto{\pgfqpoint{3.124033in}{0.638889in}}%
\pgfpathlineto{\pgfqpoint{3.124033in}{0.659990in}}%
\pgfpathlineto{\pgfqpoint{3.098282in}{0.659990in}}%
\pgfpathlineto{\pgfqpoint{3.098282in}{0.638889in}}%
\pgfpathclose%
\pgfusepath{stroke,fill}%
\end{pgfscope}%
\begin{pgfscope}%
\pgfpathrectangle{\pgfqpoint{0.781250in}{0.638889in}}{\pgfqpoint{4.843750in}{2.172222in}}%
\pgfusepath{clip}%
\pgfsetbuttcap%
\pgfsetmiterjoin%
\definecolor{currentfill}{rgb}{0.227451,0.192157,0.427451}%
\pgfsetfillcolor{currentfill}%
\pgfsetlinewidth{0.100375pt}%
\definecolor{currentstroke}{rgb}{0.266667,0.266667,0.266667}%
\pgfsetstrokecolor{currentstroke}%
\pgfsetdash{}{0pt}%
\pgfpathmoveto{\pgfqpoint{3.135069in}{0.638889in}}%
\pgfpathlineto{\pgfqpoint{3.160820in}{0.638889in}}%
\pgfpathlineto{\pgfqpoint{3.160820in}{0.792186in}}%
\pgfpathlineto{\pgfqpoint{3.135069in}{0.792186in}}%
\pgfpathlineto{\pgfqpoint{3.135069in}{0.638889in}}%
\pgfpathclose%
\pgfusepath{stroke,fill}%
\end{pgfscope}%
\begin{pgfscope}%
\pgfpathrectangle{\pgfqpoint{0.781250in}{0.638889in}}{\pgfqpoint{4.843750in}{2.172222in}}%
\pgfusepath{clip}%
\pgfsetbuttcap%
\pgfsetmiterjoin%
\definecolor{currentfill}{rgb}{0.227451,0.192157,0.427451}%
\pgfsetfillcolor{currentfill}%
\pgfsetlinewidth{0.100375pt}%
\definecolor{currentstroke}{rgb}{0.266667,0.266667,0.266667}%
\pgfsetstrokecolor{currentstroke}%
\pgfsetdash{}{0pt}%
\pgfpathmoveto{\pgfqpoint{3.171856in}{0.638889in}}%
\pgfpathlineto{\pgfqpoint{3.197607in}{0.638889in}}%
\pgfpathlineto{\pgfqpoint{3.197607in}{0.735087in}}%
\pgfpathlineto{\pgfqpoint{3.171856in}{0.735087in}}%
\pgfpathlineto{\pgfqpoint{3.171856in}{0.638889in}}%
\pgfpathclose%
\pgfusepath{stroke,fill}%
\end{pgfscope}%
\begin{pgfscope}%
\pgfpathrectangle{\pgfqpoint{0.781250in}{0.638889in}}{\pgfqpoint{4.843750in}{2.172222in}}%
\pgfusepath{clip}%
\pgfsetbuttcap%
\pgfsetmiterjoin%
\definecolor{currentfill}{rgb}{0.227451,0.192157,0.427451}%
\pgfsetfillcolor{currentfill}%
\pgfsetlinewidth{0.100375pt}%
\definecolor{currentstroke}{rgb}{0.266667,0.266667,0.266667}%
\pgfsetstrokecolor{currentstroke}%
\pgfsetdash{}{0pt}%
\pgfpathmoveto{\pgfqpoint{3.208643in}{0.638889in}}%
\pgfpathlineto{\pgfqpoint{3.234394in}{0.638889in}}%
\pgfpathlineto{\pgfqpoint{3.234394in}{0.740052in}}%
\pgfpathlineto{\pgfqpoint{3.208643in}{0.740052in}}%
\pgfpathlineto{\pgfqpoint{3.208643in}{0.638889in}}%
\pgfpathclose%
\pgfusepath{stroke,fill}%
\end{pgfscope}%
\begin{pgfscope}%
\pgfpathrectangle{\pgfqpoint{0.781250in}{0.638889in}}{\pgfqpoint{4.843750in}{2.172222in}}%
\pgfusepath{clip}%
\pgfsetbuttcap%
\pgfsetmiterjoin%
\definecolor{currentfill}{rgb}{0.227451,0.192157,0.427451}%
\pgfsetfillcolor{currentfill}%
\pgfsetlinewidth{0.100375pt}%
\definecolor{currentstroke}{rgb}{0.266667,0.266667,0.266667}%
\pgfsetstrokecolor{currentstroke}%
\pgfsetdash{}{0pt}%
\pgfpathmoveto{\pgfqpoint{3.245430in}{0.638889in}}%
\pgfpathlineto{\pgfqpoint{3.271181in}{0.638889in}}%
\pgfpathlineto{\pgfqpoint{3.271181in}{0.711193in}}%
\pgfpathlineto{\pgfqpoint{3.245430in}{0.711193in}}%
\pgfpathlineto{\pgfqpoint{3.245430in}{0.638889in}}%
\pgfpathclose%
\pgfusepath{stroke,fill}%
\end{pgfscope}%
\begin{pgfscope}%
\pgfpathrectangle{\pgfqpoint{0.781250in}{0.638889in}}{\pgfqpoint{4.843750in}{2.172222in}}%
\pgfusepath{clip}%
\pgfsetbuttcap%
\pgfsetmiterjoin%
\definecolor{currentfill}{rgb}{0.227451,0.192157,0.427451}%
\pgfsetfillcolor{currentfill}%
\pgfsetlinewidth{0.100375pt}%
\definecolor{currentstroke}{rgb}{0.266667,0.266667,0.266667}%
\pgfsetstrokecolor{currentstroke}%
\pgfsetdash{}{0pt}%
\pgfpathmoveto{\pgfqpoint{3.282217in}{0.638889in}}%
\pgfpathlineto{\pgfqpoint{3.307968in}{0.638889in}}%
\pgfpathlineto{\pgfqpoint{3.307968in}{0.950448in}}%
\pgfpathlineto{\pgfqpoint{3.282217in}{0.950448in}}%
\pgfpathlineto{\pgfqpoint{3.282217in}{0.638889in}}%
\pgfpathclose%
\pgfusepath{stroke,fill}%
\end{pgfscope}%
\begin{pgfscope}%
\pgfpathrectangle{\pgfqpoint{0.781250in}{0.638889in}}{\pgfqpoint{4.843750in}{2.172222in}}%
\pgfusepath{clip}%
\pgfsetbuttcap%
\pgfsetmiterjoin%
\definecolor{currentfill}{rgb}{0.227451,0.192157,0.427451}%
\pgfsetfillcolor{currentfill}%
\pgfsetlinewidth{0.100375pt}%
\definecolor{currentstroke}{rgb}{0.266667,0.266667,0.266667}%
\pgfsetstrokecolor{currentstroke}%
\pgfsetdash{}{0pt}%
\pgfpathmoveto{\pgfqpoint{3.319004in}{0.638889in}}%
\pgfpathlineto{\pgfqpoint{3.344755in}{0.638889in}}%
\pgfpathlineto{\pgfqpoint{3.344755in}{0.782876in}}%
\pgfpathlineto{\pgfqpoint{3.319004in}{0.782876in}}%
\pgfpathlineto{\pgfqpoint{3.319004in}{0.638889in}}%
\pgfpathclose%
\pgfusepath{stroke,fill}%
\end{pgfscope}%
\begin{pgfscope}%
\pgfpathrectangle{\pgfqpoint{0.781250in}{0.638889in}}{\pgfqpoint{4.843750in}{2.172222in}}%
\pgfusepath{clip}%
\pgfsetbuttcap%
\pgfsetmiterjoin%
\definecolor{currentfill}{rgb}{0.227451,0.192157,0.427451}%
\pgfsetfillcolor{currentfill}%
\pgfsetlinewidth{0.100375pt}%
\definecolor{currentstroke}{rgb}{0.266667,0.266667,0.266667}%
\pgfsetstrokecolor{currentstroke}%
\pgfsetdash{}{0pt}%
\pgfpathmoveto{\pgfqpoint{3.355791in}{0.638889in}}%
\pgfpathlineto{\pgfqpoint{3.381542in}{0.638889in}}%
\pgfpathlineto{\pgfqpoint{3.381542in}{0.892418in}}%
\pgfpathlineto{\pgfqpoint{3.355791in}{0.892418in}}%
\pgfpathlineto{\pgfqpoint{3.355791in}{0.638889in}}%
\pgfpathclose%
\pgfusepath{stroke,fill}%
\end{pgfscope}%
\begin{pgfscope}%
\pgfpathrectangle{\pgfqpoint{0.781250in}{0.638889in}}{\pgfqpoint{4.843750in}{2.172222in}}%
\pgfusepath{clip}%
\pgfsetbuttcap%
\pgfsetmiterjoin%
\definecolor{currentfill}{rgb}{0.164706,0.615686,0.560784}%
\pgfsetfillcolor{currentfill}%
\pgfsetlinewidth{0.100375pt}%
\definecolor{currentstroke}{rgb}{0.266667,0.266667,0.266667}%
\pgfsetstrokecolor{currentstroke}%
\pgfsetdash{}{0pt}%
\pgfpathmoveto{\pgfqpoint{3.392578in}{0.638889in}}%
\pgfpathlineto{\pgfqpoint{3.418329in}{0.638889in}}%
\pgfpathlineto{\pgfqpoint{3.418329in}{0.940207in}}%
\pgfpathlineto{\pgfqpoint{3.392578in}{0.940207in}}%
\pgfpathlineto{\pgfqpoint{3.392578in}{0.638889in}}%
\pgfpathclose%
\pgfusepath{stroke,fill}%
\end{pgfscope}%
\begin{pgfscope}%
\pgfpathrectangle{\pgfqpoint{0.781250in}{0.638889in}}{\pgfqpoint{4.843750in}{2.172222in}}%
\pgfusepath{clip}%
\pgfsetbuttcap%
\pgfsetmiterjoin%
\definecolor{currentfill}{rgb}{0.164706,0.615686,0.560784}%
\pgfsetfillcolor{currentfill}%
\pgfsetlinewidth{0.100375pt}%
\definecolor{currentstroke}{rgb}{0.266667,0.266667,0.266667}%
\pgfsetstrokecolor{currentstroke}%
\pgfsetdash{}{0pt}%
\pgfpathmoveto{\pgfqpoint{3.429365in}{0.638889in}}%
\pgfpathlineto{\pgfqpoint{3.455116in}{0.638889in}}%
\pgfpathlineto{\pgfqpoint{3.455116in}{1.335862in}}%
\pgfpathlineto{\pgfqpoint{3.429365in}{1.335862in}}%
\pgfpathlineto{\pgfqpoint{3.429365in}{0.638889in}}%
\pgfpathclose%
\pgfusepath{stroke,fill}%
\end{pgfscope}%
\begin{pgfscope}%
\pgfpathrectangle{\pgfqpoint{0.781250in}{0.638889in}}{\pgfqpoint{4.843750in}{2.172222in}}%
\pgfusepath{clip}%
\pgfsetbuttcap%
\pgfsetmiterjoin%
\definecolor{currentfill}{rgb}{0.164706,0.615686,0.560784}%
\pgfsetfillcolor{currentfill}%
\pgfsetlinewidth{0.100375pt}%
\definecolor{currentstroke}{rgb}{0.266667,0.266667,0.266667}%
\pgfsetstrokecolor{currentstroke}%
\pgfsetdash{}{0pt}%
\pgfpathmoveto{\pgfqpoint{3.466152in}{0.638889in}}%
\pgfpathlineto{\pgfqpoint{3.491903in}{0.638889in}}%
\pgfpathlineto{\pgfqpoint{3.491903in}{1.910260in}}%
\pgfpathlineto{\pgfqpoint{3.466152in}{1.910260in}}%
\pgfpathlineto{\pgfqpoint{3.466152in}{0.638889in}}%
\pgfpathclose%
\pgfusepath{stroke,fill}%
\end{pgfscope}%
\begin{pgfscope}%
\pgfpathrectangle{\pgfqpoint{0.781250in}{0.638889in}}{\pgfqpoint{4.843750in}{2.172222in}}%
\pgfusepath{clip}%
\pgfsetbuttcap%
\pgfsetmiterjoin%
\definecolor{currentfill}{rgb}{0.164706,0.615686,0.560784}%
\pgfsetfillcolor{currentfill}%
\pgfsetlinewidth{0.100375pt}%
\definecolor{currentstroke}{rgb}{0.266667,0.266667,0.266667}%
\pgfsetstrokecolor{currentstroke}%
\pgfsetdash{}{0pt}%
\pgfpathmoveto{\pgfqpoint{3.502939in}{0.638889in}}%
\pgfpathlineto{\pgfqpoint{3.528690in}{0.638889in}}%
\pgfpathlineto{\pgfqpoint{3.528690in}{1.155878in}}%
\pgfpathlineto{\pgfqpoint{3.502939in}{1.155878in}}%
\pgfpathlineto{\pgfqpoint{3.502939in}{0.638889in}}%
\pgfpathclose%
\pgfusepath{stroke,fill}%
\end{pgfscope}%
\begin{pgfscope}%
\pgfpathrectangle{\pgfqpoint{0.781250in}{0.638889in}}{\pgfqpoint{4.843750in}{2.172222in}}%
\pgfusepath{clip}%
\pgfsetbuttcap%
\pgfsetmiterjoin%
\definecolor{currentfill}{rgb}{0.227451,0.192157,0.427451}%
\pgfsetfillcolor{currentfill}%
\pgfsetlinewidth{0.100375pt}%
\definecolor{currentstroke}{rgb}{0.266667,0.266667,0.266667}%
\pgfsetstrokecolor{currentstroke}%
\pgfsetdash{}{0pt}%
\pgfpathmoveto{\pgfqpoint{3.539726in}{0.638889in}}%
\pgfpathlineto{\pgfqpoint{3.565477in}{0.638889in}}%
\pgfpathlineto{\pgfqpoint{3.565477in}{0.794979in}}%
\pgfpathlineto{\pgfqpoint{3.539726in}{0.794979in}}%
\pgfpathlineto{\pgfqpoint{3.539726in}{0.638889in}}%
\pgfpathclose%
\pgfusepath{stroke,fill}%
\end{pgfscope}%
\begin{pgfscope}%
\pgfpathrectangle{\pgfqpoint{0.781250in}{0.638889in}}{\pgfqpoint{4.843750in}{2.172222in}}%
\pgfusepath{clip}%
\pgfsetbuttcap%
\pgfsetmiterjoin%
\definecolor{currentfill}{rgb}{0.227451,0.192157,0.427451}%
\pgfsetfillcolor{currentfill}%
\pgfsetlinewidth{0.100375pt}%
\definecolor{currentstroke}{rgb}{0.266667,0.266667,0.266667}%
\pgfsetstrokecolor{currentstroke}%
\pgfsetdash{}{0pt}%
\pgfpathmoveto{\pgfqpoint{3.576513in}{0.638889in}}%
\pgfpathlineto{\pgfqpoint{3.602264in}{0.638889in}}%
\pgfpathlineto{\pgfqpoint{3.602264in}{0.950137in}}%
\pgfpathlineto{\pgfqpoint{3.576513in}{0.950137in}}%
\pgfpathlineto{\pgfqpoint{3.576513in}{0.638889in}}%
\pgfpathclose%
\pgfusepath{stroke,fill}%
\end{pgfscope}%
\begin{pgfscope}%
\pgfpathrectangle{\pgfqpoint{0.781250in}{0.638889in}}{\pgfqpoint{4.843750in}{2.172222in}}%
\pgfusepath{clip}%
\pgfsetbuttcap%
\pgfsetmiterjoin%
\definecolor{currentfill}{rgb}{0.227451,0.192157,0.427451}%
\pgfsetfillcolor{currentfill}%
\pgfsetlinewidth{0.100375pt}%
\definecolor{currentstroke}{rgb}{0.266667,0.266667,0.266667}%
\pgfsetstrokecolor{currentstroke}%
\pgfsetdash{}{0pt}%
\pgfpathmoveto{\pgfqpoint{3.613301in}{0.638889in}}%
\pgfpathlineto{\pgfqpoint{3.639051in}{0.638889in}}%
\pgfpathlineto{\pgfqpoint{3.639051in}{0.704056in}}%
\pgfpathlineto{\pgfqpoint{3.613301in}{0.704056in}}%
\pgfpathlineto{\pgfqpoint{3.613301in}{0.638889in}}%
\pgfpathclose%
\pgfusepath{stroke,fill}%
\end{pgfscope}%
\begin{pgfscope}%
\pgfpathrectangle{\pgfqpoint{0.781250in}{0.638889in}}{\pgfqpoint{4.843750in}{2.172222in}}%
\pgfusepath{clip}%
\pgfsetbuttcap%
\pgfsetmiterjoin%
\definecolor{currentfill}{rgb}{0.227451,0.192157,0.427451}%
\pgfsetfillcolor{currentfill}%
\pgfsetlinewidth{0.100375pt}%
\definecolor{currentstroke}{rgb}{0.266667,0.266667,0.266667}%
\pgfsetstrokecolor{currentstroke}%
\pgfsetdash{}{0pt}%
\pgfpathmoveto{\pgfqpoint{3.650088in}{0.638889in}}%
\pgfpathlineto{\pgfqpoint{3.675839in}{0.638889in}}%
\pgfpathlineto{\pgfqpoint{3.675839in}{0.655025in}}%
\pgfpathlineto{\pgfqpoint{3.650088in}{0.655025in}}%
\pgfpathlineto{\pgfqpoint{3.650088in}{0.638889in}}%
\pgfpathclose%
\pgfusepath{stroke,fill}%
\end{pgfscope}%
\begin{pgfscope}%
\pgfpathrectangle{\pgfqpoint{0.781250in}{0.638889in}}{\pgfqpoint{4.843750in}{2.172222in}}%
\pgfusepath{clip}%
\pgfsetbuttcap%
\pgfsetmiterjoin%
\definecolor{currentfill}{rgb}{0.227451,0.192157,0.427451}%
\pgfsetfillcolor{currentfill}%
\pgfsetlinewidth{0.100375pt}%
\definecolor{currentstroke}{rgb}{0.266667,0.266667,0.266667}%
\pgfsetstrokecolor{currentstroke}%
\pgfsetdash{}{0pt}%
\pgfpathmoveto{\pgfqpoint{3.686875in}{0.638889in}}%
\pgfpathlineto{\pgfqpoint{3.712626in}{0.638889in}}%
\pgfpathlineto{\pgfqpoint{3.712626in}{0.673024in}}%
\pgfpathlineto{\pgfqpoint{3.686875in}{0.673024in}}%
\pgfpathlineto{\pgfqpoint{3.686875in}{0.638889in}}%
\pgfpathclose%
\pgfusepath{stroke,fill}%
\end{pgfscope}%
\begin{pgfscope}%
\pgfpathrectangle{\pgfqpoint{0.781250in}{0.638889in}}{\pgfqpoint{4.843750in}{2.172222in}}%
\pgfusepath{clip}%
\pgfsetbuttcap%
\pgfsetmiterjoin%
\definecolor{currentfill}{rgb}{0.227451,0.192157,0.427451}%
\pgfsetfillcolor{currentfill}%
\pgfsetlinewidth{0.100375pt}%
\definecolor{currentstroke}{rgb}{0.266667,0.266667,0.266667}%
\pgfsetstrokecolor{currentstroke}%
\pgfsetdash{}{0pt}%
\pgfpathmoveto{\pgfqpoint{3.723662in}{0.638889in}}%
\pgfpathlineto{\pgfqpoint{3.749413in}{0.638889in}}%
\pgfpathlineto{\pgfqpoint{3.749413in}{0.855180in}}%
\pgfpathlineto{\pgfqpoint{3.723662in}{0.855180in}}%
\pgfpathlineto{\pgfqpoint{3.723662in}{0.638889in}}%
\pgfpathclose%
\pgfusepath{stroke,fill}%
\end{pgfscope}%
\begin{pgfscope}%
\pgfpathrectangle{\pgfqpoint{0.781250in}{0.638889in}}{\pgfqpoint{4.843750in}{2.172222in}}%
\pgfusepath{clip}%
\pgfsetbuttcap%
\pgfsetmiterjoin%
\definecolor{currentfill}{rgb}{0.227451,0.192157,0.427451}%
\pgfsetfillcolor{currentfill}%
\pgfsetlinewidth{0.100375pt}%
\definecolor{currentstroke}{rgb}{0.266667,0.266667,0.266667}%
\pgfsetstrokecolor{currentstroke}%
\pgfsetdash{}{0pt}%
\pgfpathmoveto{\pgfqpoint{3.760449in}{0.638889in}}%
\pgfpathlineto{\pgfqpoint{3.786200in}{0.638889in}}%
\pgfpathlineto{\pgfqpoint{3.786200in}{0.819804in}}%
\pgfpathlineto{\pgfqpoint{3.760449in}{0.819804in}}%
\pgfpathlineto{\pgfqpoint{3.760449in}{0.638889in}}%
\pgfpathclose%
\pgfusepath{stroke,fill}%
\end{pgfscope}%
\begin{pgfscope}%
\pgfpathrectangle{\pgfqpoint{0.781250in}{0.638889in}}{\pgfqpoint{4.843750in}{2.172222in}}%
\pgfusepath{clip}%
\pgfsetbuttcap%
\pgfsetmiterjoin%
\definecolor{currentfill}{rgb}{0.227451,0.192157,0.427451}%
\pgfsetfillcolor{currentfill}%
\pgfsetlinewidth{0.100375pt}%
\definecolor{currentstroke}{rgb}{0.266667,0.266667,0.266667}%
\pgfsetstrokecolor{currentstroke}%
\pgfsetdash{}{0pt}%
\pgfpathmoveto{\pgfqpoint{3.797236in}{0.638889in}}%
\pgfpathlineto{\pgfqpoint{3.822987in}{0.638889in}}%
\pgfpathlineto{\pgfqpoint{3.822987in}{0.661232in}}%
\pgfpathlineto{\pgfqpoint{3.797236in}{0.661232in}}%
\pgfpathlineto{\pgfqpoint{3.797236in}{0.638889in}}%
\pgfpathclose%
\pgfusepath{stroke,fill}%
\end{pgfscope}%
\begin{pgfscope}%
\pgfpathrectangle{\pgfqpoint{0.781250in}{0.638889in}}{\pgfqpoint{4.843750in}{2.172222in}}%
\pgfusepath{clip}%
\pgfsetbuttcap%
\pgfsetmiterjoin%
\definecolor{currentfill}{rgb}{0.164706,0.615686,0.560784}%
\pgfsetfillcolor{currentfill}%
\pgfsetlinewidth{0.100375pt}%
\definecolor{currentstroke}{rgb}{0.266667,0.266667,0.266667}%
\pgfsetstrokecolor{currentstroke}%
\pgfsetdash{}{0pt}%
\pgfpathmoveto{\pgfqpoint{3.834023in}{0.638889in}}%
\pgfpathlineto{\pgfqpoint{3.859774in}{0.638889in}}%
\pgfpathlineto{\pgfqpoint{3.859774in}{1.175117in}}%
\pgfpathlineto{\pgfqpoint{3.834023in}{1.175117in}}%
\pgfpathlineto{\pgfqpoint{3.834023in}{0.638889in}}%
\pgfpathclose%
\pgfusepath{stroke,fill}%
\end{pgfscope}%
\begin{pgfscope}%
\pgfpathrectangle{\pgfqpoint{0.781250in}{0.638889in}}{\pgfqpoint{4.843750in}{2.172222in}}%
\pgfusepath{clip}%
\pgfsetbuttcap%
\pgfsetmiterjoin%
\definecolor{currentfill}{rgb}{0.164706,0.615686,0.560784}%
\pgfsetfillcolor{currentfill}%
\pgfsetlinewidth{0.100375pt}%
\definecolor{currentstroke}{rgb}{0.266667,0.266667,0.266667}%
\pgfsetstrokecolor{currentstroke}%
\pgfsetdash{}{0pt}%
\pgfpathmoveto{\pgfqpoint{3.870810in}{0.638889in}}%
\pgfpathlineto{\pgfqpoint{3.896561in}{0.638889in}}%
\pgfpathlineto{\pgfqpoint{3.896561in}{1.030199in}}%
\pgfpathlineto{\pgfqpoint{3.870810in}{1.030199in}}%
\pgfpathlineto{\pgfqpoint{3.870810in}{0.638889in}}%
\pgfpathclose%
\pgfusepath{stroke,fill}%
\end{pgfscope}%
\begin{pgfscope}%
\pgfpathrectangle{\pgfqpoint{0.781250in}{0.638889in}}{\pgfqpoint{4.843750in}{2.172222in}}%
\pgfusepath{clip}%
\pgfsetbuttcap%
\pgfsetmiterjoin%
\definecolor{currentfill}{rgb}{0.164706,0.615686,0.560784}%
\pgfsetfillcolor{currentfill}%
\pgfsetlinewidth{0.100375pt}%
\definecolor{currentstroke}{rgb}{0.266667,0.266667,0.266667}%
\pgfsetstrokecolor{currentstroke}%
\pgfsetdash{}{0pt}%
\pgfpathmoveto{\pgfqpoint{3.907597in}{0.638889in}}%
\pgfpathlineto{\pgfqpoint{3.933348in}{0.638889in}}%
\pgfpathlineto{\pgfqpoint{3.933348in}{1.497537in}}%
\pgfpathlineto{\pgfqpoint{3.907597in}{1.497537in}}%
\pgfpathlineto{\pgfqpoint{3.907597in}{0.638889in}}%
\pgfpathclose%
\pgfusepath{stroke,fill}%
\end{pgfscope}%
\begin{pgfscope}%
\pgfpathrectangle{\pgfqpoint{0.781250in}{0.638889in}}{\pgfqpoint{4.843750in}{2.172222in}}%
\pgfusepath{clip}%
\pgfsetbuttcap%
\pgfsetmiterjoin%
\definecolor{currentfill}{rgb}{0.164706,0.615686,0.560784}%
\pgfsetfillcolor{currentfill}%
\pgfsetlinewidth{0.100375pt}%
\definecolor{currentstroke}{rgb}{0.266667,0.266667,0.266667}%
\pgfsetstrokecolor{currentstroke}%
\pgfsetdash{}{0pt}%
\pgfpathmoveto{\pgfqpoint{3.944384in}{0.638889in}}%
\pgfpathlineto{\pgfqpoint{3.970135in}{0.638889in}}%
\pgfpathlineto{\pgfqpoint{3.970135in}{0.905452in}}%
\pgfpathlineto{\pgfqpoint{3.944384in}{0.905452in}}%
\pgfpathlineto{\pgfqpoint{3.944384in}{0.638889in}}%
\pgfpathclose%
\pgfusepath{stroke,fill}%
\end{pgfscope}%
\begin{pgfscope}%
\pgfpathrectangle{\pgfqpoint{0.781250in}{0.638889in}}{\pgfqpoint{4.843750in}{2.172222in}}%
\pgfusepath{clip}%
\pgfsetbuttcap%
\pgfsetmiterjoin%
\definecolor{currentfill}{rgb}{0.227451,0.192157,0.427451}%
\pgfsetfillcolor{currentfill}%
\pgfsetlinewidth{0.100375pt}%
\definecolor{currentstroke}{rgb}{0.266667,0.266667,0.266667}%
\pgfsetstrokecolor{currentstroke}%
\pgfsetdash{}{0pt}%
\pgfpathmoveto{\pgfqpoint{3.981171in}{0.638889in}}%
\pgfpathlineto{\pgfqpoint{4.006922in}{0.638889in}}%
\pgfpathlineto{\pgfqpoint{4.006922in}{0.755568in}}%
\pgfpathlineto{\pgfqpoint{3.981171in}{0.755568in}}%
\pgfpathlineto{\pgfqpoint{3.981171in}{0.638889in}}%
\pgfpathclose%
\pgfusepath{stroke,fill}%
\end{pgfscope}%
\begin{pgfscope}%
\pgfpathrectangle{\pgfqpoint{0.781250in}{0.638889in}}{\pgfqpoint{4.843750in}{2.172222in}}%
\pgfusepath{clip}%
\pgfsetbuttcap%
\pgfsetmiterjoin%
\definecolor{currentfill}{rgb}{0.227451,0.192157,0.427451}%
\pgfsetfillcolor{currentfill}%
\pgfsetlinewidth{0.100375pt}%
\definecolor{currentstroke}{rgb}{0.266667,0.266667,0.266667}%
\pgfsetstrokecolor{currentstroke}%
\pgfsetdash{}{0pt}%
\pgfpathmoveto{\pgfqpoint{4.017958in}{0.638889in}}%
\pgfpathlineto{\pgfqpoint{4.043709in}{0.638889in}}%
\pgfpathlineto{\pgfqpoint{4.043709in}{0.836561in}}%
\pgfpathlineto{\pgfqpoint{4.017958in}{0.836561in}}%
\pgfpathlineto{\pgfqpoint{4.017958in}{0.638889in}}%
\pgfpathclose%
\pgfusepath{stroke,fill}%
\end{pgfscope}%
\begin{pgfscope}%
\pgfpathrectangle{\pgfqpoint{0.781250in}{0.638889in}}{\pgfqpoint{4.843750in}{2.172222in}}%
\pgfusepath{clip}%
\pgfsetbuttcap%
\pgfsetmiterjoin%
\definecolor{currentfill}{rgb}{0.227451,0.192157,0.427451}%
\pgfsetfillcolor{currentfill}%
\pgfsetlinewidth{0.100375pt}%
\definecolor{currentstroke}{rgb}{0.266667,0.266667,0.266667}%
\pgfsetstrokecolor{currentstroke}%
\pgfsetdash{}{0pt}%
\pgfpathmoveto{\pgfqpoint{4.054745in}{0.638889in}}%
\pgfpathlineto{\pgfqpoint{4.080496in}{0.638889in}}%
\pgfpathlineto{\pgfqpoint{4.080496in}{0.815770in}}%
\pgfpathlineto{\pgfqpoint{4.054745in}{0.815770in}}%
\pgfpathlineto{\pgfqpoint{4.054745in}{0.638889in}}%
\pgfpathclose%
\pgfusepath{stroke,fill}%
\end{pgfscope}%
\begin{pgfscope}%
\pgfpathrectangle{\pgfqpoint{0.781250in}{0.638889in}}{\pgfqpoint{4.843750in}{2.172222in}}%
\pgfusepath{clip}%
\pgfsetbuttcap%
\pgfsetmiterjoin%
\definecolor{currentfill}{rgb}{0.227451,0.192157,0.427451}%
\pgfsetfillcolor{currentfill}%
\pgfsetlinewidth{0.100375pt}%
\definecolor{currentstroke}{rgb}{0.266667,0.266667,0.266667}%
\pgfsetstrokecolor{currentstroke}%
\pgfsetdash{}{0pt}%
\pgfpathmoveto{\pgfqpoint{4.091532in}{0.638889in}}%
\pgfpathlineto{\pgfqpoint{4.117283in}{0.638889in}}%
\pgfpathlineto{\pgfqpoint{4.117283in}{0.748741in}}%
\pgfpathlineto{\pgfqpoint{4.091532in}{0.748741in}}%
\pgfpathlineto{\pgfqpoint{4.091532in}{0.638889in}}%
\pgfpathclose%
\pgfusepath{stroke,fill}%
\end{pgfscope}%
\begin{pgfscope}%
\pgfpathrectangle{\pgfqpoint{0.781250in}{0.638889in}}{\pgfqpoint{4.843750in}{2.172222in}}%
\pgfusepath{clip}%
\pgfsetbuttcap%
\pgfsetmiterjoin%
\definecolor{currentfill}{rgb}{0.227451,0.192157,0.427451}%
\pgfsetfillcolor{currentfill}%
\pgfsetlinewidth{0.100375pt}%
\definecolor{currentstroke}{rgb}{0.266667,0.266667,0.266667}%
\pgfsetstrokecolor{currentstroke}%
\pgfsetdash{}{0pt}%
\pgfpathmoveto{\pgfqpoint{4.128319in}{0.638889in}}%
\pgfpathlineto{\pgfqpoint{4.154070in}{0.638889in}}%
\pgfpathlineto{\pgfqpoint{4.154070in}{0.653784in}}%
\pgfpathlineto{\pgfqpoint{4.128319in}{0.653784in}}%
\pgfpathlineto{\pgfqpoint{4.128319in}{0.638889in}}%
\pgfpathclose%
\pgfusepath{stroke,fill}%
\end{pgfscope}%
\begin{pgfscope}%
\pgfpathrectangle{\pgfqpoint{0.781250in}{0.638889in}}{\pgfqpoint{4.843750in}{2.172222in}}%
\pgfusepath{clip}%
\pgfsetbuttcap%
\pgfsetmiterjoin%
\definecolor{currentfill}{rgb}{0.227451,0.192157,0.427451}%
\pgfsetfillcolor{currentfill}%
\pgfsetlinewidth{0.100375pt}%
\definecolor{currentstroke}{rgb}{0.266667,0.266667,0.266667}%
\pgfsetstrokecolor{currentstroke}%
\pgfsetdash{}{0pt}%
\pgfpathmoveto{\pgfqpoint{4.165106in}{0.638889in}}%
\pgfpathlineto{\pgfqpoint{4.190857in}{0.638889in}}%
\pgfpathlineto{\pgfqpoint{4.190857in}{0.700332in}}%
\pgfpathlineto{\pgfqpoint{4.165106in}{0.700332in}}%
\pgfpathlineto{\pgfqpoint{4.165106in}{0.638889in}}%
\pgfpathclose%
\pgfusepath{stroke,fill}%
\end{pgfscope}%
\begin{pgfscope}%
\pgfpathrectangle{\pgfqpoint{0.781250in}{0.638889in}}{\pgfqpoint{4.843750in}{2.172222in}}%
\pgfusepath{clip}%
\pgfsetbuttcap%
\pgfsetmiterjoin%
\definecolor{currentfill}{rgb}{0.227451,0.192157,0.427451}%
\pgfsetfillcolor{currentfill}%
\pgfsetlinewidth{0.100375pt}%
\definecolor{currentstroke}{rgb}{0.266667,0.266667,0.266667}%
\pgfsetstrokecolor{currentstroke}%
\pgfsetdash{}{0pt}%
\pgfpathmoveto{\pgfqpoint{4.201893in}{0.638889in}}%
\pgfpathlineto{\pgfqpoint{4.227644in}{0.638889in}}%
\pgfpathlineto{\pgfqpoint{4.227644in}{0.789703in}}%
\pgfpathlineto{\pgfqpoint{4.201893in}{0.789703in}}%
\pgfpathlineto{\pgfqpoint{4.201893in}{0.638889in}}%
\pgfpathclose%
\pgfusepath{stroke,fill}%
\end{pgfscope}%
\begin{pgfscope}%
\pgfpathrectangle{\pgfqpoint{0.781250in}{0.638889in}}{\pgfqpoint{4.843750in}{2.172222in}}%
\pgfusepath{clip}%
\pgfsetbuttcap%
\pgfsetmiterjoin%
\definecolor{currentfill}{rgb}{0.227451,0.192157,0.427451}%
\pgfsetfillcolor{currentfill}%
\pgfsetlinewidth{0.100375pt}%
\definecolor{currentstroke}{rgb}{0.266667,0.266667,0.266667}%
\pgfsetstrokecolor{currentstroke}%
\pgfsetdash{}{0pt}%
\pgfpathmoveto{\pgfqpoint{4.238680in}{0.638889in}}%
\pgfpathlineto{\pgfqpoint{4.264431in}{0.638889in}}%
\pgfpathlineto{\pgfqpoint{4.264431in}{1.373410in}}%
\pgfpathlineto{\pgfqpoint{4.238680in}{1.373410in}}%
\pgfpathlineto{\pgfqpoint{4.238680in}{0.638889in}}%
\pgfpathclose%
\pgfusepath{stroke,fill}%
\end{pgfscope}%
\begin{pgfscope}%
\pgfpathrectangle{\pgfqpoint{0.781250in}{0.638889in}}{\pgfqpoint{4.843750in}{2.172222in}}%
\pgfusepath{clip}%
\pgfsetbuttcap%
\pgfsetmiterjoin%
\definecolor{currentfill}{rgb}{0.164706,0.615686,0.560784}%
\pgfsetfillcolor{currentfill}%
\pgfsetlinewidth{0.100375pt}%
\definecolor{currentstroke}{rgb}{0.266667,0.266667,0.266667}%
\pgfsetstrokecolor{currentstroke}%
\pgfsetdash{}{0pt}%
\pgfpathmoveto{\pgfqpoint{4.275467in}{0.638889in}}%
\pgfpathlineto{\pgfqpoint{4.301218in}{0.638889in}}%
\pgfpathlineto{\pgfqpoint{4.301218in}{1.405994in}}%
\pgfpathlineto{\pgfqpoint{4.275467in}{1.405994in}}%
\pgfpathlineto{\pgfqpoint{4.275467in}{0.638889in}}%
\pgfpathclose%
\pgfusepath{stroke,fill}%
\end{pgfscope}%
\begin{pgfscope}%
\pgfpathrectangle{\pgfqpoint{0.781250in}{0.638889in}}{\pgfqpoint{4.843750in}{2.172222in}}%
\pgfusepath{clip}%
\pgfsetbuttcap%
\pgfsetmiterjoin%
\definecolor{currentfill}{rgb}{0.164706,0.615686,0.560784}%
\pgfsetfillcolor{currentfill}%
\pgfsetlinewidth{0.100375pt}%
\definecolor{currentstroke}{rgb}{0.266667,0.266667,0.266667}%
\pgfsetstrokecolor{currentstroke}%
\pgfsetdash{}{0pt}%
\pgfpathmoveto{\pgfqpoint{4.312254in}{0.638889in}}%
\pgfpathlineto{\pgfqpoint{4.338005in}{0.638889in}}%
\pgfpathlineto{\pgfqpoint{4.338005in}{2.647263in}}%
\pgfpathlineto{\pgfqpoint{4.312254in}{2.647263in}}%
\pgfpathlineto{\pgfqpoint{4.312254in}{0.638889in}}%
\pgfpathclose%
\pgfusepath{stroke,fill}%
\end{pgfscope}%
\begin{pgfscope}%
\pgfpathrectangle{\pgfqpoint{0.781250in}{0.638889in}}{\pgfqpoint{4.843750in}{2.172222in}}%
\pgfusepath{clip}%
\pgfsetbuttcap%
\pgfsetmiterjoin%
\definecolor{currentfill}{rgb}{0.164706,0.615686,0.560784}%
\pgfsetfillcolor{currentfill}%
\pgfsetlinewidth{0.100375pt}%
\definecolor{currentstroke}{rgb}{0.266667,0.266667,0.266667}%
\pgfsetstrokecolor{currentstroke}%
\pgfsetdash{}{0pt}%
\pgfpathmoveto{\pgfqpoint{4.349041in}{0.638889in}}%
\pgfpathlineto{\pgfqpoint{4.374792in}{0.638889in}}%
\pgfpathlineto{\pgfqpoint{4.374792in}{1.161774in}}%
\pgfpathlineto{\pgfqpoint{4.349041in}{1.161774in}}%
\pgfpathlineto{\pgfqpoint{4.349041in}{0.638889in}}%
\pgfpathclose%
\pgfusepath{stroke,fill}%
\end{pgfscope}%
\begin{pgfscope}%
\pgfpathrectangle{\pgfqpoint{0.781250in}{0.638889in}}{\pgfqpoint{4.843750in}{2.172222in}}%
\pgfusepath{clip}%
\pgfsetbuttcap%
\pgfsetmiterjoin%
\definecolor{currentfill}{rgb}{0.164706,0.615686,0.560784}%
\pgfsetfillcolor{currentfill}%
\pgfsetlinewidth{0.100375pt}%
\definecolor{currentstroke}{rgb}{0.266667,0.266667,0.266667}%
\pgfsetstrokecolor{currentstroke}%
\pgfsetdash{}{0pt}%
\pgfpathmoveto{\pgfqpoint{4.385828in}{0.638889in}}%
\pgfpathlineto{\pgfqpoint{4.411579in}{0.638889in}}%
\pgfpathlineto{\pgfqpoint{4.411579in}{1.232216in}}%
\pgfpathlineto{\pgfqpoint{4.385828in}{1.232216in}}%
\pgfpathlineto{\pgfqpoint{4.385828in}{0.638889in}}%
\pgfpathclose%
\pgfusepath{stroke,fill}%
\end{pgfscope}%
\begin{pgfscope}%
\pgfpathrectangle{\pgfqpoint{0.781250in}{0.638889in}}{\pgfqpoint{4.843750in}{2.172222in}}%
\pgfusepath{clip}%
\pgfsetbuttcap%
\pgfsetmiterjoin%
\definecolor{currentfill}{rgb}{0.227451,0.192157,0.427451}%
\pgfsetfillcolor{currentfill}%
\pgfsetlinewidth{0.100375pt}%
\definecolor{currentstroke}{rgb}{0.266667,0.266667,0.266667}%
\pgfsetstrokecolor{currentstroke}%
\pgfsetdash{}{0pt}%
\pgfpathmoveto{\pgfqpoint{4.422615in}{0.638889in}}%
\pgfpathlineto{\pgfqpoint{4.448366in}{0.638889in}}%
\pgfpathlineto{\pgfqpoint{4.448366in}{0.680161in}}%
\pgfpathlineto{\pgfqpoint{4.422615in}{0.680161in}}%
\pgfpathlineto{\pgfqpoint{4.422615in}{0.638889in}}%
\pgfpathclose%
\pgfusepath{stroke,fill}%
\end{pgfscope}%
\begin{pgfscope}%
\pgfpathrectangle{\pgfqpoint{0.781250in}{0.638889in}}{\pgfqpoint{4.843750in}{2.172222in}}%
\pgfusepath{clip}%
\pgfsetbuttcap%
\pgfsetmiterjoin%
\definecolor{currentfill}{rgb}{0.227451,0.192157,0.427451}%
\pgfsetfillcolor{currentfill}%
\pgfsetlinewidth{0.100375pt}%
\definecolor{currentstroke}{rgb}{0.266667,0.266667,0.266667}%
\pgfsetstrokecolor{currentstroke}%
\pgfsetdash{}{0pt}%
\pgfpathmoveto{\pgfqpoint{4.459403in}{0.638889in}}%
\pgfpathlineto{\pgfqpoint{4.485153in}{0.638889in}}%
\pgfpathlineto{\pgfqpoint{4.485153in}{0.963791in}}%
\pgfpathlineto{\pgfqpoint{4.459403in}{0.963791in}}%
\pgfpathlineto{\pgfqpoint{4.459403in}{0.638889in}}%
\pgfpathclose%
\pgfusepath{stroke,fill}%
\end{pgfscope}%
\begin{pgfscope}%
\pgfpathrectangle{\pgfqpoint{0.781250in}{0.638889in}}{\pgfqpoint{4.843750in}{2.172222in}}%
\pgfusepath{clip}%
\pgfsetbuttcap%
\pgfsetmiterjoin%
\definecolor{currentfill}{rgb}{0.227451,0.192157,0.427451}%
\pgfsetfillcolor{currentfill}%
\pgfsetlinewidth{0.100375pt}%
\definecolor{currentstroke}{rgb}{0.266667,0.266667,0.266667}%
\pgfsetstrokecolor{currentstroke}%
\pgfsetdash{}{0pt}%
\pgfpathmoveto{\pgfqpoint{4.496190in}{0.638889in}}%
\pgfpathlineto{\pgfqpoint{4.521941in}{0.638889in}}%
\pgfpathlineto{\pgfqpoint{4.521941in}{1.061541in}}%
\pgfpathlineto{\pgfqpoint{4.496190in}{1.061541in}}%
\pgfpathlineto{\pgfqpoint{4.496190in}{0.638889in}}%
\pgfpathclose%
\pgfusepath{stroke,fill}%
\end{pgfscope}%
\begin{pgfscope}%
\pgfpathrectangle{\pgfqpoint{0.781250in}{0.638889in}}{\pgfqpoint{4.843750in}{2.172222in}}%
\pgfusepath{clip}%
\pgfsetbuttcap%
\pgfsetmiterjoin%
\definecolor{currentfill}{rgb}{0.227451,0.192157,0.427451}%
\pgfsetfillcolor{currentfill}%
\pgfsetlinewidth{0.100375pt}%
\definecolor{currentstroke}{rgb}{0.266667,0.266667,0.266667}%
\pgfsetstrokecolor{currentstroke}%
\pgfsetdash{}{0pt}%
\pgfpathmoveto{\pgfqpoint{4.532977in}{0.638889in}}%
\pgfpathlineto{\pgfqpoint{4.558728in}{0.638889in}}%
\pgfpathlineto{\pgfqpoint{4.558728in}{0.808633in}}%
\pgfpathlineto{\pgfqpoint{4.532977in}{0.808633in}}%
\pgfpathlineto{\pgfqpoint{4.532977in}{0.638889in}}%
\pgfpathclose%
\pgfusepath{stroke,fill}%
\end{pgfscope}%
\begin{pgfscope}%
\pgfpathrectangle{\pgfqpoint{0.781250in}{0.638889in}}{\pgfqpoint{4.843750in}{2.172222in}}%
\pgfusepath{clip}%
\pgfsetbuttcap%
\pgfsetmiterjoin%
\definecolor{currentfill}{rgb}{0.227451,0.192157,0.427451}%
\pgfsetfillcolor{currentfill}%
\pgfsetlinewidth{0.100375pt}%
\definecolor{currentstroke}{rgb}{0.266667,0.266667,0.266667}%
\pgfsetstrokecolor{currentstroke}%
\pgfsetdash{}{0pt}%
\pgfpathmoveto{\pgfqpoint{4.569764in}{0.638889in}}%
\pgfpathlineto{\pgfqpoint{4.595515in}{0.638889in}}%
\pgfpathlineto{\pgfqpoint{4.595515in}{1.045094in}}%
\pgfpathlineto{\pgfqpoint{4.569764in}{1.045094in}}%
\pgfpathlineto{\pgfqpoint{4.569764in}{0.638889in}}%
\pgfpathclose%
\pgfusepath{stroke,fill}%
\end{pgfscope}%
\begin{pgfscope}%
\pgfpathrectangle{\pgfqpoint{0.781250in}{0.638889in}}{\pgfqpoint{4.843750in}{2.172222in}}%
\pgfusepath{clip}%
\pgfsetbuttcap%
\pgfsetmiterjoin%
\definecolor{currentfill}{rgb}{0.227451,0.192157,0.427451}%
\pgfsetfillcolor{currentfill}%
\pgfsetlinewidth{0.100375pt}%
\definecolor{currentstroke}{rgb}{0.266667,0.266667,0.266667}%
\pgfsetstrokecolor{currentstroke}%
\pgfsetdash{}{0pt}%
\pgfpathmoveto{\pgfqpoint{4.606551in}{0.638889in}}%
\pgfpathlineto{\pgfqpoint{4.632302in}{0.638889in}}%
\pgfpathlineto{\pgfqpoint{4.632302in}{0.872248in}}%
\pgfpathlineto{\pgfqpoint{4.606551in}{0.872248in}}%
\pgfpathlineto{\pgfqpoint{4.606551in}{0.638889in}}%
\pgfpathclose%
\pgfusepath{stroke,fill}%
\end{pgfscope}%
\begin{pgfscope}%
\pgfpathrectangle{\pgfqpoint{0.781250in}{0.638889in}}{\pgfqpoint{4.843750in}{2.172222in}}%
\pgfusepath{clip}%
\pgfsetbuttcap%
\pgfsetmiterjoin%
\definecolor{currentfill}{rgb}{0.227451,0.192157,0.427451}%
\pgfsetfillcolor{currentfill}%
\pgfsetlinewidth{0.100375pt}%
\definecolor{currentstroke}{rgb}{0.266667,0.266667,0.266667}%
\pgfsetstrokecolor{currentstroke}%
\pgfsetdash{}{0pt}%
\pgfpathmoveto{\pgfqpoint{4.643338in}{0.638889in}}%
\pgfpathlineto{\pgfqpoint{4.669089in}{0.638889in}}%
\pgfpathlineto{\pgfqpoint{4.669089in}{0.814529in}}%
\pgfpathlineto{\pgfqpoint{4.643338in}{0.814529in}}%
\pgfpathlineto{\pgfqpoint{4.643338in}{0.638889in}}%
\pgfpathclose%
\pgfusepath{stroke,fill}%
\end{pgfscope}%
\begin{pgfscope}%
\pgfpathrectangle{\pgfqpoint{0.781250in}{0.638889in}}{\pgfqpoint{4.843750in}{2.172222in}}%
\pgfusepath{clip}%
\pgfsetbuttcap%
\pgfsetmiterjoin%
\definecolor{currentfill}{rgb}{0.227451,0.192157,0.427451}%
\pgfsetfillcolor{currentfill}%
\pgfsetlinewidth{0.100375pt}%
\definecolor{currentstroke}{rgb}{0.266667,0.266667,0.266667}%
\pgfsetstrokecolor{currentstroke}%
\pgfsetdash{}{0pt}%
\pgfpathmoveto{\pgfqpoint{4.680125in}{0.638889in}}%
\pgfpathlineto{\pgfqpoint{4.705876in}{0.638889in}}%
\pgfpathlineto{\pgfqpoint{4.705876in}{0.912899in}}%
\pgfpathlineto{\pgfqpoint{4.680125in}{0.912899in}}%
\pgfpathlineto{\pgfqpoint{4.680125in}{0.638889in}}%
\pgfpathclose%
\pgfusepath{stroke,fill}%
\end{pgfscope}%
\begin{pgfscope}%
\pgfpathrectangle{\pgfqpoint{0.781250in}{0.638889in}}{\pgfqpoint{4.843750in}{2.172222in}}%
\pgfusepath{clip}%
\pgfsetbuttcap%
\pgfsetmiterjoin%
\definecolor{currentfill}{rgb}{0.164706,0.615686,0.560784}%
\pgfsetfillcolor{currentfill}%
\pgfsetlinewidth{0.100375pt}%
\definecolor{currentstroke}{rgb}{0.266667,0.266667,0.266667}%
\pgfsetstrokecolor{currentstroke}%
\pgfsetdash{}{0pt}%
\pgfpathmoveto{\pgfqpoint{4.716912in}{0.638889in}}%
\pgfpathlineto{\pgfqpoint{4.742663in}{0.638889in}}%
\pgfpathlineto{\pgfqpoint{4.742663in}{0.880006in}}%
\pgfpathlineto{\pgfqpoint{4.716912in}{0.880006in}}%
\pgfpathlineto{\pgfqpoint{4.716912in}{0.638889in}}%
\pgfpathclose%
\pgfusepath{stroke,fill}%
\end{pgfscope}%
\begin{pgfscope}%
\pgfpathrectangle{\pgfqpoint{0.781250in}{0.638889in}}{\pgfqpoint{4.843750in}{2.172222in}}%
\pgfusepath{clip}%
\pgfsetbuttcap%
\pgfsetmiterjoin%
\definecolor{currentfill}{rgb}{0.164706,0.615686,0.560784}%
\pgfsetfillcolor{currentfill}%
\pgfsetlinewidth{0.100375pt}%
\definecolor{currentstroke}{rgb}{0.266667,0.266667,0.266667}%
\pgfsetstrokecolor{currentstroke}%
\pgfsetdash{}{0pt}%
\pgfpathmoveto{\pgfqpoint{4.753699in}{0.638889in}}%
\pgfpathlineto{\pgfqpoint{4.779450in}{0.638889in}}%
\pgfpathlineto{\pgfqpoint{4.779450in}{1.648972in}}%
\pgfpathlineto{\pgfqpoint{4.753699in}{1.648972in}}%
\pgfpathlineto{\pgfqpoint{4.753699in}{0.638889in}}%
\pgfpathclose%
\pgfusepath{stroke,fill}%
\end{pgfscope}%
\begin{pgfscope}%
\pgfpathrectangle{\pgfqpoint{0.781250in}{0.638889in}}{\pgfqpoint{4.843750in}{2.172222in}}%
\pgfusepath{clip}%
\pgfsetbuttcap%
\pgfsetmiterjoin%
\definecolor{currentfill}{rgb}{0.164706,0.615686,0.560784}%
\pgfsetfillcolor{currentfill}%
\pgfsetlinewidth{0.100375pt}%
\definecolor{currentstroke}{rgb}{0.266667,0.266667,0.266667}%
\pgfsetstrokecolor{currentstroke}%
\pgfsetdash{}{0pt}%
\pgfpathmoveto{\pgfqpoint{4.790486in}{0.638889in}}%
\pgfpathlineto{\pgfqpoint{4.816237in}{0.638889in}}%
\pgfpathlineto{\pgfqpoint{4.816237in}{0.754327in}}%
\pgfpathlineto{\pgfqpoint{4.790486in}{0.754327in}}%
\pgfpathlineto{\pgfqpoint{4.790486in}{0.638889in}}%
\pgfpathclose%
\pgfusepath{stroke,fill}%
\end{pgfscope}%
\begin{pgfscope}%
\pgfpathrectangle{\pgfqpoint{0.781250in}{0.638889in}}{\pgfqpoint{4.843750in}{2.172222in}}%
\pgfusepath{clip}%
\pgfsetbuttcap%
\pgfsetmiterjoin%
\definecolor{currentfill}{rgb}{0.164706,0.615686,0.560784}%
\pgfsetfillcolor{currentfill}%
\pgfsetlinewidth{0.100375pt}%
\definecolor{currentstroke}{rgb}{0.266667,0.266667,0.266667}%
\pgfsetstrokecolor{currentstroke}%
\pgfsetdash{}{0pt}%
\pgfpathmoveto{\pgfqpoint{4.827273in}{0.638889in}}%
\pgfpathlineto{\pgfqpoint{4.853024in}{0.638889in}}%
\pgfpathlineto{\pgfqpoint{4.853024in}{1.151223in}}%
\pgfpathlineto{\pgfqpoint{4.827273in}{1.151223in}}%
\pgfpathlineto{\pgfqpoint{4.827273in}{0.638889in}}%
\pgfpathclose%
\pgfusepath{stroke,fill}%
\end{pgfscope}%
\begin{pgfscope}%
\pgfpathrectangle{\pgfqpoint{0.781250in}{0.638889in}}{\pgfqpoint{4.843750in}{2.172222in}}%
\pgfusepath{clip}%
\pgfsetbuttcap%
\pgfsetmiterjoin%
\definecolor{currentfill}{rgb}{0.227451,0.192157,0.427451}%
\pgfsetfillcolor{currentfill}%
\pgfsetlinewidth{0.100375pt}%
\definecolor{currentstroke}{rgb}{0.266667,0.266667,0.266667}%
\pgfsetstrokecolor{currentstroke}%
\pgfsetdash{}{0pt}%
\pgfpathmoveto{\pgfqpoint{4.864060in}{0.638889in}}%
\pgfpathlineto{\pgfqpoint{4.889811in}{0.638889in}}%
\pgfpathlineto{\pgfqpoint{4.889811in}{0.861076in}}%
\pgfpathlineto{\pgfqpoint{4.864060in}{0.861076in}}%
\pgfpathlineto{\pgfqpoint{4.864060in}{0.638889in}}%
\pgfpathclose%
\pgfusepath{stroke,fill}%
\end{pgfscope}%
\begin{pgfscope}%
\pgfpathrectangle{\pgfqpoint{0.781250in}{0.638889in}}{\pgfqpoint{4.843750in}{2.172222in}}%
\pgfusepath{clip}%
\pgfsetbuttcap%
\pgfsetmiterjoin%
\definecolor{currentfill}{rgb}{0.227451,0.192157,0.427451}%
\pgfsetfillcolor{currentfill}%
\pgfsetlinewidth{0.100375pt}%
\definecolor{currentstroke}{rgb}{0.266667,0.266667,0.266667}%
\pgfsetstrokecolor{currentstroke}%
\pgfsetdash{}{0pt}%
\pgfpathmoveto{\pgfqpoint{4.900847in}{0.638889in}}%
\pgfpathlineto{\pgfqpoint{4.926598in}{0.638889in}}%
\pgfpathlineto{\pgfqpoint{4.926598in}{0.795289in}}%
\pgfpathlineto{\pgfqpoint{4.900847in}{0.795289in}}%
\pgfpathlineto{\pgfqpoint{4.900847in}{0.638889in}}%
\pgfpathclose%
\pgfusepath{stroke,fill}%
\end{pgfscope}%
\begin{pgfscope}%
\pgfpathrectangle{\pgfqpoint{0.781250in}{0.638889in}}{\pgfqpoint{4.843750in}{2.172222in}}%
\pgfusepath{clip}%
\pgfsetbuttcap%
\pgfsetmiterjoin%
\definecolor{currentfill}{rgb}{0.227451,0.192157,0.427451}%
\pgfsetfillcolor{currentfill}%
\pgfsetlinewidth{0.100375pt}%
\definecolor{currentstroke}{rgb}{0.266667,0.266667,0.266667}%
\pgfsetstrokecolor{currentstroke}%
\pgfsetdash{}{0pt}%
\pgfpathmoveto{\pgfqpoint{4.937634in}{0.638889in}}%
\pgfpathlineto{\pgfqpoint{4.963385in}{0.638889in}}%
\pgfpathlineto{\pgfqpoint{4.963385in}{0.747500in}}%
\pgfpathlineto{\pgfqpoint{4.937634in}{0.747500in}}%
\pgfpathlineto{\pgfqpoint{4.937634in}{0.638889in}}%
\pgfpathclose%
\pgfusepath{stroke,fill}%
\end{pgfscope}%
\begin{pgfscope}%
\pgfpathrectangle{\pgfqpoint{0.781250in}{0.638889in}}{\pgfqpoint{4.843750in}{2.172222in}}%
\pgfusepath{clip}%
\pgfsetbuttcap%
\pgfsetmiterjoin%
\definecolor{currentfill}{rgb}{0.227451,0.192157,0.427451}%
\pgfsetfillcolor{currentfill}%
\pgfsetlinewidth{0.100375pt}%
\definecolor{currentstroke}{rgb}{0.266667,0.266667,0.266667}%
\pgfsetstrokecolor{currentstroke}%
\pgfsetdash{}{0pt}%
\pgfpathmoveto{\pgfqpoint{4.974421in}{0.638889in}}%
\pgfpathlineto{\pgfqpoint{5.000172in}{0.638889in}}%
\pgfpathlineto{\pgfqpoint{5.000172in}{0.777911in}}%
\pgfpathlineto{\pgfqpoint{4.974421in}{0.777911in}}%
\pgfpathlineto{\pgfqpoint{4.974421in}{0.638889in}}%
\pgfpathclose%
\pgfusepath{stroke,fill}%
\end{pgfscope}%
\begin{pgfscope}%
\pgfpathrectangle{\pgfqpoint{0.781250in}{0.638889in}}{\pgfqpoint{4.843750in}{2.172222in}}%
\pgfusepath{clip}%
\pgfsetbuttcap%
\pgfsetmiterjoin%
\definecolor{currentfill}{rgb}{0.227451,0.192157,0.427451}%
\pgfsetfillcolor{currentfill}%
\pgfsetlinewidth{0.100375pt}%
\definecolor{currentstroke}{rgb}{0.266667,0.266667,0.266667}%
\pgfsetstrokecolor{currentstroke}%
\pgfsetdash{}{0pt}%
\pgfpathmoveto{\pgfqpoint{5.011208in}{0.638889in}}%
\pgfpathlineto{\pgfqpoint{5.036959in}{0.638889in}}%
\pgfpathlineto{\pgfqpoint{5.036959in}{0.749362in}}%
\pgfpathlineto{\pgfqpoint{5.011208in}{0.749362in}}%
\pgfpathlineto{\pgfqpoint{5.011208in}{0.638889in}}%
\pgfpathclose%
\pgfusepath{stroke,fill}%
\end{pgfscope}%
\begin{pgfscope}%
\pgfpathrectangle{\pgfqpoint{0.781250in}{0.638889in}}{\pgfqpoint{4.843750in}{2.172222in}}%
\pgfusepath{clip}%
\pgfsetbuttcap%
\pgfsetmiterjoin%
\definecolor{currentfill}{rgb}{0.227451,0.192157,0.427451}%
\pgfsetfillcolor{currentfill}%
\pgfsetlinewidth{0.100375pt}%
\definecolor{currentstroke}{rgb}{0.266667,0.266667,0.266667}%
\pgfsetstrokecolor{currentstroke}%
\pgfsetdash{}{0pt}%
\pgfpathmoveto{\pgfqpoint{5.047995in}{0.638889in}}%
\pgfpathlineto{\pgfqpoint{5.073746in}{0.638889in}}%
\pgfpathlineto{\pgfqpoint{5.073746in}{0.772325in}}%
\pgfpathlineto{\pgfqpoint{5.047995in}{0.772325in}}%
\pgfpathlineto{\pgfqpoint{5.047995in}{0.638889in}}%
\pgfpathclose%
\pgfusepath{stroke,fill}%
\end{pgfscope}%
\begin{pgfscope}%
\pgfpathrectangle{\pgfqpoint{0.781250in}{0.638889in}}{\pgfqpoint{4.843750in}{2.172222in}}%
\pgfusepath{clip}%
\pgfsetbuttcap%
\pgfsetmiterjoin%
\definecolor{currentfill}{rgb}{0.227451,0.192157,0.427451}%
\pgfsetfillcolor{currentfill}%
\pgfsetlinewidth{0.100375pt}%
\definecolor{currentstroke}{rgb}{0.266667,0.266667,0.266667}%
\pgfsetstrokecolor{currentstroke}%
\pgfsetdash{}{0pt}%
\pgfpathmoveto{\pgfqpoint{5.084782in}{0.638889in}}%
\pgfpathlineto{\pgfqpoint{5.110533in}{0.638889in}}%
\pgfpathlineto{\pgfqpoint{5.110533in}{0.833148in}}%
\pgfpathlineto{\pgfqpoint{5.084782in}{0.833148in}}%
\pgfpathlineto{\pgfqpoint{5.084782in}{0.638889in}}%
\pgfpathclose%
\pgfusepath{stroke,fill}%
\end{pgfscope}%
\begin{pgfscope}%
\pgfpathrectangle{\pgfqpoint{0.781250in}{0.638889in}}{\pgfqpoint{4.843750in}{2.172222in}}%
\pgfusepath{clip}%
\pgfsetbuttcap%
\pgfsetmiterjoin%
\definecolor{currentfill}{rgb}{0.227451,0.192157,0.427451}%
\pgfsetfillcolor{currentfill}%
\pgfsetlinewidth{0.100375pt}%
\definecolor{currentstroke}{rgb}{0.266667,0.266667,0.266667}%
\pgfsetstrokecolor{currentstroke}%
\pgfsetdash{}{0pt}%
\pgfpathmoveto{\pgfqpoint{5.121569in}{0.638889in}}%
\pgfpathlineto{\pgfqpoint{5.147320in}{0.638889in}}%
\pgfpathlineto{\pgfqpoint{5.147320in}{0.896142in}}%
\pgfpathlineto{\pgfqpoint{5.121569in}{0.896142in}}%
\pgfpathlineto{\pgfqpoint{5.121569in}{0.638889in}}%
\pgfpathclose%
\pgfusepath{stroke,fill}%
\end{pgfscope}%
\begin{pgfscope}%
\pgfpathrectangle{\pgfqpoint{0.781250in}{0.638889in}}{\pgfqpoint{4.843750in}{2.172222in}}%
\pgfusepath{clip}%
\pgfsetbuttcap%
\pgfsetmiterjoin%
\definecolor{currentfill}{rgb}{0.164706,0.615686,0.560784}%
\pgfsetfillcolor{currentfill}%
\pgfsetlinewidth{0.100375pt}%
\definecolor{currentstroke}{rgb}{0.266667,0.266667,0.266667}%
\pgfsetstrokecolor{currentstroke}%
\pgfsetdash{}{0pt}%
\pgfpathmoveto{\pgfqpoint{5.158356in}{0.638889in}}%
\pgfpathlineto{\pgfqpoint{5.184107in}{0.638889in}}%
\pgfpathlineto{\pgfqpoint{5.184107in}{1.624457in}}%
\pgfpathlineto{\pgfqpoint{5.158356in}{1.624457in}}%
\pgfpathlineto{\pgfqpoint{5.158356in}{0.638889in}}%
\pgfpathclose%
\pgfusepath{stroke,fill}%
\end{pgfscope}%
\begin{pgfscope}%
\pgfpathrectangle{\pgfqpoint{0.781250in}{0.638889in}}{\pgfqpoint{4.843750in}{2.172222in}}%
\pgfusepath{clip}%
\pgfsetbuttcap%
\pgfsetmiterjoin%
\definecolor{currentfill}{rgb}{0.164706,0.615686,0.560784}%
\pgfsetfillcolor{currentfill}%
\pgfsetlinewidth{0.100375pt}%
\definecolor{currentstroke}{rgb}{0.266667,0.266667,0.266667}%
\pgfsetstrokecolor{currentstroke}%
\pgfsetdash{}{0pt}%
\pgfpathmoveto{\pgfqpoint{5.195143in}{0.638889in}}%
\pgfpathlineto{\pgfqpoint{5.220894in}{0.638889in}}%
\pgfpathlineto{\pgfqpoint{5.220894in}{1.287452in}}%
\pgfpathlineto{\pgfqpoint{5.195143in}{1.287452in}}%
\pgfpathlineto{\pgfqpoint{5.195143in}{0.638889in}}%
\pgfpathclose%
\pgfusepath{stroke,fill}%
\end{pgfscope}%
\begin{pgfscope}%
\pgfpathrectangle{\pgfqpoint{0.781250in}{0.638889in}}{\pgfqpoint{4.843750in}{2.172222in}}%
\pgfusepath{clip}%
\pgfsetbuttcap%
\pgfsetmiterjoin%
\definecolor{currentfill}{rgb}{0.164706,0.615686,0.560784}%
\pgfsetfillcolor{currentfill}%
\pgfsetlinewidth{0.100375pt}%
\definecolor{currentstroke}{rgb}{0.266667,0.266667,0.266667}%
\pgfsetstrokecolor{currentstroke}%
\pgfsetdash{}{0pt}%
\pgfpathmoveto{\pgfqpoint{5.231930in}{0.638889in}}%
\pgfpathlineto{\pgfqpoint{5.257681in}{0.638889in}}%
\pgfpathlineto{\pgfqpoint{5.257681in}{1.206770in}}%
\pgfpathlineto{\pgfqpoint{5.231930in}{1.206770in}}%
\pgfpathlineto{\pgfqpoint{5.231930in}{0.638889in}}%
\pgfpathclose%
\pgfusepath{stroke,fill}%
\end{pgfscope}%
\begin{pgfscope}%
\pgfpathrectangle{\pgfqpoint{0.781250in}{0.638889in}}{\pgfqpoint{4.843750in}{2.172222in}}%
\pgfusepath{clip}%
\pgfsetbuttcap%
\pgfsetmiterjoin%
\definecolor{currentfill}{rgb}{0.164706,0.615686,0.560784}%
\pgfsetfillcolor{currentfill}%
\pgfsetlinewidth{0.100375pt}%
\definecolor{currentstroke}{rgb}{0.266667,0.266667,0.266667}%
\pgfsetstrokecolor{currentstroke}%
\pgfsetdash{}{0pt}%
\pgfpathmoveto{\pgfqpoint{5.268717in}{0.638889in}}%
\pgfpathlineto{\pgfqpoint{5.294468in}{0.638889in}}%
\pgfpathlineto{\pgfqpoint{5.294468in}{1.719104in}}%
\pgfpathlineto{\pgfqpoint{5.268717in}{1.719104in}}%
\pgfpathlineto{\pgfqpoint{5.268717in}{0.638889in}}%
\pgfpathclose%
\pgfusepath{stroke,fill}%
\end{pgfscope}%
\begin{pgfscope}%
\pgfpathrectangle{\pgfqpoint{0.781250in}{0.638889in}}{\pgfqpoint{4.843750in}{2.172222in}}%
\pgfusepath{clip}%
\pgfsetbuttcap%
\pgfsetmiterjoin%
\definecolor{currentfill}{rgb}{0.227451,0.192157,0.427451}%
\pgfsetfillcolor{currentfill}%
\pgfsetlinewidth{0.100375pt}%
\definecolor{currentstroke}{rgb}{0.266667,0.266667,0.266667}%
\pgfsetstrokecolor{currentstroke}%
\pgfsetdash{}{0pt}%
\pgfpathmoveto{\pgfqpoint{5.305505in}{0.638889in}}%
\pgfpathlineto{\pgfqpoint{5.331255in}{0.638889in}}%
\pgfpathlineto{\pgfqpoint{5.331255in}{0.920347in}}%
\pgfpathlineto{\pgfqpoint{5.305505in}{0.920347in}}%
\pgfpathlineto{\pgfqpoint{5.305505in}{0.638889in}}%
\pgfpathclose%
\pgfusepath{stroke,fill}%
\end{pgfscope}%
\begin{pgfscope}%
\pgfpathrectangle{\pgfqpoint{0.781250in}{0.638889in}}{\pgfqpoint{4.843750in}{2.172222in}}%
\pgfusepath{clip}%
\pgfsetbuttcap%
\pgfsetmiterjoin%
\definecolor{currentfill}{rgb}{0.227451,0.192157,0.427451}%
\pgfsetfillcolor{currentfill}%
\pgfsetlinewidth{0.100375pt}%
\definecolor{currentstroke}{rgb}{0.266667,0.266667,0.266667}%
\pgfsetstrokecolor{currentstroke}%
\pgfsetdash{}{0pt}%
\pgfpathmoveto{\pgfqpoint{5.342292in}{0.638889in}}%
\pgfpathlineto{\pgfqpoint{5.368043in}{0.638889in}}%
\pgfpathlineto{\pgfqpoint{5.368043in}{0.705607in}}%
\pgfpathlineto{\pgfqpoint{5.342292in}{0.705607in}}%
\pgfpathlineto{\pgfqpoint{5.342292in}{0.638889in}}%
\pgfpathclose%
\pgfusepath{stroke,fill}%
\end{pgfscope}%
\begin{pgfscope}%
\pgfpathrectangle{\pgfqpoint{0.781250in}{0.638889in}}{\pgfqpoint{4.843750in}{2.172222in}}%
\pgfusepath{clip}%
\pgfsetbuttcap%
\pgfsetmiterjoin%
\definecolor{currentfill}{rgb}{0.227451,0.192157,0.427451}%
\pgfsetfillcolor{currentfill}%
\pgfsetlinewidth{0.100375pt}%
\definecolor{currentstroke}{rgb}{0.266667,0.266667,0.266667}%
\pgfsetstrokecolor{currentstroke}%
\pgfsetdash{}{0pt}%
\pgfpathmoveto{\pgfqpoint{5.379079in}{0.638889in}}%
\pgfpathlineto{\pgfqpoint{5.404830in}{0.638889in}}%
\pgfpathlineto{\pgfqpoint{5.404830in}{0.645406in}}%
\pgfpathlineto{\pgfqpoint{5.379079in}{0.645406in}}%
\pgfpathlineto{\pgfqpoint{5.379079in}{0.638889in}}%
\pgfpathclose%
\pgfusepath{stroke,fill}%
\end{pgfscope}%
\begin{pgfscope}%
\definecolor{textcolor}{rgb}{0.333333,0.333333,0.333333}%
\pgfsetstrokecolor{textcolor}%
\pgfsetfillcolor{textcolor}%
\pgftext[x=1.875000in,y=0.319444in,,top]{\color{textcolor}{\ifdefined\pdftexversion\else\setmainfont{NanumMyeongjo}\rmfamily\fi\fontsize{9.000000}{10.800000}\selectfont\catcode`\^=\active\def^{\ifmmode\sp\else\^{}\fi}\catcode`\%=\active\def%{\%}출처: 기상자료개방포털 자료 기반 저자 작성}}%
\end{pgfscope}%
\begin{pgfscope}%
\definecolor{textcolor}{rgb}{0.333333,0.333333,0.333333}%
\pgfsetstrokecolor{textcolor}%
\pgfsetfillcolor{textcolor}%
\pgftext[x=5.312500in,y=2.970833in,,top]{\color{textcolor}{\ifdefined\pdftexversion\else\setmainfont{NanumMyeongjo}\rmfamily\fi\fontsize{9.000000}{10.800000}\selectfont\catcode`\^=\active\def^{\ifmmode\sp\else\^{}\fi}\catcode`\%=\active\def%{\%}(단위: mm)}}%
\end{pgfscope}%
\end{pgfpicture}%
\makeatother%
\endgroup%
}
\end{center}
}


\slide
{\maintitle}
{\chapterfive}
{소결}{

\begin{tcolorbox}[colback=white, colframe=black, boxrule=0.8pt, rounded corners]
\begin{itemize}
    \item 논은 일반적으로 지대가 낮고 물빠짐이 좋지 않기에 \\ 적지에 논콩을 재배하는 것이 중요함
    \item 우리나라는 콩의 생장에 중요한 늦여름에 \\강수량이 몰려있기에 배수 관리에 각별히 신경써야 함
    \item 최근 발생한 초여름과 초가을 장마로 \\논콩 침수 피해가 큰 규모로 발생함  
    \item 논콩 생산 농가는 기상이변에 철저히 대비해야하며,\\ 정부 또한 재해보장제도를 완비해야 함
\end{itemize}
\end{tcolorbox}

}



