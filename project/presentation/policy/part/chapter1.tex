

\slide
{\maintitle}
{\chapterone}
{식량 자급률}{

\begin{center}
    \hspace*{-30pt}\resizebox{1.2\textwidth}{!}{%% Creator: Matplotlib, PGF backend
%%
%% To include the figure in your LaTeX document, write
%%   \input{<filename>.pgf}
%%
%% Make sure the required packages are loaded in your preamble
%%   \usepackage{pgf}
%%
%% Also ensure that all the required font packages are loaded; for instance,
%% the lmodern package is sometimes necessary when using math font.
%%   \usepackage{lmodern}
%%
%% Figures using additional raster images can only be included by \input if
%% they are in the same directory as the main LaTeX file. For loading figures
%% from other directories you can use the `import` package
%%   \usepackage{import}
%%
%% and then include the figures with
%%   \import{<path to file>}{<filename>.pgf}
%%
%% Matplotlib used the following preamble
%%   \def\mathdefault#1{#1}
%%   \everymath=\expandafter{\the\everymath\displaystyle}
%%   \IfFileExists{scrextend.sty}{
%%     \usepackage[fontsize=9.000000pt]{scrextend}
%%   }{
%%     \renewcommand{\normalsize}{\fontsize{9.000000}{10.800000}\selectfont}
%%     \normalsize
%%   }
%%   
%%   \ifdefined\pdftexversion\else  % non-pdftex case.
%%     \usepackage{fontspec}
%%     \setmainfont{DejaVuSerif.ttf}[Path=\detokenize{/home/user/.cache/pypoetry/virtualenvs/graph-KASAOWVd-py3.12/lib/python3.12/site-packages/matplotlib/mpl-data/fonts/ttf/}]
%%     \setsansfont{DejaVuSans.ttf}[Path=\detokenize{/home/user/.cache/pypoetry/virtualenvs/graph-KASAOWVd-py3.12/lib/python3.12/site-packages/matplotlib/mpl-data/fonts/ttf/}]
%%     \setmonofont{DejaVuSansMono.ttf}[Path=\detokenize{/home/user/.cache/pypoetry/virtualenvs/graph-KASAOWVd-py3.12/lib/python3.12/site-packages/matplotlib/mpl-data/fonts/ttf/}]
%%   \fi
%%   \makeatletter\@ifpackageloaded{underscore}{}{\usepackage[strings]{underscore}}\makeatother
%%
\begingroup%
\makeatletter%
\begin{pgfpicture}%
\pgfpathrectangle{\pgfpointorigin}{\pgfqpoint{5.833333in}{3.194444in}}%
\pgfusepath{use as bounding box, clip}%
\begin{pgfscope}%
\pgfsetbuttcap%
\pgfsetmiterjoin%
\definecolor{currentfill}{rgb}{1.000000,1.000000,1.000000}%
\pgfsetfillcolor{currentfill}%
\pgfsetlinewidth{0.000000pt}%
\definecolor{currentstroke}{rgb}{1.000000,1.000000,1.000000}%
\pgfsetstrokecolor{currentstroke}%
\pgfsetdash{}{0pt}%
\pgfpathmoveto{\pgfqpoint{0.000000in}{0.000000in}}%
\pgfpathlineto{\pgfqpoint{5.833333in}{0.000000in}}%
\pgfpathlineto{\pgfqpoint{5.833333in}{3.194444in}}%
\pgfpathlineto{\pgfqpoint{0.000000in}{3.194444in}}%
\pgfpathlineto{\pgfqpoint{0.000000in}{0.000000in}}%
\pgfpathclose%
\pgfusepath{fill}%
\end{pgfscope}%
\begin{pgfscope}%
\pgfsetbuttcap%
\pgfsetmiterjoin%
\definecolor{currentfill}{rgb}{1.000000,1.000000,1.000000}%
\pgfsetfillcolor{currentfill}%
\pgfsetlinewidth{0.000000pt}%
\definecolor{currentstroke}{rgb}{0.000000,0.000000,0.000000}%
\pgfsetstrokecolor{currentstroke}%
\pgfsetstrokeopacity{0.000000}%
\pgfsetdash{}{0pt}%
\pgfpathmoveto{\pgfqpoint{0.729167in}{0.638889in}}%
\pgfpathlineto{\pgfqpoint{4.666667in}{0.638889in}}%
\pgfpathlineto{\pgfqpoint{4.666667in}{2.811111in}}%
\pgfpathlineto{\pgfqpoint{0.729167in}{2.811111in}}%
\pgfpathlineto{\pgfqpoint{0.729167in}{0.638889in}}%
\pgfpathclose%
\pgfusepath{fill}%
\end{pgfscope}%
\begin{pgfscope}%
\pgfsetbuttcap%
\pgfsetroundjoin%
\definecolor{currentfill}{rgb}{0.000000,0.000000,0.000000}%
\pgfsetfillcolor{currentfill}%
\pgfsetlinewidth{0.752812pt}%
\definecolor{currentstroke}{rgb}{0.000000,0.000000,0.000000}%
\pgfsetstrokecolor{currentstroke}%
\pgfsetdash{}{0pt}%
\pgfsys@defobject{currentmarker}{\pgfqpoint{0.000000in}{-0.013889in}}{\pgfqpoint{0.000000in}{0.000000in}}{%
\pgfpathmoveto{\pgfqpoint{0.000000in}{0.000000in}}%
\pgfpathlineto{\pgfqpoint{0.000000in}{-0.013889in}}%
\pgfusepath{stroke,fill}%
}%
\begin{pgfscope}%
\pgfsys@transformshift{0.908144in}{0.638889in}%
\pgfsys@useobject{currentmarker}{}%
\end{pgfscope}%
\end{pgfscope}%
\begin{pgfscope}%
\definecolor{textcolor}{rgb}{0.000000,0.000000,0.000000}%
\pgfsetstrokecolor{textcolor}%
\pgfsetfillcolor{textcolor}%
\pgftext[x=0.838271in, y=0.315885in, left, base,rotate=45.000000]{\color{textcolor}{\ifdefined\pdftexversion\else\setmainfont{NanumMyeongjo}\rmfamily\fi\fontsize{9.000000}{10.800000}\selectfont\catcode`\^=\active\def^{\ifmmode\sp\else\^{}\fi}\catcode`\%=\active\def%{\%}2010}}%
\end{pgfscope}%
\begin{pgfscope}%
\pgfsetbuttcap%
\pgfsetroundjoin%
\definecolor{currentfill}{rgb}{0.000000,0.000000,0.000000}%
\pgfsetfillcolor{currentfill}%
\pgfsetlinewidth{0.752812pt}%
\definecolor{currentstroke}{rgb}{0.000000,0.000000,0.000000}%
\pgfsetstrokecolor{currentstroke}%
\pgfsetdash{}{0pt}%
\pgfsys@defobject{currentmarker}{\pgfqpoint{0.000000in}{-0.013889in}}{\pgfqpoint{0.000000in}{0.000000in}}{%
\pgfpathmoveto{\pgfqpoint{0.000000in}{0.000000in}}%
\pgfpathlineto{\pgfqpoint{0.000000in}{-0.013889in}}%
\pgfusepath{stroke,fill}%
}%
\begin{pgfscope}%
\pgfsys@transformshift{1.183494in}{0.638889in}%
\pgfsys@useobject{currentmarker}{}%
\end{pgfscope}%
\end{pgfscope}%
\begin{pgfscope}%
\definecolor{textcolor}{rgb}{0.000000,0.000000,0.000000}%
\pgfsetstrokecolor{textcolor}%
\pgfsetfillcolor{textcolor}%
\pgftext[x=1.113620in, y=0.315885in, left, base,rotate=45.000000]{\color{textcolor}{\ifdefined\pdftexversion\else\setmainfont{NanumMyeongjo}\rmfamily\fi\fontsize{9.000000}{10.800000}\selectfont\catcode`\^=\active\def^{\ifmmode\sp\else\^{}\fi}\catcode`\%=\active\def%{\%}2011}}%
\end{pgfscope}%
\begin{pgfscope}%
\pgfsetbuttcap%
\pgfsetroundjoin%
\definecolor{currentfill}{rgb}{0.000000,0.000000,0.000000}%
\pgfsetfillcolor{currentfill}%
\pgfsetlinewidth{0.752812pt}%
\definecolor{currentstroke}{rgb}{0.000000,0.000000,0.000000}%
\pgfsetstrokecolor{currentstroke}%
\pgfsetdash{}{0pt}%
\pgfsys@defobject{currentmarker}{\pgfqpoint{0.000000in}{-0.013889in}}{\pgfqpoint{0.000000in}{0.000000in}}{%
\pgfpathmoveto{\pgfqpoint{0.000000in}{0.000000in}}%
\pgfpathlineto{\pgfqpoint{0.000000in}{-0.013889in}}%
\pgfusepath{stroke,fill}%
}%
\begin{pgfscope}%
\pgfsys@transformshift{1.458843in}{0.638889in}%
\pgfsys@useobject{currentmarker}{}%
\end{pgfscope}%
\end{pgfscope}%
\begin{pgfscope}%
\definecolor{textcolor}{rgb}{0.000000,0.000000,0.000000}%
\pgfsetstrokecolor{textcolor}%
\pgfsetfillcolor{textcolor}%
\pgftext[x=1.388970in, y=0.315885in, left, base,rotate=45.000000]{\color{textcolor}{\ifdefined\pdftexversion\else\setmainfont{NanumMyeongjo}\rmfamily\fi\fontsize{9.000000}{10.800000}\selectfont\catcode`\^=\active\def^{\ifmmode\sp\else\^{}\fi}\catcode`\%=\active\def%{\%}2012}}%
\end{pgfscope}%
\begin{pgfscope}%
\pgfsetbuttcap%
\pgfsetroundjoin%
\definecolor{currentfill}{rgb}{0.000000,0.000000,0.000000}%
\pgfsetfillcolor{currentfill}%
\pgfsetlinewidth{0.752812pt}%
\definecolor{currentstroke}{rgb}{0.000000,0.000000,0.000000}%
\pgfsetstrokecolor{currentstroke}%
\pgfsetdash{}{0pt}%
\pgfsys@defobject{currentmarker}{\pgfqpoint{0.000000in}{-0.013889in}}{\pgfqpoint{0.000000in}{0.000000in}}{%
\pgfpathmoveto{\pgfqpoint{0.000000in}{0.000000in}}%
\pgfpathlineto{\pgfqpoint{0.000000in}{-0.013889in}}%
\pgfusepath{stroke,fill}%
}%
\begin{pgfscope}%
\pgfsys@transformshift{1.734193in}{0.638889in}%
\pgfsys@useobject{currentmarker}{}%
\end{pgfscope}%
\end{pgfscope}%
\begin{pgfscope}%
\definecolor{textcolor}{rgb}{0.000000,0.000000,0.000000}%
\pgfsetstrokecolor{textcolor}%
\pgfsetfillcolor{textcolor}%
\pgftext[x=1.664319in, y=0.315885in, left, base,rotate=45.000000]{\color{textcolor}{\ifdefined\pdftexversion\else\setmainfont{NanumMyeongjo}\rmfamily\fi\fontsize{9.000000}{10.800000}\selectfont\catcode`\^=\active\def^{\ifmmode\sp\else\^{}\fi}\catcode`\%=\active\def%{\%}2013}}%
\end{pgfscope}%
\begin{pgfscope}%
\pgfsetbuttcap%
\pgfsetroundjoin%
\definecolor{currentfill}{rgb}{0.000000,0.000000,0.000000}%
\pgfsetfillcolor{currentfill}%
\pgfsetlinewidth{0.752812pt}%
\definecolor{currentstroke}{rgb}{0.000000,0.000000,0.000000}%
\pgfsetstrokecolor{currentstroke}%
\pgfsetdash{}{0pt}%
\pgfsys@defobject{currentmarker}{\pgfqpoint{0.000000in}{-0.013889in}}{\pgfqpoint{0.000000in}{0.000000in}}{%
\pgfpathmoveto{\pgfqpoint{0.000000in}{0.000000in}}%
\pgfpathlineto{\pgfqpoint{0.000000in}{-0.013889in}}%
\pgfusepath{stroke,fill}%
}%
\begin{pgfscope}%
\pgfsys@transformshift{2.009543in}{0.638889in}%
\pgfsys@useobject{currentmarker}{}%
\end{pgfscope}%
\end{pgfscope}%
\begin{pgfscope}%
\definecolor{textcolor}{rgb}{0.000000,0.000000,0.000000}%
\pgfsetstrokecolor{textcolor}%
\pgfsetfillcolor{textcolor}%
\pgftext[x=1.939669in, y=0.315885in, left, base,rotate=45.000000]{\color{textcolor}{\ifdefined\pdftexversion\else\setmainfont{NanumMyeongjo}\rmfamily\fi\fontsize{9.000000}{10.800000}\selectfont\catcode`\^=\active\def^{\ifmmode\sp\else\^{}\fi}\catcode`\%=\active\def%{\%}2014}}%
\end{pgfscope}%
\begin{pgfscope}%
\pgfsetbuttcap%
\pgfsetroundjoin%
\definecolor{currentfill}{rgb}{0.000000,0.000000,0.000000}%
\pgfsetfillcolor{currentfill}%
\pgfsetlinewidth{0.752812pt}%
\definecolor{currentstroke}{rgb}{0.000000,0.000000,0.000000}%
\pgfsetstrokecolor{currentstroke}%
\pgfsetdash{}{0pt}%
\pgfsys@defobject{currentmarker}{\pgfqpoint{0.000000in}{-0.013889in}}{\pgfqpoint{0.000000in}{0.000000in}}{%
\pgfpathmoveto{\pgfqpoint{0.000000in}{0.000000in}}%
\pgfpathlineto{\pgfqpoint{0.000000in}{-0.013889in}}%
\pgfusepath{stroke,fill}%
}%
\begin{pgfscope}%
\pgfsys@transformshift{2.284892in}{0.638889in}%
\pgfsys@useobject{currentmarker}{}%
\end{pgfscope}%
\end{pgfscope}%
\begin{pgfscope}%
\definecolor{textcolor}{rgb}{0.000000,0.000000,0.000000}%
\pgfsetstrokecolor{textcolor}%
\pgfsetfillcolor{textcolor}%
\pgftext[x=2.215019in, y=0.315885in, left, base,rotate=45.000000]{\color{textcolor}{\ifdefined\pdftexversion\else\setmainfont{NanumMyeongjo}\rmfamily\fi\fontsize{9.000000}{10.800000}\selectfont\catcode`\^=\active\def^{\ifmmode\sp\else\^{}\fi}\catcode`\%=\active\def%{\%}2015}}%
\end{pgfscope}%
\begin{pgfscope}%
\pgfsetbuttcap%
\pgfsetroundjoin%
\definecolor{currentfill}{rgb}{0.000000,0.000000,0.000000}%
\pgfsetfillcolor{currentfill}%
\pgfsetlinewidth{0.752812pt}%
\definecolor{currentstroke}{rgb}{0.000000,0.000000,0.000000}%
\pgfsetstrokecolor{currentstroke}%
\pgfsetdash{}{0pt}%
\pgfsys@defobject{currentmarker}{\pgfqpoint{0.000000in}{-0.013889in}}{\pgfqpoint{0.000000in}{0.000000in}}{%
\pgfpathmoveto{\pgfqpoint{0.000000in}{0.000000in}}%
\pgfpathlineto{\pgfqpoint{0.000000in}{-0.013889in}}%
\pgfusepath{stroke,fill}%
}%
\begin{pgfscope}%
\pgfsys@transformshift{2.560242in}{0.638889in}%
\pgfsys@useobject{currentmarker}{}%
\end{pgfscope}%
\end{pgfscope}%
\begin{pgfscope}%
\definecolor{textcolor}{rgb}{0.000000,0.000000,0.000000}%
\pgfsetstrokecolor{textcolor}%
\pgfsetfillcolor{textcolor}%
\pgftext[x=2.490368in, y=0.315885in, left, base,rotate=45.000000]{\color{textcolor}{\ifdefined\pdftexversion\else\setmainfont{NanumMyeongjo}\rmfamily\fi\fontsize{9.000000}{10.800000}\selectfont\catcode`\^=\active\def^{\ifmmode\sp\else\^{}\fi}\catcode`\%=\active\def%{\%}2016}}%
\end{pgfscope}%
\begin{pgfscope}%
\pgfsetbuttcap%
\pgfsetroundjoin%
\definecolor{currentfill}{rgb}{0.000000,0.000000,0.000000}%
\pgfsetfillcolor{currentfill}%
\pgfsetlinewidth{0.752812pt}%
\definecolor{currentstroke}{rgb}{0.000000,0.000000,0.000000}%
\pgfsetstrokecolor{currentstroke}%
\pgfsetdash{}{0pt}%
\pgfsys@defobject{currentmarker}{\pgfqpoint{0.000000in}{-0.013889in}}{\pgfqpoint{0.000000in}{0.000000in}}{%
\pgfpathmoveto{\pgfqpoint{0.000000in}{0.000000in}}%
\pgfpathlineto{\pgfqpoint{0.000000in}{-0.013889in}}%
\pgfusepath{stroke,fill}%
}%
\begin{pgfscope}%
\pgfsys@transformshift{2.835591in}{0.638889in}%
\pgfsys@useobject{currentmarker}{}%
\end{pgfscope}%
\end{pgfscope}%
\begin{pgfscope}%
\definecolor{textcolor}{rgb}{0.000000,0.000000,0.000000}%
\pgfsetstrokecolor{textcolor}%
\pgfsetfillcolor{textcolor}%
\pgftext[x=2.765718in, y=0.315885in, left, base,rotate=45.000000]{\color{textcolor}{\ifdefined\pdftexversion\else\setmainfont{NanumMyeongjo}\rmfamily\fi\fontsize{9.000000}{10.800000}\selectfont\catcode`\^=\active\def^{\ifmmode\sp\else\^{}\fi}\catcode`\%=\active\def%{\%}2017}}%
\end{pgfscope}%
\begin{pgfscope}%
\pgfsetbuttcap%
\pgfsetroundjoin%
\definecolor{currentfill}{rgb}{0.000000,0.000000,0.000000}%
\pgfsetfillcolor{currentfill}%
\pgfsetlinewidth{0.752812pt}%
\definecolor{currentstroke}{rgb}{0.000000,0.000000,0.000000}%
\pgfsetstrokecolor{currentstroke}%
\pgfsetdash{}{0pt}%
\pgfsys@defobject{currentmarker}{\pgfqpoint{0.000000in}{-0.013889in}}{\pgfqpoint{0.000000in}{0.000000in}}{%
\pgfpathmoveto{\pgfqpoint{0.000000in}{0.000000in}}%
\pgfpathlineto{\pgfqpoint{0.000000in}{-0.013889in}}%
\pgfusepath{stroke,fill}%
}%
\begin{pgfscope}%
\pgfsys@transformshift{3.110941in}{0.638889in}%
\pgfsys@useobject{currentmarker}{}%
\end{pgfscope}%
\end{pgfscope}%
\begin{pgfscope}%
\definecolor{textcolor}{rgb}{0.000000,0.000000,0.000000}%
\pgfsetstrokecolor{textcolor}%
\pgfsetfillcolor{textcolor}%
\pgftext[x=3.041068in, y=0.315885in, left, base,rotate=45.000000]{\color{textcolor}{\ifdefined\pdftexversion\else\setmainfont{NanumMyeongjo}\rmfamily\fi\fontsize{9.000000}{10.800000}\selectfont\catcode`\^=\active\def^{\ifmmode\sp\else\^{}\fi}\catcode`\%=\active\def%{\%}2018}}%
\end{pgfscope}%
\begin{pgfscope}%
\pgfsetbuttcap%
\pgfsetroundjoin%
\definecolor{currentfill}{rgb}{0.000000,0.000000,0.000000}%
\pgfsetfillcolor{currentfill}%
\pgfsetlinewidth{0.752812pt}%
\definecolor{currentstroke}{rgb}{0.000000,0.000000,0.000000}%
\pgfsetstrokecolor{currentstroke}%
\pgfsetdash{}{0pt}%
\pgfsys@defobject{currentmarker}{\pgfqpoint{0.000000in}{-0.013889in}}{\pgfqpoint{0.000000in}{0.000000in}}{%
\pgfpathmoveto{\pgfqpoint{0.000000in}{0.000000in}}%
\pgfpathlineto{\pgfqpoint{0.000000in}{-0.013889in}}%
\pgfusepath{stroke,fill}%
}%
\begin{pgfscope}%
\pgfsys@transformshift{3.386291in}{0.638889in}%
\pgfsys@useobject{currentmarker}{}%
\end{pgfscope}%
\end{pgfscope}%
\begin{pgfscope}%
\definecolor{textcolor}{rgb}{0.000000,0.000000,0.000000}%
\pgfsetstrokecolor{textcolor}%
\pgfsetfillcolor{textcolor}%
\pgftext[x=3.316417in, y=0.315885in, left, base,rotate=45.000000]{\color{textcolor}{\ifdefined\pdftexversion\else\setmainfont{NanumMyeongjo}\rmfamily\fi\fontsize{9.000000}{10.800000}\selectfont\catcode`\^=\active\def^{\ifmmode\sp\else\^{}\fi}\catcode`\%=\active\def%{\%}2019}}%
\end{pgfscope}%
\begin{pgfscope}%
\pgfsetbuttcap%
\pgfsetroundjoin%
\definecolor{currentfill}{rgb}{0.000000,0.000000,0.000000}%
\pgfsetfillcolor{currentfill}%
\pgfsetlinewidth{0.752812pt}%
\definecolor{currentstroke}{rgb}{0.000000,0.000000,0.000000}%
\pgfsetstrokecolor{currentstroke}%
\pgfsetdash{}{0pt}%
\pgfsys@defobject{currentmarker}{\pgfqpoint{0.000000in}{-0.013889in}}{\pgfqpoint{0.000000in}{0.000000in}}{%
\pgfpathmoveto{\pgfqpoint{0.000000in}{0.000000in}}%
\pgfpathlineto{\pgfqpoint{0.000000in}{-0.013889in}}%
\pgfusepath{stroke,fill}%
}%
\begin{pgfscope}%
\pgfsys@transformshift{3.661640in}{0.638889in}%
\pgfsys@useobject{currentmarker}{}%
\end{pgfscope}%
\end{pgfscope}%
\begin{pgfscope}%
\definecolor{textcolor}{rgb}{0.000000,0.000000,0.000000}%
\pgfsetstrokecolor{textcolor}%
\pgfsetfillcolor{textcolor}%
\pgftext[x=3.591767in, y=0.315885in, left, base,rotate=45.000000]{\color{textcolor}{\ifdefined\pdftexversion\else\setmainfont{NanumMyeongjo}\rmfamily\fi\fontsize{9.000000}{10.800000}\selectfont\catcode`\^=\active\def^{\ifmmode\sp\else\^{}\fi}\catcode`\%=\active\def%{\%}2020}}%
\end{pgfscope}%
\begin{pgfscope}%
\pgfsetbuttcap%
\pgfsetroundjoin%
\definecolor{currentfill}{rgb}{0.000000,0.000000,0.000000}%
\pgfsetfillcolor{currentfill}%
\pgfsetlinewidth{0.752812pt}%
\definecolor{currentstroke}{rgb}{0.000000,0.000000,0.000000}%
\pgfsetstrokecolor{currentstroke}%
\pgfsetdash{}{0pt}%
\pgfsys@defobject{currentmarker}{\pgfqpoint{0.000000in}{-0.013889in}}{\pgfqpoint{0.000000in}{0.000000in}}{%
\pgfpathmoveto{\pgfqpoint{0.000000in}{0.000000in}}%
\pgfpathlineto{\pgfqpoint{0.000000in}{-0.013889in}}%
\pgfusepath{stroke,fill}%
}%
\begin{pgfscope}%
\pgfsys@transformshift{3.936990in}{0.638889in}%
\pgfsys@useobject{currentmarker}{}%
\end{pgfscope}%
\end{pgfscope}%
\begin{pgfscope}%
\definecolor{textcolor}{rgb}{0.000000,0.000000,0.000000}%
\pgfsetstrokecolor{textcolor}%
\pgfsetfillcolor{textcolor}%
\pgftext[x=3.867117in, y=0.315885in, left, base,rotate=45.000000]{\color{textcolor}{\ifdefined\pdftexversion\else\setmainfont{NanumMyeongjo}\rmfamily\fi\fontsize{9.000000}{10.800000}\selectfont\catcode`\^=\active\def^{\ifmmode\sp\else\^{}\fi}\catcode`\%=\active\def%{\%}2021}}%
\end{pgfscope}%
\begin{pgfscope}%
\pgfsetbuttcap%
\pgfsetroundjoin%
\definecolor{currentfill}{rgb}{0.000000,0.000000,0.000000}%
\pgfsetfillcolor{currentfill}%
\pgfsetlinewidth{0.752812pt}%
\definecolor{currentstroke}{rgb}{0.000000,0.000000,0.000000}%
\pgfsetstrokecolor{currentstroke}%
\pgfsetdash{}{0pt}%
\pgfsys@defobject{currentmarker}{\pgfqpoint{0.000000in}{-0.013889in}}{\pgfqpoint{0.000000in}{0.000000in}}{%
\pgfpathmoveto{\pgfqpoint{0.000000in}{0.000000in}}%
\pgfpathlineto{\pgfqpoint{0.000000in}{-0.013889in}}%
\pgfusepath{stroke,fill}%
}%
\begin{pgfscope}%
\pgfsys@transformshift{4.212340in}{0.638889in}%
\pgfsys@useobject{currentmarker}{}%
\end{pgfscope}%
\end{pgfscope}%
\begin{pgfscope}%
\definecolor{textcolor}{rgb}{0.000000,0.000000,0.000000}%
\pgfsetstrokecolor{textcolor}%
\pgfsetfillcolor{textcolor}%
\pgftext[x=4.142466in, y=0.315885in, left, base,rotate=45.000000]{\color{textcolor}{\ifdefined\pdftexversion\else\setmainfont{NanumMyeongjo}\rmfamily\fi\fontsize{9.000000}{10.800000}\selectfont\catcode`\^=\active\def^{\ifmmode\sp\else\^{}\fi}\catcode`\%=\active\def%{\%}2022}}%
\end{pgfscope}%
\begin{pgfscope}%
\pgfsetbuttcap%
\pgfsetroundjoin%
\definecolor{currentfill}{rgb}{0.000000,0.000000,0.000000}%
\pgfsetfillcolor{currentfill}%
\pgfsetlinewidth{0.752812pt}%
\definecolor{currentstroke}{rgb}{0.000000,0.000000,0.000000}%
\pgfsetstrokecolor{currentstroke}%
\pgfsetdash{}{0pt}%
\pgfsys@defobject{currentmarker}{\pgfqpoint{0.000000in}{-0.013889in}}{\pgfqpoint{0.000000in}{0.000000in}}{%
\pgfpathmoveto{\pgfqpoint{0.000000in}{0.000000in}}%
\pgfpathlineto{\pgfqpoint{0.000000in}{-0.013889in}}%
\pgfusepath{stroke,fill}%
}%
\begin{pgfscope}%
\pgfsys@transformshift{4.487689in}{0.638889in}%
\pgfsys@useobject{currentmarker}{}%
\end{pgfscope}%
\end{pgfscope}%
\begin{pgfscope}%
\definecolor{textcolor}{rgb}{0.000000,0.000000,0.000000}%
\pgfsetstrokecolor{textcolor}%
\pgfsetfillcolor{textcolor}%
\pgftext[x=4.417816in, y=0.315885in, left, base,rotate=45.000000]{\color{textcolor}{\ifdefined\pdftexversion\else\setmainfont{NanumMyeongjo}\rmfamily\fi\fontsize{9.000000}{10.800000}\selectfont\catcode`\^=\active\def^{\ifmmode\sp\else\^{}\fi}\catcode`\%=\active\def%{\%}2023}}%
\end{pgfscope}%
\begin{pgfscope}%
\pgfpathrectangle{\pgfqpoint{0.729167in}{0.638889in}}{\pgfqpoint{3.937500in}{2.172222in}}%
\pgfusepath{clip}%
\pgfsetbuttcap%
\pgfsetroundjoin%
\pgfsetlinewidth{0.602250pt}%
\definecolor{currentstroke}{rgb}{0.690196,0.690196,0.690196}%
\pgfsetstrokecolor{currentstroke}%
\pgfsetstrokeopacity{0.400000}%
\pgfsetdash{{2.220000pt}{0.960000pt}}{0.000000pt}%
\pgfpathmoveto{\pgfqpoint{0.729167in}{0.638889in}}%
\pgfpathlineto{\pgfqpoint{4.666667in}{0.638889in}}%
\pgfusepath{stroke}%
\end{pgfscope}%
\begin{pgfscope}%
\pgfsetbuttcap%
\pgfsetroundjoin%
\definecolor{currentfill}{rgb}{0.000000,0.000000,0.000000}%
\pgfsetfillcolor{currentfill}%
\pgfsetlinewidth{0.752812pt}%
\definecolor{currentstroke}{rgb}{0.000000,0.000000,0.000000}%
\pgfsetstrokecolor{currentstroke}%
\pgfsetdash{}{0pt}%
\pgfsys@defobject{currentmarker}{\pgfqpoint{-0.013889in}{0.000000in}}{\pgfqpoint{-0.000000in}{0.000000in}}{%
\pgfpathmoveto{\pgfqpoint{-0.000000in}{0.000000in}}%
\pgfpathlineto{\pgfqpoint{-0.013889in}{0.000000in}}%
\pgfusepath{stroke,fill}%
}%
\begin{pgfscope}%
\pgfsys@transformshift{0.729167in}{0.638889in}%
\pgfsys@useobject{currentmarker}{}%
\end{pgfscope}%
\end{pgfscope}%
\begin{pgfscope}%
\definecolor{textcolor}{rgb}{0.000000,0.000000,0.000000}%
\pgfsetstrokecolor{textcolor}%
\pgfsetfillcolor{textcolor}%
\pgftext[x=0.599528in, y=0.588962in, left, base]{\color{textcolor}{\ifdefined\pdftexversion\else\setmainfont{NanumMyeongjo}\rmfamily\fi\fontsize{9.000000}{10.800000}\selectfont\catcode`\^=\active\def^{\ifmmode\sp\else\^{}\fi}\catcode`\%=\active\def%{\%}0}}%
\end{pgfscope}%
\begin{pgfscope}%
\pgfpathrectangle{\pgfqpoint{0.729167in}{0.638889in}}{\pgfqpoint{3.937500in}{2.172222in}}%
\pgfusepath{clip}%
\pgfsetbuttcap%
\pgfsetroundjoin%
\pgfsetlinewidth{0.602250pt}%
\definecolor{currentstroke}{rgb}{0.690196,0.690196,0.690196}%
\pgfsetstrokecolor{currentstroke}%
\pgfsetstrokeopacity{0.400000}%
\pgfsetdash{{2.220000pt}{0.960000pt}}{0.000000pt}%
\pgfpathmoveto{\pgfqpoint{0.729167in}{0.805983in}}%
\pgfpathlineto{\pgfqpoint{4.666667in}{0.805983in}}%
\pgfusepath{stroke}%
\end{pgfscope}%
\begin{pgfscope}%
\pgfsetbuttcap%
\pgfsetroundjoin%
\definecolor{currentfill}{rgb}{0.000000,0.000000,0.000000}%
\pgfsetfillcolor{currentfill}%
\pgfsetlinewidth{0.752812pt}%
\definecolor{currentstroke}{rgb}{0.000000,0.000000,0.000000}%
\pgfsetstrokecolor{currentstroke}%
\pgfsetdash{}{0pt}%
\pgfsys@defobject{currentmarker}{\pgfqpoint{-0.013889in}{0.000000in}}{\pgfqpoint{-0.000000in}{0.000000in}}{%
\pgfpathmoveto{\pgfqpoint{-0.000000in}{0.000000in}}%
\pgfpathlineto{\pgfqpoint{-0.013889in}{0.000000in}}%
\pgfusepath{stroke,fill}%
}%
\begin{pgfscope}%
\pgfsys@transformshift{0.729167in}{0.805983in}%
\pgfsys@useobject{currentmarker}{}%
\end{pgfscope}%
\end{pgfscope}%
\begin{pgfscope}%
\definecolor{textcolor}{rgb}{0.000000,0.000000,0.000000}%
\pgfsetstrokecolor{textcolor}%
\pgfsetfillcolor{textcolor}%
\pgftext[x=0.599528in, y=0.756056in, left, base]{\color{textcolor}{\ifdefined\pdftexversion\else\setmainfont{NanumMyeongjo}\rmfamily\fi\fontsize{9.000000}{10.800000}\selectfont\catcode`\^=\active\def^{\ifmmode\sp\else\^{}\fi}\catcode`\%=\active\def%{\%}5}}%
\end{pgfscope}%
\begin{pgfscope}%
\pgfpathrectangle{\pgfqpoint{0.729167in}{0.638889in}}{\pgfqpoint{3.937500in}{2.172222in}}%
\pgfusepath{clip}%
\pgfsetbuttcap%
\pgfsetroundjoin%
\pgfsetlinewidth{0.602250pt}%
\definecolor{currentstroke}{rgb}{0.690196,0.690196,0.690196}%
\pgfsetstrokecolor{currentstroke}%
\pgfsetstrokeopacity{0.400000}%
\pgfsetdash{{2.220000pt}{0.960000pt}}{0.000000pt}%
\pgfpathmoveto{\pgfqpoint{0.729167in}{0.973077in}}%
\pgfpathlineto{\pgfqpoint{4.666667in}{0.973077in}}%
\pgfusepath{stroke}%
\end{pgfscope}%
\begin{pgfscope}%
\pgfsetbuttcap%
\pgfsetroundjoin%
\definecolor{currentfill}{rgb}{0.000000,0.000000,0.000000}%
\pgfsetfillcolor{currentfill}%
\pgfsetlinewidth{0.752812pt}%
\definecolor{currentstroke}{rgb}{0.000000,0.000000,0.000000}%
\pgfsetstrokecolor{currentstroke}%
\pgfsetdash{}{0pt}%
\pgfsys@defobject{currentmarker}{\pgfqpoint{-0.013889in}{0.000000in}}{\pgfqpoint{-0.000000in}{0.000000in}}{%
\pgfpathmoveto{\pgfqpoint{-0.000000in}{0.000000in}}%
\pgfpathlineto{\pgfqpoint{-0.013889in}{0.000000in}}%
\pgfusepath{stroke,fill}%
}%
\begin{pgfscope}%
\pgfsys@transformshift{0.729167in}{0.973077in}%
\pgfsys@useobject{currentmarker}{}%
\end{pgfscope}%
\end{pgfscope}%
\begin{pgfscope}%
\definecolor{textcolor}{rgb}{0.000000,0.000000,0.000000}%
\pgfsetstrokecolor{textcolor}%
\pgfsetfillcolor{textcolor}%
\pgftext[x=0.532389in, y=0.923150in, left, base]{\color{textcolor}{\ifdefined\pdftexversion\else\setmainfont{NanumMyeongjo}\rmfamily\fi\fontsize{9.000000}{10.800000}\selectfont\catcode`\^=\active\def^{\ifmmode\sp\else\^{}\fi}\catcode`\%=\active\def%{\%}10}}%
\end{pgfscope}%
\begin{pgfscope}%
\pgfpathrectangle{\pgfqpoint{0.729167in}{0.638889in}}{\pgfqpoint{3.937500in}{2.172222in}}%
\pgfusepath{clip}%
\pgfsetbuttcap%
\pgfsetroundjoin%
\pgfsetlinewidth{0.602250pt}%
\definecolor{currentstroke}{rgb}{0.690196,0.690196,0.690196}%
\pgfsetstrokecolor{currentstroke}%
\pgfsetstrokeopacity{0.400000}%
\pgfsetdash{{2.220000pt}{0.960000pt}}{0.000000pt}%
\pgfpathmoveto{\pgfqpoint{0.729167in}{1.140171in}}%
\pgfpathlineto{\pgfqpoint{4.666667in}{1.140171in}}%
\pgfusepath{stroke}%
\end{pgfscope}%
\begin{pgfscope}%
\pgfsetbuttcap%
\pgfsetroundjoin%
\definecolor{currentfill}{rgb}{0.000000,0.000000,0.000000}%
\pgfsetfillcolor{currentfill}%
\pgfsetlinewidth{0.752812pt}%
\definecolor{currentstroke}{rgb}{0.000000,0.000000,0.000000}%
\pgfsetstrokecolor{currentstroke}%
\pgfsetdash{}{0pt}%
\pgfsys@defobject{currentmarker}{\pgfqpoint{-0.013889in}{0.000000in}}{\pgfqpoint{-0.000000in}{0.000000in}}{%
\pgfpathmoveto{\pgfqpoint{-0.000000in}{0.000000in}}%
\pgfpathlineto{\pgfqpoint{-0.013889in}{0.000000in}}%
\pgfusepath{stroke,fill}%
}%
\begin{pgfscope}%
\pgfsys@transformshift{0.729167in}{1.140171in}%
\pgfsys@useobject{currentmarker}{}%
\end{pgfscope}%
\end{pgfscope}%
\begin{pgfscope}%
\definecolor{textcolor}{rgb}{0.000000,0.000000,0.000000}%
\pgfsetstrokecolor{textcolor}%
\pgfsetfillcolor{textcolor}%
\pgftext[x=0.532389in, y=1.090244in, left, base]{\color{textcolor}{\ifdefined\pdftexversion\else\setmainfont{NanumMyeongjo}\rmfamily\fi\fontsize{9.000000}{10.800000}\selectfont\catcode`\^=\active\def^{\ifmmode\sp\else\^{}\fi}\catcode`\%=\active\def%{\%}15}}%
\end{pgfscope}%
\begin{pgfscope}%
\pgfpathrectangle{\pgfqpoint{0.729167in}{0.638889in}}{\pgfqpoint{3.937500in}{2.172222in}}%
\pgfusepath{clip}%
\pgfsetbuttcap%
\pgfsetroundjoin%
\pgfsetlinewidth{0.602250pt}%
\definecolor{currentstroke}{rgb}{0.690196,0.690196,0.690196}%
\pgfsetstrokecolor{currentstroke}%
\pgfsetstrokeopacity{0.400000}%
\pgfsetdash{{2.220000pt}{0.960000pt}}{0.000000pt}%
\pgfpathmoveto{\pgfqpoint{0.729167in}{1.307265in}}%
\pgfpathlineto{\pgfqpoint{4.666667in}{1.307265in}}%
\pgfusepath{stroke}%
\end{pgfscope}%
\begin{pgfscope}%
\pgfsetbuttcap%
\pgfsetroundjoin%
\definecolor{currentfill}{rgb}{0.000000,0.000000,0.000000}%
\pgfsetfillcolor{currentfill}%
\pgfsetlinewidth{0.752812pt}%
\definecolor{currentstroke}{rgb}{0.000000,0.000000,0.000000}%
\pgfsetstrokecolor{currentstroke}%
\pgfsetdash{}{0pt}%
\pgfsys@defobject{currentmarker}{\pgfqpoint{-0.013889in}{0.000000in}}{\pgfqpoint{-0.000000in}{0.000000in}}{%
\pgfpathmoveto{\pgfqpoint{-0.000000in}{0.000000in}}%
\pgfpathlineto{\pgfqpoint{-0.013889in}{0.000000in}}%
\pgfusepath{stroke,fill}%
}%
\begin{pgfscope}%
\pgfsys@transformshift{0.729167in}{1.307265in}%
\pgfsys@useobject{currentmarker}{}%
\end{pgfscope}%
\end{pgfscope}%
\begin{pgfscope}%
\definecolor{textcolor}{rgb}{0.000000,0.000000,0.000000}%
\pgfsetstrokecolor{textcolor}%
\pgfsetfillcolor{textcolor}%
\pgftext[x=0.532389in, y=1.257338in, left, base]{\color{textcolor}{\ifdefined\pdftexversion\else\setmainfont{NanumMyeongjo}\rmfamily\fi\fontsize{9.000000}{10.800000}\selectfont\catcode`\^=\active\def^{\ifmmode\sp\else\^{}\fi}\catcode`\%=\active\def%{\%}20}}%
\end{pgfscope}%
\begin{pgfscope}%
\pgfpathrectangle{\pgfqpoint{0.729167in}{0.638889in}}{\pgfqpoint{3.937500in}{2.172222in}}%
\pgfusepath{clip}%
\pgfsetbuttcap%
\pgfsetroundjoin%
\pgfsetlinewidth{0.602250pt}%
\definecolor{currentstroke}{rgb}{0.690196,0.690196,0.690196}%
\pgfsetstrokecolor{currentstroke}%
\pgfsetstrokeopacity{0.400000}%
\pgfsetdash{{2.220000pt}{0.960000pt}}{0.000000pt}%
\pgfpathmoveto{\pgfqpoint{0.729167in}{1.474359in}}%
\pgfpathlineto{\pgfqpoint{4.666667in}{1.474359in}}%
\pgfusepath{stroke}%
\end{pgfscope}%
\begin{pgfscope}%
\pgfsetbuttcap%
\pgfsetroundjoin%
\definecolor{currentfill}{rgb}{0.000000,0.000000,0.000000}%
\pgfsetfillcolor{currentfill}%
\pgfsetlinewidth{0.752812pt}%
\definecolor{currentstroke}{rgb}{0.000000,0.000000,0.000000}%
\pgfsetstrokecolor{currentstroke}%
\pgfsetdash{}{0pt}%
\pgfsys@defobject{currentmarker}{\pgfqpoint{-0.013889in}{0.000000in}}{\pgfqpoint{-0.000000in}{0.000000in}}{%
\pgfpathmoveto{\pgfqpoint{-0.000000in}{0.000000in}}%
\pgfpathlineto{\pgfqpoint{-0.013889in}{0.000000in}}%
\pgfusepath{stroke,fill}%
}%
\begin{pgfscope}%
\pgfsys@transformshift{0.729167in}{1.474359in}%
\pgfsys@useobject{currentmarker}{}%
\end{pgfscope}%
\end{pgfscope}%
\begin{pgfscope}%
\definecolor{textcolor}{rgb}{0.000000,0.000000,0.000000}%
\pgfsetstrokecolor{textcolor}%
\pgfsetfillcolor{textcolor}%
\pgftext[x=0.532389in, y=1.424432in, left, base]{\color{textcolor}{\ifdefined\pdftexversion\else\setmainfont{NanumMyeongjo}\rmfamily\fi\fontsize{9.000000}{10.800000}\selectfont\catcode`\^=\active\def^{\ifmmode\sp\else\^{}\fi}\catcode`\%=\active\def%{\%}25}}%
\end{pgfscope}%
\begin{pgfscope}%
\pgfpathrectangle{\pgfqpoint{0.729167in}{0.638889in}}{\pgfqpoint{3.937500in}{2.172222in}}%
\pgfusepath{clip}%
\pgfsetbuttcap%
\pgfsetroundjoin%
\pgfsetlinewidth{0.602250pt}%
\definecolor{currentstroke}{rgb}{0.690196,0.690196,0.690196}%
\pgfsetstrokecolor{currentstroke}%
\pgfsetstrokeopacity{0.400000}%
\pgfsetdash{{2.220000pt}{0.960000pt}}{0.000000pt}%
\pgfpathmoveto{\pgfqpoint{0.729167in}{1.641453in}}%
\pgfpathlineto{\pgfqpoint{4.666667in}{1.641453in}}%
\pgfusepath{stroke}%
\end{pgfscope}%
\begin{pgfscope}%
\pgfsetbuttcap%
\pgfsetroundjoin%
\definecolor{currentfill}{rgb}{0.000000,0.000000,0.000000}%
\pgfsetfillcolor{currentfill}%
\pgfsetlinewidth{0.752812pt}%
\definecolor{currentstroke}{rgb}{0.000000,0.000000,0.000000}%
\pgfsetstrokecolor{currentstroke}%
\pgfsetdash{}{0pt}%
\pgfsys@defobject{currentmarker}{\pgfqpoint{-0.013889in}{0.000000in}}{\pgfqpoint{-0.000000in}{0.000000in}}{%
\pgfpathmoveto{\pgfqpoint{-0.000000in}{0.000000in}}%
\pgfpathlineto{\pgfqpoint{-0.013889in}{0.000000in}}%
\pgfusepath{stroke,fill}%
}%
\begin{pgfscope}%
\pgfsys@transformshift{0.729167in}{1.641453in}%
\pgfsys@useobject{currentmarker}{}%
\end{pgfscope}%
\end{pgfscope}%
\begin{pgfscope}%
\definecolor{textcolor}{rgb}{0.000000,0.000000,0.000000}%
\pgfsetstrokecolor{textcolor}%
\pgfsetfillcolor{textcolor}%
\pgftext[x=0.532389in, y=1.591526in, left, base]{\color{textcolor}{\ifdefined\pdftexversion\else\setmainfont{NanumMyeongjo}\rmfamily\fi\fontsize{9.000000}{10.800000}\selectfont\catcode`\^=\active\def^{\ifmmode\sp\else\^{}\fi}\catcode`\%=\active\def%{\%}30}}%
\end{pgfscope}%
\begin{pgfscope}%
\pgfpathrectangle{\pgfqpoint{0.729167in}{0.638889in}}{\pgfqpoint{3.937500in}{2.172222in}}%
\pgfusepath{clip}%
\pgfsetbuttcap%
\pgfsetroundjoin%
\pgfsetlinewidth{0.602250pt}%
\definecolor{currentstroke}{rgb}{0.690196,0.690196,0.690196}%
\pgfsetstrokecolor{currentstroke}%
\pgfsetstrokeopacity{0.400000}%
\pgfsetdash{{2.220000pt}{0.960000pt}}{0.000000pt}%
\pgfpathmoveto{\pgfqpoint{0.729167in}{1.808547in}}%
\pgfpathlineto{\pgfqpoint{4.666667in}{1.808547in}}%
\pgfusepath{stroke}%
\end{pgfscope}%
\begin{pgfscope}%
\pgfsetbuttcap%
\pgfsetroundjoin%
\definecolor{currentfill}{rgb}{0.000000,0.000000,0.000000}%
\pgfsetfillcolor{currentfill}%
\pgfsetlinewidth{0.752812pt}%
\definecolor{currentstroke}{rgb}{0.000000,0.000000,0.000000}%
\pgfsetstrokecolor{currentstroke}%
\pgfsetdash{}{0pt}%
\pgfsys@defobject{currentmarker}{\pgfqpoint{-0.013889in}{0.000000in}}{\pgfqpoint{-0.000000in}{0.000000in}}{%
\pgfpathmoveto{\pgfqpoint{-0.000000in}{0.000000in}}%
\pgfpathlineto{\pgfqpoint{-0.013889in}{0.000000in}}%
\pgfusepath{stroke,fill}%
}%
\begin{pgfscope}%
\pgfsys@transformshift{0.729167in}{1.808547in}%
\pgfsys@useobject{currentmarker}{}%
\end{pgfscope}%
\end{pgfscope}%
\begin{pgfscope}%
\definecolor{textcolor}{rgb}{0.000000,0.000000,0.000000}%
\pgfsetstrokecolor{textcolor}%
\pgfsetfillcolor{textcolor}%
\pgftext[x=0.532389in, y=1.758620in, left, base]{\color{textcolor}{\ifdefined\pdftexversion\else\setmainfont{NanumMyeongjo}\rmfamily\fi\fontsize{9.000000}{10.800000}\selectfont\catcode`\^=\active\def^{\ifmmode\sp\else\^{}\fi}\catcode`\%=\active\def%{\%}35}}%
\end{pgfscope}%
\begin{pgfscope}%
\pgfpathrectangle{\pgfqpoint{0.729167in}{0.638889in}}{\pgfqpoint{3.937500in}{2.172222in}}%
\pgfusepath{clip}%
\pgfsetbuttcap%
\pgfsetroundjoin%
\pgfsetlinewidth{0.602250pt}%
\definecolor{currentstroke}{rgb}{0.690196,0.690196,0.690196}%
\pgfsetstrokecolor{currentstroke}%
\pgfsetstrokeopacity{0.400000}%
\pgfsetdash{{2.220000pt}{0.960000pt}}{0.000000pt}%
\pgfpathmoveto{\pgfqpoint{0.729167in}{1.975641in}}%
\pgfpathlineto{\pgfqpoint{4.666667in}{1.975641in}}%
\pgfusepath{stroke}%
\end{pgfscope}%
\begin{pgfscope}%
\pgfsetbuttcap%
\pgfsetroundjoin%
\definecolor{currentfill}{rgb}{0.000000,0.000000,0.000000}%
\pgfsetfillcolor{currentfill}%
\pgfsetlinewidth{0.752812pt}%
\definecolor{currentstroke}{rgb}{0.000000,0.000000,0.000000}%
\pgfsetstrokecolor{currentstroke}%
\pgfsetdash{}{0pt}%
\pgfsys@defobject{currentmarker}{\pgfqpoint{-0.013889in}{0.000000in}}{\pgfqpoint{-0.000000in}{0.000000in}}{%
\pgfpathmoveto{\pgfqpoint{-0.000000in}{0.000000in}}%
\pgfpathlineto{\pgfqpoint{-0.013889in}{0.000000in}}%
\pgfusepath{stroke,fill}%
}%
\begin{pgfscope}%
\pgfsys@transformshift{0.729167in}{1.975641in}%
\pgfsys@useobject{currentmarker}{}%
\end{pgfscope}%
\end{pgfscope}%
\begin{pgfscope}%
\definecolor{textcolor}{rgb}{0.000000,0.000000,0.000000}%
\pgfsetstrokecolor{textcolor}%
\pgfsetfillcolor{textcolor}%
\pgftext[x=0.532389in, y=1.925714in, left, base]{\color{textcolor}{\ifdefined\pdftexversion\else\setmainfont{NanumMyeongjo}\rmfamily\fi\fontsize{9.000000}{10.800000}\selectfont\catcode`\^=\active\def^{\ifmmode\sp\else\^{}\fi}\catcode`\%=\active\def%{\%}40}}%
\end{pgfscope}%
\begin{pgfscope}%
\pgfpathrectangle{\pgfqpoint{0.729167in}{0.638889in}}{\pgfqpoint{3.937500in}{2.172222in}}%
\pgfusepath{clip}%
\pgfsetbuttcap%
\pgfsetroundjoin%
\pgfsetlinewidth{0.602250pt}%
\definecolor{currentstroke}{rgb}{0.690196,0.690196,0.690196}%
\pgfsetstrokecolor{currentstroke}%
\pgfsetstrokeopacity{0.400000}%
\pgfsetdash{{2.220000pt}{0.960000pt}}{0.000000pt}%
\pgfpathmoveto{\pgfqpoint{0.729167in}{2.142735in}}%
\pgfpathlineto{\pgfqpoint{4.666667in}{2.142735in}}%
\pgfusepath{stroke}%
\end{pgfscope}%
\begin{pgfscope}%
\pgfsetbuttcap%
\pgfsetroundjoin%
\definecolor{currentfill}{rgb}{0.000000,0.000000,0.000000}%
\pgfsetfillcolor{currentfill}%
\pgfsetlinewidth{0.752812pt}%
\definecolor{currentstroke}{rgb}{0.000000,0.000000,0.000000}%
\pgfsetstrokecolor{currentstroke}%
\pgfsetdash{}{0pt}%
\pgfsys@defobject{currentmarker}{\pgfqpoint{-0.013889in}{0.000000in}}{\pgfqpoint{-0.000000in}{0.000000in}}{%
\pgfpathmoveto{\pgfqpoint{-0.000000in}{0.000000in}}%
\pgfpathlineto{\pgfqpoint{-0.013889in}{0.000000in}}%
\pgfusepath{stroke,fill}%
}%
\begin{pgfscope}%
\pgfsys@transformshift{0.729167in}{2.142735in}%
\pgfsys@useobject{currentmarker}{}%
\end{pgfscope}%
\end{pgfscope}%
\begin{pgfscope}%
\definecolor{textcolor}{rgb}{0.000000,0.000000,0.000000}%
\pgfsetstrokecolor{textcolor}%
\pgfsetfillcolor{textcolor}%
\pgftext[x=0.532389in, y=2.092808in, left, base]{\color{textcolor}{\ifdefined\pdftexversion\else\setmainfont{NanumMyeongjo}\rmfamily\fi\fontsize{9.000000}{10.800000}\selectfont\catcode`\^=\active\def^{\ifmmode\sp\else\^{}\fi}\catcode`\%=\active\def%{\%}45}}%
\end{pgfscope}%
\begin{pgfscope}%
\pgfpathrectangle{\pgfqpoint{0.729167in}{0.638889in}}{\pgfqpoint{3.937500in}{2.172222in}}%
\pgfusepath{clip}%
\pgfsetbuttcap%
\pgfsetroundjoin%
\pgfsetlinewidth{0.602250pt}%
\definecolor{currentstroke}{rgb}{0.690196,0.690196,0.690196}%
\pgfsetstrokecolor{currentstroke}%
\pgfsetstrokeopacity{0.400000}%
\pgfsetdash{{2.220000pt}{0.960000pt}}{0.000000pt}%
\pgfpathmoveto{\pgfqpoint{0.729167in}{2.309829in}}%
\pgfpathlineto{\pgfqpoint{4.666667in}{2.309829in}}%
\pgfusepath{stroke}%
\end{pgfscope}%
\begin{pgfscope}%
\pgfsetbuttcap%
\pgfsetroundjoin%
\definecolor{currentfill}{rgb}{0.000000,0.000000,0.000000}%
\pgfsetfillcolor{currentfill}%
\pgfsetlinewidth{0.752812pt}%
\definecolor{currentstroke}{rgb}{0.000000,0.000000,0.000000}%
\pgfsetstrokecolor{currentstroke}%
\pgfsetdash{}{0pt}%
\pgfsys@defobject{currentmarker}{\pgfqpoint{-0.013889in}{0.000000in}}{\pgfqpoint{-0.000000in}{0.000000in}}{%
\pgfpathmoveto{\pgfqpoint{-0.000000in}{0.000000in}}%
\pgfpathlineto{\pgfqpoint{-0.013889in}{0.000000in}}%
\pgfusepath{stroke,fill}%
}%
\begin{pgfscope}%
\pgfsys@transformshift{0.729167in}{2.309829in}%
\pgfsys@useobject{currentmarker}{}%
\end{pgfscope}%
\end{pgfscope}%
\begin{pgfscope}%
\definecolor{textcolor}{rgb}{0.000000,0.000000,0.000000}%
\pgfsetstrokecolor{textcolor}%
\pgfsetfillcolor{textcolor}%
\pgftext[x=0.532389in, y=2.259902in, left, base]{\color{textcolor}{\ifdefined\pdftexversion\else\setmainfont{NanumMyeongjo}\rmfamily\fi\fontsize{9.000000}{10.800000}\selectfont\catcode`\^=\active\def^{\ifmmode\sp\else\^{}\fi}\catcode`\%=\active\def%{\%}50}}%
\end{pgfscope}%
\begin{pgfscope}%
\pgfpathrectangle{\pgfqpoint{0.729167in}{0.638889in}}{\pgfqpoint{3.937500in}{2.172222in}}%
\pgfusepath{clip}%
\pgfsetbuttcap%
\pgfsetroundjoin%
\pgfsetlinewidth{0.602250pt}%
\definecolor{currentstroke}{rgb}{0.690196,0.690196,0.690196}%
\pgfsetstrokecolor{currentstroke}%
\pgfsetstrokeopacity{0.400000}%
\pgfsetdash{{2.220000pt}{0.960000pt}}{0.000000pt}%
\pgfpathmoveto{\pgfqpoint{0.729167in}{2.476923in}}%
\pgfpathlineto{\pgfqpoint{4.666667in}{2.476923in}}%
\pgfusepath{stroke}%
\end{pgfscope}%
\begin{pgfscope}%
\pgfsetbuttcap%
\pgfsetroundjoin%
\definecolor{currentfill}{rgb}{0.000000,0.000000,0.000000}%
\pgfsetfillcolor{currentfill}%
\pgfsetlinewidth{0.752812pt}%
\definecolor{currentstroke}{rgb}{0.000000,0.000000,0.000000}%
\pgfsetstrokecolor{currentstroke}%
\pgfsetdash{}{0pt}%
\pgfsys@defobject{currentmarker}{\pgfqpoint{-0.013889in}{0.000000in}}{\pgfqpoint{-0.000000in}{0.000000in}}{%
\pgfpathmoveto{\pgfqpoint{-0.000000in}{0.000000in}}%
\pgfpathlineto{\pgfqpoint{-0.013889in}{0.000000in}}%
\pgfusepath{stroke,fill}%
}%
\begin{pgfscope}%
\pgfsys@transformshift{0.729167in}{2.476923in}%
\pgfsys@useobject{currentmarker}{}%
\end{pgfscope}%
\end{pgfscope}%
\begin{pgfscope}%
\definecolor{textcolor}{rgb}{0.000000,0.000000,0.000000}%
\pgfsetstrokecolor{textcolor}%
\pgfsetfillcolor{textcolor}%
\pgftext[x=0.532389in, y=2.426996in, left, base]{\color{textcolor}{\ifdefined\pdftexversion\else\setmainfont{NanumMyeongjo}\rmfamily\fi\fontsize{9.000000}{10.800000}\selectfont\catcode`\^=\active\def^{\ifmmode\sp\else\^{}\fi}\catcode`\%=\active\def%{\%}55}}%
\end{pgfscope}%
\begin{pgfscope}%
\pgfpathrectangle{\pgfqpoint{0.729167in}{0.638889in}}{\pgfqpoint{3.937500in}{2.172222in}}%
\pgfusepath{clip}%
\pgfsetbuttcap%
\pgfsetroundjoin%
\pgfsetlinewidth{0.602250pt}%
\definecolor{currentstroke}{rgb}{0.690196,0.690196,0.690196}%
\pgfsetstrokecolor{currentstroke}%
\pgfsetstrokeopacity{0.400000}%
\pgfsetdash{{2.220000pt}{0.960000pt}}{0.000000pt}%
\pgfpathmoveto{\pgfqpoint{0.729167in}{2.644017in}}%
\pgfpathlineto{\pgfqpoint{4.666667in}{2.644017in}}%
\pgfusepath{stroke}%
\end{pgfscope}%
\begin{pgfscope}%
\pgfsetbuttcap%
\pgfsetroundjoin%
\definecolor{currentfill}{rgb}{0.000000,0.000000,0.000000}%
\pgfsetfillcolor{currentfill}%
\pgfsetlinewidth{0.752812pt}%
\definecolor{currentstroke}{rgb}{0.000000,0.000000,0.000000}%
\pgfsetstrokecolor{currentstroke}%
\pgfsetdash{}{0pt}%
\pgfsys@defobject{currentmarker}{\pgfqpoint{-0.013889in}{0.000000in}}{\pgfqpoint{-0.000000in}{0.000000in}}{%
\pgfpathmoveto{\pgfqpoint{-0.000000in}{0.000000in}}%
\pgfpathlineto{\pgfqpoint{-0.013889in}{0.000000in}}%
\pgfusepath{stroke,fill}%
}%
\begin{pgfscope}%
\pgfsys@transformshift{0.729167in}{2.644017in}%
\pgfsys@useobject{currentmarker}{}%
\end{pgfscope}%
\end{pgfscope}%
\begin{pgfscope}%
\definecolor{textcolor}{rgb}{0.000000,0.000000,0.000000}%
\pgfsetstrokecolor{textcolor}%
\pgfsetfillcolor{textcolor}%
\pgftext[x=0.532389in, y=2.594090in, left, base]{\color{textcolor}{\ifdefined\pdftexversion\else\setmainfont{NanumMyeongjo}\rmfamily\fi\fontsize{9.000000}{10.800000}\selectfont\catcode`\^=\active\def^{\ifmmode\sp\else\^{}\fi}\catcode`\%=\active\def%{\%}60}}%
\end{pgfscope}%
\begin{pgfscope}%
\pgfpathrectangle{\pgfqpoint{0.729167in}{0.638889in}}{\pgfqpoint{3.937500in}{2.172222in}}%
\pgfusepath{clip}%
\pgfsetbuttcap%
\pgfsetroundjoin%
\pgfsetlinewidth{0.602250pt}%
\definecolor{currentstroke}{rgb}{0.690196,0.690196,0.690196}%
\pgfsetstrokecolor{currentstroke}%
\pgfsetstrokeopacity{0.400000}%
\pgfsetdash{{2.220000pt}{0.960000pt}}{0.000000pt}%
\pgfpathmoveto{\pgfqpoint{0.729167in}{2.811111in}}%
\pgfpathlineto{\pgfqpoint{4.666667in}{2.811111in}}%
\pgfusepath{stroke}%
\end{pgfscope}%
\begin{pgfscope}%
\pgfsetbuttcap%
\pgfsetroundjoin%
\definecolor{currentfill}{rgb}{0.000000,0.000000,0.000000}%
\pgfsetfillcolor{currentfill}%
\pgfsetlinewidth{0.752812pt}%
\definecolor{currentstroke}{rgb}{0.000000,0.000000,0.000000}%
\pgfsetstrokecolor{currentstroke}%
\pgfsetdash{}{0pt}%
\pgfsys@defobject{currentmarker}{\pgfqpoint{-0.013889in}{0.000000in}}{\pgfqpoint{-0.000000in}{0.000000in}}{%
\pgfpathmoveto{\pgfqpoint{-0.000000in}{0.000000in}}%
\pgfpathlineto{\pgfqpoint{-0.013889in}{0.000000in}}%
\pgfusepath{stroke,fill}%
}%
\begin{pgfscope}%
\pgfsys@transformshift{0.729167in}{2.811111in}%
\pgfsys@useobject{currentmarker}{}%
\end{pgfscope}%
\end{pgfscope}%
\begin{pgfscope}%
\definecolor{textcolor}{rgb}{0.000000,0.000000,0.000000}%
\pgfsetstrokecolor{textcolor}%
\pgfsetfillcolor{textcolor}%
\pgftext[x=0.532389in, y=2.761184in, left, base]{\color{textcolor}{\ifdefined\pdftexversion\else\setmainfont{NanumMyeongjo}\rmfamily\fi\fontsize{9.000000}{10.800000}\selectfont\catcode`\^=\active\def^{\ifmmode\sp\else\^{}\fi}\catcode`\%=\active\def%{\%}65}}%
\end{pgfscope}%
\begin{pgfscope}%
\pgfpathrectangle{\pgfqpoint{0.729167in}{0.638889in}}{\pgfqpoint{3.937500in}{2.172222in}}%
\pgfusepath{clip}%
\pgfsetrectcap%
\pgfsetroundjoin%
\pgfsetlinewidth{1.204500pt}%
\definecolor{currentstroke}{rgb}{0.235294,0.490196,0.764706}%
\pgfsetstrokecolor{currentstroke}%
\pgfsetdash{}{0pt}%
\pgfpathmoveto{\pgfqpoint{0.908144in}{2.446846in}}%
\pgfpathlineto{\pgfqpoint{1.183494in}{2.149419in}}%
\pgfpathlineto{\pgfqpoint{1.458843in}{2.166128in}}%
\pgfpathlineto{\pgfqpoint{1.734193in}{2.226282in}}%
\pgfpathlineto{\pgfqpoint{2.009543in}{2.299803in}}%
\pgfpathlineto{\pgfqpoint{2.284892in}{2.316513in}}%
\pgfpathlineto{\pgfqpoint{2.560242in}{2.336564in}}%
\pgfpathlineto{\pgfqpoint{2.835591in}{2.266385in}}%
\pgfpathlineto{\pgfqpoint{3.110941in}{2.206231in}}%
\pgfpathlineto{\pgfqpoint{3.386291in}{2.169470in}}%
\pgfpathlineto{\pgfqpoint{3.661640in}{2.169470in}}%
\pgfpathlineto{\pgfqpoint{3.936990in}{1.985667in}}%
\pgfpathlineto{\pgfqpoint{4.212340in}{2.179496in}}%
\pgfpathlineto{\pgfqpoint{4.487689in}{2.159444in}}%
\pgfusepath{stroke}%
\end{pgfscope}%
\begin{pgfscope}%
\pgfpathrectangle{\pgfqpoint{0.729167in}{0.638889in}}{\pgfqpoint{3.937500in}{2.172222in}}%
\pgfusepath{clip}%
\pgfsetbuttcap%
\pgfsetroundjoin%
\definecolor{currentfill}{rgb}{0.235294,0.490196,0.764706}%
\pgfsetfillcolor{currentfill}%
\pgfsetlinewidth{1.003750pt}%
\definecolor{currentstroke}{rgb}{0.235294,0.490196,0.764706}%
\pgfsetstrokecolor{currentstroke}%
\pgfsetdash{}{0pt}%
\pgfsys@defobject{currentmarker}{\pgfqpoint{-0.020833in}{-0.020833in}}{\pgfqpoint{0.020833in}{0.020833in}}{%
\pgfpathmoveto{\pgfqpoint{0.000000in}{-0.020833in}}%
\pgfpathcurveto{\pgfqpoint{0.005525in}{-0.020833in}}{\pgfqpoint{0.010825in}{-0.018638in}}{\pgfqpoint{0.014731in}{-0.014731in}}%
\pgfpathcurveto{\pgfqpoint{0.018638in}{-0.010825in}}{\pgfqpoint{0.020833in}{-0.005525in}}{\pgfqpoint{0.020833in}{0.000000in}}%
\pgfpathcurveto{\pgfqpoint{0.020833in}{0.005525in}}{\pgfqpoint{0.018638in}{0.010825in}}{\pgfqpoint{0.014731in}{0.014731in}}%
\pgfpathcurveto{\pgfqpoint{0.010825in}{0.018638in}}{\pgfqpoint{0.005525in}{0.020833in}}{\pgfqpoint{0.000000in}{0.020833in}}%
\pgfpathcurveto{\pgfqpoint{-0.005525in}{0.020833in}}{\pgfqpoint{-0.010825in}{0.018638in}}{\pgfqpoint{-0.014731in}{0.014731in}}%
\pgfpathcurveto{\pgfqpoint{-0.018638in}{0.010825in}}{\pgfqpoint{-0.020833in}{0.005525in}}{\pgfqpoint{-0.020833in}{0.000000in}}%
\pgfpathcurveto{\pgfqpoint{-0.020833in}{-0.005525in}}{\pgfqpoint{-0.018638in}{-0.010825in}}{\pgfqpoint{-0.014731in}{-0.014731in}}%
\pgfpathcurveto{\pgfqpoint{-0.010825in}{-0.018638in}}{\pgfqpoint{-0.005525in}{-0.020833in}}{\pgfqpoint{0.000000in}{-0.020833in}}%
\pgfpathlineto{\pgfqpoint{0.000000in}{-0.020833in}}%
\pgfpathclose%
\pgfusepath{stroke,fill}%
}%
\begin{pgfscope}%
\pgfsys@transformshift{0.908144in}{2.446846in}%
\pgfsys@useobject{currentmarker}{}%
\end{pgfscope}%
\begin{pgfscope}%
\pgfsys@transformshift{1.183494in}{2.149419in}%
\pgfsys@useobject{currentmarker}{}%
\end{pgfscope}%
\begin{pgfscope}%
\pgfsys@transformshift{1.458843in}{2.166128in}%
\pgfsys@useobject{currentmarker}{}%
\end{pgfscope}%
\begin{pgfscope}%
\pgfsys@transformshift{1.734193in}{2.226282in}%
\pgfsys@useobject{currentmarker}{}%
\end{pgfscope}%
\begin{pgfscope}%
\pgfsys@transformshift{2.009543in}{2.299803in}%
\pgfsys@useobject{currentmarker}{}%
\end{pgfscope}%
\begin{pgfscope}%
\pgfsys@transformshift{2.284892in}{2.316513in}%
\pgfsys@useobject{currentmarker}{}%
\end{pgfscope}%
\begin{pgfscope}%
\pgfsys@transformshift{2.560242in}{2.336564in}%
\pgfsys@useobject{currentmarker}{}%
\end{pgfscope}%
\begin{pgfscope}%
\pgfsys@transformshift{2.835591in}{2.266385in}%
\pgfsys@useobject{currentmarker}{}%
\end{pgfscope}%
\begin{pgfscope}%
\pgfsys@transformshift{3.110941in}{2.206231in}%
\pgfsys@useobject{currentmarker}{}%
\end{pgfscope}%
\begin{pgfscope}%
\pgfsys@transformshift{3.386291in}{2.169470in}%
\pgfsys@useobject{currentmarker}{}%
\end{pgfscope}%
\begin{pgfscope}%
\pgfsys@transformshift{3.661640in}{2.169470in}%
\pgfsys@useobject{currentmarker}{}%
\end{pgfscope}%
\begin{pgfscope}%
\pgfsys@transformshift{3.936990in}{1.985667in}%
\pgfsys@useobject{currentmarker}{}%
\end{pgfscope}%
\begin{pgfscope}%
\pgfsys@transformshift{4.212340in}{2.179496in}%
\pgfsys@useobject{currentmarker}{}%
\end{pgfscope}%
\begin{pgfscope}%
\pgfsys@transformshift{4.487689in}{2.159444in}%
\pgfsys@useobject{currentmarker}{}%
\end{pgfscope}%
\end{pgfscope}%
\begin{pgfscope}%
\pgfpathrectangle{\pgfqpoint{0.729167in}{0.638889in}}{\pgfqpoint{3.937500in}{2.172222in}}%
\pgfusepath{clip}%
\pgfsetrectcap%
\pgfsetroundjoin%
\pgfsetlinewidth{1.204500pt}%
\definecolor{currentstroke}{rgb}{0.337255,0.713725,0.627451}%
\pgfsetstrokecolor{currentstroke}%
\pgfsetdash{}{0pt}%
\pgfpathmoveto{\pgfqpoint{0.908144in}{1.561248in}}%
\pgfpathlineto{\pgfqpoint{1.183494in}{1.450966in}}%
\pgfpathlineto{\pgfqpoint{1.458843in}{1.430915in}}%
\pgfpathlineto{\pgfqpoint{1.734193in}{1.417547in}}%
\pgfpathlineto{\pgfqpoint{2.009543in}{1.440940in}}%
\pgfpathlineto{\pgfqpoint{2.284892in}{1.437598in}}%
\pgfpathlineto{\pgfqpoint{2.560242in}{1.430915in}}%
\pgfpathlineto{\pgfqpoint{2.835591in}{1.420889in}}%
\pgfpathlineto{\pgfqpoint{3.110941in}{1.367419in}}%
\pgfpathlineto{\pgfqpoint{3.386291in}{1.340684in}}%
\pgfpathlineto{\pgfqpoint{3.661640in}{1.313949in}}%
\pgfpathlineto{\pgfqpoint{3.936990in}{1.253795in}}%
\pgfpathlineto{\pgfqpoint{4.212340in}{1.313949in}}%
\pgfpathlineto{\pgfqpoint{4.487689in}{1.303923in}}%
\pgfusepath{stroke}%
\end{pgfscope}%
\begin{pgfscope}%
\pgfpathrectangle{\pgfqpoint{0.729167in}{0.638889in}}{\pgfqpoint{3.937500in}{2.172222in}}%
\pgfusepath{clip}%
\pgfsetbuttcap%
\pgfsetroundjoin%
\definecolor{currentfill}{rgb}{0.337255,0.713725,0.627451}%
\pgfsetfillcolor{currentfill}%
\pgfsetlinewidth{1.003750pt}%
\definecolor{currentstroke}{rgb}{0.337255,0.713725,0.627451}%
\pgfsetstrokecolor{currentstroke}%
\pgfsetdash{}{0pt}%
\pgfsys@defobject{currentmarker}{\pgfqpoint{-0.020833in}{-0.020833in}}{\pgfqpoint{0.020833in}{0.020833in}}{%
\pgfpathmoveto{\pgfqpoint{0.000000in}{-0.020833in}}%
\pgfpathcurveto{\pgfqpoint{0.005525in}{-0.020833in}}{\pgfqpoint{0.010825in}{-0.018638in}}{\pgfqpoint{0.014731in}{-0.014731in}}%
\pgfpathcurveto{\pgfqpoint{0.018638in}{-0.010825in}}{\pgfqpoint{0.020833in}{-0.005525in}}{\pgfqpoint{0.020833in}{0.000000in}}%
\pgfpathcurveto{\pgfqpoint{0.020833in}{0.005525in}}{\pgfqpoint{0.018638in}{0.010825in}}{\pgfqpoint{0.014731in}{0.014731in}}%
\pgfpathcurveto{\pgfqpoint{0.010825in}{0.018638in}}{\pgfqpoint{0.005525in}{0.020833in}}{\pgfqpoint{0.000000in}{0.020833in}}%
\pgfpathcurveto{\pgfqpoint{-0.005525in}{0.020833in}}{\pgfqpoint{-0.010825in}{0.018638in}}{\pgfqpoint{-0.014731in}{0.014731in}}%
\pgfpathcurveto{\pgfqpoint{-0.018638in}{0.010825in}}{\pgfqpoint{-0.020833in}{0.005525in}}{\pgfqpoint{-0.020833in}{0.000000in}}%
\pgfpathcurveto{\pgfqpoint{-0.020833in}{-0.005525in}}{\pgfqpoint{-0.018638in}{-0.010825in}}{\pgfqpoint{-0.014731in}{-0.014731in}}%
\pgfpathcurveto{\pgfqpoint{-0.010825in}{-0.018638in}}{\pgfqpoint{-0.005525in}{-0.020833in}}{\pgfqpoint{0.000000in}{-0.020833in}}%
\pgfpathlineto{\pgfqpoint{0.000000in}{-0.020833in}}%
\pgfpathclose%
\pgfusepath{stroke,fill}%
}%
\begin{pgfscope}%
\pgfsys@transformshift{0.908144in}{1.561248in}%
\pgfsys@useobject{currentmarker}{}%
\end{pgfscope}%
\begin{pgfscope}%
\pgfsys@transformshift{1.183494in}{1.450966in}%
\pgfsys@useobject{currentmarker}{}%
\end{pgfscope}%
\begin{pgfscope}%
\pgfsys@transformshift{1.458843in}{1.430915in}%
\pgfsys@useobject{currentmarker}{}%
\end{pgfscope}%
\begin{pgfscope}%
\pgfsys@transformshift{1.734193in}{1.417547in}%
\pgfsys@useobject{currentmarker}{}%
\end{pgfscope}%
\begin{pgfscope}%
\pgfsys@transformshift{2.009543in}{1.440940in}%
\pgfsys@useobject{currentmarker}{}%
\end{pgfscope}%
\begin{pgfscope}%
\pgfsys@transformshift{2.284892in}{1.437598in}%
\pgfsys@useobject{currentmarker}{}%
\end{pgfscope}%
\begin{pgfscope}%
\pgfsys@transformshift{2.560242in}{1.430915in}%
\pgfsys@useobject{currentmarker}{}%
\end{pgfscope}%
\begin{pgfscope}%
\pgfsys@transformshift{2.835591in}{1.420889in}%
\pgfsys@useobject{currentmarker}{}%
\end{pgfscope}%
\begin{pgfscope}%
\pgfsys@transformshift{3.110941in}{1.367419in}%
\pgfsys@useobject{currentmarker}{}%
\end{pgfscope}%
\begin{pgfscope}%
\pgfsys@transformshift{3.386291in}{1.340684in}%
\pgfsys@useobject{currentmarker}{}%
\end{pgfscope}%
\begin{pgfscope}%
\pgfsys@transformshift{3.661640in}{1.313949in}%
\pgfsys@useobject{currentmarker}{}%
\end{pgfscope}%
\begin{pgfscope}%
\pgfsys@transformshift{3.936990in}{1.253795in}%
\pgfsys@useobject{currentmarker}{}%
\end{pgfscope}%
\begin{pgfscope}%
\pgfsys@transformshift{4.212340in}{1.313949in}%
\pgfsys@useobject{currentmarker}{}%
\end{pgfscope}%
\begin{pgfscope}%
\pgfsys@transformshift{4.487689in}{1.303923in}%
\pgfsys@useobject{currentmarker}{}%
\end{pgfscope}%
\end{pgfscope}%
\begin{pgfscope}%
\pgfsetrectcap%
\pgfsetmiterjoin%
\pgfsetlinewidth{0.752812pt}%
\definecolor{currentstroke}{rgb}{0.000000,0.000000,0.000000}%
\pgfsetstrokecolor{currentstroke}%
\pgfsetdash{}{0pt}%
\pgfpathmoveto{\pgfqpoint{0.729167in}{0.638889in}}%
\pgfpathlineto{\pgfqpoint{0.729167in}{2.811111in}}%
\pgfusepath{stroke}%
\end{pgfscope}%
\begin{pgfscope}%
\pgfsetrectcap%
\pgfsetmiterjoin%
\pgfsetlinewidth{0.752812pt}%
\definecolor{currentstroke}{rgb}{0.000000,0.000000,0.000000}%
\pgfsetstrokecolor{currentstroke}%
\pgfsetdash{}{0pt}%
\pgfpathmoveto{\pgfqpoint{0.729167in}{0.638889in}}%
\pgfpathlineto{\pgfqpoint{4.666667in}{0.638889in}}%
\pgfusepath{stroke}%
\end{pgfscope}%
\begin{pgfscope}%
\definecolor{textcolor}{rgb}{0.133333,0.133333,0.133333}%
\pgfsetstrokecolor{textcolor}%
\pgfsetfillcolor{textcolor}%
\pgftext[x=0.908144in,y=2.502402in,,bottom]{\color{textcolor}{\ifdefined\pdftexversion\else\setmainfont{NanumMyeongjo}\rmfamily\fi\fontsize{7.000000}{8.400000}\selectfont\catcode`\^=\active\def^{\ifmmode\sp\else\^{}\fi}\catcode`\%=\active\def%{\%}54.1}}%
\end{pgfscope}%
\begin{pgfscope}%
\definecolor{textcolor}{rgb}{0.133333,0.133333,0.133333}%
\pgfsetstrokecolor{textcolor}%
\pgfsetfillcolor{textcolor}%
\pgftext[x=1.183494in,y=2.204974in,,bottom]{\color{textcolor}{\ifdefined\pdftexversion\else\setmainfont{NanumMyeongjo}\rmfamily\fi\fontsize{7.000000}{8.400000}\selectfont\catcode`\^=\active\def^{\ifmmode\sp\else\^{}\fi}\catcode`\%=\active\def%{\%}45.2}}%
\end{pgfscope}%
\begin{pgfscope}%
\definecolor{textcolor}{rgb}{0.133333,0.133333,0.133333}%
\pgfsetstrokecolor{textcolor}%
\pgfsetfillcolor{textcolor}%
\pgftext[x=1.458843in,y=2.221684in,,bottom]{\color{textcolor}{\ifdefined\pdftexversion\else\setmainfont{NanumMyeongjo}\rmfamily\fi\fontsize{7.000000}{8.400000}\selectfont\catcode`\^=\active\def^{\ifmmode\sp\else\^{}\fi}\catcode`\%=\active\def%{\%}45.7}}%
\end{pgfscope}%
\begin{pgfscope}%
\definecolor{textcolor}{rgb}{0.133333,0.133333,0.133333}%
\pgfsetstrokecolor{textcolor}%
\pgfsetfillcolor{textcolor}%
\pgftext[x=1.734193in,y=2.281838in,,bottom]{\color{textcolor}{\ifdefined\pdftexversion\else\setmainfont{NanumMyeongjo}\rmfamily\fi\fontsize{7.000000}{8.400000}\selectfont\catcode`\^=\active\def^{\ifmmode\sp\else\^{}\fi}\catcode`\%=\active\def%{\%}47.5}}%
\end{pgfscope}%
\begin{pgfscope}%
\definecolor{textcolor}{rgb}{0.133333,0.133333,0.133333}%
\pgfsetstrokecolor{textcolor}%
\pgfsetfillcolor{textcolor}%
\pgftext[x=2.009543in,y=2.355359in,,bottom]{\color{textcolor}{\ifdefined\pdftexversion\else\setmainfont{NanumMyeongjo}\rmfamily\fi\fontsize{7.000000}{8.400000}\selectfont\catcode`\^=\active\def^{\ifmmode\sp\else\^{}\fi}\catcode`\%=\active\def%{\%}49.7}}%
\end{pgfscope}%
\begin{pgfscope}%
\definecolor{textcolor}{rgb}{0.133333,0.133333,0.133333}%
\pgfsetstrokecolor{textcolor}%
\pgfsetfillcolor{textcolor}%
\pgftext[x=2.284892in,y=2.372068in,,bottom]{\color{textcolor}{\ifdefined\pdftexversion\else\setmainfont{NanumMyeongjo}\rmfamily\fi\fontsize{7.000000}{8.400000}\selectfont\catcode`\^=\active\def^{\ifmmode\sp\else\^{}\fi}\catcode`\%=\active\def%{\%}50.2}}%
\end{pgfscope}%
\begin{pgfscope}%
\definecolor{textcolor}{rgb}{0.133333,0.133333,0.133333}%
\pgfsetstrokecolor{textcolor}%
\pgfsetfillcolor{textcolor}%
\pgftext[x=2.560242in,y=2.392120in,,bottom]{\color{textcolor}{\ifdefined\pdftexversion\else\setmainfont{NanumMyeongjo}\rmfamily\fi\fontsize{7.000000}{8.400000}\selectfont\catcode`\^=\active\def^{\ifmmode\sp\else\^{}\fi}\catcode`\%=\active\def%{\%}50.8}}%
\end{pgfscope}%
\begin{pgfscope}%
\definecolor{textcolor}{rgb}{0.133333,0.133333,0.133333}%
\pgfsetstrokecolor{textcolor}%
\pgfsetfillcolor{textcolor}%
\pgftext[x=2.835591in,y=2.321940in,,bottom]{\color{textcolor}{\ifdefined\pdftexversion\else\setmainfont{NanumMyeongjo}\rmfamily\fi\fontsize{7.000000}{8.400000}\selectfont\catcode`\^=\active\def^{\ifmmode\sp\else\^{}\fi}\catcode`\%=\active\def%{\%}48.7}}%
\end{pgfscope}%
\begin{pgfscope}%
\definecolor{textcolor}{rgb}{0.133333,0.133333,0.133333}%
\pgfsetstrokecolor{textcolor}%
\pgfsetfillcolor{textcolor}%
\pgftext[x=3.110941in,y=2.261786in,,bottom]{\color{textcolor}{\ifdefined\pdftexversion\else\setmainfont{NanumMyeongjo}\rmfamily\fi\fontsize{7.000000}{8.400000}\selectfont\catcode`\^=\active\def^{\ifmmode\sp\else\^{}\fi}\catcode`\%=\active\def%{\%}46.9}}%
\end{pgfscope}%
\begin{pgfscope}%
\definecolor{textcolor}{rgb}{0.133333,0.133333,0.133333}%
\pgfsetstrokecolor{textcolor}%
\pgfsetfillcolor{textcolor}%
\pgftext[x=3.386291in,y=2.225026in,,bottom]{\color{textcolor}{\ifdefined\pdftexversion\else\setmainfont{NanumMyeongjo}\rmfamily\fi\fontsize{7.000000}{8.400000}\selectfont\catcode`\^=\active\def^{\ifmmode\sp\else\^{}\fi}\catcode`\%=\active\def%{\%}45.8}}%
\end{pgfscope}%
\begin{pgfscope}%
\definecolor{textcolor}{rgb}{0.133333,0.133333,0.133333}%
\pgfsetstrokecolor{textcolor}%
\pgfsetfillcolor{textcolor}%
\pgftext[x=3.661640in,y=2.225026in,,bottom]{\color{textcolor}{\ifdefined\pdftexversion\else\setmainfont{NanumMyeongjo}\rmfamily\fi\fontsize{7.000000}{8.400000}\selectfont\catcode`\^=\active\def^{\ifmmode\sp\else\^{}\fi}\catcode`\%=\active\def%{\%}45.8}}%
\end{pgfscope}%
\begin{pgfscope}%
\definecolor{textcolor}{rgb}{0.133333,0.133333,0.133333}%
\pgfsetstrokecolor{textcolor}%
\pgfsetfillcolor{textcolor}%
\pgftext[x=3.936990in,y=2.041222in,,bottom]{\color{textcolor}{\ifdefined\pdftexversion\else\setmainfont{NanumMyeongjo}\rmfamily\fi\fontsize{7.000000}{8.400000}\selectfont\catcode`\^=\active\def^{\ifmmode\sp\else\^{}\fi}\catcode`\%=\active\def%{\%}40.3}}%
\end{pgfscope}%
\begin{pgfscope}%
\definecolor{textcolor}{rgb}{0.133333,0.133333,0.133333}%
\pgfsetstrokecolor{textcolor}%
\pgfsetfillcolor{textcolor}%
\pgftext[x=4.212340in,y=2.235051in,,bottom]{\color{textcolor}{\ifdefined\pdftexversion\else\setmainfont{NanumMyeongjo}\rmfamily\fi\fontsize{7.000000}{8.400000}\selectfont\catcode`\^=\active\def^{\ifmmode\sp\else\^{}\fi}\catcode`\%=\active\def%{\%}46.1}}%
\end{pgfscope}%
\begin{pgfscope}%
\definecolor{textcolor}{rgb}{0.133333,0.133333,0.133333}%
\pgfsetstrokecolor{textcolor}%
\pgfsetfillcolor{textcolor}%
\pgftext[x=4.487689in,y=2.215000in,,bottom]{\color{textcolor}{\ifdefined\pdftexversion\else\setmainfont{NanumMyeongjo}\rmfamily\fi\fontsize{7.000000}{8.400000}\selectfont\catcode`\^=\active\def^{\ifmmode\sp\else\^{}\fi}\catcode`\%=\active\def%{\%}45.5}}%
\end{pgfscope}%
\begin{pgfscope}%
\definecolor{textcolor}{rgb}{0.133333,0.133333,0.133333}%
\pgfsetstrokecolor{textcolor}%
\pgfsetfillcolor{textcolor}%
\pgftext[x=0.908144in,y=1.616803in,,bottom]{\color{textcolor}{\ifdefined\pdftexversion\else\setmainfont{NanumMyeongjo}\rmfamily\fi\fontsize{7.000000}{8.400000}\selectfont\catcode`\^=\active\def^{\ifmmode\sp\else\^{}\fi}\catcode`\%=\active\def%{\%}27.6}}%
\end{pgfscope}%
\begin{pgfscope}%
\definecolor{textcolor}{rgb}{0.133333,0.133333,0.133333}%
\pgfsetstrokecolor{textcolor}%
\pgfsetfillcolor{textcolor}%
\pgftext[x=1.183494in,y=1.506521in,,bottom]{\color{textcolor}{\ifdefined\pdftexversion\else\setmainfont{NanumMyeongjo}\rmfamily\fi\fontsize{7.000000}{8.400000}\selectfont\catcode`\^=\active\def^{\ifmmode\sp\else\^{}\fi}\catcode`\%=\active\def%{\%}24.3}}%
\end{pgfscope}%
\begin{pgfscope}%
\definecolor{textcolor}{rgb}{0.133333,0.133333,0.133333}%
\pgfsetstrokecolor{textcolor}%
\pgfsetfillcolor{textcolor}%
\pgftext[x=1.458843in,y=1.486470in,,bottom]{\color{textcolor}{\ifdefined\pdftexversion\else\setmainfont{NanumMyeongjo}\rmfamily\fi\fontsize{7.000000}{8.400000}\selectfont\catcode`\^=\active\def^{\ifmmode\sp\else\^{}\fi}\catcode`\%=\active\def%{\%}23.7}}%
\end{pgfscope}%
\begin{pgfscope}%
\definecolor{textcolor}{rgb}{0.133333,0.133333,0.133333}%
\pgfsetstrokecolor{textcolor}%
\pgfsetfillcolor{textcolor}%
\pgftext[x=1.734193in,y=1.473103in,,bottom]{\color{textcolor}{\ifdefined\pdftexversion\else\setmainfont{NanumMyeongjo}\rmfamily\fi\fontsize{7.000000}{8.400000}\selectfont\catcode`\^=\active\def^{\ifmmode\sp\else\^{}\fi}\catcode`\%=\active\def%{\%}23.3}}%
\end{pgfscope}%
\begin{pgfscope}%
\definecolor{textcolor}{rgb}{0.133333,0.133333,0.133333}%
\pgfsetstrokecolor{textcolor}%
\pgfsetfillcolor{textcolor}%
\pgftext[x=2.009543in,y=1.496496in,,bottom]{\color{textcolor}{\ifdefined\pdftexversion\else\setmainfont{NanumMyeongjo}\rmfamily\fi\fontsize{7.000000}{8.400000}\selectfont\catcode`\^=\active\def^{\ifmmode\sp\else\^{}\fi}\catcode`\%=\active\def%{\%}24.0}}%
\end{pgfscope}%
\begin{pgfscope}%
\definecolor{textcolor}{rgb}{0.133333,0.133333,0.133333}%
\pgfsetstrokecolor{textcolor}%
\pgfsetfillcolor{textcolor}%
\pgftext[x=2.284892in,y=1.493154in,,bottom]{\color{textcolor}{\ifdefined\pdftexversion\else\setmainfont{NanumMyeongjo}\rmfamily\fi\fontsize{7.000000}{8.400000}\selectfont\catcode`\^=\active\def^{\ifmmode\sp\else\^{}\fi}\catcode`\%=\active\def%{\%}23.9}}%
\end{pgfscope}%
\begin{pgfscope}%
\definecolor{textcolor}{rgb}{0.133333,0.133333,0.133333}%
\pgfsetstrokecolor{textcolor}%
\pgfsetfillcolor{textcolor}%
\pgftext[x=2.560242in,y=1.486470in,,bottom]{\color{textcolor}{\ifdefined\pdftexversion\else\setmainfont{NanumMyeongjo}\rmfamily\fi\fontsize{7.000000}{8.400000}\selectfont\catcode`\^=\active\def^{\ifmmode\sp\else\^{}\fi}\catcode`\%=\active\def%{\%}23.7}}%
\end{pgfscope}%
\begin{pgfscope}%
\definecolor{textcolor}{rgb}{0.133333,0.133333,0.133333}%
\pgfsetstrokecolor{textcolor}%
\pgfsetfillcolor{textcolor}%
\pgftext[x=2.835591in,y=1.476444in,,bottom]{\color{textcolor}{\ifdefined\pdftexversion\else\setmainfont{NanumMyeongjo}\rmfamily\fi\fontsize{7.000000}{8.400000}\selectfont\catcode`\^=\active\def^{\ifmmode\sp\else\^{}\fi}\catcode`\%=\active\def%{\%}23.4}}%
\end{pgfscope}%
\begin{pgfscope}%
\definecolor{textcolor}{rgb}{0.133333,0.133333,0.133333}%
\pgfsetstrokecolor{textcolor}%
\pgfsetfillcolor{textcolor}%
\pgftext[x=3.110941in,y=1.422974in,,bottom]{\color{textcolor}{\ifdefined\pdftexversion\else\setmainfont{NanumMyeongjo}\rmfamily\fi\fontsize{7.000000}{8.400000}\selectfont\catcode`\^=\active\def^{\ifmmode\sp\else\^{}\fi}\catcode`\%=\active\def%{\%}21.8}}%
\end{pgfscope}%
\begin{pgfscope}%
\definecolor{textcolor}{rgb}{0.133333,0.133333,0.133333}%
\pgfsetstrokecolor{textcolor}%
\pgfsetfillcolor{textcolor}%
\pgftext[x=3.386291in,y=1.396239in,,bottom]{\color{textcolor}{\ifdefined\pdftexversion\else\setmainfont{NanumMyeongjo}\rmfamily\fi\fontsize{7.000000}{8.400000}\selectfont\catcode`\^=\active\def^{\ifmmode\sp\else\^{}\fi}\catcode`\%=\active\def%{\%}21.0}}%
\end{pgfscope}%
\begin{pgfscope}%
\definecolor{textcolor}{rgb}{0.133333,0.133333,0.133333}%
\pgfsetstrokecolor{textcolor}%
\pgfsetfillcolor{textcolor}%
\pgftext[x=3.661640in,y=1.369504in,,bottom]{\color{textcolor}{\ifdefined\pdftexversion\else\setmainfont{NanumMyeongjo}\rmfamily\fi\fontsize{7.000000}{8.400000}\selectfont\catcode`\^=\active\def^{\ifmmode\sp\else\^{}\fi}\catcode`\%=\active\def%{\%}20.2}}%
\end{pgfscope}%
\begin{pgfscope}%
\definecolor{textcolor}{rgb}{0.133333,0.133333,0.133333}%
\pgfsetstrokecolor{textcolor}%
\pgfsetfillcolor{textcolor}%
\pgftext[x=3.936990in,y=1.309350in,,bottom]{\color{textcolor}{\ifdefined\pdftexversion\else\setmainfont{NanumMyeongjo}\rmfamily\fi\fontsize{7.000000}{8.400000}\selectfont\catcode`\^=\active\def^{\ifmmode\sp\else\^{}\fi}\catcode`\%=\active\def%{\%}18.4}}%
\end{pgfscope}%
\begin{pgfscope}%
\definecolor{textcolor}{rgb}{0.133333,0.133333,0.133333}%
\pgfsetstrokecolor{textcolor}%
\pgfsetfillcolor{textcolor}%
\pgftext[x=4.212340in,y=1.369504in,,bottom]{\color{textcolor}{\ifdefined\pdftexversion\else\setmainfont{NanumMyeongjo}\rmfamily\fi\fontsize{7.000000}{8.400000}\selectfont\catcode`\^=\active\def^{\ifmmode\sp\else\^{}\fi}\catcode`\%=\active\def%{\%}20.2}}%
\end{pgfscope}%
\begin{pgfscope}%
\definecolor{textcolor}{rgb}{0.133333,0.133333,0.133333}%
\pgfsetstrokecolor{textcolor}%
\pgfsetfillcolor{textcolor}%
\pgftext[x=4.487689in,y=1.359479in,,bottom]{\color{textcolor}{\ifdefined\pdftexversion\else\setmainfont{NanumMyeongjo}\rmfamily\fi\fontsize{7.000000}{8.400000}\selectfont\catcode`\^=\active\def^{\ifmmode\sp\else\^{}\fi}\catcode`\%=\active\def%{\%}19.9}}%
\end{pgfscope}%
\begin{pgfscope}%
\pgfsetrectcap%
\pgfsetroundjoin%
\pgfsetlinewidth{1.204500pt}%
\definecolor{currentstroke}{rgb}{0.235294,0.490196,0.764706}%
\pgfsetstrokecolor{currentstroke}%
\pgfsetdash{}{0pt}%
\pgfpathmoveto{\pgfqpoint{4.779167in}{2.642508in}}%
\pgfpathlineto{\pgfqpoint{4.904167in}{2.642508in}}%
\pgfpathlineto{\pgfqpoint{5.029167in}{2.642508in}}%
\pgfusepath{stroke}%
\end{pgfscope}%
\begin{pgfscope}%
\pgfsetbuttcap%
\pgfsetroundjoin%
\definecolor{currentfill}{rgb}{0.235294,0.490196,0.764706}%
\pgfsetfillcolor{currentfill}%
\pgfsetlinewidth{1.003750pt}%
\definecolor{currentstroke}{rgb}{0.235294,0.490196,0.764706}%
\pgfsetstrokecolor{currentstroke}%
\pgfsetdash{}{0pt}%
\pgfsys@defobject{currentmarker}{\pgfqpoint{-0.020833in}{-0.020833in}}{\pgfqpoint{0.020833in}{0.020833in}}{%
\pgfpathmoveto{\pgfqpoint{0.000000in}{-0.020833in}}%
\pgfpathcurveto{\pgfqpoint{0.005525in}{-0.020833in}}{\pgfqpoint{0.010825in}{-0.018638in}}{\pgfqpoint{0.014731in}{-0.014731in}}%
\pgfpathcurveto{\pgfqpoint{0.018638in}{-0.010825in}}{\pgfqpoint{0.020833in}{-0.005525in}}{\pgfqpoint{0.020833in}{0.000000in}}%
\pgfpathcurveto{\pgfqpoint{0.020833in}{0.005525in}}{\pgfqpoint{0.018638in}{0.010825in}}{\pgfqpoint{0.014731in}{0.014731in}}%
\pgfpathcurveto{\pgfqpoint{0.010825in}{0.018638in}}{\pgfqpoint{0.005525in}{0.020833in}}{\pgfqpoint{0.000000in}{0.020833in}}%
\pgfpathcurveto{\pgfqpoint{-0.005525in}{0.020833in}}{\pgfqpoint{-0.010825in}{0.018638in}}{\pgfqpoint{-0.014731in}{0.014731in}}%
\pgfpathcurveto{\pgfqpoint{-0.018638in}{0.010825in}}{\pgfqpoint{-0.020833in}{0.005525in}}{\pgfqpoint{-0.020833in}{0.000000in}}%
\pgfpathcurveto{\pgfqpoint{-0.020833in}{-0.005525in}}{\pgfqpoint{-0.018638in}{-0.010825in}}{\pgfqpoint{-0.014731in}{-0.014731in}}%
\pgfpathcurveto{\pgfqpoint{-0.010825in}{-0.018638in}}{\pgfqpoint{-0.005525in}{-0.020833in}}{\pgfqpoint{0.000000in}{-0.020833in}}%
\pgfpathlineto{\pgfqpoint{0.000000in}{-0.020833in}}%
\pgfpathclose%
\pgfusepath{stroke,fill}%
}%
\begin{pgfscope}%
\pgfsys@transformshift{4.904167in}{2.642508in}%
\pgfsys@useobject{currentmarker}{}%
\end{pgfscope}%
\end{pgfscope}%
\begin{pgfscope}%
\definecolor{textcolor}{rgb}{0.000000,0.000000,0.000000}%
\pgfsetstrokecolor{textcolor}%
\pgfsetfillcolor{textcolor}%
\pgftext[x=5.129167in,y=2.598758in,left,base]{\color{textcolor}{\ifdefined\pdftexversion\else\setmainfont{NanumMyeongjo}\rmfamily\fi\fontsize{9.000000}{10.800000}\selectfont\catcode`\^=\active\def^{\ifmmode\sp\else\^{}\fi}\catcode`\%=\active\def%{\%}사료제외}}%
\end{pgfscope}%
\begin{pgfscope}%
\pgfsetrectcap%
\pgfsetroundjoin%
\pgfsetlinewidth{1.204500pt}%
\definecolor{currentstroke}{rgb}{0.337255,0.713725,0.627451}%
\pgfsetstrokecolor{currentstroke}%
\pgfsetdash{}{0pt}%
\pgfpathmoveto{\pgfqpoint{4.779167in}{2.451224in}}%
\pgfpathlineto{\pgfqpoint{4.904167in}{2.451224in}}%
\pgfpathlineto{\pgfqpoint{5.029167in}{2.451224in}}%
\pgfusepath{stroke}%
\end{pgfscope}%
\begin{pgfscope}%
\pgfsetbuttcap%
\pgfsetroundjoin%
\definecolor{currentfill}{rgb}{0.337255,0.713725,0.627451}%
\pgfsetfillcolor{currentfill}%
\pgfsetlinewidth{1.003750pt}%
\definecolor{currentstroke}{rgb}{0.337255,0.713725,0.627451}%
\pgfsetstrokecolor{currentstroke}%
\pgfsetdash{}{0pt}%
\pgfsys@defobject{currentmarker}{\pgfqpoint{-0.020833in}{-0.020833in}}{\pgfqpoint{0.020833in}{0.020833in}}{%
\pgfpathmoveto{\pgfqpoint{0.000000in}{-0.020833in}}%
\pgfpathcurveto{\pgfqpoint{0.005525in}{-0.020833in}}{\pgfqpoint{0.010825in}{-0.018638in}}{\pgfqpoint{0.014731in}{-0.014731in}}%
\pgfpathcurveto{\pgfqpoint{0.018638in}{-0.010825in}}{\pgfqpoint{0.020833in}{-0.005525in}}{\pgfqpoint{0.020833in}{0.000000in}}%
\pgfpathcurveto{\pgfqpoint{0.020833in}{0.005525in}}{\pgfqpoint{0.018638in}{0.010825in}}{\pgfqpoint{0.014731in}{0.014731in}}%
\pgfpathcurveto{\pgfqpoint{0.010825in}{0.018638in}}{\pgfqpoint{0.005525in}{0.020833in}}{\pgfqpoint{0.000000in}{0.020833in}}%
\pgfpathcurveto{\pgfqpoint{-0.005525in}{0.020833in}}{\pgfqpoint{-0.010825in}{0.018638in}}{\pgfqpoint{-0.014731in}{0.014731in}}%
\pgfpathcurveto{\pgfqpoint{-0.018638in}{0.010825in}}{\pgfqpoint{-0.020833in}{0.005525in}}{\pgfqpoint{-0.020833in}{0.000000in}}%
\pgfpathcurveto{\pgfqpoint{-0.020833in}{-0.005525in}}{\pgfqpoint{-0.018638in}{-0.010825in}}{\pgfqpoint{-0.014731in}{-0.014731in}}%
\pgfpathcurveto{\pgfqpoint{-0.010825in}{-0.018638in}}{\pgfqpoint{-0.005525in}{-0.020833in}}{\pgfqpoint{0.000000in}{-0.020833in}}%
\pgfpathlineto{\pgfqpoint{0.000000in}{-0.020833in}}%
\pgfpathclose%
\pgfusepath{stroke,fill}%
}%
\begin{pgfscope}%
\pgfsys@transformshift{4.904167in}{2.451224in}%
\pgfsys@useobject{currentmarker}{}%
\end{pgfscope}%
\end{pgfscope}%
\begin{pgfscope}%
\definecolor{textcolor}{rgb}{0.000000,0.000000,0.000000}%
\pgfsetstrokecolor{textcolor}%
\pgfsetfillcolor{textcolor}%
\pgftext[x=5.129167in,y=2.407474in,left,base]{\color{textcolor}{\ifdefined\pdftexversion\else\setmainfont{NanumMyeongjo}\rmfamily\fi\fontsize{9.000000}{10.800000}\selectfont\catcode`\^=\active\def^{\ifmmode\sp\else\^{}\fi}\catcode`\%=\active\def%{\%}사료포함}}%
\end{pgfscope}%
\begin{pgfscope}%
\definecolor{textcolor}{rgb}{0.333333,0.333333,0.333333}%
\pgfsetstrokecolor{textcolor}%
\pgfsetfillcolor{textcolor}%
\pgftext[x=2.041667in,y=0.159722in,,top]{\color{textcolor}{\ifdefined\pdftexversion\else\setmainfont{NanumMyeongjo}\rmfamily\fi\fontsize{9.000000}{10.800000}\selectfont\catcode`\^=\active\def^{\ifmmode\sp\else\^{}\fi}\catcode`\%=\active\def%{\%}출처: 국가농식품통계서비스(KASS) 자료 기반 저자 작성}}%
\end{pgfscope}%
\begin{pgfscope}%
\definecolor{textcolor}{rgb}{0.333333,0.333333,0.333333}%
\pgfsetstrokecolor{textcolor}%
\pgfsetfillcolor{textcolor}%
\pgftext[x=4.375000in,y=2.970833in,,top]{\color{textcolor}{\ifdefined\pdftexversion\else\setmainfont{NanumMyeongjo}\rmfamily\fi\fontsize{9.000000}{10.800000}\selectfont\catcode`\^=\active\def^{\ifmmode\sp\else\^{}\fi}\catcode`\%=\active\def%{\%}(단위: %)}}%
\end{pgfscope}%
\end{pgfpicture}%
\makeatother%
\endgroup%
}
\end{center}
}


\slide
{\maintitle}
{\chapterone}
{콩 자급률}{
\begin{center}
    \hspace*{-35pt}\resizebox{1.2\textwidth}{!}{%% Creator: Matplotlib, PGF backend
%%
%% To include the figure in your LaTeX document, write
%%   \input{<filename>.pgf}
%%
%% Make sure the required packages are loaded in your preamble
%%   \usepackage{pgf}
%%
%% Also ensure that all the required font packages are loaded; for instance,
%% the lmodern package is sometimes necessary when using math font.
%%   \usepackage{lmodern}
%%
%% Figures using additional raster images can only be included by \input if
%% they are in the same directory as the main LaTeX file. For loading figures
%% from other directories you can use the `import` package
%%   \usepackage{import}
%%
%% and then include the figures with
%%   \import{<path to file>}{<filename>.pgf}
%%
%% Matplotlib used the following preamble
%%   \def\mathdefault#1{#1}
%%   \everymath=\expandafter{\the\everymath\displaystyle}
%%   \IfFileExists{scrextend.sty}{
%%     \usepackage[fontsize=9.000000pt]{scrextend}
%%   }{
%%     \renewcommand{\normalsize}{\fontsize{9.000000}{10.800000}\selectfont}
%%     \normalsize
%%   }
%%   
%%   \ifdefined\pdftexversion\else  % non-pdftex case.
%%     \usepackage{fontspec}
%%     \setmainfont{DejaVuSerif.ttf}[Path=\detokenize{/home/user/.cache/pypoetry/virtualenvs/graph-KASAOWVd-py3.12/lib/python3.12/site-packages/matplotlib/mpl-data/fonts/ttf/}]
%%     \setsansfont{DejaVuSans.ttf}[Path=\detokenize{/home/user/.cache/pypoetry/virtualenvs/graph-KASAOWVd-py3.12/lib/python3.12/site-packages/matplotlib/mpl-data/fonts/ttf/}]
%%     \setmonofont{DejaVuSansMono.ttf}[Path=\detokenize{/home/user/.cache/pypoetry/virtualenvs/graph-KASAOWVd-py3.12/lib/python3.12/site-packages/matplotlib/mpl-data/fonts/ttf/}]
%%   \fi
%%   \makeatletter\@ifpackageloaded{underscore}{}{\usepackage[strings]{underscore}}\makeatother
%%
\begingroup%
\makeatletter%
\begin{pgfpicture}%
\pgfpathrectangle{\pgfpointorigin}{\pgfqpoint{6.250000in}{3.194444in}}%
\pgfusepath{use as bounding box, clip}%
\begin{pgfscope}%
\pgfsetbuttcap%
\pgfsetmiterjoin%
\definecolor{currentfill}{rgb}{1.000000,1.000000,1.000000}%
\pgfsetfillcolor{currentfill}%
\pgfsetlinewidth{0.000000pt}%
\definecolor{currentstroke}{rgb}{1.000000,1.000000,1.000000}%
\pgfsetstrokecolor{currentstroke}%
\pgfsetdash{}{0pt}%
\pgfpathmoveto{\pgfqpoint{0.000000in}{0.000000in}}%
\pgfpathlineto{\pgfqpoint{6.250000in}{0.000000in}}%
\pgfpathlineto{\pgfqpoint{6.250000in}{3.194444in}}%
\pgfpathlineto{\pgfqpoint{0.000000in}{3.194444in}}%
\pgfpathlineto{\pgfqpoint{0.000000in}{0.000000in}}%
\pgfpathclose%
\pgfusepath{fill}%
\end{pgfscope}%
\begin{pgfscope}%
\pgfsetbuttcap%
\pgfsetmiterjoin%
\definecolor{currentfill}{rgb}{1.000000,1.000000,1.000000}%
\pgfsetfillcolor{currentfill}%
\pgfsetlinewidth{0.000000pt}%
\definecolor{currentstroke}{rgb}{0.000000,0.000000,0.000000}%
\pgfsetstrokecolor{currentstroke}%
\pgfsetstrokeopacity{0.000000}%
\pgfsetdash{}{0pt}%
\pgfpathmoveto{\pgfqpoint{0.781250in}{0.638889in}}%
\pgfpathlineto{\pgfqpoint{5.000000in}{0.638889in}}%
\pgfpathlineto{\pgfqpoint{5.000000in}{2.811111in}}%
\pgfpathlineto{\pgfqpoint{0.781250in}{2.811111in}}%
\pgfpathlineto{\pgfqpoint{0.781250in}{0.638889in}}%
\pgfpathclose%
\pgfusepath{fill}%
\end{pgfscope}%
\begin{pgfscope}%
\pgfsetbuttcap%
\pgfsetroundjoin%
\definecolor{currentfill}{rgb}{0.000000,0.000000,0.000000}%
\pgfsetfillcolor{currentfill}%
\pgfsetlinewidth{0.752812pt}%
\definecolor{currentstroke}{rgb}{0.000000,0.000000,0.000000}%
\pgfsetstrokecolor{currentstroke}%
\pgfsetdash{}{0pt}%
\pgfsys@defobject{currentmarker}{\pgfqpoint{0.000000in}{-0.013889in}}{\pgfqpoint{0.000000in}{0.000000in}}{%
\pgfpathmoveto{\pgfqpoint{0.000000in}{0.000000in}}%
\pgfpathlineto{\pgfqpoint{0.000000in}{-0.013889in}}%
\pgfusepath{stroke,fill}%
}%
\begin{pgfscope}%
\pgfsys@transformshift{0.973011in}{0.638889in}%
\pgfsys@useobject{currentmarker}{}%
\end{pgfscope}%
\end{pgfscope}%
\begin{pgfscope}%
\definecolor{textcolor}{rgb}{0.000000,0.000000,0.000000}%
\pgfsetstrokecolor{textcolor}%
\pgfsetfillcolor{textcolor}%
\pgftext[x=0.903138in, y=0.315885in, left, base,rotate=45.000000]{\color{textcolor}{\ifdefined\pdftexversion\else\setmainfont{NanumMyeongjo}\rmfamily\fi\fontsize{9.000000}{10.800000}\selectfont\catcode`\^=\active\def^{\ifmmode\sp\else\^{}\fi}\catcode`\%=\active\def%{\%}2010}}%
\end{pgfscope}%
\begin{pgfscope}%
\pgfsetbuttcap%
\pgfsetroundjoin%
\definecolor{currentfill}{rgb}{0.000000,0.000000,0.000000}%
\pgfsetfillcolor{currentfill}%
\pgfsetlinewidth{0.752812pt}%
\definecolor{currentstroke}{rgb}{0.000000,0.000000,0.000000}%
\pgfsetstrokecolor{currentstroke}%
\pgfsetdash{}{0pt}%
\pgfsys@defobject{currentmarker}{\pgfqpoint{0.000000in}{-0.013889in}}{\pgfqpoint{0.000000in}{0.000000in}}{%
\pgfpathmoveto{\pgfqpoint{0.000000in}{0.000000in}}%
\pgfpathlineto{\pgfqpoint{0.000000in}{-0.013889in}}%
\pgfusepath{stroke,fill}%
}%
\begin{pgfscope}%
\pgfsys@transformshift{1.246956in}{0.638889in}%
\pgfsys@useobject{currentmarker}{}%
\end{pgfscope}%
\end{pgfscope}%
\begin{pgfscope}%
\definecolor{textcolor}{rgb}{0.000000,0.000000,0.000000}%
\pgfsetstrokecolor{textcolor}%
\pgfsetfillcolor{textcolor}%
\pgftext[x=1.177083in, y=0.315885in, left, base,rotate=45.000000]{\color{textcolor}{\ifdefined\pdftexversion\else\setmainfont{NanumMyeongjo}\rmfamily\fi\fontsize{9.000000}{10.800000}\selectfont\catcode`\^=\active\def^{\ifmmode\sp\else\^{}\fi}\catcode`\%=\active\def%{\%}2011}}%
\end{pgfscope}%
\begin{pgfscope}%
\pgfsetbuttcap%
\pgfsetroundjoin%
\definecolor{currentfill}{rgb}{0.000000,0.000000,0.000000}%
\pgfsetfillcolor{currentfill}%
\pgfsetlinewidth{0.752812pt}%
\definecolor{currentstroke}{rgb}{0.000000,0.000000,0.000000}%
\pgfsetstrokecolor{currentstroke}%
\pgfsetdash{}{0pt}%
\pgfsys@defobject{currentmarker}{\pgfqpoint{0.000000in}{-0.013889in}}{\pgfqpoint{0.000000in}{0.000000in}}{%
\pgfpathmoveto{\pgfqpoint{0.000000in}{0.000000in}}%
\pgfpathlineto{\pgfqpoint{0.000000in}{-0.013889in}}%
\pgfusepath{stroke,fill}%
}%
\begin{pgfscope}%
\pgfsys@transformshift{1.520901in}{0.638889in}%
\pgfsys@useobject{currentmarker}{}%
\end{pgfscope}%
\end{pgfscope}%
\begin{pgfscope}%
\definecolor{textcolor}{rgb}{0.000000,0.000000,0.000000}%
\pgfsetstrokecolor{textcolor}%
\pgfsetfillcolor{textcolor}%
\pgftext[x=1.451028in, y=0.315885in, left, base,rotate=45.000000]{\color{textcolor}{\ifdefined\pdftexversion\else\setmainfont{NanumMyeongjo}\rmfamily\fi\fontsize{9.000000}{10.800000}\selectfont\catcode`\^=\active\def^{\ifmmode\sp\else\^{}\fi}\catcode`\%=\active\def%{\%}2012}}%
\end{pgfscope}%
\begin{pgfscope}%
\pgfsetbuttcap%
\pgfsetroundjoin%
\definecolor{currentfill}{rgb}{0.000000,0.000000,0.000000}%
\pgfsetfillcolor{currentfill}%
\pgfsetlinewidth{0.752812pt}%
\definecolor{currentstroke}{rgb}{0.000000,0.000000,0.000000}%
\pgfsetstrokecolor{currentstroke}%
\pgfsetdash{}{0pt}%
\pgfsys@defobject{currentmarker}{\pgfqpoint{0.000000in}{-0.013889in}}{\pgfqpoint{0.000000in}{0.000000in}}{%
\pgfpathmoveto{\pgfqpoint{0.000000in}{0.000000in}}%
\pgfpathlineto{\pgfqpoint{0.000000in}{-0.013889in}}%
\pgfusepath{stroke,fill}%
}%
\begin{pgfscope}%
\pgfsys@transformshift{1.794846in}{0.638889in}%
\pgfsys@useobject{currentmarker}{}%
\end{pgfscope}%
\end{pgfscope}%
\begin{pgfscope}%
\definecolor{textcolor}{rgb}{0.000000,0.000000,0.000000}%
\pgfsetstrokecolor{textcolor}%
\pgfsetfillcolor{textcolor}%
\pgftext[x=1.724972in, y=0.315885in, left, base,rotate=45.000000]{\color{textcolor}{\ifdefined\pdftexversion\else\setmainfont{NanumMyeongjo}\rmfamily\fi\fontsize{9.000000}{10.800000}\selectfont\catcode`\^=\active\def^{\ifmmode\sp\else\^{}\fi}\catcode`\%=\active\def%{\%}2013}}%
\end{pgfscope}%
\begin{pgfscope}%
\pgfsetbuttcap%
\pgfsetroundjoin%
\definecolor{currentfill}{rgb}{0.000000,0.000000,0.000000}%
\pgfsetfillcolor{currentfill}%
\pgfsetlinewidth{0.752812pt}%
\definecolor{currentstroke}{rgb}{0.000000,0.000000,0.000000}%
\pgfsetstrokecolor{currentstroke}%
\pgfsetdash{}{0pt}%
\pgfsys@defobject{currentmarker}{\pgfqpoint{0.000000in}{-0.013889in}}{\pgfqpoint{0.000000in}{0.000000in}}{%
\pgfpathmoveto{\pgfqpoint{0.000000in}{0.000000in}}%
\pgfpathlineto{\pgfqpoint{0.000000in}{-0.013889in}}%
\pgfusepath{stroke,fill}%
}%
\begin{pgfscope}%
\pgfsys@transformshift{2.068791in}{0.638889in}%
\pgfsys@useobject{currentmarker}{}%
\end{pgfscope}%
\end{pgfscope}%
\begin{pgfscope}%
\definecolor{textcolor}{rgb}{0.000000,0.000000,0.000000}%
\pgfsetstrokecolor{textcolor}%
\pgfsetfillcolor{textcolor}%
\pgftext[x=1.998917in, y=0.315885in, left, base,rotate=45.000000]{\color{textcolor}{\ifdefined\pdftexversion\else\setmainfont{NanumMyeongjo}\rmfamily\fi\fontsize{9.000000}{10.800000}\selectfont\catcode`\^=\active\def^{\ifmmode\sp\else\^{}\fi}\catcode`\%=\active\def%{\%}2014}}%
\end{pgfscope}%
\begin{pgfscope}%
\pgfsetbuttcap%
\pgfsetroundjoin%
\definecolor{currentfill}{rgb}{0.000000,0.000000,0.000000}%
\pgfsetfillcolor{currentfill}%
\pgfsetlinewidth{0.752812pt}%
\definecolor{currentstroke}{rgb}{0.000000,0.000000,0.000000}%
\pgfsetstrokecolor{currentstroke}%
\pgfsetdash{}{0pt}%
\pgfsys@defobject{currentmarker}{\pgfqpoint{0.000000in}{-0.013889in}}{\pgfqpoint{0.000000in}{0.000000in}}{%
\pgfpathmoveto{\pgfqpoint{0.000000in}{0.000000in}}%
\pgfpathlineto{\pgfqpoint{0.000000in}{-0.013889in}}%
\pgfusepath{stroke,fill}%
}%
\begin{pgfscope}%
\pgfsys@transformshift{2.342735in}{0.638889in}%
\pgfsys@useobject{currentmarker}{}%
\end{pgfscope}%
\end{pgfscope}%
\begin{pgfscope}%
\definecolor{textcolor}{rgb}{0.000000,0.000000,0.000000}%
\pgfsetstrokecolor{textcolor}%
\pgfsetfillcolor{textcolor}%
\pgftext[x=2.272862in, y=0.315885in, left, base,rotate=45.000000]{\color{textcolor}{\ifdefined\pdftexversion\else\setmainfont{NanumMyeongjo}\rmfamily\fi\fontsize{9.000000}{10.800000}\selectfont\catcode`\^=\active\def^{\ifmmode\sp\else\^{}\fi}\catcode`\%=\active\def%{\%}2015}}%
\end{pgfscope}%
\begin{pgfscope}%
\pgfsetbuttcap%
\pgfsetroundjoin%
\definecolor{currentfill}{rgb}{0.000000,0.000000,0.000000}%
\pgfsetfillcolor{currentfill}%
\pgfsetlinewidth{0.752812pt}%
\definecolor{currentstroke}{rgb}{0.000000,0.000000,0.000000}%
\pgfsetstrokecolor{currentstroke}%
\pgfsetdash{}{0pt}%
\pgfsys@defobject{currentmarker}{\pgfqpoint{0.000000in}{-0.013889in}}{\pgfqpoint{0.000000in}{0.000000in}}{%
\pgfpathmoveto{\pgfqpoint{0.000000in}{0.000000in}}%
\pgfpathlineto{\pgfqpoint{0.000000in}{-0.013889in}}%
\pgfusepath{stroke,fill}%
}%
\begin{pgfscope}%
\pgfsys@transformshift{2.616680in}{0.638889in}%
\pgfsys@useobject{currentmarker}{}%
\end{pgfscope}%
\end{pgfscope}%
\begin{pgfscope}%
\definecolor{textcolor}{rgb}{0.000000,0.000000,0.000000}%
\pgfsetstrokecolor{textcolor}%
\pgfsetfillcolor{textcolor}%
\pgftext[x=2.546807in, y=0.315885in, left, base,rotate=45.000000]{\color{textcolor}{\ifdefined\pdftexversion\else\setmainfont{NanumMyeongjo}\rmfamily\fi\fontsize{9.000000}{10.800000}\selectfont\catcode`\^=\active\def^{\ifmmode\sp\else\^{}\fi}\catcode`\%=\active\def%{\%}2016}}%
\end{pgfscope}%
\begin{pgfscope}%
\pgfsetbuttcap%
\pgfsetroundjoin%
\definecolor{currentfill}{rgb}{0.000000,0.000000,0.000000}%
\pgfsetfillcolor{currentfill}%
\pgfsetlinewidth{0.752812pt}%
\definecolor{currentstroke}{rgb}{0.000000,0.000000,0.000000}%
\pgfsetstrokecolor{currentstroke}%
\pgfsetdash{}{0pt}%
\pgfsys@defobject{currentmarker}{\pgfqpoint{0.000000in}{-0.013889in}}{\pgfqpoint{0.000000in}{0.000000in}}{%
\pgfpathmoveto{\pgfqpoint{0.000000in}{0.000000in}}%
\pgfpathlineto{\pgfqpoint{0.000000in}{-0.013889in}}%
\pgfusepath{stroke,fill}%
}%
\begin{pgfscope}%
\pgfsys@transformshift{2.890625in}{0.638889in}%
\pgfsys@useobject{currentmarker}{}%
\end{pgfscope}%
\end{pgfscope}%
\begin{pgfscope}%
\definecolor{textcolor}{rgb}{0.000000,0.000000,0.000000}%
\pgfsetstrokecolor{textcolor}%
\pgfsetfillcolor{textcolor}%
\pgftext[x=2.820752in, y=0.315885in, left, base,rotate=45.000000]{\color{textcolor}{\ifdefined\pdftexversion\else\setmainfont{NanumMyeongjo}\rmfamily\fi\fontsize{9.000000}{10.800000}\selectfont\catcode`\^=\active\def^{\ifmmode\sp\else\^{}\fi}\catcode`\%=\active\def%{\%}2017}}%
\end{pgfscope}%
\begin{pgfscope}%
\pgfsetbuttcap%
\pgfsetroundjoin%
\definecolor{currentfill}{rgb}{0.000000,0.000000,0.000000}%
\pgfsetfillcolor{currentfill}%
\pgfsetlinewidth{0.752812pt}%
\definecolor{currentstroke}{rgb}{0.000000,0.000000,0.000000}%
\pgfsetstrokecolor{currentstroke}%
\pgfsetdash{}{0pt}%
\pgfsys@defobject{currentmarker}{\pgfqpoint{0.000000in}{-0.013889in}}{\pgfqpoint{0.000000in}{0.000000in}}{%
\pgfpathmoveto{\pgfqpoint{0.000000in}{0.000000in}}%
\pgfpathlineto{\pgfqpoint{0.000000in}{-0.013889in}}%
\pgfusepath{stroke,fill}%
}%
\begin{pgfscope}%
\pgfsys@transformshift{3.164570in}{0.638889in}%
\pgfsys@useobject{currentmarker}{}%
\end{pgfscope}%
\end{pgfscope}%
\begin{pgfscope}%
\definecolor{textcolor}{rgb}{0.000000,0.000000,0.000000}%
\pgfsetstrokecolor{textcolor}%
\pgfsetfillcolor{textcolor}%
\pgftext[x=3.094696in, y=0.315885in, left, base,rotate=45.000000]{\color{textcolor}{\ifdefined\pdftexversion\else\setmainfont{NanumMyeongjo}\rmfamily\fi\fontsize{9.000000}{10.800000}\selectfont\catcode`\^=\active\def^{\ifmmode\sp\else\^{}\fi}\catcode`\%=\active\def%{\%}2018}}%
\end{pgfscope}%
\begin{pgfscope}%
\pgfsetbuttcap%
\pgfsetroundjoin%
\definecolor{currentfill}{rgb}{0.000000,0.000000,0.000000}%
\pgfsetfillcolor{currentfill}%
\pgfsetlinewidth{0.752812pt}%
\definecolor{currentstroke}{rgb}{0.000000,0.000000,0.000000}%
\pgfsetstrokecolor{currentstroke}%
\pgfsetdash{}{0pt}%
\pgfsys@defobject{currentmarker}{\pgfqpoint{0.000000in}{-0.013889in}}{\pgfqpoint{0.000000in}{0.000000in}}{%
\pgfpathmoveto{\pgfqpoint{0.000000in}{0.000000in}}%
\pgfpathlineto{\pgfqpoint{0.000000in}{-0.013889in}}%
\pgfusepath{stroke,fill}%
}%
\begin{pgfscope}%
\pgfsys@transformshift{3.438515in}{0.638889in}%
\pgfsys@useobject{currentmarker}{}%
\end{pgfscope}%
\end{pgfscope}%
\begin{pgfscope}%
\definecolor{textcolor}{rgb}{0.000000,0.000000,0.000000}%
\pgfsetstrokecolor{textcolor}%
\pgfsetfillcolor{textcolor}%
\pgftext[x=3.368641in, y=0.315885in, left, base,rotate=45.000000]{\color{textcolor}{\ifdefined\pdftexversion\else\setmainfont{NanumMyeongjo}\rmfamily\fi\fontsize{9.000000}{10.800000}\selectfont\catcode`\^=\active\def^{\ifmmode\sp\else\^{}\fi}\catcode`\%=\active\def%{\%}2019}}%
\end{pgfscope}%
\begin{pgfscope}%
\pgfsetbuttcap%
\pgfsetroundjoin%
\definecolor{currentfill}{rgb}{0.000000,0.000000,0.000000}%
\pgfsetfillcolor{currentfill}%
\pgfsetlinewidth{0.752812pt}%
\definecolor{currentstroke}{rgb}{0.000000,0.000000,0.000000}%
\pgfsetstrokecolor{currentstroke}%
\pgfsetdash{}{0pt}%
\pgfsys@defobject{currentmarker}{\pgfqpoint{0.000000in}{-0.013889in}}{\pgfqpoint{0.000000in}{0.000000in}}{%
\pgfpathmoveto{\pgfqpoint{0.000000in}{0.000000in}}%
\pgfpathlineto{\pgfqpoint{0.000000in}{-0.013889in}}%
\pgfusepath{stroke,fill}%
}%
\begin{pgfscope}%
\pgfsys@transformshift{3.712459in}{0.638889in}%
\pgfsys@useobject{currentmarker}{}%
\end{pgfscope}%
\end{pgfscope}%
\begin{pgfscope}%
\definecolor{textcolor}{rgb}{0.000000,0.000000,0.000000}%
\pgfsetstrokecolor{textcolor}%
\pgfsetfillcolor{textcolor}%
\pgftext[x=3.642586in, y=0.315885in, left, base,rotate=45.000000]{\color{textcolor}{\ifdefined\pdftexversion\else\setmainfont{NanumMyeongjo}\rmfamily\fi\fontsize{9.000000}{10.800000}\selectfont\catcode`\^=\active\def^{\ifmmode\sp\else\^{}\fi}\catcode`\%=\active\def%{\%}2020}}%
\end{pgfscope}%
\begin{pgfscope}%
\pgfsetbuttcap%
\pgfsetroundjoin%
\definecolor{currentfill}{rgb}{0.000000,0.000000,0.000000}%
\pgfsetfillcolor{currentfill}%
\pgfsetlinewidth{0.752812pt}%
\definecolor{currentstroke}{rgb}{0.000000,0.000000,0.000000}%
\pgfsetstrokecolor{currentstroke}%
\pgfsetdash{}{0pt}%
\pgfsys@defobject{currentmarker}{\pgfqpoint{0.000000in}{-0.013889in}}{\pgfqpoint{0.000000in}{0.000000in}}{%
\pgfpathmoveto{\pgfqpoint{0.000000in}{0.000000in}}%
\pgfpathlineto{\pgfqpoint{0.000000in}{-0.013889in}}%
\pgfusepath{stroke,fill}%
}%
\begin{pgfscope}%
\pgfsys@transformshift{3.986404in}{0.638889in}%
\pgfsys@useobject{currentmarker}{}%
\end{pgfscope}%
\end{pgfscope}%
\begin{pgfscope}%
\definecolor{textcolor}{rgb}{0.000000,0.000000,0.000000}%
\pgfsetstrokecolor{textcolor}%
\pgfsetfillcolor{textcolor}%
\pgftext[x=3.916531in, y=0.315885in, left, base,rotate=45.000000]{\color{textcolor}{\ifdefined\pdftexversion\else\setmainfont{NanumMyeongjo}\rmfamily\fi\fontsize{9.000000}{10.800000}\selectfont\catcode`\^=\active\def^{\ifmmode\sp\else\^{}\fi}\catcode`\%=\active\def%{\%}2021}}%
\end{pgfscope}%
\begin{pgfscope}%
\pgfsetbuttcap%
\pgfsetroundjoin%
\definecolor{currentfill}{rgb}{0.000000,0.000000,0.000000}%
\pgfsetfillcolor{currentfill}%
\pgfsetlinewidth{0.752812pt}%
\definecolor{currentstroke}{rgb}{0.000000,0.000000,0.000000}%
\pgfsetstrokecolor{currentstroke}%
\pgfsetdash{}{0pt}%
\pgfsys@defobject{currentmarker}{\pgfqpoint{0.000000in}{-0.013889in}}{\pgfqpoint{0.000000in}{0.000000in}}{%
\pgfpathmoveto{\pgfqpoint{0.000000in}{0.000000in}}%
\pgfpathlineto{\pgfqpoint{0.000000in}{-0.013889in}}%
\pgfusepath{stroke,fill}%
}%
\begin{pgfscope}%
\pgfsys@transformshift{4.260349in}{0.638889in}%
\pgfsys@useobject{currentmarker}{}%
\end{pgfscope}%
\end{pgfscope}%
\begin{pgfscope}%
\definecolor{textcolor}{rgb}{0.000000,0.000000,0.000000}%
\pgfsetstrokecolor{textcolor}%
\pgfsetfillcolor{textcolor}%
\pgftext[x=4.190476in, y=0.315885in, left, base,rotate=45.000000]{\color{textcolor}{\ifdefined\pdftexversion\else\setmainfont{NanumMyeongjo}\rmfamily\fi\fontsize{9.000000}{10.800000}\selectfont\catcode`\^=\active\def^{\ifmmode\sp\else\^{}\fi}\catcode`\%=\active\def%{\%}2022}}%
\end{pgfscope}%
\begin{pgfscope}%
\pgfsetbuttcap%
\pgfsetroundjoin%
\definecolor{currentfill}{rgb}{0.000000,0.000000,0.000000}%
\pgfsetfillcolor{currentfill}%
\pgfsetlinewidth{0.752812pt}%
\definecolor{currentstroke}{rgb}{0.000000,0.000000,0.000000}%
\pgfsetstrokecolor{currentstroke}%
\pgfsetdash{}{0pt}%
\pgfsys@defobject{currentmarker}{\pgfqpoint{0.000000in}{-0.013889in}}{\pgfqpoint{0.000000in}{0.000000in}}{%
\pgfpathmoveto{\pgfqpoint{0.000000in}{0.000000in}}%
\pgfpathlineto{\pgfqpoint{0.000000in}{-0.013889in}}%
\pgfusepath{stroke,fill}%
}%
\begin{pgfscope}%
\pgfsys@transformshift{4.534294in}{0.638889in}%
\pgfsys@useobject{currentmarker}{}%
\end{pgfscope}%
\end{pgfscope}%
\begin{pgfscope}%
\definecolor{textcolor}{rgb}{0.000000,0.000000,0.000000}%
\pgfsetstrokecolor{textcolor}%
\pgfsetfillcolor{textcolor}%
\pgftext[x=4.464420in, y=0.315885in, left, base,rotate=45.000000]{\color{textcolor}{\ifdefined\pdftexversion\else\setmainfont{NanumMyeongjo}\rmfamily\fi\fontsize{9.000000}{10.800000}\selectfont\catcode`\^=\active\def^{\ifmmode\sp\else\^{}\fi}\catcode`\%=\active\def%{\%}2023}}%
\end{pgfscope}%
\begin{pgfscope}%
\pgfsetbuttcap%
\pgfsetroundjoin%
\definecolor{currentfill}{rgb}{0.000000,0.000000,0.000000}%
\pgfsetfillcolor{currentfill}%
\pgfsetlinewidth{0.752812pt}%
\definecolor{currentstroke}{rgb}{0.000000,0.000000,0.000000}%
\pgfsetstrokecolor{currentstroke}%
\pgfsetdash{}{0pt}%
\pgfsys@defobject{currentmarker}{\pgfqpoint{0.000000in}{-0.013889in}}{\pgfqpoint{0.000000in}{0.000000in}}{%
\pgfpathmoveto{\pgfqpoint{0.000000in}{0.000000in}}%
\pgfpathlineto{\pgfqpoint{0.000000in}{-0.013889in}}%
\pgfusepath{stroke,fill}%
}%
\begin{pgfscope}%
\pgfsys@transformshift{4.808239in}{0.638889in}%
\pgfsys@useobject{currentmarker}{}%
\end{pgfscope}%
\end{pgfscope}%
\begin{pgfscope}%
\definecolor{textcolor}{rgb}{0.000000,0.000000,0.000000}%
\pgfsetstrokecolor{textcolor}%
\pgfsetfillcolor{textcolor}%
\pgftext[x=4.738365in, y=0.315885in, left, base,rotate=45.000000]{\color{textcolor}{\ifdefined\pdftexversion\else\setmainfont{NanumMyeongjo}\rmfamily\fi\fontsize{9.000000}{10.800000}\selectfont\catcode`\^=\active\def^{\ifmmode\sp\else\^{}\fi}\catcode`\%=\active\def%{\%}2024}}%
\end{pgfscope}%
\begin{pgfscope}%
\pgfpathrectangle{\pgfqpoint{0.781250in}{0.638889in}}{\pgfqpoint{4.218750in}{2.172222in}}%
\pgfusepath{clip}%
\pgfsetbuttcap%
\pgfsetroundjoin%
\pgfsetlinewidth{0.602250pt}%
\definecolor{currentstroke}{rgb}{0.690196,0.690196,0.690196}%
\pgfsetstrokecolor{currentstroke}%
\pgfsetstrokeopacity{0.400000}%
\pgfsetdash{{2.220000pt}{0.960000pt}}{0.000000pt}%
\pgfpathmoveto{\pgfqpoint{0.781250in}{0.638889in}}%
\pgfpathlineto{\pgfqpoint{5.000000in}{0.638889in}}%
\pgfusepath{stroke}%
\end{pgfscope}%
\begin{pgfscope}%
\pgfsetbuttcap%
\pgfsetroundjoin%
\definecolor{currentfill}{rgb}{0.000000,0.000000,0.000000}%
\pgfsetfillcolor{currentfill}%
\pgfsetlinewidth{0.752812pt}%
\definecolor{currentstroke}{rgb}{0.000000,0.000000,0.000000}%
\pgfsetstrokecolor{currentstroke}%
\pgfsetdash{}{0pt}%
\pgfsys@defobject{currentmarker}{\pgfqpoint{-0.013889in}{0.000000in}}{\pgfqpoint{-0.000000in}{0.000000in}}{%
\pgfpathmoveto{\pgfqpoint{-0.000000in}{0.000000in}}%
\pgfpathlineto{\pgfqpoint{-0.013889in}{0.000000in}}%
\pgfusepath{stroke,fill}%
}%
\begin{pgfscope}%
\pgfsys@transformshift{0.781250in}{0.638889in}%
\pgfsys@useobject{currentmarker}{}%
\end{pgfscope}%
\end{pgfscope}%
\begin{pgfscope}%
\definecolor{textcolor}{rgb}{0.000000,0.000000,0.000000}%
\pgfsetstrokecolor{textcolor}%
\pgfsetfillcolor{textcolor}%
\pgftext[x=0.651611in, y=0.588962in, left, base]{\color{textcolor}{\ifdefined\pdftexversion\else\setmainfont{NanumMyeongjo}\rmfamily\fi\fontsize{9.000000}{10.800000}\selectfont\catcode`\^=\active\def^{\ifmmode\sp\else\^{}\fi}\catcode`\%=\active\def%{\%}0}}%
\end{pgfscope}%
\begin{pgfscope}%
\pgfpathrectangle{\pgfqpoint{0.781250in}{0.638889in}}{\pgfqpoint{4.218750in}{2.172222in}}%
\pgfusepath{clip}%
\pgfsetbuttcap%
\pgfsetroundjoin%
\pgfsetlinewidth{0.602250pt}%
\definecolor{currentstroke}{rgb}{0.690196,0.690196,0.690196}%
\pgfsetstrokecolor{currentstroke}%
\pgfsetstrokeopacity{0.400000}%
\pgfsetdash{{2.220000pt}{0.960000pt}}{0.000000pt}%
\pgfpathmoveto{\pgfqpoint{0.781250in}{0.880247in}}%
\pgfpathlineto{\pgfqpoint{5.000000in}{0.880247in}}%
\pgfusepath{stroke}%
\end{pgfscope}%
\begin{pgfscope}%
\pgfsetbuttcap%
\pgfsetroundjoin%
\definecolor{currentfill}{rgb}{0.000000,0.000000,0.000000}%
\pgfsetfillcolor{currentfill}%
\pgfsetlinewidth{0.752812pt}%
\definecolor{currentstroke}{rgb}{0.000000,0.000000,0.000000}%
\pgfsetstrokecolor{currentstroke}%
\pgfsetdash{}{0pt}%
\pgfsys@defobject{currentmarker}{\pgfqpoint{-0.013889in}{0.000000in}}{\pgfqpoint{-0.000000in}{0.000000in}}{%
\pgfpathmoveto{\pgfqpoint{-0.000000in}{0.000000in}}%
\pgfpathlineto{\pgfqpoint{-0.013889in}{0.000000in}}%
\pgfusepath{stroke,fill}%
}%
\begin{pgfscope}%
\pgfsys@transformshift{0.781250in}{0.880247in}%
\pgfsys@useobject{currentmarker}{}%
\end{pgfscope}%
\end{pgfscope}%
\begin{pgfscope}%
\definecolor{textcolor}{rgb}{0.000000,0.000000,0.000000}%
\pgfsetstrokecolor{textcolor}%
\pgfsetfillcolor{textcolor}%
\pgftext[x=0.651611in, y=0.830320in, left, base]{\color{textcolor}{\ifdefined\pdftexversion\else\setmainfont{NanumMyeongjo}\rmfamily\fi\fontsize{9.000000}{10.800000}\selectfont\catcode`\^=\active\def^{\ifmmode\sp\else\^{}\fi}\catcode`\%=\active\def%{\%}5}}%
\end{pgfscope}%
\begin{pgfscope}%
\pgfpathrectangle{\pgfqpoint{0.781250in}{0.638889in}}{\pgfqpoint{4.218750in}{2.172222in}}%
\pgfusepath{clip}%
\pgfsetbuttcap%
\pgfsetroundjoin%
\pgfsetlinewidth{0.602250pt}%
\definecolor{currentstroke}{rgb}{0.690196,0.690196,0.690196}%
\pgfsetstrokecolor{currentstroke}%
\pgfsetstrokeopacity{0.400000}%
\pgfsetdash{{2.220000pt}{0.960000pt}}{0.000000pt}%
\pgfpathmoveto{\pgfqpoint{0.781250in}{1.121605in}}%
\pgfpathlineto{\pgfqpoint{5.000000in}{1.121605in}}%
\pgfusepath{stroke}%
\end{pgfscope}%
\begin{pgfscope}%
\pgfsetbuttcap%
\pgfsetroundjoin%
\definecolor{currentfill}{rgb}{0.000000,0.000000,0.000000}%
\pgfsetfillcolor{currentfill}%
\pgfsetlinewidth{0.752812pt}%
\definecolor{currentstroke}{rgb}{0.000000,0.000000,0.000000}%
\pgfsetstrokecolor{currentstroke}%
\pgfsetdash{}{0pt}%
\pgfsys@defobject{currentmarker}{\pgfqpoint{-0.013889in}{0.000000in}}{\pgfqpoint{-0.000000in}{0.000000in}}{%
\pgfpathmoveto{\pgfqpoint{-0.000000in}{0.000000in}}%
\pgfpathlineto{\pgfqpoint{-0.013889in}{0.000000in}}%
\pgfusepath{stroke,fill}%
}%
\begin{pgfscope}%
\pgfsys@transformshift{0.781250in}{1.121605in}%
\pgfsys@useobject{currentmarker}{}%
\end{pgfscope}%
\end{pgfscope}%
\begin{pgfscope}%
\definecolor{textcolor}{rgb}{0.000000,0.000000,0.000000}%
\pgfsetstrokecolor{textcolor}%
\pgfsetfillcolor{textcolor}%
\pgftext[x=0.584473in, y=1.071678in, left, base]{\color{textcolor}{\ifdefined\pdftexversion\else\setmainfont{NanumMyeongjo}\rmfamily\fi\fontsize{9.000000}{10.800000}\selectfont\catcode`\^=\active\def^{\ifmmode\sp\else\^{}\fi}\catcode`\%=\active\def%{\%}10}}%
\end{pgfscope}%
\begin{pgfscope}%
\pgfpathrectangle{\pgfqpoint{0.781250in}{0.638889in}}{\pgfqpoint{4.218750in}{2.172222in}}%
\pgfusepath{clip}%
\pgfsetbuttcap%
\pgfsetroundjoin%
\pgfsetlinewidth{0.602250pt}%
\definecolor{currentstroke}{rgb}{0.690196,0.690196,0.690196}%
\pgfsetstrokecolor{currentstroke}%
\pgfsetstrokeopacity{0.400000}%
\pgfsetdash{{2.220000pt}{0.960000pt}}{0.000000pt}%
\pgfpathmoveto{\pgfqpoint{0.781250in}{1.362963in}}%
\pgfpathlineto{\pgfqpoint{5.000000in}{1.362963in}}%
\pgfusepath{stroke}%
\end{pgfscope}%
\begin{pgfscope}%
\pgfsetbuttcap%
\pgfsetroundjoin%
\definecolor{currentfill}{rgb}{0.000000,0.000000,0.000000}%
\pgfsetfillcolor{currentfill}%
\pgfsetlinewidth{0.752812pt}%
\definecolor{currentstroke}{rgb}{0.000000,0.000000,0.000000}%
\pgfsetstrokecolor{currentstroke}%
\pgfsetdash{}{0pt}%
\pgfsys@defobject{currentmarker}{\pgfqpoint{-0.013889in}{0.000000in}}{\pgfqpoint{-0.000000in}{0.000000in}}{%
\pgfpathmoveto{\pgfqpoint{-0.000000in}{0.000000in}}%
\pgfpathlineto{\pgfqpoint{-0.013889in}{0.000000in}}%
\pgfusepath{stroke,fill}%
}%
\begin{pgfscope}%
\pgfsys@transformshift{0.781250in}{1.362963in}%
\pgfsys@useobject{currentmarker}{}%
\end{pgfscope}%
\end{pgfscope}%
\begin{pgfscope}%
\definecolor{textcolor}{rgb}{0.000000,0.000000,0.000000}%
\pgfsetstrokecolor{textcolor}%
\pgfsetfillcolor{textcolor}%
\pgftext[x=0.584473in, y=1.313036in, left, base]{\color{textcolor}{\ifdefined\pdftexversion\else\setmainfont{NanumMyeongjo}\rmfamily\fi\fontsize{9.000000}{10.800000}\selectfont\catcode`\^=\active\def^{\ifmmode\sp\else\^{}\fi}\catcode`\%=\active\def%{\%}15}}%
\end{pgfscope}%
\begin{pgfscope}%
\pgfpathrectangle{\pgfqpoint{0.781250in}{0.638889in}}{\pgfqpoint{4.218750in}{2.172222in}}%
\pgfusepath{clip}%
\pgfsetbuttcap%
\pgfsetroundjoin%
\pgfsetlinewidth{0.602250pt}%
\definecolor{currentstroke}{rgb}{0.690196,0.690196,0.690196}%
\pgfsetstrokecolor{currentstroke}%
\pgfsetstrokeopacity{0.400000}%
\pgfsetdash{{2.220000pt}{0.960000pt}}{0.000000pt}%
\pgfpathmoveto{\pgfqpoint{0.781250in}{1.604321in}}%
\pgfpathlineto{\pgfqpoint{5.000000in}{1.604321in}}%
\pgfusepath{stroke}%
\end{pgfscope}%
\begin{pgfscope}%
\pgfsetbuttcap%
\pgfsetroundjoin%
\definecolor{currentfill}{rgb}{0.000000,0.000000,0.000000}%
\pgfsetfillcolor{currentfill}%
\pgfsetlinewidth{0.752812pt}%
\definecolor{currentstroke}{rgb}{0.000000,0.000000,0.000000}%
\pgfsetstrokecolor{currentstroke}%
\pgfsetdash{}{0pt}%
\pgfsys@defobject{currentmarker}{\pgfqpoint{-0.013889in}{0.000000in}}{\pgfqpoint{-0.000000in}{0.000000in}}{%
\pgfpathmoveto{\pgfqpoint{-0.000000in}{0.000000in}}%
\pgfpathlineto{\pgfqpoint{-0.013889in}{0.000000in}}%
\pgfusepath{stroke,fill}%
}%
\begin{pgfscope}%
\pgfsys@transformshift{0.781250in}{1.604321in}%
\pgfsys@useobject{currentmarker}{}%
\end{pgfscope}%
\end{pgfscope}%
\begin{pgfscope}%
\definecolor{textcolor}{rgb}{0.000000,0.000000,0.000000}%
\pgfsetstrokecolor{textcolor}%
\pgfsetfillcolor{textcolor}%
\pgftext[x=0.584473in, y=1.554394in, left, base]{\color{textcolor}{\ifdefined\pdftexversion\else\setmainfont{NanumMyeongjo}\rmfamily\fi\fontsize{9.000000}{10.800000}\selectfont\catcode`\^=\active\def^{\ifmmode\sp\else\^{}\fi}\catcode`\%=\active\def%{\%}20}}%
\end{pgfscope}%
\begin{pgfscope}%
\pgfpathrectangle{\pgfqpoint{0.781250in}{0.638889in}}{\pgfqpoint{4.218750in}{2.172222in}}%
\pgfusepath{clip}%
\pgfsetbuttcap%
\pgfsetroundjoin%
\pgfsetlinewidth{0.602250pt}%
\definecolor{currentstroke}{rgb}{0.690196,0.690196,0.690196}%
\pgfsetstrokecolor{currentstroke}%
\pgfsetstrokeopacity{0.400000}%
\pgfsetdash{{2.220000pt}{0.960000pt}}{0.000000pt}%
\pgfpathmoveto{\pgfqpoint{0.781250in}{1.845679in}}%
\pgfpathlineto{\pgfqpoint{5.000000in}{1.845679in}}%
\pgfusepath{stroke}%
\end{pgfscope}%
\begin{pgfscope}%
\pgfsetbuttcap%
\pgfsetroundjoin%
\definecolor{currentfill}{rgb}{0.000000,0.000000,0.000000}%
\pgfsetfillcolor{currentfill}%
\pgfsetlinewidth{0.752812pt}%
\definecolor{currentstroke}{rgb}{0.000000,0.000000,0.000000}%
\pgfsetstrokecolor{currentstroke}%
\pgfsetdash{}{0pt}%
\pgfsys@defobject{currentmarker}{\pgfqpoint{-0.013889in}{0.000000in}}{\pgfqpoint{-0.000000in}{0.000000in}}{%
\pgfpathmoveto{\pgfqpoint{-0.000000in}{0.000000in}}%
\pgfpathlineto{\pgfqpoint{-0.013889in}{0.000000in}}%
\pgfusepath{stroke,fill}%
}%
\begin{pgfscope}%
\pgfsys@transformshift{0.781250in}{1.845679in}%
\pgfsys@useobject{currentmarker}{}%
\end{pgfscope}%
\end{pgfscope}%
\begin{pgfscope}%
\definecolor{textcolor}{rgb}{0.000000,0.000000,0.000000}%
\pgfsetstrokecolor{textcolor}%
\pgfsetfillcolor{textcolor}%
\pgftext[x=0.584473in, y=1.795752in, left, base]{\color{textcolor}{\ifdefined\pdftexversion\else\setmainfont{NanumMyeongjo}\rmfamily\fi\fontsize{9.000000}{10.800000}\selectfont\catcode`\^=\active\def^{\ifmmode\sp\else\^{}\fi}\catcode`\%=\active\def%{\%}25}}%
\end{pgfscope}%
\begin{pgfscope}%
\pgfpathrectangle{\pgfqpoint{0.781250in}{0.638889in}}{\pgfqpoint{4.218750in}{2.172222in}}%
\pgfusepath{clip}%
\pgfsetbuttcap%
\pgfsetroundjoin%
\pgfsetlinewidth{0.602250pt}%
\definecolor{currentstroke}{rgb}{0.690196,0.690196,0.690196}%
\pgfsetstrokecolor{currentstroke}%
\pgfsetstrokeopacity{0.400000}%
\pgfsetdash{{2.220000pt}{0.960000pt}}{0.000000pt}%
\pgfpathmoveto{\pgfqpoint{0.781250in}{2.087037in}}%
\pgfpathlineto{\pgfqpoint{5.000000in}{2.087037in}}%
\pgfusepath{stroke}%
\end{pgfscope}%
\begin{pgfscope}%
\pgfsetbuttcap%
\pgfsetroundjoin%
\definecolor{currentfill}{rgb}{0.000000,0.000000,0.000000}%
\pgfsetfillcolor{currentfill}%
\pgfsetlinewidth{0.752812pt}%
\definecolor{currentstroke}{rgb}{0.000000,0.000000,0.000000}%
\pgfsetstrokecolor{currentstroke}%
\pgfsetdash{}{0pt}%
\pgfsys@defobject{currentmarker}{\pgfqpoint{-0.013889in}{0.000000in}}{\pgfqpoint{-0.000000in}{0.000000in}}{%
\pgfpathmoveto{\pgfqpoint{-0.000000in}{0.000000in}}%
\pgfpathlineto{\pgfqpoint{-0.013889in}{0.000000in}}%
\pgfusepath{stroke,fill}%
}%
\begin{pgfscope}%
\pgfsys@transformshift{0.781250in}{2.087037in}%
\pgfsys@useobject{currentmarker}{}%
\end{pgfscope}%
\end{pgfscope}%
\begin{pgfscope}%
\definecolor{textcolor}{rgb}{0.000000,0.000000,0.000000}%
\pgfsetstrokecolor{textcolor}%
\pgfsetfillcolor{textcolor}%
\pgftext[x=0.584473in, y=2.037110in, left, base]{\color{textcolor}{\ifdefined\pdftexversion\else\setmainfont{NanumMyeongjo}\rmfamily\fi\fontsize{9.000000}{10.800000}\selectfont\catcode`\^=\active\def^{\ifmmode\sp\else\^{}\fi}\catcode`\%=\active\def%{\%}30}}%
\end{pgfscope}%
\begin{pgfscope}%
\pgfpathrectangle{\pgfqpoint{0.781250in}{0.638889in}}{\pgfqpoint{4.218750in}{2.172222in}}%
\pgfusepath{clip}%
\pgfsetbuttcap%
\pgfsetroundjoin%
\pgfsetlinewidth{0.602250pt}%
\definecolor{currentstroke}{rgb}{0.690196,0.690196,0.690196}%
\pgfsetstrokecolor{currentstroke}%
\pgfsetstrokeopacity{0.400000}%
\pgfsetdash{{2.220000pt}{0.960000pt}}{0.000000pt}%
\pgfpathmoveto{\pgfqpoint{0.781250in}{2.328395in}}%
\pgfpathlineto{\pgfqpoint{5.000000in}{2.328395in}}%
\pgfusepath{stroke}%
\end{pgfscope}%
\begin{pgfscope}%
\pgfsetbuttcap%
\pgfsetroundjoin%
\definecolor{currentfill}{rgb}{0.000000,0.000000,0.000000}%
\pgfsetfillcolor{currentfill}%
\pgfsetlinewidth{0.752812pt}%
\definecolor{currentstroke}{rgb}{0.000000,0.000000,0.000000}%
\pgfsetstrokecolor{currentstroke}%
\pgfsetdash{}{0pt}%
\pgfsys@defobject{currentmarker}{\pgfqpoint{-0.013889in}{0.000000in}}{\pgfqpoint{-0.000000in}{0.000000in}}{%
\pgfpathmoveto{\pgfqpoint{-0.000000in}{0.000000in}}%
\pgfpathlineto{\pgfqpoint{-0.013889in}{0.000000in}}%
\pgfusepath{stroke,fill}%
}%
\begin{pgfscope}%
\pgfsys@transformshift{0.781250in}{2.328395in}%
\pgfsys@useobject{currentmarker}{}%
\end{pgfscope}%
\end{pgfscope}%
\begin{pgfscope}%
\definecolor{textcolor}{rgb}{0.000000,0.000000,0.000000}%
\pgfsetstrokecolor{textcolor}%
\pgfsetfillcolor{textcolor}%
\pgftext[x=0.584473in, y=2.278468in, left, base]{\color{textcolor}{\ifdefined\pdftexversion\else\setmainfont{NanumMyeongjo}\rmfamily\fi\fontsize{9.000000}{10.800000}\selectfont\catcode`\^=\active\def^{\ifmmode\sp\else\^{}\fi}\catcode`\%=\active\def%{\%}35}}%
\end{pgfscope}%
\begin{pgfscope}%
\pgfpathrectangle{\pgfqpoint{0.781250in}{0.638889in}}{\pgfqpoint{4.218750in}{2.172222in}}%
\pgfusepath{clip}%
\pgfsetbuttcap%
\pgfsetroundjoin%
\pgfsetlinewidth{0.602250pt}%
\definecolor{currentstroke}{rgb}{0.690196,0.690196,0.690196}%
\pgfsetstrokecolor{currentstroke}%
\pgfsetstrokeopacity{0.400000}%
\pgfsetdash{{2.220000pt}{0.960000pt}}{0.000000pt}%
\pgfpathmoveto{\pgfqpoint{0.781250in}{2.569753in}}%
\pgfpathlineto{\pgfqpoint{5.000000in}{2.569753in}}%
\pgfusepath{stroke}%
\end{pgfscope}%
\begin{pgfscope}%
\pgfsetbuttcap%
\pgfsetroundjoin%
\definecolor{currentfill}{rgb}{0.000000,0.000000,0.000000}%
\pgfsetfillcolor{currentfill}%
\pgfsetlinewidth{0.752812pt}%
\definecolor{currentstroke}{rgb}{0.000000,0.000000,0.000000}%
\pgfsetstrokecolor{currentstroke}%
\pgfsetdash{}{0pt}%
\pgfsys@defobject{currentmarker}{\pgfqpoint{-0.013889in}{0.000000in}}{\pgfqpoint{-0.000000in}{0.000000in}}{%
\pgfpathmoveto{\pgfqpoint{-0.000000in}{0.000000in}}%
\pgfpathlineto{\pgfqpoint{-0.013889in}{0.000000in}}%
\pgfusepath{stroke,fill}%
}%
\begin{pgfscope}%
\pgfsys@transformshift{0.781250in}{2.569753in}%
\pgfsys@useobject{currentmarker}{}%
\end{pgfscope}%
\end{pgfscope}%
\begin{pgfscope}%
\definecolor{textcolor}{rgb}{0.000000,0.000000,0.000000}%
\pgfsetstrokecolor{textcolor}%
\pgfsetfillcolor{textcolor}%
\pgftext[x=0.584473in, y=2.519826in, left, base]{\color{textcolor}{\ifdefined\pdftexversion\else\setmainfont{NanumMyeongjo}\rmfamily\fi\fontsize{9.000000}{10.800000}\selectfont\catcode`\^=\active\def^{\ifmmode\sp\else\^{}\fi}\catcode`\%=\active\def%{\%}40}}%
\end{pgfscope}%
\begin{pgfscope}%
\pgfpathrectangle{\pgfqpoint{0.781250in}{0.638889in}}{\pgfqpoint{4.218750in}{2.172222in}}%
\pgfusepath{clip}%
\pgfsetbuttcap%
\pgfsetroundjoin%
\pgfsetlinewidth{0.602250pt}%
\definecolor{currentstroke}{rgb}{0.690196,0.690196,0.690196}%
\pgfsetstrokecolor{currentstroke}%
\pgfsetstrokeopacity{0.400000}%
\pgfsetdash{{2.220000pt}{0.960000pt}}{0.000000pt}%
\pgfpathmoveto{\pgfqpoint{0.781250in}{2.811111in}}%
\pgfpathlineto{\pgfqpoint{5.000000in}{2.811111in}}%
\pgfusepath{stroke}%
\end{pgfscope}%
\begin{pgfscope}%
\pgfsetbuttcap%
\pgfsetroundjoin%
\definecolor{currentfill}{rgb}{0.000000,0.000000,0.000000}%
\pgfsetfillcolor{currentfill}%
\pgfsetlinewidth{0.752812pt}%
\definecolor{currentstroke}{rgb}{0.000000,0.000000,0.000000}%
\pgfsetstrokecolor{currentstroke}%
\pgfsetdash{}{0pt}%
\pgfsys@defobject{currentmarker}{\pgfqpoint{-0.013889in}{0.000000in}}{\pgfqpoint{-0.000000in}{0.000000in}}{%
\pgfpathmoveto{\pgfqpoint{-0.000000in}{0.000000in}}%
\pgfpathlineto{\pgfqpoint{-0.013889in}{0.000000in}}%
\pgfusepath{stroke,fill}%
}%
\begin{pgfscope}%
\pgfsys@transformshift{0.781250in}{2.811111in}%
\pgfsys@useobject{currentmarker}{}%
\end{pgfscope}%
\end{pgfscope}%
\begin{pgfscope}%
\definecolor{textcolor}{rgb}{0.000000,0.000000,0.000000}%
\pgfsetstrokecolor{textcolor}%
\pgfsetfillcolor{textcolor}%
\pgftext[x=0.584473in, y=2.761184in, left, base]{\color{textcolor}{\ifdefined\pdftexversion\else\setmainfont{NanumMyeongjo}\rmfamily\fi\fontsize{9.000000}{10.800000}\selectfont\catcode`\^=\active\def^{\ifmmode\sp\else\^{}\fi}\catcode`\%=\active\def%{\%}45}}%
\end{pgfscope}%
\begin{pgfscope}%
\pgfpathrectangle{\pgfqpoint{0.781250in}{0.638889in}}{\pgfqpoint{4.218750in}{2.172222in}}%
\pgfusepath{clip}%
\pgfsetrectcap%
\pgfsetroundjoin%
\pgfsetlinewidth{1.204500pt}%
\definecolor{currentstroke}{rgb}{0.235294,0.490196,0.764706}%
\pgfsetstrokecolor{currentstroke}%
\pgfsetdash{}{0pt}%
\pgfpathmoveto{\pgfqpoint{0.973011in}{2.202889in}}%
\pgfpathlineto{\pgfqpoint{1.246956in}{1.898778in}}%
\pgfpathlineto{\pgfqpoint{1.520901in}{2.125654in}}%
\pgfpathlineto{\pgfqpoint{1.794846in}{2.067728in}}%
\pgfpathlineto{\pgfqpoint{2.068791in}{2.371840in}}%
\pgfpathlineto{\pgfqpoint{2.342735in}{2.188407in}}%
\pgfpathlineto{\pgfqpoint{2.616680in}{1.826370in}}%
\pgfpathlineto{\pgfqpoint{2.890625in}{1.700864in}}%
\pgfpathlineto{\pgfqpoint{3.164570in}{1.860160in}}%
\pgfpathlineto{\pgfqpoint{3.438515in}{1.927741in}}%
\pgfpathlineto{\pgfqpoint{3.712459in}{2.106346in}}%
\pgfpathlineto{\pgfqpoint{3.986404in}{1.782926in}}%
\pgfpathlineto{\pgfqpoint{4.260349in}{2.019457in}}%
\pgfpathlineto{\pgfqpoint{4.534294in}{2.362185in}}%
\pgfpathlineto{\pgfqpoint{4.808239in}{2.502173in}}%
\pgfusepath{stroke}%
\end{pgfscope}%
\begin{pgfscope}%
\pgfpathrectangle{\pgfqpoint{0.781250in}{0.638889in}}{\pgfqpoint{4.218750in}{2.172222in}}%
\pgfusepath{clip}%
\pgfsetbuttcap%
\pgfsetroundjoin%
\definecolor{currentfill}{rgb}{0.235294,0.490196,0.764706}%
\pgfsetfillcolor{currentfill}%
\pgfsetlinewidth{1.003750pt}%
\definecolor{currentstroke}{rgb}{0.235294,0.490196,0.764706}%
\pgfsetstrokecolor{currentstroke}%
\pgfsetdash{}{0pt}%
\pgfsys@defobject{currentmarker}{\pgfqpoint{-0.020833in}{-0.020833in}}{\pgfqpoint{0.020833in}{0.020833in}}{%
\pgfpathmoveto{\pgfqpoint{0.000000in}{-0.020833in}}%
\pgfpathcurveto{\pgfqpoint{0.005525in}{-0.020833in}}{\pgfqpoint{0.010825in}{-0.018638in}}{\pgfqpoint{0.014731in}{-0.014731in}}%
\pgfpathcurveto{\pgfqpoint{0.018638in}{-0.010825in}}{\pgfqpoint{0.020833in}{-0.005525in}}{\pgfqpoint{0.020833in}{0.000000in}}%
\pgfpathcurveto{\pgfqpoint{0.020833in}{0.005525in}}{\pgfqpoint{0.018638in}{0.010825in}}{\pgfqpoint{0.014731in}{0.014731in}}%
\pgfpathcurveto{\pgfqpoint{0.010825in}{0.018638in}}{\pgfqpoint{0.005525in}{0.020833in}}{\pgfqpoint{0.000000in}{0.020833in}}%
\pgfpathcurveto{\pgfqpoint{-0.005525in}{0.020833in}}{\pgfqpoint{-0.010825in}{0.018638in}}{\pgfqpoint{-0.014731in}{0.014731in}}%
\pgfpathcurveto{\pgfqpoint{-0.018638in}{0.010825in}}{\pgfqpoint{-0.020833in}{0.005525in}}{\pgfqpoint{-0.020833in}{0.000000in}}%
\pgfpathcurveto{\pgfqpoint{-0.020833in}{-0.005525in}}{\pgfqpoint{-0.018638in}{-0.010825in}}{\pgfqpoint{-0.014731in}{-0.014731in}}%
\pgfpathcurveto{\pgfqpoint{-0.010825in}{-0.018638in}}{\pgfqpoint{-0.005525in}{-0.020833in}}{\pgfqpoint{0.000000in}{-0.020833in}}%
\pgfpathlineto{\pgfqpoint{0.000000in}{-0.020833in}}%
\pgfpathclose%
\pgfusepath{stroke,fill}%
}%
\begin{pgfscope}%
\pgfsys@transformshift{0.973011in}{2.202889in}%
\pgfsys@useobject{currentmarker}{}%
\end{pgfscope}%
\begin{pgfscope}%
\pgfsys@transformshift{1.246956in}{1.898778in}%
\pgfsys@useobject{currentmarker}{}%
\end{pgfscope}%
\begin{pgfscope}%
\pgfsys@transformshift{1.520901in}{2.125654in}%
\pgfsys@useobject{currentmarker}{}%
\end{pgfscope}%
\begin{pgfscope}%
\pgfsys@transformshift{1.794846in}{2.067728in}%
\pgfsys@useobject{currentmarker}{}%
\end{pgfscope}%
\begin{pgfscope}%
\pgfsys@transformshift{2.068791in}{2.371840in}%
\pgfsys@useobject{currentmarker}{}%
\end{pgfscope}%
\begin{pgfscope}%
\pgfsys@transformshift{2.342735in}{2.188407in}%
\pgfsys@useobject{currentmarker}{}%
\end{pgfscope}%
\begin{pgfscope}%
\pgfsys@transformshift{2.616680in}{1.826370in}%
\pgfsys@useobject{currentmarker}{}%
\end{pgfscope}%
\begin{pgfscope}%
\pgfsys@transformshift{2.890625in}{1.700864in}%
\pgfsys@useobject{currentmarker}{}%
\end{pgfscope}%
\begin{pgfscope}%
\pgfsys@transformshift{3.164570in}{1.860160in}%
\pgfsys@useobject{currentmarker}{}%
\end{pgfscope}%
\begin{pgfscope}%
\pgfsys@transformshift{3.438515in}{1.927741in}%
\pgfsys@useobject{currentmarker}{}%
\end{pgfscope}%
\begin{pgfscope}%
\pgfsys@transformshift{3.712459in}{2.106346in}%
\pgfsys@useobject{currentmarker}{}%
\end{pgfscope}%
\begin{pgfscope}%
\pgfsys@transformshift{3.986404in}{1.782926in}%
\pgfsys@useobject{currentmarker}{}%
\end{pgfscope}%
\begin{pgfscope}%
\pgfsys@transformshift{4.260349in}{2.019457in}%
\pgfsys@useobject{currentmarker}{}%
\end{pgfscope}%
\begin{pgfscope}%
\pgfsys@transformshift{4.534294in}{2.362185in}%
\pgfsys@useobject{currentmarker}{}%
\end{pgfscope}%
\begin{pgfscope}%
\pgfsys@transformshift{4.808239in}{2.502173in}%
\pgfsys@useobject{currentmarker}{}%
\end{pgfscope}%
\end{pgfscope}%
\begin{pgfscope}%
\pgfpathrectangle{\pgfqpoint{0.781250in}{0.638889in}}{\pgfqpoint{4.218750in}{2.172222in}}%
\pgfusepath{clip}%
\pgfsetrectcap%
\pgfsetroundjoin%
\pgfsetlinewidth{1.204500pt}%
\definecolor{currentstroke}{rgb}{0.337255,0.713725,0.627451}%
\pgfsetstrokecolor{currentstroke}%
\pgfsetdash{}{0pt}%
\pgfpathmoveto{\pgfqpoint{0.973011in}{1.126432in}}%
\pgfpathlineto{\pgfqpoint{1.246956in}{1.025062in}}%
\pgfpathlineto{\pgfqpoint{1.520901in}{1.140914in}}%
\pgfpathlineto{\pgfqpoint{1.794846in}{1.107123in}}%
\pgfpathlineto{\pgfqpoint{2.068791in}{1.184358in}}%
\pgfpathlineto{\pgfqpoint{2.342735in}{1.092642in}}%
\pgfpathlineto{\pgfqpoint{2.616680in}{0.976790in}}%
\pgfpathlineto{\pgfqpoint{2.890625in}{0.899556in}}%
\pgfpathlineto{\pgfqpoint{3.164570in}{0.943000in}}%
\pgfpathlineto{\pgfqpoint{3.438515in}{0.957481in}}%
\pgfpathlineto{\pgfqpoint{3.712459in}{1.000926in}}%
\pgfpathlineto{\pgfqpoint{3.986404in}{0.923691in}}%
\pgfpathlineto{\pgfqpoint{4.260349in}{1.010580in}}%
\pgfpathlineto{\pgfqpoint{4.534294in}{1.087815in}}%
\pgfusepath{stroke}%
\end{pgfscope}%
\begin{pgfscope}%
\pgfpathrectangle{\pgfqpoint{0.781250in}{0.638889in}}{\pgfqpoint{4.218750in}{2.172222in}}%
\pgfusepath{clip}%
\pgfsetbuttcap%
\pgfsetroundjoin%
\definecolor{currentfill}{rgb}{0.337255,0.713725,0.627451}%
\pgfsetfillcolor{currentfill}%
\pgfsetlinewidth{1.003750pt}%
\definecolor{currentstroke}{rgb}{0.337255,0.713725,0.627451}%
\pgfsetstrokecolor{currentstroke}%
\pgfsetdash{}{0pt}%
\pgfsys@defobject{currentmarker}{\pgfqpoint{-0.020833in}{-0.020833in}}{\pgfqpoint{0.020833in}{0.020833in}}{%
\pgfpathmoveto{\pgfqpoint{0.000000in}{-0.020833in}}%
\pgfpathcurveto{\pgfqpoint{0.005525in}{-0.020833in}}{\pgfqpoint{0.010825in}{-0.018638in}}{\pgfqpoint{0.014731in}{-0.014731in}}%
\pgfpathcurveto{\pgfqpoint{0.018638in}{-0.010825in}}{\pgfqpoint{0.020833in}{-0.005525in}}{\pgfqpoint{0.020833in}{0.000000in}}%
\pgfpathcurveto{\pgfqpoint{0.020833in}{0.005525in}}{\pgfqpoint{0.018638in}{0.010825in}}{\pgfqpoint{0.014731in}{0.014731in}}%
\pgfpathcurveto{\pgfqpoint{0.010825in}{0.018638in}}{\pgfqpoint{0.005525in}{0.020833in}}{\pgfqpoint{0.000000in}{0.020833in}}%
\pgfpathcurveto{\pgfqpoint{-0.005525in}{0.020833in}}{\pgfqpoint{-0.010825in}{0.018638in}}{\pgfqpoint{-0.014731in}{0.014731in}}%
\pgfpathcurveto{\pgfqpoint{-0.018638in}{0.010825in}}{\pgfqpoint{-0.020833in}{0.005525in}}{\pgfqpoint{-0.020833in}{0.000000in}}%
\pgfpathcurveto{\pgfqpoint{-0.020833in}{-0.005525in}}{\pgfqpoint{-0.018638in}{-0.010825in}}{\pgfqpoint{-0.014731in}{-0.014731in}}%
\pgfpathcurveto{\pgfqpoint{-0.010825in}{-0.018638in}}{\pgfqpoint{-0.005525in}{-0.020833in}}{\pgfqpoint{0.000000in}{-0.020833in}}%
\pgfpathlineto{\pgfqpoint{0.000000in}{-0.020833in}}%
\pgfpathclose%
\pgfusepath{stroke,fill}%
}%
\begin{pgfscope}%
\pgfsys@transformshift{0.973011in}{1.126432in}%
\pgfsys@useobject{currentmarker}{}%
\end{pgfscope}%
\begin{pgfscope}%
\pgfsys@transformshift{1.246956in}{1.025062in}%
\pgfsys@useobject{currentmarker}{}%
\end{pgfscope}%
\begin{pgfscope}%
\pgfsys@transformshift{1.520901in}{1.140914in}%
\pgfsys@useobject{currentmarker}{}%
\end{pgfscope}%
\begin{pgfscope}%
\pgfsys@transformshift{1.794846in}{1.107123in}%
\pgfsys@useobject{currentmarker}{}%
\end{pgfscope}%
\begin{pgfscope}%
\pgfsys@transformshift{2.068791in}{1.184358in}%
\pgfsys@useobject{currentmarker}{}%
\end{pgfscope}%
\begin{pgfscope}%
\pgfsys@transformshift{2.342735in}{1.092642in}%
\pgfsys@useobject{currentmarker}{}%
\end{pgfscope}%
\begin{pgfscope}%
\pgfsys@transformshift{2.616680in}{0.976790in}%
\pgfsys@useobject{currentmarker}{}%
\end{pgfscope}%
\begin{pgfscope}%
\pgfsys@transformshift{2.890625in}{0.899556in}%
\pgfsys@useobject{currentmarker}{}%
\end{pgfscope}%
\begin{pgfscope}%
\pgfsys@transformshift{3.164570in}{0.943000in}%
\pgfsys@useobject{currentmarker}{}%
\end{pgfscope}%
\begin{pgfscope}%
\pgfsys@transformshift{3.438515in}{0.957481in}%
\pgfsys@useobject{currentmarker}{}%
\end{pgfscope}%
\begin{pgfscope}%
\pgfsys@transformshift{3.712459in}{1.000926in}%
\pgfsys@useobject{currentmarker}{}%
\end{pgfscope}%
\begin{pgfscope}%
\pgfsys@transformshift{3.986404in}{0.923691in}%
\pgfsys@useobject{currentmarker}{}%
\end{pgfscope}%
\begin{pgfscope}%
\pgfsys@transformshift{4.260349in}{1.010580in}%
\pgfsys@useobject{currentmarker}{}%
\end{pgfscope}%
\begin{pgfscope}%
\pgfsys@transformshift{4.534294in}{1.087815in}%
\pgfsys@useobject{currentmarker}{}%
\end{pgfscope}%
\end{pgfscope}%
\begin{pgfscope}%
\pgfsetrectcap%
\pgfsetmiterjoin%
\pgfsetlinewidth{0.752812pt}%
\definecolor{currentstroke}{rgb}{0.000000,0.000000,0.000000}%
\pgfsetstrokecolor{currentstroke}%
\pgfsetdash{}{0pt}%
\pgfpathmoveto{\pgfqpoint{0.781250in}{0.638889in}}%
\pgfpathlineto{\pgfqpoint{0.781250in}{2.811111in}}%
\pgfusepath{stroke}%
\end{pgfscope}%
\begin{pgfscope}%
\pgfsetrectcap%
\pgfsetmiterjoin%
\pgfsetlinewidth{0.752812pt}%
\definecolor{currentstroke}{rgb}{0.000000,0.000000,0.000000}%
\pgfsetstrokecolor{currentstroke}%
\pgfsetdash{}{0pt}%
\pgfpathmoveto{\pgfqpoint{0.781250in}{0.638889in}}%
\pgfpathlineto{\pgfqpoint{5.000000in}{0.638889in}}%
\pgfusepath{stroke}%
\end{pgfscope}%
\begin{pgfscope}%
\definecolor{textcolor}{rgb}{0.133333,0.133333,0.133333}%
\pgfsetstrokecolor{textcolor}%
\pgfsetfillcolor{textcolor}%
\pgftext[x=0.973011in,y=2.258444in,,bottom]{\color{textcolor}{\ifdefined\pdftexversion\else\setmainfont{NanumMyeongjo}\rmfamily\fi\fontsize{7.000000}{8.400000}\selectfont\catcode`\^=\active\def^{\ifmmode\sp\else\^{}\fi}\catcode`\%=\active\def%{\%}32.4}}%
\end{pgfscope}%
\begin{pgfscope}%
\definecolor{textcolor}{rgb}{0.133333,0.133333,0.133333}%
\pgfsetstrokecolor{textcolor}%
\pgfsetfillcolor{textcolor}%
\pgftext[x=1.246956in,y=1.954333in,,bottom]{\color{textcolor}{\ifdefined\pdftexversion\else\setmainfont{NanumMyeongjo}\rmfamily\fi\fontsize{7.000000}{8.400000}\selectfont\catcode`\^=\active\def^{\ifmmode\sp\else\^{}\fi}\catcode`\%=\active\def%{\%}26.1}}%
\end{pgfscope}%
\begin{pgfscope}%
\definecolor{textcolor}{rgb}{0.133333,0.133333,0.133333}%
\pgfsetstrokecolor{textcolor}%
\pgfsetfillcolor{textcolor}%
\pgftext[x=1.520901in,y=2.181210in,,bottom]{\color{textcolor}{\ifdefined\pdftexversion\else\setmainfont{NanumMyeongjo}\rmfamily\fi\fontsize{7.000000}{8.400000}\selectfont\catcode`\^=\active\def^{\ifmmode\sp\else\^{}\fi}\catcode`\%=\active\def%{\%}30.8}}%
\end{pgfscope}%
\begin{pgfscope}%
\definecolor{textcolor}{rgb}{0.133333,0.133333,0.133333}%
\pgfsetstrokecolor{textcolor}%
\pgfsetfillcolor{textcolor}%
\pgftext[x=1.794846in,y=2.123284in,,bottom]{\color{textcolor}{\ifdefined\pdftexversion\else\setmainfont{NanumMyeongjo}\rmfamily\fi\fontsize{7.000000}{8.400000}\selectfont\catcode`\^=\active\def^{\ifmmode\sp\else\^{}\fi}\catcode`\%=\active\def%{\%}29.6}}%
\end{pgfscope}%
\begin{pgfscope}%
\definecolor{textcolor}{rgb}{0.133333,0.133333,0.133333}%
\pgfsetstrokecolor{textcolor}%
\pgfsetfillcolor{textcolor}%
\pgftext[x=2.068791in,y=2.427395in,,bottom]{\color{textcolor}{\ifdefined\pdftexversion\else\setmainfont{NanumMyeongjo}\rmfamily\fi\fontsize{7.000000}{8.400000}\selectfont\catcode`\^=\active\def^{\ifmmode\sp\else\^{}\fi}\catcode`\%=\active\def%{\%}35.9}}%
\end{pgfscope}%
\begin{pgfscope}%
\definecolor{textcolor}{rgb}{0.133333,0.133333,0.133333}%
\pgfsetstrokecolor{textcolor}%
\pgfsetfillcolor{textcolor}%
\pgftext[x=2.342735in,y=2.243963in,,bottom]{\color{textcolor}{\ifdefined\pdftexversion\else\setmainfont{NanumMyeongjo}\rmfamily\fi\fontsize{7.000000}{8.400000}\selectfont\catcode`\^=\active\def^{\ifmmode\sp\else\^{}\fi}\catcode`\%=\active\def%{\%}32.1}}%
\end{pgfscope}%
\begin{pgfscope}%
\definecolor{textcolor}{rgb}{0.133333,0.133333,0.133333}%
\pgfsetstrokecolor{textcolor}%
\pgfsetfillcolor{textcolor}%
\pgftext[x=2.616680in,y=1.881926in,,bottom]{\color{textcolor}{\ifdefined\pdftexversion\else\setmainfont{NanumMyeongjo}\rmfamily\fi\fontsize{7.000000}{8.400000}\selectfont\catcode`\^=\active\def^{\ifmmode\sp\else\^{}\fi}\catcode`\%=\active\def%{\%}24.6}}%
\end{pgfscope}%
\begin{pgfscope}%
\definecolor{textcolor}{rgb}{0.133333,0.133333,0.133333}%
\pgfsetstrokecolor{textcolor}%
\pgfsetfillcolor{textcolor}%
\pgftext[x=2.890625in,y=1.756420in,,bottom]{\color{textcolor}{\ifdefined\pdftexversion\else\setmainfont{NanumMyeongjo}\rmfamily\fi\fontsize{7.000000}{8.400000}\selectfont\catcode`\^=\active\def^{\ifmmode\sp\else\^{}\fi}\catcode`\%=\active\def%{\%}22.0}}%
\end{pgfscope}%
\begin{pgfscope}%
\definecolor{textcolor}{rgb}{0.133333,0.133333,0.133333}%
\pgfsetstrokecolor{textcolor}%
\pgfsetfillcolor{textcolor}%
\pgftext[x=3.164570in,y=1.915716in,,bottom]{\color{textcolor}{\ifdefined\pdftexversion\else\setmainfont{NanumMyeongjo}\rmfamily\fi\fontsize{7.000000}{8.400000}\selectfont\catcode`\^=\active\def^{\ifmmode\sp\else\^{}\fi}\catcode`\%=\active\def%{\%}25.3}}%
\end{pgfscope}%
\begin{pgfscope}%
\definecolor{textcolor}{rgb}{0.133333,0.133333,0.133333}%
\pgfsetstrokecolor{textcolor}%
\pgfsetfillcolor{textcolor}%
\pgftext[x=3.438515in,y=1.983296in,,bottom]{\color{textcolor}{\ifdefined\pdftexversion\else\setmainfont{NanumMyeongjo}\rmfamily\fi\fontsize{7.000000}{8.400000}\selectfont\catcode`\^=\active\def^{\ifmmode\sp\else\^{}\fi}\catcode`\%=\active\def%{\%}26.7}}%
\end{pgfscope}%
\begin{pgfscope}%
\definecolor{textcolor}{rgb}{0.133333,0.133333,0.133333}%
\pgfsetstrokecolor{textcolor}%
\pgfsetfillcolor{textcolor}%
\pgftext[x=3.712459in,y=2.161901in,,bottom]{\color{textcolor}{\ifdefined\pdftexversion\else\setmainfont{NanumMyeongjo}\rmfamily\fi\fontsize{7.000000}{8.400000}\selectfont\catcode`\^=\active\def^{\ifmmode\sp\else\^{}\fi}\catcode`\%=\active\def%{\%}30.4}}%
\end{pgfscope}%
\begin{pgfscope}%
\definecolor{textcolor}{rgb}{0.133333,0.133333,0.133333}%
\pgfsetstrokecolor{textcolor}%
\pgfsetfillcolor{textcolor}%
\pgftext[x=3.986404in,y=1.838481in,,bottom]{\color{textcolor}{\ifdefined\pdftexversion\else\setmainfont{NanumMyeongjo}\rmfamily\fi\fontsize{7.000000}{8.400000}\selectfont\catcode`\^=\active\def^{\ifmmode\sp\else\^{}\fi}\catcode`\%=\active\def%{\%}23.7}}%
\end{pgfscope}%
\begin{pgfscope}%
\definecolor{textcolor}{rgb}{0.133333,0.133333,0.133333}%
\pgfsetstrokecolor{textcolor}%
\pgfsetfillcolor{textcolor}%
\pgftext[x=4.260349in,y=2.075012in,,bottom]{\color{textcolor}{\ifdefined\pdftexversion\else\setmainfont{NanumMyeongjo}\rmfamily\fi\fontsize{7.000000}{8.400000}\selectfont\catcode`\^=\active\def^{\ifmmode\sp\else\^{}\fi}\catcode`\%=\active\def%{\%}28.6}}%
\end{pgfscope}%
\begin{pgfscope}%
\definecolor{textcolor}{rgb}{0.133333,0.133333,0.133333}%
\pgfsetstrokecolor{textcolor}%
\pgfsetfillcolor{textcolor}%
\pgftext[x=4.534294in,y=2.417741in,,bottom]{\color{textcolor}{\ifdefined\pdftexversion\else\setmainfont{NanumMyeongjo}\rmfamily\fi\fontsize{7.000000}{8.400000}\selectfont\catcode`\^=\active\def^{\ifmmode\sp\else\^{}\fi}\catcode`\%=\active\def%{\%}35.7}}%
\end{pgfscope}%
\begin{pgfscope}%
\definecolor{textcolor}{rgb}{0.133333,0.133333,0.133333}%
\pgfsetstrokecolor{textcolor}%
\pgfsetfillcolor{textcolor}%
\pgftext[x=4.808239in,y=2.557728in,,bottom]{\color{textcolor}{\ifdefined\pdftexversion\else\setmainfont{NanumMyeongjo}\rmfamily\fi\fontsize{7.000000}{8.400000}\selectfont\catcode`\^=\active\def^{\ifmmode\sp\else\^{}\fi}\catcode`\%=\active\def%{\%}38.6}}%
\end{pgfscope}%
\begin{pgfscope}%
\definecolor{textcolor}{rgb}{0.133333,0.133333,0.133333}%
\pgfsetstrokecolor{textcolor}%
\pgfsetfillcolor{textcolor}%
\pgftext[x=0.973011in,y=1.181988in,,bottom]{\color{textcolor}{\ifdefined\pdftexversion\else\setmainfont{NanumMyeongjo}\rmfamily\fi\fontsize{7.000000}{8.400000}\selectfont\catcode`\^=\active\def^{\ifmmode\sp\else\^{}\fi}\catcode`\%=\active\def%{\%}10.1}}%
\end{pgfscope}%
\begin{pgfscope}%
\definecolor{textcolor}{rgb}{0.133333,0.133333,0.133333}%
\pgfsetstrokecolor{textcolor}%
\pgfsetfillcolor{textcolor}%
\pgftext[x=1.246956in,y=1.080617in,,bottom]{\color{textcolor}{\ifdefined\pdftexversion\else\setmainfont{NanumMyeongjo}\rmfamily\fi\fontsize{7.000000}{8.400000}\selectfont\catcode`\^=\active\def^{\ifmmode\sp\else\^{}\fi}\catcode`\%=\active\def%{\%}8.0}}%
\end{pgfscope}%
\begin{pgfscope}%
\definecolor{textcolor}{rgb}{0.133333,0.133333,0.133333}%
\pgfsetstrokecolor{textcolor}%
\pgfsetfillcolor{textcolor}%
\pgftext[x=1.520901in,y=1.196469in,,bottom]{\color{textcolor}{\ifdefined\pdftexversion\else\setmainfont{NanumMyeongjo}\rmfamily\fi\fontsize{7.000000}{8.400000}\selectfont\catcode`\^=\active\def^{\ifmmode\sp\else\^{}\fi}\catcode`\%=\active\def%{\%}10.4}}%
\end{pgfscope}%
\begin{pgfscope}%
\definecolor{textcolor}{rgb}{0.133333,0.133333,0.133333}%
\pgfsetstrokecolor{textcolor}%
\pgfsetfillcolor{textcolor}%
\pgftext[x=1.794846in,y=1.162679in,,bottom]{\color{textcolor}{\ifdefined\pdftexversion\else\setmainfont{NanumMyeongjo}\rmfamily\fi\fontsize{7.000000}{8.400000}\selectfont\catcode`\^=\active\def^{\ifmmode\sp\else\^{}\fi}\catcode`\%=\active\def%{\%}9.7}}%
\end{pgfscope}%
\begin{pgfscope}%
\definecolor{textcolor}{rgb}{0.133333,0.133333,0.133333}%
\pgfsetstrokecolor{textcolor}%
\pgfsetfillcolor{textcolor}%
\pgftext[x=2.068791in,y=1.239914in,,bottom]{\color{textcolor}{\ifdefined\pdftexversion\else\setmainfont{NanumMyeongjo}\rmfamily\fi\fontsize{7.000000}{8.400000}\selectfont\catcode`\^=\active\def^{\ifmmode\sp\else\^{}\fi}\catcode`\%=\active\def%{\%}11.3}}%
\end{pgfscope}%
\begin{pgfscope}%
\definecolor{textcolor}{rgb}{0.133333,0.133333,0.133333}%
\pgfsetstrokecolor{textcolor}%
\pgfsetfillcolor{textcolor}%
\pgftext[x=2.342735in,y=1.148198in,,bottom]{\color{textcolor}{\ifdefined\pdftexversion\else\setmainfont{NanumMyeongjo}\rmfamily\fi\fontsize{7.000000}{8.400000}\selectfont\catcode`\^=\active\def^{\ifmmode\sp\else\^{}\fi}\catcode`\%=\active\def%{\%}9.4}}%
\end{pgfscope}%
\begin{pgfscope}%
\definecolor{textcolor}{rgb}{0.133333,0.133333,0.133333}%
\pgfsetstrokecolor{textcolor}%
\pgfsetfillcolor{textcolor}%
\pgftext[x=2.616680in,y=1.032346in,,bottom]{\color{textcolor}{\ifdefined\pdftexversion\else\setmainfont{NanumMyeongjo}\rmfamily\fi\fontsize{7.000000}{8.400000}\selectfont\catcode`\^=\active\def^{\ifmmode\sp\else\^{}\fi}\catcode`\%=\active\def%{\%}7.0}}%
\end{pgfscope}%
\begin{pgfscope}%
\definecolor{textcolor}{rgb}{0.133333,0.133333,0.133333}%
\pgfsetstrokecolor{textcolor}%
\pgfsetfillcolor{textcolor}%
\pgftext[x=2.890625in,y=0.955111in,,bottom]{\color{textcolor}{\ifdefined\pdftexversion\else\setmainfont{NanumMyeongjo}\rmfamily\fi\fontsize{7.000000}{8.400000}\selectfont\catcode`\^=\active\def^{\ifmmode\sp\else\^{}\fi}\catcode`\%=\active\def%{\%}5.4}}%
\end{pgfscope}%
\begin{pgfscope}%
\definecolor{textcolor}{rgb}{0.133333,0.133333,0.133333}%
\pgfsetstrokecolor{textcolor}%
\pgfsetfillcolor{textcolor}%
\pgftext[x=3.164570in,y=0.998556in,,bottom]{\color{textcolor}{\ifdefined\pdftexversion\else\setmainfont{NanumMyeongjo}\rmfamily\fi\fontsize{7.000000}{8.400000}\selectfont\catcode`\^=\active\def^{\ifmmode\sp\else\^{}\fi}\catcode`\%=\active\def%{\%}6.3}}%
\end{pgfscope}%
\begin{pgfscope}%
\definecolor{textcolor}{rgb}{0.133333,0.133333,0.133333}%
\pgfsetstrokecolor{textcolor}%
\pgfsetfillcolor{textcolor}%
\pgftext[x=3.438515in,y=1.013037in,,bottom]{\color{textcolor}{\ifdefined\pdftexversion\else\setmainfont{NanumMyeongjo}\rmfamily\fi\fontsize{7.000000}{8.400000}\selectfont\catcode`\^=\active\def^{\ifmmode\sp\else\^{}\fi}\catcode`\%=\active\def%{\%}6.6}}%
\end{pgfscope}%
\begin{pgfscope}%
\definecolor{textcolor}{rgb}{0.133333,0.133333,0.133333}%
\pgfsetstrokecolor{textcolor}%
\pgfsetfillcolor{textcolor}%
\pgftext[x=3.712459in,y=1.056481in,,bottom]{\color{textcolor}{\ifdefined\pdftexversion\else\setmainfont{NanumMyeongjo}\rmfamily\fi\fontsize{7.000000}{8.400000}\selectfont\catcode`\^=\active\def^{\ifmmode\sp\else\^{}\fi}\catcode`\%=\active\def%{\%}7.5}}%
\end{pgfscope}%
\begin{pgfscope}%
\definecolor{textcolor}{rgb}{0.133333,0.133333,0.133333}%
\pgfsetstrokecolor{textcolor}%
\pgfsetfillcolor{textcolor}%
\pgftext[x=3.986404in,y=0.979247in,,bottom]{\color{textcolor}{\ifdefined\pdftexversion\else\setmainfont{NanumMyeongjo}\rmfamily\fi\fontsize{7.000000}{8.400000}\selectfont\catcode`\^=\active\def^{\ifmmode\sp\else\^{}\fi}\catcode`\%=\active\def%{\%}5.9}}%
\end{pgfscope}%
\begin{pgfscope}%
\definecolor{textcolor}{rgb}{0.133333,0.133333,0.133333}%
\pgfsetstrokecolor{textcolor}%
\pgfsetfillcolor{textcolor}%
\pgftext[x=4.260349in,y=1.066136in,,bottom]{\color{textcolor}{\ifdefined\pdftexversion\else\setmainfont{NanumMyeongjo}\rmfamily\fi\fontsize{7.000000}{8.400000}\selectfont\catcode`\^=\active\def^{\ifmmode\sp\else\^{}\fi}\catcode`\%=\active\def%{\%}7.7}}%
\end{pgfscope}%
\begin{pgfscope}%
\definecolor{textcolor}{rgb}{0.133333,0.133333,0.133333}%
\pgfsetstrokecolor{textcolor}%
\pgfsetfillcolor{textcolor}%
\pgftext[x=4.534294in,y=1.143370in,,bottom]{\color{textcolor}{\ifdefined\pdftexversion\else\setmainfont{NanumMyeongjo}\rmfamily\fi\fontsize{7.000000}{8.400000}\selectfont\catcode`\^=\active\def^{\ifmmode\sp\else\^{}\fi}\catcode`\%=\active\def%{\%}9.3}}%
\end{pgfscope}%
\begin{pgfscope}%
\pgfsetrectcap%
\pgfsetroundjoin%
\pgfsetlinewidth{1.204500pt}%
\definecolor{currentstroke}{rgb}{0.235294,0.490196,0.764706}%
\pgfsetstrokecolor{currentstroke}%
\pgfsetdash{}{0pt}%
\pgfpathmoveto{\pgfqpoint{5.112500in}{2.642508in}}%
\pgfpathlineto{\pgfqpoint{5.237500in}{2.642508in}}%
\pgfpathlineto{\pgfqpoint{5.362500in}{2.642508in}}%
\pgfusepath{stroke}%
\end{pgfscope}%
\begin{pgfscope}%
\pgfsetbuttcap%
\pgfsetroundjoin%
\definecolor{currentfill}{rgb}{0.235294,0.490196,0.764706}%
\pgfsetfillcolor{currentfill}%
\pgfsetlinewidth{1.003750pt}%
\definecolor{currentstroke}{rgb}{0.235294,0.490196,0.764706}%
\pgfsetstrokecolor{currentstroke}%
\pgfsetdash{}{0pt}%
\pgfsys@defobject{currentmarker}{\pgfqpoint{-0.020833in}{-0.020833in}}{\pgfqpoint{0.020833in}{0.020833in}}{%
\pgfpathmoveto{\pgfqpoint{0.000000in}{-0.020833in}}%
\pgfpathcurveto{\pgfqpoint{0.005525in}{-0.020833in}}{\pgfqpoint{0.010825in}{-0.018638in}}{\pgfqpoint{0.014731in}{-0.014731in}}%
\pgfpathcurveto{\pgfqpoint{0.018638in}{-0.010825in}}{\pgfqpoint{0.020833in}{-0.005525in}}{\pgfqpoint{0.020833in}{0.000000in}}%
\pgfpathcurveto{\pgfqpoint{0.020833in}{0.005525in}}{\pgfqpoint{0.018638in}{0.010825in}}{\pgfqpoint{0.014731in}{0.014731in}}%
\pgfpathcurveto{\pgfqpoint{0.010825in}{0.018638in}}{\pgfqpoint{0.005525in}{0.020833in}}{\pgfqpoint{0.000000in}{0.020833in}}%
\pgfpathcurveto{\pgfqpoint{-0.005525in}{0.020833in}}{\pgfqpoint{-0.010825in}{0.018638in}}{\pgfqpoint{-0.014731in}{0.014731in}}%
\pgfpathcurveto{\pgfqpoint{-0.018638in}{0.010825in}}{\pgfqpoint{-0.020833in}{0.005525in}}{\pgfqpoint{-0.020833in}{0.000000in}}%
\pgfpathcurveto{\pgfqpoint{-0.020833in}{-0.005525in}}{\pgfqpoint{-0.018638in}{-0.010825in}}{\pgfqpoint{-0.014731in}{-0.014731in}}%
\pgfpathcurveto{\pgfqpoint{-0.010825in}{-0.018638in}}{\pgfqpoint{-0.005525in}{-0.020833in}}{\pgfqpoint{0.000000in}{-0.020833in}}%
\pgfpathlineto{\pgfqpoint{0.000000in}{-0.020833in}}%
\pgfpathclose%
\pgfusepath{stroke,fill}%
}%
\begin{pgfscope}%
\pgfsys@transformshift{5.237500in}{2.642508in}%
\pgfsys@useobject{currentmarker}{}%
\end{pgfscope}%
\end{pgfscope}%
\begin{pgfscope}%
\definecolor{textcolor}{rgb}{0.000000,0.000000,0.000000}%
\pgfsetstrokecolor{textcolor}%
\pgfsetfillcolor{textcolor}%
\pgftext[x=5.462500in,y=2.598758in,left,base]{\color{textcolor}{\ifdefined\pdftexversion\else\setmainfont{NanumMyeongjo}\rmfamily\fi\fontsize{9.000000}{10.800000}\selectfont\catcode`\^=\active\def^{\ifmmode\sp\else\^{}\fi}\catcode`\%=\active\def%{\%}사료제외}}%
\end{pgfscope}%
\begin{pgfscope}%
\pgfsetrectcap%
\pgfsetroundjoin%
\pgfsetlinewidth{1.204500pt}%
\definecolor{currentstroke}{rgb}{0.337255,0.713725,0.627451}%
\pgfsetstrokecolor{currentstroke}%
\pgfsetdash{}{0pt}%
\pgfpathmoveto{\pgfqpoint{5.112500in}{2.451224in}}%
\pgfpathlineto{\pgfqpoint{5.237500in}{2.451224in}}%
\pgfpathlineto{\pgfqpoint{5.362500in}{2.451224in}}%
\pgfusepath{stroke}%
\end{pgfscope}%
\begin{pgfscope}%
\pgfsetbuttcap%
\pgfsetroundjoin%
\definecolor{currentfill}{rgb}{0.337255,0.713725,0.627451}%
\pgfsetfillcolor{currentfill}%
\pgfsetlinewidth{1.003750pt}%
\definecolor{currentstroke}{rgb}{0.337255,0.713725,0.627451}%
\pgfsetstrokecolor{currentstroke}%
\pgfsetdash{}{0pt}%
\pgfsys@defobject{currentmarker}{\pgfqpoint{-0.020833in}{-0.020833in}}{\pgfqpoint{0.020833in}{0.020833in}}{%
\pgfpathmoveto{\pgfqpoint{0.000000in}{-0.020833in}}%
\pgfpathcurveto{\pgfqpoint{0.005525in}{-0.020833in}}{\pgfqpoint{0.010825in}{-0.018638in}}{\pgfqpoint{0.014731in}{-0.014731in}}%
\pgfpathcurveto{\pgfqpoint{0.018638in}{-0.010825in}}{\pgfqpoint{0.020833in}{-0.005525in}}{\pgfqpoint{0.020833in}{0.000000in}}%
\pgfpathcurveto{\pgfqpoint{0.020833in}{0.005525in}}{\pgfqpoint{0.018638in}{0.010825in}}{\pgfqpoint{0.014731in}{0.014731in}}%
\pgfpathcurveto{\pgfqpoint{0.010825in}{0.018638in}}{\pgfqpoint{0.005525in}{0.020833in}}{\pgfqpoint{0.000000in}{0.020833in}}%
\pgfpathcurveto{\pgfqpoint{-0.005525in}{0.020833in}}{\pgfqpoint{-0.010825in}{0.018638in}}{\pgfqpoint{-0.014731in}{0.014731in}}%
\pgfpathcurveto{\pgfqpoint{-0.018638in}{0.010825in}}{\pgfqpoint{-0.020833in}{0.005525in}}{\pgfqpoint{-0.020833in}{0.000000in}}%
\pgfpathcurveto{\pgfqpoint{-0.020833in}{-0.005525in}}{\pgfqpoint{-0.018638in}{-0.010825in}}{\pgfqpoint{-0.014731in}{-0.014731in}}%
\pgfpathcurveto{\pgfqpoint{-0.010825in}{-0.018638in}}{\pgfqpoint{-0.005525in}{-0.020833in}}{\pgfqpoint{0.000000in}{-0.020833in}}%
\pgfpathlineto{\pgfqpoint{0.000000in}{-0.020833in}}%
\pgfpathclose%
\pgfusepath{stroke,fill}%
}%
\begin{pgfscope}%
\pgfsys@transformshift{5.237500in}{2.451224in}%
\pgfsys@useobject{currentmarker}{}%
\end{pgfscope}%
\end{pgfscope}%
\begin{pgfscope}%
\definecolor{textcolor}{rgb}{0.000000,0.000000,0.000000}%
\pgfsetstrokecolor{textcolor}%
\pgfsetfillcolor{textcolor}%
\pgftext[x=5.462500in,y=2.407474in,left,base]{\color{textcolor}{\ifdefined\pdftexversion\else\setmainfont{NanumMyeongjo}\rmfamily\fi\fontsize{9.000000}{10.800000}\selectfont\catcode`\^=\active\def^{\ifmmode\sp\else\^{}\fi}\catcode`\%=\active\def%{\%}사료포함}}%
\end{pgfscope}%
\begin{pgfscope}%
\definecolor{textcolor}{rgb}{0.333333,0.333333,0.333333}%
\pgfsetstrokecolor{textcolor}%
\pgfsetfillcolor{textcolor}%
\pgftext[x=2.187500in,y=0.159722in,,top]{\color{textcolor}{\ifdefined\pdftexversion\else\setmainfont{NanumMyeongjo}\rmfamily\fi\fontsize{9.000000}{10.800000}\selectfont\catcode`\^=\active\def^{\ifmmode\sp\else\^{}\fi}\catcode`\%=\active\def%{\%}출처: 국가농식품통계서비스(KASS) 자료 기반 저자 작성}}%
\end{pgfscope}%
\begin{pgfscope}%
\definecolor{textcolor}{rgb}{0.333333,0.333333,0.333333}%
\pgfsetstrokecolor{textcolor}%
\pgfsetfillcolor{textcolor}%
\pgftext[x=4.687500in,y=2.970833in,,top]{\color{textcolor}{\ifdefined\pdftexversion\else\setmainfont{NanumMyeongjo}\rmfamily\fi\fontsize{9.000000}{10.800000}\selectfont\catcode`\^=\active\def^{\ifmmode\sp\else\^{}\fi}\catcode`\%=\active\def%{\%}(단위: %)}}%
\end{pgfscope}%
\end{pgfpicture}%
\makeatother%
\endgroup%
}
\end{center}
}

\slide
{\maintitle}
{\chapterone}
{콩 수급방정식}{
\hfill \normalsize (단위: 1,000톤)
\large{
\begin{center}
\renewcommand{\arraystretch}{1} 
\begin{tabularx}{\textwidth}{c|Y|c|Y|Y|Y}
\hline
\rowcolor{gray!30}
연도 & 국내 생산량 & = & 국내 소비량 & 순수출량 & 재고 변화량 \\
\hline\hline
% 여기부터 CSV 데이터 행 자동 삽입
\csvreader[
  late after line=\\\hline
]{asset/콩_수급균형_modified.csv}{}{% head to column names 제거, 인덱스 기반 컬럼 사용
  \csvcoli & \csvcolii & = & \csvcoliii & \csvcoliv & \csvcolv 
}
\end{tabularx}
\end{center}
}
\small {출처: FAOSTAT} \\ [5pt]
{\large - 대외무역과 재고행위를 고려한 수급균형방정식 : \par
\hspace{5pt} 국내 생산량 = 국내 소비량 + 순수출량 + 재고 변화량}
}


\slide
{\maintitle}
{\chapterone}
{콩 수급 변동 분석}{
\hfill \normalsize (단위: 1,000톤)
\large{
\begin{center}
\renewcommand{\arraystretch}{1} 
\begin{tabularx}{\textwidth}{c|Y|c|Y|Y|Y}
\hline
\rowcolor{gray!30}
연도 & Δ국내 생산량 & = & Δ국내 소비량 & Δ순수출량 & Δ재고 변화량 \\
\hline\hline
% 여기부터 CSV 데이터 행 자동 삽입
\csvreader[
  late after line=\\\hline
]{asset/콩_수급균형_diff.csv}{}{% head to column names 제거, 인덱스 기반 컬럼 사용
  \csvcoli & \csvcolii & = & \csvcoliii & \csvcoliv & \csvcolv
}
\end{tabularx}
\end{center}
}
\normalsize { Δ국내 생산량 = Δ국내 소비량 + Δ순수출량 + Δ재고 변화량} 
}


\slide
{\maintitle}
{\chapterone}
{국내 콩 생산량}{
\begin{center}
    \hspace*{-55pt}\resizebox{1.5\textwidth}{!}{%% Creator: Matplotlib, PGF backend
%%
%% To include the figure in your LaTeX document, write
%%   \input{<filename>.pgf}
%%
%% Make sure the required packages are loaded in your preamble
%%   \usepackage{pgf}
%%
%% Also ensure that all the required font packages are loaded; for instance,
%% the lmodern package is sometimes necessary when using math font.
%%   \usepackage{lmodern}
%%
%% Figures using additional raster images can only be included by \input if
%% they are in the same directory as the main LaTeX file. For loading figures
%% from other directories you can use the `import` package
%%   \usepackage{import}
%%
%% and then include the figures with
%%   \import{<path to file>}{<filename>.pgf}
%%
%% Matplotlib used the following preamble
%%   \def\mathdefault#1{#1}
%%   \everymath=\expandafter{\the\everymath\displaystyle}
%%   \IfFileExists{scrextend.sty}{
%%     \usepackage[fontsize=5.000000pt]{scrextend}
%%   }{
%%     \renewcommand{\normalsize}{\fontsize{5.000000}{6.000000}\selectfont}
%%     \normalsize
%%   }
%%   
%%   \ifdefined\pdftexversion\else  % non-pdftex case.
%%     \usepackage{fontspec}
%%     \setmainfont{DejaVuSerif.ttf}[Path=\detokenize{/home/user/.cache/pypoetry/virtualenvs/graph-KASAOWVd-py3.12/lib/python3.12/site-packages/matplotlib/mpl-data/fonts/ttf/}]
%%     \setsansfont{DejaVuSans.ttf}[Path=\detokenize{/home/user/.cache/pypoetry/virtualenvs/graph-KASAOWVd-py3.12/lib/python3.12/site-packages/matplotlib/mpl-data/fonts/ttf/}]
%%     \setmonofont{DejaVuSansMono.ttf}[Path=\detokenize{/home/user/.cache/pypoetry/virtualenvs/graph-KASAOWVd-py3.12/lib/python3.12/site-packages/matplotlib/mpl-data/fonts/ttf/}]
%%   \fi
%%   \makeatletter\@ifpackageloaded{underscore}{}{\usepackage[strings]{underscore}}\makeatother
%%
\begingroup%
\makeatletter%
\begin{pgfpicture}%
\pgfpathrectangle{\pgfpointorigin}{\pgfqpoint{6.944444in}{2.777778in}}%
\pgfusepath{use as bounding box, clip}%
\begin{pgfscope}%
\pgfsetbuttcap%
\pgfsetmiterjoin%
\definecolor{currentfill}{rgb}{1.000000,1.000000,1.000000}%
\pgfsetfillcolor{currentfill}%
\pgfsetlinewidth{0.000000pt}%
\definecolor{currentstroke}{rgb}{1.000000,1.000000,1.000000}%
\pgfsetstrokecolor{currentstroke}%
\pgfsetdash{}{0pt}%
\pgfpathmoveto{\pgfqpoint{0.000000in}{0.000000in}}%
\pgfpathlineto{\pgfqpoint{6.944444in}{0.000000in}}%
\pgfpathlineto{\pgfqpoint{6.944444in}{2.777778in}}%
\pgfpathlineto{\pgfqpoint{0.000000in}{2.777778in}}%
\pgfpathlineto{\pgfqpoint{0.000000in}{0.000000in}}%
\pgfpathclose%
\pgfusepath{fill}%
\end{pgfscope}%
\begin{pgfscope}%
\pgfsetbuttcap%
\pgfsetmiterjoin%
\definecolor{currentfill}{rgb}{1.000000,1.000000,1.000000}%
\pgfsetfillcolor{currentfill}%
\pgfsetlinewidth{0.000000pt}%
\definecolor{currentstroke}{rgb}{0.000000,0.000000,0.000000}%
\pgfsetstrokecolor{currentstroke}%
\pgfsetstrokeopacity{0.000000}%
\pgfsetdash{}{0pt}%
\pgfpathmoveto{\pgfqpoint{0.868056in}{0.555556in}}%
\pgfpathlineto{\pgfqpoint{4.861111in}{0.555556in}}%
\pgfpathlineto{\pgfqpoint{4.861111in}{2.444444in}}%
\pgfpathlineto{\pgfqpoint{0.868056in}{2.444444in}}%
\pgfpathlineto{\pgfqpoint{0.868056in}{0.555556in}}%
\pgfpathclose%
\pgfusepath{fill}%
\end{pgfscope}%
\begin{pgfscope}%
\pgfsetbuttcap%
\pgfsetroundjoin%
\definecolor{currentfill}{rgb}{0.000000,0.000000,0.000000}%
\pgfsetfillcolor{currentfill}%
\pgfsetlinewidth{0.752812pt}%
\definecolor{currentstroke}{rgb}{0.000000,0.000000,0.000000}%
\pgfsetstrokecolor{currentstroke}%
\pgfsetdash{}{0pt}%
\pgfsys@defobject{currentmarker}{\pgfqpoint{0.000000in}{-0.013889in}}{\pgfqpoint{0.000000in}{0.000000in}}{%
\pgfpathmoveto{\pgfqpoint{0.000000in}{0.000000in}}%
\pgfpathlineto{\pgfqpoint{0.000000in}{-0.013889in}}%
\pgfusepath{stroke,fill}%
}%
\begin{pgfscope}%
\pgfsys@transformshift{1.168298in}{0.555556in}%
\pgfsys@useobject{currentmarker}{}%
\end{pgfscope}%
\end{pgfscope}%
\begin{pgfscope}%
\definecolor{textcolor}{rgb}{0.000000,0.000000,0.000000}%
\pgfsetstrokecolor{textcolor}%
\pgfsetfillcolor{textcolor}%
\pgftext[x=1.168298in,y=0.493056in,,top]{\color{textcolor}{\ifdefined\pdftexversion\else\setmainfont{NanumMyeongjo}\rmfamily\fi\fontsize{5.000000}{6.000000}\selectfont\catcode`\^=\active\def^{\ifmmode\sp\else\^{}\fi}\catcode`\%=\active\def%{\%}2014}}%
\end{pgfscope}%
\begin{pgfscope}%
\pgfsetbuttcap%
\pgfsetroundjoin%
\definecolor{currentfill}{rgb}{0.000000,0.000000,0.000000}%
\pgfsetfillcolor{currentfill}%
\pgfsetlinewidth{0.752812pt}%
\definecolor{currentstroke}{rgb}{0.000000,0.000000,0.000000}%
\pgfsetstrokecolor{currentstroke}%
\pgfsetdash{}{0pt}%
\pgfsys@defobject{currentmarker}{\pgfqpoint{0.000000in}{-0.013889in}}{\pgfqpoint{0.000000in}{0.000000in}}{%
\pgfpathmoveto{\pgfqpoint{0.000000in}{0.000000in}}%
\pgfpathlineto{\pgfqpoint{0.000000in}{-0.013889in}}%
\pgfusepath{stroke,fill}%
}%
\begin{pgfscope}%
\pgfsys@transformshift{1.507555in}{0.555556in}%
\pgfsys@useobject{currentmarker}{}%
\end{pgfscope}%
\end{pgfscope}%
\begin{pgfscope}%
\definecolor{textcolor}{rgb}{0.000000,0.000000,0.000000}%
\pgfsetstrokecolor{textcolor}%
\pgfsetfillcolor{textcolor}%
\pgftext[x=1.507555in,y=0.493056in,,top]{\color{textcolor}{\ifdefined\pdftexversion\else\setmainfont{NanumMyeongjo}\rmfamily\fi\fontsize{5.000000}{6.000000}\selectfont\catcode`\^=\active\def^{\ifmmode\sp\else\^{}\fi}\catcode`\%=\active\def%{\%}2015}}%
\end{pgfscope}%
\begin{pgfscope}%
\pgfsetbuttcap%
\pgfsetroundjoin%
\definecolor{currentfill}{rgb}{0.000000,0.000000,0.000000}%
\pgfsetfillcolor{currentfill}%
\pgfsetlinewidth{0.752812pt}%
\definecolor{currentstroke}{rgb}{0.000000,0.000000,0.000000}%
\pgfsetstrokecolor{currentstroke}%
\pgfsetdash{}{0pt}%
\pgfsys@defobject{currentmarker}{\pgfqpoint{0.000000in}{-0.013889in}}{\pgfqpoint{0.000000in}{0.000000in}}{%
\pgfpathmoveto{\pgfqpoint{0.000000in}{0.000000in}}%
\pgfpathlineto{\pgfqpoint{0.000000in}{-0.013889in}}%
\pgfusepath{stroke,fill}%
}%
\begin{pgfscope}%
\pgfsys@transformshift{1.846812in}{0.555556in}%
\pgfsys@useobject{currentmarker}{}%
\end{pgfscope}%
\end{pgfscope}%
\begin{pgfscope}%
\definecolor{textcolor}{rgb}{0.000000,0.000000,0.000000}%
\pgfsetstrokecolor{textcolor}%
\pgfsetfillcolor{textcolor}%
\pgftext[x=1.846812in,y=0.493056in,,top]{\color{textcolor}{\ifdefined\pdftexversion\else\setmainfont{NanumMyeongjo}\rmfamily\fi\fontsize{5.000000}{6.000000}\selectfont\catcode`\^=\active\def^{\ifmmode\sp\else\^{}\fi}\catcode`\%=\active\def%{\%}2016}}%
\end{pgfscope}%
\begin{pgfscope}%
\pgfsetbuttcap%
\pgfsetroundjoin%
\definecolor{currentfill}{rgb}{0.000000,0.000000,0.000000}%
\pgfsetfillcolor{currentfill}%
\pgfsetlinewidth{0.752812pt}%
\definecolor{currentstroke}{rgb}{0.000000,0.000000,0.000000}%
\pgfsetstrokecolor{currentstroke}%
\pgfsetdash{}{0pt}%
\pgfsys@defobject{currentmarker}{\pgfqpoint{0.000000in}{-0.013889in}}{\pgfqpoint{0.000000in}{0.000000in}}{%
\pgfpathmoveto{\pgfqpoint{0.000000in}{0.000000in}}%
\pgfpathlineto{\pgfqpoint{0.000000in}{-0.013889in}}%
\pgfusepath{stroke,fill}%
}%
\begin{pgfscope}%
\pgfsys@transformshift{2.186069in}{0.555556in}%
\pgfsys@useobject{currentmarker}{}%
\end{pgfscope}%
\end{pgfscope}%
\begin{pgfscope}%
\definecolor{textcolor}{rgb}{0.000000,0.000000,0.000000}%
\pgfsetstrokecolor{textcolor}%
\pgfsetfillcolor{textcolor}%
\pgftext[x=2.186069in,y=0.493056in,,top]{\color{textcolor}{\ifdefined\pdftexversion\else\setmainfont{NanumMyeongjo}\rmfamily\fi\fontsize{5.000000}{6.000000}\selectfont\catcode`\^=\active\def^{\ifmmode\sp\else\^{}\fi}\catcode`\%=\active\def%{\%}2017}}%
\end{pgfscope}%
\begin{pgfscope}%
\pgfsetbuttcap%
\pgfsetroundjoin%
\definecolor{currentfill}{rgb}{0.000000,0.000000,0.000000}%
\pgfsetfillcolor{currentfill}%
\pgfsetlinewidth{0.752812pt}%
\definecolor{currentstroke}{rgb}{0.000000,0.000000,0.000000}%
\pgfsetstrokecolor{currentstroke}%
\pgfsetdash{}{0pt}%
\pgfsys@defobject{currentmarker}{\pgfqpoint{0.000000in}{-0.013889in}}{\pgfqpoint{0.000000in}{0.000000in}}{%
\pgfpathmoveto{\pgfqpoint{0.000000in}{0.000000in}}%
\pgfpathlineto{\pgfqpoint{0.000000in}{-0.013889in}}%
\pgfusepath{stroke,fill}%
}%
\begin{pgfscope}%
\pgfsys@transformshift{2.525326in}{0.555556in}%
\pgfsys@useobject{currentmarker}{}%
\end{pgfscope}%
\end{pgfscope}%
\begin{pgfscope}%
\definecolor{textcolor}{rgb}{0.000000,0.000000,0.000000}%
\pgfsetstrokecolor{textcolor}%
\pgfsetfillcolor{textcolor}%
\pgftext[x=2.525326in,y=0.493056in,,top]{\color{textcolor}{\ifdefined\pdftexversion\else\setmainfont{NanumMyeongjo}\rmfamily\fi\fontsize{5.000000}{6.000000}\selectfont\catcode`\^=\active\def^{\ifmmode\sp\else\^{}\fi}\catcode`\%=\active\def%{\%}2018}}%
\end{pgfscope}%
\begin{pgfscope}%
\pgfsetbuttcap%
\pgfsetroundjoin%
\definecolor{currentfill}{rgb}{0.000000,0.000000,0.000000}%
\pgfsetfillcolor{currentfill}%
\pgfsetlinewidth{0.752812pt}%
\definecolor{currentstroke}{rgb}{0.000000,0.000000,0.000000}%
\pgfsetstrokecolor{currentstroke}%
\pgfsetdash{}{0pt}%
\pgfsys@defobject{currentmarker}{\pgfqpoint{0.000000in}{-0.013889in}}{\pgfqpoint{0.000000in}{0.000000in}}{%
\pgfpathmoveto{\pgfqpoint{0.000000in}{0.000000in}}%
\pgfpathlineto{\pgfqpoint{0.000000in}{-0.013889in}}%
\pgfusepath{stroke,fill}%
}%
\begin{pgfscope}%
\pgfsys@transformshift{2.864583in}{0.555556in}%
\pgfsys@useobject{currentmarker}{}%
\end{pgfscope}%
\end{pgfscope}%
\begin{pgfscope}%
\definecolor{textcolor}{rgb}{0.000000,0.000000,0.000000}%
\pgfsetstrokecolor{textcolor}%
\pgfsetfillcolor{textcolor}%
\pgftext[x=2.864583in,y=0.493056in,,top]{\color{textcolor}{\ifdefined\pdftexversion\else\setmainfont{NanumMyeongjo}\rmfamily\fi\fontsize{5.000000}{6.000000}\selectfont\catcode`\^=\active\def^{\ifmmode\sp\else\^{}\fi}\catcode`\%=\active\def%{\%}2019}}%
\end{pgfscope}%
\begin{pgfscope}%
\pgfsetbuttcap%
\pgfsetroundjoin%
\definecolor{currentfill}{rgb}{0.000000,0.000000,0.000000}%
\pgfsetfillcolor{currentfill}%
\pgfsetlinewidth{0.752812pt}%
\definecolor{currentstroke}{rgb}{0.000000,0.000000,0.000000}%
\pgfsetstrokecolor{currentstroke}%
\pgfsetdash{}{0pt}%
\pgfsys@defobject{currentmarker}{\pgfqpoint{0.000000in}{-0.013889in}}{\pgfqpoint{0.000000in}{0.000000in}}{%
\pgfpathmoveto{\pgfqpoint{0.000000in}{0.000000in}}%
\pgfpathlineto{\pgfqpoint{0.000000in}{-0.013889in}}%
\pgfusepath{stroke,fill}%
}%
\begin{pgfscope}%
\pgfsys@transformshift{3.203840in}{0.555556in}%
\pgfsys@useobject{currentmarker}{}%
\end{pgfscope}%
\end{pgfscope}%
\begin{pgfscope}%
\definecolor{textcolor}{rgb}{0.000000,0.000000,0.000000}%
\pgfsetstrokecolor{textcolor}%
\pgfsetfillcolor{textcolor}%
\pgftext[x=3.203840in,y=0.493056in,,top]{\color{textcolor}{\ifdefined\pdftexversion\else\setmainfont{NanumMyeongjo}\rmfamily\fi\fontsize{5.000000}{6.000000}\selectfont\catcode`\^=\active\def^{\ifmmode\sp\else\^{}\fi}\catcode`\%=\active\def%{\%}2020}}%
\end{pgfscope}%
\begin{pgfscope}%
\pgfsetbuttcap%
\pgfsetroundjoin%
\definecolor{currentfill}{rgb}{0.000000,0.000000,0.000000}%
\pgfsetfillcolor{currentfill}%
\pgfsetlinewidth{0.752812pt}%
\definecolor{currentstroke}{rgb}{0.000000,0.000000,0.000000}%
\pgfsetstrokecolor{currentstroke}%
\pgfsetdash{}{0pt}%
\pgfsys@defobject{currentmarker}{\pgfqpoint{0.000000in}{-0.013889in}}{\pgfqpoint{0.000000in}{0.000000in}}{%
\pgfpathmoveto{\pgfqpoint{0.000000in}{0.000000in}}%
\pgfpathlineto{\pgfqpoint{0.000000in}{-0.013889in}}%
\pgfusepath{stroke,fill}%
}%
\begin{pgfscope}%
\pgfsys@transformshift{3.543097in}{0.555556in}%
\pgfsys@useobject{currentmarker}{}%
\end{pgfscope}%
\end{pgfscope}%
\begin{pgfscope}%
\definecolor{textcolor}{rgb}{0.000000,0.000000,0.000000}%
\pgfsetstrokecolor{textcolor}%
\pgfsetfillcolor{textcolor}%
\pgftext[x=3.543097in,y=0.493056in,,top]{\color{textcolor}{\ifdefined\pdftexversion\else\setmainfont{NanumMyeongjo}\rmfamily\fi\fontsize{5.000000}{6.000000}\selectfont\catcode`\^=\active\def^{\ifmmode\sp\else\^{}\fi}\catcode`\%=\active\def%{\%}2021}}%
\end{pgfscope}%
\begin{pgfscope}%
\pgfsetbuttcap%
\pgfsetroundjoin%
\definecolor{currentfill}{rgb}{0.000000,0.000000,0.000000}%
\pgfsetfillcolor{currentfill}%
\pgfsetlinewidth{0.752812pt}%
\definecolor{currentstroke}{rgb}{0.000000,0.000000,0.000000}%
\pgfsetstrokecolor{currentstroke}%
\pgfsetdash{}{0pt}%
\pgfsys@defobject{currentmarker}{\pgfqpoint{0.000000in}{-0.013889in}}{\pgfqpoint{0.000000in}{0.000000in}}{%
\pgfpathmoveto{\pgfqpoint{0.000000in}{0.000000in}}%
\pgfpathlineto{\pgfqpoint{0.000000in}{-0.013889in}}%
\pgfusepath{stroke,fill}%
}%
\begin{pgfscope}%
\pgfsys@transformshift{3.882355in}{0.555556in}%
\pgfsys@useobject{currentmarker}{}%
\end{pgfscope}%
\end{pgfscope}%
\begin{pgfscope}%
\definecolor{textcolor}{rgb}{0.000000,0.000000,0.000000}%
\pgfsetstrokecolor{textcolor}%
\pgfsetfillcolor{textcolor}%
\pgftext[x=3.882355in,y=0.493056in,,top]{\color{textcolor}{\ifdefined\pdftexversion\else\setmainfont{NanumMyeongjo}\rmfamily\fi\fontsize{5.000000}{6.000000}\selectfont\catcode`\^=\active\def^{\ifmmode\sp\else\^{}\fi}\catcode`\%=\active\def%{\%}2022}}%
\end{pgfscope}%
\begin{pgfscope}%
\pgfsetbuttcap%
\pgfsetroundjoin%
\definecolor{currentfill}{rgb}{0.000000,0.000000,0.000000}%
\pgfsetfillcolor{currentfill}%
\pgfsetlinewidth{0.752812pt}%
\definecolor{currentstroke}{rgb}{0.000000,0.000000,0.000000}%
\pgfsetstrokecolor{currentstroke}%
\pgfsetdash{}{0pt}%
\pgfsys@defobject{currentmarker}{\pgfqpoint{0.000000in}{-0.013889in}}{\pgfqpoint{0.000000in}{0.000000in}}{%
\pgfpathmoveto{\pgfqpoint{0.000000in}{0.000000in}}%
\pgfpathlineto{\pgfqpoint{0.000000in}{-0.013889in}}%
\pgfusepath{stroke,fill}%
}%
\begin{pgfscope}%
\pgfsys@transformshift{4.221612in}{0.555556in}%
\pgfsys@useobject{currentmarker}{}%
\end{pgfscope}%
\end{pgfscope}%
\begin{pgfscope}%
\definecolor{textcolor}{rgb}{0.000000,0.000000,0.000000}%
\pgfsetstrokecolor{textcolor}%
\pgfsetfillcolor{textcolor}%
\pgftext[x=4.221612in,y=0.493056in,,top]{\color{textcolor}{\ifdefined\pdftexversion\else\setmainfont{NanumMyeongjo}\rmfamily\fi\fontsize{5.000000}{6.000000}\selectfont\catcode`\^=\active\def^{\ifmmode\sp\else\^{}\fi}\catcode`\%=\active\def%{\%}2023}}%
\end{pgfscope}%
\begin{pgfscope}%
\pgfsetbuttcap%
\pgfsetroundjoin%
\definecolor{currentfill}{rgb}{0.000000,0.000000,0.000000}%
\pgfsetfillcolor{currentfill}%
\pgfsetlinewidth{0.752812pt}%
\definecolor{currentstroke}{rgb}{0.000000,0.000000,0.000000}%
\pgfsetstrokecolor{currentstroke}%
\pgfsetdash{}{0pt}%
\pgfsys@defobject{currentmarker}{\pgfqpoint{0.000000in}{-0.013889in}}{\pgfqpoint{0.000000in}{0.000000in}}{%
\pgfpathmoveto{\pgfqpoint{0.000000in}{0.000000in}}%
\pgfpathlineto{\pgfqpoint{0.000000in}{-0.013889in}}%
\pgfusepath{stroke,fill}%
}%
\begin{pgfscope}%
\pgfsys@transformshift{4.560869in}{0.555556in}%
\pgfsys@useobject{currentmarker}{}%
\end{pgfscope}%
\end{pgfscope}%
\begin{pgfscope}%
\definecolor{textcolor}{rgb}{0.000000,0.000000,0.000000}%
\pgfsetstrokecolor{textcolor}%
\pgfsetfillcolor{textcolor}%
\pgftext[x=4.560869in,y=0.493056in,,top]{\color{textcolor}{\ifdefined\pdftexversion\else\setmainfont{NanumMyeongjo}\rmfamily\fi\fontsize{5.000000}{6.000000}\selectfont\catcode`\^=\active\def^{\ifmmode\sp\else\^{}\fi}\catcode`\%=\active\def%{\%}2024}}%
\end{pgfscope}%
\begin{pgfscope}%
\pgfpathrectangle{\pgfqpoint{0.868056in}{0.555556in}}{\pgfqpoint{3.993056in}{1.888889in}}%
\pgfusepath{clip}%
\pgfsetbuttcap%
\pgfsetroundjoin%
\pgfsetlinewidth{0.602250pt}%
\definecolor{currentstroke}{rgb}{0.690196,0.690196,0.690196}%
\pgfsetstrokecolor{currentstroke}%
\pgfsetstrokeopacity{0.450000}%
\pgfsetdash{{2.220000pt}{0.960000pt}}{0.000000pt}%
\pgfpathmoveto{\pgfqpoint{0.868056in}{0.555556in}}%
\pgfpathlineto{\pgfqpoint{4.861111in}{0.555556in}}%
\pgfusepath{stroke}%
\end{pgfscope}%
\begin{pgfscope}%
\pgfsetbuttcap%
\pgfsetroundjoin%
\definecolor{currentfill}{rgb}{0.000000,0.000000,0.000000}%
\pgfsetfillcolor{currentfill}%
\pgfsetlinewidth{0.752812pt}%
\definecolor{currentstroke}{rgb}{0.000000,0.000000,0.000000}%
\pgfsetstrokecolor{currentstroke}%
\pgfsetdash{}{0pt}%
\pgfsys@defobject{currentmarker}{\pgfqpoint{-0.013889in}{0.000000in}}{\pgfqpoint{-0.000000in}{0.000000in}}{%
\pgfpathmoveto{\pgfqpoint{-0.000000in}{0.000000in}}%
\pgfpathlineto{\pgfqpoint{-0.013889in}{0.000000in}}%
\pgfusepath{stroke,fill}%
}%
\begin{pgfscope}%
\pgfsys@transformshift{0.868056in}{0.555556in}%
\pgfsys@useobject{currentmarker}{}%
\end{pgfscope}%
\end{pgfscope}%
\begin{pgfscope}%
\definecolor{textcolor}{rgb}{0.000000,0.000000,0.000000}%
\pgfsetstrokecolor{textcolor}%
\pgfsetfillcolor{textcolor}%
\pgftext[x=0.768256in, y=0.527818in, left, base]{\color{textcolor}{\ifdefined\pdftexversion\else\setmainfont{NanumMyeongjo}\rmfamily\fi\fontsize{5.000000}{6.000000}\selectfont\catcode`\^=\active\def^{\ifmmode\sp\else\^{}\fi}\catcode`\%=\active\def%{\%}0}}%
\end{pgfscope}%
\begin{pgfscope}%
\pgfpathrectangle{\pgfqpoint{0.868056in}{0.555556in}}{\pgfqpoint{3.993056in}{1.888889in}}%
\pgfusepath{clip}%
\pgfsetbuttcap%
\pgfsetroundjoin%
\pgfsetlinewidth{0.602250pt}%
\definecolor{currentstroke}{rgb}{0.690196,0.690196,0.690196}%
\pgfsetstrokecolor{currentstroke}%
\pgfsetstrokeopacity{0.450000}%
\pgfsetdash{{2.220000pt}{0.960000pt}}{0.000000pt}%
\pgfpathmoveto{\pgfqpoint{0.868056in}{0.762577in}}%
\pgfpathlineto{\pgfqpoint{4.861111in}{0.762577in}}%
\pgfusepath{stroke}%
\end{pgfscope}%
\begin{pgfscope}%
\pgfsetbuttcap%
\pgfsetroundjoin%
\definecolor{currentfill}{rgb}{0.000000,0.000000,0.000000}%
\pgfsetfillcolor{currentfill}%
\pgfsetlinewidth{0.752812pt}%
\definecolor{currentstroke}{rgb}{0.000000,0.000000,0.000000}%
\pgfsetstrokecolor{currentstroke}%
\pgfsetdash{}{0pt}%
\pgfsys@defobject{currentmarker}{\pgfqpoint{-0.013889in}{0.000000in}}{\pgfqpoint{-0.000000in}{0.000000in}}{%
\pgfpathmoveto{\pgfqpoint{-0.000000in}{0.000000in}}%
\pgfpathlineto{\pgfqpoint{-0.013889in}{0.000000in}}%
\pgfusepath{stroke,fill}%
}%
\begin{pgfscope}%
\pgfsys@transformshift{0.868056in}{0.762577in}%
\pgfsys@useobject{currentmarker}{}%
\end{pgfscope}%
\end{pgfscope}%
\begin{pgfscope}%
\definecolor{textcolor}{rgb}{0.000000,0.000000,0.000000}%
\pgfsetstrokecolor{textcolor}%
\pgfsetfillcolor{textcolor}%
\pgftext[x=0.702271in, y=0.734840in, left, base]{\color{textcolor}{\ifdefined\pdftexversion\else\setmainfont{NanumMyeongjo}\rmfamily\fi\fontsize{5.000000}{6.000000}\selectfont\catcode`\^=\active\def^{\ifmmode\sp\else\^{}\fi}\catcode`\%=\active\def%{\%}2만}}%
\end{pgfscope}%
\begin{pgfscope}%
\pgfpathrectangle{\pgfqpoint{0.868056in}{0.555556in}}{\pgfqpoint{3.993056in}{1.888889in}}%
\pgfusepath{clip}%
\pgfsetbuttcap%
\pgfsetroundjoin%
\pgfsetlinewidth{0.602250pt}%
\definecolor{currentstroke}{rgb}{0.690196,0.690196,0.690196}%
\pgfsetstrokecolor{currentstroke}%
\pgfsetstrokeopacity{0.450000}%
\pgfsetdash{{2.220000pt}{0.960000pt}}{0.000000pt}%
\pgfpathmoveto{\pgfqpoint{0.868056in}{0.969598in}}%
\pgfpathlineto{\pgfqpoint{4.861111in}{0.969598in}}%
\pgfusepath{stroke}%
\end{pgfscope}%
\begin{pgfscope}%
\pgfsetbuttcap%
\pgfsetroundjoin%
\definecolor{currentfill}{rgb}{0.000000,0.000000,0.000000}%
\pgfsetfillcolor{currentfill}%
\pgfsetlinewidth{0.752812pt}%
\definecolor{currentstroke}{rgb}{0.000000,0.000000,0.000000}%
\pgfsetstrokecolor{currentstroke}%
\pgfsetdash{}{0pt}%
\pgfsys@defobject{currentmarker}{\pgfqpoint{-0.013889in}{0.000000in}}{\pgfqpoint{-0.000000in}{0.000000in}}{%
\pgfpathmoveto{\pgfqpoint{-0.000000in}{0.000000in}}%
\pgfpathlineto{\pgfqpoint{-0.013889in}{0.000000in}}%
\pgfusepath{stroke,fill}%
}%
\begin{pgfscope}%
\pgfsys@transformshift{0.868056in}{0.969598in}%
\pgfsys@useobject{currentmarker}{}%
\end{pgfscope}%
\end{pgfscope}%
\begin{pgfscope}%
\definecolor{textcolor}{rgb}{0.000000,0.000000,0.000000}%
\pgfsetstrokecolor{textcolor}%
\pgfsetfillcolor{textcolor}%
\pgftext[x=0.702271in, y=0.941861in, left, base]{\color{textcolor}{\ifdefined\pdftexversion\else\setmainfont{NanumMyeongjo}\rmfamily\fi\fontsize{5.000000}{6.000000}\selectfont\catcode`\^=\active\def^{\ifmmode\sp\else\^{}\fi}\catcode`\%=\active\def%{\%}4만}}%
\end{pgfscope}%
\begin{pgfscope}%
\pgfpathrectangle{\pgfqpoint{0.868056in}{0.555556in}}{\pgfqpoint{3.993056in}{1.888889in}}%
\pgfusepath{clip}%
\pgfsetbuttcap%
\pgfsetroundjoin%
\pgfsetlinewidth{0.602250pt}%
\definecolor{currentstroke}{rgb}{0.690196,0.690196,0.690196}%
\pgfsetstrokecolor{currentstroke}%
\pgfsetstrokeopacity{0.450000}%
\pgfsetdash{{2.220000pt}{0.960000pt}}{0.000000pt}%
\pgfpathmoveto{\pgfqpoint{0.868056in}{1.176619in}}%
\pgfpathlineto{\pgfqpoint{4.861111in}{1.176619in}}%
\pgfusepath{stroke}%
\end{pgfscope}%
\begin{pgfscope}%
\pgfsetbuttcap%
\pgfsetroundjoin%
\definecolor{currentfill}{rgb}{0.000000,0.000000,0.000000}%
\pgfsetfillcolor{currentfill}%
\pgfsetlinewidth{0.752812pt}%
\definecolor{currentstroke}{rgb}{0.000000,0.000000,0.000000}%
\pgfsetstrokecolor{currentstroke}%
\pgfsetdash{}{0pt}%
\pgfsys@defobject{currentmarker}{\pgfqpoint{-0.013889in}{0.000000in}}{\pgfqpoint{-0.000000in}{0.000000in}}{%
\pgfpathmoveto{\pgfqpoint{-0.000000in}{0.000000in}}%
\pgfpathlineto{\pgfqpoint{-0.013889in}{0.000000in}}%
\pgfusepath{stroke,fill}%
}%
\begin{pgfscope}%
\pgfsys@transformshift{0.868056in}{1.176619in}%
\pgfsys@useobject{currentmarker}{}%
\end{pgfscope}%
\end{pgfscope}%
\begin{pgfscope}%
\definecolor{textcolor}{rgb}{0.000000,0.000000,0.000000}%
\pgfsetstrokecolor{textcolor}%
\pgfsetfillcolor{textcolor}%
\pgftext[x=0.702271in, y=1.148882in, left, base]{\color{textcolor}{\ifdefined\pdftexversion\else\setmainfont{NanumMyeongjo}\rmfamily\fi\fontsize{5.000000}{6.000000}\selectfont\catcode`\^=\active\def^{\ifmmode\sp\else\^{}\fi}\catcode`\%=\active\def%{\%}6만}}%
\end{pgfscope}%
\begin{pgfscope}%
\pgfpathrectangle{\pgfqpoint{0.868056in}{0.555556in}}{\pgfqpoint{3.993056in}{1.888889in}}%
\pgfusepath{clip}%
\pgfsetbuttcap%
\pgfsetroundjoin%
\pgfsetlinewidth{0.602250pt}%
\definecolor{currentstroke}{rgb}{0.690196,0.690196,0.690196}%
\pgfsetstrokecolor{currentstroke}%
\pgfsetstrokeopacity{0.450000}%
\pgfsetdash{{2.220000pt}{0.960000pt}}{0.000000pt}%
\pgfpathmoveto{\pgfqpoint{0.868056in}{1.383640in}}%
\pgfpathlineto{\pgfqpoint{4.861111in}{1.383640in}}%
\pgfusepath{stroke}%
\end{pgfscope}%
\begin{pgfscope}%
\pgfsetbuttcap%
\pgfsetroundjoin%
\definecolor{currentfill}{rgb}{0.000000,0.000000,0.000000}%
\pgfsetfillcolor{currentfill}%
\pgfsetlinewidth{0.752812pt}%
\definecolor{currentstroke}{rgb}{0.000000,0.000000,0.000000}%
\pgfsetstrokecolor{currentstroke}%
\pgfsetdash{}{0pt}%
\pgfsys@defobject{currentmarker}{\pgfqpoint{-0.013889in}{0.000000in}}{\pgfqpoint{-0.000000in}{0.000000in}}{%
\pgfpathmoveto{\pgfqpoint{-0.000000in}{0.000000in}}%
\pgfpathlineto{\pgfqpoint{-0.013889in}{0.000000in}}%
\pgfusepath{stroke,fill}%
}%
\begin{pgfscope}%
\pgfsys@transformshift{0.868056in}{1.383640in}%
\pgfsys@useobject{currentmarker}{}%
\end{pgfscope}%
\end{pgfscope}%
\begin{pgfscope}%
\definecolor{textcolor}{rgb}{0.000000,0.000000,0.000000}%
\pgfsetstrokecolor{textcolor}%
\pgfsetfillcolor{textcolor}%
\pgftext[x=0.702271in, y=1.355903in, left, base]{\color{textcolor}{\ifdefined\pdftexversion\else\setmainfont{NanumMyeongjo}\rmfamily\fi\fontsize{5.000000}{6.000000}\selectfont\catcode`\^=\active\def^{\ifmmode\sp\else\^{}\fi}\catcode`\%=\active\def%{\%}8만}}%
\end{pgfscope}%
\begin{pgfscope}%
\pgfpathrectangle{\pgfqpoint{0.868056in}{0.555556in}}{\pgfqpoint{3.993056in}{1.888889in}}%
\pgfusepath{clip}%
\pgfsetbuttcap%
\pgfsetroundjoin%
\pgfsetlinewidth{0.602250pt}%
\definecolor{currentstroke}{rgb}{0.690196,0.690196,0.690196}%
\pgfsetstrokecolor{currentstroke}%
\pgfsetstrokeopacity{0.450000}%
\pgfsetdash{{2.220000pt}{0.960000pt}}{0.000000pt}%
\pgfpathmoveto{\pgfqpoint{0.868056in}{1.590662in}}%
\pgfpathlineto{\pgfqpoint{4.861111in}{1.590662in}}%
\pgfusepath{stroke}%
\end{pgfscope}%
\begin{pgfscope}%
\pgfsetbuttcap%
\pgfsetroundjoin%
\definecolor{currentfill}{rgb}{0.000000,0.000000,0.000000}%
\pgfsetfillcolor{currentfill}%
\pgfsetlinewidth{0.752812pt}%
\definecolor{currentstroke}{rgb}{0.000000,0.000000,0.000000}%
\pgfsetstrokecolor{currentstroke}%
\pgfsetdash{}{0pt}%
\pgfsys@defobject{currentmarker}{\pgfqpoint{-0.013889in}{0.000000in}}{\pgfqpoint{-0.000000in}{0.000000in}}{%
\pgfpathmoveto{\pgfqpoint{-0.000000in}{0.000000in}}%
\pgfpathlineto{\pgfqpoint{-0.013889in}{0.000000in}}%
\pgfusepath{stroke,fill}%
}%
\begin{pgfscope}%
\pgfsys@transformshift{0.868056in}{1.590662in}%
\pgfsys@useobject{currentmarker}{}%
\end{pgfscope}%
\end{pgfscope}%
\begin{pgfscope}%
\definecolor{textcolor}{rgb}{0.000000,0.000000,0.000000}%
\pgfsetstrokecolor{textcolor}%
\pgfsetfillcolor{textcolor}%
\pgftext[x=0.664971in, y=1.562924in, left, base]{\color{textcolor}{\ifdefined\pdftexversion\else\setmainfont{NanumMyeongjo}\rmfamily\fi\fontsize{5.000000}{6.000000}\selectfont\catcode`\^=\active\def^{\ifmmode\sp\else\^{}\fi}\catcode`\%=\active\def%{\%}10만}}%
\end{pgfscope}%
\begin{pgfscope}%
\pgfpathrectangle{\pgfqpoint{0.868056in}{0.555556in}}{\pgfqpoint{3.993056in}{1.888889in}}%
\pgfusepath{clip}%
\pgfsetbuttcap%
\pgfsetroundjoin%
\pgfsetlinewidth{0.602250pt}%
\definecolor{currentstroke}{rgb}{0.690196,0.690196,0.690196}%
\pgfsetstrokecolor{currentstroke}%
\pgfsetstrokeopacity{0.450000}%
\pgfsetdash{{2.220000pt}{0.960000pt}}{0.000000pt}%
\pgfpathmoveto{\pgfqpoint{0.868056in}{1.797683in}}%
\pgfpathlineto{\pgfqpoint{4.861111in}{1.797683in}}%
\pgfusepath{stroke}%
\end{pgfscope}%
\begin{pgfscope}%
\pgfsetbuttcap%
\pgfsetroundjoin%
\definecolor{currentfill}{rgb}{0.000000,0.000000,0.000000}%
\pgfsetfillcolor{currentfill}%
\pgfsetlinewidth{0.752812pt}%
\definecolor{currentstroke}{rgb}{0.000000,0.000000,0.000000}%
\pgfsetstrokecolor{currentstroke}%
\pgfsetdash{}{0pt}%
\pgfsys@defobject{currentmarker}{\pgfqpoint{-0.013889in}{0.000000in}}{\pgfqpoint{-0.000000in}{0.000000in}}{%
\pgfpathmoveto{\pgfqpoint{-0.000000in}{0.000000in}}%
\pgfpathlineto{\pgfqpoint{-0.013889in}{0.000000in}}%
\pgfusepath{stroke,fill}%
}%
\begin{pgfscope}%
\pgfsys@transformshift{0.868056in}{1.797683in}%
\pgfsys@useobject{currentmarker}{}%
\end{pgfscope}%
\end{pgfscope}%
\begin{pgfscope}%
\definecolor{textcolor}{rgb}{0.000000,0.000000,0.000000}%
\pgfsetstrokecolor{textcolor}%
\pgfsetfillcolor{textcolor}%
\pgftext[x=0.664971in, y=1.769946in, left, base]{\color{textcolor}{\ifdefined\pdftexversion\else\setmainfont{NanumMyeongjo}\rmfamily\fi\fontsize{5.000000}{6.000000}\selectfont\catcode`\^=\active\def^{\ifmmode\sp\else\^{}\fi}\catcode`\%=\active\def%{\%}12만}}%
\end{pgfscope}%
\begin{pgfscope}%
\pgfpathrectangle{\pgfqpoint{0.868056in}{0.555556in}}{\pgfqpoint{3.993056in}{1.888889in}}%
\pgfusepath{clip}%
\pgfsetbuttcap%
\pgfsetroundjoin%
\pgfsetlinewidth{0.602250pt}%
\definecolor{currentstroke}{rgb}{0.690196,0.690196,0.690196}%
\pgfsetstrokecolor{currentstroke}%
\pgfsetstrokeopacity{0.450000}%
\pgfsetdash{{2.220000pt}{0.960000pt}}{0.000000pt}%
\pgfpathmoveto{\pgfqpoint{0.868056in}{2.004704in}}%
\pgfpathlineto{\pgfqpoint{4.861111in}{2.004704in}}%
\pgfusepath{stroke}%
\end{pgfscope}%
\begin{pgfscope}%
\pgfsetbuttcap%
\pgfsetroundjoin%
\definecolor{currentfill}{rgb}{0.000000,0.000000,0.000000}%
\pgfsetfillcolor{currentfill}%
\pgfsetlinewidth{0.752812pt}%
\definecolor{currentstroke}{rgb}{0.000000,0.000000,0.000000}%
\pgfsetstrokecolor{currentstroke}%
\pgfsetdash{}{0pt}%
\pgfsys@defobject{currentmarker}{\pgfqpoint{-0.013889in}{0.000000in}}{\pgfqpoint{-0.000000in}{0.000000in}}{%
\pgfpathmoveto{\pgfqpoint{-0.000000in}{0.000000in}}%
\pgfpathlineto{\pgfqpoint{-0.013889in}{0.000000in}}%
\pgfusepath{stroke,fill}%
}%
\begin{pgfscope}%
\pgfsys@transformshift{0.868056in}{2.004704in}%
\pgfsys@useobject{currentmarker}{}%
\end{pgfscope}%
\end{pgfscope}%
\begin{pgfscope}%
\definecolor{textcolor}{rgb}{0.000000,0.000000,0.000000}%
\pgfsetstrokecolor{textcolor}%
\pgfsetfillcolor{textcolor}%
\pgftext[x=0.664971in, y=1.976967in, left, base]{\color{textcolor}{\ifdefined\pdftexversion\else\setmainfont{NanumMyeongjo}\rmfamily\fi\fontsize{5.000000}{6.000000}\selectfont\catcode`\^=\active\def^{\ifmmode\sp\else\^{}\fi}\catcode`\%=\active\def%{\%}14만}}%
\end{pgfscope}%
\begin{pgfscope}%
\pgfpathrectangle{\pgfqpoint{0.868056in}{0.555556in}}{\pgfqpoint{3.993056in}{1.888889in}}%
\pgfusepath{clip}%
\pgfsetbuttcap%
\pgfsetroundjoin%
\pgfsetlinewidth{0.602250pt}%
\definecolor{currentstroke}{rgb}{0.690196,0.690196,0.690196}%
\pgfsetstrokecolor{currentstroke}%
\pgfsetstrokeopacity{0.450000}%
\pgfsetdash{{2.220000pt}{0.960000pt}}{0.000000pt}%
\pgfpathmoveto{\pgfqpoint{0.868056in}{2.211725in}}%
\pgfpathlineto{\pgfqpoint{4.861111in}{2.211725in}}%
\pgfusepath{stroke}%
\end{pgfscope}%
\begin{pgfscope}%
\pgfsetbuttcap%
\pgfsetroundjoin%
\definecolor{currentfill}{rgb}{0.000000,0.000000,0.000000}%
\pgfsetfillcolor{currentfill}%
\pgfsetlinewidth{0.752812pt}%
\definecolor{currentstroke}{rgb}{0.000000,0.000000,0.000000}%
\pgfsetstrokecolor{currentstroke}%
\pgfsetdash{}{0pt}%
\pgfsys@defobject{currentmarker}{\pgfqpoint{-0.013889in}{0.000000in}}{\pgfqpoint{-0.000000in}{0.000000in}}{%
\pgfpathmoveto{\pgfqpoint{-0.000000in}{0.000000in}}%
\pgfpathlineto{\pgfqpoint{-0.013889in}{0.000000in}}%
\pgfusepath{stroke,fill}%
}%
\begin{pgfscope}%
\pgfsys@transformshift{0.868056in}{2.211725in}%
\pgfsys@useobject{currentmarker}{}%
\end{pgfscope}%
\end{pgfscope}%
\begin{pgfscope}%
\definecolor{textcolor}{rgb}{0.000000,0.000000,0.000000}%
\pgfsetstrokecolor{textcolor}%
\pgfsetfillcolor{textcolor}%
\pgftext[x=0.664971in, y=2.183988in, left, base]{\color{textcolor}{\ifdefined\pdftexversion\else\setmainfont{NanumMyeongjo}\rmfamily\fi\fontsize{5.000000}{6.000000}\selectfont\catcode`\^=\active\def^{\ifmmode\sp\else\^{}\fi}\catcode`\%=\active\def%{\%}16만}}%
\end{pgfscope}%
\begin{pgfscope}%
\pgfpathrectangle{\pgfqpoint{0.868056in}{0.555556in}}{\pgfqpoint{3.993056in}{1.888889in}}%
\pgfusepath{clip}%
\pgfsetbuttcap%
\pgfsetroundjoin%
\pgfsetlinewidth{0.602250pt}%
\definecolor{currentstroke}{rgb}{0.690196,0.690196,0.690196}%
\pgfsetstrokecolor{currentstroke}%
\pgfsetstrokeopacity{0.450000}%
\pgfsetdash{{2.220000pt}{0.960000pt}}{0.000000pt}%
\pgfpathmoveto{\pgfqpoint{0.868056in}{2.418746in}}%
\pgfpathlineto{\pgfqpoint{4.861111in}{2.418746in}}%
\pgfusepath{stroke}%
\end{pgfscope}%
\begin{pgfscope}%
\pgfsetbuttcap%
\pgfsetroundjoin%
\definecolor{currentfill}{rgb}{0.000000,0.000000,0.000000}%
\pgfsetfillcolor{currentfill}%
\pgfsetlinewidth{0.752812pt}%
\definecolor{currentstroke}{rgb}{0.000000,0.000000,0.000000}%
\pgfsetstrokecolor{currentstroke}%
\pgfsetdash{}{0pt}%
\pgfsys@defobject{currentmarker}{\pgfqpoint{-0.013889in}{0.000000in}}{\pgfqpoint{-0.000000in}{0.000000in}}{%
\pgfpathmoveto{\pgfqpoint{-0.000000in}{0.000000in}}%
\pgfpathlineto{\pgfqpoint{-0.013889in}{0.000000in}}%
\pgfusepath{stroke,fill}%
}%
\begin{pgfscope}%
\pgfsys@transformshift{0.868056in}{2.418746in}%
\pgfsys@useobject{currentmarker}{}%
\end{pgfscope}%
\end{pgfscope}%
\begin{pgfscope}%
\definecolor{textcolor}{rgb}{0.000000,0.000000,0.000000}%
\pgfsetstrokecolor{textcolor}%
\pgfsetfillcolor{textcolor}%
\pgftext[x=0.664971in, y=2.391009in, left, base]{\color{textcolor}{\ifdefined\pdftexversion\else\setmainfont{NanumMyeongjo}\rmfamily\fi\fontsize{5.000000}{6.000000}\selectfont\catcode`\^=\active\def^{\ifmmode\sp\else\^{}\fi}\catcode`\%=\active\def%{\%}18만}}%
\end{pgfscope}%
\begin{pgfscope}%
\pgfsetrectcap%
\pgfsetmiterjoin%
\pgfsetlinewidth{0.752812pt}%
\definecolor{currentstroke}{rgb}{0.000000,0.000000,0.000000}%
\pgfsetstrokecolor{currentstroke}%
\pgfsetdash{}{0pt}%
\pgfpathmoveto{\pgfqpoint{0.868056in}{0.555556in}}%
\pgfpathlineto{\pgfqpoint{0.868056in}{2.444444in}}%
\pgfusepath{stroke}%
\end{pgfscope}%
\begin{pgfscope}%
\pgfsetrectcap%
\pgfsetmiterjoin%
\pgfsetlinewidth{0.752812pt}%
\definecolor{currentstroke}{rgb}{0.000000,0.000000,0.000000}%
\pgfsetstrokecolor{currentstroke}%
\pgfsetdash{}{0pt}%
\pgfpathmoveto{\pgfqpoint{0.868056in}{0.555556in}}%
\pgfpathlineto{\pgfqpoint{4.861111in}{0.555556in}}%
\pgfusepath{stroke}%
\end{pgfscope}%
\begin{pgfscope}%
\pgfpathrectangle{\pgfqpoint{0.868056in}{0.555556in}}{\pgfqpoint{3.993056in}{1.888889in}}%
\pgfusepath{clip}%
\pgfsetbuttcap%
\pgfsetmiterjoin%
\definecolor{currentfill}{rgb}{0.227451,0.192157,0.427451}%
\pgfsetfillcolor{currentfill}%
\pgfsetlinewidth{1.003750pt}%
\definecolor{currentstroke}{rgb}{0.266667,0.266667,0.266667}%
\pgfsetstrokecolor{currentstroke}%
\pgfsetdash{}{0pt}%
\pgfpathmoveto{\pgfqpoint{1.049558in}{0.555556in}}%
\pgfpathlineto{\pgfqpoint{1.287038in}{0.555556in}}%
\pgfpathlineto{\pgfqpoint{1.287038in}{2.197058in}}%
\pgfpathlineto{\pgfqpoint{1.049558in}{2.197058in}}%
\pgfpathlineto{\pgfqpoint{1.049558in}{0.555556in}}%
\pgfpathclose%
\pgfusepath{stroke,fill}%
\end{pgfscope}%
\begin{pgfscope}%
\pgfpathrectangle{\pgfqpoint{0.868056in}{0.555556in}}{\pgfqpoint{3.993056in}{1.888889in}}%
\pgfusepath{clip}%
\pgfsetbuttcap%
\pgfsetmiterjoin%
\definecolor{currentfill}{rgb}{0.227451,0.192157,0.427451}%
\pgfsetfillcolor{currentfill}%
\pgfsetlinewidth{1.003750pt}%
\definecolor{currentstroke}{rgb}{0.266667,0.266667,0.266667}%
\pgfsetstrokecolor{currentstroke}%
\pgfsetdash{}{0pt}%
\pgfpathmoveto{\pgfqpoint{1.388815in}{0.555556in}}%
\pgfpathlineto{\pgfqpoint{1.626295in}{0.555556in}}%
\pgfpathlineto{\pgfqpoint{1.626295in}{1.788160in}}%
\pgfpathlineto{\pgfqpoint{1.388815in}{1.788160in}}%
\pgfpathlineto{\pgfqpoint{1.388815in}{0.555556in}}%
\pgfpathclose%
\pgfusepath{stroke,fill}%
\end{pgfscope}%
\begin{pgfscope}%
\pgfpathrectangle{\pgfqpoint{0.868056in}{0.555556in}}{\pgfqpoint{3.993056in}{1.888889in}}%
\pgfusepath{clip}%
\pgfsetbuttcap%
\pgfsetmiterjoin%
\definecolor{currentfill}{rgb}{0.227451,0.192157,0.427451}%
\pgfsetfillcolor{currentfill}%
\pgfsetlinewidth{1.003750pt}%
\definecolor{currentstroke}{rgb}{0.266667,0.266667,0.266667}%
\pgfsetstrokecolor{currentstroke}%
\pgfsetdash{}{0pt}%
\pgfpathmoveto{\pgfqpoint{1.728072in}{0.555556in}}%
\pgfpathlineto{\pgfqpoint{1.965552in}{0.555556in}}%
\pgfpathlineto{\pgfqpoint{1.965552in}{1.494593in}}%
\pgfpathlineto{\pgfqpoint{1.728072in}{1.494593in}}%
\pgfpathlineto{\pgfqpoint{1.728072in}{0.555556in}}%
\pgfpathclose%
\pgfusepath{stroke,fill}%
\end{pgfscope}%
\begin{pgfscope}%
\pgfpathrectangle{\pgfqpoint{0.868056in}{0.555556in}}{\pgfqpoint{3.993056in}{1.888889in}}%
\pgfusepath{clip}%
\pgfsetbuttcap%
\pgfsetmiterjoin%
\definecolor{currentfill}{rgb}{0.227451,0.192157,0.427451}%
\pgfsetfillcolor{currentfill}%
\pgfsetlinewidth{1.003750pt}%
\definecolor{currentstroke}{rgb}{0.266667,0.266667,0.266667}%
\pgfsetstrokecolor{currentstroke}%
\pgfsetdash{}{0pt}%
\pgfpathmoveto{\pgfqpoint{2.067329in}{0.555556in}}%
\pgfpathlineto{\pgfqpoint{2.304809in}{0.555556in}}%
\pgfpathlineto{\pgfqpoint{2.304809in}{1.602731in}}%
\pgfpathlineto{\pgfqpoint{2.067329in}{1.602731in}}%
\pgfpathlineto{\pgfqpoint{2.067329in}{0.555556in}}%
\pgfpathclose%
\pgfusepath{stroke,fill}%
\end{pgfscope}%
\begin{pgfscope}%
\pgfpathrectangle{\pgfqpoint{0.868056in}{0.555556in}}{\pgfqpoint{3.993056in}{1.888889in}}%
\pgfusepath{clip}%
\pgfsetbuttcap%
\pgfsetmiterjoin%
\definecolor{currentfill}{rgb}{0.227451,0.192157,0.427451}%
\pgfsetfillcolor{currentfill}%
\pgfsetlinewidth{1.003750pt}%
\definecolor{currentstroke}{rgb}{0.266667,0.266667,0.266667}%
\pgfsetstrokecolor{currentstroke}%
\pgfsetdash{}{0pt}%
\pgfpathmoveto{\pgfqpoint{2.406586in}{0.555556in}}%
\pgfpathlineto{\pgfqpoint{2.644066in}{0.555556in}}%
\pgfpathlineto{\pgfqpoint{2.644066in}{1.653617in}}%
\pgfpathlineto{\pgfqpoint{2.406586in}{1.653617in}}%
\pgfpathlineto{\pgfqpoint{2.406586in}{0.555556in}}%
\pgfpathclose%
\pgfusepath{stroke,fill}%
\end{pgfscope}%
\begin{pgfscope}%
\pgfpathrectangle{\pgfqpoint{0.868056in}{0.555556in}}{\pgfqpoint{3.993056in}{1.888889in}}%
\pgfusepath{clip}%
\pgfsetbuttcap%
\pgfsetmiterjoin%
\definecolor{currentfill}{rgb}{0.227451,0.192157,0.427451}%
\pgfsetfillcolor{currentfill}%
\pgfsetlinewidth{1.003750pt}%
\definecolor{currentstroke}{rgb}{0.266667,0.266667,0.266667}%
\pgfsetstrokecolor{currentstroke}%
\pgfsetdash{}{0pt}%
\pgfpathmoveto{\pgfqpoint{2.745843in}{0.555556in}}%
\pgfpathlineto{\pgfqpoint{2.983323in}{0.555556in}}%
\pgfpathlineto{\pgfqpoint{2.983323in}{1.851001in}}%
\pgfpathlineto{\pgfqpoint{2.745843in}{1.851001in}}%
\pgfpathlineto{\pgfqpoint{2.745843in}{0.555556in}}%
\pgfpathclose%
\pgfusepath{stroke,fill}%
\end{pgfscope}%
\begin{pgfscope}%
\pgfpathrectangle{\pgfqpoint{0.868056in}{0.555556in}}{\pgfqpoint{3.993056in}{1.888889in}}%
\pgfusepath{clip}%
\pgfsetbuttcap%
\pgfsetmiterjoin%
\definecolor{currentfill}{rgb}{0.227451,0.192157,0.427451}%
\pgfsetfillcolor{currentfill}%
\pgfsetlinewidth{1.003750pt}%
\definecolor{currentstroke}{rgb}{0.266667,0.266667,0.266667}%
\pgfsetstrokecolor{currentstroke}%
\pgfsetdash{}{0pt}%
\pgfpathmoveto{\pgfqpoint{3.085100in}{0.555556in}}%
\pgfpathlineto{\pgfqpoint{3.322580in}{0.555556in}}%
\pgfpathlineto{\pgfqpoint{3.322580in}{1.576967in}}%
\pgfpathlineto{\pgfqpoint{3.085100in}{1.576967in}}%
\pgfpathlineto{\pgfqpoint{3.085100in}{0.555556in}}%
\pgfpathclose%
\pgfusepath{stroke,fill}%
\end{pgfscope}%
\begin{pgfscope}%
\pgfpathrectangle{\pgfqpoint{0.868056in}{0.555556in}}{\pgfqpoint{3.993056in}{1.888889in}}%
\pgfusepath{clip}%
\pgfsetbuttcap%
\pgfsetmiterjoin%
\definecolor{currentfill}{rgb}{0.227451,0.192157,0.427451}%
\pgfsetfillcolor{currentfill}%
\pgfsetlinewidth{1.003750pt}%
\definecolor{currentstroke}{rgb}{0.266667,0.266667,0.266667}%
\pgfsetstrokecolor{currentstroke}%
\pgfsetdash{}{0pt}%
\pgfpathmoveto{\pgfqpoint{3.424357in}{0.555556in}}%
\pgfpathlineto{\pgfqpoint{3.661837in}{0.555556in}}%
\pgfpathlineto{\pgfqpoint{3.661837in}{1.881185in}}%
\pgfpathlineto{\pgfqpoint{3.424357in}{1.881185in}}%
\pgfpathlineto{\pgfqpoint{3.424357in}{0.555556in}}%
\pgfpathclose%
\pgfusepath{stroke,fill}%
\end{pgfscope}%
\begin{pgfscope}%
\pgfpathrectangle{\pgfqpoint{0.868056in}{0.555556in}}{\pgfqpoint{3.993056in}{1.888889in}}%
\pgfusepath{clip}%
\pgfsetbuttcap%
\pgfsetmiterjoin%
\definecolor{currentfill}{rgb}{0.227451,0.192157,0.427451}%
\pgfsetfillcolor{currentfill}%
\pgfsetlinewidth{1.003750pt}%
\definecolor{currentstroke}{rgb}{0.266667,0.266667,0.266667}%
\pgfsetstrokecolor{currentstroke}%
\pgfsetdash{}{0pt}%
\pgfpathmoveto{\pgfqpoint{3.763615in}{0.555556in}}%
\pgfpathlineto{\pgfqpoint{4.001094in}{0.555556in}}%
\pgfpathlineto{\pgfqpoint{4.001094in}{2.104602in}}%
\pgfpathlineto{\pgfqpoint{3.763615in}{2.104602in}}%
\pgfpathlineto{\pgfqpoint{3.763615in}{0.555556in}}%
\pgfpathclose%
\pgfusepath{stroke,fill}%
\end{pgfscope}%
\begin{pgfscope}%
\pgfpathrectangle{\pgfqpoint{0.868056in}{0.555556in}}{\pgfqpoint{3.993056in}{1.888889in}}%
\pgfusepath{clip}%
\pgfsetbuttcap%
\pgfsetmiterjoin%
\definecolor{currentfill}{rgb}{0.227451,0.192157,0.427451}%
\pgfsetfillcolor{currentfill}%
\pgfsetlinewidth{1.003750pt}%
\definecolor{currentstroke}{rgb}{0.266667,0.266667,0.266667}%
\pgfsetstrokecolor{currentstroke}%
\pgfsetdash{}{0pt}%
\pgfpathmoveto{\pgfqpoint{4.102872in}{0.555556in}}%
\pgfpathlineto{\pgfqpoint{4.340352in}{0.555556in}}%
\pgfpathlineto{\pgfqpoint{4.340352in}{2.185195in}}%
\pgfpathlineto{\pgfqpoint{4.102872in}{2.185195in}}%
\pgfpathlineto{\pgfqpoint{4.102872in}{0.555556in}}%
\pgfpathclose%
\pgfusepath{stroke,fill}%
\end{pgfscope}%
\begin{pgfscope}%
\pgfpathrectangle{\pgfqpoint{0.868056in}{0.555556in}}{\pgfqpoint{3.993056in}{1.888889in}}%
\pgfusepath{clip}%
\pgfsetbuttcap%
\pgfsetmiterjoin%
\definecolor{currentfill}{rgb}{0.227451,0.192157,0.427451}%
\pgfsetfillcolor{currentfill}%
\pgfsetlinewidth{1.003750pt}%
\definecolor{currentstroke}{rgb}{0.266667,0.266667,0.266667}%
\pgfsetstrokecolor{currentstroke}%
\pgfsetdash{}{0pt}%
\pgfpathmoveto{\pgfqpoint{4.442129in}{0.555556in}}%
\pgfpathlineto{\pgfqpoint{4.679609in}{0.555556in}}%
\pgfpathlineto{\pgfqpoint{4.679609in}{2.354497in}}%
\pgfpathlineto{\pgfqpoint{4.442129in}{2.354497in}}%
\pgfpathlineto{\pgfqpoint{4.442129in}{0.555556in}}%
\pgfpathclose%
\pgfusepath{stroke,fill}%
\end{pgfscope}%
\begin{pgfscope}%
\definecolor{textcolor}{rgb}{0.000000,0.000000,0.000000}%
\pgfsetstrokecolor{textcolor}%
\pgfsetfillcolor{textcolor}%
\pgftext[x=1.168298in,y=2.224836in,,bottom]{\color{textcolor}{\ifdefined\pdftexversion\else\setmainfont{NanumMyeongjo}\rmfamily\fi\fontsize{5.000000}{6.000000}\bfseries\selectfont\catcode`\^=\active\def^{\ifmmode\sp\else\^{}\fi}\catcode`\%=\active\def%{\%}158,583}}%
\end{pgfscope}%
\begin{pgfscope}%
\definecolor{textcolor}{rgb}{0.000000,0.000000,0.000000}%
\pgfsetstrokecolor{textcolor}%
\pgfsetfillcolor{textcolor}%
\pgftext[x=1.507555in,y=1.815938in,,bottom]{\color{textcolor}{\ifdefined\pdftexversion\else\setmainfont{NanumMyeongjo}\rmfamily\fi\fontsize{5.000000}{6.000000}\bfseries\selectfont\catcode`\^=\active\def^{\ifmmode\sp\else\^{}\fi}\catcode`\%=\active\def%{\%}119,080}}%
\end{pgfscope}%
\begin{pgfscope}%
\definecolor{textcolor}{rgb}{0.000000,0.000000,0.000000}%
\pgfsetstrokecolor{textcolor}%
\pgfsetfillcolor{textcolor}%
\pgftext[x=1.846812in,y=1.522371in,,bottom]{\color{textcolor}{\ifdefined\pdftexversion\else\setmainfont{NanumMyeongjo}\rmfamily\fi\fontsize{5.000000}{6.000000}\bfseries\selectfont\catcode`\^=\active\def^{\ifmmode\sp\else\^{}\fi}\catcode`\%=\active\def%{\%}90,719}}%
\end{pgfscope}%
\begin{pgfscope}%
\definecolor{textcolor}{rgb}{0.000000,0.000000,0.000000}%
\pgfsetstrokecolor{textcolor}%
\pgfsetfillcolor{textcolor}%
\pgftext[x=2.186069in,y=1.630509in,,bottom]{\color{textcolor}{\ifdefined\pdftexversion\else\setmainfont{NanumMyeongjo}\rmfamily\fi\fontsize{5.000000}{6.000000}\bfseries\selectfont\catcode`\^=\active\def^{\ifmmode\sp\else\^{}\fi}\catcode`\%=\active\def%{\%}101,166}}%
\end{pgfscope}%
\begin{pgfscope}%
\definecolor{textcolor}{rgb}{0.000000,0.000000,0.000000}%
\pgfsetstrokecolor{textcolor}%
\pgfsetfillcolor{textcolor}%
\pgftext[x=2.525326in,y=1.681394in,,bottom]{\color{textcolor}{\ifdefined\pdftexversion\else\setmainfont{NanumMyeongjo}\rmfamily\fi\fontsize{5.000000}{6.000000}\bfseries\selectfont\catcode`\^=\active\def^{\ifmmode\sp\else\^{}\fi}\catcode`\%=\active\def%{\%}106,082}}%
\end{pgfscope}%
\begin{pgfscope}%
\definecolor{textcolor}{rgb}{0.000000,0.000000,0.000000}%
\pgfsetstrokecolor{textcolor}%
\pgfsetfillcolor{textcolor}%
\pgftext[x=2.864583in,y=1.878779in,,bottom]{\color{textcolor}{\ifdefined\pdftexversion\else\setmainfont{NanumMyeongjo}\rmfamily\fi\fontsize{5.000000}{6.000000}\bfseries\selectfont\catcode`\^=\active\def^{\ifmmode\sp\else\^{}\fi}\catcode`\%=\active\def%{\%}125,151}}%
\end{pgfscope}%
\begin{pgfscope}%
\definecolor{textcolor}{rgb}{0.000000,0.000000,0.000000}%
\pgfsetstrokecolor{textcolor}%
\pgfsetfillcolor{textcolor}%
\pgftext[x=3.203840in,y=1.604745in,,bottom]{\color{textcolor}{\ifdefined\pdftexversion\else\setmainfont{NanumMyeongjo}\rmfamily\fi\fontsize{5.000000}{6.000000}\bfseries\selectfont\catcode`\^=\active\def^{\ifmmode\sp\else\^{}\fi}\catcode`\%=\active\def%{\%}98,677}}%
\end{pgfscope}%
\begin{pgfscope}%
\definecolor{textcolor}{rgb}{0.000000,0.000000,0.000000}%
\pgfsetstrokecolor{textcolor}%
\pgfsetfillcolor{textcolor}%
\pgftext[x=3.543097in,y=1.908963in,,bottom]{\color{textcolor}{\ifdefined\pdftexversion\else\setmainfont{NanumMyeongjo}\rmfamily\fi\fontsize{5.000000}{6.000000}\bfseries\selectfont\catcode`\^=\active\def^{\ifmmode\sp\else\^{}\fi}\catcode`\%=\active\def%{\%}128,067}}%
\end{pgfscope}%
\begin{pgfscope}%
\definecolor{textcolor}{rgb}{0.000000,0.000000,0.000000}%
\pgfsetstrokecolor{textcolor}%
\pgfsetfillcolor{textcolor}%
\pgftext[x=3.882355in,y=2.132380in,,bottom]{\color{textcolor}{\ifdefined\pdftexversion\else\setmainfont{NanumMyeongjo}\rmfamily\fi\fontsize{5.000000}{6.000000}\bfseries\selectfont\catcode`\^=\active\def^{\ifmmode\sp\else\^{}\fi}\catcode`\%=\active\def%{\%}149,651}}%
\end{pgfscope}%
\begin{pgfscope}%
\definecolor{textcolor}{rgb}{0.000000,0.000000,0.000000}%
\pgfsetstrokecolor{textcolor}%
\pgfsetfillcolor{textcolor}%
\pgftext[x=4.221612in,y=2.212973in,,bottom]{\color{textcolor}{\ifdefined\pdftexversion\else\setmainfont{NanumMyeongjo}\rmfamily\fi\fontsize{5.000000}{6.000000}\bfseries\selectfont\catcode`\^=\active\def^{\ifmmode\sp\else\^{}\fi}\catcode`\%=\active\def%{\%}157,437}}%
\end{pgfscope}%
\begin{pgfscope}%
\definecolor{textcolor}{rgb}{0.000000,0.000000,0.000000}%
\pgfsetstrokecolor{textcolor}%
\pgfsetfillcolor{textcolor}%
\pgftext[x=4.560869in,y=2.382275in,,bottom]{\color{textcolor}{\ifdefined\pdftexversion\else\setmainfont{NanumMyeongjo}\rmfamily\fi\fontsize{5.000000}{6.000000}\bfseries\selectfont\catcode`\^=\active\def^{\ifmmode\sp\else\^{}\fi}\catcode`\%=\active\def%{\%}173,793}}%
\end{pgfscope}%
\begin{pgfscope}%
\definecolor{textcolor}{rgb}{0.333333,0.333333,0.333333}%
\pgfsetstrokecolor{textcolor}%
\pgfsetfillcolor{textcolor}%
\pgftext[x=1.736111in,y=0.333333in,,top]{\color{textcolor}{\ifdefined\pdftexversion\else\setmainfont{NanumMyeongjo}\rmfamily\fi\fontsize{5.000000}{6.000000}\selectfont\catcode`\^=\active\def^{\ifmmode\sp\else\^{}\fi}\catcode`\%=\active\def%{\%}출처: 국가농식품통계서비스(KASS) 자료 기반 저자 작성}}%
\end{pgfscope}%
\begin{pgfscope}%
\definecolor{textcolor}{rgb}{0.333333,0.333333,0.333333}%
\pgfsetstrokecolor{textcolor}%
\pgfsetfillcolor{textcolor}%
\pgftext[x=4.513889in,y=2.583333in,,top]{\color{textcolor}{\ifdefined\pdftexversion\else\setmainfont{NanumMyeongjo}\rmfamily\fi\fontsize{5.000000}{6.000000}\selectfont\catcode`\^=\active\def^{\ifmmode\sp\else\^{}\fi}\catcode`\%=\active\def%{\%}(단위: 톤)}}%
\end{pgfscope}%
\end{pgfpicture}%
\makeatother%
\endgroup%
}
\end{center}
}

\slide
{\maintitle}
{\chapterone}
{국내 콩 생산량}{
\begin{center}
    \hspace*{-55pt}\resizebox{1.5\textwidth}{!}{%% Creator: Matplotlib, PGF backend
%%
%% To include the figure in your LaTeX document, write
%%   \input{<filename>.pgf}
%%
%% Make sure the required packages are loaded in your preamble
%%   \usepackage{pgf}
%%
%% Also ensure that all the required font packages are loaded; for instance,
%% the lmodern package is sometimes necessary when using math font.
%%   \usepackage{lmodern}
%%
%% Figures using additional raster images can only be included by \input if
%% they are in the same directory as the main LaTeX file. For loading figures
%% from other directories you can use the `import` package
%%   \usepackage{import}
%%
%% and then include the figures with
%%   \import{<path to file>}{<filename>.pgf}
%%
%% Matplotlib used the following preamble
%%   \def\mathdefault#1{#1}
%%   \everymath=\expandafter{\the\everymath\displaystyle}
%%   \IfFileExists{scrextend.sty}{
%%     \usepackage[fontsize=5.000000pt]{scrextend}
%%   }{
%%     \renewcommand{\normalsize}{\fontsize{5.000000}{6.000000}\selectfont}
%%     \normalsize
%%   }
%%   
%%   \ifdefined\pdftexversion\else  % non-pdftex case.
%%     \usepackage{fontspec}
%%     \setmainfont{DejaVuSerif.ttf}[Path=\detokenize{/home/user/.cache/pypoetry/virtualenvs/graph-KASAOWVd-py3.12/lib/python3.12/site-packages/matplotlib/mpl-data/fonts/ttf/}]
%%     \setsansfont{DejaVuSans.ttf}[Path=\detokenize{/home/user/.cache/pypoetry/virtualenvs/graph-KASAOWVd-py3.12/lib/python3.12/site-packages/matplotlib/mpl-data/fonts/ttf/}]
%%     \setmonofont{DejaVuSansMono.ttf}[Path=\detokenize{/home/user/.cache/pypoetry/virtualenvs/graph-KASAOWVd-py3.12/lib/python3.12/site-packages/matplotlib/mpl-data/fonts/ttf/}]
%%   \fi
%%   \makeatletter\@ifpackageloaded{underscore}{}{\usepackage[strings]{underscore}}\makeatother
%%
\begingroup%
\makeatletter%
\begin{pgfpicture}%
\pgfpathrectangle{\pgfpointorigin}{\pgfqpoint{6.944444in}{2.777778in}}%
\pgfusepath{use as bounding box, clip}%
\begin{pgfscope}%
\pgfsetbuttcap%
\pgfsetmiterjoin%
\definecolor{currentfill}{rgb}{1.000000,1.000000,1.000000}%
\pgfsetfillcolor{currentfill}%
\pgfsetlinewidth{0.000000pt}%
\definecolor{currentstroke}{rgb}{1.000000,1.000000,1.000000}%
\pgfsetstrokecolor{currentstroke}%
\pgfsetdash{}{0pt}%
\pgfpathmoveto{\pgfqpoint{0.000000in}{0.000000in}}%
\pgfpathlineto{\pgfqpoint{6.944444in}{0.000000in}}%
\pgfpathlineto{\pgfqpoint{6.944444in}{2.777778in}}%
\pgfpathlineto{\pgfqpoint{0.000000in}{2.777778in}}%
\pgfpathlineto{\pgfqpoint{0.000000in}{0.000000in}}%
\pgfpathclose%
\pgfusepath{fill}%
\end{pgfscope}%
\begin{pgfscope}%
\pgfsetbuttcap%
\pgfsetmiterjoin%
\definecolor{currentfill}{rgb}{1.000000,1.000000,1.000000}%
\pgfsetfillcolor{currentfill}%
\pgfsetlinewidth{0.000000pt}%
\definecolor{currentstroke}{rgb}{0.000000,0.000000,0.000000}%
\pgfsetstrokecolor{currentstroke}%
\pgfsetstrokeopacity{0.000000}%
\pgfsetdash{}{0pt}%
\pgfpathmoveto{\pgfqpoint{0.868056in}{0.555556in}}%
\pgfpathlineto{\pgfqpoint{4.861111in}{0.555556in}}%
\pgfpathlineto{\pgfqpoint{4.861111in}{2.444444in}}%
\pgfpathlineto{\pgfqpoint{0.868056in}{2.444444in}}%
\pgfpathlineto{\pgfqpoint{0.868056in}{0.555556in}}%
\pgfpathclose%
\pgfusepath{fill}%
\end{pgfscope}%
\begin{pgfscope}%
\pgfsetbuttcap%
\pgfsetroundjoin%
\definecolor{currentfill}{rgb}{0.000000,0.000000,0.000000}%
\pgfsetfillcolor{currentfill}%
\pgfsetlinewidth{0.752812pt}%
\definecolor{currentstroke}{rgb}{0.000000,0.000000,0.000000}%
\pgfsetstrokecolor{currentstroke}%
\pgfsetdash{}{0pt}%
\pgfsys@defobject{currentmarker}{\pgfqpoint{0.000000in}{-0.013889in}}{\pgfqpoint{0.000000in}{0.000000in}}{%
\pgfpathmoveto{\pgfqpoint{0.000000in}{0.000000in}}%
\pgfpathlineto{\pgfqpoint{0.000000in}{-0.013889in}}%
\pgfusepath{stroke,fill}%
}%
\begin{pgfscope}%
\pgfsys@transformshift{1.168298in}{0.555556in}%
\pgfsys@useobject{currentmarker}{}%
\end{pgfscope}%
\end{pgfscope}%
\begin{pgfscope}%
\definecolor{textcolor}{rgb}{0.000000,0.000000,0.000000}%
\pgfsetstrokecolor{textcolor}%
\pgfsetfillcolor{textcolor}%
\pgftext[x=1.168298in,y=0.493056in,,top]{\color{textcolor}{\ifdefined\pdftexversion\else\setmainfont{NanumMyeongjo}\rmfamily\fi\fontsize{5.000000}{6.000000}\selectfont\catcode`\^=\active\def^{\ifmmode\sp\else\^{}\fi}\catcode`\%=\active\def%{\%}2014}}%
\end{pgfscope}%
\begin{pgfscope}%
\pgfsetbuttcap%
\pgfsetroundjoin%
\definecolor{currentfill}{rgb}{0.000000,0.000000,0.000000}%
\pgfsetfillcolor{currentfill}%
\pgfsetlinewidth{0.752812pt}%
\definecolor{currentstroke}{rgb}{0.000000,0.000000,0.000000}%
\pgfsetstrokecolor{currentstroke}%
\pgfsetdash{}{0pt}%
\pgfsys@defobject{currentmarker}{\pgfqpoint{0.000000in}{-0.013889in}}{\pgfqpoint{0.000000in}{0.000000in}}{%
\pgfpathmoveto{\pgfqpoint{0.000000in}{0.000000in}}%
\pgfpathlineto{\pgfqpoint{0.000000in}{-0.013889in}}%
\pgfusepath{stroke,fill}%
}%
\begin{pgfscope}%
\pgfsys@transformshift{1.507555in}{0.555556in}%
\pgfsys@useobject{currentmarker}{}%
\end{pgfscope}%
\end{pgfscope}%
\begin{pgfscope}%
\definecolor{textcolor}{rgb}{0.000000,0.000000,0.000000}%
\pgfsetstrokecolor{textcolor}%
\pgfsetfillcolor{textcolor}%
\pgftext[x=1.507555in,y=0.493056in,,top]{\color{textcolor}{\ifdefined\pdftexversion\else\setmainfont{NanumMyeongjo}\rmfamily\fi\fontsize{5.000000}{6.000000}\selectfont\catcode`\^=\active\def^{\ifmmode\sp\else\^{}\fi}\catcode`\%=\active\def%{\%}2015}}%
\end{pgfscope}%
\begin{pgfscope}%
\pgfsetbuttcap%
\pgfsetroundjoin%
\definecolor{currentfill}{rgb}{0.000000,0.000000,0.000000}%
\pgfsetfillcolor{currentfill}%
\pgfsetlinewidth{0.752812pt}%
\definecolor{currentstroke}{rgb}{0.000000,0.000000,0.000000}%
\pgfsetstrokecolor{currentstroke}%
\pgfsetdash{}{0pt}%
\pgfsys@defobject{currentmarker}{\pgfqpoint{0.000000in}{-0.013889in}}{\pgfqpoint{0.000000in}{0.000000in}}{%
\pgfpathmoveto{\pgfqpoint{0.000000in}{0.000000in}}%
\pgfpathlineto{\pgfqpoint{0.000000in}{-0.013889in}}%
\pgfusepath{stroke,fill}%
}%
\begin{pgfscope}%
\pgfsys@transformshift{1.846812in}{0.555556in}%
\pgfsys@useobject{currentmarker}{}%
\end{pgfscope}%
\end{pgfscope}%
\begin{pgfscope}%
\definecolor{textcolor}{rgb}{0.000000,0.000000,0.000000}%
\pgfsetstrokecolor{textcolor}%
\pgfsetfillcolor{textcolor}%
\pgftext[x=1.846812in,y=0.493056in,,top]{\color{textcolor}{\ifdefined\pdftexversion\else\setmainfont{NanumMyeongjo}\rmfamily\fi\fontsize{5.000000}{6.000000}\selectfont\catcode`\^=\active\def^{\ifmmode\sp\else\^{}\fi}\catcode`\%=\active\def%{\%}2016}}%
\end{pgfscope}%
\begin{pgfscope}%
\pgfsetbuttcap%
\pgfsetroundjoin%
\definecolor{currentfill}{rgb}{0.000000,0.000000,0.000000}%
\pgfsetfillcolor{currentfill}%
\pgfsetlinewidth{0.752812pt}%
\definecolor{currentstroke}{rgb}{0.000000,0.000000,0.000000}%
\pgfsetstrokecolor{currentstroke}%
\pgfsetdash{}{0pt}%
\pgfsys@defobject{currentmarker}{\pgfqpoint{0.000000in}{-0.013889in}}{\pgfqpoint{0.000000in}{0.000000in}}{%
\pgfpathmoveto{\pgfqpoint{0.000000in}{0.000000in}}%
\pgfpathlineto{\pgfqpoint{0.000000in}{-0.013889in}}%
\pgfusepath{stroke,fill}%
}%
\begin{pgfscope}%
\pgfsys@transformshift{2.186069in}{0.555556in}%
\pgfsys@useobject{currentmarker}{}%
\end{pgfscope}%
\end{pgfscope}%
\begin{pgfscope}%
\definecolor{textcolor}{rgb}{0.000000,0.000000,0.000000}%
\pgfsetstrokecolor{textcolor}%
\pgfsetfillcolor{textcolor}%
\pgftext[x=2.186069in,y=0.493056in,,top]{\color{textcolor}{\ifdefined\pdftexversion\else\setmainfont{NanumMyeongjo}\rmfamily\fi\fontsize{5.000000}{6.000000}\selectfont\catcode`\^=\active\def^{\ifmmode\sp\else\^{}\fi}\catcode`\%=\active\def%{\%}2017}}%
\end{pgfscope}%
\begin{pgfscope}%
\pgfsetbuttcap%
\pgfsetroundjoin%
\definecolor{currentfill}{rgb}{0.000000,0.000000,0.000000}%
\pgfsetfillcolor{currentfill}%
\pgfsetlinewidth{0.752812pt}%
\definecolor{currentstroke}{rgb}{0.000000,0.000000,0.000000}%
\pgfsetstrokecolor{currentstroke}%
\pgfsetdash{}{0pt}%
\pgfsys@defobject{currentmarker}{\pgfqpoint{0.000000in}{-0.013889in}}{\pgfqpoint{0.000000in}{0.000000in}}{%
\pgfpathmoveto{\pgfqpoint{0.000000in}{0.000000in}}%
\pgfpathlineto{\pgfqpoint{0.000000in}{-0.013889in}}%
\pgfusepath{stroke,fill}%
}%
\begin{pgfscope}%
\pgfsys@transformshift{2.525326in}{0.555556in}%
\pgfsys@useobject{currentmarker}{}%
\end{pgfscope}%
\end{pgfscope}%
\begin{pgfscope}%
\definecolor{textcolor}{rgb}{0.000000,0.000000,0.000000}%
\pgfsetstrokecolor{textcolor}%
\pgfsetfillcolor{textcolor}%
\pgftext[x=2.525326in,y=0.493056in,,top]{\color{textcolor}{\ifdefined\pdftexversion\else\setmainfont{NanumMyeongjo}\rmfamily\fi\fontsize{5.000000}{6.000000}\selectfont\catcode`\^=\active\def^{\ifmmode\sp\else\^{}\fi}\catcode`\%=\active\def%{\%}2018}}%
\end{pgfscope}%
\begin{pgfscope}%
\pgfsetbuttcap%
\pgfsetroundjoin%
\definecolor{currentfill}{rgb}{0.000000,0.000000,0.000000}%
\pgfsetfillcolor{currentfill}%
\pgfsetlinewidth{0.752812pt}%
\definecolor{currentstroke}{rgb}{0.000000,0.000000,0.000000}%
\pgfsetstrokecolor{currentstroke}%
\pgfsetdash{}{0pt}%
\pgfsys@defobject{currentmarker}{\pgfqpoint{0.000000in}{-0.013889in}}{\pgfqpoint{0.000000in}{0.000000in}}{%
\pgfpathmoveto{\pgfqpoint{0.000000in}{0.000000in}}%
\pgfpathlineto{\pgfqpoint{0.000000in}{-0.013889in}}%
\pgfusepath{stroke,fill}%
}%
\begin{pgfscope}%
\pgfsys@transformshift{2.864583in}{0.555556in}%
\pgfsys@useobject{currentmarker}{}%
\end{pgfscope}%
\end{pgfscope}%
\begin{pgfscope}%
\definecolor{textcolor}{rgb}{0.000000,0.000000,0.000000}%
\pgfsetstrokecolor{textcolor}%
\pgfsetfillcolor{textcolor}%
\pgftext[x=2.864583in,y=0.493056in,,top]{\color{textcolor}{\ifdefined\pdftexversion\else\setmainfont{NanumMyeongjo}\rmfamily\fi\fontsize{5.000000}{6.000000}\selectfont\catcode`\^=\active\def^{\ifmmode\sp\else\^{}\fi}\catcode`\%=\active\def%{\%}2019}}%
\end{pgfscope}%
\begin{pgfscope}%
\pgfsetbuttcap%
\pgfsetroundjoin%
\definecolor{currentfill}{rgb}{0.000000,0.000000,0.000000}%
\pgfsetfillcolor{currentfill}%
\pgfsetlinewidth{0.752812pt}%
\definecolor{currentstroke}{rgb}{0.000000,0.000000,0.000000}%
\pgfsetstrokecolor{currentstroke}%
\pgfsetdash{}{0pt}%
\pgfsys@defobject{currentmarker}{\pgfqpoint{0.000000in}{-0.013889in}}{\pgfqpoint{0.000000in}{0.000000in}}{%
\pgfpathmoveto{\pgfqpoint{0.000000in}{0.000000in}}%
\pgfpathlineto{\pgfqpoint{0.000000in}{-0.013889in}}%
\pgfusepath{stroke,fill}%
}%
\begin{pgfscope}%
\pgfsys@transformshift{3.203840in}{0.555556in}%
\pgfsys@useobject{currentmarker}{}%
\end{pgfscope}%
\end{pgfscope}%
\begin{pgfscope}%
\definecolor{textcolor}{rgb}{0.000000,0.000000,0.000000}%
\pgfsetstrokecolor{textcolor}%
\pgfsetfillcolor{textcolor}%
\pgftext[x=3.203840in,y=0.493056in,,top]{\color{textcolor}{\ifdefined\pdftexversion\else\setmainfont{NanumMyeongjo}\rmfamily\fi\fontsize{5.000000}{6.000000}\selectfont\catcode`\^=\active\def^{\ifmmode\sp\else\^{}\fi}\catcode`\%=\active\def%{\%}2020}}%
\end{pgfscope}%
\begin{pgfscope}%
\pgfsetbuttcap%
\pgfsetroundjoin%
\definecolor{currentfill}{rgb}{0.000000,0.000000,0.000000}%
\pgfsetfillcolor{currentfill}%
\pgfsetlinewidth{0.752812pt}%
\definecolor{currentstroke}{rgb}{0.000000,0.000000,0.000000}%
\pgfsetstrokecolor{currentstroke}%
\pgfsetdash{}{0pt}%
\pgfsys@defobject{currentmarker}{\pgfqpoint{0.000000in}{-0.013889in}}{\pgfqpoint{0.000000in}{0.000000in}}{%
\pgfpathmoveto{\pgfqpoint{0.000000in}{0.000000in}}%
\pgfpathlineto{\pgfqpoint{0.000000in}{-0.013889in}}%
\pgfusepath{stroke,fill}%
}%
\begin{pgfscope}%
\pgfsys@transformshift{3.543097in}{0.555556in}%
\pgfsys@useobject{currentmarker}{}%
\end{pgfscope}%
\end{pgfscope}%
\begin{pgfscope}%
\definecolor{textcolor}{rgb}{0.000000,0.000000,0.000000}%
\pgfsetstrokecolor{textcolor}%
\pgfsetfillcolor{textcolor}%
\pgftext[x=3.543097in,y=0.493056in,,top]{\color{textcolor}{\ifdefined\pdftexversion\else\setmainfont{NanumMyeongjo}\rmfamily\fi\fontsize{5.000000}{6.000000}\selectfont\catcode`\^=\active\def^{\ifmmode\sp\else\^{}\fi}\catcode`\%=\active\def%{\%}2021}}%
\end{pgfscope}%
\begin{pgfscope}%
\pgfsetbuttcap%
\pgfsetroundjoin%
\definecolor{currentfill}{rgb}{0.000000,0.000000,0.000000}%
\pgfsetfillcolor{currentfill}%
\pgfsetlinewidth{0.752812pt}%
\definecolor{currentstroke}{rgb}{0.000000,0.000000,0.000000}%
\pgfsetstrokecolor{currentstroke}%
\pgfsetdash{}{0pt}%
\pgfsys@defobject{currentmarker}{\pgfqpoint{0.000000in}{-0.013889in}}{\pgfqpoint{0.000000in}{0.000000in}}{%
\pgfpathmoveto{\pgfqpoint{0.000000in}{0.000000in}}%
\pgfpathlineto{\pgfqpoint{0.000000in}{-0.013889in}}%
\pgfusepath{stroke,fill}%
}%
\begin{pgfscope}%
\pgfsys@transformshift{3.882355in}{0.555556in}%
\pgfsys@useobject{currentmarker}{}%
\end{pgfscope}%
\end{pgfscope}%
\begin{pgfscope}%
\definecolor{textcolor}{rgb}{0.000000,0.000000,0.000000}%
\pgfsetstrokecolor{textcolor}%
\pgfsetfillcolor{textcolor}%
\pgftext[x=3.882355in,y=0.493056in,,top]{\color{textcolor}{\ifdefined\pdftexversion\else\setmainfont{NanumMyeongjo}\rmfamily\fi\fontsize{5.000000}{6.000000}\selectfont\catcode`\^=\active\def^{\ifmmode\sp\else\^{}\fi}\catcode`\%=\active\def%{\%}2022}}%
\end{pgfscope}%
\begin{pgfscope}%
\pgfsetbuttcap%
\pgfsetroundjoin%
\definecolor{currentfill}{rgb}{0.000000,0.000000,0.000000}%
\pgfsetfillcolor{currentfill}%
\pgfsetlinewidth{0.752812pt}%
\definecolor{currentstroke}{rgb}{0.000000,0.000000,0.000000}%
\pgfsetstrokecolor{currentstroke}%
\pgfsetdash{}{0pt}%
\pgfsys@defobject{currentmarker}{\pgfqpoint{0.000000in}{-0.013889in}}{\pgfqpoint{0.000000in}{0.000000in}}{%
\pgfpathmoveto{\pgfqpoint{0.000000in}{0.000000in}}%
\pgfpathlineto{\pgfqpoint{0.000000in}{-0.013889in}}%
\pgfusepath{stroke,fill}%
}%
\begin{pgfscope}%
\pgfsys@transformshift{4.221612in}{0.555556in}%
\pgfsys@useobject{currentmarker}{}%
\end{pgfscope}%
\end{pgfscope}%
\begin{pgfscope}%
\definecolor{textcolor}{rgb}{0.000000,0.000000,0.000000}%
\pgfsetstrokecolor{textcolor}%
\pgfsetfillcolor{textcolor}%
\pgftext[x=4.221612in,y=0.493056in,,top]{\color{textcolor}{\ifdefined\pdftexversion\else\setmainfont{NanumMyeongjo}\rmfamily\fi\fontsize{5.000000}{6.000000}\selectfont\catcode`\^=\active\def^{\ifmmode\sp\else\^{}\fi}\catcode`\%=\active\def%{\%}2023}}%
\end{pgfscope}%
\begin{pgfscope}%
\pgfsetbuttcap%
\pgfsetroundjoin%
\definecolor{currentfill}{rgb}{0.000000,0.000000,0.000000}%
\pgfsetfillcolor{currentfill}%
\pgfsetlinewidth{0.752812pt}%
\definecolor{currentstroke}{rgb}{0.000000,0.000000,0.000000}%
\pgfsetstrokecolor{currentstroke}%
\pgfsetdash{}{0pt}%
\pgfsys@defobject{currentmarker}{\pgfqpoint{0.000000in}{-0.013889in}}{\pgfqpoint{0.000000in}{0.000000in}}{%
\pgfpathmoveto{\pgfqpoint{0.000000in}{0.000000in}}%
\pgfpathlineto{\pgfqpoint{0.000000in}{-0.013889in}}%
\pgfusepath{stroke,fill}%
}%
\begin{pgfscope}%
\pgfsys@transformshift{4.560869in}{0.555556in}%
\pgfsys@useobject{currentmarker}{}%
\end{pgfscope}%
\end{pgfscope}%
\begin{pgfscope}%
\definecolor{textcolor}{rgb}{0.000000,0.000000,0.000000}%
\pgfsetstrokecolor{textcolor}%
\pgfsetfillcolor{textcolor}%
\pgftext[x=4.560869in,y=0.493056in,,top]{\color{textcolor}{\ifdefined\pdftexversion\else\setmainfont{NanumMyeongjo}\rmfamily\fi\fontsize{5.000000}{6.000000}\selectfont\catcode`\^=\active\def^{\ifmmode\sp\else\^{}\fi}\catcode`\%=\active\def%{\%}2024}}%
\end{pgfscope}%
\begin{pgfscope}%
\pgfpathrectangle{\pgfqpoint{0.868056in}{0.555556in}}{\pgfqpoint{3.993056in}{1.888889in}}%
\pgfusepath{clip}%
\pgfsetbuttcap%
\pgfsetroundjoin%
\pgfsetlinewidth{0.602250pt}%
\definecolor{currentstroke}{rgb}{0.690196,0.690196,0.690196}%
\pgfsetstrokecolor{currentstroke}%
\pgfsetstrokeopacity{0.400000}%
\pgfsetdash{{2.220000pt}{0.960000pt}}{0.000000pt}%
\pgfpathmoveto{\pgfqpoint{0.868056in}{0.555556in}}%
\pgfpathlineto{\pgfqpoint{4.861111in}{0.555556in}}%
\pgfusepath{stroke}%
\end{pgfscope}%
\begin{pgfscope}%
\pgfsetbuttcap%
\pgfsetroundjoin%
\definecolor{currentfill}{rgb}{0.000000,0.000000,0.000000}%
\pgfsetfillcolor{currentfill}%
\pgfsetlinewidth{0.752812pt}%
\definecolor{currentstroke}{rgb}{0.000000,0.000000,0.000000}%
\pgfsetstrokecolor{currentstroke}%
\pgfsetdash{}{0pt}%
\pgfsys@defobject{currentmarker}{\pgfqpoint{-0.013889in}{0.000000in}}{\pgfqpoint{-0.000000in}{0.000000in}}{%
\pgfpathmoveto{\pgfqpoint{-0.000000in}{0.000000in}}%
\pgfpathlineto{\pgfqpoint{-0.013889in}{0.000000in}}%
\pgfusepath{stroke,fill}%
}%
\begin{pgfscope}%
\pgfsys@transformshift{0.868056in}{0.555556in}%
\pgfsys@useobject{currentmarker}{}%
\end{pgfscope}%
\end{pgfscope}%
\begin{pgfscope}%
\definecolor{textcolor}{rgb}{0.000000,0.000000,0.000000}%
\pgfsetstrokecolor{textcolor}%
\pgfsetfillcolor{textcolor}%
\pgftext[x=0.768256in, y=0.527818in, left, base]{\color{textcolor}{\ifdefined\pdftexversion\else\setmainfont{NanumMyeongjo}\rmfamily\fi\fontsize{5.000000}{6.000000}\selectfont\catcode`\^=\active\def^{\ifmmode\sp\else\^{}\fi}\catcode`\%=\active\def%{\%}0}}%
\end{pgfscope}%
\begin{pgfscope}%
\pgfpathrectangle{\pgfqpoint{0.868056in}{0.555556in}}{\pgfqpoint{3.993056in}{1.888889in}}%
\pgfusepath{clip}%
\pgfsetbuttcap%
\pgfsetroundjoin%
\pgfsetlinewidth{0.602250pt}%
\definecolor{currentstroke}{rgb}{0.690196,0.690196,0.690196}%
\pgfsetstrokecolor{currentstroke}%
\pgfsetstrokeopacity{0.400000}%
\pgfsetdash{{2.220000pt}{0.960000pt}}{0.000000pt}%
\pgfpathmoveto{\pgfqpoint{0.868056in}{0.762577in}}%
\pgfpathlineto{\pgfqpoint{4.861111in}{0.762577in}}%
\pgfusepath{stroke}%
\end{pgfscope}%
\begin{pgfscope}%
\pgfsetbuttcap%
\pgfsetroundjoin%
\definecolor{currentfill}{rgb}{0.000000,0.000000,0.000000}%
\pgfsetfillcolor{currentfill}%
\pgfsetlinewidth{0.752812pt}%
\definecolor{currentstroke}{rgb}{0.000000,0.000000,0.000000}%
\pgfsetstrokecolor{currentstroke}%
\pgfsetdash{}{0pt}%
\pgfsys@defobject{currentmarker}{\pgfqpoint{-0.013889in}{0.000000in}}{\pgfqpoint{-0.000000in}{0.000000in}}{%
\pgfpathmoveto{\pgfqpoint{-0.000000in}{0.000000in}}%
\pgfpathlineto{\pgfqpoint{-0.013889in}{0.000000in}}%
\pgfusepath{stroke,fill}%
}%
\begin{pgfscope}%
\pgfsys@transformshift{0.868056in}{0.762577in}%
\pgfsys@useobject{currentmarker}{}%
\end{pgfscope}%
\end{pgfscope}%
\begin{pgfscope}%
\definecolor{textcolor}{rgb}{0.000000,0.000000,0.000000}%
\pgfsetstrokecolor{textcolor}%
\pgfsetfillcolor{textcolor}%
\pgftext[x=0.702271in, y=0.734840in, left, base]{\color{textcolor}{\ifdefined\pdftexversion\else\setmainfont{NanumMyeongjo}\rmfamily\fi\fontsize{5.000000}{6.000000}\selectfont\catcode`\^=\active\def^{\ifmmode\sp\else\^{}\fi}\catcode`\%=\active\def%{\%}2만}}%
\end{pgfscope}%
\begin{pgfscope}%
\pgfpathrectangle{\pgfqpoint{0.868056in}{0.555556in}}{\pgfqpoint{3.993056in}{1.888889in}}%
\pgfusepath{clip}%
\pgfsetbuttcap%
\pgfsetroundjoin%
\pgfsetlinewidth{0.602250pt}%
\definecolor{currentstroke}{rgb}{0.690196,0.690196,0.690196}%
\pgfsetstrokecolor{currentstroke}%
\pgfsetstrokeopacity{0.400000}%
\pgfsetdash{{2.220000pt}{0.960000pt}}{0.000000pt}%
\pgfpathmoveto{\pgfqpoint{0.868056in}{0.969598in}}%
\pgfpathlineto{\pgfqpoint{4.861111in}{0.969598in}}%
\pgfusepath{stroke}%
\end{pgfscope}%
\begin{pgfscope}%
\pgfsetbuttcap%
\pgfsetroundjoin%
\definecolor{currentfill}{rgb}{0.000000,0.000000,0.000000}%
\pgfsetfillcolor{currentfill}%
\pgfsetlinewidth{0.752812pt}%
\definecolor{currentstroke}{rgb}{0.000000,0.000000,0.000000}%
\pgfsetstrokecolor{currentstroke}%
\pgfsetdash{}{0pt}%
\pgfsys@defobject{currentmarker}{\pgfqpoint{-0.013889in}{0.000000in}}{\pgfqpoint{-0.000000in}{0.000000in}}{%
\pgfpathmoveto{\pgfqpoint{-0.000000in}{0.000000in}}%
\pgfpathlineto{\pgfqpoint{-0.013889in}{0.000000in}}%
\pgfusepath{stroke,fill}%
}%
\begin{pgfscope}%
\pgfsys@transformshift{0.868056in}{0.969598in}%
\pgfsys@useobject{currentmarker}{}%
\end{pgfscope}%
\end{pgfscope}%
\begin{pgfscope}%
\definecolor{textcolor}{rgb}{0.000000,0.000000,0.000000}%
\pgfsetstrokecolor{textcolor}%
\pgfsetfillcolor{textcolor}%
\pgftext[x=0.702271in, y=0.941861in, left, base]{\color{textcolor}{\ifdefined\pdftexversion\else\setmainfont{NanumMyeongjo}\rmfamily\fi\fontsize{5.000000}{6.000000}\selectfont\catcode`\^=\active\def^{\ifmmode\sp\else\^{}\fi}\catcode`\%=\active\def%{\%}4만}}%
\end{pgfscope}%
\begin{pgfscope}%
\pgfpathrectangle{\pgfqpoint{0.868056in}{0.555556in}}{\pgfqpoint{3.993056in}{1.888889in}}%
\pgfusepath{clip}%
\pgfsetbuttcap%
\pgfsetroundjoin%
\pgfsetlinewidth{0.602250pt}%
\definecolor{currentstroke}{rgb}{0.690196,0.690196,0.690196}%
\pgfsetstrokecolor{currentstroke}%
\pgfsetstrokeopacity{0.400000}%
\pgfsetdash{{2.220000pt}{0.960000pt}}{0.000000pt}%
\pgfpathmoveto{\pgfqpoint{0.868056in}{1.176619in}}%
\pgfpathlineto{\pgfqpoint{4.861111in}{1.176619in}}%
\pgfusepath{stroke}%
\end{pgfscope}%
\begin{pgfscope}%
\pgfsetbuttcap%
\pgfsetroundjoin%
\definecolor{currentfill}{rgb}{0.000000,0.000000,0.000000}%
\pgfsetfillcolor{currentfill}%
\pgfsetlinewidth{0.752812pt}%
\definecolor{currentstroke}{rgb}{0.000000,0.000000,0.000000}%
\pgfsetstrokecolor{currentstroke}%
\pgfsetdash{}{0pt}%
\pgfsys@defobject{currentmarker}{\pgfqpoint{-0.013889in}{0.000000in}}{\pgfqpoint{-0.000000in}{0.000000in}}{%
\pgfpathmoveto{\pgfqpoint{-0.000000in}{0.000000in}}%
\pgfpathlineto{\pgfqpoint{-0.013889in}{0.000000in}}%
\pgfusepath{stroke,fill}%
}%
\begin{pgfscope}%
\pgfsys@transformshift{0.868056in}{1.176619in}%
\pgfsys@useobject{currentmarker}{}%
\end{pgfscope}%
\end{pgfscope}%
\begin{pgfscope}%
\definecolor{textcolor}{rgb}{0.000000,0.000000,0.000000}%
\pgfsetstrokecolor{textcolor}%
\pgfsetfillcolor{textcolor}%
\pgftext[x=0.702271in, y=1.148882in, left, base]{\color{textcolor}{\ifdefined\pdftexversion\else\setmainfont{NanumMyeongjo}\rmfamily\fi\fontsize{5.000000}{6.000000}\selectfont\catcode`\^=\active\def^{\ifmmode\sp\else\^{}\fi}\catcode`\%=\active\def%{\%}6만}}%
\end{pgfscope}%
\begin{pgfscope}%
\pgfpathrectangle{\pgfqpoint{0.868056in}{0.555556in}}{\pgfqpoint{3.993056in}{1.888889in}}%
\pgfusepath{clip}%
\pgfsetbuttcap%
\pgfsetroundjoin%
\pgfsetlinewidth{0.602250pt}%
\definecolor{currentstroke}{rgb}{0.690196,0.690196,0.690196}%
\pgfsetstrokecolor{currentstroke}%
\pgfsetstrokeopacity{0.400000}%
\pgfsetdash{{2.220000pt}{0.960000pt}}{0.000000pt}%
\pgfpathmoveto{\pgfqpoint{0.868056in}{1.383640in}}%
\pgfpathlineto{\pgfqpoint{4.861111in}{1.383640in}}%
\pgfusepath{stroke}%
\end{pgfscope}%
\begin{pgfscope}%
\pgfsetbuttcap%
\pgfsetroundjoin%
\definecolor{currentfill}{rgb}{0.000000,0.000000,0.000000}%
\pgfsetfillcolor{currentfill}%
\pgfsetlinewidth{0.752812pt}%
\definecolor{currentstroke}{rgb}{0.000000,0.000000,0.000000}%
\pgfsetstrokecolor{currentstroke}%
\pgfsetdash{}{0pt}%
\pgfsys@defobject{currentmarker}{\pgfqpoint{-0.013889in}{0.000000in}}{\pgfqpoint{-0.000000in}{0.000000in}}{%
\pgfpathmoveto{\pgfqpoint{-0.000000in}{0.000000in}}%
\pgfpathlineto{\pgfqpoint{-0.013889in}{0.000000in}}%
\pgfusepath{stroke,fill}%
}%
\begin{pgfscope}%
\pgfsys@transformshift{0.868056in}{1.383640in}%
\pgfsys@useobject{currentmarker}{}%
\end{pgfscope}%
\end{pgfscope}%
\begin{pgfscope}%
\definecolor{textcolor}{rgb}{0.000000,0.000000,0.000000}%
\pgfsetstrokecolor{textcolor}%
\pgfsetfillcolor{textcolor}%
\pgftext[x=0.702271in, y=1.355903in, left, base]{\color{textcolor}{\ifdefined\pdftexversion\else\setmainfont{NanumMyeongjo}\rmfamily\fi\fontsize{5.000000}{6.000000}\selectfont\catcode`\^=\active\def^{\ifmmode\sp\else\^{}\fi}\catcode`\%=\active\def%{\%}8만}}%
\end{pgfscope}%
\begin{pgfscope}%
\pgfpathrectangle{\pgfqpoint{0.868056in}{0.555556in}}{\pgfqpoint{3.993056in}{1.888889in}}%
\pgfusepath{clip}%
\pgfsetbuttcap%
\pgfsetroundjoin%
\pgfsetlinewidth{0.602250pt}%
\definecolor{currentstroke}{rgb}{0.690196,0.690196,0.690196}%
\pgfsetstrokecolor{currentstroke}%
\pgfsetstrokeopacity{0.400000}%
\pgfsetdash{{2.220000pt}{0.960000pt}}{0.000000pt}%
\pgfpathmoveto{\pgfqpoint{0.868056in}{1.590662in}}%
\pgfpathlineto{\pgfqpoint{4.861111in}{1.590662in}}%
\pgfusepath{stroke}%
\end{pgfscope}%
\begin{pgfscope}%
\pgfsetbuttcap%
\pgfsetroundjoin%
\definecolor{currentfill}{rgb}{0.000000,0.000000,0.000000}%
\pgfsetfillcolor{currentfill}%
\pgfsetlinewidth{0.752812pt}%
\definecolor{currentstroke}{rgb}{0.000000,0.000000,0.000000}%
\pgfsetstrokecolor{currentstroke}%
\pgfsetdash{}{0pt}%
\pgfsys@defobject{currentmarker}{\pgfqpoint{-0.013889in}{0.000000in}}{\pgfqpoint{-0.000000in}{0.000000in}}{%
\pgfpathmoveto{\pgfqpoint{-0.000000in}{0.000000in}}%
\pgfpathlineto{\pgfqpoint{-0.013889in}{0.000000in}}%
\pgfusepath{stroke,fill}%
}%
\begin{pgfscope}%
\pgfsys@transformshift{0.868056in}{1.590662in}%
\pgfsys@useobject{currentmarker}{}%
\end{pgfscope}%
\end{pgfscope}%
\begin{pgfscope}%
\definecolor{textcolor}{rgb}{0.000000,0.000000,0.000000}%
\pgfsetstrokecolor{textcolor}%
\pgfsetfillcolor{textcolor}%
\pgftext[x=0.664971in, y=1.562924in, left, base]{\color{textcolor}{\ifdefined\pdftexversion\else\setmainfont{NanumMyeongjo}\rmfamily\fi\fontsize{5.000000}{6.000000}\selectfont\catcode`\^=\active\def^{\ifmmode\sp\else\^{}\fi}\catcode`\%=\active\def%{\%}10만}}%
\end{pgfscope}%
\begin{pgfscope}%
\pgfpathrectangle{\pgfqpoint{0.868056in}{0.555556in}}{\pgfqpoint{3.993056in}{1.888889in}}%
\pgfusepath{clip}%
\pgfsetbuttcap%
\pgfsetroundjoin%
\pgfsetlinewidth{0.602250pt}%
\definecolor{currentstroke}{rgb}{0.690196,0.690196,0.690196}%
\pgfsetstrokecolor{currentstroke}%
\pgfsetstrokeopacity{0.400000}%
\pgfsetdash{{2.220000pt}{0.960000pt}}{0.000000pt}%
\pgfpathmoveto{\pgfqpoint{0.868056in}{1.797683in}}%
\pgfpathlineto{\pgfqpoint{4.861111in}{1.797683in}}%
\pgfusepath{stroke}%
\end{pgfscope}%
\begin{pgfscope}%
\pgfsetbuttcap%
\pgfsetroundjoin%
\definecolor{currentfill}{rgb}{0.000000,0.000000,0.000000}%
\pgfsetfillcolor{currentfill}%
\pgfsetlinewidth{0.752812pt}%
\definecolor{currentstroke}{rgb}{0.000000,0.000000,0.000000}%
\pgfsetstrokecolor{currentstroke}%
\pgfsetdash{}{0pt}%
\pgfsys@defobject{currentmarker}{\pgfqpoint{-0.013889in}{0.000000in}}{\pgfqpoint{-0.000000in}{0.000000in}}{%
\pgfpathmoveto{\pgfqpoint{-0.000000in}{0.000000in}}%
\pgfpathlineto{\pgfqpoint{-0.013889in}{0.000000in}}%
\pgfusepath{stroke,fill}%
}%
\begin{pgfscope}%
\pgfsys@transformshift{0.868056in}{1.797683in}%
\pgfsys@useobject{currentmarker}{}%
\end{pgfscope}%
\end{pgfscope}%
\begin{pgfscope}%
\definecolor{textcolor}{rgb}{0.000000,0.000000,0.000000}%
\pgfsetstrokecolor{textcolor}%
\pgfsetfillcolor{textcolor}%
\pgftext[x=0.664971in, y=1.769946in, left, base]{\color{textcolor}{\ifdefined\pdftexversion\else\setmainfont{NanumMyeongjo}\rmfamily\fi\fontsize{5.000000}{6.000000}\selectfont\catcode`\^=\active\def^{\ifmmode\sp\else\^{}\fi}\catcode`\%=\active\def%{\%}12만}}%
\end{pgfscope}%
\begin{pgfscope}%
\pgfpathrectangle{\pgfqpoint{0.868056in}{0.555556in}}{\pgfqpoint{3.993056in}{1.888889in}}%
\pgfusepath{clip}%
\pgfsetbuttcap%
\pgfsetroundjoin%
\pgfsetlinewidth{0.602250pt}%
\definecolor{currentstroke}{rgb}{0.690196,0.690196,0.690196}%
\pgfsetstrokecolor{currentstroke}%
\pgfsetstrokeopacity{0.400000}%
\pgfsetdash{{2.220000pt}{0.960000pt}}{0.000000pt}%
\pgfpathmoveto{\pgfqpoint{0.868056in}{2.004704in}}%
\pgfpathlineto{\pgfqpoint{4.861111in}{2.004704in}}%
\pgfusepath{stroke}%
\end{pgfscope}%
\begin{pgfscope}%
\pgfsetbuttcap%
\pgfsetroundjoin%
\definecolor{currentfill}{rgb}{0.000000,0.000000,0.000000}%
\pgfsetfillcolor{currentfill}%
\pgfsetlinewidth{0.752812pt}%
\definecolor{currentstroke}{rgb}{0.000000,0.000000,0.000000}%
\pgfsetstrokecolor{currentstroke}%
\pgfsetdash{}{0pt}%
\pgfsys@defobject{currentmarker}{\pgfqpoint{-0.013889in}{0.000000in}}{\pgfqpoint{-0.000000in}{0.000000in}}{%
\pgfpathmoveto{\pgfqpoint{-0.000000in}{0.000000in}}%
\pgfpathlineto{\pgfqpoint{-0.013889in}{0.000000in}}%
\pgfusepath{stroke,fill}%
}%
\begin{pgfscope}%
\pgfsys@transformshift{0.868056in}{2.004704in}%
\pgfsys@useobject{currentmarker}{}%
\end{pgfscope}%
\end{pgfscope}%
\begin{pgfscope}%
\definecolor{textcolor}{rgb}{0.000000,0.000000,0.000000}%
\pgfsetstrokecolor{textcolor}%
\pgfsetfillcolor{textcolor}%
\pgftext[x=0.664971in, y=1.976967in, left, base]{\color{textcolor}{\ifdefined\pdftexversion\else\setmainfont{NanumMyeongjo}\rmfamily\fi\fontsize{5.000000}{6.000000}\selectfont\catcode`\^=\active\def^{\ifmmode\sp\else\^{}\fi}\catcode`\%=\active\def%{\%}14만}}%
\end{pgfscope}%
\begin{pgfscope}%
\pgfpathrectangle{\pgfqpoint{0.868056in}{0.555556in}}{\pgfqpoint{3.993056in}{1.888889in}}%
\pgfusepath{clip}%
\pgfsetbuttcap%
\pgfsetroundjoin%
\pgfsetlinewidth{0.602250pt}%
\definecolor{currentstroke}{rgb}{0.690196,0.690196,0.690196}%
\pgfsetstrokecolor{currentstroke}%
\pgfsetstrokeopacity{0.400000}%
\pgfsetdash{{2.220000pt}{0.960000pt}}{0.000000pt}%
\pgfpathmoveto{\pgfqpoint{0.868056in}{2.211725in}}%
\pgfpathlineto{\pgfqpoint{4.861111in}{2.211725in}}%
\pgfusepath{stroke}%
\end{pgfscope}%
\begin{pgfscope}%
\pgfsetbuttcap%
\pgfsetroundjoin%
\definecolor{currentfill}{rgb}{0.000000,0.000000,0.000000}%
\pgfsetfillcolor{currentfill}%
\pgfsetlinewidth{0.752812pt}%
\definecolor{currentstroke}{rgb}{0.000000,0.000000,0.000000}%
\pgfsetstrokecolor{currentstroke}%
\pgfsetdash{}{0pt}%
\pgfsys@defobject{currentmarker}{\pgfqpoint{-0.013889in}{0.000000in}}{\pgfqpoint{-0.000000in}{0.000000in}}{%
\pgfpathmoveto{\pgfqpoint{-0.000000in}{0.000000in}}%
\pgfpathlineto{\pgfqpoint{-0.013889in}{0.000000in}}%
\pgfusepath{stroke,fill}%
}%
\begin{pgfscope}%
\pgfsys@transformshift{0.868056in}{2.211725in}%
\pgfsys@useobject{currentmarker}{}%
\end{pgfscope}%
\end{pgfscope}%
\begin{pgfscope}%
\definecolor{textcolor}{rgb}{0.000000,0.000000,0.000000}%
\pgfsetstrokecolor{textcolor}%
\pgfsetfillcolor{textcolor}%
\pgftext[x=0.664971in, y=2.183988in, left, base]{\color{textcolor}{\ifdefined\pdftexversion\else\setmainfont{NanumMyeongjo}\rmfamily\fi\fontsize{5.000000}{6.000000}\selectfont\catcode`\^=\active\def^{\ifmmode\sp\else\^{}\fi}\catcode`\%=\active\def%{\%}16만}}%
\end{pgfscope}%
\begin{pgfscope}%
\pgfpathrectangle{\pgfqpoint{0.868056in}{0.555556in}}{\pgfqpoint{3.993056in}{1.888889in}}%
\pgfusepath{clip}%
\pgfsetbuttcap%
\pgfsetroundjoin%
\pgfsetlinewidth{0.602250pt}%
\definecolor{currentstroke}{rgb}{0.690196,0.690196,0.690196}%
\pgfsetstrokecolor{currentstroke}%
\pgfsetstrokeopacity{0.400000}%
\pgfsetdash{{2.220000pt}{0.960000pt}}{0.000000pt}%
\pgfpathmoveto{\pgfqpoint{0.868056in}{2.418746in}}%
\pgfpathlineto{\pgfqpoint{4.861111in}{2.418746in}}%
\pgfusepath{stroke}%
\end{pgfscope}%
\begin{pgfscope}%
\pgfsetbuttcap%
\pgfsetroundjoin%
\definecolor{currentfill}{rgb}{0.000000,0.000000,0.000000}%
\pgfsetfillcolor{currentfill}%
\pgfsetlinewidth{0.752812pt}%
\definecolor{currentstroke}{rgb}{0.000000,0.000000,0.000000}%
\pgfsetstrokecolor{currentstroke}%
\pgfsetdash{}{0pt}%
\pgfsys@defobject{currentmarker}{\pgfqpoint{-0.013889in}{0.000000in}}{\pgfqpoint{-0.000000in}{0.000000in}}{%
\pgfpathmoveto{\pgfqpoint{-0.000000in}{0.000000in}}%
\pgfpathlineto{\pgfqpoint{-0.013889in}{0.000000in}}%
\pgfusepath{stroke,fill}%
}%
\begin{pgfscope}%
\pgfsys@transformshift{0.868056in}{2.418746in}%
\pgfsys@useobject{currentmarker}{}%
\end{pgfscope}%
\end{pgfscope}%
\begin{pgfscope}%
\definecolor{textcolor}{rgb}{0.000000,0.000000,0.000000}%
\pgfsetstrokecolor{textcolor}%
\pgfsetfillcolor{textcolor}%
\pgftext[x=0.664971in, y=2.391009in, left, base]{\color{textcolor}{\ifdefined\pdftexversion\else\setmainfont{NanumMyeongjo}\rmfamily\fi\fontsize{5.000000}{6.000000}\selectfont\catcode`\^=\active\def^{\ifmmode\sp\else\^{}\fi}\catcode`\%=\active\def%{\%}18만}}%
\end{pgfscope}%
\begin{pgfscope}%
\pgfsetrectcap%
\pgfsetmiterjoin%
\pgfsetlinewidth{0.752812pt}%
\definecolor{currentstroke}{rgb}{0.000000,0.000000,0.000000}%
\pgfsetstrokecolor{currentstroke}%
\pgfsetdash{}{0pt}%
\pgfpathmoveto{\pgfqpoint{0.868056in}{0.555556in}}%
\pgfpathlineto{\pgfqpoint{0.868056in}{2.444444in}}%
\pgfusepath{stroke}%
\end{pgfscope}%
\begin{pgfscope}%
\pgfsetrectcap%
\pgfsetmiterjoin%
\pgfsetlinewidth{0.752812pt}%
\definecolor{currentstroke}{rgb}{0.000000,0.000000,0.000000}%
\pgfsetstrokecolor{currentstroke}%
\pgfsetdash{}{0pt}%
\pgfpathmoveto{\pgfqpoint{0.868056in}{0.555556in}}%
\pgfpathlineto{\pgfqpoint{4.861111in}{0.555556in}}%
\pgfusepath{stroke}%
\end{pgfscope}%
\begin{pgfscope}%
\pgfpathrectangle{\pgfqpoint{0.868056in}{0.555556in}}{\pgfqpoint{3.993056in}{1.888889in}}%
\pgfusepath{clip}%
\pgfsetbuttcap%
\pgfsetmiterjoin%
\definecolor{currentfill}{rgb}{0.337255,0.713725,0.627451}%
\pgfsetfillcolor{currentfill}%
\pgfsetlinewidth{1.003750pt}%
\definecolor{currentstroke}{rgb}{0.266667,0.266667,0.266667}%
\pgfsetstrokecolor{currentstroke}%
\pgfsetdash{}{0pt}%
\pgfpathmoveto{\pgfqpoint{1.049558in}{0.555556in}}%
\pgfpathlineto{\pgfqpoint{1.287038in}{0.555556in}}%
\pgfpathlineto{\pgfqpoint{1.287038in}{0.710221in}}%
\pgfpathlineto{\pgfqpoint{1.049558in}{0.710221in}}%
\pgfpathlineto{\pgfqpoint{1.049558in}{0.555556in}}%
\pgfpathclose%
\pgfusepath{stroke,fill}%
\end{pgfscope}%
\begin{pgfscope}%
\pgfpathrectangle{\pgfqpoint{0.868056in}{0.555556in}}{\pgfqpoint{3.993056in}{1.888889in}}%
\pgfusepath{clip}%
\pgfsetbuttcap%
\pgfsetmiterjoin%
\definecolor{currentfill}{rgb}{0.337255,0.713725,0.627451}%
\pgfsetfillcolor{currentfill}%
\pgfsetlinewidth{1.003750pt}%
\definecolor{currentstroke}{rgb}{0.266667,0.266667,0.266667}%
\pgfsetstrokecolor{currentstroke}%
\pgfsetdash{}{0pt}%
\pgfpathmoveto{\pgfqpoint{1.388815in}{0.555556in}}%
\pgfpathlineto{\pgfqpoint{1.626295in}{0.555556in}}%
\pgfpathlineto{\pgfqpoint{1.626295in}{0.670535in}}%
\pgfpathlineto{\pgfqpoint{1.388815in}{0.670535in}}%
\pgfpathlineto{\pgfqpoint{1.388815in}{0.555556in}}%
\pgfpathclose%
\pgfusepath{stroke,fill}%
\end{pgfscope}%
\begin{pgfscope}%
\pgfpathrectangle{\pgfqpoint{0.868056in}{0.555556in}}{\pgfqpoint{3.993056in}{1.888889in}}%
\pgfusepath{clip}%
\pgfsetbuttcap%
\pgfsetmiterjoin%
\definecolor{currentfill}{rgb}{0.337255,0.713725,0.627451}%
\pgfsetfillcolor{currentfill}%
\pgfsetlinewidth{1.003750pt}%
\definecolor{currentstroke}{rgb}{0.266667,0.266667,0.266667}%
\pgfsetstrokecolor{currentstroke}%
\pgfsetdash{}{0pt}%
\pgfpathmoveto{\pgfqpoint{1.728072in}{0.555556in}}%
\pgfpathlineto{\pgfqpoint{1.965552in}{0.555556in}}%
\pgfpathlineto{\pgfqpoint{1.965552in}{0.631284in}}%
\pgfpathlineto{\pgfqpoint{1.728072in}{0.631284in}}%
\pgfpathlineto{\pgfqpoint{1.728072in}{0.555556in}}%
\pgfpathclose%
\pgfusepath{stroke,fill}%
\end{pgfscope}%
\begin{pgfscope}%
\pgfpathrectangle{\pgfqpoint{0.868056in}{0.555556in}}{\pgfqpoint{3.993056in}{1.888889in}}%
\pgfusepath{clip}%
\pgfsetbuttcap%
\pgfsetmiterjoin%
\definecolor{currentfill}{rgb}{0.337255,0.713725,0.627451}%
\pgfsetfillcolor{currentfill}%
\pgfsetlinewidth{1.003750pt}%
\definecolor{currentstroke}{rgb}{0.266667,0.266667,0.266667}%
\pgfsetstrokecolor{currentstroke}%
\pgfsetdash{}{0pt}%
\pgfpathmoveto{\pgfqpoint{2.067329in}{0.555556in}}%
\pgfpathlineto{\pgfqpoint{2.304809in}{0.555556in}}%
\pgfpathlineto{\pgfqpoint{2.304809in}{0.686952in}}%
\pgfpathlineto{\pgfqpoint{2.067329in}{0.686952in}}%
\pgfpathlineto{\pgfqpoint{2.067329in}{0.555556in}}%
\pgfpathclose%
\pgfusepath{stroke,fill}%
\end{pgfscope}%
\begin{pgfscope}%
\pgfpathrectangle{\pgfqpoint{0.868056in}{0.555556in}}{\pgfqpoint{3.993056in}{1.888889in}}%
\pgfusepath{clip}%
\pgfsetbuttcap%
\pgfsetmiterjoin%
\definecolor{currentfill}{rgb}{0.337255,0.713725,0.627451}%
\pgfsetfillcolor{currentfill}%
\pgfsetlinewidth{1.003750pt}%
\definecolor{currentstroke}{rgb}{0.266667,0.266667,0.266667}%
\pgfsetstrokecolor{currentstroke}%
\pgfsetdash{}{0pt}%
\pgfpathmoveto{\pgfqpoint{2.406586in}{0.555556in}}%
\pgfpathlineto{\pgfqpoint{2.644066in}{0.555556in}}%
\pgfpathlineto{\pgfqpoint{2.644066in}{0.746574in}}%
\pgfpathlineto{\pgfqpoint{2.406586in}{0.746574in}}%
\pgfpathlineto{\pgfqpoint{2.406586in}{0.555556in}}%
\pgfpathclose%
\pgfusepath{stroke,fill}%
\end{pgfscope}%
\begin{pgfscope}%
\pgfpathrectangle{\pgfqpoint{0.868056in}{0.555556in}}{\pgfqpoint{3.993056in}{1.888889in}}%
\pgfusepath{clip}%
\pgfsetbuttcap%
\pgfsetmiterjoin%
\definecolor{currentfill}{rgb}{0.337255,0.713725,0.627451}%
\pgfsetfillcolor{currentfill}%
\pgfsetlinewidth{1.003750pt}%
\definecolor{currentstroke}{rgb}{0.266667,0.266667,0.266667}%
\pgfsetstrokecolor{currentstroke}%
\pgfsetdash{}{0pt}%
\pgfpathmoveto{\pgfqpoint{2.745843in}{0.555556in}}%
\pgfpathlineto{\pgfqpoint{2.983323in}{0.555556in}}%
\pgfpathlineto{\pgfqpoint{2.983323in}{0.811713in}}%
\pgfpathlineto{\pgfqpoint{2.745843in}{0.811713in}}%
\pgfpathlineto{\pgfqpoint{2.745843in}{0.555556in}}%
\pgfpathclose%
\pgfusepath{stroke,fill}%
\end{pgfscope}%
\begin{pgfscope}%
\pgfpathrectangle{\pgfqpoint{0.868056in}{0.555556in}}{\pgfqpoint{3.993056in}{1.888889in}}%
\pgfusepath{clip}%
\pgfsetbuttcap%
\pgfsetmiterjoin%
\definecolor{currentfill}{rgb}{0.337255,0.713725,0.627451}%
\pgfsetfillcolor{currentfill}%
\pgfsetlinewidth{1.003750pt}%
\definecolor{currentstroke}{rgb}{0.266667,0.266667,0.266667}%
\pgfsetstrokecolor{currentstroke}%
\pgfsetdash{}{0pt}%
\pgfpathmoveto{\pgfqpoint{3.085100in}{0.555556in}}%
\pgfpathlineto{\pgfqpoint{3.322580in}{0.555556in}}%
\pgfpathlineto{\pgfqpoint{3.322580in}{0.720614in}}%
\pgfpathlineto{\pgfqpoint{3.085100in}{0.720614in}}%
\pgfpathlineto{\pgfqpoint{3.085100in}{0.555556in}}%
\pgfpathclose%
\pgfusepath{stroke,fill}%
\end{pgfscope}%
\begin{pgfscope}%
\pgfpathrectangle{\pgfqpoint{0.868056in}{0.555556in}}{\pgfqpoint{3.993056in}{1.888889in}}%
\pgfusepath{clip}%
\pgfsetbuttcap%
\pgfsetmiterjoin%
\definecolor{currentfill}{rgb}{0.337255,0.713725,0.627451}%
\pgfsetfillcolor{currentfill}%
\pgfsetlinewidth{1.003750pt}%
\definecolor{currentstroke}{rgb}{0.266667,0.266667,0.266667}%
\pgfsetstrokecolor{currentstroke}%
\pgfsetdash{}{0pt}%
\pgfpathmoveto{\pgfqpoint{3.424357in}{0.555556in}}%
\pgfpathlineto{\pgfqpoint{3.661837in}{0.555556in}}%
\pgfpathlineto{\pgfqpoint{3.661837in}{0.808618in}}%
\pgfpathlineto{\pgfqpoint{3.424357in}{0.808618in}}%
\pgfpathlineto{\pgfqpoint{3.424357in}{0.555556in}}%
\pgfpathclose%
\pgfusepath{stroke,fill}%
\end{pgfscope}%
\begin{pgfscope}%
\pgfpathrectangle{\pgfqpoint{0.868056in}{0.555556in}}{\pgfqpoint{3.993056in}{1.888889in}}%
\pgfusepath{clip}%
\pgfsetbuttcap%
\pgfsetmiterjoin%
\definecolor{currentfill}{rgb}{0.337255,0.713725,0.627451}%
\pgfsetfillcolor{currentfill}%
\pgfsetlinewidth{1.003750pt}%
\definecolor{currentstroke}{rgb}{0.266667,0.266667,0.266667}%
\pgfsetstrokecolor{currentstroke}%
\pgfsetdash{}{0pt}%
\pgfpathmoveto{\pgfqpoint{3.763615in}{0.555556in}}%
\pgfpathlineto{\pgfqpoint{4.001094in}{0.555556in}}%
\pgfpathlineto{\pgfqpoint{4.001094in}{0.916394in}}%
\pgfpathlineto{\pgfqpoint{3.763615in}{0.916394in}}%
\pgfpathlineto{\pgfqpoint{3.763615in}{0.555556in}}%
\pgfpathclose%
\pgfusepath{stroke,fill}%
\end{pgfscope}%
\begin{pgfscope}%
\pgfpathrectangle{\pgfqpoint{0.868056in}{0.555556in}}{\pgfqpoint{3.993056in}{1.888889in}}%
\pgfusepath{clip}%
\pgfsetbuttcap%
\pgfsetmiterjoin%
\definecolor{currentfill}{rgb}{0.337255,0.713725,0.627451}%
\pgfsetfillcolor{currentfill}%
\pgfsetlinewidth{1.003750pt}%
\definecolor{currentstroke}{rgb}{0.266667,0.266667,0.266667}%
\pgfsetstrokecolor{currentstroke}%
\pgfsetdash{}{0pt}%
\pgfpathmoveto{\pgfqpoint{4.102872in}{0.555556in}}%
\pgfpathlineto{\pgfqpoint{4.340352in}{0.555556in}}%
\pgfpathlineto{\pgfqpoint{4.340352in}{0.965023in}}%
\pgfpathlineto{\pgfqpoint{4.102872in}{0.965023in}}%
\pgfpathlineto{\pgfqpoint{4.102872in}{0.555556in}}%
\pgfpathclose%
\pgfusepath{stroke,fill}%
\end{pgfscope}%
\begin{pgfscope}%
\pgfpathrectangle{\pgfqpoint{0.868056in}{0.555556in}}{\pgfqpoint{3.993056in}{1.888889in}}%
\pgfusepath{clip}%
\pgfsetbuttcap%
\pgfsetmiterjoin%
\definecolor{currentfill}{rgb}{0.337255,0.713725,0.627451}%
\pgfsetfillcolor{currentfill}%
\pgfsetlinewidth{1.003750pt}%
\definecolor{currentstroke}{rgb}{0.266667,0.266667,0.266667}%
\pgfsetstrokecolor{currentstroke}%
\pgfsetdash{}{0pt}%
\pgfpathmoveto{\pgfqpoint{4.442129in}{0.555556in}}%
\pgfpathlineto{\pgfqpoint{4.679609in}{0.555556in}}%
\pgfpathlineto{\pgfqpoint{4.679609in}{1.095591in}}%
\pgfpathlineto{\pgfqpoint{4.442129in}{1.095591in}}%
\pgfpathlineto{\pgfqpoint{4.442129in}{0.555556in}}%
\pgfpathclose%
\pgfusepath{stroke,fill}%
\end{pgfscope}%
\begin{pgfscope}%
\pgfpathrectangle{\pgfqpoint{0.868056in}{0.555556in}}{\pgfqpoint{3.993056in}{1.888889in}}%
\pgfusepath{clip}%
\pgfsetbuttcap%
\pgfsetmiterjoin%
\definecolor{currentfill}{rgb}{0.235294,0.490196,0.764706}%
\pgfsetfillcolor{currentfill}%
\pgfsetlinewidth{1.003750pt}%
\definecolor{currentstroke}{rgb}{0.266667,0.266667,0.266667}%
\pgfsetstrokecolor{currentstroke}%
\pgfsetdash{}{0pt}%
\pgfpathmoveto{\pgfqpoint{1.049558in}{0.710221in}}%
\pgfpathlineto{\pgfqpoint{1.287038in}{0.710221in}}%
\pgfpathlineto{\pgfqpoint{1.287038in}{1.003425in}}%
\pgfpathlineto{\pgfqpoint{1.049558in}{1.003425in}}%
\pgfpathlineto{\pgfqpoint{1.049558in}{0.710221in}}%
\pgfpathclose%
\pgfusepath{stroke,fill}%
\end{pgfscope}%
\begin{pgfscope}%
\pgfpathrectangle{\pgfqpoint{0.868056in}{0.555556in}}{\pgfqpoint{3.993056in}{1.888889in}}%
\pgfusepath{clip}%
\pgfsetbuttcap%
\pgfsetmiterjoin%
\definecolor{currentfill}{rgb}{0.235294,0.490196,0.764706}%
\pgfsetfillcolor{currentfill}%
\pgfsetlinewidth{1.003750pt}%
\definecolor{currentstroke}{rgb}{0.266667,0.266667,0.266667}%
\pgfsetstrokecolor{currentstroke}%
\pgfsetdash{}{0pt}%
\pgfpathmoveto{\pgfqpoint{1.388815in}{0.670535in}}%
\pgfpathlineto{\pgfqpoint{1.626295in}{0.670535in}}%
\pgfpathlineto{\pgfqpoint{1.626295in}{0.877670in}}%
\pgfpathlineto{\pgfqpoint{1.388815in}{0.877670in}}%
\pgfpathlineto{\pgfqpoint{1.388815in}{0.670535in}}%
\pgfpathclose%
\pgfusepath{stroke,fill}%
\end{pgfscope}%
\begin{pgfscope}%
\pgfpathrectangle{\pgfqpoint{0.868056in}{0.555556in}}{\pgfqpoint{3.993056in}{1.888889in}}%
\pgfusepath{clip}%
\pgfsetbuttcap%
\pgfsetmiterjoin%
\definecolor{currentfill}{rgb}{0.235294,0.490196,0.764706}%
\pgfsetfillcolor{currentfill}%
\pgfsetlinewidth{1.003750pt}%
\definecolor{currentstroke}{rgb}{0.266667,0.266667,0.266667}%
\pgfsetstrokecolor{currentstroke}%
\pgfsetdash{}{0pt}%
\pgfpathmoveto{\pgfqpoint{1.728072in}{0.631284in}}%
\pgfpathlineto{\pgfqpoint{1.965552in}{0.631284in}}%
\pgfpathlineto{\pgfqpoint{1.965552in}{0.770040in}}%
\pgfpathlineto{\pgfqpoint{1.728072in}{0.770040in}}%
\pgfpathlineto{\pgfqpoint{1.728072in}{0.631284in}}%
\pgfpathclose%
\pgfusepath{stroke,fill}%
\end{pgfscope}%
\begin{pgfscope}%
\pgfpathrectangle{\pgfqpoint{0.868056in}{0.555556in}}{\pgfqpoint{3.993056in}{1.888889in}}%
\pgfusepath{clip}%
\pgfsetbuttcap%
\pgfsetmiterjoin%
\definecolor{currentfill}{rgb}{0.235294,0.490196,0.764706}%
\pgfsetfillcolor{currentfill}%
\pgfsetlinewidth{1.003750pt}%
\definecolor{currentstroke}{rgb}{0.266667,0.266667,0.266667}%
\pgfsetstrokecolor{currentstroke}%
\pgfsetdash{}{0pt}%
\pgfpathmoveto{\pgfqpoint{2.067329in}{0.686952in}}%
\pgfpathlineto{\pgfqpoint{2.304809in}{0.686952in}}%
\pgfpathlineto{\pgfqpoint{2.304809in}{0.868209in}}%
\pgfpathlineto{\pgfqpoint{2.067329in}{0.868209in}}%
\pgfpathlineto{\pgfqpoint{2.067329in}{0.686952in}}%
\pgfpathclose%
\pgfusepath{stroke,fill}%
\end{pgfscope}%
\begin{pgfscope}%
\pgfpathrectangle{\pgfqpoint{0.868056in}{0.555556in}}{\pgfqpoint{3.993056in}{1.888889in}}%
\pgfusepath{clip}%
\pgfsetbuttcap%
\pgfsetmiterjoin%
\definecolor{currentfill}{rgb}{0.235294,0.490196,0.764706}%
\pgfsetfillcolor{currentfill}%
\pgfsetlinewidth{1.003750pt}%
\definecolor{currentstroke}{rgb}{0.266667,0.266667,0.266667}%
\pgfsetstrokecolor{currentstroke}%
\pgfsetdash{}{0pt}%
\pgfpathmoveto{\pgfqpoint{2.406586in}{0.746574in}}%
\pgfpathlineto{\pgfqpoint{2.644066in}{0.746574in}}%
\pgfpathlineto{\pgfqpoint{2.644066in}{0.940926in}}%
\pgfpathlineto{\pgfqpoint{2.406586in}{0.940926in}}%
\pgfpathlineto{\pgfqpoint{2.406586in}{0.746574in}}%
\pgfpathclose%
\pgfusepath{stroke,fill}%
\end{pgfscope}%
\begin{pgfscope}%
\pgfpathrectangle{\pgfqpoint{0.868056in}{0.555556in}}{\pgfqpoint{3.993056in}{1.888889in}}%
\pgfusepath{clip}%
\pgfsetbuttcap%
\pgfsetmiterjoin%
\definecolor{currentfill}{rgb}{0.235294,0.490196,0.764706}%
\pgfsetfillcolor{currentfill}%
\pgfsetlinewidth{1.003750pt}%
\definecolor{currentstroke}{rgb}{0.266667,0.266667,0.266667}%
\pgfsetstrokecolor{currentstroke}%
\pgfsetdash{}{0pt}%
\pgfpathmoveto{\pgfqpoint{2.745843in}{0.811713in}}%
\pgfpathlineto{\pgfqpoint{2.983323in}{0.811713in}}%
\pgfpathlineto{\pgfqpoint{2.983323in}{1.031818in}}%
\pgfpathlineto{\pgfqpoint{2.745843in}{1.031818in}}%
\pgfpathlineto{\pgfqpoint{2.745843in}{0.811713in}}%
\pgfpathclose%
\pgfusepath{stroke,fill}%
\end{pgfscope}%
\begin{pgfscope}%
\pgfpathrectangle{\pgfqpoint{0.868056in}{0.555556in}}{\pgfqpoint{3.993056in}{1.888889in}}%
\pgfusepath{clip}%
\pgfsetbuttcap%
\pgfsetmiterjoin%
\definecolor{currentfill}{rgb}{0.235294,0.490196,0.764706}%
\pgfsetfillcolor{currentfill}%
\pgfsetlinewidth{1.003750pt}%
\definecolor{currentstroke}{rgb}{0.266667,0.266667,0.266667}%
\pgfsetstrokecolor{currentstroke}%
\pgfsetdash{}{0pt}%
\pgfpathmoveto{\pgfqpoint{3.085100in}{0.720614in}}%
\pgfpathlineto{\pgfqpoint{3.322580in}{0.720614in}}%
\pgfpathlineto{\pgfqpoint{3.322580in}{0.899252in}}%
\pgfpathlineto{\pgfqpoint{3.085100in}{0.899252in}}%
\pgfpathlineto{\pgfqpoint{3.085100in}{0.720614in}}%
\pgfpathclose%
\pgfusepath{stroke,fill}%
\end{pgfscope}%
\begin{pgfscope}%
\pgfpathrectangle{\pgfqpoint{0.868056in}{0.555556in}}{\pgfqpoint{3.993056in}{1.888889in}}%
\pgfusepath{clip}%
\pgfsetbuttcap%
\pgfsetmiterjoin%
\definecolor{currentfill}{rgb}{0.235294,0.490196,0.764706}%
\pgfsetfillcolor{currentfill}%
\pgfsetlinewidth{1.003750pt}%
\definecolor{currentstroke}{rgb}{0.266667,0.266667,0.266667}%
\pgfsetstrokecolor{currentstroke}%
\pgfsetdash{}{0pt}%
\pgfpathmoveto{\pgfqpoint{3.424357in}{0.808618in}}%
\pgfpathlineto{\pgfqpoint{3.661837in}{0.808618in}}%
\pgfpathlineto{\pgfqpoint{3.661837in}{1.067198in}}%
\pgfpathlineto{\pgfqpoint{3.424357in}{1.067198in}}%
\pgfpathlineto{\pgfqpoint{3.424357in}{0.808618in}}%
\pgfpathclose%
\pgfusepath{stroke,fill}%
\end{pgfscope}%
\begin{pgfscope}%
\pgfpathrectangle{\pgfqpoint{0.868056in}{0.555556in}}{\pgfqpoint{3.993056in}{1.888889in}}%
\pgfusepath{clip}%
\pgfsetbuttcap%
\pgfsetmiterjoin%
\definecolor{currentfill}{rgb}{0.235294,0.490196,0.764706}%
\pgfsetfillcolor{currentfill}%
\pgfsetlinewidth{1.003750pt}%
\definecolor{currentstroke}{rgb}{0.266667,0.266667,0.266667}%
\pgfsetstrokecolor{currentstroke}%
\pgfsetdash{}{0pt}%
\pgfpathmoveto{\pgfqpoint{3.763615in}{0.916394in}}%
\pgfpathlineto{\pgfqpoint{4.001094in}{0.916394in}}%
\pgfpathlineto{\pgfqpoint{4.001094in}{1.174187in}}%
\pgfpathlineto{\pgfqpoint{3.763615in}{1.174187in}}%
\pgfpathlineto{\pgfqpoint{3.763615in}{0.916394in}}%
\pgfpathclose%
\pgfusepath{stroke,fill}%
\end{pgfscope}%
\begin{pgfscope}%
\pgfpathrectangle{\pgfqpoint{0.868056in}{0.555556in}}{\pgfqpoint{3.993056in}{1.888889in}}%
\pgfusepath{clip}%
\pgfsetbuttcap%
\pgfsetmiterjoin%
\definecolor{currentfill}{rgb}{0.235294,0.490196,0.764706}%
\pgfsetfillcolor{currentfill}%
\pgfsetlinewidth{1.003750pt}%
\definecolor{currentstroke}{rgb}{0.266667,0.266667,0.266667}%
\pgfsetstrokecolor{currentstroke}%
\pgfsetdash{}{0pt}%
\pgfpathmoveto{\pgfqpoint{4.102872in}{0.965023in}}%
\pgfpathlineto{\pgfqpoint{4.340352in}{0.965023in}}%
\pgfpathlineto{\pgfqpoint{4.340352in}{1.201545in}}%
\pgfpathlineto{\pgfqpoint{4.102872in}{1.201545in}}%
\pgfpathlineto{\pgfqpoint{4.102872in}{0.965023in}}%
\pgfpathclose%
\pgfusepath{stroke,fill}%
\end{pgfscope}%
\begin{pgfscope}%
\pgfpathrectangle{\pgfqpoint{0.868056in}{0.555556in}}{\pgfqpoint{3.993056in}{1.888889in}}%
\pgfusepath{clip}%
\pgfsetbuttcap%
\pgfsetmiterjoin%
\definecolor{currentfill}{rgb}{0.235294,0.490196,0.764706}%
\pgfsetfillcolor{currentfill}%
\pgfsetlinewidth{1.003750pt}%
\definecolor{currentstroke}{rgb}{0.266667,0.266667,0.266667}%
\pgfsetstrokecolor{currentstroke}%
\pgfsetdash{}{0pt}%
\pgfpathmoveto{\pgfqpoint{4.442129in}{1.095591in}}%
\pgfpathlineto{\pgfqpoint{4.679609in}{1.095591in}}%
\pgfpathlineto{\pgfqpoint{4.679609in}{1.352608in}}%
\pgfpathlineto{\pgfqpoint{4.442129in}{1.352608in}}%
\pgfpathlineto{\pgfqpoint{4.442129in}{1.095591in}}%
\pgfpathclose%
\pgfusepath{stroke,fill}%
\end{pgfscope}%
\begin{pgfscope}%
\pgfpathrectangle{\pgfqpoint{0.868056in}{0.555556in}}{\pgfqpoint{3.993056in}{1.888889in}}%
\pgfusepath{clip}%
\pgfsetbuttcap%
\pgfsetmiterjoin%
\definecolor{currentfill}{rgb}{0.725490,0.486275,0.164706}%
\pgfsetfillcolor{currentfill}%
\pgfsetlinewidth{1.003750pt}%
\definecolor{currentstroke}{rgb}{0.266667,0.266667,0.266667}%
\pgfsetstrokecolor{currentstroke}%
\pgfsetdash{}{0pt}%
\pgfpathmoveto{\pgfqpoint{1.049558in}{1.003425in}}%
\pgfpathlineto{\pgfqpoint{1.287038in}{1.003425in}}%
\pgfpathlineto{\pgfqpoint{1.287038in}{1.215819in}}%
\pgfpathlineto{\pgfqpoint{1.049558in}{1.215819in}}%
\pgfpathlineto{\pgfqpoint{1.049558in}{1.003425in}}%
\pgfpathclose%
\pgfusepath{stroke,fill}%
\end{pgfscope}%
\begin{pgfscope}%
\pgfpathrectangle{\pgfqpoint{0.868056in}{0.555556in}}{\pgfqpoint{3.993056in}{1.888889in}}%
\pgfusepath{clip}%
\pgfsetbuttcap%
\pgfsetmiterjoin%
\definecolor{currentfill}{rgb}{0.725490,0.486275,0.164706}%
\pgfsetfillcolor{currentfill}%
\pgfsetlinewidth{1.003750pt}%
\definecolor{currentstroke}{rgb}{0.266667,0.266667,0.266667}%
\pgfsetstrokecolor{currentstroke}%
\pgfsetdash{}{0pt}%
\pgfpathmoveto{\pgfqpoint{1.388815in}{0.877670in}}%
\pgfpathlineto{\pgfqpoint{1.626295in}{0.877670in}}%
\pgfpathlineto{\pgfqpoint{1.626295in}{1.087434in}}%
\pgfpathlineto{\pgfqpoint{1.388815in}{1.087434in}}%
\pgfpathlineto{\pgfqpoint{1.388815in}{0.877670in}}%
\pgfpathclose%
\pgfusepath{stroke,fill}%
\end{pgfscope}%
\begin{pgfscope}%
\pgfpathrectangle{\pgfqpoint{0.868056in}{0.555556in}}{\pgfqpoint{3.993056in}{1.888889in}}%
\pgfusepath{clip}%
\pgfsetbuttcap%
\pgfsetmiterjoin%
\definecolor{currentfill}{rgb}{0.725490,0.486275,0.164706}%
\pgfsetfillcolor{currentfill}%
\pgfsetlinewidth{1.003750pt}%
\definecolor{currentstroke}{rgb}{0.266667,0.266667,0.266667}%
\pgfsetstrokecolor{currentstroke}%
\pgfsetdash{}{0pt}%
\pgfpathmoveto{\pgfqpoint{1.728072in}{0.770040in}}%
\pgfpathlineto{\pgfqpoint{1.965552in}{0.770040in}}%
\pgfpathlineto{\pgfqpoint{1.965552in}{0.937354in}}%
\pgfpathlineto{\pgfqpoint{1.728072in}{0.937354in}}%
\pgfpathlineto{\pgfqpoint{1.728072in}{0.770040in}}%
\pgfpathclose%
\pgfusepath{stroke,fill}%
\end{pgfscope}%
\begin{pgfscope}%
\pgfpathrectangle{\pgfqpoint{0.868056in}{0.555556in}}{\pgfqpoint{3.993056in}{1.888889in}}%
\pgfusepath{clip}%
\pgfsetbuttcap%
\pgfsetmiterjoin%
\definecolor{currentfill}{rgb}{0.725490,0.486275,0.164706}%
\pgfsetfillcolor{currentfill}%
\pgfsetlinewidth{1.003750pt}%
\definecolor{currentstroke}{rgb}{0.266667,0.266667,0.266667}%
\pgfsetstrokecolor{currentstroke}%
\pgfsetdash{}{0pt}%
\pgfpathmoveto{\pgfqpoint{2.067329in}{0.868209in}}%
\pgfpathlineto{\pgfqpoint{2.304809in}{0.868209in}}%
\pgfpathlineto{\pgfqpoint{2.304809in}{1.037449in}}%
\pgfpathlineto{\pgfqpoint{2.067329in}{1.037449in}}%
\pgfpathlineto{\pgfqpoint{2.067329in}{0.868209in}}%
\pgfpathclose%
\pgfusepath{stroke,fill}%
\end{pgfscope}%
\begin{pgfscope}%
\pgfpathrectangle{\pgfqpoint{0.868056in}{0.555556in}}{\pgfqpoint{3.993056in}{1.888889in}}%
\pgfusepath{clip}%
\pgfsetbuttcap%
\pgfsetmiterjoin%
\definecolor{currentfill}{rgb}{0.725490,0.486275,0.164706}%
\pgfsetfillcolor{currentfill}%
\pgfsetlinewidth{1.003750pt}%
\definecolor{currentstroke}{rgb}{0.266667,0.266667,0.266667}%
\pgfsetstrokecolor{currentstroke}%
\pgfsetdash{}{0pt}%
\pgfpathmoveto{\pgfqpoint{2.406586in}{0.940926in}}%
\pgfpathlineto{\pgfqpoint{2.644066in}{0.940926in}}%
\pgfpathlineto{\pgfqpoint{2.644066in}{1.105476in}}%
\pgfpathlineto{\pgfqpoint{2.406586in}{1.105476in}}%
\pgfpathlineto{\pgfqpoint{2.406586in}{0.940926in}}%
\pgfpathclose%
\pgfusepath{stroke,fill}%
\end{pgfscope}%
\begin{pgfscope}%
\pgfpathrectangle{\pgfqpoint{0.868056in}{0.555556in}}{\pgfqpoint{3.993056in}{1.888889in}}%
\pgfusepath{clip}%
\pgfsetbuttcap%
\pgfsetmiterjoin%
\definecolor{currentfill}{rgb}{0.725490,0.486275,0.164706}%
\pgfsetfillcolor{currentfill}%
\pgfsetlinewidth{1.003750pt}%
\definecolor{currentstroke}{rgb}{0.266667,0.266667,0.266667}%
\pgfsetstrokecolor{currentstroke}%
\pgfsetdash{}{0pt}%
\pgfpathmoveto{\pgfqpoint{2.745843in}{1.031818in}}%
\pgfpathlineto{\pgfqpoint{2.983323in}{1.031818in}}%
\pgfpathlineto{\pgfqpoint{2.983323in}{1.207310in}}%
\pgfpathlineto{\pgfqpoint{2.745843in}{1.207310in}}%
\pgfpathlineto{\pgfqpoint{2.745843in}{1.031818in}}%
\pgfpathclose%
\pgfusepath{stroke,fill}%
\end{pgfscope}%
\begin{pgfscope}%
\pgfpathrectangle{\pgfqpoint{0.868056in}{0.555556in}}{\pgfqpoint{3.993056in}{1.888889in}}%
\pgfusepath{clip}%
\pgfsetbuttcap%
\pgfsetmiterjoin%
\definecolor{currentfill}{rgb}{0.725490,0.486275,0.164706}%
\pgfsetfillcolor{currentfill}%
\pgfsetlinewidth{1.003750pt}%
\definecolor{currentstroke}{rgb}{0.266667,0.266667,0.266667}%
\pgfsetstrokecolor{currentstroke}%
\pgfsetdash{}{0pt}%
\pgfpathmoveto{\pgfqpoint{3.085100in}{0.899252in}}%
\pgfpathlineto{\pgfqpoint{3.322580in}{0.899252in}}%
\pgfpathlineto{\pgfqpoint{3.322580in}{1.026974in}}%
\pgfpathlineto{\pgfqpoint{3.085100in}{1.026974in}}%
\pgfpathlineto{\pgfqpoint{3.085100in}{0.899252in}}%
\pgfpathclose%
\pgfusepath{stroke,fill}%
\end{pgfscope}%
\begin{pgfscope}%
\pgfpathrectangle{\pgfqpoint{0.868056in}{0.555556in}}{\pgfqpoint{3.993056in}{1.888889in}}%
\pgfusepath{clip}%
\pgfsetbuttcap%
\pgfsetmiterjoin%
\definecolor{currentfill}{rgb}{0.725490,0.486275,0.164706}%
\pgfsetfillcolor{currentfill}%
\pgfsetlinewidth{1.003750pt}%
\definecolor{currentstroke}{rgb}{0.266667,0.266667,0.266667}%
\pgfsetstrokecolor{currentstroke}%
\pgfsetdash{}{0pt}%
\pgfpathmoveto{\pgfqpoint{3.424357in}{1.067198in}}%
\pgfpathlineto{\pgfqpoint{3.661837in}{1.067198in}}%
\pgfpathlineto{\pgfqpoint{3.661837in}{1.231966in}}%
\pgfpathlineto{\pgfqpoint{3.424357in}{1.231966in}}%
\pgfpathlineto{\pgfqpoint{3.424357in}{1.067198in}}%
\pgfpathclose%
\pgfusepath{stroke,fill}%
\end{pgfscope}%
\begin{pgfscope}%
\pgfpathrectangle{\pgfqpoint{0.868056in}{0.555556in}}{\pgfqpoint{3.993056in}{1.888889in}}%
\pgfusepath{clip}%
\pgfsetbuttcap%
\pgfsetmiterjoin%
\definecolor{currentfill}{rgb}{0.725490,0.486275,0.164706}%
\pgfsetfillcolor{currentfill}%
\pgfsetlinewidth{1.003750pt}%
\definecolor{currentstroke}{rgb}{0.266667,0.266667,0.266667}%
\pgfsetstrokecolor{currentstroke}%
\pgfsetdash{}{0pt}%
\pgfpathmoveto{\pgfqpoint{3.763615in}{1.174187in}}%
\pgfpathlineto{\pgfqpoint{4.001094in}{1.174187in}}%
\pgfpathlineto{\pgfqpoint{4.001094in}{1.388184in}}%
\pgfpathlineto{\pgfqpoint{3.763615in}{1.388184in}}%
\pgfpathlineto{\pgfqpoint{3.763615in}{1.174187in}}%
\pgfpathclose%
\pgfusepath{stroke,fill}%
\end{pgfscope}%
\begin{pgfscope}%
\pgfpathrectangle{\pgfqpoint{0.868056in}{0.555556in}}{\pgfqpoint{3.993056in}{1.888889in}}%
\pgfusepath{clip}%
\pgfsetbuttcap%
\pgfsetmiterjoin%
\definecolor{currentfill}{rgb}{0.725490,0.486275,0.164706}%
\pgfsetfillcolor{currentfill}%
\pgfsetlinewidth{1.003750pt}%
\definecolor{currentstroke}{rgb}{0.266667,0.266667,0.266667}%
\pgfsetstrokecolor{currentstroke}%
\pgfsetdash{}{0pt}%
\pgfpathmoveto{\pgfqpoint{4.102872in}{1.201545in}}%
\pgfpathlineto{\pgfqpoint{4.340352in}{1.201545in}}%
\pgfpathlineto{\pgfqpoint{4.340352in}{1.414010in}}%
\pgfpathlineto{\pgfqpoint{4.102872in}{1.414010in}}%
\pgfpathlineto{\pgfqpoint{4.102872in}{1.201545in}}%
\pgfpathclose%
\pgfusepath{stroke,fill}%
\end{pgfscope}%
\begin{pgfscope}%
\pgfpathrectangle{\pgfqpoint{0.868056in}{0.555556in}}{\pgfqpoint{3.993056in}{1.888889in}}%
\pgfusepath{clip}%
\pgfsetbuttcap%
\pgfsetmiterjoin%
\definecolor{currentfill}{rgb}{0.725490,0.486275,0.164706}%
\pgfsetfillcolor{currentfill}%
\pgfsetlinewidth{1.003750pt}%
\definecolor{currentstroke}{rgb}{0.266667,0.266667,0.266667}%
\pgfsetstrokecolor{currentstroke}%
\pgfsetdash{}{0pt}%
\pgfpathmoveto{\pgfqpoint{4.442129in}{1.352608in}}%
\pgfpathlineto{\pgfqpoint{4.679609in}{1.352608in}}%
\pgfpathlineto{\pgfqpoint{4.679609in}{1.580838in}}%
\pgfpathlineto{\pgfqpoint{4.442129in}{1.580838in}}%
\pgfpathlineto{\pgfqpoint{4.442129in}{1.352608in}}%
\pgfpathclose%
\pgfusepath{stroke,fill}%
\end{pgfscope}%
\begin{pgfscope}%
\pgfpathrectangle{\pgfqpoint{0.868056in}{0.555556in}}{\pgfqpoint{3.993056in}{1.888889in}}%
\pgfusepath{clip}%
\pgfsetbuttcap%
\pgfsetmiterjoin%
\definecolor{currentfill}{rgb}{0.733333,0.321569,0.733333}%
\pgfsetfillcolor{currentfill}%
\pgfsetlinewidth{1.003750pt}%
\definecolor{currentstroke}{rgb}{0.266667,0.266667,0.266667}%
\pgfsetstrokecolor{currentstroke}%
\pgfsetdash{}{0pt}%
\pgfpathmoveto{\pgfqpoint{1.049558in}{1.215819in}}%
\pgfpathlineto{\pgfqpoint{1.287038in}{1.215819in}}%
\pgfpathlineto{\pgfqpoint{1.287038in}{1.368497in}}%
\pgfpathlineto{\pgfqpoint{1.049558in}{1.368497in}}%
\pgfpathlineto{\pgfqpoint{1.049558in}{1.215819in}}%
\pgfpathclose%
\pgfusepath{stroke,fill}%
\end{pgfscope}%
\begin{pgfscope}%
\pgfpathrectangle{\pgfqpoint{0.868056in}{0.555556in}}{\pgfqpoint{3.993056in}{1.888889in}}%
\pgfusepath{clip}%
\pgfsetbuttcap%
\pgfsetmiterjoin%
\definecolor{currentfill}{rgb}{0.733333,0.321569,0.733333}%
\pgfsetfillcolor{currentfill}%
\pgfsetlinewidth{1.003750pt}%
\definecolor{currentstroke}{rgb}{0.266667,0.266667,0.266667}%
\pgfsetstrokecolor{currentstroke}%
\pgfsetdash{}{0pt}%
\pgfpathmoveto{\pgfqpoint{1.388815in}{1.087434in}}%
\pgfpathlineto{\pgfqpoint{1.626295in}{1.087434in}}%
\pgfpathlineto{\pgfqpoint{1.626295in}{1.193729in}}%
\pgfpathlineto{\pgfqpoint{1.388815in}{1.193729in}}%
\pgfpathlineto{\pgfqpoint{1.388815in}{1.087434in}}%
\pgfpathclose%
\pgfusepath{stroke,fill}%
\end{pgfscope}%
\begin{pgfscope}%
\pgfpathrectangle{\pgfqpoint{0.868056in}{0.555556in}}{\pgfqpoint{3.993056in}{1.888889in}}%
\pgfusepath{clip}%
\pgfsetbuttcap%
\pgfsetmiterjoin%
\definecolor{currentfill}{rgb}{0.733333,0.321569,0.733333}%
\pgfsetfillcolor{currentfill}%
\pgfsetlinewidth{1.003750pt}%
\definecolor{currentstroke}{rgb}{0.266667,0.266667,0.266667}%
\pgfsetstrokecolor{currentstroke}%
\pgfsetdash{}{0pt}%
\pgfpathmoveto{\pgfqpoint{1.728072in}{0.937354in}}%
\pgfpathlineto{\pgfqpoint{1.965552in}{0.937354in}}%
\pgfpathlineto{\pgfqpoint{1.965552in}{1.024873in}}%
\pgfpathlineto{\pgfqpoint{1.728072in}{1.024873in}}%
\pgfpathlineto{\pgfqpoint{1.728072in}{0.937354in}}%
\pgfpathclose%
\pgfusepath{stroke,fill}%
\end{pgfscope}%
\begin{pgfscope}%
\pgfpathrectangle{\pgfqpoint{0.868056in}{0.555556in}}{\pgfqpoint{3.993056in}{1.888889in}}%
\pgfusepath{clip}%
\pgfsetbuttcap%
\pgfsetmiterjoin%
\definecolor{currentfill}{rgb}{0.733333,0.321569,0.733333}%
\pgfsetfillcolor{currentfill}%
\pgfsetlinewidth{1.003750pt}%
\definecolor{currentstroke}{rgb}{0.266667,0.266667,0.266667}%
\pgfsetstrokecolor{currentstroke}%
\pgfsetdash{}{0pt}%
\pgfpathmoveto{\pgfqpoint{2.067329in}{1.037449in}}%
\pgfpathlineto{\pgfqpoint{2.304809in}{1.037449in}}%
\pgfpathlineto{\pgfqpoint{2.304809in}{1.134107in}}%
\pgfpathlineto{\pgfqpoint{2.067329in}{1.134107in}}%
\pgfpathlineto{\pgfqpoint{2.067329in}{1.037449in}}%
\pgfpathclose%
\pgfusepath{stroke,fill}%
\end{pgfscope}%
\begin{pgfscope}%
\pgfpathrectangle{\pgfqpoint{0.868056in}{0.555556in}}{\pgfqpoint{3.993056in}{1.888889in}}%
\pgfusepath{clip}%
\pgfsetbuttcap%
\pgfsetmiterjoin%
\definecolor{currentfill}{rgb}{0.733333,0.321569,0.733333}%
\pgfsetfillcolor{currentfill}%
\pgfsetlinewidth{1.003750pt}%
\definecolor{currentstroke}{rgb}{0.266667,0.266667,0.266667}%
\pgfsetstrokecolor{currentstroke}%
\pgfsetdash{}{0pt}%
\pgfpathmoveto{\pgfqpoint{2.406586in}{1.105476in}}%
\pgfpathlineto{\pgfqpoint{2.644066in}{1.105476in}}%
\pgfpathlineto{\pgfqpoint{2.644066in}{1.205757in}}%
\pgfpathlineto{\pgfqpoint{2.406586in}{1.205757in}}%
\pgfpathlineto{\pgfqpoint{2.406586in}{1.105476in}}%
\pgfpathclose%
\pgfusepath{stroke,fill}%
\end{pgfscope}%
\begin{pgfscope}%
\pgfpathrectangle{\pgfqpoint{0.868056in}{0.555556in}}{\pgfqpoint{3.993056in}{1.888889in}}%
\pgfusepath{clip}%
\pgfsetbuttcap%
\pgfsetmiterjoin%
\definecolor{currentfill}{rgb}{0.733333,0.321569,0.733333}%
\pgfsetfillcolor{currentfill}%
\pgfsetlinewidth{1.003750pt}%
\definecolor{currentstroke}{rgb}{0.266667,0.266667,0.266667}%
\pgfsetstrokecolor{currentstroke}%
\pgfsetdash{}{0pt}%
\pgfpathmoveto{\pgfqpoint{2.745843in}{1.207310in}}%
\pgfpathlineto{\pgfqpoint{2.983323in}{1.207310in}}%
\pgfpathlineto{\pgfqpoint{2.983323in}{1.328552in}}%
\pgfpathlineto{\pgfqpoint{2.745843in}{1.328552in}}%
\pgfpathlineto{\pgfqpoint{2.745843in}{1.207310in}}%
\pgfpathclose%
\pgfusepath{stroke,fill}%
\end{pgfscope}%
\begin{pgfscope}%
\pgfpathrectangle{\pgfqpoint{0.868056in}{0.555556in}}{\pgfqpoint{3.993056in}{1.888889in}}%
\pgfusepath{clip}%
\pgfsetbuttcap%
\pgfsetmiterjoin%
\definecolor{currentfill}{rgb}{0.733333,0.321569,0.733333}%
\pgfsetfillcolor{currentfill}%
\pgfsetlinewidth{1.003750pt}%
\definecolor{currentstroke}{rgb}{0.266667,0.266667,0.266667}%
\pgfsetstrokecolor{currentstroke}%
\pgfsetdash{}{0pt}%
\pgfpathmoveto{\pgfqpoint{3.085100in}{1.026974in}}%
\pgfpathlineto{\pgfqpoint{3.322580in}{1.026974in}}%
\pgfpathlineto{\pgfqpoint{3.322580in}{1.132834in}}%
\pgfpathlineto{\pgfqpoint{3.085100in}{1.132834in}}%
\pgfpathlineto{\pgfqpoint{3.085100in}{1.026974in}}%
\pgfpathclose%
\pgfusepath{stroke,fill}%
\end{pgfscope}%
\begin{pgfscope}%
\pgfpathrectangle{\pgfqpoint{0.868056in}{0.555556in}}{\pgfqpoint{3.993056in}{1.888889in}}%
\pgfusepath{clip}%
\pgfsetbuttcap%
\pgfsetmiterjoin%
\definecolor{currentfill}{rgb}{0.733333,0.321569,0.733333}%
\pgfsetfillcolor{currentfill}%
\pgfsetlinewidth{1.003750pt}%
\definecolor{currentstroke}{rgb}{0.266667,0.266667,0.266667}%
\pgfsetstrokecolor{currentstroke}%
\pgfsetdash{}{0pt}%
\pgfpathmoveto{\pgfqpoint{3.424357in}{1.231966in}}%
\pgfpathlineto{\pgfqpoint{3.661837in}{1.231966in}}%
\pgfpathlineto{\pgfqpoint{3.661837in}{1.404270in}}%
\pgfpathlineto{\pgfqpoint{3.424357in}{1.404270in}}%
\pgfpathlineto{\pgfqpoint{3.424357in}{1.231966in}}%
\pgfpathclose%
\pgfusepath{stroke,fill}%
\end{pgfscope}%
\begin{pgfscope}%
\pgfpathrectangle{\pgfqpoint{0.868056in}{0.555556in}}{\pgfqpoint{3.993056in}{1.888889in}}%
\pgfusepath{clip}%
\pgfsetbuttcap%
\pgfsetmiterjoin%
\definecolor{currentfill}{rgb}{0.733333,0.321569,0.733333}%
\pgfsetfillcolor{currentfill}%
\pgfsetlinewidth{1.003750pt}%
\definecolor{currentstroke}{rgb}{0.266667,0.266667,0.266667}%
\pgfsetstrokecolor{currentstroke}%
\pgfsetdash{}{0pt}%
\pgfpathmoveto{\pgfqpoint{3.763615in}{1.388184in}}%
\pgfpathlineto{\pgfqpoint{4.001094in}{1.388184in}}%
\pgfpathlineto{\pgfqpoint{4.001094in}{1.577909in}}%
\pgfpathlineto{\pgfqpoint{3.763615in}{1.577909in}}%
\pgfpathlineto{\pgfqpoint{3.763615in}{1.388184in}}%
\pgfpathclose%
\pgfusepath{stroke,fill}%
\end{pgfscope}%
\begin{pgfscope}%
\pgfpathrectangle{\pgfqpoint{0.868056in}{0.555556in}}{\pgfqpoint{3.993056in}{1.888889in}}%
\pgfusepath{clip}%
\pgfsetbuttcap%
\pgfsetmiterjoin%
\definecolor{currentfill}{rgb}{0.733333,0.321569,0.733333}%
\pgfsetfillcolor{currentfill}%
\pgfsetlinewidth{1.003750pt}%
\definecolor{currentstroke}{rgb}{0.266667,0.266667,0.266667}%
\pgfsetstrokecolor{currentstroke}%
\pgfsetdash{}{0pt}%
\pgfpathmoveto{\pgfqpoint{4.102872in}{1.414010in}}%
\pgfpathlineto{\pgfqpoint{4.340352in}{1.414010in}}%
\pgfpathlineto{\pgfqpoint{4.340352in}{1.615100in}}%
\pgfpathlineto{\pgfqpoint{4.102872in}{1.615100in}}%
\pgfpathlineto{\pgfqpoint{4.102872in}{1.414010in}}%
\pgfpathclose%
\pgfusepath{stroke,fill}%
\end{pgfscope}%
\begin{pgfscope}%
\pgfpathrectangle{\pgfqpoint{0.868056in}{0.555556in}}{\pgfqpoint{3.993056in}{1.888889in}}%
\pgfusepath{clip}%
\pgfsetbuttcap%
\pgfsetmiterjoin%
\definecolor{currentfill}{rgb}{0.733333,0.321569,0.733333}%
\pgfsetfillcolor{currentfill}%
\pgfsetlinewidth{1.003750pt}%
\definecolor{currentstroke}{rgb}{0.266667,0.266667,0.266667}%
\pgfsetstrokecolor{currentstroke}%
\pgfsetdash{}{0pt}%
\pgfpathmoveto{\pgfqpoint{4.442129in}{1.580838in}}%
\pgfpathlineto{\pgfqpoint{4.679609in}{1.580838in}}%
\pgfpathlineto{\pgfqpoint{4.679609in}{1.788708in}}%
\pgfpathlineto{\pgfqpoint{4.442129in}{1.788708in}}%
\pgfpathlineto{\pgfqpoint{4.442129in}{1.580838in}}%
\pgfpathclose%
\pgfusepath{stroke,fill}%
\end{pgfscope}%
\begin{pgfscope}%
\pgfpathrectangle{\pgfqpoint{0.868056in}{0.555556in}}{\pgfqpoint{3.993056in}{1.888889in}}%
\pgfusepath{clip}%
\pgfsetbuttcap%
\pgfsetmiterjoin%
\definecolor{currentfill}{rgb}{0.549020,0.247059,0.121569}%
\pgfsetfillcolor{currentfill}%
\pgfsetlinewidth{1.003750pt}%
\definecolor{currentstroke}{rgb}{0.266667,0.266667,0.266667}%
\pgfsetstrokecolor{currentstroke}%
\pgfsetdash{}{0pt}%
\pgfpathmoveto{\pgfqpoint{1.049558in}{1.368497in}}%
\pgfpathlineto{\pgfqpoint{1.287038in}{1.368497in}}%
\pgfpathlineto{\pgfqpoint{1.287038in}{1.623071in}}%
\pgfpathlineto{\pgfqpoint{1.049558in}{1.623071in}}%
\pgfpathlineto{\pgfqpoint{1.049558in}{1.368497in}}%
\pgfpathclose%
\pgfusepath{stroke,fill}%
\end{pgfscope}%
\begin{pgfscope}%
\pgfpathrectangle{\pgfqpoint{0.868056in}{0.555556in}}{\pgfqpoint{3.993056in}{1.888889in}}%
\pgfusepath{clip}%
\pgfsetbuttcap%
\pgfsetmiterjoin%
\definecolor{currentfill}{rgb}{0.549020,0.247059,0.121569}%
\pgfsetfillcolor{currentfill}%
\pgfsetlinewidth{1.003750pt}%
\definecolor{currentstroke}{rgb}{0.266667,0.266667,0.266667}%
\pgfsetstrokecolor{currentstroke}%
\pgfsetdash{}{0pt}%
\pgfpathmoveto{\pgfqpoint{1.388815in}{1.193729in}}%
\pgfpathlineto{\pgfqpoint{1.626295in}{1.193729in}}%
\pgfpathlineto{\pgfqpoint{1.626295in}{1.369159in}}%
\pgfpathlineto{\pgfqpoint{1.388815in}{1.369159in}}%
\pgfpathlineto{\pgfqpoint{1.388815in}{1.193729in}}%
\pgfpathclose%
\pgfusepath{stroke,fill}%
\end{pgfscope}%
\begin{pgfscope}%
\pgfpathrectangle{\pgfqpoint{0.868056in}{0.555556in}}{\pgfqpoint{3.993056in}{1.888889in}}%
\pgfusepath{clip}%
\pgfsetbuttcap%
\pgfsetmiterjoin%
\definecolor{currentfill}{rgb}{0.549020,0.247059,0.121569}%
\pgfsetfillcolor{currentfill}%
\pgfsetlinewidth{1.003750pt}%
\definecolor{currentstroke}{rgb}{0.266667,0.266667,0.266667}%
\pgfsetstrokecolor{currentstroke}%
\pgfsetdash{}{0pt}%
\pgfpathmoveto{\pgfqpoint{1.728072in}{1.024873in}}%
\pgfpathlineto{\pgfqpoint{1.965552in}{1.024873in}}%
\pgfpathlineto{\pgfqpoint{1.965552in}{1.145411in}}%
\pgfpathlineto{\pgfqpoint{1.728072in}{1.145411in}}%
\pgfpathlineto{\pgfqpoint{1.728072in}{1.024873in}}%
\pgfpathclose%
\pgfusepath{stroke,fill}%
\end{pgfscope}%
\begin{pgfscope}%
\pgfpathrectangle{\pgfqpoint{0.868056in}{0.555556in}}{\pgfqpoint{3.993056in}{1.888889in}}%
\pgfusepath{clip}%
\pgfsetbuttcap%
\pgfsetmiterjoin%
\definecolor{currentfill}{rgb}{0.549020,0.247059,0.121569}%
\pgfsetfillcolor{currentfill}%
\pgfsetlinewidth{1.003750pt}%
\definecolor{currentstroke}{rgb}{0.266667,0.266667,0.266667}%
\pgfsetstrokecolor{currentstroke}%
\pgfsetdash{}{0pt}%
\pgfpathmoveto{\pgfqpoint{2.067329in}{1.134107in}}%
\pgfpathlineto{\pgfqpoint{2.304809in}{1.134107in}}%
\pgfpathlineto{\pgfqpoint{2.304809in}{1.266715in}}%
\pgfpathlineto{\pgfqpoint{2.067329in}{1.266715in}}%
\pgfpathlineto{\pgfqpoint{2.067329in}{1.134107in}}%
\pgfpathclose%
\pgfusepath{stroke,fill}%
\end{pgfscope}%
\begin{pgfscope}%
\pgfpathrectangle{\pgfqpoint{0.868056in}{0.555556in}}{\pgfqpoint{3.993056in}{1.888889in}}%
\pgfusepath{clip}%
\pgfsetbuttcap%
\pgfsetmiterjoin%
\definecolor{currentfill}{rgb}{0.549020,0.247059,0.121569}%
\pgfsetfillcolor{currentfill}%
\pgfsetlinewidth{1.003750pt}%
\definecolor{currentstroke}{rgb}{0.266667,0.266667,0.266667}%
\pgfsetstrokecolor{currentstroke}%
\pgfsetdash{}{0pt}%
\pgfpathmoveto{\pgfqpoint{2.406586in}{1.205757in}}%
\pgfpathlineto{\pgfqpoint{2.644066in}{1.205757in}}%
\pgfpathlineto{\pgfqpoint{2.644066in}{1.337671in}}%
\pgfpathlineto{\pgfqpoint{2.406586in}{1.337671in}}%
\pgfpathlineto{\pgfqpoint{2.406586in}{1.205757in}}%
\pgfpathclose%
\pgfusepath{stroke,fill}%
\end{pgfscope}%
\begin{pgfscope}%
\pgfpathrectangle{\pgfqpoint{0.868056in}{0.555556in}}{\pgfqpoint{3.993056in}{1.888889in}}%
\pgfusepath{clip}%
\pgfsetbuttcap%
\pgfsetmiterjoin%
\definecolor{currentfill}{rgb}{0.549020,0.247059,0.121569}%
\pgfsetfillcolor{currentfill}%
\pgfsetlinewidth{1.003750pt}%
\definecolor{currentstroke}{rgb}{0.266667,0.266667,0.266667}%
\pgfsetstrokecolor{currentstroke}%
\pgfsetdash{}{0pt}%
\pgfpathmoveto{\pgfqpoint{2.745843in}{1.328552in}}%
\pgfpathlineto{\pgfqpoint{2.983323in}{1.328552in}}%
\pgfpathlineto{\pgfqpoint{2.983323in}{1.502926in}}%
\pgfpathlineto{\pgfqpoint{2.745843in}{1.502926in}}%
\pgfpathlineto{\pgfqpoint{2.745843in}{1.328552in}}%
\pgfpathclose%
\pgfusepath{stroke,fill}%
\end{pgfscope}%
\begin{pgfscope}%
\pgfpathrectangle{\pgfqpoint{0.868056in}{0.555556in}}{\pgfqpoint{3.993056in}{1.888889in}}%
\pgfusepath{clip}%
\pgfsetbuttcap%
\pgfsetmiterjoin%
\definecolor{currentfill}{rgb}{0.549020,0.247059,0.121569}%
\pgfsetfillcolor{currentfill}%
\pgfsetlinewidth{1.003750pt}%
\definecolor{currentstroke}{rgb}{0.266667,0.266667,0.266667}%
\pgfsetstrokecolor{currentstroke}%
\pgfsetdash{}{0pt}%
\pgfpathmoveto{\pgfqpoint{3.085100in}{1.132834in}}%
\pgfpathlineto{\pgfqpoint{3.322580in}{1.132834in}}%
\pgfpathlineto{\pgfqpoint{3.322580in}{1.266239in}}%
\pgfpathlineto{\pgfqpoint{3.085100in}{1.266239in}}%
\pgfpathlineto{\pgfqpoint{3.085100in}{1.132834in}}%
\pgfpathclose%
\pgfusepath{stroke,fill}%
\end{pgfscope}%
\begin{pgfscope}%
\pgfpathrectangle{\pgfqpoint{0.868056in}{0.555556in}}{\pgfqpoint{3.993056in}{1.888889in}}%
\pgfusepath{clip}%
\pgfsetbuttcap%
\pgfsetmiterjoin%
\definecolor{currentfill}{rgb}{0.549020,0.247059,0.121569}%
\pgfsetfillcolor{currentfill}%
\pgfsetlinewidth{1.003750pt}%
\definecolor{currentstroke}{rgb}{0.266667,0.266667,0.266667}%
\pgfsetstrokecolor{currentstroke}%
\pgfsetdash{}{0pt}%
\pgfpathmoveto{\pgfqpoint{3.424357in}{1.404270in}}%
\pgfpathlineto{\pgfqpoint{3.661837in}{1.404270in}}%
\pgfpathlineto{\pgfqpoint{3.661837in}{1.517324in}}%
\pgfpathlineto{\pgfqpoint{3.424357in}{1.517324in}}%
\pgfpathlineto{\pgfqpoint{3.424357in}{1.404270in}}%
\pgfpathclose%
\pgfusepath{stroke,fill}%
\end{pgfscope}%
\begin{pgfscope}%
\pgfpathrectangle{\pgfqpoint{0.868056in}{0.555556in}}{\pgfqpoint{3.993056in}{1.888889in}}%
\pgfusepath{clip}%
\pgfsetbuttcap%
\pgfsetmiterjoin%
\definecolor{currentfill}{rgb}{0.549020,0.247059,0.121569}%
\pgfsetfillcolor{currentfill}%
\pgfsetlinewidth{1.003750pt}%
\definecolor{currentstroke}{rgb}{0.266667,0.266667,0.266667}%
\pgfsetstrokecolor{currentstroke}%
\pgfsetdash{}{0pt}%
\pgfpathmoveto{\pgfqpoint{3.763615in}{1.577909in}}%
\pgfpathlineto{\pgfqpoint{4.001094in}{1.577909in}}%
\pgfpathlineto{\pgfqpoint{4.001094in}{1.728341in}}%
\pgfpathlineto{\pgfqpoint{3.763615in}{1.728341in}}%
\pgfpathlineto{\pgfqpoint{3.763615in}{1.577909in}}%
\pgfpathclose%
\pgfusepath{stroke,fill}%
\end{pgfscope}%
\begin{pgfscope}%
\pgfpathrectangle{\pgfqpoint{0.868056in}{0.555556in}}{\pgfqpoint{3.993056in}{1.888889in}}%
\pgfusepath{clip}%
\pgfsetbuttcap%
\pgfsetmiterjoin%
\definecolor{currentfill}{rgb}{0.549020,0.247059,0.121569}%
\pgfsetfillcolor{currentfill}%
\pgfsetlinewidth{1.003750pt}%
\definecolor{currentstroke}{rgb}{0.266667,0.266667,0.266667}%
\pgfsetstrokecolor{currentstroke}%
\pgfsetdash{}{0pt}%
\pgfpathmoveto{\pgfqpoint{4.102872in}{1.615100in}}%
\pgfpathlineto{\pgfqpoint{4.340352in}{1.615100in}}%
\pgfpathlineto{\pgfqpoint{4.340352in}{1.781204in}}%
\pgfpathlineto{\pgfqpoint{4.102872in}{1.781204in}}%
\pgfpathlineto{\pgfqpoint{4.102872in}{1.615100in}}%
\pgfpathclose%
\pgfusepath{stroke,fill}%
\end{pgfscope}%
\begin{pgfscope}%
\pgfpathrectangle{\pgfqpoint{0.868056in}{0.555556in}}{\pgfqpoint{3.993056in}{1.888889in}}%
\pgfusepath{clip}%
\pgfsetbuttcap%
\pgfsetmiterjoin%
\definecolor{currentfill}{rgb}{0.549020,0.247059,0.121569}%
\pgfsetfillcolor{currentfill}%
\pgfsetlinewidth{1.003750pt}%
\definecolor{currentstroke}{rgb}{0.266667,0.266667,0.266667}%
\pgfsetstrokecolor{currentstroke}%
\pgfsetdash{}{0pt}%
\pgfpathmoveto{\pgfqpoint{4.442129in}{1.788708in}}%
\pgfpathlineto{\pgfqpoint{4.679609in}{1.788708in}}%
\pgfpathlineto{\pgfqpoint{4.679609in}{1.949605in}}%
\pgfpathlineto{\pgfqpoint{4.442129in}{1.949605in}}%
\pgfpathlineto{\pgfqpoint{4.442129in}{1.788708in}}%
\pgfpathclose%
\pgfusepath{stroke,fill}%
\end{pgfscope}%
\begin{pgfscope}%
\pgfpathrectangle{\pgfqpoint{0.868056in}{0.555556in}}{\pgfqpoint{3.993056in}{1.888889in}}%
\pgfusepath{clip}%
\pgfsetbuttcap%
\pgfsetmiterjoin%
\definecolor{currentfill}{rgb}{0.701961,0.760784,0.360784}%
\pgfsetfillcolor{currentfill}%
\pgfsetlinewidth{1.003750pt}%
\definecolor{currentstroke}{rgb}{0.266667,0.266667,0.266667}%
\pgfsetstrokecolor{currentstroke}%
\pgfsetdash{}{0pt}%
\pgfpathmoveto{\pgfqpoint{1.049558in}{1.623071in}}%
\pgfpathlineto{\pgfqpoint{1.287038in}{1.623071in}}%
\pgfpathlineto{\pgfqpoint{1.287038in}{1.798014in}}%
\pgfpathlineto{\pgfqpoint{1.049558in}{1.798014in}}%
\pgfpathlineto{\pgfqpoint{1.049558in}{1.623071in}}%
\pgfpathclose%
\pgfusepath{stroke,fill}%
\end{pgfscope}%
\begin{pgfscope}%
\pgfpathrectangle{\pgfqpoint{0.868056in}{0.555556in}}{\pgfqpoint{3.993056in}{1.888889in}}%
\pgfusepath{clip}%
\pgfsetbuttcap%
\pgfsetmiterjoin%
\definecolor{currentfill}{rgb}{0.701961,0.760784,0.360784}%
\pgfsetfillcolor{currentfill}%
\pgfsetlinewidth{1.003750pt}%
\definecolor{currentstroke}{rgb}{0.266667,0.266667,0.266667}%
\pgfsetstrokecolor{currentstroke}%
\pgfsetdash{}{0pt}%
\pgfpathmoveto{\pgfqpoint{1.388815in}{1.369159in}}%
\pgfpathlineto{\pgfqpoint{1.626295in}{1.369159in}}%
\pgfpathlineto{\pgfqpoint{1.626295in}{1.461294in}}%
\pgfpathlineto{\pgfqpoint{1.388815in}{1.461294in}}%
\pgfpathlineto{\pgfqpoint{1.388815in}{1.369159in}}%
\pgfpathclose%
\pgfusepath{stroke,fill}%
\end{pgfscope}%
\begin{pgfscope}%
\pgfpathrectangle{\pgfqpoint{0.868056in}{0.555556in}}{\pgfqpoint{3.993056in}{1.888889in}}%
\pgfusepath{clip}%
\pgfsetbuttcap%
\pgfsetmiterjoin%
\definecolor{currentfill}{rgb}{0.701961,0.760784,0.360784}%
\pgfsetfillcolor{currentfill}%
\pgfsetlinewidth{1.003750pt}%
\definecolor{currentstroke}{rgb}{0.266667,0.266667,0.266667}%
\pgfsetstrokecolor{currentstroke}%
\pgfsetdash{}{0pt}%
\pgfpathmoveto{\pgfqpoint{1.728072in}{1.145411in}}%
\pgfpathlineto{\pgfqpoint{1.965552in}{1.145411in}}%
\pgfpathlineto{\pgfqpoint{1.965552in}{1.213707in}}%
\pgfpathlineto{\pgfqpoint{1.728072in}{1.213707in}}%
\pgfpathlineto{\pgfqpoint{1.728072in}{1.145411in}}%
\pgfpathclose%
\pgfusepath{stroke,fill}%
\end{pgfscope}%
\begin{pgfscope}%
\pgfpathrectangle{\pgfqpoint{0.868056in}{0.555556in}}{\pgfqpoint{3.993056in}{1.888889in}}%
\pgfusepath{clip}%
\pgfsetbuttcap%
\pgfsetmiterjoin%
\definecolor{currentfill}{rgb}{0.701961,0.760784,0.360784}%
\pgfsetfillcolor{currentfill}%
\pgfsetlinewidth{1.003750pt}%
\definecolor{currentstroke}{rgb}{0.266667,0.266667,0.266667}%
\pgfsetstrokecolor{currentstroke}%
\pgfsetdash{}{0pt}%
\pgfpathmoveto{\pgfqpoint{2.067329in}{1.266715in}}%
\pgfpathlineto{\pgfqpoint{2.304809in}{1.266715in}}%
\pgfpathlineto{\pgfqpoint{2.304809in}{1.350786in}}%
\pgfpathlineto{\pgfqpoint{2.067329in}{1.350786in}}%
\pgfpathlineto{\pgfqpoint{2.067329in}{1.266715in}}%
\pgfpathclose%
\pgfusepath{stroke,fill}%
\end{pgfscope}%
\begin{pgfscope}%
\pgfpathrectangle{\pgfqpoint{0.868056in}{0.555556in}}{\pgfqpoint{3.993056in}{1.888889in}}%
\pgfusepath{clip}%
\pgfsetbuttcap%
\pgfsetmiterjoin%
\definecolor{currentfill}{rgb}{0.701961,0.760784,0.360784}%
\pgfsetfillcolor{currentfill}%
\pgfsetlinewidth{1.003750pt}%
\definecolor{currentstroke}{rgb}{0.266667,0.266667,0.266667}%
\pgfsetstrokecolor{currentstroke}%
\pgfsetdash{}{0pt}%
\pgfpathmoveto{\pgfqpoint{2.406586in}{1.337671in}}%
\pgfpathlineto{\pgfqpoint{2.644066in}{1.337671in}}%
\pgfpathlineto{\pgfqpoint{2.644066in}{1.423927in}}%
\pgfpathlineto{\pgfqpoint{2.406586in}{1.423927in}}%
\pgfpathlineto{\pgfqpoint{2.406586in}{1.337671in}}%
\pgfpathclose%
\pgfusepath{stroke,fill}%
\end{pgfscope}%
\begin{pgfscope}%
\pgfpathrectangle{\pgfqpoint{0.868056in}{0.555556in}}{\pgfqpoint{3.993056in}{1.888889in}}%
\pgfusepath{clip}%
\pgfsetbuttcap%
\pgfsetmiterjoin%
\definecolor{currentfill}{rgb}{0.701961,0.760784,0.360784}%
\pgfsetfillcolor{currentfill}%
\pgfsetlinewidth{1.003750pt}%
\definecolor{currentstroke}{rgb}{0.266667,0.266667,0.266667}%
\pgfsetstrokecolor{currentstroke}%
\pgfsetdash{}{0pt}%
\pgfpathmoveto{\pgfqpoint{2.745843in}{1.502926in}}%
\pgfpathlineto{\pgfqpoint{2.983323in}{1.502926in}}%
\pgfpathlineto{\pgfqpoint{2.983323in}{1.610256in}}%
\pgfpathlineto{\pgfqpoint{2.745843in}{1.610256in}}%
\pgfpathlineto{\pgfqpoint{2.745843in}{1.502926in}}%
\pgfpathclose%
\pgfusepath{stroke,fill}%
\end{pgfscope}%
\begin{pgfscope}%
\pgfpathrectangle{\pgfqpoint{0.868056in}{0.555556in}}{\pgfqpoint{3.993056in}{1.888889in}}%
\pgfusepath{clip}%
\pgfsetbuttcap%
\pgfsetmiterjoin%
\definecolor{currentfill}{rgb}{0.701961,0.760784,0.360784}%
\pgfsetfillcolor{currentfill}%
\pgfsetlinewidth{1.003750pt}%
\definecolor{currentstroke}{rgb}{0.266667,0.266667,0.266667}%
\pgfsetstrokecolor{currentstroke}%
\pgfsetdash{}{0pt}%
\pgfpathmoveto{\pgfqpoint{3.085100in}{1.266239in}}%
\pgfpathlineto{\pgfqpoint{3.322580in}{1.266239in}}%
\pgfpathlineto{\pgfqpoint{3.322580in}{1.356852in}}%
\pgfpathlineto{\pgfqpoint{3.085100in}{1.356852in}}%
\pgfpathlineto{\pgfqpoint{3.085100in}{1.266239in}}%
\pgfpathclose%
\pgfusepath{stroke,fill}%
\end{pgfscope}%
\begin{pgfscope}%
\pgfpathrectangle{\pgfqpoint{0.868056in}{0.555556in}}{\pgfqpoint{3.993056in}{1.888889in}}%
\pgfusepath{clip}%
\pgfsetbuttcap%
\pgfsetmiterjoin%
\definecolor{currentfill}{rgb}{0.701961,0.760784,0.360784}%
\pgfsetfillcolor{currentfill}%
\pgfsetlinewidth{1.003750pt}%
\definecolor{currentstroke}{rgb}{0.266667,0.266667,0.266667}%
\pgfsetstrokecolor{currentstroke}%
\pgfsetdash{}{0pt}%
\pgfpathmoveto{\pgfqpoint{3.424357in}{1.517324in}}%
\pgfpathlineto{\pgfqpoint{3.661837in}{1.517324in}}%
\pgfpathlineto{\pgfqpoint{3.661837in}{1.626373in}}%
\pgfpathlineto{\pgfqpoint{3.424357in}{1.626373in}}%
\pgfpathlineto{\pgfqpoint{3.424357in}{1.517324in}}%
\pgfpathclose%
\pgfusepath{stroke,fill}%
\end{pgfscope}%
\begin{pgfscope}%
\pgfpathrectangle{\pgfqpoint{0.868056in}{0.555556in}}{\pgfqpoint{3.993056in}{1.888889in}}%
\pgfusepath{clip}%
\pgfsetbuttcap%
\pgfsetmiterjoin%
\definecolor{currentfill}{rgb}{0.701961,0.760784,0.360784}%
\pgfsetfillcolor{currentfill}%
\pgfsetlinewidth{1.003750pt}%
\definecolor{currentstroke}{rgb}{0.266667,0.266667,0.266667}%
\pgfsetstrokecolor{currentstroke}%
\pgfsetdash{}{0pt}%
\pgfpathmoveto{\pgfqpoint{3.763615in}{1.728341in}}%
\pgfpathlineto{\pgfqpoint{4.001094in}{1.728341in}}%
\pgfpathlineto{\pgfqpoint{4.001094in}{1.843455in}}%
\pgfpathlineto{\pgfqpoint{3.763615in}{1.843455in}}%
\pgfpathlineto{\pgfqpoint{3.763615in}{1.728341in}}%
\pgfpathclose%
\pgfusepath{stroke,fill}%
\end{pgfscope}%
\begin{pgfscope}%
\pgfpathrectangle{\pgfqpoint{0.868056in}{0.555556in}}{\pgfqpoint{3.993056in}{1.888889in}}%
\pgfusepath{clip}%
\pgfsetbuttcap%
\pgfsetmiterjoin%
\definecolor{currentfill}{rgb}{0.701961,0.760784,0.360784}%
\pgfsetfillcolor{currentfill}%
\pgfsetlinewidth{1.003750pt}%
\definecolor{currentstroke}{rgb}{0.266667,0.266667,0.266667}%
\pgfsetstrokecolor{currentstroke}%
\pgfsetdash{}{0pt}%
\pgfpathmoveto{\pgfqpoint{4.102872in}{1.781204in}}%
\pgfpathlineto{\pgfqpoint{4.340352in}{1.781204in}}%
\pgfpathlineto{\pgfqpoint{4.340352in}{1.906731in}}%
\pgfpathlineto{\pgfqpoint{4.102872in}{1.906731in}}%
\pgfpathlineto{\pgfqpoint{4.102872in}{1.781204in}}%
\pgfpathclose%
\pgfusepath{stroke,fill}%
\end{pgfscope}%
\begin{pgfscope}%
\pgfpathrectangle{\pgfqpoint{0.868056in}{0.555556in}}{\pgfqpoint{3.993056in}{1.888889in}}%
\pgfusepath{clip}%
\pgfsetbuttcap%
\pgfsetmiterjoin%
\definecolor{currentfill}{rgb}{0.701961,0.760784,0.360784}%
\pgfsetfillcolor{currentfill}%
\pgfsetlinewidth{1.003750pt}%
\definecolor{currentstroke}{rgb}{0.266667,0.266667,0.266667}%
\pgfsetstrokecolor{currentstroke}%
\pgfsetdash{}{0pt}%
\pgfpathmoveto{\pgfqpoint{4.442129in}{1.949605in}}%
\pgfpathlineto{\pgfqpoint{4.679609in}{1.949605in}}%
\pgfpathlineto{\pgfqpoint{4.679609in}{2.086767in}}%
\pgfpathlineto{\pgfqpoint{4.442129in}{2.086767in}}%
\pgfpathlineto{\pgfqpoint{4.442129in}{1.949605in}}%
\pgfpathclose%
\pgfusepath{stroke,fill}%
\end{pgfscope}%
\begin{pgfscope}%
\pgfpathrectangle{\pgfqpoint{0.868056in}{0.555556in}}{\pgfqpoint{3.993056in}{1.888889in}}%
\pgfusepath{clip}%
\pgfsetbuttcap%
\pgfsetmiterjoin%
\definecolor{currentfill}{rgb}{0.447059,0.447059,0.447059}%
\pgfsetfillcolor{currentfill}%
\pgfsetlinewidth{1.003750pt}%
\definecolor{currentstroke}{rgb}{0.266667,0.266667,0.266667}%
\pgfsetstrokecolor{currentstroke}%
\pgfsetdash{}{0pt}%
\pgfpathmoveto{\pgfqpoint{1.049558in}{1.798014in}}%
\pgfpathlineto{\pgfqpoint{1.287038in}{1.798014in}}%
\pgfpathlineto{\pgfqpoint{1.287038in}{1.972730in}}%
\pgfpathlineto{\pgfqpoint{1.049558in}{1.972730in}}%
\pgfpathlineto{\pgfqpoint{1.049558in}{1.798014in}}%
\pgfpathclose%
\pgfusepath{stroke,fill}%
\end{pgfscope}%
\begin{pgfscope}%
\pgfpathrectangle{\pgfqpoint{0.868056in}{0.555556in}}{\pgfqpoint{3.993056in}{1.888889in}}%
\pgfusepath{clip}%
\pgfsetbuttcap%
\pgfsetmiterjoin%
\definecolor{currentfill}{rgb}{0.447059,0.447059,0.447059}%
\pgfsetfillcolor{currentfill}%
\pgfsetlinewidth{1.003750pt}%
\definecolor{currentstroke}{rgb}{0.266667,0.266667,0.266667}%
\pgfsetstrokecolor{currentstroke}%
\pgfsetdash{}{0pt}%
\pgfpathmoveto{\pgfqpoint{1.388815in}{1.461294in}}%
\pgfpathlineto{\pgfqpoint{1.626295in}{1.461294in}}%
\pgfpathlineto{\pgfqpoint{1.626295in}{1.577816in}}%
\pgfpathlineto{\pgfqpoint{1.388815in}{1.577816in}}%
\pgfpathlineto{\pgfqpoint{1.388815in}{1.461294in}}%
\pgfpathclose%
\pgfusepath{stroke,fill}%
\end{pgfscope}%
\begin{pgfscope}%
\pgfpathrectangle{\pgfqpoint{0.868056in}{0.555556in}}{\pgfqpoint{3.993056in}{1.888889in}}%
\pgfusepath{clip}%
\pgfsetbuttcap%
\pgfsetmiterjoin%
\definecolor{currentfill}{rgb}{0.447059,0.447059,0.447059}%
\pgfsetfillcolor{currentfill}%
\pgfsetlinewidth{1.003750pt}%
\definecolor{currentstroke}{rgb}{0.266667,0.266667,0.266667}%
\pgfsetstrokecolor{currentstroke}%
\pgfsetdash{}{0pt}%
\pgfpathmoveto{\pgfqpoint{1.728072in}{1.213707in}}%
\pgfpathlineto{\pgfqpoint{1.965552in}{1.213707in}}%
\pgfpathlineto{\pgfqpoint{1.965552in}{1.317300in}}%
\pgfpathlineto{\pgfqpoint{1.728072in}{1.317300in}}%
\pgfpathlineto{\pgfqpoint{1.728072in}{1.213707in}}%
\pgfpathclose%
\pgfusepath{stroke,fill}%
\end{pgfscope}%
\begin{pgfscope}%
\pgfpathrectangle{\pgfqpoint{0.868056in}{0.555556in}}{\pgfqpoint{3.993056in}{1.888889in}}%
\pgfusepath{clip}%
\pgfsetbuttcap%
\pgfsetmiterjoin%
\definecolor{currentfill}{rgb}{0.447059,0.447059,0.447059}%
\pgfsetfillcolor{currentfill}%
\pgfsetlinewidth{1.003750pt}%
\definecolor{currentstroke}{rgb}{0.266667,0.266667,0.266667}%
\pgfsetstrokecolor{currentstroke}%
\pgfsetdash{}{0pt}%
\pgfpathmoveto{\pgfqpoint{2.067329in}{1.350786in}}%
\pgfpathlineto{\pgfqpoint{2.304809in}{1.350786in}}%
\pgfpathlineto{\pgfqpoint{2.304809in}{1.430903in}}%
\pgfpathlineto{\pgfqpoint{2.067329in}{1.430903in}}%
\pgfpathlineto{\pgfqpoint{2.067329in}{1.350786in}}%
\pgfpathclose%
\pgfusepath{stroke,fill}%
\end{pgfscope}%
\begin{pgfscope}%
\pgfpathrectangle{\pgfqpoint{0.868056in}{0.555556in}}{\pgfqpoint{3.993056in}{1.888889in}}%
\pgfusepath{clip}%
\pgfsetbuttcap%
\pgfsetmiterjoin%
\definecolor{currentfill}{rgb}{0.447059,0.447059,0.447059}%
\pgfsetfillcolor{currentfill}%
\pgfsetlinewidth{1.003750pt}%
\definecolor{currentstroke}{rgb}{0.266667,0.266667,0.266667}%
\pgfsetstrokecolor{currentstroke}%
\pgfsetdash{}{0pt}%
\pgfpathmoveto{\pgfqpoint{2.406586in}{1.423927in}}%
\pgfpathlineto{\pgfqpoint{2.644066in}{1.423927in}}%
\pgfpathlineto{\pgfqpoint{2.644066in}{1.517583in}}%
\pgfpathlineto{\pgfqpoint{2.406586in}{1.517583in}}%
\pgfpathlineto{\pgfqpoint{2.406586in}{1.423927in}}%
\pgfpathclose%
\pgfusepath{stroke,fill}%
\end{pgfscope}%
\begin{pgfscope}%
\pgfpathrectangle{\pgfqpoint{0.868056in}{0.555556in}}{\pgfqpoint{3.993056in}{1.888889in}}%
\pgfusepath{clip}%
\pgfsetbuttcap%
\pgfsetmiterjoin%
\definecolor{currentfill}{rgb}{0.447059,0.447059,0.447059}%
\pgfsetfillcolor{currentfill}%
\pgfsetlinewidth{1.003750pt}%
\definecolor{currentstroke}{rgb}{0.266667,0.266667,0.266667}%
\pgfsetstrokecolor{currentstroke}%
\pgfsetdash{}{0pt}%
\pgfpathmoveto{\pgfqpoint{2.745843in}{1.610256in}}%
\pgfpathlineto{\pgfqpoint{2.983323in}{1.610256in}}%
\pgfpathlineto{\pgfqpoint{2.983323in}{1.723621in}}%
\pgfpathlineto{\pgfqpoint{2.745843in}{1.723621in}}%
\pgfpathlineto{\pgfqpoint{2.745843in}{1.610256in}}%
\pgfpathclose%
\pgfusepath{stroke,fill}%
\end{pgfscope}%
\begin{pgfscope}%
\pgfpathrectangle{\pgfqpoint{0.868056in}{0.555556in}}{\pgfqpoint{3.993056in}{1.888889in}}%
\pgfusepath{clip}%
\pgfsetbuttcap%
\pgfsetmiterjoin%
\definecolor{currentfill}{rgb}{0.447059,0.447059,0.447059}%
\pgfsetfillcolor{currentfill}%
\pgfsetlinewidth{1.003750pt}%
\definecolor{currentstroke}{rgb}{0.266667,0.266667,0.266667}%
\pgfsetstrokecolor{currentstroke}%
\pgfsetdash{}{0pt}%
\pgfpathmoveto{\pgfqpoint{3.085100in}{1.356852in}}%
\pgfpathlineto{\pgfqpoint{3.322580in}{1.356852in}}%
\pgfpathlineto{\pgfqpoint{3.322580in}{1.455808in}}%
\pgfpathlineto{\pgfqpoint{3.085100in}{1.455808in}}%
\pgfpathlineto{\pgfqpoint{3.085100in}{1.356852in}}%
\pgfpathclose%
\pgfusepath{stroke,fill}%
\end{pgfscope}%
\begin{pgfscope}%
\pgfpathrectangle{\pgfqpoint{0.868056in}{0.555556in}}{\pgfqpoint{3.993056in}{1.888889in}}%
\pgfusepath{clip}%
\pgfsetbuttcap%
\pgfsetmiterjoin%
\definecolor{currentfill}{rgb}{0.447059,0.447059,0.447059}%
\pgfsetfillcolor{currentfill}%
\pgfsetlinewidth{1.003750pt}%
\definecolor{currentstroke}{rgb}{0.266667,0.266667,0.266667}%
\pgfsetstrokecolor{currentstroke}%
\pgfsetdash{}{0pt}%
\pgfpathmoveto{\pgfqpoint{3.424357in}{1.626373in}}%
\pgfpathlineto{\pgfqpoint{3.661837in}{1.626373in}}%
\pgfpathlineto{\pgfqpoint{3.661837in}{1.739189in}}%
\pgfpathlineto{\pgfqpoint{3.424357in}{1.739189in}}%
\pgfpathlineto{\pgfqpoint{3.424357in}{1.626373in}}%
\pgfpathclose%
\pgfusepath{stroke,fill}%
\end{pgfscope}%
\begin{pgfscope}%
\pgfpathrectangle{\pgfqpoint{0.868056in}{0.555556in}}{\pgfqpoint{3.993056in}{1.888889in}}%
\pgfusepath{clip}%
\pgfsetbuttcap%
\pgfsetmiterjoin%
\definecolor{currentfill}{rgb}{0.447059,0.447059,0.447059}%
\pgfsetfillcolor{currentfill}%
\pgfsetlinewidth{1.003750pt}%
\definecolor{currentstroke}{rgb}{0.266667,0.266667,0.266667}%
\pgfsetstrokecolor{currentstroke}%
\pgfsetdash{}{0pt}%
\pgfpathmoveto{\pgfqpoint{3.763615in}{1.843455in}}%
\pgfpathlineto{\pgfqpoint{4.001094in}{1.843455in}}%
\pgfpathlineto{\pgfqpoint{4.001094in}{1.947525in}}%
\pgfpathlineto{\pgfqpoint{3.763615in}{1.947525in}}%
\pgfpathlineto{\pgfqpoint{3.763615in}{1.843455in}}%
\pgfpathclose%
\pgfusepath{stroke,fill}%
\end{pgfscope}%
\begin{pgfscope}%
\pgfpathrectangle{\pgfqpoint{0.868056in}{0.555556in}}{\pgfqpoint{3.993056in}{1.888889in}}%
\pgfusepath{clip}%
\pgfsetbuttcap%
\pgfsetmiterjoin%
\definecolor{currentfill}{rgb}{0.447059,0.447059,0.447059}%
\pgfsetfillcolor{currentfill}%
\pgfsetlinewidth{1.003750pt}%
\definecolor{currentstroke}{rgb}{0.266667,0.266667,0.266667}%
\pgfsetstrokecolor{currentstroke}%
\pgfsetdash{}{0pt}%
\pgfpathmoveto{\pgfqpoint{4.102872in}{1.906731in}}%
\pgfpathlineto{\pgfqpoint{4.340352in}{1.906731in}}%
\pgfpathlineto{\pgfqpoint{4.340352in}{2.014672in}}%
\pgfpathlineto{\pgfqpoint{4.102872in}{2.014672in}}%
\pgfpathlineto{\pgfqpoint{4.102872in}{1.906731in}}%
\pgfpathclose%
\pgfusepath{stroke,fill}%
\end{pgfscope}%
\begin{pgfscope}%
\pgfpathrectangle{\pgfqpoint{0.868056in}{0.555556in}}{\pgfqpoint{3.993056in}{1.888889in}}%
\pgfusepath{clip}%
\pgfsetbuttcap%
\pgfsetmiterjoin%
\definecolor{currentfill}{rgb}{0.447059,0.447059,0.447059}%
\pgfsetfillcolor{currentfill}%
\pgfsetlinewidth{1.003750pt}%
\definecolor{currentstroke}{rgb}{0.266667,0.266667,0.266667}%
\pgfsetstrokecolor{currentstroke}%
\pgfsetdash{}{0pt}%
\pgfpathmoveto{\pgfqpoint{4.442129in}{2.086767in}}%
\pgfpathlineto{\pgfqpoint{4.679609in}{2.086767in}}%
\pgfpathlineto{\pgfqpoint{4.679609in}{2.193725in}}%
\pgfpathlineto{\pgfqpoint{4.442129in}{2.193725in}}%
\pgfpathlineto{\pgfqpoint{4.442129in}{2.086767in}}%
\pgfpathclose%
\pgfusepath{stroke,fill}%
\end{pgfscope}%
\begin{pgfscope}%
\pgfpathrectangle{\pgfqpoint{0.868056in}{0.555556in}}{\pgfqpoint{3.993056in}{1.888889in}}%
\pgfusepath{clip}%
\pgfsetbuttcap%
\pgfsetmiterjoin%
\definecolor{currentfill}{rgb}{0.447059,0.447059,0.447059}%
\pgfsetfillcolor{currentfill}%
\pgfsetlinewidth{1.003750pt}%
\definecolor{currentstroke}{rgb}{0.266667,0.266667,0.266667}%
\pgfsetstrokecolor{currentstroke}%
\pgfsetdash{}{0pt}%
\pgfpathmoveto{\pgfqpoint{1.049558in}{1.972730in}}%
\pgfpathlineto{\pgfqpoint{1.287038in}{1.972730in}}%
\pgfpathlineto{\pgfqpoint{1.287038in}{2.063529in}}%
\pgfpathlineto{\pgfqpoint{1.049558in}{2.063529in}}%
\pgfpathlineto{\pgfqpoint{1.049558in}{1.972730in}}%
\pgfpathclose%
\pgfusepath{stroke,fill}%
\end{pgfscope}%
\begin{pgfscope}%
\pgfpathrectangle{\pgfqpoint{0.868056in}{0.555556in}}{\pgfqpoint{3.993056in}{1.888889in}}%
\pgfusepath{clip}%
\pgfsetbuttcap%
\pgfsetmiterjoin%
\definecolor{currentfill}{rgb}{0.447059,0.447059,0.447059}%
\pgfsetfillcolor{currentfill}%
\pgfsetlinewidth{1.003750pt}%
\definecolor{currentstroke}{rgb}{0.266667,0.266667,0.266667}%
\pgfsetstrokecolor{currentstroke}%
\pgfsetdash{}{0pt}%
\pgfpathmoveto{\pgfqpoint{1.388815in}{1.577816in}}%
\pgfpathlineto{\pgfqpoint{1.626295in}{1.577816in}}%
\pgfpathlineto{\pgfqpoint{1.626295in}{1.665407in}}%
\pgfpathlineto{\pgfqpoint{1.388815in}{1.665407in}}%
\pgfpathlineto{\pgfqpoint{1.388815in}{1.577816in}}%
\pgfpathclose%
\pgfusepath{stroke,fill}%
\end{pgfscope}%
\begin{pgfscope}%
\pgfpathrectangle{\pgfqpoint{0.868056in}{0.555556in}}{\pgfqpoint{3.993056in}{1.888889in}}%
\pgfusepath{clip}%
\pgfsetbuttcap%
\pgfsetmiterjoin%
\definecolor{currentfill}{rgb}{0.447059,0.447059,0.447059}%
\pgfsetfillcolor{currentfill}%
\pgfsetlinewidth{1.003750pt}%
\definecolor{currentstroke}{rgb}{0.266667,0.266667,0.266667}%
\pgfsetstrokecolor{currentstroke}%
\pgfsetdash{}{0pt}%
\pgfpathmoveto{\pgfqpoint{1.728072in}{1.317300in}}%
\pgfpathlineto{\pgfqpoint{1.965552in}{1.317300in}}%
\pgfpathlineto{\pgfqpoint{1.965552in}{1.412986in}}%
\pgfpathlineto{\pgfqpoint{1.728072in}{1.412986in}}%
\pgfpathlineto{\pgfqpoint{1.728072in}{1.317300in}}%
\pgfpathclose%
\pgfusepath{stroke,fill}%
\end{pgfscope}%
\begin{pgfscope}%
\pgfpathrectangle{\pgfqpoint{0.868056in}{0.555556in}}{\pgfqpoint{3.993056in}{1.888889in}}%
\pgfusepath{clip}%
\pgfsetbuttcap%
\pgfsetmiterjoin%
\definecolor{currentfill}{rgb}{0.447059,0.447059,0.447059}%
\pgfsetfillcolor{currentfill}%
\pgfsetlinewidth{1.003750pt}%
\definecolor{currentstroke}{rgb}{0.266667,0.266667,0.266667}%
\pgfsetstrokecolor{currentstroke}%
\pgfsetdash{}{0pt}%
\pgfpathmoveto{\pgfqpoint{2.067329in}{1.430903in}}%
\pgfpathlineto{\pgfqpoint{2.304809in}{1.430903in}}%
\pgfpathlineto{\pgfqpoint{2.304809in}{1.502150in}}%
\pgfpathlineto{\pgfqpoint{2.067329in}{1.502150in}}%
\pgfpathlineto{\pgfqpoint{2.067329in}{1.430903in}}%
\pgfpathclose%
\pgfusepath{stroke,fill}%
\end{pgfscope}%
\begin{pgfscope}%
\pgfpathrectangle{\pgfqpoint{0.868056in}{0.555556in}}{\pgfqpoint{3.993056in}{1.888889in}}%
\pgfusepath{clip}%
\pgfsetbuttcap%
\pgfsetmiterjoin%
\definecolor{currentfill}{rgb}{0.447059,0.447059,0.447059}%
\pgfsetfillcolor{currentfill}%
\pgfsetlinewidth{1.003750pt}%
\definecolor{currentstroke}{rgb}{0.266667,0.266667,0.266667}%
\pgfsetstrokecolor{currentstroke}%
\pgfsetdash{}{0pt}%
\pgfpathmoveto{\pgfqpoint{2.406586in}{1.517583in}}%
\pgfpathlineto{\pgfqpoint{2.644066in}{1.517583in}}%
\pgfpathlineto{\pgfqpoint{2.644066in}{1.572299in}}%
\pgfpathlineto{\pgfqpoint{2.406586in}{1.572299in}}%
\pgfpathlineto{\pgfqpoint{2.406586in}{1.517583in}}%
\pgfpathclose%
\pgfusepath{stroke,fill}%
\end{pgfscope}%
\begin{pgfscope}%
\pgfpathrectangle{\pgfqpoint{0.868056in}{0.555556in}}{\pgfqpoint{3.993056in}{1.888889in}}%
\pgfusepath{clip}%
\pgfsetbuttcap%
\pgfsetmiterjoin%
\definecolor{currentfill}{rgb}{0.447059,0.447059,0.447059}%
\pgfsetfillcolor{currentfill}%
\pgfsetlinewidth{1.003750pt}%
\definecolor{currentstroke}{rgb}{0.266667,0.266667,0.266667}%
\pgfsetstrokecolor{currentstroke}%
\pgfsetdash{}{0pt}%
\pgfpathmoveto{\pgfqpoint{2.745843in}{1.723621in}}%
\pgfpathlineto{\pgfqpoint{2.983323in}{1.723621in}}%
\pgfpathlineto{\pgfqpoint{2.983323in}{1.766205in}}%
\pgfpathlineto{\pgfqpoint{2.745843in}{1.766205in}}%
\pgfpathlineto{\pgfqpoint{2.745843in}{1.723621in}}%
\pgfpathclose%
\pgfusepath{stroke,fill}%
\end{pgfscope}%
\begin{pgfscope}%
\pgfpathrectangle{\pgfqpoint{0.868056in}{0.555556in}}{\pgfqpoint{3.993056in}{1.888889in}}%
\pgfusepath{clip}%
\pgfsetbuttcap%
\pgfsetmiterjoin%
\definecolor{currentfill}{rgb}{0.447059,0.447059,0.447059}%
\pgfsetfillcolor{currentfill}%
\pgfsetlinewidth{1.003750pt}%
\definecolor{currentstroke}{rgb}{0.266667,0.266667,0.266667}%
\pgfsetstrokecolor{currentstroke}%
\pgfsetdash{}{0pt}%
\pgfpathmoveto{\pgfqpoint{3.085100in}{1.455808in}}%
\pgfpathlineto{\pgfqpoint{3.322580in}{1.455808in}}%
\pgfpathlineto{\pgfqpoint{3.322580in}{1.496643in}}%
\pgfpathlineto{\pgfqpoint{3.085100in}{1.496643in}}%
\pgfpathlineto{\pgfqpoint{3.085100in}{1.455808in}}%
\pgfpathclose%
\pgfusepath{stroke,fill}%
\end{pgfscope}%
\begin{pgfscope}%
\pgfpathrectangle{\pgfqpoint{0.868056in}{0.555556in}}{\pgfqpoint{3.993056in}{1.888889in}}%
\pgfusepath{clip}%
\pgfsetbuttcap%
\pgfsetmiterjoin%
\definecolor{currentfill}{rgb}{0.447059,0.447059,0.447059}%
\pgfsetfillcolor{currentfill}%
\pgfsetlinewidth{1.003750pt}%
\definecolor{currentstroke}{rgb}{0.266667,0.266667,0.266667}%
\pgfsetstrokecolor{currentstroke}%
\pgfsetdash{}{0pt}%
\pgfpathmoveto{\pgfqpoint{3.424357in}{1.739189in}}%
\pgfpathlineto{\pgfqpoint{3.661837in}{1.739189in}}%
\pgfpathlineto{\pgfqpoint{3.661837in}{1.798583in}}%
\pgfpathlineto{\pgfqpoint{3.424357in}{1.798583in}}%
\pgfpathlineto{\pgfqpoint{3.424357in}{1.739189in}}%
\pgfpathclose%
\pgfusepath{stroke,fill}%
\end{pgfscope}%
\begin{pgfscope}%
\pgfpathrectangle{\pgfqpoint{0.868056in}{0.555556in}}{\pgfqpoint{3.993056in}{1.888889in}}%
\pgfusepath{clip}%
\pgfsetbuttcap%
\pgfsetmiterjoin%
\definecolor{currentfill}{rgb}{0.447059,0.447059,0.447059}%
\pgfsetfillcolor{currentfill}%
\pgfsetlinewidth{1.003750pt}%
\definecolor{currentstroke}{rgb}{0.266667,0.266667,0.266667}%
\pgfsetstrokecolor{currentstroke}%
\pgfsetdash{}{0pt}%
\pgfpathmoveto{\pgfqpoint{3.763615in}{1.947525in}}%
\pgfpathlineto{\pgfqpoint{4.001094in}{1.947525in}}%
\pgfpathlineto{\pgfqpoint{4.001094in}{2.016390in}}%
\pgfpathlineto{\pgfqpoint{3.763615in}{2.016390in}}%
\pgfpathlineto{\pgfqpoint{3.763615in}{1.947525in}}%
\pgfpathclose%
\pgfusepath{stroke,fill}%
\end{pgfscope}%
\begin{pgfscope}%
\pgfpathrectangle{\pgfqpoint{0.868056in}{0.555556in}}{\pgfqpoint{3.993056in}{1.888889in}}%
\pgfusepath{clip}%
\pgfsetbuttcap%
\pgfsetmiterjoin%
\definecolor{currentfill}{rgb}{0.447059,0.447059,0.447059}%
\pgfsetfillcolor{currentfill}%
\pgfsetlinewidth{1.003750pt}%
\definecolor{currentstroke}{rgb}{0.266667,0.266667,0.266667}%
\pgfsetstrokecolor{currentstroke}%
\pgfsetdash{}{0pt}%
\pgfpathmoveto{\pgfqpoint{4.102872in}{2.014672in}}%
\pgfpathlineto{\pgfqpoint{4.340352in}{2.014672in}}%
\pgfpathlineto{\pgfqpoint{4.340352in}{2.092388in}}%
\pgfpathlineto{\pgfqpoint{4.102872in}{2.092388in}}%
\pgfpathlineto{\pgfqpoint{4.102872in}{2.014672in}}%
\pgfpathclose%
\pgfusepath{stroke,fill}%
\end{pgfscope}%
\begin{pgfscope}%
\pgfpathrectangle{\pgfqpoint{0.868056in}{0.555556in}}{\pgfqpoint{3.993056in}{1.888889in}}%
\pgfusepath{clip}%
\pgfsetbuttcap%
\pgfsetmiterjoin%
\definecolor{currentfill}{rgb}{0.447059,0.447059,0.447059}%
\pgfsetfillcolor{currentfill}%
\pgfsetlinewidth{1.003750pt}%
\definecolor{currentstroke}{rgb}{0.266667,0.266667,0.266667}%
\pgfsetstrokecolor{currentstroke}%
\pgfsetdash{}{0pt}%
\pgfpathmoveto{\pgfqpoint{4.442129in}{2.193725in}}%
\pgfpathlineto{\pgfqpoint{4.679609in}{2.193725in}}%
\pgfpathlineto{\pgfqpoint{4.679609in}{2.256297in}}%
\pgfpathlineto{\pgfqpoint{4.442129in}{2.256297in}}%
\pgfpathlineto{\pgfqpoint{4.442129in}{2.193725in}}%
\pgfpathclose%
\pgfusepath{stroke,fill}%
\end{pgfscope}%
\begin{pgfscope}%
\pgfpathrectangle{\pgfqpoint{0.868056in}{0.555556in}}{\pgfqpoint{3.993056in}{1.888889in}}%
\pgfusepath{clip}%
\pgfsetbuttcap%
\pgfsetmiterjoin%
\definecolor{currentfill}{rgb}{0.447059,0.447059,0.447059}%
\pgfsetfillcolor{currentfill}%
\pgfsetlinewidth{1.003750pt}%
\definecolor{currentstroke}{rgb}{0.266667,0.266667,0.266667}%
\pgfsetstrokecolor{currentstroke}%
\pgfsetdash{}{0pt}%
\pgfpathmoveto{\pgfqpoint{1.049558in}{2.063529in}}%
\pgfpathlineto{\pgfqpoint{1.287038in}{2.063529in}}%
\pgfpathlineto{\pgfqpoint{1.287038in}{2.166491in}}%
\pgfpathlineto{\pgfqpoint{1.049558in}{2.166491in}}%
\pgfpathlineto{\pgfqpoint{1.049558in}{2.063529in}}%
\pgfpathclose%
\pgfusepath{stroke,fill}%
\end{pgfscope}%
\begin{pgfscope}%
\pgfpathrectangle{\pgfqpoint{0.868056in}{0.555556in}}{\pgfqpoint{3.993056in}{1.888889in}}%
\pgfusepath{clip}%
\pgfsetbuttcap%
\pgfsetmiterjoin%
\definecolor{currentfill}{rgb}{0.447059,0.447059,0.447059}%
\pgfsetfillcolor{currentfill}%
\pgfsetlinewidth{1.003750pt}%
\definecolor{currentstroke}{rgb}{0.266667,0.266667,0.266667}%
\pgfsetstrokecolor{currentstroke}%
\pgfsetdash{}{0pt}%
\pgfpathmoveto{\pgfqpoint{1.388815in}{1.665407in}}%
\pgfpathlineto{\pgfqpoint{1.626295in}{1.665407in}}%
\pgfpathlineto{\pgfqpoint{1.626295in}{1.769217in}}%
\pgfpathlineto{\pgfqpoint{1.388815in}{1.769217in}}%
\pgfpathlineto{\pgfqpoint{1.388815in}{1.665407in}}%
\pgfpathclose%
\pgfusepath{stroke,fill}%
\end{pgfscope}%
\begin{pgfscope}%
\pgfpathrectangle{\pgfqpoint{0.868056in}{0.555556in}}{\pgfqpoint{3.993056in}{1.888889in}}%
\pgfusepath{clip}%
\pgfsetbuttcap%
\pgfsetmiterjoin%
\definecolor{currentfill}{rgb}{0.447059,0.447059,0.447059}%
\pgfsetfillcolor{currentfill}%
\pgfsetlinewidth{1.003750pt}%
\definecolor{currentstroke}{rgb}{0.266667,0.266667,0.266667}%
\pgfsetstrokecolor{currentstroke}%
\pgfsetdash{}{0pt}%
\pgfpathmoveto{\pgfqpoint{1.728072in}{1.412986in}}%
\pgfpathlineto{\pgfqpoint{1.965552in}{1.412986in}}%
\pgfpathlineto{\pgfqpoint{1.965552in}{1.476127in}}%
\pgfpathlineto{\pgfqpoint{1.728072in}{1.476127in}}%
\pgfpathlineto{\pgfqpoint{1.728072in}{1.412986in}}%
\pgfpathclose%
\pgfusepath{stroke,fill}%
\end{pgfscope}%
\begin{pgfscope}%
\pgfpathrectangle{\pgfqpoint{0.868056in}{0.555556in}}{\pgfqpoint{3.993056in}{1.888889in}}%
\pgfusepath{clip}%
\pgfsetbuttcap%
\pgfsetmiterjoin%
\definecolor{currentfill}{rgb}{0.447059,0.447059,0.447059}%
\pgfsetfillcolor{currentfill}%
\pgfsetlinewidth{1.003750pt}%
\definecolor{currentstroke}{rgb}{0.266667,0.266667,0.266667}%
\pgfsetstrokecolor{currentstroke}%
\pgfsetdash{}{0pt}%
\pgfpathmoveto{\pgfqpoint{2.067329in}{1.502150in}}%
\pgfpathlineto{\pgfqpoint{2.304809in}{1.502150in}}%
\pgfpathlineto{\pgfqpoint{2.304809in}{1.584627in}}%
\pgfpathlineto{\pgfqpoint{2.067329in}{1.584627in}}%
\pgfpathlineto{\pgfqpoint{2.067329in}{1.502150in}}%
\pgfpathclose%
\pgfusepath{stroke,fill}%
\end{pgfscope}%
\begin{pgfscope}%
\pgfpathrectangle{\pgfqpoint{0.868056in}{0.555556in}}{\pgfqpoint{3.993056in}{1.888889in}}%
\pgfusepath{clip}%
\pgfsetbuttcap%
\pgfsetmiterjoin%
\definecolor{currentfill}{rgb}{0.447059,0.447059,0.447059}%
\pgfsetfillcolor{currentfill}%
\pgfsetlinewidth{1.003750pt}%
\definecolor{currentstroke}{rgb}{0.266667,0.266667,0.266667}%
\pgfsetstrokecolor{currentstroke}%
\pgfsetdash{}{0pt}%
\pgfpathmoveto{\pgfqpoint{2.406586in}{1.572299in}}%
\pgfpathlineto{\pgfqpoint{2.644066in}{1.572299in}}%
\pgfpathlineto{\pgfqpoint{2.644066in}{1.635958in}}%
\pgfpathlineto{\pgfqpoint{2.406586in}{1.635958in}}%
\pgfpathlineto{\pgfqpoint{2.406586in}{1.572299in}}%
\pgfpathclose%
\pgfusepath{stroke,fill}%
\end{pgfscope}%
\begin{pgfscope}%
\pgfpathrectangle{\pgfqpoint{0.868056in}{0.555556in}}{\pgfqpoint{3.993056in}{1.888889in}}%
\pgfusepath{clip}%
\pgfsetbuttcap%
\pgfsetmiterjoin%
\definecolor{currentfill}{rgb}{0.447059,0.447059,0.447059}%
\pgfsetfillcolor{currentfill}%
\pgfsetlinewidth{1.003750pt}%
\definecolor{currentstroke}{rgb}{0.266667,0.266667,0.266667}%
\pgfsetstrokecolor{currentstroke}%
\pgfsetdash{}{0pt}%
\pgfpathmoveto{\pgfqpoint{2.745843in}{1.766205in}}%
\pgfpathlineto{\pgfqpoint{2.983323in}{1.766205in}}%
\pgfpathlineto{\pgfqpoint{2.983323in}{1.831251in}}%
\pgfpathlineto{\pgfqpoint{2.745843in}{1.831251in}}%
\pgfpathlineto{\pgfqpoint{2.745843in}{1.766205in}}%
\pgfpathclose%
\pgfusepath{stroke,fill}%
\end{pgfscope}%
\begin{pgfscope}%
\pgfpathrectangle{\pgfqpoint{0.868056in}{0.555556in}}{\pgfqpoint{3.993056in}{1.888889in}}%
\pgfusepath{clip}%
\pgfsetbuttcap%
\pgfsetmiterjoin%
\definecolor{currentfill}{rgb}{0.447059,0.447059,0.447059}%
\pgfsetfillcolor{currentfill}%
\pgfsetlinewidth{1.003750pt}%
\definecolor{currentstroke}{rgb}{0.266667,0.266667,0.266667}%
\pgfsetstrokecolor{currentstroke}%
\pgfsetdash{}{0pt}%
\pgfpathmoveto{\pgfqpoint{3.085100in}{1.496643in}}%
\pgfpathlineto{\pgfqpoint{3.322580in}{1.496643in}}%
\pgfpathlineto{\pgfqpoint{3.322580in}{1.558832in}}%
\pgfpathlineto{\pgfqpoint{3.085100in}{1.558832in}}%
\pgfpathlineto{\pgfqpoint{3.085100in}{1.496643in}}%
\pgfpathclose%
\pgfusepath{stroke,fill}%
\end{pgfscope}%
\begin{pgfscope}%
\pgfpathrectangle{\pgfqpoint{0.868056in}{0.555556in}}{\pgfqpoint{3.993056in}{1.888889in}}%
\pgfusepath{clip}%
\pgfsetbuttcap%
\pgfsetmiterjoin%
\definecolor{currentfill}{rgb}{0.447059,0.447059,0.447059}%
\pgfsetfillcolor{currentfill}%
\pgfsetlinewidth{1.003750pt}%
\definecolor{currentstroke}{rgb}{0.266667,0.266667,0.266667}%
\pgfsetstrokecolor{currentstroke}%
\pgfsetdash{}{0pt}%
\pgfpathmoveto{\pgfqpoint{3.424357in}{1.798583in}}%
\pgfpathlineto{\pgfqpoint{3.661837in}{1.798583in}}%
\pgfpathlineto{\pgfqpoint{3.661837in}{1.860058in}}%
\pgfpathlineto{\pgfqpoint{3.424357in}{1.860058in}}%
\pgfpathlineto{\pgfqpoint{3.424357in}{1.798583in}}%
\pgfpathclose%
\pgfusepath{stroke,fill}%
\end{pgfscope}%
\begin{pgfscope}%
\pgfpathrectangle{\pgfqpoint{0.868056in}{0.555556in}}{\pgfqpoint{3.993056in}{1.888889in}}%
\pgfusepath{clip}%
\pgfsetbuttcap%
\pgfsetmiterjoin%
\definecolor{currentfill}{rgb}{0.447059,0.447059,0.447059}%
\pgfsetfillcolor{currentfill}%
\pgfsetlinewidth{1.003750pt}%
\definecolor{currentstroke}{rgb}{0.266667,0.266667,0.266667}%
\pgfsetstrokecolor{currentstroke}%
\pgfsetdash{}{0pt}%
\pgfpathmoveto{\pgfqpoint{3.763615in}{2.016390in}}%
\pgfpathlineto{\pgfqpoint{4.001094in}{2.016390in}}%
\pgfpathlineto{\pgfqpoint{4.001094in}{2.082730in}}%
\pgfpathlineto{\pgfqpoint{3.763615in}{2.082730in}}%
\pgfpathlineto{\pgfqpoint{3.763615in}{2.016390in}}%
\pgfpathclose%
\pgfusepath{stroke,fill}%
\end{pgfscope}%
\begin{pgfscope}%
\pgfpathrectangle{\pgfqpoint{0.868056in}{0.555556in}}{\pgfqpoint{3.993056in}{1.888889in}}%
\pgfusepath{clip}%
\pgfsetbuttcap%
\pgfsetmiterjoin%
\definecolor{currentfill}{rgb}{0.447059,0.447059,0.447059}%
\pgfsetfillcolor{currentfill}%
\pgfsetlinewidth{1.003750pt}%
\definecolor{currentstroke}{rgb}{0.266667,0.266667,0.266667}%
\pgfsetstrokecolor{currentstroke}%
\pgfsetdash{}{0pt}%
\pgfpathmoveto{\pgfqpoint{4.102872in}{2.092388in}}%
\pgfpathlineto{\pgfqpoint{4.340352in}{2.092388in}}%
\pgfpathlineto{\pgfqpoint{4.340352in}{2.161678in}}%
\pgfpathlineto{\pgfqpoint{4.102872in}{2.161678in}}%
\pgfpathlineto{\pgfqpoint{4.102872in}{2.092388in}}%
\pgfpathclose%
\pgfusepath{stroke,fill}%
\end{pgfscope}%
\begin{pgfscope}%
\pgfpathrectangle{\pgfqpoint{0.868056in}{0.555556in}}{\pgfqpoint{3.993056in}{1.888889in}}%
\pgfusepath{clip}%
\pgfsetbuttcap%
\pgfsetmiterjoin%
\definecolor{currentfill}{rgb}{0.447059,0.447059,0.447059}%
\pgfsetfillcolor{currentfill}%
\pgfsetlinewidth{1.003750pt}%
\definecolor{currentstroke}{rgb}{0.266667,0.266667,0.266667}%
\pgfsetstrokecolor{currentstroke}%
\pgfsetdash{}{0pt}%
\pgfpathmoveto{\pgfqpoint{4.442129in}{2.256297in}}%
\pgfpathlineto{\pgfqpoint{4.679609in}{2.256297in}}%
\pgfpathlineto{\pgfqpoint{4.679609in}{2.324541in}}%
\pgfpathlineto{\pgfqpoint{4.442129in}{2.324541in}}%
\pgfpathlineto{\pgfqpoint{4.442129in}{2.256297in}}%
\pgfpathclose%
\pgfusepath{stroke,fill}%
\end{pgfscope}%
\begin{pgfscope}%
\pgfpathrectangle{\pgfqpoint{0.868056in}{0.555556in}}{\pgfqpoint{3.993056in}{1.888889in}}%
\pgfusepath{clip}%
\pgfsetbuttcap%
\pgfsetmiterjoin%
\definecolor{currentfill}{rgb}{0.447059,0.447059,0.447059}%
\pgfsetfillcolor{currentfill}%
\pgfsetlinewidth{1.003750pt}%
\definecolor{currentstroke}{rgb}{0.266667,0.266667,0.266667}%
\pgfsetstrokecolor{currentstroke}%
\pgfsetdash{}{0pt}%
\pgfpathmoveto{\pgfqpoint{1.049558in}{2.166491in}}%
\pgfpathlineto{\pgfqpoint{1.287038in}{2.166491in}}%
\pgfpathlineto{\pgfqpoint{1.287038in}{2.180807in}}%
\pgfpathlineto{\pgfqpoint{1.049558in}{2.180807in}}%
\pgfpathlineto{\pgfqpoint{1.049558in}{2.166491in}}%
\pgfpathclose%
\pgfusepath{stroke,fill}%
\end{pgfscope}%
\begin{pgfscope}%
\pgfpathrectangle{\pgfqpoint{0.868056in}{0.555556in}}{\pgfqpoint{3.993056in}{1.888889in}}%
\pgfusepath{clip}%
\pgfsetbuttcap%
\pgfsetmiterjoin%
\definecolor{currentfill}{rgb}{0.447059,0.447059,0.447059}%
\pgfsetfillcolor{currentfill}%
\pgfsetlinewidth{1.003750pt}%
\definecolor{currentstroke}{rgb}{0.266667,0.266667,0.266667}%
\pgfsetstrokecolor{currentstroke}%
\pgfsetdash{}{0pt}%
\pgfpathmoveto{\pgfqpoint{1.388815in}{1.769217in}}%
\pgfpathlineto{\pgfqpoint{1.626295in}{1.769217in}}%
\pgfpathlineto{\pgfqpoint{1.626295in}{1.773710in}}%
\pgfpathlineto{\pgfqpoint{1.388815in}{1.773710in}}%
\pgfpathlineto{\pgfqpoint{1.388815in}{1.769217in}}%
\pgfpathclose%
\pgfusepath{stroke,fill}%
\end{pgfscope}%
\begin{pgfscope}%
\pgfpathrectangle{\pgfqpoint{0.868056in}{0.555556in}}{\pgfqpoint{3.993056in}{1.888889in}}%
\pgfusepath{clip}%
\pgfsetbuttcap%
\pgfsetmiterjoin%
\definecolor{currentfill}{rgb}{0.447059,0.447059,0.447059}%
\pgfsetfillcolor{currentfill}%
\pgfsetlinewidth{1.003750pt}%
\definecolor{currentstroke}{rgb}{0.266667,0.266667,0.266667}%
\pgfsetstrokecolor{currentstroke}%
\pgfsetdash{}{0pt}%
\pgfpathmoveto{\pgfqpoint{1.728072in}{1.476127in}}%
\pgfpathlineto{\pgfqpoint{1.965552in}{1.476127in}}%
\pgfpathlineto{\pgfqpoint{1.965552in}{1.479036in}}%
\pgfpathlineto{\pgfqpoint{1.728072in}{1.479036in}}%
\pgfpathlineto{\pgfqpoint{1.728072in}{1.476127in}}%
\pgfpathclose%
\pgfusepath{stroke,fill}%
\end{pgfscope}%
\begin{pgfscope}%
\pgfpathrectangle{\pgfqpoint{0.868056in}{0.555556in}}{\pgfqpoint{3.993056in}{1.888889in}}%
\pgfusepath{clip}%
\pgfsetbuttcap%
\pgfsetmiterjoin%
\definecolor{currentfill}{rgb}{0.447059,0.447059,0.447059}%
\pgfsetfillcolor{currentfill}%
\pgfsetlinewidth{1.003750pt}%
\definecolor{currentstroke}{rgb}{0.266667,0.266667,0.266667}%
\pgfsetstrokecolor{currentstroke}%
\pgfsetdash{}{0pt}%
\pgfpathmoveto{\pgfqpoint{2.067329in}{1.584627in}}%
\pgfpathlineto{\pgfqpoint{2.304809in}{1.584627in}}%
\pgfpathlineto{\pgfqpoint{2.304809in}{1.588591in}}%
\pgfpathlineto{\pgfqpoint{2.067329in}{1.588591in}}%
\pgfpathlineto{\pgfqpoint{2.067329in}{1.584627in}}%
\pgfpathclose%
\pgfusepath{stroke,fill}%
\end{pgfscope}%
\begin{pgfscope}%
\pgfpathrectangle{\pgfqpoint{0.868056in}{0.555556in}}{\pgfqpoint{3.993056in}{1.888889in}}%
\pgfusepath{clip}%
\pgfsetbuttcap%
\pgfsetmiterjoin%
\definecolor{currentfill}{rgb}{0.447059,0.447059,0.447059}%
\pgfsetfillcolor{currentfill}%
\pgfsetlinewidth{1.003750pt}%
\definecolor{currentstroke}{rgb}{0.266667,0.266667,0.266667}%
\pgfsetstrokecolor{currentstroke}%
\pgfsetdash{}{0pt}%
\pgfpathmoveto{\pgfqpoint{2.406586in}{1.635958in}}%
\pgfpathlineto{\pgfqpoint{2.644066in}{1.635958in}}%
\pgfpathlineto{\pgfqpoint{2.644066in}{1.639591in}}%
\pgfpathlineto{\pgfqpoint{2.406586in}{1.639591in}}%
\pgfpathlineto{\pgfqpoint{2.406586in}{1.635958in}}%
\pgfpathclose%
\pgfusepath{stroke,fill}%
\end{pgfscope}%
\begin{pgfscope}%
\pgfpathrectangle{\pgfqpoint{0.868056in}{0.555556in}}{\pgfqpoint{3.993056in}{1.888889in}}%
\pgfusepath{clip}%
\pgfsetbuttcap%
\pgfsetmiterjoin%
\definecolor{currentfill}{rgb}{0.447059,0.447059,0.447059}%
\pgfsetfillcolor{currentfill}%
\pgfsetlinewidth{1.003750pt}%
\definecolor{currentstroke}{rgb}{0.266667,0.266667,0.266667}%
\pgfsetstrokecolor{currentstroke}%
\pgfsetdash{}{0pt}%
\pgfpathmoveto{\pgfqpoint{2.745843in}{1.831251in}}%
\pgfpathlineto{\pgfqpoint{2.983323in}{1.831251in}}%
\pgfpathlineto{\pgfqpoint{2.983323in}{1.835495in}}%
\pgfpathlineto{\pgfqpoint{2.745843in}{1.835495in}}%
\pgfpathlineto{\pgfqpoint{2.745843in}{1.831251in}}%
\pgfpathclose%
\pgfusepath{stroke,fill}%
\end{pgfscope}%
\begin{pgfscope}%
\pgfpathrectangle{\pgfqpoint{0.868056in}{0.555556in}}{\pgfqpoint{3.993056in}{1.888889in}}%
\pgfusepath{clip}%
\pgfsetbuttcap%
\pgfsetmiterjoin%
\definecolor{currentfill}{rgb}{0.447059,0.447059,0.447059}%
\pgfsetfillcolor{currentfill}%
\pgfsetlinewidth{1.003750pt}%
\definecolor{currentstroke}{rgb}{0.266667,0.266667,0.266667}%
\pgfsetstrokecolor{currentstroke}%
\pgfsetdash{}{0pt}%
\pgfpathmoveto{\pgfqpoint{3.085100in}{1.558832in}}%
\pgfpathlineto{\pgfqpoint{3.322580in}{1.558832in}}%
\pgfpathlineto{\pgfqpoint{3.322580in}{1.564815in}}%
\pgfpathlineto{\pgfqpoint{3.085100in}{1.564815in}}%
\pgfpathlineto{\pgfqpoint{3.085100in}{1.558832in}}%
\pgfpathclose%
\pgfusepath{stroke,fill}%
\end{pgfscope}%
\begin{pgfscope}%
\pgfpathrectangle{\pgfqpoint{0.868056in}{0.555556in}}{\pgfqpoint{3.993056in}{1.888889in}}%
\pgfusepath{clip}%
\pgfsetbuttcap%
\pgfsetmiterjoin%
\definecolor{currentfill}{rgb}{0.447059,0.447059,0.447059}%
\pgfsetfillcolor{currentfill}%
\pgfsetlinewidth{1.003750pt}%
\definecolor{currentstroke}{rgb}{0.266667,0.266667,0.266667}%
\pgfsetstrokecolor{currentstroke}%
\pgfsetdash{}{0pt}%
\pgfpathmoveto{\pgfqpoint{3.424357in}{1.860058in}}%
\pgfpathlineto{\pgfqpoint{3.661837in}{1.860058in}}%
\pgfpathlineto{\pgfqpoint{3.661837in}{1.866466in}}%
\pgfpathlineto{\pgfqpoint{3.424357in}{1.866466in}}%
\pgfpathlineto{\pgfqpoint{3.424357in}{1.860058in}}%
\pgfpathclose%
\pgfusepath{stroke,fill}%
\end{pgfscope}%
\begin{pgfscope}%
\pgfpathrectangle{\pgfqpoint{0.868056in}{0.555556in}}{\pgfqpoint{3.993056in}{1.888889in}}%
\pgfusepath{clip}%
\pgfsetbuttcap%
\pgfsetmiterjoin%
\definecolor{currentfill}{rgb}{0.447059,0.447059,0.447059}%
\pgfsetfillcolor{currentfill}%
\pgfsetlinewidth{1.003750pt}%
\definecolor{currentstroke}{rgb}{0.266667,0.266667,0.266667}%
\pgfsetstrokecolor{currentstroke}%
\pgfsetdash{}{0pt}%
\pgfpathmoveto{\pgfqpoint{3.763615in}{2.082730in}}%
\pgfpathlineto{\pgfqpoint{4.001094in}{2.082730in}}%
\pgfpathlineto{\pgfqpoint{4.001094in}{2.089779in}}%
\pgfpathlineto{\pgfqpoint{3.763615in}{2.089779in}}%
\pgfpathlineto{\pgfqpoint{3.763615in}{2.082730in}}%
\pgfpathclose%
\pgfusepath{stroke,fill}%
\end{pgfscope}%
\begin{pgfscope}%
\pgfpathrectangle{\pgfqpoint{0.868056in}{0.555556in}}{\pgfqpoint{3.993056in}{1.888889in}}%
\pgfusepath{clip}%
\pgfsetbuttcap%
\pgfsetmiterjoin%
\definecolor{currentfill}{rgb}{0.447059,0.447059,0.447059}%
\pgfsetfillcolor{currentfill}%
\pgfsetlinewidth{1.003750pt}%
\definecolor{currentstroke}{rgb}{0.266667,0.266667,0.266667}%
\pgfsetstrokecolor{currentstroke}%
\pgfsetdash{}{0pt}%
\pgfpathmoveto{\pgfqpoint{4.102872in}{2.161678in}}%
\pgfpathlineto{\pgfqpoint{4.340352in}{2.161678in}}%
\pgfpathlineto{\pgfqpoint{4.340352in}{2.169462in}}%
\pgfpathlineto{\pgfqpoint{4.102872in}{2.169462in}}%
\pgfpathlineto{\pgfqpoint{4.102872in}{2.161678in}}%
\pgfpathclose%
\pgfusepath{stroke,fill}%
\end{pgfscope}%
\begin{pgfscope}%
\pgfpathrectangle{\pgfqpoint{0.868056in}{0.555556in}}{\pgfqpoint{3.993056in}{1.888889in}}%
\pgfusepath{clip}%
\pgfsetbuttcap%
\pgfsetmiterjoin%
\definecolor{currentfill}{rgb}{0.447059,0.447059,0.447059}%
\pgfsetfillcolor{currentfill}%
\pgfsetlinewidth{1.003750pt}%
\definecolor{currentstroke}{rgb}{0.266667,0.266667,0.266667}%
\pgfsetstrokecolor{currentstroke}%
\pgfsetdash{}{0pt}%
\pgfpathmoveto{\pgfqpoint{4.442129in}{2.324541in}}%
\pgfpathlineto{\pgfqpoint{4.679609in}{2.324541in}}%
\pgfpathlineto{\pgfqpoint{4.679609in}{2.331383in}}%
\pgfpathlineto{\pgfqpoint{4.442129in}{2.331383in}}%
\pgfpathlineto{\pgfqpoint{4.442129in}{2.324541in}}%
\pgfpathclose%
\pgfusepath{stroke,fill}%
\end{pgfscope}%
\begin{pgfscope}%
\pgfpathrectangle{\pgfqpoint{0.868056in}{0.555556in}}{\pgfqpoint{3.993056in}{1.888889in}}%
\pgfusepath{clip}%
\pgfsetbuttcap%
\pgfsetmiterjoin%
\definecolor{currentfill}{rgb}{0.447059,0.447059,0.447059}%
\pgfsetfillcolor{currentfill}%
\pgfsetlinewidth{1.003750pt}%
\definecolor{currentstroke}{rgb}{0.266667,0.266667,0.266667}%
\pgfsetstrokecolor{currentstroke}%
\pgfsetdash{}{0pt}%
\pgfpathmoveto{\pgfqpoint{1.049558in}{2.180807in}}%
\pgfpathlineto{\pgfqpoint{1.287038in}{2.180807in}}%
\pgfpathlineto{\pgfqpoint{1.287038in}{2.184233in}}%
\pgfpathlineto{\pgfqpoint{1.049558in}{2.184233in}}%
\pgfpathlineto{\pgfqpoint{1.049558in}{2.180807in}}%
\pgfpathclose%
\pgfusepath{stroke,fill}%
\end{pgfscope}%
\begin{pgfscope}%
\pgfpathrectangle{\pgfqpoint{0.868056in}{0.555556in}}{\pgfqpoint{3.993056in}{1.888889in}}%
\pgfusepath{clip}%
\pgfsetbuttcap%
\pgfsetmiterjoin%
\definecolor{currentfill}{rgb}{0.447059,0.447059,0.447059}%
\pgfsetfillcolor{currentfill}%
\pgfsetlinewidth{1.003750pt}%
\definecolor{currentstroke}{rgb}{0.266667,0.266667,0.266667}%
\pgfsetstrokecolor{currentstroke}%
\pgfsetdash{}{0pt}%
\pgfpathmoveto{\pgfqpoint{1.388815in}{1.773710in}}%
\pgfpathlineto{\pgfqpoint{1.626295in}{1.773710in}}%
\pgfpathlineto{\pgfqpoint{1.626295in}{1.776712in}}%
\pgfpathlineto{\pgfqpoint{1.388815in}{1.776712in}}%
\pgfpathlineto{\pgfqpoint{1.388815in}{1.773710in}}%
\pgfpathclose%
\pgfusepath{stroke,fill}%
\end{pgfscope}%
\begin{pgfscope}%
\pgfpathrectangle{\pgfqpoint{0.868056in}{0.555556in}}{\pgfqpoint{3.993056in}{1.888889in}}%
\pgfusepath{clip}%
\pgfsetbuttcap%
\pgfsetmiterjoin%
\definecolor{currentfill}{rgb}{0.447059,0.447059,0.447059}%
\pgfsetfillcolor{currentfill}%
\pgfsetlinewidth{1.003750pt}%
\definecolor{currentstroke}{rgb}{0.266667,0.266667,0.266667}%
\pgfsetstrokecolor{currentstroke}%
\pgfsetdash{}{0pt}%
\pgfpathmoveto{\pgfqpoint{1.728072in}{1.479036in}}%
\pgfpathlineto{\pgfqpoint{1.965552in}{1.479036in}}%
\pgfpathlineto{\pgfqpoint{1.965552in}{1.483249in}}%
\pgfpathlineto{\pgfqpoint{1.728072in}{1.483249in}}%
\pgfpathlineto{\pgfqpoint{1.728072in}{1.479036in}}%
\pgfpathclose%
\pgfusepath{stroke,fill}%
\end{pgfscope}%
\begin{pgfscope}%
\pgfpathrectangle{\pgfqpoint{0.868056in}{0.555556in}}{\pgfqpoint{3.993056in}{1.888889in}}%
\pgfusepath{clip}%
\pgfsetbuttcap%
\pgfsetmiterjoin%
\definecolor{currentfill}{rgb}{0.447059,0.447059,0.447059}%
\pgfsetfillcolor{currentfill}%
\pgfsetlinewidth{1.003750pt}%
\definecolor{currentstroke}{rgb}{0.266667,0.266667,0.266667}%
\pgfsetstrokecolor{currentstroke}%
\pgfsetdash{}{0pt}%
\pgfpathmoveto{\pgfqpoint{2.067329in}{1.588591in}}%
\pgfpathlineto{\pgfqpoint{2.304809in}{1.588591in}}%
\pgfpathlineto{\pgfqpoint{2.304809in}{1.591510in}}%
\pgfpathlineto{\pgfqpoint{2.067329in}{1.591510in}}%
\pgfpathlineto{\pgfqpoint{2.067329in}{1.588591in}}%
\pgfpathclose%
\pgfusepath{stroke,fill}%
\end{pgfscope}%
\begin{pgfscope}%
\pgfpathrectangle{\pgfqpoint{0.868056in}{0.555556in}}{\pgfqpoint{3.993056in}{1.888889in}}%
\pgfusepath{clip}%
\pgfsetbuttcap%
\pgfsetmiterjoin%
\definecolor{currentfill}{rgb}{0.447059,0.447059,0.447059}%
\pgfsetfillcolor{currentfill}%
\pgfsetlinewidth{1.003750pt}%
\definecolor{currentstroke}{rgb}{0.266667,0.266667,0.266667}%
\pgfsetstrokecolor{currentstroke}%
\pgfsetdash{}{0pt}%
\pgfpathmoveto{\pgfqpoint{2.406586in}{1.639591in}}%
\pgfpathlineto{\pgfqpoint{2.644066in}{1.639591in}}%
\pgfpathlineto{\pgfqpoint{2.644066in}{1.642282in}}%
\pgfpathlineto{\pgfqpoint{2.406586in}{1.642282in}}%
\pgfpathlineto{\pgfqpoint{2.406586in}{1.639591in}}%
\pgfpathclose%
\pgfusepath{stroke,fill}%
\end{pgfscope}%
\begin{pgfscope}%
\pgfpathrectangle{\pgfqpoint{0.868056in}{0.555556in}}{\pgfqpoint{3.993056in}{1.888889in}}%
\pgfusepath{clip}%
\pgfsetbuttcap%
\pgfsetmiterjoin%
\definecolor{currentfill}{rgb}{0.447059,0.447059,0.447059}%
\pgfsetfillcolor{currentfill}%
\pgfsetlinewidth{1.003750pt}%
\definecolor{currentstroke}{rgb}{0.266667,0.266667,0.266667}%
\pgfsetstrokecolor{currentstroke}%
\pgfsetdash{}{0pt}%
\pgfpathmoveto{\pgfqpoint{2.745843in}{1.835495in}}%
\pgfpathlineto{\pgfqpoint{2.983323in}{1.835495in}}%
\pgfpathlineto{\pgfqpoint{2.983323in}{1.838528in}}%
\pgfpathlineto{\pgfqpoint{2.745843in}{1.838528in}}%
\pgfpathlineto{\pgfqpoint{2.745843in}{1.835495in}}%
\pgfpathclose%
\pgfusepath{stroke,fill}%
\end{pgfscope}%
\begin{pgfscope}%
\pgfpathrectangle{\pgfqpoint{0.868056in}{0.555556in}}{\pgfqpoint{3.993056in}{1.888889in}}%
\pgfusepath{clip}%
\pgfsetbuttcap%
\pgfsetmiterjoin%
\definecolor{currentfill}{rgb}{0.447059,0.447059,0.447059}%
\pgfsetfillcolor{currentfill}%
\pgfsetlinewidth{1.003750pt}%
\definecolor{currentstroke}{rgb}{0.266667,0.266667,0.266667}%
\pgfsetstrokecolor{currentstroke}%
\pgfsetdash{}{0pt}%
\pgfpathmoveto{\pgfqpoint{3.085100in}{1.564815in}}%
\pgfpathlineto{\pgfqpoint{3.322580in}{1.564815in}}%
\pgfpathlineto{\pgfqpoint{3.322580in}{1.567817in}}%
\pgfpathlineto{\pgfqpoint{3.085100in}{1.567817in}}%
\pgfpathlineto{\pgfqpoint{3.085100in}{1.564815in}}%
\pgfpathclose%
\pgfusepath{stroke,fill}%
\end{pgfscope}%
\begin{pgfscope}%
\pgfpathrectangle{\pgfqpoint{0.868056in}{0.555556in}}{\pgfqpoint{3.993056in}{1.888889in}}%
\pgfusepath{clip}%
\pgfsetbuttcap%
\pgfsetmiterjoin%
\definecolor{currentfill}{rgb}{0.447059,0.447059,0.447059}%
\pgfsetfillcolor{currentfill}%
\pgfsetlinewidth{1.003750pt}%
\definecolor{currentstroke}{rgb}{0.266667,0.266667,0.266667}%
\pgfsetstrokecolor{currentstroke}%
\pgfsetdash{}{0pt}%
\pgfpathmoveto{\pgfqpoint{3.424357in}{1.866466in}}%
\pgfpathlineto{\pgfqpoint{3.661837in}{1.866466in}}%
\pgfpathlineto{\pgfqpoint{3.661837in}{1.869799in}}%
\pgfpathlineto{\pgfqpoint{3.424357in}{1.869799in}}%
\pgfpathlineto{\pgfqpoint{3.424357in}{1.866466in}}%
\pgfpathclose%
\pgfusepath{stroke,fill}%
\end{pgfscope}%
\begin{pgfscope}%
\pgfpathrectangle{\pgfqpoint{0.868056in}{0.555556in}}{\pgfqpoint{3.993056in}{1.888889in}}%
\pgfusepath{clip}%
\pgfsetbuttcap%
\pgfsetmiterjoin%
\definecolor{currentfill}{rgb}{0.447059,0.447059,0.447059}%
\pgfsetfillcolor{currentfill}%
\pgfsetlinewidth{1.003750pt}%
\definecolor{currentstroke}{rgb}{0.266667,0.266667,0.266667}%
\pgfsetstrokecolor{currentstroke}%
\pgfsetdash{}{0pt}%
\pgfpathmoveto{\pgfqpoint{3.763615in}{2.089779in}}%
\pgfpathlineto{\pgfqpoint{4.001094in}{2.089779in}}%
\pgfpathlineto{\pgfqpoint{4.001094in}{2.093733in}}%
\pgfpathlineto{\pgfqpoint{3.763615in}{2.093733in}}%
\pgfpathlineto{\pgfqpoint{3.763615in}{2.089779in}}%
\pgfpathclose%
\pgfusepath{stroke,fill}%
\end{pgfscope}%
\begin{pgfscope}%
\pgfpathrectangle{\pgfqpoint{0.868056in}{0.555556in}}{\pgfqpoint{3.993056in}{1.888889in}}%
\pgfusepath{clip}%
\pgfsetbuttcap%
\pgfsetmiterjoin%
\definecolor{currentfill}{rgb}{0.447059,0.447059,0.447059}%
\pgfsetfillcolor{currentfill}%
\pgfsetlinewidth{1.003750pt}%
\definecolor{currentstroke}{rgb}{0.266667,0.266667,0.266667}%
\pgfsetstrokecolor{currentstroke}%
\pgfsetdash{}{0pt}%
\pgfpathmoveto{\pgfqpoint{4.102872in}{2.169462in}}%
\pgfpathlineto{\pgfqpoint{4.340352in}{2.169462in}}%
\pgfpathlineto{\pgfqpoint{4.340352in}{2.175321in}}%
\pgfpathlineto{\pgfqpoint{4.102872in}{2.175321in}}%
\pgfpathlineto{\pgfqpoint{4.102872in}{2.169462in}}%
\pgfpathclose%
\pgfusepath{stroke,fill}%
\end{pgfscope}%
\begin{pgfscope}%
\pgfpathrectangle{\pgfqpoint{0.868056in}{0.555556in}}{\pgfqpoint{3.993056in}{1.888889in}}%
\pgfusepath{clip}%
\pgfsetbuttcap%
\pgfsetmiterjoin%
\definecolor{currentfill}{rgb}{0.447059,0.447059,0.447059}%
\pgfsetfillcolor{currentfill}%
\pgfsetlinewidth{1.003750pt}%
\definecolor{currentstroke}{rgb}{0.266667,0.266667,0.266667}%
\pgfsetstrokecolor{currentstroke}%
\pgfsetdash{}{0pt}%
\pgfpathmoveto{\pgfqpoint{4.442129in}{2.331383in}}%
\pgfpathlineto{\pgfqpoint{4.679609in}{2.331383in}}%
\pgfpathlineto{\pgfqpoint{4.679609in}{2.339923in}}%
\pgfpathlineto{\pgfqpoint{4.442129in}{2.339923in}}%
\pgfpathlineto{\pgfqpoint{4.442129in}{2.331383in}}%
\pgfpathclose%
\pgfusepath{stroke,fill}%
\end{pgfscope}%
\begin{pgfscope}%
\pgfpathrectangle{\pgfqpoint{0.868056in}{0.555556in}}{\pgfqpoint{3.993056in}{1.888889in}}%
\pgfusepath{clip}%
\pgfsetbuttcap%
\pgfsetmiterjoin%
\definecolor{currentfill}{rgb}{0.447059,0.447059,0.447059}%
\pgfsetfillcolor{currentfill}%
\pgfsetlinewidth{1.003750pt}%
\definecolor{currentstroke}{rgb}{0.266667,0.266667,0.266667}%
\pgfsetstrokecolor{currentstroke}%
\pgfsetdash{}{0pt}%
\pgfpathmoveto{\pgfqpoint{1.049558in}{2.184233in}}%
\pgfpathlineto{\pgfqpoint{1.287038in}{2.184233in}}%
\pgfpathlineto{\pgfqpoint{1.287038in}{2.184233in}}%
\pgfpathlineto{\pgfqpoint{1.049558in}{2.184233in}}%
\pgfpathlineto{\pgfqpoint{1.049558in}{2.184233in}}%
\pgfpathclose%
\pgfusepath{stroke,fill}%
\end{pgfscope}%
\begin{pgfscope}%
\pgfpathrectangle{\pgfqpoint{0.868056in}{0.555556in}}{\pgfqpoint{3.993056in}{1.888889in}}%
\pgfusepath{clip}%
\pgfsetbuttcap%
\pgfsetmiterjoin%
\definecolor{currentfill}{rgb}{0.447059,0.447059,0.447059}%
\pgfsetfillcolor{currentfill}%
\pgfsetlinewidth{1.003750pt}%
\definecolor{currentstroke}{rgb}{0.266667,0.266667,0.266667}%
\pgfsetstrokecolor{currentstroke}%
\pgfsetdash{}{0pt}%
\pgfpathmoveto{\pgfqpoint{1.388815in}{1.776712in}}%
\pgfpathlineto{\pgfqpoint{1.626295in}{1.776712in}}%
\pgfpathlineto{\pgfqpoint{1.626295in}{1.776712in}}%
\pgfpathlineto{\pgfqpoint{1.388815in}{1.776712in}}%
\pgfpathlineto{\pgfqpoint{1.388815in}{1.776712in}}%
\pgfpathclose%
\pgfusepath{stroke,fill}%
\end{pgfscope}%
\begin{pgfscope}%
\pgfpathrectangle{\pgfqpoint{0.868056in}{0.555556in}}{\pgfqpoint{3.993056in}{1.888889in}}%
\pgfusepath{clip}%
\pgfsetbuttcap%
\pgfsetmiterjoin%
\definecolor{currentfill}{rgb}{0.447059,0.447059,0.447059}%
\pgfsetfillcolor{currentfill}%
\pgfsetlinewidth{1.003750pt}%
\definecolor{currentstroke}{rgb}{0.266667,0.266667,0.266667}%
\pgfsetstrokecolor{currentstroke}%
\pgfsetdash{}{0pt}%
\pgfpathmoveto{\pgfqpoint{1.728072in}{1.483249in}}%
\pgfpathlineto{\pgfqpoint{1.965552in}{1.483249in}}%
\pgfpathlineto{\pgfqpoint{1.965552in}{1.486489in}}%
\pgfpathlineto{\pgfqpoint{1.728072in}{1.486489in}}%
\pgfpathlineto{\pgfqpoint{1.728072in}{1.483249in}}%
\pgfpathclose%
\pgfusepath{stroke,fill}%
\end{pgfscope}%
\begin{pgfscope}%
\pgfpathrectangle{\pgfqpoint{0.868056in}{0.555556in}}{\pgfqpoint{3.993056in}{1.888889in}}%
\pgfusepath{clip}%
\pgfsetbuttcap%
\pgfsetmiterjoin%
\definecolor{currentfill}{rgb}{0.447059,0.447059,0.447059}%
\pgfsetfillcolor{currentfill}%
\pgfsetlinewidth{1.003750pt}%
\definecolor{currentstroke}{rgb}{0.266667,0.266667,0.266667}%
\pgfsetstrokecolor{currentstroke}%
\pgfsetdash{}{0pt}%
\pgfpathmoveto{\pgfqpoint{2.067329in}{1.591510in}}%
\pgfpathlineto{\pgfqpoint{2.304809in}{1.591510in}}%
\pgfpathlineto{\pgfqpoint{2.304809in}{1.594088in}}%
\pgfpathlineto{\pgfqpoint{2.067329in}{1.594088in}}%
\pgfpathlineto{\pgfqpoint{2.067329in}{1.591510in}}%
\pgfpathclose%
\pgfusepath{stroke,fill}%
\end{pgfscope}%
\begin{pgfscope}%
\pgfpathrectangle{\pgfqpoint{0.868056in}{0.555556in}}{\pgfqpoint{3.993056in}{1.888889in}}%
\pgfusepath{clip}%
\pgfsetbuttcap%
\pgfsetmiterjoin%
\definecolor{currentfill}{rgb}{0.447059,0.447059,0.447059}%
\pgfsetfillcolor{currentfill}%
\pgfsetlinewidth{1.003750pt}%
\definecolor{currentstroke}{rgb}{0.266667,0.266667,0.266667}%
\pgfsetstrokecolor{currentstroke}%
\pgfsetdash{}{0pt}%
\pgfpathmoveto{\pgfqpoint{2.406586in}{1.642282in}}%
\pgfpathlineto{\pgfqpoint{2.644066in}{1.642282in}}%
\pgfpathlineto{\pgfqpoint{2.644066in}{1.644166in}}%
\pgfpathlineto{\pgfqpoint{2.406586in}{1.644166in}}%
\pgfpathlineto{\pgfqpoint{2.406586in}{1.642282in}}%
\pgfpathclose%
\pgfusepath{stroke,fill}%
\end{pgfscope}%
\begin{pgfscope}%
\pgfpathrectangle{\pgfqpoint{0.868056in}{0.555556in}}{\pgfqpoint{3.993056in}{1.888889in}}%
\pgfusepath{clip}%
\pgfsetbuttcap%
\pgfsetmiterjoin%
\definecolor{currentfill}{rgb}{0.447059,0.447059,0.447059}%
\pgfsetfillcolor{currentfill}%
\pgfsetlinewidth{1.003750pt}%
\definecolor{currentstroke}{rgb}{0.266667,0.266667,0.266667}%
\pgfsetstrokecolor{currentstroke}%
\pgfsetdash{}{0pt}%
\pgfpathmoveto{\pgfqpoint{2.745843in}{1.838528in}}%
\pgfpathlineto{\pgfqpoint{2.983323in}{1.838528in}}%
\pgfpathlineto{\pgfqpoint{2.983323in}{1.842762in}}%
\pgfpathlineto{\pgfqpoint{2.745843in}{1.842762in}}%
\pgfpathlineto{\pgfqpoint{2.745843in}{1.838528in}}%
\pgfpathclose%
\pgfusepath{stroke,fill}%
\end{pgfscope}%
\begin{pgfscope}%
\pgfpathrectangle{\pgfqpoint{0.868056in}{0.555556in}}{\pgfqpoint{3.993056in}{1.888889in}}%
\pgfusepath{clip}%
\pgfsetbuttcap%
\pgfsetmiterjoin%
\definecolor{currentfill}{rgb}{0.447059,0.447059,0.447059}%
\pgfsetfillcolor{currentfill}%
\pgfsetlinewidth{1.003750pt}%
\definecolor{currentstroke}{rgb}{0.266667,0.266667,0.266667}%
\pgfsetstrokecolor{currentstroke}%
\pgfsetdash{}{0pt}%
\pgfpathmoveto{\pgfqpoint{3.085100in}{1.567817in}}%
\pgfpathlineto{\pgfqpoint{3.322580in}{1.567817in}}%
\pgfpathlineto{\pgfqpoint{3.322580in}{1.570467in}}%
\pgfpathlineto{\pgfqpoint{3.085100in}{1.570467in}}%
\pgfpathlineto{\pgfqpoint{3.085100in}{1.567817in}}%
\pgfpathclose%
\pgfusepath{stroke,fill}%
\end{pgfscope}%
\begin{pgfscope}%
\pgfpathrectangle{\pgfqpoint{0.868056in}{0.555556in}}{\pgfqpoint{3.993056in}{1.888889in}}%
\pgfusepath{clip}%
\pgfsetbuttcap%
\pgfsetmiterjoin%
\definecolor{currentfill}{rgb}{0.447059,0.447059,0.447059}%
\pgfsetfillcolor{currentfill}%
\pgfsetlinewidth{1.003750pt}%
\definecolor{currentstroke}{rgb}{0.266667,0.266667,0.266667}%
\pgfsetstrokecolor{currentstroke}%
\pgfsetdash{}{0pt}%
\pgfpathmoveto{\pgfqpoint{3.424357in}{1.869799in}}%
\pgfpathlineto{\pgfqpoint{3.661837in}{1.869799in}}%
\pgfpathlineto{\pgfqpoint{3.661837in}{1.873194in}}%
\pgfpathlineto{\pgfqpoint{3.424357in}{1.873194in}}%
\pgfpathlineto{\pgfqpoint{3.424357in}{1.869799in}}%
\pgfpathclose%
\pgfusepath{stroke,fill}%
\end{pgfscope}%
\begin{pgfscope}%
\pgfpathrectangle{\pgfqpoint{0.868056in}{0.555556in}}{\pgfqpoint{3.993056in}{1.888889in}}%
\pgfusepath{clip}%
\pgfsetbuttcap%
\pgfsetmiterjoin%
\definecolor{currentfill}{rgb}{0.447059,0.447059,0.447059}%
\pgfsetfillcolor{currentfill}%
\pgfsetlinewidth{1.003750pt}%
\definecolor{currentstroke}{rgb}{0.266667,0.266667,0.266667}%
\pgfsetstrokecolor{currentstroke}%
\pgfsetdash{}{0pt}%
\pgfpathmoveto{\pgfqpoint{3.763615in}{2.093733in}}%
\pgfpathlineto{\pgfqpoint{4.001094in}{2.093733in}}%
\pgfpathlineto{\pgfqpoint{4.001094in}{2.097222in}}%
\pgfpathlineto{\pgfqpoint{3.763615in}{2.097222in}}%
\pgfpathlineto{\pgfqpoint{3.763615in}{2.093733in}}%
\pgfpathclose%
\pgfusepath{stroke,fill}%
\end{pgfscope}%
\begin{pgfscope}%
\pgfpathrectangle{\pgfqpoint{0.868056in}{0.555556in}}{\pgfqpoint{3.993056in}{1.888889in}}%
\pgfusepath{clip}%
\pgfsetbuttcap%
\pgfsetmiterjoin%
\definecolor{currentfill}{rgb}{0.447059,0.447059,0.447059}%
\pgfsetfillcolor{currentfill}%
\pgfsetlinewidth{1.003750pt}%
\definecolor{currentstroke}{rgb}{0.266667,0.266667,0.266667}%
\pgfsetstrokecolor{currentstroke}%
\pgfsetdash{}{0pt}%
\pgfpathmoveto{\pgfqpoint{4.102872in}{2.175321in}}%
\pgfpathlineto{\pgfqpoint{4.340352in}{2.175321in}}%
\pgfpathlineto{\pgfqpoint{4.340352in}{2.178001in}}%
\pgfpathlineto{\pgfqpoint{4.102872in}{2.178001in}}%
\pgfpathlineto{\pgfqpoint{4.102872in}{2.175321in}}%
\pgfpathclose%
\pgfusepath{stroke,fill}%
\end{pgfscope}%
\begin{pgfscope}%
\pgfpathrectangle{\pgfqpoint{0.868056in}{0.555556in}}{\pgfqpoint{3.993056in}{1.888889in}}%
\pgfusepath{clip}%
\pgfsetbuttcap%
\pgfsetmiterjoin%
\definecolor{currentfill}{rgb}{0.447059,0.447059,0.447059}%
\pgfsetfillcolor{currentfill}%
\pgfsetlinewidth{1.003750pt}%
\definecolor{currentstroke}{rgb}{0.266667,0.266667,0.266667}%
\pgfsetstrokecolor{currentstroke}%
\pgfsetdash{}{0pt}%
\pgfpathmoveto{\pgfqpoint{4.442129in}{2.339923in}}%
\pgfpathlineto{\pgfqpoint{4.679609in}{2.339923in}}%
\pgfpathlineto{\pgfqpoint{4.679609in}{2.343587in}}%
\pgfpathlineto{\pgfqpoint{4.442129in}{2.343587in}}%
\pgfpathlineto{\pgfqpoint{4.442129in}{2.339923in}}%
\pgfpathclose%
\pgfusepath{stroke,fill}%
\end{pgfscope}%
\begin{pgfscope}%
\pgfpathrectangle{\pgfqpoint{0.868056in}{0.555556in}}{\pgfqpoint{3.993056in}{1.888889in}}%
\pgfusepath{clip}%
\pgfsetbuttcap%
\pgfsetmiterjoin%
\definecolor{currentfill}{rgb}{0.447059,0.447059,0.447059}%
\pgfsetfillcolor{currentfill}%
\pgfsetlinewidth{1.003750pt}%
\definecolor{currentstroke}{rgb}{0.266667,0.266667,0.266667}%
\pgfsetstrokecolor{currentstroke}%
\pgfsetdash{}{0pt}%
\pgfpathmoveto{\pgfqpoint{1.049558in}{2.184233in}}%
\pgfpathlineto{\pgfqpoint{1.287038in}{2.184233in}}%
\pgfpathlineto{\pgfqpoint{1.287038in}{2.188197in}}%
\pgfpathlineto{\pgfqpoint{1.049558in}{2.188197in}}%
\pgfpathlineto{\pgfqpoint{1.049558in}{2.184233in}}%
\pgfpathclose%
\pgfusepath{stroke,fill}%
\end{pgfscope}%
\begin{pgfscope}%
\pgfpathrectangle{\pgfqpoint{0.868056in}{0.555556in}}{\pgfqpoint{3.993056in}{1.888889in}}%
\pgfusepath{clip}%
\pgfsetbuttcap%
\pgfsetmiterjoin%
\definecolor{currentfill}{rgb}{0.447059,0.447059,0.447059}%
\pgfsetfillcolor{currentfill}%
\pgfsetlinewidth{1.003750pt}%
\definecolor{currentstroke}{rgb}{0.266667,0.266667,0.266667}%
\pgfsetstrokecolor{currentstroke}%
\pgfsetdash{}{0pt}%
\pgfpathmoveto{\pgfqpoint{1.388815in}{1.776712in}}%
\pgfpathlineto{\pgfqpoint{1.626295in}{1.776712in}}%
\pgfpathlineto{\pgfqpoint{1.626295in}{1.780893in}}%
\pgfpathlineto{\pgfqpoint{1.388815in}{1.780893in}}%
\pgfpathlineto{\pgfqpoint{1.388815in}{1.776712in}}%
\pgfpathclose%
\pgfusepath{stroke,fill}%
\end{pgfscope}%
\begin{pgfscope}%
\pgfpathrectangle{\pgfqpoint{0.868056in}{0.555556in}}{\pgfqpoint{3.993056in}{1.888889in}}%
\pgfusepath{clip}%
\pgfsetbuttcap%
\pgfsetmiterjoin%
\definecolor{currentfill}{rgb}{0.447059,0.447059,0.447059}%
\pgfsetfillcolor{currentfill}%
\pgfsetlinewidth{1.003750pt}%
\definecolor{currentstroke}{rgb}{0.266667,0.266667,0.266667}%
\pgfsetstrokecolor{currentstroke}%
\pgfsetdash{}{0pt}%
\pgfpathmoveto{\pgfqpoint{1.728072in}{1.486489in}}%
\pgfpathlineto{\pgfqpoint{1.965552in}{1.486489in}}%
\pgfpathlineto{\pgfqpoint{1.965552in}{1.489873in}}%
\pgfpathlineto{\pgfqpoint{1.728072in}{1.489873in}}%
\pgfpathlineto{\pgfqpoint{1.728072in}{1.486489in}}%
\pgfpathclose%
\pgfusepath{stroke,fill}%
\end{pgfscope}%
\begin{pgfscope}%
\pgfpathrectangle{\pgfqpoint{0.868056in}{0.555556in}}{\pgfqpoint{3.993056in}{1.888889in}}%
\pgfusepath{clip}%
\pgfsetbuttcap%
\pgfsetmiterjoin%
\definecolor{currentfill}{rgb}{0.447059,0.447059,0.447059}%
\pgfsetfillcolor{currentfill}%
\pgfsetlinewidth{1.003750pt}%
\definecolor{currentstroke}{rgb}{0.266667,0.266667,0.266667}%
\pgfsetstrokecolor{currentstroke}%
\pgfsetdash{}{0pt}%
\pgfpathmoveto{\pgfqpoint{2.067329in}{1.594088in}}%
\pgfpathlineto{\pgfqpoint{2.304809in}{1.594088in}}%
\pgfpathlineto{\pgfqpoint{2.304809in}{1.597048in}}%
\pgfpathlineto{\pgfqpoint{2.067329in}{1.597048in}}%
\pgfpathlineto{\pgfqpoint{2.067329in}{1.594088in}}%
\pgfpathclose%
\pgfusepath{stroke,fill}%
\end{pgfscope}%
\begin{pgfscope}%
\pgfpathrectangle{\pgfqpoint{0.868056in}{0.555556in}}{\pgfqpoint{3.993056in}{1.888889in}}%
\pgfusepath{clip}%
\pgfsetbuttcap%
\pgfsetmiterjoin%
\definecolor{currentfill}{rgb}{0.447059,0.447059,0.447059}%
\pgfsetfillcolor{currentfill}%
\pgfsetlinewidth{1.003750pt}%
\definecolor{currentstroke}{rgb}{0.266667,0.266667,0.266667}%
\pgfsetstrokecolor{currentstroke}%
\pgfsetdash{}{0pt}%
\pgfpathmoveto{\pgfqpoint{2.406586in}{1.644166in}}%
\pgfpathlineto{\pgfqpoint{2.644066in}{1.644166in}}%
\pgfpathlineto{\pgfqpoint{2.644066in}{1.647468in}}%
\pgfpathlineto{\pgfqpoint{2.406586in}{1.647468in}}%
\pgfpathlineto{\pgfqpoint{2.406586in}{1.644166in}}%
\pgfpathclose%
\pgfusepath{stroke,fill}%
\end{pgfscope}%
\begin{pgfscope}%
\pgfpathrectangle{\pgfqpoint{0.868056in}{0.555556in}}{\pgfqpoint{3.993056in}{1.888889in}}%
\pgfusepath{clip}%
\pgfsetbuttcap%
\pgfsetmiterjoin%
\definecolor{currentfill}{rgb}{0.447059,0.447059,0.447059}%
\pgfsetfillcolor{currentfill}%
\pgfsetlinewidth{1.003750pt}%
\definecolor{currentstroke}{rgb}{0.266667,0.266667,0.266667}%
\pgfsetstrokecolor{currentstroke}%
\pgfsetdash{}{0pt}%
\pgfpathmoveto{\pgfqpoint{2.745843in}{1.842762in}}%
\pgfpathlineto{\pgfqpoint{2.983323in}{1.842762in}}%
\pgfpathlineto{\pgfqpoint{2.983323in}{1.845784in}}%
\pgfpathlineto{\pgfqpoint{2.745843in}{1.845784in}}%
\pgfpathlineto{\pgfqpoint{2.745843in}{1.842762in}}%
\pgfpathclose%
\pgfusepath{stroke,fill}%
\end{pgfscope}%
\begin{pgfscope}%
\pgfpathrectangle{\pgfqpoint{0.868056in}{0.555556in}}{\pgfqpoint{3.993056in}{1.888889in}}%
\pgfusepath{clip}%
\pgfsetbuttcap%
\pgfsetmiterjoin%
\definecolor{currentfill}{rgb}{0.447059,0.447059,0.447059}%
\pgfsetfillcolor{currentfill}%
\pgfsetlinewidth{1.003750pt}%
\definecolor{currentstroke}{rgb}{0.266667,0.266667,0.266667}%
\pgfsetstrokecolor{currentstroke}%
\pgfsetdash{}{0pt}%
\pgfpathmoveto{\pgfqpoint{3.085100in}{1.570467in}}%
\pgfpathlineto{\pgfqpoint{3.322580in}{1.570467in}}%
\pgfpathlineto{\pgfqpoint{3.322580in}{1.573261in}}%
\pgfpathlineto{\pgfqpoint{3.085100in}{1.573261in}}%
\pgfpathlineto{\pgfqpoint{3.085100in}{1.570467in}}%
\pgfpathclose%
\pgfusepath{stroke,fill}%
\end{pgfscope}%
\begin{pgfscope}%
\pgfpathrectangle{\pgfqpoint{0.868056in}{0.555556in}}{\pgfqpoint{3.993056in}{1.888889in}}%
\pgfusepath{clip}%
\pgfsetbuttcap%
\pgfsetmiterjoin%
\definecolor{currentfill}{rgb}{0.447059,0.447059,0.447059}%
\pgfsetfillcolor{currentfill}%
\pgfsetlinewidth{1.003750pt}%
\definecolor{currentstroke}{rgb}{0.266667,0.266667,0.266667}%
\pgfsetstrokecolor{currentstroke}%
\pgfsetdash{}{0pt}%
\pgfpathmoveto{\pgfqpoint{3.424357in}{1.873194in}}%
\pgfpathlineto{\pgfqpoint{3.661837in}{1.873194in}}%
\pgfpathlineto{\pgfqpoint{3.661837in}{1.875999in}}%
\pgfpathlineto{\pgfqpoint{3.424357in}{1.875999in}}%
\pgfpathlineto{\pgfqpoint{3.424357in}{1.873194in}}%
\pgfpathclose%
\pgfusepath{stroke,fill}%
\end{pgfscope}%
\begin{pgfscope}%
\pgfpathrectangle{\pgfqpoint{0.868056in}{0.555556in}}{\pgfqpoint{3.993056in}{1.888889in}}%
\pgfusepath{clip}%
\pgfsetbuttcap%
\pgfsetmiterjoin%
\definecolor{currentfill}{rgb}{0.447059,0.447059,0.447059}%
\pgfsetfillcolor{currentfill}%
\pgfsetlinewidth{1.003750pt}%
\definecolor{currentstroke}{rgb}{0.266667,0.266667,0.266667}%
\pgfsetstrokecolor{currentstroke}%
\pgfsetdash{}{0pt}%
\pgfpathmoveto{\pgfqpoint{3.763615in}{2.097222in}}%
\pgfpathlineto{\pgfqpoint{4.001094in}{2.097222in}}%
\pgfpathlineto{\pgfqpoint{4.001094in}{2.099602in}}%
\pgfpathlineto{\pgfqpoint{3.763615in}{2.099602in}}%
\pgfpathlineto{\pgfqpoint{3.763615in}{2.097222in}}%
\pgfpathclose%
\pgfusepath{stroke,fill}%
\end{pgfscope}%
\begin{pgfscope}%
\pgfpathrectangle{\pgfqpoint{0.868056in}{0.555556in}}{\pgfqpoint{3.993056in}{1.888889in}}%
\pgfusepath{clip}%
\pgfsetbuttcap%
\pgfsetmiterjoin%
\definecolor{currentfill}{rgb}{0.447059,0.447059,0.447059}%
\pgfsetfillcolor{currentfill}%
\pgfsetlinewidth{1.003750pt}%
\definecolor{currentstroke}{rgb}{0.266667,0.266667,0.266667}%
\pgfsetstrokecolor{currentstroke}%
\pgfsetdash{}{0pt}%
\pgfpathmoveto{\pgfqpoint{4.102872in}{2.178001in}}%
\pgfpathlineto{\pgfqpoint{4.340352in}{2.178001in}}%
\pgfpathlineto{\pgfqpoint{4.340352in}{2.180610in}}%
\pgfpathlineto{\pgfqpoint{4.102872in}{2.180610in}}%
\pgfpathlineto{\pgfqpoint{4.102872in}{2.178001in}}%
\pgfpathclose%
\pgfusepath{stroke,fill}%
\end{pgfscope}%
\begin{pgfscope}%
\pgfpathrectangle{\pgfqpoint{0.868056in}{0.555556in}}{\pgfqpoint{3.993056in}{1.888889in}}%
\pgfusepath{clip}%
\pgfsetbuttcap%
\pgfsetmiterjoin%
\definecolor{currentfill}{rgb}{0.447059,0.447059,0.447059}%
\pgfsetfillcolor{currentfill}%
\pgfsetlinewidth{1.003750pt}%
\definecolor{currentstroke}{rgb}{0.266667,0.266667,0.266667}%
\pgfsetstrokecolor{currentstroke}%
\pgfsetdash{}{0pt}%
\pgfpathmoveto{\pgfqpoint{4.442129in}{2.343587in}}%
\pgfpathlineto{\pgfqpoint{4.679609in}{2.343587in}}%
\pgfpathlineto{\pgfqpoint{4.679609in}{2.345513in}}%
\pgfpathlineto{\pgfqpoint{4.442129in}{2.345513in}}%
\pgfpathlineto{\pgfqpoint{4.442129in}{2.343587in}}%
\pgfpathclose%
\pgfusepath{stroke,fill}%
\end{pgfscope}%
\begin{pgfscope}%
\pgfpathrectangle{\pgfqpoint{0.868056in}{0.555556in}}{\pgfqpoint{3.993056in}{1.888889in}}%
\pgfusepath{clip}%
\pgfsetbuttcap%
\pgfsetmiterjoin%
\definecolor{currentfill}{rgb}{0.447059,0.447059,0.447059}%
\pgfsetfillcolor{currentfill}%
\pgfsetlinewidth{1.003750pt}%
\definecolor{currentstroke}{rgb}{0.266667,0.266667,0.266667}%
\pgfsetstrokecolor{currentstroke}%
\pgfsetdash{}{0pt}%
\pgfpathmoveto{\pgfqpoint{1.049558in}{2.188197in}}%
\pgfpathlineto{\pgfqpoint{1.287038in}{2.188197in}}%
\pgfpathlineto{\pgfqpoint{1.287038in}{2.191613in}}%
\pgfpathlineto{\pgfqpoint{1.049558in}{2.191613in}}%
\pgfpathlineto{\pgfqpoint{1.049558in}{2.188197in}}%
\pgfpathclose%
\pgfusepath{stroke,fill}%
\end{pgfscope}%
\begin{pgfscope}%
\pgfpathrectangle{\pgfqpoint{0.868056in}{0.555556in}}{\pgfqpoint{3.993056in}{1.888889in}}%
\pgfusepath{clip}%
\pgfsetbuttcap%
\pgfsetmiterjoin%
\definecolor{currentfill}{rgb}{0.447059,0.447059,0.447059}%
\pgfsetfillcolor{currentfill}%
\pgfsetlinewidth{1.003750pt}%
\definecolor{currentstroke}{rgb}{0.266667,0.266667,0.266667}%
\pgfsetstrokecolor{currentstroke}%
\pgfsetdash{}{0pt}%
\pgfpathmoveto{\pgfqpoint{1.388815in}{1.780893in}}%
\pgfpathlineto{\pgfqpoint{1.626295in}{1.780893in}}%
\pgfpathlineto{\pgfqpoint{1.626295in}{1.783284in}}%
\pgfpathlineto{\pgfqpoint{1.388815in}{1.783284in}}%
\pgfpathlineto{\pgfqpoint{1.388815in}{1.780893in}}%
\pgfpathclose%
\pgfusepath{stroke,fill}%
\end{pgfscope}%
\begin{pgfscope}%
\pgfpathrectangle{\pgfqpoint{0.868056in}{0.555556in}}{\pgfqpoint{3.993056in}{1.888889in}}%
\pgfusepath{clip}%
\pgfsetbuttcap%
\pgfsetmiterjoin%
\definecolor{currentfill}{rgb}{0.447059,0.447059,0.447059}%
\pgfsetfillcolor{currentfill}%
\pgfsetlinewidth{1.003750pt}%
\definecolor{currentstroke}{rgb}{0.266667,0.266667,0.266667}%
\pgfsetstrokecolor{currentstroke}%
\pgfsetdash{}{0pt}%
\pgfpathmoveto{\pgfqpoint{1.728072in}{1.489873in}}%
\pgfpathlineto{\pgfqpoint{1.965552in}{1.489873in}}%
\pgfpathlineto{\pgfqpoint{1.965552in}{1.491664in}}%
\pgfpathlineto{\pgfqpoint{1.728072in}{1.491664in}}%
\pgfpathlineto{\pgfqpoint{1.728072in}{1.489873in}}%
\pgfpathclose%
\pgfusepath{stroke,fill}%
\end{pgfscope}%
\begin{pgfscope}%
\pgfpathrectangle{\pgfqpoint{0.868056in}{0.555556in}}{\pgfqpoint{3.993056in}{1.888889in}}%
\pgfusepath{clip}%
\pgfsetbuttcap%
\pgfsetmiterjoin%
\definecolor{currentfill}{rgb}{0.447059,0.447059,0.447059}%
\pgfsetfillcolor{currentfill}%
\pgfsetlinewidth{1.003750pt}%
\definecolor{currentstroke}{rgb}{0.266667,0.266667,0.266667}%
\pgfsetstrokecolor{currentstroke}%
\pgfsetdash{}{0pt}%
\pgfpathmoveto{\pgfqpoint{2.067329in}{1.597048in}}%
\pgfpathlineto{\pgfqpoint{2.304809in}{1.597048in}}%
\pgfpathlineto{\pgfqpoint{2.304809in}{1.598466in}}%
\pgfpathlineto{\pgfqpoint{2.067329in}{1.598466in}}%
\pgfpathlineto{\pgfqpoint{2.067329in}{1.597048in}}%
\pgfpathclose%
\pgfusepath{stroke,fill}%
\end{pgfscope}%
\begin{pgfscope}%
\pgfpathrectangle{\pgfqpoint{0.868056in}{0.555556in}}{\pgfqpoint{3.993056in}{1.888889in}}%
\pgfusepath{clip}%
\pgfsetbuttcap%
\pgfsetmiterjoin%
\definecolor{currentfill}{rgb}{0.447059,0.447059,0.447059}%
\pgfsetfillcolor{currentfill}%
\pgfsetlinewidth{1.003750pt}%
\definecolor{currentstroke}{rgb}{0.266667,0.266667,0.266667}%
\pgfsetstrokecolor{currentstroke}%
\pgfsetdash{}{0pt}%
\pgfpathmoveto{\pgfqpoint{2.406586in}{1.647468in}}%
\pgfpathlineto{\pgfqpoint{2.644066in}{1.647468in}}%
\pgfpathlineto{\pgfqpoint{2.644066in}{1.648659in}}%
\pgfpathlineto{\pgfqpoint{2.406586in}{1.648659in}}%
\pgfpathlineto{\pgfqpoint{2.406586in}{1.647468in}}%
\pgfpathclose%
\pgfusepath{stroke,fill}%
\end{pgfscope}%
\begin{pgfscope}%
\pgfpathrectangle{\pgfqpoint{0.868056in}{0.555556in}}{\pgfqpoint{3.993056in}{1.888889in}}%
\pgfusepath{clip}%
\pgfsetbuttcap%
\pgfsetmiterjoin%
\definecolor{currentfill}{rgb}{0.447059,0.447059,0.447059}%
\pgfsetfillcolor{currentfill}%
\pgfsetlinewidth{1.003750pt}%
\definecolor{currentstroke}{rgb}{0.266667,0.266667,0.266667}%
\pgfsetstrokecolor{currentstroke}%
\pgfsetdash{}{0pt}%
\pgfpathmoveto{\pgfqpoint{2.745843in}{1.845784in}}%
\pgfpathlineto{\pgfqpoint{2.983323in}{1.845784in}}%
\pgfpathlineto{\pgfqpoint{2.983323in}{1.847057in}}%
\pgfpathlineto{\pgfqpoint{2.745843in}{1.847057in}}%
\pgfpathlineto{\pgfqpoint{2.745843in}{1.845784in}}%
\pgfpathclose%
\pgfusepath{stroke,fill}%
\end{pgfscope}%
\begin{pgfscope}%
\pgfpathrectangle{\pgfqpoint{0.868056in}{0.555556in}}{\pgfqpoint{3.993056in}{1.888889in}}%
\pgfusepath{clip}%
\pgfsetbuttcap%
\pgfsetmiterjoin%
\definecolor{currentfill}{rgb}{0.447059,0.447059,0.447059}%
\pgfsetfillcolor{currentfill}%
\pgfsetlinewidth{1.003750pt}%
\definecolor{currentstroke}{rgb}{0.266667,0.266667,0.266667}%
\pgfsetstrokecolor{currentstroke}%
\pgfsetdash{}{0pt}%
\pgfpathmoveto{\pgfqpoint{3.085100in}{1.573261in}}%
\pgfpathlineto{\pgfqpoint{3.322580in}{1.573261in}}%
\pgfpathlineto{\pgfqpoint{3.322580in}{1.574193in}}%
\pgfpathlineto{\pgfqpoint{3.085100in}{1.574193in}}%
\pgfpathlineto{\pgfqpoint{3.085100in}{1.573261in}}%
\pgfpathclose%
\pgfusepath{stroke,fill}%
\end{pgfscope}%
\begin{pgfscope}%
\pgfpathrectangle{\pgfqpoint{0.868056in}{0.555556in}}{\pgfqpoint{3.993056in}{1.888889in}}%
\pgfusepath{clip}%
\pgfsetbuttcap%
\pgfsetmiterjoin%
\definecolor{currentfill}{rgb}{0.447059,0.447059,0.447059}%
\pgfsetfillcolor{currentfill}%
\pgfsetlinewidth{1.003750pt}%
\definecolor{currentstroke}{rgb}{0.266667,0.266667,0.266667}%
\pgfsetstrokecolor{currentstroke}%
\pgfsetdash{}{0pt}%
\pgfpathmoveto{\pgfqpoint{3.424357in}{1.875999in}}%
\pgfpathlineto{\pgfqpoint{3.661837in}{1.875999in}}%
\pgfpathlineto{\pgfqpoint{3.661837in}{1.878038in}}%
\pgfpathlineto{\pgfqpoint{3.424357in}{1.878038in}}%
\pgfpathlineto{\pgfqpoint{3.424357in}{1.875999in}}%
\pgfpathclose%
\pgfusepath{stroke,fill}%
\end{pgfscope}%
\begin{pgfscope}%
\pgfpathrectangle{\pgfqpoint{0.868056in}{0.555556in}}{\pgfqpoint{3.993056in}{1.888889in}}%
\pgfusepath{clip}%
\pgfsetbuttcap%
\pgfsetmiterjoin%
\definecolor{currentfill}{rgb}{0.447059,0.447059,0.447059}%
\pgfsetfillcolor{currentfill}%
\pgfsetlinewidth{1.003750pt}%
\definecolor{currentstroke}{rgb}{0.266667,0.266667,0.266667}%
\pgfsetstrokecolor{currentstroke}%
\pgfsetdash{}{0pt}%
\pgfpathmoveto{\pgfqpoint{3.763615in}{2.099602in}}%
\pgfpathlineto{\pgfqpoint{4.001094in}{2.099602in}}%
\pgfpathlineto{\pgfqpoint{4.001094in}{2.102076in}}%
\pgfpathlineto{\pgfqpoint{3.763615in}{2.102076in}}%
\pgfpathlineto{\pgfqpoint{3.763615in}{2.099602in}}%
\pgfpathclose%
\pgfusepath{stroke,fill}%
\end{pgfscope}%
\begin{pgfscope}%
\pgfpathrectangle{\pgfqpoint{0.868056in}{0.555556in}}{\pgfqpoint{3.993056in}{1.888889in}}%
\pgfusepath{clip}%
\pgfsetbuttcap%
\pgfsetmiterjoin%
\definecolor{currentfill}{rgb}{0.447059,0.447059,0.447059}%
\pgfsetfillcolor{currentfill}%
\pgfsetlinewidth{1.003750pt}%
\definecolor{currentstroke}{rgb}{0.266667,0.266667,0.266667}%
\pgfsetstrokecolor{currentstroke}%
\pgfsetdash{}{0pt}%
\pgfpathmoveto{\pgfqpoint{4.102872in}{2.180610in}}%
\pgfpathlineto{\pgfqpoint{4.340352in}{2.180610in}}%
\pgfpathlineto{\pgfqpoint{4.340352in}{2.182659in}}%
\pgfpathlineto{\pgfqpoint{4.102872in}{2.182659in}}%
\pgfpathlineto{\pgfqpoint{4.102872in}{2.180610in}}%
\pgfpathclose%
\pgfusepath{stroke,fill}%
\end{pgfscope}%
\begin{pgfscope}%
\pgfpathrectangle{\pgfqpoint{0.868056in}{0.555556in}}{\pgfqpoint{3.993056in}{1.888889in}}%
\pgfusepath{clip}%
\pgfsetbuttcap%
\pgfsetmiterjoin%
\definecolor{currentfill}{rgb}{0.447059,0.447059,0.447059}%
\pgfsetfillcolor{currentfill}%
\pgfsetlinewidth{1.003750pt}%
\definecolor{currentstroke}{rgb}{0.266667,0.266667,0.266667}%
\pgfsetstrokecolor{currentstroke}%
\pgfsetdash{}{0pt}%
\pgfpathmoveto{\pgfqpoint{4.442129in}{2.345513in}}%
\pgfpathlineto{\pgfqpoint{4.679609in}{2.345513in}}%
\pgfpathlineto{\pgfqpoint{4.679609in}{2.346693in}}%
\pgfpathlineto{\pgfqpoint{4.442129in}{2.346693in}}%
\pgfpathlineto{\pgfqpoint{4.442129in}{2.345513in}}%
\pgfpathclose%
\pgfusepath{stroke,fill}%
\end{pgfscope}%
\begin{pgfscope}%
\pgfpathrectangle{\pgfqpoint{0.868056in}{0.555556in}}{\pgfqpoint{3.993056in}{1.888889in}}%
\pgfusepath{clip}%
\pgfsetbuttcap%
\pgfsetmiterjoin%
\definecolor{currentfill}{rgb}{0.447059,0.447059,0.447059}%
\pgfsetfillcolor{currentfill}%
\pgfsetlinewidth{1.003750pt}%
\definecolor{currentstroke}{rgb}{0.266667,0.266667,0.266667}%
\pgfsetstrokecolor{currentstroke}%
\pgfsetdash{}{0pt}%
\pgfpathmoveto{\pgfqpoint{1.049558in}{2.191613in}}%
\pgfpathlineto{\pgfqpoint{1.287038in}{2.191613in}}%
\pgfpathlineto{\pgfqpoint{1.287038in}{2.195308in}}%
\pgfpathlineto{\pgfqpoint{1.049558in}{2.195308in}}%
\pgfpathlineto{\pgfqpoint{1.049558in}{2.191613in}}%
\pgfpathclose%
\pgfusepath{stroke,fill}%
\end{pgfscope}%
\begin{pgfscope}%
\pgfpathrectangle{\pgfqpoint{0.868056in}{0.555556in}}{\pgfqpoint{3.993056in}{1.888889in}}%
\pgfusepath{clip}%
\pgfsetbuttcap%
\pgfsetmiterjoin%
\definecolor{currentfill}{rgb}{0.447059,0.447059,0.447059}%
\pgfsetfillcolor{currentfill}%
\pgfsetlinewidth{1.003750pt}%
\definecolor{currentstroke}{rgb}{0.266667,0.266667,0.266667}%
\pgfsetstrokecolor{currentstroke}%
\pgfsetdash{}{0pt}%
\pgfpathmoveto{\pgfqpoint{1.388815in}{1.783284in}}%
\pgfpathlineto{\pgfqpoint{1.626295in}{1.783284in}}%
\pgfpathlineto{\pgfqpoint{1.626295in}{1.786617in}}%
\pgfpathlineto{\pgfqpoint{1.388815in}{1.786617in}}%
\pgfpathlineto{\pgfqpoint{1.388815in}{1.783284in}}%
\pgfpathclose%
\pgfusepath{stroke,fill}%
\end{pgfscope}%
\begin{pgfscope}%
\pgfpathrectangle{\pgfqpoint{0.868056in}{0.555556in}}{\pgfqpoint{3.993056in}{1.888889in}}%
\pgfusepath{clip}%
\pgfsetbuttcap%
\pgfsetmiterjoin%
\definecolor{currentfill}{rgb}{0.447059,0.447059,0.447059}%
\pgfsetfillcolor{currentfill}%
\pgfsetlinewidth{1.003750pt}%
\definecolor{currentstroke}{rgb}{0.266667,0.266667,0.266667}%
\pgfsetstrokecolor{currentstroke}%
\pgfsetdash{}{0pt}%
\pgfpathmoveto{\pgfqpoint{1.728072in}{1.491664in}}%
\pgfpathlineto{\pgfqpoint{1.965552in}{1.491664in}}%
\pgfpathlineto{\pgfqpoint{1.965552in}{1.493507in}}%
\pgfpathlineto{\pgfqpoint{1.728072in}{1.493507in}}%
\pgfpathlineto{\pgfqpoint{1.728072in}{1.491664in}}%
\pgfpathclose%
\pgfusepath{stroke,fill}%
\end{pgfscope}%
\begin{pgfscope}%
\pgfpathrectangle{\pgfqpoint{0.868056in}{0.555556in}}{\pgfqpoint{3.993056in}{1.888889in}}%
\pgfusepath{clip}%
\pgfsetbuttcap%
\pgfsetmiterjoin%
\definecolor{currentfill}{rgb}{0.447059,0.447059,0.447059}%
\pgfsetfillcolor{currentfill}%
\pgfsetlinewidth{1.003750pt}%
\definecolor{currentstroke}{rgb}{0.266667,0.266667,0.266667}%
\pgfsetstrokecolor{currentstroke}%
\pgfsetdash{}{0pt}%
\pgfpathmoveto{\pgfqpoint{2.067329in}{1.598466in}}%
\pgfpathlineto{\pgfqpoint{2.304809in}{1.598466in}}%
\pgfpathlineto{\pgfqpoint{2.304809in}{1.601385in}}%
\pgfpathlineto{\pgfqpoint{2.067329in}{1.601385in}}%
\pgfpathlineto{\pgfqpoint{2.067329in}{1.598466in}}%
\pgfpathclose%
\pgfusepath{stroke,fill}%
\end{pgfscope}%
\begin{pgfscope}%
\pgfpathrectangle{\pgfqpoint{0.868056in}{0.555556in}}{\pgfqpoint{3.993056in}{1.888889in}}%
\pgfusepath{clip}%
\pgfsetbuttcap%
\pgfsetmiterjoin%
\definecolor{currentfill}{rgb}{0.447059,0.447059,0.447059}%
\pgfsetfillcolor{currentfill}%
\pgfsetlinewidth{1.003750pt}%
\definecolor{currentstroke}{rgb}{0.266667,0.266667,0.266667}%
\pgfsetstrokecolor{currentstroke}%
\pgfsetdash{}{0pt}%
\pgfpathmoveto{\pgfqpoint{2.406586in}{1.648659in}}%
\pgfpathlineto{\pgfqpoint{2.644066in}{1.648659in}}%
\pgfpathlineto{\pgfqpoint{2.644066in}{1.651019in}}%
\pgfpathlineto{\pgfqpoint{2.406586in}{1.651019in}}%
\pgfpathlineto{\pgfqpoint{2.406586in}{1.648659in}}%
\pgfpathclose%
\pgfusepath{stroke,fill}%
\end{pgfscope}%
\begin{pgfscope}%
\pgfpathrectangle{\pgfqpoint{0.868056in}{0.555556in}}{\pgfqpoint{3.993056in}{1.888889in}}%
\pgfusepath{clip}%
\pgfsetbuttcap%
\pgfsetmiterjoin%
\definecolor{currentfill}{rgb}{0.447059,0.447059,0.447059}%
\pgfsetfillcolor{currentfill}%
\pgfsetlinewidth{1.003750pt}%
\definecolor{currentstroke}{rgb}{0.266667,0.266667,0.266667}%
\pgfsetstrokecolor{currentstroke}%
\pgfsetdash{}{0pt}%
\pgfpathmoveto{\pgfqpoint{2.745843in}{1.847057in}}%
\pgfpathlineto{\pgfqpoint{2.983323in}{1.847057in}}%
\pgfpathlineto{\pgfqpoint{2.983323in}{1.849272in}}%
\pgfpathlineto{\pgfqpoint{2.745843in}{1.849272in}}%
\pgfpathlineto{\pgfqpoint{2.745843in}{1.847057in}}%
\pgfpathclose%
\pgfusepath{stroke,fill}%
\end{pgfscope}%
\begin{pgfscope}%
\pgfpathrectangle{\pgfqpoint{0.868056in}{0.555556in}}{\pgfqpoint{3.993056in}{1.888889in}}%
\pgfusepath{clip}%
\pgfsetbuttcap%
\pgfsetmiterjoin%
\definecolor{currentfill}{rgb}{0.447059,0.447059,0.447059}%
\pgfsetfillcolor{currentfill}%
\pgfsetlinewidth{1.003750pt}%
\definecolor{currentstroke}{rgb}{0.266667,0.266667,0.266667}%
\pgfsetstrokecolor{currentstroke}%
\pgfsetdash{}{0pt}%
\pgfpathmoveto{\pgfqpoint{3.085100in}{1.574193in}}%
\pgfpathlineto{\pgfqpoint{3.322580in}{1.574193in}}%
\pgfpathlineto{\pgfqpoint{3.322580in}{1.575611in}}%
\pgfpathlineto{\pgfqpoint{3.085100in}{1.575611in}}%
\pgfpathlineto{\pgfqpoint{3.085100in}{1.574193in}}%
\pgfpathclose%
\pgfusepath{stroke,fill}%
\end{pgfscope}%
\begin{pgfscope}%
\pgfpathrectangle{\pgfqpoint{0.868056in}{0.555556in}}{\pgfqpoint{3.993056in}{1.888889in}}%
\pgfusepath{clip}%
\pgfsetbuttcap%
\pgfsetmiterjoin%
\definecolor{currentfill}{rgb}{0.447059,0.447059,0.447059}%
\pgfsetfillcolor{currentfill}%
\pgfsetlinewidth{1.003750pt}%
\definecolor{currentstroke}{rgb}{0.266667,0.266667,0.266667}%
\pgfsetstrokecolor{currentstroke}%
\pgfsetdash{}{0pt}%
\pgfpathmoveto{\pgfqpoint{3.424357in}{1.878038in}}%
\pgfpathlineto{\pgfqpoint{3.661837in}{1.878038in}}%
\pgfpathlineto{\pgfqpoint{3.661837in}{1.879829in}}%
\pgfpathlineto{\pgfqpoint{3.424357in}{1.879829in}}%
\pgfpathlineto{\pgfqpoint{3.424357in}{1.878038in}}%
\pgfpathclose%
\pgfusepath{stroke,fill}%
\end{pgfscope}%
\begin{pgfscope}%
\pgfpathrectangle{\pgfqpoint{0.868056in}{0.555556in}}{\pgfqpoint{3.993056in}{1.888889in}}%
\pgfusepath{clip}%
\pgfsetbuttcap%
\pgfsetmiterjoin%
\definecolor{currentfill}{rgb}{0.447059,0.447059,0.447059}%
\pgfsetfillcolor{currentfill}%
\pgfsetlinewidth{1.003750pt}%
\definecolor{currentstroke}{rgb}{0.266667,0.266667,0.266667}%
\pgfsetstrokecolor{currentstroke}%
\pgfsetdash{}{0pt}%
\pgfpathmoveto{\pgfqpoint{3.763615in}{2.102076in}}%
\pgfpathlineto{\pgfqpoint{4.001094in}{2.102076in}}%
\pgfpathlineto{\pgfqpoint{4.001094in}{2.103505in}}%
\pgfpathlineto{\pgfqpoint{3.763615in}{2.103505in}}%
\pgfpathlineto{\pgfqpoint{3.763615in}{2.102076in}}%
\pgfpathclose%
\pgfusepath{stroke,fill}%
\end{pgfscope}%
\begin{pgfscope}%
\pgfpathrectangle{\pgfqpoint{0.868056in}{0.555556in}}{\pgfqpoint{3.993056in}{1.888889in}}%
\pgfusepath{clip}%
\pgfsetbuttcap%
\pgfsetmiterjoin%
\definecolor{currentfill}{rgb}{0.447059,0.447059,0.447059}%
\pgfsetfillcolor{currentfill}%
\pgfsetlinewidth{1.003750pt}%
\definecolor{currentstroke}{rgb}{0.266667,0.266667,0.266667}%
\pgfsetstrokecolor{currentstroke}%
\pgfsetdash{}{0pt}%
\pgfpathmoveto{\pgfqpoint{4.102872in}{2.182659in}}%
\pgfpathlineto{\pgfqpoint{4.340352in}{2.182659in}}%
\pgfpathlineto{\pgfqpoint{4.340352in}{2.184243in}}%
\pgfpathlineto{\pgfqpoint{4.102872in}{2.184243in}}%
\pgfpathlineto{\pgfqpoint{4.102872in}{2.182659in}}%
\pgfpathclose%
\pgfusepath{stroke,fill}%
\end{pgfscope}%
\begin{pgfscope}%
\pgfpathrectangle{\pgfqpoint{0.868056in}{0.555556in}}{\pgfqpoint{3.993056in}{1.888889in}}%
\pgfusepath{clip}%
\pgfsetbuttcap%
\pgfsetmiterjoin%
\definecolor{currentfill}{rgb}{0.447059,0.447059,0.447059}%
\pgfsetfillcolor{currentfill}%
\pgfsetlinewidth{1.003750pt}%
\definecolor{currentstroke}{rgb}{0.266667,0.266667,0.266667}%
\pgfsetstrokecolor{currentstroke}%
\pgfsetdash{}{0pt}%
\pgfpathmoveto{\pgfqpoint{4.442129in}{2.346693in}}%
\pgfpathlineto{\pgfqpoint{4.679609in}{2.346693in}}%
\pgfpathlineto{\pgfqpoint{4.679609in}{2.353028in}}%
\pgfpathlineto{\pgfqpoint{4.442129in}{2.353028in}}%
\pgfpathlineto{\pgfqpoint{4.442129in}{2.346693in}}%
\pgfpathclose%
\pgfusepath{stroke,fill}%
\end{pgfscope}%
\begin{pgfscope}%
\pgfpathrectangle{\pgfqpoint{0.868056in}{0.555556in}}{\pgfqpoint{3.993056in}{1.888889in}}%
\pgfusepath{clip}%
\pgfsetbuttcap%
\pgfsetmiterjoin%
\definecolor{currentfill}{rgb}{0.447059,0.447059,0.447059}%
\pgfsetfillcolor{currentfill}%
\pgfsetlinewidth{1.003750pt}%
\definecolor{currentstroke}{rgb}{0.266667,0.266667,0.266667}%
\pgfsetstrokecolor{currentstroke}%
\pgfsetdash{}{0pt}%
\pgfpathmoveto{\pgfqpoint{1.049558in}{2.195308in}}%
\pgfpathlineto{\pgfqpoint{1.287038in}{2.195308in}}%
\pgfpathlineto{\pgfqpoint{1.287038in}{2.196592in}}%
\pgfpathlineto{\pgfqpoint{1.049558in}{2.196592in}}%
\pgfpathlineto{\pgfqpoint{1.049558in}{2.195308in}}%
\pgfpathclose%
\pgfusepath{stroke,fill}%
\end{pgfscope}%
\begin{pgfscope}%
\pgfpathrectangle{\pgfqpoint{0.868056in}{0.555556in}}{\pgfqpoint{3.993056in}{1.888889in}}%
\pgfusepath{clip}%
\pgfsetbuttcap%
\pgfsetmiterjoin%
\definecolor{currentfill}{rgb}{0.447059,0.447059,0.447059}%
\pgfsetfillcolor{currentfill}%
\pgfsetlinewidth{1.003750pt}%
\definecolor{currentstroke}{rgb}{0.266667,0.266667,0.266667}%
\pgfsetstrokecolor{currentstroke}%
\pgfsetdash{}{0pt}%
\pgfpathmoveto{\pgfqpoint{1.388815in}{1.786617in}}%
\pgfpathlineto{\pgfqpoint{1.626295in}{1.786617in}}%
\pgfpathlineto{\pgfqpoint{1.626295in}{1.787984in}}%
\pgfpathlineto{\pgfqpoint{1.388815in}{1.787984in}}%
\pgfpathlineto{\pgfqpoint{1.388815in}{1.786617in}}%
\pgfpathclose%
\pgfusepath{stroke,fill}%
\end{pgfscope}%
\begin{pgfscope}%
\pgfpathrectangle{\pgfqpoint{0.868056in}{0.555556in}}{\pgfqpoint{3.993056in}{1.888889in}}%
\pgfusepath{clip}%
\pgfsetbuttcap%
\pgfsetmiterjoin%
\definecolor{currentfill}{rgb}{0.447059,0.447059,0.447059}%
\pgfsetfillcolor{currentfill}%
\pgfsetlinewidth{1.003750pt}%
\definecolor{currentstroke}{rgb}{0.266667,0.266667,0.266667}%
\pgfsetstrokecolor{currentstroke}%
\pgfsetdash{}{0pt}%
\pgfpathmoveto{\pgfqpoint{1.728072in}{1.493507in}}%
\pgfpathlineto{\pgfqpoint{1.965552in}{1.493507in}}%
\pgfpathlineto{\pgfqpoint{1.965552in}{1.494562in}}%
\pgfpathlineto{\pgfqpoint{1.728072in}{1.494562in}}%
\pgfpathlineto{\pgfqpoint{1.728072in}{1.493507in}}%
\pgfpathclose%
\pgfusepath{stroke,fill}%
\end{pgfscope}%
\begin{pgfscope}%
\pgfpathrectangle{\pgfqpoint{0.868056in}{0.555556in}}{\pgfqpoint{3.993056in}{1.888889in}}%
\pgfusepath{clip}%
\pgfsetbuttcap%
\pgfsetmiterjoin%
\definecolor{currentfill}{rgb}{0.447059,0.447059,0.447059}%
\pgfsetfillcolor{currentfill}%
\pgfsetlinewidth{1.003750pt}%
\definecolor{currentstroke}{rgb}{0.266667,0.266667,0.266667}%
\pgfsetstrokecolor{currentstroke}%
\pgfsetdash{}{0pt}%
\pgfpathmoveto{\pgfqpoint{2.067329in}{1.601385in}}%
\pgfpathlineto{\pgfqpoint{2.304809in}{1.601385in}}%
\pgfpathlineto{\pgfqpoint{2.304809in}{1.602638in}}%
\pgfpathlineto{\pgfqpoint{2.067329in}{1.602638in}}%
\pgfpathlineto{\pgfqpoint{2.067329in}{1.601385in}}%
\pgfpathclose%
\pgfusepath{stroke,fill}%
\end{pgfscope}%
\begin{pgfscope}%
\pgfpathrectangle{\pgfqpoint{0.868056in}{0.555556in}}{\pgfqpoint{3.993056in}{1.888889in}}%
\pgfusepath{clip}%
\pgfsetbuttcap%
\pgfsetmiterjoin%
\definecolor{currentfill}{rgb}{0.447059,0.447059,0.447059}%
\pgfsetfillcolor{currentfill}%
\pgfsetlinewidth{1.003750pt}%
\definecolor{currentstroke}{rgb}{0.266667,0.266667,0.266667}%
\pgfsetstrokecolor{currentstroke}%
\pgfsetdash{}{0pt}%
\pgfpathmoveto{\pgfqpoint{2.406586in}{1.651019in}}%
\pgfpathlineto{\pgfqpoint{2.644066in}{1.651019in}}%
\pgfpathlineto{\pgfqpoint{2.644066in}{1.653472in}}%
\pgfpathlineto{\pgfqpoint{2.406586in}{1.653472in}}%
\pgfpathlineto{\pgfqpoint{2.406586in}{1.651019in}}%
\pgfpathclose%
\pgfusepath{stroke,fill}%
\end{pgfscope}%
\begin{pgfscope}%
\pgfpathrectangle{\pgfqpoint{0.868056in}{0.555556in}}{\pgfqpoint{3.993056in}{1.888889in}}%
\pgfusepath{clip}%
\pgfsetbuttcap%
\pgfsetmiterjoin%
\definecolor{currentfill}{rgb}{0.447059,0.447059,0.447059}%
\pgfsetfillcolor{currentfill}%
\pgfsetlinewidth{1.003750pt}%
\definecolor{currentstroke}{rgb}{0.266667,0.266667,0.266667}%
\pgfsetstrokecolor{currentstroke}%
\pgfsetdash{}{0pt}%
\pgfpathmoveto{\pgfqpoint{2.745843in}{1.849272in}}%
\pgfpathlineto{\pgfqpoint{2.983323in}{1.849272in}}%
\pgfpathlineto{\pgfqpoint{2.983323in}{1.850887in}}%
\pgfpathlineto{\pgfqpoint{2.745843in}{1.850887in}}%
\pgfpathlineto{\pgfqpoint{2.745843in}{1.849272in}}%
\pgfpathclose%
\pgfusepath{stroke,fill}%
\end{pgfscope}%
\begin{pgfscope}%
\pgfpathrectangle{\pgfqpoint{0.868056in}{0.555556in}}{\pgfqpoint{3.993056in}{1.888889in}}%
\pgfusepath{clip}%
\pgfsetbuttcap%
\pgfsetmiterjoin%
\definecolor{currentfill}{rgb}{0.447059,0.447059,0.447059}%
\pgfsetfillcolor{currentfill}%
\pgfsetlinewidth{1.003750pt}%
\definecolor{currentstroke}{rgb}{0.266667,0.266667,0.266667}%
\pgfsetstrokecolor{currentstroke}%
\pgfsetdash{}{0pt}%
\pgfpathmoveto{\pgfqpoint{3.085100in}{1.575611in}}%
\pgfpathlineto{\pgfqpoint{3.322580in}{1.575611in}}%
\pgfpathlineto{\pgfqpoint{3.322580in}{1.576884in}}%
\pgfpathlineto{\pgfqpoint{3.085100in}{1.576884in}}%
\pgfpathlineto{\pgfqpoint{3.085100in}{1.575611in}}%
\pgfpathclose%
\pgfusepath{stroke,fill}%
\end{pgfscope}%
\begin{pgfscope}%
\pgfpathrectangle{\pgfqpoint{0.868056in}{0.555556in}}{\pgfqpoint{3.993056in}{1.888889in}}%
\pgfusepath{clip}%
\pgfsetbuttcap%
\pgfsetmiterjoin%
\definecolor{currentfill}{rgb}{0.447059,0.447059,0.447059}%
\pgfsetfillcolor{currentfill}%
\pgfsetlinewidth{1.003750pt}%
\definecolor{currentstroke}{rgb}{0.266667,0.266667,0.266667}%
\pgfsetstrokecolor{currentstroke}%
\pgfsetdash{}{0pt}%
\pgfpathmoveto{\pgfqpoint{3.424357in}{1.879829in}}%
\pgfpathlineto{\pgfqpoint{3.661837in}{1.879829in}}%
\pgfpathlineto{\pgfqpoint{3.661837in}{1.880967in}}%
\pgfpathlineto{\pgfqpoint{3.424357in}{1.880967in}}%
\pgfpathlineto{\pgfqpoint{3.424357in}{1.879829in}}%
\pgfpathclose%
\pgfusepath{stroke,fill}%
\end{pgfscope}%
\begin{pgfscope}%
\pgfpathrectangle{\pgfqpoint{0.868056in}{0.555556in}}{\pgfqpoint{3.993056in}{1.888889in}}%
\pgfusepath{clip}%
\pgfsetbuttcap%
\pgfsetmiterjoin%
\definecolor{currentfill}{rgb}{0.447059,0.447059,0.447059}%
\pgfsetfillcolor{currentfill}%
\pgfsetlinewidth{1.003750pt}%
\definecolor{currentstroke}{rgb}{0.266667,0.266667,0.266667}%
\pgfsetstrokecolor{currentstroke}%
\pgfsetdash{}{0pt}%
\pgfpathmoveto{\pgfqpoint{3.763615in}{2.103505in}}%
\pgfpathlineto{\pgfqpoint{4.001094in}{2.103505in}}%
\pgfpathlineto{\pgfqpoint{4.001094in}{2.104488in}}%
\pgfpathlineto{\pgfqpoint{3.763615in}{2.104488in}}%
\pgfpathlineto{\pgfqpoint{3.763615in}{2.103505in}}%
\pgfpathclose%
\pgfusepath{stroke,fill}%
\end{pgfscope}%
\begin{pgfscope}%
\pgfpathrectangle{\pgfqpoint{0.868056in}{0.555556in}}{\pgfqpoint{3.993056in}{1.888889in}}%
\pgfusepath{clip}%
\pgfsetbuttcap%
\pgfsetmiterjoin%
\definecolor{currentfill}{rgb}{0.447059,0.447059,0.447059}%
\pgfsetfillcolor{currentfill}%
\pgfsetlinewidth{1.003750pt}%
\definecolor{currentstroke}{rgb}{0.266667,0.266667,0.266667}%
\pgfsetstrokecolor{currentstroke}%
\pgfsetdash{}{0pt}%
\pgfpathmoveto{\pgfqpoint{4.102872in}{2.184243in}}%
\pgfpathlineto{\pgfqpoint{4.340352in}{2.184243in}}%
\pgfpathlineto{\pgfqpoint{4.340352in}{2.185113in}}%
\pgfpathlineto{\pgfqpoint{4.102872in}{2.185113in}}%
\pgfpathlineto{\pgfqpoint{4.102872in}{2.184243in}}%
\pgfpathclose%
\pgfusepath{stroke,fill}%
\end{pgfscope}%
\begin{pgfscope}%
\pgfpathrectangle{\pgfqpoint{0.868056in}{0.555556in}}{\pgfqpoint{3.993056in}{1.888889in}}%
\pgfusepath{clip}%
\pgfsetbuttcap%
\pgfsetmiterjoin%
\definecolor{currentfill}{rgb}{0.447059,0.447059,0.447059}%
\pgfsetfillcolor{currentfill}%
\pgfsetlinewidth{1.003750pt}%
\definecolor{currentstroke}{rgb}{0.266667,0.266667,0.266667}%
\pgfsetstrokecolor{currentstroke}%
\pgfsetdash{}{0pt}%
\pgfpathmoveto{\pgfqpoint{4.442129in}{2.353028in}}%
\pgfpathlineto{\pgfqpoint{4.679609in}{2.353028in}}%
\pgfpathlineto{\pgfqpoint{4.679609in}{2.354063in}}%
\pgfpathlineto{\pgfqpoint{4.442129in}{2.354063in}}%
\pgfpathlineto{\pgfqpoint{4.442129in}{2.353028in}}%
\pgfpathclose%
\pgfusepath{stroke,fill}%
\end{pgfscope}%
\begin{pgfscope}%
\pgfpathrectangle{\pgfqpoint{0.868056in}{0.555556in}}{\pgfqpoint{3.993056in}{1.888889in}}%
\pgfusepath{clip}%
\pgfsetbuttcap%
\pgfsetmiterjoin%
\definecolor{currentfill}{rgb}{0.447059,0.447059,0.447059}%
\pgfsetfillcolor{currentfill}%
\pgfsetlinewidth{1.003750pt}%
\definecolor{currentstroke}{rgb}{0.266667,0.266667,0.266667}%
\pgfsetstrokecolor{currentstroke}%
\pgfsetdash{}{0pt}%
\pgfpathmoveto{\pgfqpoint{1.049558in}{2.196592in}}%
\pgfpathlineto{\pgfqpoint{1.287038in}{2.196592in}}%
\pgfpathlineto{\pgfqpoint{1.287038in}{2.197037in}}%
\pgfpathlineto{\pgfqpoint{1.049558in}{2.197037in}}%
\pgfpathlineto{\pgfqpoint{1.049558in}{2.196592in}}%
\pgfpathclose%
\pgfusepath{stroke,fill}%
\end{pgfscope}%
\begin{pgfscope}%
\pgfpathrectangle{\pgfqpoint{0.868056in}{0.555556in}}{\pgfqpoint{3.993056in}{1.888889in}}%
\pgfusepath{clip}%
\pgfsetbuttcap%
\pgfsetmiterjoin%
\definecolor{currentfill}{rgb}{0.447059,0.447059,0.447059}%
\pgfsetfillcolor{currentfill}%
\pgfsetlinewidth{1.003750pt}%
\definecolor{currentstroke}{rgb}{0.266667,0.266667,0.266667}%
\pgfsetstrokecolor{currentstroke}%
\pgfsetdash{}{0pt}%
\pgfpathmoveto{\pgfqpoint{1.388815in}{1.787984in}}%
\pgfpathlineto{\pgfqpoint{1.626295in}{1.787984in}}%
\pgfpathlineto{\pgfqpoint{1.626295in}{1.788149in}}%
\pgfpathlineto{\pgfqpoint{1.388815in}{1.788149in}}%
\pgfpathlineto{\pgfqpoint{1.388815in}{1.787984in}}%
\pgfpathclose%
\pgfusepath{stroke,fill}%
\end{pgfscope}%
\begin{pgfscope}%
\pgfpathrectangle{\pgfqpoint{0.868056in}{0.555556in}}{\pgfqpoint{3.993056in}{1.888889in}}%
\pgfusepath{clip}%
\pgfsetbuttcap%
\pgfsetmiterjoin%
\definecolor{currentfill}{rgb}{0.447059,0.447059,0.447059}%
\pgfsetfillcolor{currentfill}%
\pgfsetlinewidth{1.003750pt}%
\definecolor{currentstroke}{rgb}{0.266667,0.266667,0.266667}%
\pgfsetstrokecolor{currentstroke}%
\pgfsetdash{}{0pt}%
\pgfpathmoveto{\pgfqpoint{1.728072in}{1.494562in}}%
\pgfpathlineto{\pgfqpoint{1.965552in}{1.494562in}}%
\pgfpathlineto{\pgfqpoint{1.965552in}{1.494593in}}%
\pgfpathlineto{\pgfqpoint{1.728072in}{1.494593in}}%
\pgfpathlineto{\pgfqpoint{1.728072in}{1.494562in}}%
\pgfpathclose%
\pgfusepath{stroke,fill}%
\end{pgfscope}%
\begin{pgfscope}%
\pgfpathrectangle{\pgfqpoint{0.868056in}{0.555556in}}{\pgfqpoint{3.993056in}{1.888889in}}%
\pgfusepath{clip}%
\pgfsetbuttcap%
\pgfsetmiterjoin%
\definecolor{currentfill}{rgb}{0.447059,0.447059,0.447059}%
\pgfsetfillcolor{currentfill}%
\pgfsetlinewidth{1.003750pt}%
\definecolor{currentstroke}{rgb}{0.266667,0.266667,0.266667}%
\pgfsetstrokecolor{currentstroke}%
\pgfsetdash{}{0pt}%
\pgfpathmoveto{\pgfqpoint{2.067329in}{1.602638in}}%
\pgfpathlineto{\pgfqpoint{2.304809in}{1.602638in}}%
\pgfpathlineto{\pgfqpoint{2.304809in}{1.602731in}}%
\pgfpathlineto{\pgfqpoint{2.067329in}{1.602731in}}%
\pgfpathlineto{\pgfqpoint{2.067329in}{1.602638in}}%
\pgfpathclose%
\pgfusepath{stroke,fill}%
\end{pgfscope}%
\begin{pgfscope}%
\pgfpathrectangle{\pgfqpoint{0.868056in}{0.555556in}}{\pgfqpoint{3.993056in}{1.888889in}}%
\pgfusepath{clip}%
\pgfsetbuttcap%
\pgfsetmiterjoin%
\definecolor{currentfill}{rgb}{0.447059,0.447059,0.447059}%
\pgfsetfillcolor{currentfill}%
\pgfsetlinewidth{1.003750pt}%
\definecolor{currentstroke}{rgb}{0.266667,0.266667,0.266667}%
\pgfsetstrokecolor{currentstroke}%
\pgfsetdash{}{0pt}%
\pgfpathmoveto{\pgfqpoint{2.406586in}{1.653472in}}%
\pgfpathlineto{\pgfqpoint{2.644066in}{1.653472in}}%
\pgfpathlineto{\pgfqpoint{2.644066in}{1.653575in}}%
\pgfpathlineto{\pgfqpoint{2.406586in}{1.653575in}}%
\pgfpathlineto{\pgfqpoint{2.406586in}{1.653472in}}%
\pgfpathclose%
\pgfusepath{stroke,fill}%
\end{pgfscope}%
\begin{pgfscope}%
\pgfpathrectangle{\pgfqpoint{0.868056in}{0.555556in}}{\pgfqpoint{3.993056in}{1.888889in}}%
\pgfusepath{clip}%
\pgfsetbuttcap%
\pgfsetmiterjoin%
\definecolor{currentfill}{rgb}{0.447059,0.447059,0.447059}%
\pgfsetfillcolor{currentfill}%
\pgfsetlinewidth{1.003750pt}%
\definecolor{currentstroke}{rgb}{0.266667,0.266667,0.266667}%
\pgfsetstrokecolor{currentstroke}%
\pgfsetdash{}{0pt}%
\pgfpathmoveto{\pgfqpoint{2.745843in}{1.850887in}}%
\pgfpathlineto{\pgfqpoint{2.983323in}{1.850887in}}%
\pgfpathlineto{\pgfqpoint{2.983323in}{1.850991in}}%
\pgfpathlineto{\pgfqpoint{2.745843in}{1.850991in}}%
\pgfpathlineto{\pgfqpoint{2.745843in}{1.850887in}}%
\pgfpathclose%
\pgfusepath{stroke,fill}%
\end{pgfscope}%
\begin{pgfscope}%
\pgfpathrectangle{\pgfqpoint{0.868056in}{0.555556in}}{\pgfqpoint{3.993056in}{1.888889in}}%
\pgfusepath{clip}%
\pgfsetbuttcap%
\pgfsetmiterjoin%
\definecolor{currentfill}{rgb}{0.447059,0.447059,0.447059}%
\pgfsetfillcolor{currentfill}%
\pgfsetlinewidth{1.003750pt}%
\definecolor{currentstroke}{rgb}{0.266667,0.266667,0.266667}%
\pgfsetstrokecolor{currentstroke}%
\pgfsetdash{}{0pt}%
\pgfpathmoveto{\pgfqpoint{3.085100in}{1.576884in}}%
\pgfpathlineto{\pgfqpoint{3.322580in}{1.576884in}}%
\pgfpathlineto{\pgfqpoint{3.322580in}{1.576977in}}%
\pgfpathlineto{\pgfqpoint{3.085100in}{1.576977in}}%
\pgfpathlineto{\pgfqpoint{3.085100in}{1.576884in}}%
\pgfpathclose%
\pgfusepath{stroke,fill}%
\end{pgfscope}%
\begin{pgfscope}%
\pgfpathrectangle{\pgfqpoint{0.868056in}{0.555556in}}{\pgfqpoint{3.993056in}{1.888889in}}%
\pgfusepath{clip}%
\pgfsetbuttcap%
\pgfsetmiterjoin%
\definecolor{currentfill}{rgb}{0.447059,0.447059,0.447059}%
\pgfsetfillcolor{currentfill}%
\pgfsetlinewidth{1.003750pt}%
\definecolor{currentstroke}{rgb}{0.266667,0.266667,0.266667}%
\pgfsetstrokecolor{currentstroke}%
\pgfsetdash{}{0pt}%
\pgfpathmoveto{\pgfqpoint{3.424357in}{1.880967in}}%
\pgfpathlineto{\pgfqpoint{3.661837in}{1.880967in}}%
\pgfpathlineto{\pgfqpoint{3.661837in}{1.881174in}}%
\pgfpathlineto{\pgfqpoint{3.424357in}{1.881174in}}%
\pgfpathlineto{\pgfqpoint{3.424357in}{1.880967in}}%
\pgfpathclose%
\pgfusepath{stroke,fill}%
\end{pgfscope}%
\begin{pgfscope}%
\pgfpathrectangle{\pgfqpoint{0.868056in}{0.555556in}}{\pgfqpoint{3.993056in}{1.888889in}}%
\pgfusepath{clip}%
\pgfsetbuttcap%
\pgfsetmiterjoin%
\definecolor{currentfill}{rgb}{0.447059,0.447059,0.447059}%
\pgfsetfillcolor{currentfill}%
\pgfsetlinewidth{1.003750pt}%
\definecolor{currentstroke}{rgb}{0.266667,0.266667,0.266667}%
\pgfsetstrokecolor{currentstroke}%
\pgfsetdash{}{0pt}%
\pgfpathmoveto{\pgfqpoint{3.763615in}{2.104488in}}%
\pgfpathlineto{\pgfqpoint{4.001094in}{2.104488in}}%
\pgfpathlineto{\pgfqpoint{4.001094in}{2.104612in}}%
\pgfpathlineto{\pgfqpoint{3.763615in}{2.104612in}}%
\pgfpathlineto{\pgfqpoint{3.763615in}{2.104488in}}%
\pgfpathclose%
\pgfusepath{stroke,fill}%
\end{pgfscope}%
\begin{pgfscope}%
\pgfpathrectangle{\pgfqpoint{0.868056in}{0.555556in}}{\pgfqpoint{3.993056in}{1.888889in}}%
\pgfusepath{clip}%
\pgfsetbuttcap%
\pgfsetmiterjoin%
\definecolor{currentfill}{rgb}{0.447059,0.447059,0.447059}%
\pgfsetfillcolor{currentfill}%
\pgfsetlinewidth{1.003750pt}%
\definecolor{currentstroke}{rgb}{0.266667,0.266667,0.266667}%
\pgfsetstrokecolor{currentstroke}%
\pgfsetdash{}{0pt}%
\pgfpathmoveto{\pgfqpoint{4.102872in}{2.185113in}}%
\pgfpathlineto{\pgfqpoint{4.340352in}{2.185113in}}%
\pgfpathlineto{\pgfqpoint{4.340352in}{2.185195in}}%
\pgfpathlineto{\pgfqpoint{4.102872in}{2.185195in}}%
\pgfpathlineto{\pgfqpoint{4.102872in}{2.185113in}}%
\pgfpathclose%
\pgfusepath{stroke,fill}%
\end{pgfscope}%
\begin{pgfscope}%
\pgfpathrectangle{\pgfqpoint{0.868056in}{0.555556in}}{\pgfqpoint{3.993056in}{1.888889in}}%
\pgfusepath{clip}%
\pgfsetbuttcap%
\pgfsetmiterjoin%
\definecolor{currentfill}{rgb}{0.447059,0.447059,0.447059}%
\pgfsetfillcolor{currentfill}%
\pgfsetlinewidth{1.003750pt}%
\definecolor{currentstroke}{rgb}{0.266667,0.266667,0.266667}%
\pgfsetstrokecolor{currentstroke}%
\pgfsetdash{}{0pt}%
\pgfpathmoveto{\pgfqpoint{4.442129in}{2.354063in}}%
\pgfpathlineto{\pgfqpoint{4.679609in}{2.354063in}}%
\pgfpathlineto{\pgfqpoint{4.679609in}{2.354497in}}%
\pgfpathlineto{\pgfqpoint{4.442129in}{2.354497in}}%
\pgfpathlineto{\pgfqpoint{4.442129in}{2.354063in}}%
\pgfpathclose%
\pgfusepath{stroke,fill}%
\end{pgfscope}%
\begin{pgfscope}%
\definecolor{textcolor}{rgb}{0.000000,0.000000,0.000000}%
\pgfsetstrokecolor{textcolor}%
\pgfsetfillcolor{textcolor}%
\pgftext[x=1.168298in,y=2.224815in,,bottom]{\color{textcolor}{\ifdefined\pdftexversion\else\setmainfont{NanumMyeongjo}\rmfamily\fi\fontsize{5.000000}{6.000000}\selectfont\catcode`\^=\active\def^{\ifmmode\sp\else\^{}\fi}\catcode`\%=\active\def%{\%}158,581}}%
\end{pgfscope}%
\begin{pgfscope}%
\definecolor{textcolor}{rgb}{0.000000,0.000000,0.000000}%
\pgfsetstrokecolor{textcolor}%
\pgfsetfillcolor{textcolor}%
\pgftext[x=1.507555in,y=1.815927in,,bottom]{\color{textcolor}{\ifdefined\pdftexversion\else\setmainfont{NanumMyeongjo}\rmfamily\fi\fontsize{5.000000}{6.000000}\selectfont\catcode`\^=\active\def^{\ifmmode\sp\else\^{}\fi}\catcode`\%=\active\def%{\%}119,079}}%
\end{pgfscope}%
\begin{pgfscope}%
\definecolor{textcolor}{rgb}{0.000000,0.000000,0.000000}%
\pgfsetstrokecolor{textcolor}%
\pgfsetfillcolor{textcolor}%
\pgftext[x=1.846812in,y=1.522371in,,bottom]{\color{textcolor}{\ifdefined\pdftexversion\else\setmainfont{NanumMyeongjo}\rmfamily\fi\fontsize{5.000000}{6.000000}\selectfont\catcode`\^=\active\def^{\ifmmode\sp\else\^{}\fi}\catcode`\%=\active\def%{\%}90,719}}%
\end{pgfscope}%
\begin{pgfscope}%
\definecolor{textcolor}{rgb}{0.000000,0.000000,0.000000}%
\pgfsetstrokecolor{textcolor}%
\pgfsetfillcolor{textcolor}%
\pgftext[x=2.186069in,y=1.630509in,,bottom]{\color{textcolor}{\ifdefined\pdftexversion\else\setmainfont{NanumMyeongjo}\rmfamily\fi\fontsize{5.000000}{6.000000}\selectfont\catcode`\^=\active\def^{\ifmmode\sp\else\^{}\fi}\catcode`\%=\active\def%{\%}101,166}}%
\end{pgfscope}%
\begin{pgfscope}%
\definecolor{textcolor}{rgb}{0.000000,0.000000,0.000000}%
\pgfsetstrokecolor{textcolor}%
\pgfsetfillcolor{textcolor}%
\pgftext[x=2.525326in,y=1.681353in,,bottom]{\color{textcolor}{\ifdefined\pdftexversion\else\setmainfont{NanumMyeongjo}\rmfamily\fi\fontsize{5.000000}{6.000000}\selectfont\catcode`\^=\active\def^{\ifmmode\sp\else\^{}\fi}\catcode`\%=\active\def%{\%}106,078}}%
\end{pgfscope}%
\begin{pgfscope}%
\definecolor{textcolor}{rgb}{0.000000,0.000000,0.000000}%
\pgfsetstrokecolor{textcolor}%
\pgfsetfillcolor{textcolor}%
\pgftext[x=2.864583in,y=1.878769in,,bottom]{\color{textcolor}{\ifdefined\pdftexversion\else\setmainfont{NanumMyeongjo}\rmfamily\fi\fontsize{5.000000}{6.000000}\selectfont\catcode`\^=\active\def^{\ifmmode\sp\else\^{}\fi}\catcode`\%=\active\def%{\%}125,150}}%
\end{pgfscope}%
\begin{pgfscope}%
\definecolor{textcolor}{rgb}{0.000000,0.000000,0.000000}%
\pgfsetstrokecolor{textcolor}%
\pgfsetfillcolor{textcolor}%
\pgftext[x=3.203840in,y=1.604755in,,bottom]{\color{textcolor}{\ifdefined\pdftexversion\else\setmainfont{NanumMyeongjo}\rmfamily\fi\fontsize{5.000000}{6.000000}\selectfont\catcode`\^=\active\def^{\ifmmode\sp\else\^{}\fi}\catcode`\%=\active\def%{\%}98,678}}%
\end{pgfscope}%
\begin{pgfscope}%
\definecolor{textcolor}{rgb}{0.000000,0.000000,0.000000}%
\pgfsetstrokecolor{textcolor}%
\pgfsetfillcolor{textcolor}%
\pgftext[x=3.543097in,y=1.908952in,,bottom]{\color{textcolor}{\ifdefined\pdftexversion\else\setmainfont{NanumMyeongjo}\rmfamily\fi\fontsize{5.000000}{6.000000}\selectfont\catcode`\^=\active\def^{\ifmmode\sp\else\^{}\fi}\catcode`\%=\active\def%{\%}128,066}}%
\end{pgfscope}%
\begin{pgfscope}%
\definecolor{textcolor}{rgb}{0.000000,0.000000,0.000000}%
\pgfsetstrokecolor{textcolor}%
\pgfsetfillcolor{textcolor}%
\pgftext[x=3.882355in,y=2.132390in,,bottom]{\color{textcolor}{\ifdefined\pdftexversion\else\setmainfont{NanumMyeongjo}\rmfamily\fi\fontsize{5.000000}{6.000000}\selectfont\catcode`\^=\active\def^{\ifmmode\sp\else\^{}\fi}\catcode`\%=\active\def%{\%}149,652}}%
\end{pgfscope}%
\begin{pgfscope}%
\definecolor{textcolor}{rgb}{0.000000,0.000000,0.000000}%
\pgfsetstrokecolor{textcolor}%
\pgfsetfillcolor{textcolor}%
\pgftext[x=4.221612in,y=2.212973in,,bottom]{\color{textcolor}{\ifdefined\pdftexversion\else\setmainfont{NanumMyeongjo}\rmfamily\fi\fontsize{5.000000}{6.000000}\selectfont\catcode`\^=\active\def^{\ifmmode\sp\else\^{}\fi}\catcode`\%=\active\def%{\%}157,437}}%
\end{pgfscope}%
\begin{pgfscope}%
\definecolor{textcolor}{rgb}{0.000000,0.000000,0.000000}%
\pgfsetstrokecolor{textcolor}%
\pgfsetfillcolor{textcolor}%
\pgftext[x=4.560869in,y=2.382275in,,bottom]{\color{textcolor}{\ifdefined\pdftexversion\else\setmainfont{NanumMyeongjo}\rmfamily\fi\fontsize{5.000000}{6.000000}\selectfont\catcode`\^=\active\def^{\ifmmode\sp\else\^{}\fi}\catcode`\%=\active\def%{\%}173,793}}%
\end{pgfscope}%
\begin{pgfscope}%
\definecolor{textcolor}{rgb}{1.000000,1.000000,1.000000}%
\pgfsetstrokecolor{textcolor}%
\pgfsetfillcolor{textcolor}%
\pgftext[x=1.168298in,y=0.627413in,,]{\color{textcolor}{\ifdefined\pdftexversion\else\setmainfont{NanumMyeongjo}\rmfamily\fi\fontsize{5.000000}{6.000000}\selectfont\catcode`\^=\active\def^{\ifmmode\sp\else\^{}\fi}\catcode`\%=\active\def%{\%}14,942}}%
\end{pgfscope}%
\begin{pgfscope}%
\definecolor{textcolor}{rgb}{1.000000,1.000000,1.000000}%
\pgfsetstrokecolor{textcolor}%
\pgfsetfillcolor{textcolor}%
\pgftext[x=1.507555in,y=0.587727in,,]{\color{textcolor}{\ifdefined\pdftexversion\else\setmainfont{NanumMyeongjo}\rmfamily\fi\fontsize{5.000000}{6.000000}\selectfont\catcode`\^=\active\def^{\ifmmode\sp\else\^{}\fi}\catcode`\%=\active\def%{\%}11,108}}%
\end{pgfscope}%
\begin{pgfscope}%
\definecolor{textcolor}{rgb}{1.000000,1.000000,1.000000}%
\pgfsetstrokecolor{textcolor}%
\pgfsetfillcolor{textcolor}%
\pgftext[x=2.186069in,y=0.604143in,,]{\color{textcolor}{\ifdefined\pdftexversion\else\setmainfont{NanumMyeongjo}\rmfamily\fi\fontsize{5.000000}{6.000000}\selectfont\catcode`\^=\active\def^{\ifmmode\sp\else\^{}\fi}\catcode`\%=\active\def%{\%}12,694}}%
\end{pgfscope}%
\begin{pgfscope}%
\definecolor{textcolor}{rgb}{1.000000,1.000000,1.000000}%
\pgfsetstrokecolor{textcolor}%
\pgfsetfillcolor{textcolor}%
\pgftext[x=2.525326in,y=0.663766in,,]{\color{textcolor}{\ifdefined\pdftexversion\else\setmainfont{NanumMyeongjo}\rmfamily\fi\fontsize{5.000000}{6.000000}\selectfont\catcode`\^=\active\def^{\ifmmode\sp\else\^{}\fi}\catcode`\%=\active\def%{\%}18,454}}%
\end{pgfscope}%
\begin{pgfscope}%
\definecolor{textcolor}{rgb}{1.000000,1.000000,1.000000}%
\pgfsetstrokecolor{textcolor}%
\pgfsetfillcolor{textcolor}%
\pgftext[x=2.864583in,y=0.728905in,,]{\color{textcolor}{\ifdefined\pdftexversion\else\setmainfont{NanumMyeongjo}\rmfamily\fi\fontsize{5.000000}{6.000000}\selectfont\catcode`\^=\active\def^{\ifmmode\sp\else\^{}\fi}\catcode`\%=\active\def%{\%}24,747}}%
\end{pgfscope}%
\begin{pgfscope}%
\definecolor{textcolor}{rgb}{1.000000,1.000000,1.000000}%
\pgfsetstrokecolor{textcolor}%
\pgfsetfillcolor{textcolor}%
\pgftext[x=3.203840in,y=0.637805in,,]{\color{textcolor}{\ifdefined\pdftexversion\else\setmainfont{NanumMyeongjo}\rmfamily\fi\fontsize{5.000000}{6.000000}\selectfont\catcode`\^=\active\def^{\ifmmode\sp\else\^{}\fi}\catcode`\%=\active\def%{\%}15,946}}%
\end{pgfscope}%
\begin{pgfscope}%
\definecolor{textcolor}{rgb}{1.000000,1.000000,1.000000}%
\pgfsetstrokecolor{textcolor}%
\pgfsetfillcolor{textcolor}%
\pgftext[x=3.543097in,y=0.725810in,,]{\color{textcolor}{\ifdefined\pdftexversion\else\setmainfont{NanumMyeongjo}\rmfamily\fi\fontsize{5.000000}{6.000000}\selectfont\catcode`\^=\active\def^{\ifmmode\sp\else\^{}\fi}\catcode`\%=\active\def%{\%}24,448}}%
\end{pgfscope}%
\begin{pgfscope}%
\definecolor{textcolor}{rgb}{1.000000,1.000000,1.000000}%
\pgfsetstrokecolor{textcolor}%
\pgfsetfillcolor{textcolor}%
\pgftext[x=3.882355in,y=0.833585in,,]{\color{textcolor}{\ifdefined\pdftexversion\else\setmainfont{NanumMyeongjo}\rmfamily\fi\fontsize{5.000000}{6.000000}\selectfont\catcode`\^=\active\def^{\ifmmode\sp\else\^{}\fi}\catcode`\%=\active\def%{\%}34,860}}%
\end{pgfscope}%
\begin{pgfscope}%
\definecolor{textcolor}{rgb}{1.000000,1.000000,1.000000}%
\pgfsetstrokecolor{textcolor}%
\pgfsetfillcolor{textcolor}%
\pgftext[x=4.221612in,y=0.882214in,,]{\color{textcolor}{\ifdefined\pdftexversion\else\setmainfont{NanumMyeongjo}\rmfamily\fi\fontsize{5.000000}{6.000000}\selectfont\catcode`\^=\active\def^{\ifmmode\sp\else\^{}\fi}\catcode`\%=\active\def%{\%}39,558}}%
\end{pgfscope}%
\begin{pgfscope}%
\definecolor{textcolor}{rgb}{1.000000,1.000000,1.000000}%
\pgfsetstrokecolor{textcolor}%
\pgfsetfillcolor{textcolor}%
\pgftext[x=4.560869in,y=1.012783in,,]{\color{textcolor}{\ifdefined\pdftexversion\else\setmainfont{NanumMyeongjo}\rmfamily\fi\fontsize{5.000000}{6.000000}\selectfont\catcode`\^=\active\def^{\ifmmode\sp\else\^{}\fi}\catcode`\%=\active\def%{\%}52,172}}%
\end{pgfscope}%
\begin{pgfscope}%
\definecolor{textcolor}{rgb}{1.000000,1.000000,1.000000}%
\pgfsetstrokecolor{textcolor}%
\pgfsetfillcolor{textcolor}%
\pgftext[x=1.168298in,y=0.920617in,,]{\color{textcolor}{\ifdefined\pdftexversion\else\setmainfont{NanumMyeongjo}\rmfamily\fi\fontsize{5.000000}{6.000000}\selectfont\catcode`\^=\active\def^{\ifmmode\sp\else\^{}\fi}\catcode`\%=\active\def%{\%}28,326}}%
\end{pgfscope}%
\begin{pgfscope}%
\definecolor{textcolor}{rgb}{1.000000,1.000000,1.000000}%
\pgfsetstrokecolor{textcolor}%
\pgfsetfillcolor{textcolor}%
\pgftext[x=1.507555in,y=0.794862in,,]{\color{textcolor}{\ifdefined\pdftexversion\else\setmainfont{NanumMyeongjo}\rmfamily\fi\fontsize{5.000000}{6.000000}\selectfont\catcode`\^=\active\def^{\ifmmode\sp\else\^{}\fi}\catcode`\%=\active\def%{\%}20,011}}%
\end{pgfscope}%
\begin{pgfscope}%
\definecolor{textcolor}{rgb}{1.000000,1.000000,1.000000}%
\pgfsetstrokecolor{textcolor}%
\pgfsetfillcolor{textcolor}%
\pgftext[x=1.846812in,y=0.687231in,,]{\color{textcolor}{\ifdefined\pdftexversion\else\setmainfont{NanumMyeongjo}\rmfamily\fi\fontsize{5.000000}{6.000000}\selectfont\catcode`\^=\active\def^{\ifmmode\sp\else\^{}\fi}\catcode`\%=\active\def%{\%}13,405}}%
\end{pgfscope}%
\begin{pgfscope}%
\definecolor{textcolor}{rgb}{1.000000,1.000000,1.000000}%
\pgfsetstrokecolor{textcolor}%
\pgfsetfillcolor{textcolor}%
\pgftext[x=2.186069in,y=0.785401in,,]{\color{textcolor}{\ifdefined\pdftexversion\else\setmainfont{NanumMyeongjo}\rmfamily\fi\fontsize{5.000000}{6.000000}\selectfont\catcode`\^=\active\def^{\ifmmode\sp\else\^{}\fi}\catcode`\%=\active\def%{\%}17,511}}%
\end{pgfscope}%
\begin{pgfscope}%
\definecolor{textcolor}{rgb}{1.000000,1.000000,1.000000}%
\pgfsetstrokecolor{textcolor}%
\pgfsetfillcolor{textcolor}%
\pgftext[x=2.525326in,y=0.858117in,,]{\color{textcolor}{\ifdefined\pdftexversion\else\setmainfont{NanumMyeongjo}\rmfamily\fi\fontsize{5.000000}{6.000000}\selectfont\catcode`\^=\active\def^{\ifmmode\sp\else\^{}\fi}\catcode`\%=\active\def%{\%}18,776}}%
\end{pgfscope}%
\begin{pgfscope}%
\definecolor{textcolor}{rgb}{1.000000,1.000000,1.000000}%
\pgfsetstrokecolor{textcolor}%
\pgfsetfillcolor{textcolor}%
\pgftext[x=2.864583in,y=0.949010in,,]{\color{textcolor}{\ifdefined\pdftexversion\else\setmainfont{NanumMyeongjo}\rmfamily\fi\fontsize{5.000000}{6.000000}\selectfont\catcode`\^=\active\def^{\ifmmode\sp\else\^{}\fi}\catcode`\%=\active\def%{\%}21,264}}%
\end{pgfscope}%
\begin{pgfscope}%
\definecolor{textcolor}{rgb}{1.000000,1.000000,1.000000}%
\pgfsetstrokecolor{textcolor}%
\pgfsetfillcolor{textcolor}%
\pgftext[x=3.203840in,y=0.816444in,,]{\color{textcolor}{\ifdefined\pdftexversion\else\setmainfont{NanumMyeongjo}\rmfamily\fi\fontsize{5.000000}{6.000000}\selectfont\catcode`\^=\active\def^{\ifmmode\sp\else\^{}\fi}\catcode`\%=\active\def%{\%}17,258}}%
\end{pgfscope}%
\begin{pgfscope}%
\definecolor{textcolor}{rgb}{1.000000,1.000000,1.000000}%
\pgfsetstrokecolor{textcolor}%
\pgfsetfillcolor{textcolor}%
\pgftext[x=3.543097in,y=0.984390in,,]{\color{textcolor}{\ifdefined\pdftexversion\else\setmainfont{NanumMyeongjo}\rmfamily\fi\fontsize{5.000000}{6.000000}\selectfont\catcode`\^=\active\def^{\ifmmode\sp\else\^{}\fi}\catcode`\%=\active\def%{\%}24,981}}%
\end{pgfscope}%
\begin{pgfscope}%
\definecolor{textcolor}{rgb}{1.000000,1.000000,1.000000}%
\pgfsetstrokecolor{textcolor}%
\pgfsetfillcolor{textcolor}%
\pgftext[x=3.882355in,y=1.091378in,,]{\color{textcolor}{\ifdefined\pdftexversion\else\setmainfont{NanumMyeongjo}\rmfamily\fi\fontsize{5.000000}{6.000000}\selectfont\catcode`\^=\active\def^{\ifmmode\sp\else\^{}\fi}\catcode`\%=\active\def%{\%}24,905}}%
\end{pgfscope}%
\begin{pgfscope}%
\definecolor{textcolor}{rgb}{1.000000,1.000000,1.000000}%
\pgfsetstrokecolor{textcolor}%
\pgfsetfillcolor{textcolor}%
\pgftext[x=4.221612in,y=1.118736in,,]{\color{textcolor}{\ifdefined\pdftexversion\else\setmainfont{NanumMyeongjo}\rmfamily\fi\fontsize{5.000000}{6.000000}\selectfont\catcode`\^=\active\def^{\ifmmode\sp\else\^{}\fi}\catcode`\%=\active\def%{\%}22,850}}%
\end{pgfscope}%
\begin{pgfscope}%
\definecolor{textcolor}{rgb}{1.000000,1.000000,1.000000}%
\pgfsetstrokecolor{textcolor}%
\pgfsetfillcolor{textcolor}%
\pgftext[x=4.560869in,y=1.269799in,,]{\color{textcolor}{\ifdefined\pdftexversion\else\setmainfont{NanumMyeongjo}\rmfamily\fi\fontsize{5.000000}{6.000000}\selectfont\catcode`\^=\active\def^{\ifmmode\sp\else\^{}\fi}\catcode`\%=\active\def%{\%}24,830}}%
\end{pgfscope}%
\begin{pgfscope}%
\definecolor{textcolor}{rgb}{1.000000,1.000000,1.000000}%
\pgfsetstrokecolor{textcolor}%
\pgfsetfillcolor{textcolor}%
\pgftext[x=1.168298in,y=1.133010in,,]{\color{textcolor}{\ifdefined\pdftexversion\else\setmainfont{NanumMyeongjo}\rmfamily\fi\fontsize{5.000000}{6.000000}\selectfont\catcode`\^=\active\def^{\ifmmode\sp\else\^{}\fi}\catcode`\%=\active\def%{\%}20,519}}%
\end{pgfscope}%
\begin{pgfscope}%
\definecolor{textcolor}{rgb}{1.000000,1.000000,1.000000}%
\pgfsetstrokecolor{textcolor}%
\pgfsetfillcolor{textcolor}%
\pgftext[x=1.507555in,y=1.004626in,,]{\color{textcolor}{\ifdefined\pdftexversion\else\setmainfont{NanumMyeongjo}\rmfamily\fi\fontsize{5.000000}{6.000000}\selectfont\catcode`\^=\active\def^{\ifmmode\sp\else\^{}\fi}\catcode`\%=\active\def%{\%}20,265}}%
\end{pgfscope}%
\begin{pgfscope}%
\definecolor{textcolor}{rgb}{1.000000,1.000000,1.000000}%
\pgfsetstrokecolor{textcolor}%
\pgfsetfillcolor{textcolor}%
\pgftext[x=1.846812in,y=0.854546in,,]{\color{textcolor}{\ifdefined\pdftexversion\else\setmainfont{NanumMyeongjo}\rmfamily\fi\fontsize{5.000000}{6.000000}\selectfont\catcode`\^=\active\def^{\ifmmode\sp\else\^{}\fi}\catcode`\%=\active\def%{\%}16,164}}%
\end{pgfscope}%
\begin{pgfscope}%
\definecolor{textcolor}{rgb}{1.000000,1.000000,1.000000}%
\pgfsetstrokecolor{textcolor}%
\pgfsetfillcolor{textcolor}%
\pgftext[x=2.186069in,y=0.954641in,,]{\color{textcolor}{\ifdefined\pdftexversion\else\setmainfont{NanumMyeongjo}\rmfamily\fi\fontsize{5.000000}{6.000000}\selectfont\catcode`\^=\active\def^{\ifmmode\sp\else\^{}\fi}\catcode`\%=\active\def%{\%}16,350}}%
\end{pgfscope}%
\begin{pgfscope}%
\definecolor{textcolor}{rgb}{1.000000,1.000000,1.000000}%
\pgfsetstrokecolor{textcolor}%
\pgfsetfillcolor{textcolor}%
\pgftext[x=2.525326in,y=1.022668in,,]{\color{textcolor}{\ifdefined\pdftexversion\else\setmainfont{NanumMyeongjo}\rmfamily\fi\fontsize{5.000000}{6.000000}\selectfont\catcode`\^=\active\def^{\ifmmode\sp\else\^{}\fi}\catcode`\%=\active\def%{\%}15,897}}%
\end{pgfscope}%
\begin{pgfscope}%
\definecolor{textcolor}{rgb}{1.000000,1.000000,1.000000}%
\pgfsetstrokecolor{textcolor}%
\pgfsetfillcolor{textcolor}%
\pgftext[x=2.864583in,y=1.124502in,,]{\color{textcolor}{\ifdefined\pdftexversion\else\setmainfont{NanumMyeongjo}\rmfamily\fi\fontsize{5.000000}{6.000000}\selectfont\catcode`\^=\active\def^{\ifmmode\sp\else\^{}\fi}\catcode`\%=\active\def%{\%}16,954}}%
\end{pgfscope}%
\begin{pgfscope}%
\definecolor{textcolor}{rgb}{1.000000,1.000000,1.000000}%
\pgfsetstrokecolor{textcolor}%
\pgfsetfillcolor{textcolor}%
\pgftext[x=3.203840in,y=0.944165in,,]{\color{textcolor}{\ifdefined\pdftexversion\else\setmainfont{NanumMyeongjo}\rmfamily\fi\fontsize{5.000000}{6.000000}\selectfont\catcode`\^=\active\def^{\ifmmode\sp\else\^{}\fi}\catcode`\%=\active\def%{\%}12,339}}%
\end{pgfscope}%
\begin{pgfscope}%
\definecolor{textcolor}{rgb}{1.000000,1.000000,1.000000}%
\pgfsetstrokecolor{textcolor}%
\pgfsetfillcolor{textcolor}%
\pgftext[x=3.543097in,y=1.149158in,,]{\color{textcolor}{\ifdefined\pdftexversion\else\setmainfont{NanumMyeongjo}\rmfamily\fi\fontsize{5.000000}{6.000000}\selectfont\catcode`\^=\active\def^{\ifmmode\sp\else\^{}\fi}\catcode`\%=\active\def%{\%}15,918}}%
\end{pgfscope}%
\begin{pgfscope}%
\definecolor{textcolor}{rgb}{1.000000,1.000000,1.000000}%
\pgfsetstrokecolor{textcolor}%
\pgfsetfillcolor{textcolor}%
\pgftext[x=3.882355in,y=1.305376in,,]{\color{textcolor}{\ifdefined\pdftexversion\else\setmainfont{NanumMyeongjo}\rmfamily\fi\fontsize{5.000000}{6.000000}\selectfont\catcode`\^=\active\def^{\ifmmode\sp\else\^{}\fi}\catcode`\%=\active\def%{\%}20,674}}%
\end{pgfscope}%
\begin{pgfscope}%
\definecolor{textcolor}{rgb}{1.000000,1.000000,1.000000}%
\pgfsetstrokecolor{textcolor}%
\pgfsetfillcolor{textcolor}%
\pgftext[x=4.221612in,y=1.331202in,,]{\color{textcolor}{\ifdefined\pdftexversion\else\setmainfont{NanumMyeongjo}\rmfamily\fi\fontsize{5.000000}{6.000000}\selectfont\catcode`\^=\active\def^{\ifmmode\sp\else\^{}\fi}\catcode`\%=\active\def%{\%}20,526}}%
\end{pgfscope}%
\begin{pgfscope}%
\definecolor{textcolor}{rgb}{1.000000,1.000000,1.000000}%
\pgfsetstrokecolor{textcolor}%
\pgfsetfillcolor{textcolor}%
\pgftext[x=4.560869in,y=1.498030in,,]{\color{textcolor}{\ifdefined\pdftexversion\else\setmainfont{NanumMyeongjo}\rmfamily\fi\fontsize{5.000000}{6.000000}\selectfont\catcode`\^=\active\def^{\ifmmode\sp\else\^{}\fi}\catcode`\%=\active\def%{\%}22,049}}%
\end{pgfscope}%
\begin{pgfscope}%
\pgfsetbuttcap%
\pgfsetmiterjoin%
\definecolor{currentfill}{rgb}{0.337255,0.713725,0.627451}%
\pgfsetfillcolor{currentfill}%
\pgfsetlinewidth{1.003750pt}%
\definecolor{currentstroke}{rgb}{0.266667,0.266667,0.266667}%
\pgfsetstrokecolor{currentstroke}%
\pgfsetdash{}{0pt}%
\pgfpathmoveto{\pgfqpoint{4.843750in}{2.326470in}}%
\pgfpathlineto{\pgfqpoint{4.982639in}{2.326470in}}%
\pgfpathlineto{\pgfqpoint{4.982639in}{2.375081in}}%
\pgfpathlineto{\pgfqpoint{4.843750in}{2.375081in}}%
\pgfpathlineto{\pgfqpoint{4.843750in}{2.326470in}}%
\pgfpathclose%
\pgfusepath{stroke,fill}%
\end{pgfscope}%
\begin{pgfscope}%
\definecolor{textcolor}{rgb}{0.000000,0.000000,0.000000}%
\pgfsetstrokecolor{textcolor}%
\pgfsetfillcolor{textcolor}%
\pgftext[x=5.038194in,y=2.326470in,left,base]{\color{textcolor}{\ifdefined\pdftexversion\else\setmainfont{NanumMyeongjo}\rmfamily\fi\fontsize{5.000000}{6.000000}\selectfont\catcode`\^=\active\def^{\ifmmode\sp\else\^{}\fi}\catcode`\%=\active\def%{\%}전라북도}}%
\end{pgfscope}%
\begin{pgfscope}%
\pgfsetbuttcap%
\pgfsetmiterjoin%
\definecolor{currentfill}{rgb}{0.235294,0.490196,0.764706}%
\pgfsetfillcolor{currentfill}%
\pgfsetlinewidth{1.003750pt}%
\definecolor{currentstroke}{rgb}{0.266667,0.266667,0.266667}%
\pgfsetstrokecolor{currentstroke}%
\pgfsetdash{}{0pt}%
\pgfpathmoveto{\pgfqpoint{4.843750in}{2.220202in}}%
\pgfpathlineto{\pgfqpoint{4.982639in}{2.220202in}}%
\pgfpathlineto{\pgfqpoint{4.982639in}{2.268813in}}%
\pgfpathlineto{\pgfqpoint{4.843750in}{2.268813in}}%
\pgfpathlineto{\pgfqpoint{4.843750in}{2.220202in}}%
\pgfpathclose%
\pgfusepath{stroke,fill}%
\end{pgfscope}%
\begin{pgfscope}%
\definecolor{textcolor}{rgb}{0.000000,0.000000,0.000000}%
\pgfsetstrokecolor{textcolor}%
\pgfsetfillcolor{textcolor}%
\pgftext[x=5.038194in,y=2.220202in,left,base]{\color{textcolor}{\ifdefined\pdftexversion\else\setmainfont{NanumMyeongjo}\rmfamily\fi\fontsize{5.000000}{6.000000}\selectfont\catcode`\^=\active\def^{\ifmmode\sp\else\^{}\fi}\catcode`\%=\active\def%{\%}경상북도}}%
\end{pgfscope}%
\begin{pgfscope}%
\pgfsetbuttcap%
\pgfsetmiterjoin%
\definecolor{currentfill}{rgb}{0.725490,0.486275,0.164706}%
\pgfsetfillcolor{currentfill}%
\pgfsetlinewidth{1.003750pt}%
\definecolor{currentstroke}{rgb}{0.266667,0.266667,0.266667}%
\pgfsetstrokecolor{currentstroke}%
\pgfsetdash{}{0pt}%
\pgfpathmoveto{\pgfqpoint{4.843750in}{2.113933in}}%
\pgfpathlineto{\pgfqpoint{4.982639in}{2.113933in}}%
\pgfpathlineto{\pgfqpoint{4.982639in}{2.162544in}}%
\pgfpathlineto{\pgfqpoint{4.843750in}{2.162544in}}%
\pgfpathlineto{\pgfqpoint{4.843750in}{2.113933in}}%
\pgfpathclose%
\pgfusepath{stroke,fill}%
\end{pgfscope}%
\begin{pgfscope}%
\definecolor{textcolor}{rgb}{0.000000,0.000000,0.000000}%
\pgfsetstrokecolor{textcolor}%
\pgfsetfillcolor{textcolor}%
\pgftext[x=5.038194in,y=2.113933in,left,base]{\color{textcolor}{\ifdefined\pdftexversion\else\setmainfont{NanumMyeongjo}\rmfamily\fi\fontsize{5.000000}{6.000000}\selectfont\catcode`\^=\active\def^{\ifmmode\sp\else\^{}\fi}\catcode`\%=\active\def%{\%}전라남도}}%
\end{pgfscope}%
\begin{pgfscope}%
\pgfsetbuttcap%
\pgfsetmiterjoin%
\definecolor{currentfill}{rgb}{0.733333,0.321569,0.733333}%
\pgfsetfillcolor{currentfill}%
\pgfsetlinewidth{1.003750pt}%
\definecolor{currentstroke}{rgb}{0.266667,0.266667,0.266667}%
\pgfsetstrokecolor{currentstroke}%
\pgfsetdash{}{0pt}%
\pgfpathmoveto{\pgfqpoint{4.843750in}{2.007664in}}%
\pgfpathlineto{\pgfqpoint{4.982639in}{2.007664in}}%
\pgfpathlineto{\pgfqpoint{4.982639in}{2.056275in}}%
\pgfpathlineto{\pgfqpoint{4.843750in}{2.056275in}}%
\pgfpathlineto{\pgfqpoint{4.843750in}{2.007664in}}%
\pgfpathclose%
\pgfusepath{stroke,fill}%
\end{pgfscope}%
\begin{pgfscope}%
\definecolor{textcolor}{rgb}{0.000000,0.000000,0.000000}%
\pgfsetstrokecolor{textcolor}%
\pgfsetfillcolor{textcolor}%
\pgftext[x=5.038194in,y=2.007664in,left,base]{\color{textcolor}{\ifdefined\pdftexversion\else\setmainfont{NanumMyeongjo}\rmfamily\fi\fontsize{5.000000}{6.000000}\selectfont\catcode`\^=\active\def^{\ifmmode\sp\else\^{}\fi}\catcode`\%=\active\def%{\%}충청남도}}%
\end{pgfscope}%
\begin{pgfscope}%
\pgfsetbuttcap%
\pgfsetmiterjoin%
\definecolor{currentfill}{rgb}{0.549020,0.247059,0.121569}%
\pgfsetfillcolor{currentfill}%
\pgfsetlinewidth{1.003750pt}%
\definecolor{currentstroke}{rgb}{0.266667,0.266667,0.266667}%
\pgfsetstrokecolor{currentstroke}%
\pgfsetdash{}{0pt}%
\pgfpathmoveto{\pgfqpoint{4.843750in}{1.901395in}}%
\pgfpathlineto{\pgfqpoint{4.982639in}{1.901395in}}%
\pgfpathlineto{\pgfqpoint{4.982639in}{1.950006in}}%
\pgfpathlineto{\pgfqpoint{4.843750in}{1.950006in}}%
\pgfpathlineto{\pgfqpoint{4.843750in}{1.901395in}}%
\pgfpathclose%
\pgfusepath{stroke,fill}%
\end{pgfscope}%
\begin{pgfscope}%
\definecolor{textcolor}{rgb}{0.000000,0.000000,0.000000}%
\pgfsetstrokecolor{textcolor}%
\pgfsetfillcolor{textcolor}%
\pgftext[x=5.038194in,y=1.901395in,left,base]{\color{textcolor}{\ifdefined\pdftexversion\else\setmainfont{NanumMyeongjo}\rmfamily\fi\fontsize{5.000000}{6.000000}\selectfont\catcode`\^=\active\def^{\ifmmode\sp\else\^{}\fi}\catcode`\%=\active\def%{\%}충청북도}}%
\end{pgfscope}%
\begin{pgfscope}%
\pgfsetbuttcap%
\pgfsetmiterjoin%
\definecolor{currentfill}{rgb}{0.701961,0.760784,0.360784}%
\pgfsetfillcolor{currentfill}%
\pgfsetlinewidth{1.003750pt}%
\definecolor{currentstroke}{rgb}{0.266667,0.266667,0.266667}%
\pgfsetstrokecolor{currentstroke}%
\pgfsetdash{}{0pt}%
\pgfpathmoveto{\pgfqpoint{4.843750in}{1.795127in}}%
\pgfpathlineto{\pgfqpoint{4.982639in}{1.795127in}}%
\pgfpathlineto{\pgfqpoint{4.982639in}{1.843738in}}%
\pgfpathlineto{\pgfqpoint{4.843750in}{1.843738in}}%
\pgfpathlineto{\pgfqpoint{4.843750in}{1.795127in}}%
\pgfpathclose%
\pgfusepath{stroke,fill}%
\end{pgfscope}%
\begin{pgfscope}%
\definecolor{textcolor}{rgb}{0.000000,0.000000,0.000000}%
\pgfsetstrokecolor{textcolor}%
\pgfsetfillcolor{textcolor}%
\pgftext[x=5.038194in,y=1.795127in,left,base]{\color{textcolor}{\ifdefined\pdftexversion\else\setmainfont{NanumMyeongjo}\rmfamily\fi\fontsize{5.000000}{6.000000}\selectfont\catcode`\^=\active\def^{\ifmmode\sp\else\^{}\fi}\catcode`\%=\active\def%{\%}경기도}}%
\end{pgfscope}%
\begin{pgfscope}%
\pgfsetbuttcap%
\pgfsetmiterjoin%
\definecolor{currentfill}{rgb}{0.447059,0.447059,0.447059}%
\pgfsetfillcolor{currentfill}%
\pgfsetlinewidth{1.003750pt}%
\definecolor{currentstroke}{rgb}{0.266667,0.266667,0.266667}%
\pgfsetstrokecolor{currentstroke}%
\pgfsetdash{}{0pt}%
\pgfpathmoveto{\pgfqpoint{4.843750in}{1.688858in}}%
\pgfpathlineto{\pgfqpoint{4.982639in}{1.688858in}}%
\pgfpathlineto{\pgfqpoint{4.982639in}{1.737469in}}%
\pgfpathlineto{\pgfqpoint{4.843750in}{1.737469in}}%
\pgfpathlineto{\pgfqpoint{4.843750in}{1.688858in}}%
\pgfpathclose%
\pgfusepath{stroke,fill}%
\end{pgfscope}%
\begin{pgfscope}%
\definecolor{textcolor}{rgb}{0.000000,0.000000,0.000000}%
\pgfsetstrokecolor{textcolor}%
\pgfsetfillcolor{textcolor}%
\pgftext[x=5.038194in,y=1.688858in,left,base]{\color{textcolor}{\ifdefined\pdftexversion\else\setmainfont{NanumMyeongjo}\rmfamily\fi\fontsize{5.000000}{6.000000}\selectfont\catcode`\^=\active\def^{\ifmmode\sp\else\^{}\fi}\catcode`\%=\active\def%{\%}기타}}%
\end{pgfscope}%
\begin{pgfscope}%
\definecolor{textcolor}{rgb}{0.333333,0.333333,0.333333}%
\pgfsetstrokecolor{textcolor}%
\pgfsetfillcolor{textcolor}%
\pgftext[x=1.736111in,y=0.333333in,,top]{\color{textcolor}{\ifdefined\pdftexversion\else\setmainfont{NanumMyeongjo}\rmfamily\fi\fontsize{5.000000}{6.000000}\selectfont\catcode`\^=\active\def^{\ifmmode\sp\else\^{}\fi}\catcode`\%=\active\def%{\%}출처: 국가농식품통계서비스(KASS) 자료 기반 저자 작성}}%
\end{pgfscope}%
\begin{pgfscope}%
\definecolor{textcolor}{rgb}{0.333333,0.333333,0.333333}%
\pgfsetstrokecolor{textcolor}%
\pgfsetfillcolor{textcolor}%
\pgftext[x=4.513889in,y=2.583333in,,top]{\color{textcolor}{\ifdefined\pdftexversion\else\setmainfont{NanumMyeongjo}\rmfamily\fi\fontsize{5.000000}{6.000000}\selectfont\catcode`\^=\active\def^{\ifmmode\sp\else\^{}\fi}\catcode`\%=\active\def%{\%}(단위: 톤)}}%
\end{pgfscope}%
\end{pgfpicture}%
\makeatother%
\endgroup%
}
\end{center}
}

\slide
{\maintitle}
{\chapterone}
{국내 콩 생산면적}{
\begin{center}
    \hspace*{-55pt}\resizebox{1.5\textwidth}{!}{%% Creator: Matplotlib, PGF backend
%%
%% To include the figure in your LaTeX document, write
%%   \input{<filename>.pgf}
%%
%% Make sure the required packages are loaded in your preamble
%%   \usepackage{pgf}
%%
%% Also ensure that all the required font packages are loaded; for instance,
%% the lmodern package is sometimes necessary when using math font.
%%   \usepackage{lmodern}
%%
%% Figures using additional raster images can only be included by \input if
%% they are in the same directory as the main LaTeX file. For loading figures
%% from other directories you can use the `import` package
%%   \usepackage{import}
%%
%% and then include the figures with
%%   \import{<path to file>}{<filename>.pgf}
%%
%% Matplotlib used the following preamble
%%   \def\mathdefault#1{#1}
%%   \everymath=\expandafter{\the\everymath\displaystyle}
%%   \IfFileExists{scrextend.sty}{
%%     \usepackage[fontsize=9.000000pt]{scrextend}
%%   }{
%%     \renewcommand{\normalsize}{\fontsize{9.000000}{10.800000}\selectfont}
%%     \normalsize
%%   }
%%   
%%   \ifdefined\pdftexversion\else  % non-pdftex case.
%%     \usepackage{fontspec}
%%     \setmainfont{DejaVuSerif.ttf}[Path=\detokenize{/home/user/.cache/pypoetry/virtualenvs/graph-KASAOWVd-py3.12/lib/python3.12/site-packages/matplotlib/mpl-data/fonts/ttf/}]
%%     \setsansfont{DejaVuSans.ttf}[Path=\detokenize{/home/user/.cache/pypoetry/virtualenvs/graph-KASAOWVd-py3.12/lib/python3.12/site-packages/matplotlib/mpl-data/fonts/ttf/}]
%%     \setmonofont{DejaVuSansMono.ttf}[Path=\detokenize{/home/user/.cache/pypoetry/virtualenvs/graph-KASAOWVd-py3.12/lib/python3.12/site-packages/matplotlib/mpl-data/fonts/ttf/}]
%%   \fi
%%   \makeatletter\@ifpackageloaded{underscore}{}{\usepackage[strings]{underscore}}\makeatother
%%
\begingroup%
\makeatletter%
\begin{pgfpicture}%
\pgfpathrectangle{\pgfpointorigin}{\pgfqpoint{6.250000in}{3.194444in}}%
\pgfusepath{use as bounding box, clip}%
\begin{pgfscope}%
\pgfsetbuttcap%
\pgfsetmiterjoin%
\definecolor{currentfill}{rgb}{1.000000,1.000000,1.000000}%
\pgfsetfillcolor{currentfill}%
\pgfsetlinewidth{0.000000pt}%
\definecolor{currentstroke}{rgb}{1.000000,1.000000,1.000000}%
\pgfsetstrokecolor{currentstroke}%
\pgfsetdash{}{0pt}%
\pgfpathmoveto{\pgfqpoint{0.000000in}{0.000000in}}%
\pgfpathlineto{\pgfqpoint{6.250000in}{0.000000in}}%
\pgfpathlineto{\pgfqpoint{6.250000in}{3.194444in}}%
\pgfpathlineto{\pgfqpoint{0.000000in}{3.194444in}}%
\pgfpathlineto{\pgfqpoint{0.000000in}{0.000000in}}%
\pgfpathclose%
\pgfusepath{fill}%
\end{pgfscope}%
\begin{pgfscope}%
\pgfsetbuttcap%
\pgfsetmiterjoin%
\definecolor{currentfill}{rgb}{1.000000,1.000000,1.000000}%
\pgfsetfillcolor{currentfill}%
\pgfsetlinewidth{0.000000pt}%
\definecolor{currentstroke}{rgb}{0.000000,0.000000,0.000000}%
\pgfsetstrokecolor{currentstroke}%
\pgfsetstrokeopacity{0.000000}%
\pgfsetdash{}{0pt}%
\pgfpathmoveto{\pgfqpoint{0.781250in}{0.638889in}}%
\pgfpathlineto{\pgfqpoint{5.000000in}{0.638889in}}%
\pgfpathlineto{\pgfqpoint{5.000000in}{2.811111in}}%
\pgfpathlineto{\pgfqpoint{0.781250in}{2.811111in}}%
\pgfpathlineto{\pgfqpoint{0.781250in}{0.638889in}}%
\pgfpathclose%
\pgfusepath{fill}%
\end{pgfscope}%
\begin{pgfscope}%
\pgfsetbuttcap%
\pgfsetroundjoin%
\definecolor{currentfill}{rgb}{0.000000,0.000000,0.000000}%
\pgfsetfillcolor{currentfill}%
\pgfsetlinewidth{0.752812pt}%
\definecolor{currentstroke}{rgb}{0.000000,0.000000,0.000000}%
\pgfsetstrokecolor{currentstroke}%
\pgfsetdash{}{0pt}%
\pgfsys@defobject{currentmarker}{\pgfqpoint{0.000000in}{-0.013889in}}{\pgfqpoint{0.000000in}{0.000000in}}{%
\pgfpathmoveto{\pgfqpoint{0.000000in}{0.000000in}}%
\pgfpathlineto{\pgfqpoint{0.000000in}{-0.013889in}}%
\pgfusepath{stroke,fill}%
}%
\begin{pgfscope}%
\pgfsys@transformshift{1.098463in}{0.638889in}%
\pgfsys@useobject{currentmarker}{}%
\end{pgfscope}%
\end{pgfscope}%
\begin{pgfscope}%
\definecolor{textcolor}{rgb}{0.000000,0.000000,0.000000}%
\pgfsetstrokecolor{textcolor}%
\pgfsetfillcolor{textcolor}%
\pgftext[x=1.028589in, y=0.315885in, left, base,rotate=45.000000]{\color{textcolor}{\ifdefined\pdftexversion\else\setmainfont{NanumMyeongjo}\rmfamily\fi\fontsize{9.000000}{10.800000}\selectfont\catcode`\^=\active\def^{\ifmmode\sp\else\^{}\fi}\catcode`\%=\active\def%{\%}2014}}%
\end{pgfscope}%
\begin{pgfscope}%
\pgfsetbuttcap%
\pgfsetroundjoin%
\definecolor{currentfill}{rgb}{0.000000,0.000000,0.000000}%
\pgfsetfillcolor{currentfill}%
\pgfsetlinewidth{0.752812pt}%
\definecolor{currentstroke}{rgb}{0.000000,0.000000,0.000000}%
\pgfsetstrokecolor{currentstroke}%
\pgfsetdash{}{0pt}%
\pgfsys@defobject{currentmarker}{\pgfqpoint{0.000000in}{-0.013889in}}{\pgfqpoint{0.000000in}{0.000000in}}{%
\pgfpathmoveto{\pgfqpoint{0.000000in}{0.000000in}}%
\pgfpathlineto{\pgfqpoint{0.000000in}{-0.013889in}}%
\pgfusepath{stroke,fill}%
}%
\begin{pgfscope}%
\pgfsys@transformshift{1.456895in}{0.638889in}%
\pgfsys@useobject{currentmarker}{}%
\end{pgfscope}%
\end{pgfscope}%
\begin{pgfscope}%
\definecolor{textcolor}{rgb}{0.000000,0.000000,0.000000}%
\pgfsetstrokecolor{textcolor}%
\pgfsetfillcolor{textcolor}%
\pgftext[x=1.387022in, y=0.315885in, left, base,rotate=45.000000]{\color{textcolor}{\ifdefined\pdftexversion\else\setmainfont{NanumMyeongjo}\rmfamily\fi\fontsize{9.000000}{10.800000}\selectfont\catcode`\^=\active\def^{\ifmmode\sp\else\^{}\fi}\catcode`\%=\active\def%{\%}2015}}%
\end{pgfscope}%
\begin{pgfscope}%
\pgfsetbuttcap%
\pgfsetroundjoin%
\definecolor{currentfill}{rgb}{0.000000,0.000000,0.000000}%
\pgfsetfillcolor{currentfill}%
\pgfsetlinewidth{0.752812pt}%
\definecolor{currentstroke}{rgb}{0.000000,0.000000,0.000000}%
\pgfsetstrokecolor{currentstroke}%
\pgfsetdash{}{0pt}%
\pgfsys@defobject{currentmarker}{\pgfqpoint{0.000000in}{-0.013889in}}{\pgfqpoint{0.000000in}{0.000000in}}{%
\pgfpathmoveto{\pgfqpoint{0.000000in}{0.000000in}}%
\pgfpathlineto{\pgfqpoint{0.000000in}{-0.013889in}}%
\pgfusepath{stroke,fill}%
}%
\begin{pgfscope}%
\pgfsys@transformshift{1.815328in}{0.638889in}%
\pgfsys@useobject{currentmarker}{}%
\end{pgfscope}%
\end{pgfscope}%
\begin{pgfscope}%
\definecolor{textcolor}{rgb}{0.000000,0.000000,0.000000}%
\pgfsetstrokecolor{textcolor}%
\pgfsetfillcolor{textcolor}%
\pgftext[x=1.745454in, y=0.315885in, left, base,rotate=45.000000]{\color{textcolor}{\ifdefined\pdftexversion\else\setmainfont{NanumMyeongjo}\rmfamily\fi\fontsize{9.000000}{10.800000}\selectfont\catcode`\^=\active\def^{\ifmmode\sp\else\^{}\fi}\catcode`\%=\active\def%{\%}2016}}%
\end{pgfscope}%
\begin{pgfscope}%
\pgfsetbuttcap%
\pgfsetroundjoin%
\definecolor{currentfill}{rgb}{0.000000,0.000000,0.000000}%
\pgfsetfillcolor{currentfill}%
\pgfsetlinewidth{0.752812pt}%
\definecolor{currentstroke}{rgb}{0.000000,0.000000,0.000000}%
\pgfsetstrokecolor{currentstroke}%
\pgfsetdash{}{0pt}%
\pgfsys@defobject{currentmarker}{\pgfqpoint{0.000000in}{-0.013889in}}{\pgfqpoint{0.000000in}{0.000000in}}{%
\pgfpathmoveto{\pgfqpoint{0.000000in}{0.000000in}}%
\pgfpathlineto{\pgfqpoint{0.000000in}{-0.013889in}}%
\pgfusepath{stroke,fill}%
}%
\begin{pgfscope}%
\pgfsys@transformshift{2.173760in}{0.638889in}%
\pgfsys@useobject{currentmarker}{}%
\end{pgfscope}%
\end{pgfscope}%
\begin{pgfscope}%
\definecolor{textcolor}{rgb}{0.000000,0.000000,0.000000}%
\pgfsetstrokecolor{textcolor}%
\pgfsetfillcolor{textcolor}%
\pgftext[x=2.103887in, y=0.315885in, left, base,rotate=45.000000]{\color{textcolor}{\ifdefined\pdftexversion\else\setmainfont{NanumMyeongjo}\rmfamily\fi\fontsize{9.000000}{10.800000}\selectfont\catcode`\^=\active\def^{\ifmmode\sp\else\^{}\fi}\catcode`\%=\active\def%{\%}2017}}%
\end{pgfscope}%
\begin{pgfscope}%
\pgfsetbuttcap%
\pgfsetroundjoin%
\definecolor{currentfill}{rgb}{0.000000,0.000000,0.000000}%
\pgfsetfillcolor{currentfill}%
\pgfsetlinewidth{0.752812pt}%
\definecolor{currentstroke}{rgb}{0.000000,0.000000,0.000000}%
\pgfsetstrokecolor{currentstroke}%
\pgfsetdash{}{0pt}%
\pgfsys@defobject{currentmarker}{\pgfqpoint{0.000000in}{-0.013889in}}{\pgfqpoint{0.000000in}{0.000000in}}{%
\pgfpathmoveto{\pgfqpoint{0.000000in}{0.000000in}}%
\pgfpathlineto{\pgfqpoint{0.000000in}{-0.013889in}}%
\pgfusepath{stroke,fill}%
}%
\begin{pgfscope}%
\pgfsys@transformshift{2.532193in}{0.638889in}%
\pgfsys@useobject{currentmarker}{}%
\end{pgfscope}%
\end{pgfscope}%
\begin{pgfscope}%
\definecolor{textcolor}{rgb}{0.000000,0.000000,0.000000}%
\pgfsetstrokecolor{textcolor}%
\pgfsetfillcolor{textcolor}%
\pgftext[x=2.462319in, y=0.315885in, left, base,rotate=45.000000]{\color{textcolor}{\ifdefined\pdftexversion\else\setmainfont{NanumMyeongjo}\rmfamily\fi\fontsize{9.000000}{10.800000}\selectfont\catcode`\^=\active\def^{\ifmmode\sp\else\^{}\fi}\catcode`\%=\active\def%{\%}2018}}%
\end{pgfscope}%
\begin{pgfscope}%
\pgfsetbuttcap%
\pgfsetroundjoin%
\definecolor{currentfill}{rgb}{0.000000,0.000000,0.000000}%
\pgfsetfillcolor{currentfill}%
\pgfsetlinewidth{0.752812pt}%
\definecolor{currentstroke}{rgb}{0.000000,0.000000,0.000000}%
\pgfsetstrokecolor{currentstroke}%
\pgfsetdash{}{0pt}%
\pgfsys@defobject{currentmarker}{\pgfqpoint{0.000000in}{-0.013889in}}{\pgfqpoint{0.000000in}{0.000000in}}{%
\pgfpathmoveto{\pgfqpoint{0.000000in}{0.000000in}}%
\pgfpathlineto{\pgfqpoint{0.000000in}{-0.013889in}}%
\pgfusepath{stroke,fill}%
}%
\begin{pgfscope}%
\pgfsys@transformshift{2.890625in}{0.638889in}%
\pgfsys@useobject{currentmarker}{}%
\end{pgfscope}%
\end{pgfscope}%
\begin{pgfscope}%
\definecolor{textcolor}{rgb}{0.000000,0.000000,0.000000}%
\pgfsetstrokecolor{textcolor}%
\pgfsetfillcolor{textcolor}%
\pgftext[x=2.820752in, y=0.315885in, left, base,rotate=45.000000]{\color{textcolor}{\ifdefined\pdftexversion\else\setmainfont{NanumMyeongjo}\rmfamily\fi\fontsize{9.000000}{10.800000}\selectfont\catcode`\^=\active\def^{\ifmmode\sp\else\^{}\fi}\catcode`\%=\active\def%{\%}2019}}%
\end{pgfscope}%
\begin{pgfscope}%
\pgfsetbuttcap%
\pgfsetroundjoin%
\definecolor{currentfill}{rgb}{0.000000,0.000000,0.000000}%
\pgfsetfillcolor{currentfill}%
\pgfsetlinewidth{0.752812pt}%
\definecolor{currentstroke}{rgb}{0.000000,0.000000,0.000000}%
\pgfsetstrokecolor{currentstroke}%
\pgfsetdash{}{0pt}%
\pgfsys@defobject{currentmarker}{\pgfqpoint{0.000000in}{-0.013889in}}{\pgfqpoint{0.000000in}{0.000000in}}{%
\pgfpathmoveto{\pgfqpoint{0.000000in}{0.000000in}}%
\pgfpathlineto{\pgfqpoint{0.000000in}{-0.013889in}}%
\pgfusepath{stroke,fill}%
}%
\begin{pgfscope}%
\pgfsys@transformshift{3.249057in}{0.638889in}%
\pgfsys@useobject{currentmarker}{}%
\end{pgfscope}%
\end{pgfscope}%
\begin{pgfscope}%
\definecolor{textcolor}{rgb}{0.000000,0.000000,0.000000}%
\pgfsetstrokecolor{textcolor}%
\pgfsetfillcolor{textcolor}%
\pgftext[x=3.179184in, y=0.315885in, left, base,rotate=45.000000]{\color{textcolor}{\ifdefined\pdftexversion\else\setmainfont{NanumMyeongjo}\rmfamily\fi\fontsize{9.000000}{10.800000}\selectfont\catcode`\^=\active\def^{\ifmmode\sp\else\^{}\fi}\catcode`\%=\active\def%{\%}2020}}%
\end{pgfscope}%
\begin{pgfscope}%
\pgfsetbuttcap%
\pgfsetroundjoin%
\definecolor{currentfill}{rgb}{0.000000,0.000000,0.000000}%
\pgfsetfillcolor{currentfill}%
\pgfsetlinewidth{0.752812pt}%
\definecolor{currentstroke}{rgb}{0.000000,0.000000,0.000000}%
\pgfsetstrokecolor{currentstroke}%
\pgfsetdash{}{0pt}%
\pgfsys@defobject{currentmarker}{\pgfqpoint{0.000000in}{-0.013889in}}{\pgfqpoint{0.000000in}{0.000000in}}{%
\pgfpathmoveto{\pgfqpoint{0.000000in}{0.000000in}}%
\pgfpathlineto{\pgfqpoint{0.000000in}{-0.013889in}}%
\pgfusepath{stroke,fill}%
}%
\begin{pgfscope}%
\pgfsys@transformshift{3.607490in}{0.638889in}%
\pgfsys@useobject{currentmarker}{}%
\end{pgfscope}%
\end{pgfscope}%
\begin{pgfscope}%
\definecolor{textcolor}{rgb}{0.000000,0.000000,0.000000}%
\pgfsetstrokecolor{textcolor}%
\pgfsetfillcolor{textcolor}%
\pgftext[x=3.537617in, y=0.315885in, left, base,rotate=45.000000]{\color{textcolor}{\ifdefined\pdftexversion\else\setmainfont{NanumMyeongjo}\rmfamily\fi\fontsize{9.000000}{10.800000}\selectfont\catcode`\^=\active\def^{\ifmmode\sp\else\^{}\fi}\catcode`\%=\active\def%{\%}2021}}%
\end{pgfscope}%
\begin{pgfscope}%
\pgfsetbuttcap%
\pgfsetroundjoin%
\definecolor{currentfill}{rgb}{0.000000,0.000000,0.000000}%
\pgfsetfillcolor{currentfill}%
\pgfsetlinewidth{0.752812pt}%
\definecolor{currentstroke}{rgb}{0.000000,0.000000,0.000000}%
\pgfsetstrokecolor{currentstroke}%
\pgfsetdash{}{0pt}%
\pgfsys@defobject{currentmarker}{\pgfqpoint{0.000000in}{-0.013889in}}{\pgfqpoint{0.000000in}{0.000000in}}{%
\pgfpathmoveto{\pgfqpoint{0.000000in}{0.000000in}}%
\pgfpathlineto{\pgfqpoint{0.000000in}{-0.013889in}}%
\pgfusepath{stroke,fill}%
}%
\begin{pgfscope}%
\pgfsys@transformshift{3.965922in}{0.638889in}%
\pgfsys@useobject{currentmarker}{}%
\end{pgfscope}%
\end{pgfscope}%
\begin{pgfscope}%
\definecolor{textcolor}{rgb}{0.000000,0.000000,0.000000}%
\pgfsetstrokecolor{textcolor}%
\pgfsetfillcolor{textcolor}%
\pgftext[x=3.896049in, y=0.315885in, left, base,rotate=45.000000]{\color{textcolor}{\ifdefined\pdftexversion\else\setmainfont{NanumMyeongjo}\rmfamily\fi\fontsize{9.000000}{10.800000}\selectfont\catcode`\^=\active\def^{\ifmmode\sp\else\^{}\fi}\catcode`\%=\active\def%{\%}2022}}%
\end{pgfscope}%
\begin{pgfscope}%
\pgfsetbuttcap%
\pgfsetroundjoin%
\definecolor{currentfill}{rgb}{0.000000,0.000000,0.000000}%
\pgfsetfillcolor{currentfill}%
\pgfsetlinewidth{0.752812pt}%
\definecolor{currentstroke}{rgb}{0.000000,0.000000,0.000000}%
\pgfsetstrokecolor{currentstroke}%
\pgfsetdash{}{0pt}%
\pgfsys@defobject{currentmarker}{\pgfqpoint{0.000000in}{-0.013889in}}{\pgfqpoint{0.000000in}{0.000000in}}{%
\pgfpathmoveto{\pgfqpoint{0.000000in}{0.000000in}}%
\pgfpathlineto{\pgfqpoint{0.000000in}{-0.013889in}}%
\pgfusepath{stroke,fill}%
}%
\begin{pgfscope}%
\pgfsys@transformshift{4.324355in}{0.638889in}%
\pgfsys@useobject{currentmarker}{}%
\end{pgfscope}%
\end{pgfscope}%
\begin{pgfscope}%
\definecolor{textcolor}{rgb}{0.000000,0.000000,0.000000}%
\pgfsetstrokecolor{textcolor}%
\pgfsetfillcolor{textcolor}%
\pgftext[x=4.254481in, y=0.315885in, left, base,rotate=45.000000]{\color{textcolor}{\ifdefined\pdftexversion\else\setmainfont{NanumMyeongjo}\rmfamily\fi\fontsize{9.000000}{10.800000}\selectfont\catcode`\^=\active\def^{\ifmmode\sp\else\^{}\fi}\catcode`\%=\active\def%{\%}2023}}%
\end{pgfscope}%
\begin{pgfscope}%
\pgfsetbuttcap%
\pgfsetroundjoin%
\definecolor{currentfill}{rgb}{0.000000,0.000000,0.000000}%
\pgfsetfillcolor{currentfill}%
\pgfsetlinewidth{0.752812pt}%
\definecolor{currentstroke}{rgb}{0.000000,0.000000,0.000000}%
\pgfsetstrokecolor{currentstroke}%
\pgfsetdash{}{0pt}%
\pgfsys@defobject{currentmarker}{\pgfqpoint{0.000000in}{-0.013889in}}{\pgfqpoint{0.000000in}{0.000000in}}{%
\pgfpathmoveto{\pgfqpoint{0.000000in}{0.000000in}}%
\pgfpathlineto{\pgfqpoint{0.000000in}{-0.013889in}}%
\pgfusepath{stroke,fill}%
}%
\begin{pgfscope}%
\pgfsys@transformshift{4.682787in}{0.638889in}%
\pgfsys@useobject{currentmarker}{}%
\end{pgfscope}%
\end{pgfscope}%
\begin{pgfscope}%
\definecolor{textcolor}{rgb}{0.000000,0.000000,0.000000}%
\pgfsetstrokecolor{textcolor}%
\pgfsetfillcolor{textcolor}%
\pgftext[x=4.612914in, y=0.315885in, left, base,rotate=45.000000]{\color{textcolor}{\ifdefined\pdftexversion\else\setmainfont{NanumMyeongjo}\rmfamily\fi\fontsize{9.000000}{10.800000}\selectfont\catcode`\^=\active\def^{\ifmmode\sp\else\^{}\fi}\catcode`\%=\active\def%{\%}2024}}%
\end{pgfscope}%
\begin{pgfscope}%
\pgfpathrectangle{\pgfqpoint{0.781250in}{0.638889in}}{\pgfqpoint{4.218750in}{2.172222in}}%
\pgfusepath{clip}%
\pgfsetbuttcap%
\pgfsetroundjoin%
\pgfsetlinewidth{0.602250pt}%
\definecolor{currentstroke}{rgb}{0.690196,0.690196,0.690196}%
\pgfsetstrokecolor{currentstroke}%
\pgfsetstrokeopacity{0.450000}%
\pgfsetdash{{2.220000pt}{0.960000pt}}{0.000000pt}%
\pgfpathmoveto{\pgfqpoint{0.781250in}{0.638889in}}%
\pgfpathlineto{\pgfqpoint{5.000000in}{0.638889in}}%
\pgfusepath{stroke}%
\end{pgfscope}%
\begin{pgfscope}%
\pgfsetbuttcap%
\pgfsetroundjoin%
\definecolor{currentfill}{rgb}{0.000000,0.000000,0.000000}%
\pgfsetfillcolor{currentfill}%
\pgfsetlinewidth{0.752812pt}%
\definecolor{currentstroke}{rgb}{0.000000,0.000000,0.000000}%
\pgfsetstrokecolor{currentstroke}%
\pgfsetdash{}{0pt}%
\pgfsys@defobject{currentmarker}{\pgfqpoint{-0.013889in}{0.000000in}}{\pgfqpoint{-0.000000in}{0.000000in}}{%
\pgfpathmoveto{\pgfqpoint{-0.000000in}{0.000000in}}%
\pgfpathlineto{\pgfqpoint{-0.013889in}{0.000000in}}%
\pgfusepath{stroke,fill}%
}%
\begin{pgfscope}%
\pgfsys@transformshift{0.781250in}{0.638889in}%
\pgfsys@useobject{currentmarker}{}%
\end{pgfscope}%
\end{pgfscope}%
\begin{pgfscope}%
\definecolor{textcolor}{rgb}{0.000000,0.000000,0.000000}%
\pgfsetstrokecolor{textcolor}%
\pgfsetfillcolor{textcolor}%
\pgftext[x=0.651611in, y=0.588962in, left, base]{\color{textcolor}{\ifdefined\pdftexversion\else\setmainfont{NanumMyeongjo}\rmfamily\fi\fontsize{9.000000}{10.800000}\selectfont\catcode`\^=\active\def^{\ifmmode\sp\else\^{}\fi}\catcode`\%=\active\def%{\%}0}}%
\end{pgfscope}%
\begin{pgfscope}%
\pgfpathrectangle{\pgfqpoint{0.781250in}{0.638889in}}{\pgfqpoint{4.218750in}{2.172222in}}%
\pgfusepath{clip}%
\pgfsetbuttcap%
\pgfsetroundjoin%
\pgfsetlinewidth{0.602250pt}%
\definecolor{currentstroke}{rgb}{0.690196,0.690196,0.690196}%
\pgfsetstrokecolor{currentstroke}%
\pgfsetstrokeopacity{0.450000}%
\pgfsetdash{{2.220000pt}{0.960000pt}}{0.000000pt}%
\pgfpathmoveto{\pgfqpoint{0.781250in}{0.856111in}}%
\pgfpathlineto{\pgfqpoint{5.000000in}{0.856111in}}%
\pgfusepath{stroke}%
\end{pgfscope}%
\begin{pgfscope}%
\pgfsetbuttcap%
\pgfsetroundjoin%
\definecolor{currentfill}{rgb}{0.000000,0.000000,0.000000}%
\pgfsetfillcolor{currentfill}%
\pgfsetlinewidth{0.752812pt}%
\definecolor{currentstroke}{rgb}{0.000000,0.000000,0.000000}%
\pgfsetstrokecolor{currentstroke}%
\pgfsetdash{}{0pt}%
\pgfsys@defobject{currentmarker}{\pgfqpoint{-0.013889in}{0.000000in}}{\pgfqpoint{-0.000000in}{0.000000in}}{%
\pgfpathmoveto{\pgfqpoint{-0.000000in}{0.000000in}}%
\pgfpathlineto{\pgfqpoint{-0.013889in}{0.000000in}}%
\pgfusepath{stroke,fill}%
}%
\begin{pgfscope}%
\pgfsys@transformshift{0.781250in}{0.856111in}%
\pgfsys@useobject{currentmarker}{}%
\end{pgfscope}%
\end{pgfscope}%
\begin{pgfscope}%
\definecolor{textcolor}{rgb}{0.000000,0.000000,0.000000}%
\pgfsetstrokecolor{textcolor}%
\pgfsetfillcolor{textcolor}%
\pgftext[x=0.532837in, y=0.806184in, left, base]{\color{textcolor}{\ifdefined\pdftexversion\else\setmainfont{NanumMyeongjo}\rmfamily\fi\fontsize{9.000000}{10.800000}\selectfont\catcode`\^=\active\def^{\ifmmode\sp\else\^{}\fi}\catcode`\%=\active\def%{\%}1만}}%
\end{pgfscope}%
\begin{pgfscope}%
\pgfpathrectangle{\pgfqpoint{0.781250in}{0.638889in}}{\pgfqpoint{4.218750in}{2.172222in}}%
\pgfusepath{clip}%
\pgfsetbuttcap%
\pgfsetroundjoin%
\pgfsetlinewidth{0.602250pt}%
\definecolor{currentstroke}{rgb}{0.690196,0.690196,0.690196}%
\pgfsetstrokecolor{currentstroke}%
\pgfsetstrokeopacity{0.450000}%
\pgfsetdash{{2.220000pt}{0.960000pt}}{0.000000pt}%
\pgfpathmoveto{\pgfqpoint{0.781250in}{1.073333in}}%
\pgfpathlineto{\pgfqpoint{5.000000in}{1.073333in}}%
\pgfusepath{stroke}%
\end{pgfscope}%
\begin{pgfscope}%
\pgfsetbuttcap%
\pgfsetroundjoin%
\definecolor{currentfill}{rgb}{0.000000,0.000000,0.000000}%
\pgfsetfillcolor{currentfill}%
\pgfsetlinewidth{0.752812pt}%
\definecolor{currentstroke}{rgb}{0.000000,0.000000,0.000000}%
\pgfsetstrokecolor{currentstroke}%
\pgfsetdash{}{0pt}%
\pgfsys@defobject{currentmarker}{\pgfqpoint{-0.013889in}{0.000000in}}{\pgfqpoint{-0.000000in}{0.000000in}}{%
\pgfpathmoveto{\pgfqpoint{-0.000000in}{0.000000in}}%
\pgfpathlineto{\pgfqpoint{-0.013889in}{0.000000in}}%
\pgfusepath{stroke,fill}%
}%
\begin{pgfscope}%
\pgfsys@transformshift{0.781250in}{1.073333in}%
\pgfsys@useobject{currentmarker}{}%
\end{pgfscope}%
\end{pgfscope}%
\begin{pgfscope}%
\definecolor{textcolor}{rgb}{0.000000,0.000000,0.000000}%
\pgfsetstrokecolor{textcolor}%
\pgfsetfillcolor{textcolor}%
\pgftext[x=0.532837in, y=1.023407in, left, base]{\color{textcolor}{\ifdefined\pdftexversion\else\setmainfont{NanumMyeongjo}\rmfamily\fi\fontsize{9.000000}{10.800000}\selectfont\catcode`\^=\active\def^{\ifmmode\sp\else\^{}\fi}\catcode`\%=\active\def%{\%}2만}}%
\end{pgfscope}%
\begin{pgfscope}%
\pgfpathrectangle{\pgfqpoint{0.781250in}{0.638889in}}{\pgfqpoint{4.218750in}{2.172222in}}%
\pgfusepath{clip}%
\pgfsetbuttcap%
\pgfsetroundjoin%
\pgfsetlinewidth{0.602250pt}%
\definecolor{currentstroke}{rgb}{0.690196,0.690196,0.690196}%
\pgfsetstrokecolor{currentstroke}%
\pgfsetstrokeopacity{0.450000}%
\pgfsetdash{{2.220000pt}{0.960000pt}}{0.000000pt}%
\pgfpathmoveto{\pgfqpoint{0.781250in}{1.290556in}}%
\pgfpathlineto{\pgfqpoint{5.000000in}{1.290556in}}%
\pgfusepath{stroke}%
\end{pgfscope}%
\begin{pgfscope}%
\pgfsetbuttcap%
\pgfsetroundjoin%
\definecolor{currentfill}{rgb}{0.000000,0.000000,0.000000}%
\pgfsetfillcolor{currentfill}%
\pgfsetlinewidth{0.752812pt}%
\definecolor{currentstroke}{rgb}{0.000000,0.000000,0.000000}%
\pgfsetstrokecolor{currentstroke}%
\pgfsetdash{}{0pt}%
\pgfsys@defobject{currentmarker}{\pgfqpoint{-0.013889in}{0.000000in}}{\pgfqpoint{-0.000000in}{0.000000in}}{%
\pgfpathmoveto{\pgfqpoint{-0.000000in}{0.000000in}}%
\pgfpathlineto{\pgfqpoint{-0.013889in}{0.000000in}}%
\pgfusepath{stroke,fill}%
}%
\begin{pgfscope}%
\pgfsys@transformshift{0.781250in}{1.290556in}%
\pgfsys@useobject{currentmarker}{}%
\end{pgfscope}%
\end{pgfscope}%
\begin{pgfscope}%
\definecolor{textcolor}{rgb}{0.000000,0.000000,0.000000}%
\pgfsetstrokecolor{textcolor}%
\pgfsetfillcolor{textcolor}%
\pgftext[x=0.532837in, y=1.240629in, left, base]{\color{textcolor}{\ifdefined\pdftexversion\else\setmainfont{NanumMyeongjo}\rmfamily\fi\fontsize{9.000000}{10.800000}\selectfont\catcode`\^=\active\def^{\ifmmode\sp\else\^{}\fi}\catcode`\%=\active\def%{\%}3만}}%
\end{pgfscope}%
\begin{pgfscope}%
\pgfpathrectangle{\pgfqpoint{0.781250in}{0.638889in}}{\pgfqpoint{4.218750in}{2.172222in}}%
\pgfusepath{clip}%
\pgfsetbuttcap%
\pgfsetroundjoin%
\pgfsetlinewidth{0.602250pt}%
\definecolor{currentstroke}{rgb}{0.690196,0.690196,0.690196}%
\pgfsetstrokecolor{currentstroke}%
\pgfsetstrokeopacity{0.450000}%
\pgfsetdash{{2.220000pt}{0.960000pt}}{0.000000pt}%
\pgfpathmoveto{\pgfqpoint{0.781250in}{1.507778in}}%
\pgfpathlineto{\pgfqpoint{5.000000in}{1.507778in}}%
\pgfusepath{stroke}%
\end{pgfscope}%
\begin{pgfscope}%
\pgfsetbuttcap%
\pgfsetroundjoin%
\definecolor{currentfill}{rgb}{0.000000,0.000000,0.000000}%
\pgfsetfillcolor{currentfill}%
\pgfsetlinewidth{0.752812pt}%
\definecolor{currentstroke}{rgb}{0.000000,0.000000,0.000000}%
\pgfsetstrokecolor{currentstroke}%
\pgfsetdash{}{0pt}%
\pgfsys@defobject{currentmarker}{\pgfqpoint{-0.013889in}{0.000000in}}{\pgfqpoint{-0.000000in}{0.000000in}}{%
\pgfpathmoveto{\pgfqpoint{-0.000000in}{0.000000in}}%
\pgfpathlineto{\pgfqpoint{-0.013889in}{0.000000in}}%
\pgfusepath{stroke,fill}%
}%
\begin{pgfscope}%
\pgfsys@transformshift{0.781250in}{1.507778in}%
\pgfsys@useobject{currentmarker}{}%
\end{pgfscope}%
\end{pgfscope}%
\begin{pgfscope}%
\definecolor{textcolor}{rgb}{0.000000,0.000000,0.000000}%
\pgfsetstrokecolor{textcolor}%
\pgfsetfillcolor{textcolor}%
\pgftext[x=0.532837in, y=1.457851in, left, base]{\color{textcolor}{\ifdefined\pdftexversion\else\setmainfont{NanumMyeongjo}\rmfamily\fi\fontsize{9.000000}{10.800000}\selectfont\catcode`\^=\active\def^{\ifmmode\sp\else\^{}\fi}\catcode`\%=\active\def%{\%}4만}}%
\end{pgfscope}%
\begin{pgfscope}%
\pgfpathrectangle{\pgfqpoint{0.781250in}{0.638889in}}{\pgfqpoint{4.218750in}{2.172222in}}%
\pgfusepath{clip}%
\pgfsetbuttcap%
\pgfsetroundjoin%
\pgfsetlinewidth{0.602250pt}%
\definecolor{currentstroke}{rgb}{0.690196,0.690196,0.690196}%
\pgfsetstrokecolor{currentstroke}%
\pgfsetstrokeopacity{0.450000}%
\pgfsetdash{{2.220000pt}{0.960000pt}}{0.000000pt}%
\pgfpathmoveto{\pgfqpoint{0.781250in}{1.725000in}}%
\pgfpathlineto{\pgfqpoint{5.000000in}{1.725000in}}%
\pgfusepath{stroke}%
\end{pgfscope}%
\begin{pgfscope}%
\pgfsetbuttcap%
\pgfsetroundjoin%
\definecolor{currentfill}{rgb}{0.000000,0.000000,0.000000}%
\pgfsetfillcolor{currentfill}%
\pgfsetlinewidth{0.752812pt}%
\definecolor{currentstroke}{rgb}{0.000000,0.000000,0.000000}%
\pgfsetstrokecolor{currentstroke}%
\pgfsetdash{}{0pt}%
\pgfsys@defobject{currentmarker}{\pgfqpoint{-0.013889in}{0.000000in}}{\pgfqpoint{-0.000000in}{0.000000in}}{%
\pgfpathmoveto{\pgfqpoint{-0.000000in}{0.000000in}}%
\pgfpathlineto{\pgfqpoint{-0.013889in}{0.000000in}}%
\pgfusepath{stroke,fill}%
}%
\begin{pgfscope}%
\pgfsys@transformshift{0.781250in}{1.725000in}%
\pgfsys@useobject{currentmarker}{}%
\end{pgfscope}%
\end{pgfscope}%
\begin{pgfscope}%
\definecolor{textcolor}{rgb}{0.000000,0.000000,0.000000}%
\pgfsetstrokecolor{textcolor}%
\pgfsetfillcolor{textcolor}%
\pgftext[x=0.532837in, y=1.675073in, left, base]{\color{textcolor}{\ifdefined\pdftexversion\else\setmainfont{NanumMyeongjo}\rmfamily\fi\fontsize{9.000000}{10.800000}\selectfont\catcode`\^=\active\def^{\ifmmode\sp\else\^{}\fi}\catcode`\%=\active\def%{\%}5만}}%
\end{pgfscope}%
\begin{pgfscope}%
\pgfpathrectangle{\pgfqpoint{0.781250in}{0.638889in}}{\pgfqpoint{4.218750in}{2.172222in}}%
\pgfusepath{clip}%
\pgfsetbuttcap%
\pgfsetroundjoin%
\pgfsetlinewidth{0.602250pt}%
\definecolor{currentstroke}{rgb}{0.690196,0.690196,0.690196}%
\pgfsetstrokecolor{currentstroke}%
\pgfsetstrokeopacity{0.450000}%
\pgfsetdash{{2.220000pt}{0.960000pt}}{0.000000pt}%
\pgfpathmoveto{\pgfqpoint{0.781250in}{1.942222in}}%
\pgfpathlineto{\pgfqpoint{5.000000in}{1.942222in}}%
\pgfusepath{stroke}%
\end{pgfscope}%
\begin{pgfscope}%
\pgfsetbuttcap%
\pgfsetroundjoin%
\definecolor{currentfill}{rgb}{0.000000,0.000000,0.000000}%
\pgfsetfillcolor{currentfill}%
\pgfsetlinewidth{0.752812pt}%
\definecolor{currentstroke}{rgb}{0.000000,0.000000,0.000000}%
\pgfsetstrokecolor{currentstroke}%
\pgfsetdash{}{0pt}%
\pgfsys@defobject{currentmarker}{\pgfqpoint{-0.013889in}{0.000000in}}{\pgfqpoint{-0.000000in}{0.000000in}}{%
\pgfpathmoveto{\pgfqpoint{-0.000000in}{0.000000in}}%
\pgfpathlineto{\pgfqpoint{-0.013889in}{0.000000in}}%
\pgfusepath{stroke,fill}%
}%
\begin{pgfscope}%
\pgfsys@transformshift{0.781250in}{1.942222in}%
\pgfsys@useobject{currentmarker}{}%
\end{pgfscope}%
\end{pgfscope}%
\begin{pgfscope}%
\definecolor{textcolor}{rgb}{0.000000,0.000000,0.000000}%
\pgfsetstrokecolor{textcolor}%
\pgfsetfillcolor{textcolor}%
\pgftext[x=0.532837in, y=1.892295in, left, base]{\color{textcolor}{\ifdefined\pdftexversion\else\setmainfont{NanumMyeongjo}\rmfamily\fi\fontsize{9.000000}{10.800000}\selectfont\catcode`\^=\active\def^{\ifmmode\sp\else\^{}\fi}\catcode`\%=\active\def%{\%}6만}}%
\end{pgfscope}%
\begin{pgfscope}%
\pgfpathrectangle{\pgfqpoint{0.781250in}{0.638889in}}{\pgfqpoint{4.218750in}{2.172222in}}%
\pgfusepath{clip}%
\pgfsetbuttcap%
\pgfsetroundjoin%
\pgfsetlinewidth{0.602250pt}%
\definecolor{currentstroke}{rgb}{0.690196,0.690196,0.690196}%
\pgfsetstrokecolor{currentstroke}%
\pgfsetstrokeopacity{0.450000}%
\pgfsetdash{{2.220000pt}{0.960000pt}}{0.000000pt}%
\pgfpathmoveto{\pgfqpoint{0.781250in}{2.159444in}}%
\pgfpathlineto{\pgfqpoint{5.000000in}{2.159444in}}%
\pgfusepath{stroke}%
\end{pgfscope}%
\begin{pgfscope}%
\pgfsetbuttcap%
\pgfsetroundjoin%
\definecolor{currentfill}{rgb}{0.000000,0.000000,0.000000}%
\pgfsetfillcolor{currentfill}%
\pgfsetlinewidth{0.752812pt}%
\definecolor{currentstroke}{rgb}{0.000000,0.000000,0.000000}%
\pgfsetstrokecolor{currentstroke}%
\pgfsetdash{}{0pt}%
\pgfsys@defobject{currentmarker}{\pgfqpoint{-0.013889in}{0.000000in}}{\pgfqpoint{-0.000000in}{0.000000in}}{%
\pgfpathmoveto{\pgfqpoint{-0.000000in}{0.000000in}}%
\pgfpathlineto{\pgfqpoint{-0.013889in}{0.000000in}}%
\pgfusepath{stroke,fill}%
}%
\begin{pgfscope}%
\pgfsys@transformshift{0.781250in}{2.159444in}%
\pgfsys@useobject{currentmarker}{}%
\end{pgfscope}%
\end{pgfscope}%
\begin{pgfscope}%
\definecolor{textcolor}{rgb}{0.000000,0.000000,0.000000}%
\pgfsetstrokecolor{textcolor}%
\pgfsetfillcolor{textcolor}%
\pgftext[x=0.532837in, y=2.109518in, left, base]{\color{textcolor}{\ifdefined\pdftexversion\else\setmainfont{NanumMyeongjo}\rmfamily\fi\fontsize{9.000000}{10.800000}\selectfont\catcode`\^=\active\def^{\ifmmode\sp\else\^{}\fi}\catcode`\%=\active\def%{\%}7만}}%
\end{pgfscope}%
\begin{pgfscope}%
\pgfpathrectangle{\pgfqpoint{0.781250in}{0.638889in}}{\pgfqpoint{4.218750in}{2.172222in}}%
\pgfusepath{clip}%
\pgfsetbuttcap%
\pgfsetroundjoin%
\pgfsetlinewidth{0.602250pt}%
\definecolor{currentstroke}{rgb}{0.690196,0.690196,0.690196}%
\pgfsetstrokecolor{currentstroke}%
\pgfsetstrokeopacity{0.450000}%
\pgfsetdash{{2.220000pt}{0.960000pt}}{0.000000pt}%
\pgfpathmoveto{\pgfqpoint{0.781250in}{2.376667in}}%
\pgfpathlineto{\pgfqpoint{5.000000in}{2.376667in}}%
\pgfusepath{stroke}%
\end{pgfscope}%
\begin{pgfscope}%
\pgfsetbuttcap%
\pgfsetroundjoin%
\definecolor{currentfill}{rgb}{0.000000,0.000000,0.000000}%
\pgfsetfillcolor{currentfill}%
\pgfsetlinewidth{0.752812pt}%
\definecolor{currentstroke}{rgb}{0.000000,0.000000,0.000000}%
\pgfsetstrokecolor{currentstroke}%
\pgfsetdash{}{0pt}%
\pgfsys@defobject{currentmarker}{\pgfqpoint{-0.013889in}{0.000000in}}{\pgfqpoint{-0.000000in}{0.000000in}}{%
\pgfpathmoveto{\pgfqpoint{-0.000000in}{0.000000in}}%
\pgfpathlineto{\pgfqpoint{-0.013889in}{0.000000in}}%
\pgfusepath{stroke,fill}%
}%
\begin{pgfscope}%
\pgfsys@transformshift{0.781250in}{2.376667in}%
\pgfsys@useobject{currentmarker}{}%
\end{pgfscope}%
\end{pgfscope}%
\begin{pgfscope}%
\definecolor{textcolor}{rgb}{0.000000,0.000000,0.000000}%
\pgfsetstrokecolor{textcolor}%
\pgfsetfillcolor{textcolor}%
\pgftext[x=0.532837in, y=2.326740in, left, base]{\color{textcolor}{\ifdefined\pdftexversion\else\setmainfont{NanumMyeongjo}\rmfamily\fi\fontsize{9.000000}{10.800000}\selectfont\catcode`\^=\active\def^{\ifmmode\sp\else\^{}\fi}\catcode`\%=\active\def%{\%}8만}}%
\end{pgfscope}%
\begin{pgfscope}%
\pgfpathrectangle{\pgfqpoint{0.781250in}{0.638889in}}{\pgfqpoint{4.218750in}{2.172222in}}%
\pgfusepath{clip}%
\pgfsetbuttcap%
\pgfsetroundjoin%
\pgfsetlinewidth{0.602250pt}%
\definecolor{currentstroke}{rgb}{0.690196,0.690196,0.690196}%
\pgfsetstrokecolor{currentstroke}%
\pgfsetstrokeopacity{0.450000}%
\pgfsetdash{{2.220000pt}{0.960000pt}}{0.000000pt}%
\pgfpathmoveto{\pgfqpoint{0.781250in}{2.593889in}}%
\pgfpathlineto{\pgfqpoint{5.000000in}{2.593889in}}%
\pgfusepath{stroke}%
\end{pgfscope}%
\begin{pgfscope}%
\pgfsetbuttcap%
\pgfsetroundjoin%
\definecolor{currentfill}{rgb}{0.000000,0.000000,0.000000}%
\pgfsetfillcolor{currentfill}%
\pgfsetlinewidth{0.752812pt}%
\definecolor{currentstroke}{rgb}{0.000000,0.000000,0.000000}%
\pgfsetstrokecolor{currentstroke}%
\pgfsetdash{}{0pt}%
\pgfsys@defobject{currentmarker}{\pgfqpoint{-0.013889in}{0.000000in}}{\pgfqpoint{-0.000000in}{0.000000in}}{%
\pgfpathmoveto{\pgfqpoint{-0.000000in}{0.000000in}}%
\pgfpathlineto{\pgfqpoint{-0.013889in}{0.000000in}}%
\pgfusepath{stroke,fill}%
}%
\begin{pgfscope}%
\pgfsys@transformshift{0.781250in}{2.593889in}%
\pgfsys@useobject{currentmarker}{}%
\end{pgfscope}%
\end{pgfscope}%
\begin{pgfscope}%
\definecolor{textcolor}{rgb}{0.000000,0.000000,0.000000}%
\pgfsetstrokecolor{textcolor}%
\pgfsetfillcolor{textcolor}%
\pgftext[x=0.532837in, y=2.543962in, left, base]{\color{textcolor}{\ifdefined\pdftexversion\else\setmainfont{NanumMyeongjo}\rmfamily\fi\fontsize{9.000000}{10.800000}\selectfont\catcode`\^=\active\def^{\ifmmode\sp\else\^{}\fi}\catcode`\%=\active\def%{\%}9만}}%
\end{pgfscope}%
\begin{pgfscope}%
\pgfpathrectangle{\pgfqpoint{0.781250in}{0.638889in}}{\pgfqpoint{4.218750in}{2.172222in}}%
\pgfusepath{clip}%
\pgfsetbuttcap%
\pgfsetroundjoin%
\pgfsetlinewidth{0.602250pt}%
\definecolor{currentstroke}{rgb}{0.690196,0.690196,0.690196}%
\pgfsetstrokecolor{currentstroke}%
\pgfsetstrokeopacity{0.450000}%
\pgfsetdash{{2.220000pt}{0.960000pt}}{0.000000pt}%
\pgfpathmoveto{\pgfqpoint{0.781250in}{2.811111in}}%
\pgfpathlineto{\pgfqpoint{5.000000in}{2.811111in}}%
\pgfusepath{stroke}%
\end{pgfscope}%
\begin{pgfscope}%
\pgfsetbuttcap%
\pgfsetroundjoin%
\definecolor{currentfill}{rgb}{0.000000,0.000000,0.000000}%
\pgfsetfillcolor{currentfill}%
\pgfsetlinewidth{0.752812pt}%
\definecolor{currentstroke}{rgb}{0.000000,0.000000,0.000000}%
\pgfsetstrokecolor{currentstroke}%
\pgfsetdash{}{0pt}%
\pgfsys@defobject{currentmarker}{\pgfqpoint{-0.013889in}{0.000000in}}{\pgfqpoint{-0.000000in}{0.000000in}}{%
\pgfpathmoveto{\pgfqpoint{-0.000000in}{0.000000in}}%
\pgfpathlineto{\pgfqpoint{-0.013889in}{0.000000in}}%
\pgfusepath{stroke,fill}%
}%
\begin{pgfscope}%
\pgfsys@transformshift{0.781250in}{2.811111in}%
\pgfsys@useobject{currentmarker}{}%
\end{pgfscope}%
\end{pgfscope}%
\begin{pgfscope}%
\definecolor{textcolor}{rgb}{0.000000,0.000000,0.000000}%
\pgfsetstrokecolor{textcolor}%
\pgfsetfillcolor{textcolor}%
\pgftext[x=0.465698in, y=2.761184in, left, base]{\color{textcolor}{\ifdefined\pdftexversion\else\setmainfont{NanumMyeongjo}\rmfamily\fi\fontsize{9.000000}{10.800000}\selectfont\catcode`\^=\active\def^{\ifmmode\sp\else\^{}\fi}\catcode`\%=\active\def%{\%}10만}}%
\end{pgfscope}%
\begin{pgfscope}%
\pgfsetrectcap%
\pgfsetmiterjoin%
\pgfsetlinewidth{0.752812pt}%
\definecolor{currentstroke}{rgb}{0.000000,0.000000,0.000000}%
\pgfsetstrokecolor{currentstroke}%
\pgfsetdash{}{0pt}%
\pgfpathmoveto{\pgfqpoint{0.781250in}{0.638889in}}%
\pgfpathlineto{\pgfqpoint{0.781250in}{2.811111in}}%
\pgfusepath{stroke}%
\end{pgfscope}%
\begin{pgfscope}%
\pgfsetrectcap%
\pgfsetmiterjoin%
\pgfsetlinewidth{0.752812pt}%
\definecolor{currentstroke}{rgb}{0.000000,0.000000,0.000000}%
\pgfsetstrokecolor{currentstroke}%
\pgfsetdash{}{0pt}%
\pgfpathmoveto{\pgfqpoint{0.781250in}{0.638889in}}%
\pgfpathlineto{\pgfqpoint{5.000000in}{0.638889in}}%
\pgfusepath{stroke}%
\end{pgfscope}%
\begin{pgfscope}%
\pgfpathrectangle{\pgfqpoint{0.781250in}{0.638889in}}{\pgfqpoint{4.218750in}{2.172222in}}%
\pgfusepath{clip}%
\pgfsetbuttcap%
\pgfsetmiterjoin%
\definecolor{currentfill}{rgb}{0.227451,0.192157,0.427451}%
\pgfsetfillcolor{currentfill}%
\pgfsetlinewidth{1.003750pt}%
\definecolor{currentstroke}{rgb}{0.266667,0.266667,0.266667}%
\pgfsetstrokecolor{currentstroke}%
\pgfsetdash{}{0pt}%
\pgfpathmoveto{\pgfqpoint{0.973011in}{0.638889in}}%
\pgfpathlineto{\pgfqpoint{1.223914in}{0.638889in}}%
\pgfpathlineto{\pgfqpoint{1.223914in}{2.575773in}}%
\pgfpathlineto{\pgfqpoint{0.973011in}{2.575773in}}%
\pgfpathlineto{\pgfqpoint{0.973011in}{0.638889in}}%
\pgfpathclose%
\pgfusepath{stroke,fill}%
\end{pgfscope}%
\begin{pgfscope}%
\pgfpathrectangle{\pgfqpoint{0.781250in}{0.638889in}}{\pgfqpoint{4.218750in}{2.172222in}}%
\pgfusepath{clip}%
\pgfsetbuttcap%
\pgfsetmiterjoin%
\definecolor{currentfill}{rgb}{0.227451,0.192157,0.427451}%
\pgfsetfillcolor{currentfill}%
\pgfsetlinewidth{1.003750pt}%
\definecolor{currentstroke}{rgb}{0.266667,0.266667,0.266667}%
\pgfsetstrokecolor{currentstroke}%
\pgfsetdash{}{0pt}%
\pgfpathmoveto{\pgfqpoint{1.331444in}{0.638889in}}%
\pgfpathlineto{\pgfqpoint{1.582347in}{0.638889in}}%
\pgfpathlineto{\pgfqpoint{1.582347in}{2.142653in}}%
\pgfpathlineto{\pgfqpoint{1.331444in}{2.142653in}}%
\pgfpathlineto{\pgfqpoint{1.331444in}{0.638889in}}%
\pgfpathclose%
\pgfusepath{stroke,fill}%
\end{pgfscope}%
\begin{pgfscope}%
\pgfpathrectangle{\pgfqpoint{0.781250in}{0.638889in}}{\pgfqpoint{4.218750in}{2.172222in}}%
\pgfusepath{clip}%
\pgfsetbuttcap%
\pgfsetmiterjoin%
\definecolor{currentfill}{rgb}{0.227451,0.192157,0.427451}%
\pgfsetfillcolor{currentfill}%
\pgfsetlinewidth{1.003750pt}%
\definecolor{currentstroke}{rgb}{0.266667,0.266667,0.266667}%
\pgfsetstrokecolor{currentstroke}%
\pgfsetdash{}{0pt}%
\pgfpathmoveto{\pgfqpoint{1.689876in}{0.638889in}}%
\pgfpathlineto{\pgfqpoint{1.940779in}{0.638889in}}%
\pgfpathlineto{\pgfqpoint{1.940779in}{1.966073in}}%
\pgfpathlineto{\pgfqpoint{1.689876in}{1.966073in}}%
\pgfpathlineto{\pgfqpoint{1.689876in}{0.638889in}}%
\pgfpathclose%
\pgfusepath{stroke,fill}%
\end{pgfscope}%
\begin{pgfscope}%
\pgfpathrectangle{\pgfqpoint{0.781250in}{0.638889in}}{\pgfqpoint{4.218750in}{2.172222in}}%
\pgfusepath{clip}%
\pgfsetbuttcap%
\pgfsetmiterjoin%
\definecolor{currentfill}{rgb}{0.227451,0.192157,0.427451}%
\pgfsetfillcolor{currentfill}%
\pgfsetlinewidth{1.003750pt}%
\definecolor{currentstroke}{rgb}{0.266667,0.266667,0.266667}%
\pgfsetstrokecolor{currentstroke}%
\pgfsetdash{}{0pt}%
\pgfpathmoveto{\pgfqpoint{2.048309in}{0.638889in}}%
\pgfpathlineto{\pgfqpoint{2.299211in}{0.638889in}}%
\pgfpathlineto{\pgfqpoint{2.299211in}{1.899734in}}%
\pgfpathlineto{\pgfqpoint{2.048309in}{1.899734in}}%
\pgfpathlineto{\pgfqpoint{2.048309in}{0.638889in}}%
\pgfpathclose%
\pgfusepath{stroke,fill}%
\end{pgfscope}%
\begin{pgfscope}%
\pgfpathrectangle{\pgfqpoint{0.781250in}{0.638889in}}{\pgfqpoint{4.218750in}{2.172222in}}%
\pgfusepath{clip}%
\pgfsetbuttcap%
\pgfsetmiterjoin%
\definecolor{currentfill}{rgb}{0.227451,0.192157,0.427451}%
\pgfsetfillcolor{currentfill}%
\pgfsetlinewidth{1.003750pt}%
\definecolor{currentstroke}{rgb}{0.266667,0.266667,0.266667}%
\pgfsetstrokecolor{currentstroke}%
\pgfsetdash{}{0pt}%
\pgfpathmoveto{\pgfqpoint{2.406741in}{0.638889in}}%
\pgfpathlineto{\pgfqpoint{2.657644in}{0.638889in}}%
\pgfpathlineto{\pgfqpoint{2.657644in}{2.003609in}}%
\pgfpathlineto{\pgfqpoint{2.406741in}{2.003609in}}%
\pgfpathlineto{\pgfqpoint{2.406741in}{0.638889in}}%
\pgfpathclose%
\pgfusepath{stroke,fill}%
\end{pgfscope}%
\begin{pgfscope}%
\pgfpathrectangle{\pgfqpoint{0.781250in}{0.638889in}}{\pgfqpoint{4.218750in}{2.172222in}}%
\pgfusepath{clip}%
\pgfsetbuttcap%
\pgfsetmiterjoin%
\definecolor{currentfill}{rgb}{0.227451,0.192157,0.427451}%
\pgfsetfillcolor{currentfill}%
\pgfsetlinewidth{1.003750pt}%
\definecolor{currentstroke}{rgb}{0.266667,0.266667,0.266667}%
\pgfsetstrokecolor{currentstroke}%
\pgfsetdash{}{0pt}%
\pgfpathmoveto{\pgfqpoint{2.765174in}{0.638889in}}%
\pgfpathlineto{\pgfqpoint{3.016076in}{0.638889in}}%
\pgfpathlineto{\pgfqpoint{3.016076in}{2.195916in}}%
\pgfpathlineto{\pgfqpoint{2.765174in}{2.195916in}}%
\pgfpathlineto{\pgfqpoint{2.765174in}{0.638889in}}%
\pgfpathclose%
\pgfusepath{stroke,fill}%
\end{pgfscope}%
\begin{pgfscope}%
\pgfpathrectangle{\pgfqpoint{0.781250in}{0.638889in}}{\pgfqpoint{4.218750in}{2.172222in}}%
\pgfusepath{clip}%
\pgfsetbuttcap%
\pgfsetmiterjoin%
\definecolor{currentfill}{rgb}{0.227451,0.192157,0.427451}%
\pgfsetfillcolor{currentfill}%
\pgfsetlinewidth{1.003750pt}%
\definecolor{currentstroke}{rgb}{0.266667,0.266667,0.266667}%
\pgfsetstrokecolor{currentstroke}%
\pgfsetdash{}{0pt}%
\pgfpathmoveto{\pgfqpoint{3.123606in}{0.638889in}}%
\pgfpathlineto{\pgfqpoint{3.374509in}{0.638889in}}%
\pgfpathlineto{\pgfqpoint{3.374509in}{2.088521in}}%
\pgfpathlineto{\pgfqpoint{3.123606in}{2.088521in}}%
\pgfpathlineto{\pgfqpoint{3.123606in}{0.638889in}}%
\pgfpathclose%
\pgfusepath{stroke,fill}%
\end{pgfscope}%
\begin{pgfscope}%
\pgfpathrectangle{\pgfqpoint{0.781250in}{0.638889in}}{\pgfqpoint{4.218750in}{2.172222in}}%
\pgfusepath{clip}%
\pgfsetbuttcap%
\pgfsetmiterjoin%
\definecolor{currentfill}{rgb}{0.227451,0.192157,0.427451}%
\pgfsetfillcolor{currentfill}%
\pgfsetlinewidth{1.003750pt}%
\definecolor{currentstroke}{rgb}{0.266667,0.266667,0.266667}%
\pgfsetstrokecolor{currentstroke}%
\pgfsetdash{}{0pt}%
\pgfpathmoveto{\pgfqpoint{3.482039in}{0.638889in}}%
\pgfpathlineto{\pgfqpoint{3.732941in}{0.638889in}}%
\pgfpathlineto{\pgfqpoint{3.732941in}{2.056633in}}%
\pgfpathlineto{\pgfqpoint{3.482039in}{2.056633in}}%
\pgfpathlineto{\pgfqpoint{3.482039in}{0.638889in}}%
\pgfpathclose%
\pgfusepath{stroke,fill}%
\end{pgfscope}%
\begin{pgfscope}%
\pgfpathrectangle{\pgfqpoint{0.781250in}{0.638889in}}{\pgfqpoint{4.218750in}{2.172222in}}%
\pgfusepath{clip}%
\pgfsetbuttcap%
\pgfsetmiterjoin%
\definecolor{currentfill}{rgb}{0.227451,0.192157,0.427451}%
\pgfsetfillcolor{currentfill}%
\pgfsetlinewidth{1.003750pt}%
\definecolor{currentstroke}{rgb}{0.266667,0.266667,0.266667}%
\pgfsetstrokecolor{currentstroke}%
\pgfsetdash{}{0pt}%
\pgfpathmoveto{\pgfqpoint{3.840471in}{0.638889in}}%
\pgfpathlineto{\pgfqpoint{4.091374in}{0.638889in}}%
\pgfpathlineto{\pgfqpoint{4.091374in}{2.282175in}}%
\pgfpathlineto{\pgfqpoint{3.840471in}{2.282175in}}%
\pgfpathlineto{\pgfqpoint{3.840471in}{0.638889in}}%
\pgfpathclose%
\pgfusepath{stroke,fill}%
\end{pgfscope}%
\begin{pgfscope}%
\pgfpathrectangle{\pgfqpoint{0.781250in}{0.638889in}}{\pgfqpoint{4.218750in}{2.172222in}}%
\pgfusepath{clip}%
\pgfsetbuttcap%
\pgfsetmiterjoin%
\definecolor{currentfill}{rgb}{0.227451,0.192157,0.427451}%
\pgfsetfillcolor{currentfill}%
\pgfsetlinewidth{1.003750pt}%
\definecolor{currentstroke}{rgb}{0.266667,0.266667,0.266667}%
\pgfsetstrokecolor{currentstroke}%
\pgfsetdash{}{0pt}%
\pgfpathmoveto{\pgfqpoint{4.198903in}{0.638889in}}%
\pgfpathlineto{\pgfqpoint{4.449806in}{0.638889in}}%
\pgfpathlineto{\pgfqpoint{4.449806in}{2.323230in}}%
\pgfpathlineto{\pgfqpoint{4.198903in}{2.323230in}}%
\pgfpathlineto{\pgfqpoint{4.198903in}{0.638889in}}%
\pgfpathclose%
\pgfusepath{stroke,fill}%
\end{pgfscope}%
\begin{pgfscope}%
\pgfpathrectangle{\pgfqpoint{0.781250in}{0.638889in}}{\pgfqpoint{4.218750in}{2.172222in}}%
\pgfusepath{clip}%
\pgfsetbuttcap%
\pgfsetmiterjoin%
\definecolor{currentfill}{rgb}{0.227451,0.192157,0.427451}%
\pgfsetfillcolor{currentfill}%
\pgfsetlinewidth{1.003750pt}%
\definecolor{currentstroke}{rgb}{0.266667,0.266667,0.266667}%
\pgfsetstrokecolor{currentstroke}%
\pgfsetdash{}{0pt}%
\pgfpathmoveto{\pgfqpoint{4.557336in}{0.638889in}}%
\pgfpathlineto{\pgfqpoint{4.808239in}{0.638889in}}%
\pgfpathlineto{\pgfqpoint{4.808239in}{2.479826in}}%
\pgfpathlineto{\pgfqpoint{4.557336in}{2.479826in}}%
\pgfpathlineto{\pgfqpoint{4.557336in}{0.638889in}}%
\pgfpathclose%
\pgfusepath{stroke,fill}%
\end{pgfscope}%
\begin{pgfscope}%
\definecolor{textcolor}{rgb}{0.000000,0.000000,0.000000}%
\pgfsetstrokecolor{textcolor}%
\pgfsetfillcolor{textcolor}%
\pgftext[x=1.098463in,y=2.603550in,,bottom]{\color{textcolor}{\ifdefined\pdftexversion\else\setmainfont{NanumMyeongjo}\rmfamily\fi\fontsize{5.000000}{6.000000}\bfseries\selectfont\catcode`\^=\active\def^{\ifmmode\sp\else\^{}\fi}\catcode`\%=\active\def%{\%}89,166}}%
\end{pgfscope}%
\begin{pgfscope}%
\definecolor{textcolor}{rgb}{0.000000,0.000000,0.000000}%
\pgfsetstrokecolor{textcolor}%
\pgfsetfillcolor{textcolor}%
\pgftext[x=1.456895in,y=2.170431in,,bottom]{\color{textcolor}{\ifdefined\pdftexversion\else\setmainfont{NanumMyeongjo}\rmfamily\fi\fontsize{5.000000}{6.000000}\bfseries\selectfont\catcode`\^=\active\def^{\ifmmode\sp\else\^{}\fi}\catcode`\%=\active\def%{\%}69,227}}%
\end{pgfscope}%
\begin{pgfscope}%
\definecolor{textcolor}{rgb}{0.000000,0.000000,0.000000}%
\pgfsetstrokecolor{textcolor}%
\pgfsetfillcolor{textcolor}%
\pgftext[x=1.815328in,y=1.993851in,,bottom]{\color{textcolor}{\ifdefined\pdftexversion\else\setmainfont{NanumMyeongjo}\rmfamily\fi\fontsize{5.000000}{6.000000}\bfseries\selectfont\catcode`\^=\active\def^{\ifmmode\sp\else\^{}\fi}\catcode`\%=\active\def%{\%}61,098}}%
\end{pgfscope}%
\begin{pgfscope}%
\definecolor{textcolor}{rgb}{0.000000,0.000000,0.000000}%
\pgfsetstrokecolor{textcolor}%
\pgfsetfillcolor{textcolor}%
\pgftext[x=2.173760in,y=1.927511in,,bottom]{\color{textcolor}{\ifdefined\pdftexversion\else\setmainfont{NanumMyeongjo}\rmfamily\fi\fontsize{5.000000}{6.000000}\bfseries\selectfont\catcode`\^=\active\def^{\ifmmode\sp\else\^{}\fi}\catcode`\%=\active\def%{\%}58,044}}%
\end{pgfscope}%
\begin{pgfscope}%
\definecolor{textcolor}{rgb}{0.000000,0.000000,0.000000}%
\pgfsetstrokecolor{textcolor}%
\pgfsetfillcolor{textcolor}%
\pgftext[x=2.532193in,y=2.031387in,,bottom]{\color{textcolor}{\ifdefined\pdftexversion\else\setmainfont{NanumMyeongjo}\rmfamily\fi\fontsize{5.000000}{6.000000}\bfseries\selectfont\catcode`\^=\active\def^{\ifmmode\sp\else\^{}\fi}\catcode`\%=\active\def%{\%}62,826}}%
\end{pgfscope}%
\begin{pgfscope}%
\definecolor{textcolor}{rgb}{0.000000,0.000000,0.000000}%
\pgfsetstrokecolor{textcolor}%
\pgfsetfillcolor{textcolor}%
\pgftext[x=2.890625in,y=2.223694in,,bottom]{\color{textcolor}{\ifdefined\pdftexversion\else\setmainfont{NanumMyeongjo}\rmfamily\fi\fontsize{5.000000}{6.000000}\bfseries\selectfont\catcode`\^=\active\def^{\ifmmode\sp\else\^{}\fi}\catcode`\%=\active\def%{\%}71,679}}%
\end{pgfscope}%
\begin{pgfscope}%
\definecolor{textcolor}{rgb}{0.000000,0.000000,0.000000}%
\pgfsetstrokecolor{textcolor}%
\pgfsetfillcolor{textcolor}%
\pgftext[x=3.249057in,y=2.116299in,,bottom]{\color{textcolor}{\ifdefined\pdftexversion\else\setmainfont{NanumMyeongjo}\rmfamily\fi\fontsize{5.000000}{6.000000}\bfseries\selectfont\catcode`\^=\active\def^{\ifmmode\sp\else\^{}\fi}\catcode`\%=\active\def%{\%}66,735}}%
\end{pgfscope}%
\begin{pgfscope}%
\definecolor{textcolor}{rgb}{0.000000,0.000000,0.000000}%
\pgfsetstrokecolor{textcolor}%
\pgfsetfillcolor{textcolor}%
\pgftext[x=3.607490in,y=2.084411in,,bottom]{\color{textcolor}{\ifdefined\pdftexversion\else\setmainfont{NanumMyeongjo}\rmfamily\fi\fontsize{5.000000}{6.000000}\bfseries\selectfont\catcode`\^=\active\def^{\ifmmode\sp\else\^{}\fi}\catcode`\%=\active\def%{\%}65,267}}%
\end{pgfscope}%
\begin{pgfscope}%
\definecolor{textcolor}{rgb}{0.000000,0.000000,0.000000}%
\pgfsetstrokecolor{textcolor}%
\pgfsetfillcolor{textcolor}%
\pgftext[x=3.965922in,y=2.309953in,,bottom]{\color{textcolor}{\ifdefined\pdftexversion\else\setmainfont{NanumMyeongjo}\rmfamily\fi\fontsize{5.000000}{6.000000}\bfseries\selectfont\catcode`\^=\active\def^{\ifmmode\sp\else\^{}\fi}\catcode`\%=\active\def%{\%}75,650}}%
\end{pgfscope}%
\begin{pgfscope}%
\definecolor{textcolor}{rgb}{0.000000,0.000000,0.000000}%
\pgfsetstrokecolor{textcolor}%
\pgfsetfillcolor{textcolor}%
\pgftext[x=4.324355in,y=2.351008in,,bottom]{\color{textcolor}{\ifdefined\pdftexversion\else\setmainfont{NanumMyeongjo}\rmfamily\fi\fontsize{5.000000}{6.000000}\bfseries\selectfont\catcode`\^=\active\def^{\ifmmode\sp\else\^{}\fi}\catcode`\%=\active\def%{\%}77,540}}%
\end{pgfscope}%
\begin{pgfscope}%
\definecolor{textcolor}{rgb}{0.000000,0.000000,0.000000}%
\pgfsetstrokecolor{textcolor}%
\pgfsetfillcolor{textcolor}%
\pgftext[x=4.682787in,y=2.507603in,,bottom]{\color{textcolor}{\ifdefined\pdftexversion\else\setmainfont{NanumMyeongjo}\rmfamily\fi\fontsize{5.000000}{6.000000}\bfseries\selectfont\catcode`\^=\active\def^{\ifmmode\sp\else\^{}\fi}\catcode`\%=\active\def%{\%}84,749}}%
\end{pgfscope}%
\begin{pgfscope}%
\definecolor{textcolor}{rgb}{0.333333,0.333333,0.333333}%
\pgfsetstrokecolor{textcolor}%
\pgfsetfillcolor{textcolor}%
\pgftext[x=2.062500in,y=0.159722in,,top]{\color{textcolor}{\ifdefined\pdftexversion\else\setmainfont{NanumMyeongjo}\rmfamily\fi\fontsize{9.000000}{10.800000}\selectfont\catcode`\^=\active\def^{\ifmmode\sp\else\^{}\fi}\catcode`\%=\active\def%{\%}출처: 국가농식품통계서비스(KASS) 자료 기반 저자 작성}}%
\end{pgfscope}%
\begin{pgfscope}%
\definecolor{textcolor}{rgb}{0.333333,0.333333,0.333333}%
\pgfsetstrokecolor{textcolor}%
\pgfsetfillcolor{textcolor}%
\pgftext[x=4.687500in,y=3.034722in,,top]{\color{textcolor}{\ifdefined\pdftexversion\else\setmainfont{NanumMyeongjo}\rmfamily\fi\fontsize{9.000000}{10.800000}\selectfont\catcode`\^=\active\def^{\ifmmode\sp\else\^{}\fi}\catcode`\%=\active\def%{\%}(단위: ha)}}%
\end{pgfscope}%
\end{pgfpicture}%
\makeatother%
\endgroup%
}
\end{center}
}

\slide
{\maintitle}
{\chapterone}
{국내 콩 생산면적}{
\begin{center}
    \hspace*{-55pt}\resizebox{1.5\textwidth}{!}{%% Creator: Matplotlib, PGF backend
%%
%% To include the figure in your LaTeX document, write
%%   \input{<filename>.pgf}
%%
%% Make sure the required packages are loaded in your preamble
%%   \usepackage{pgf}
%%
%% Also ensure that all the required font packages are loaded; for instance,
%% the lmodern package is sometimes necessary when using math font.
%%   \usepackage{lmodern}
%%
%% Figures using additional raster images can only be included by \input if
%% they are in the same directory as the main LaTeX file. For loading figures
%% from other directories you can use the `import` package
%%   \usepackage{import}
%%
%% and then include the figures with
%%   \import{<path to file>}{<filename>.pgf}
%%
%% Matplotlib used the following preamble
%%   \def\mathdefault#1{#1}
%%   \everymath=\expandafter{\the\everymath\displaystyle}
%%   \IfFileExists{scrextend.sty}{
%%     \usepackage[fontsize=9.000000pt]{scrextend}
%%   }{
%%     \renewcommand{\normalsize}{\fontsize{9.000000}{10.800000}\selectfont}
%%     \normalsize
%%   }
%%   
%%   \ifdefined\pdftexversion\else  % non-pdftex case.
%%     \usepackage{fontspec}
%%     \setmainfont{DejaVuSerif.ttf}[Path=\detokenize{/home/user/.cache/pypoetry/virtualenvs/graph-KASAOWVd-py3.12/lib/python3.12/site-packages/matplotlib/mpl-data/fonts/ttf/}]
%%     \setsansfont{DejaVuSans.ttf}[Path=\detokenize{/home/user/.cache/pypoetry/virtualenvs/graph-KASAOWVd-py3.12/lib/python3.12/site-packages/matplotlib/mpl-data/fonts/ttf/}]
%%     \setmonofont{DejaVuSansMono.ttf}[Path=\detokenize{/home/user/.cache/pypoetry/virtualenvs/graph-KASAOWVd-py3.12/lib/python3.12/site-packages/matplotlib/mpl-data/fonts/ttf/}]
%%   \fi
%%   \makeatletter\@ifpackageloaded{underscore}{}{\usepackage[strings]{underscore}}\makeatother
%%
\begingroup%
\makeatletter%
\begin{pgfpicture}%
\pgfpathrectangle{\pgfpointorigin}{\pgfqpoint{6.250000in}{3.194444in}}%
\pgfusepath{use as bounding box, clip}%
\begin{pgfscope}%
\pgfsetbuttcap%
\pgfsetmiterjoin%
\definecolor{currentfill}{rgb}{1.000000,1.000000,1.000000}%
\pgfsetfillcolor{currentfill}%
\pgfsetlinewidth{0.000000pt}%
\definecolor{currentstroke}{rgb}{1.000000,1.000000,1.000000}%
\pgfsetstrokecolor{currentstroke}%
\pgfsetdash{}{0pt}%
\pgfpathmoveto{\pgfqpoint{0.000000in}{0.000000in}}%
\pgfpathlineto{\pgfqpoint{6.250000in}{0.000000in}}%
\pgfpathlineto{\pgfqpoint{6.250000in}{3.194444in}}%
\pgfpathlineto{\pgfqpoint{0.000000in}{3.194444in}}%
\pgfpathlineto{\pgfqpoint{0.000000in}{0.000000in}}%
\pgfpathclose%
\pgfusepath{fill}%
\end{pgfscope}%
\begin{pgfscope}%
\pgfsetbuttcap%
\pgfsetmiterjoin%
\definecolor{currentfill}{rgb}{1.000000,1.000000,1.000000}%
\pgfsetfillcolor{currentfill}%
\pgfsetlinewidth{0.000000pt}%
\definecolor{currentstroke}{rgb}{0.000000,0.000000,0.000000}%
\pgfsetstrokecolor{currentstroke}%
\pgfsetstrokeopacity{0.000000}%
\pgfsetdash{}{0pt}%
\pgfpathmoveto{\pgfqpoint{0.781250in}{0.638889in}}%
\pgfpathlineto{\pgfqpoint{5.000000in}{0.638889in}}%
\pgfpathlineto{\pgfqpoint{5.000000in}{2.811111in}}%
\pgfpathlineto{\pgfqpoint{0.781250in}{2.811111in}}%
\pgfpathlineto{\pgfqpoint{0.781250in}{0.638889in}}%
\pgfpathclose%
\pgfusepath{fill}%
\end{pgfscope}%
\begin{pgfscope}%
\pgfsetbuttcap%
\pgfsetroundjoin%
\definecolor{currentfill}{rgb}{0.000000,0.000000,0.000000}%
\pgfsetfillcolor{currentfill}%
\pgfsetlinewidth{0.752812pt}%
\definecolor{currentstroke}{rgb}{0.000000,0.000000,0.000000}%
\pgfsetstrokecolor{currentstroke}%
\pgfsetdash{}{0pt}%
\pgfsys@defobject{currentmarker}{\pgfqpoint{0.000000in}{-0.013889in}}{\pgfqpoint{0.000000in}{0.000000in}}{%
\pgfpathmoveto{\pgfqpoint{0.000000in}{0.000000in}}%
\pgfpathlineto{\pgfqpoint{0.000000in}{-0.013889in}}%
\pgfusepath{stroke,fill}%
}%
\begin{pgfscope}%
\pgfsys@transformshift{1.098463in}{0.638889in}%
\pgfsys@useobject{currentmarker}{}%
\end{pgfscope}%
\end{pgfscope}%
\begin{pgfscope}%
\definecolor{textcolor}{rgb}{0.000000,0.000000,0.000000}%
\pgfsetstrokecolor{textcolor}%
\pgfsetfillcolor{textcolor}%
\pgftext[x=1.028589in, y=0.315885in, left, base,rotate=45.000000]{\color{textcolor}{\ifdefined\pdftexversion\else\setmainfont{NanumMyeongjo}\rmfamily\fi\fontsize{9.000000}{10.800000}\selectfont\catcode`\^=\active\def^{\ifmmode\sp\else\^{}\fi}\catcode`\%=\active\def%{\%}2014}}%
\end{pgfscope}%
\begin{pgfscope}%
\pgfsetbuttcap%
\pgfsetroundjoin%
\definecolor{currentfill}{rgb}{0.000000,0.000000,0.000000}%
\pgfsetfillcolor{currentfill}%
\pgfsetlinewidth{0.752812pt}%
\definecolor{currentstroke}{rgb}{0.000000,0.000000,0.000000}%
\pgfsetstrokecolor{currentstroke}%
\pgfsetdash{}{0pt}%
\pgfsys@defobject{currentmarker}{\pgfqpoint{0.000000in}{-0.013889in}}{\pgfqpoint{0.000000in}{0.000000in}}{%
\pgfpathmoveto{\pgfqpoint{0.000000in}{0.000000in}}%
\pgfpathlineto{\pgfqpoint{0.000000in}{-0.013889in}}%
\pgfusepath{stroke,fill}%
}%
\begin{pgfscope}%
\pgfsys@transformshift{1.456895in}{0.638889in}%
\pgfsys@useobject{currentmarker}{}%
\end{pgfscope}%
\end{pgfscope}%
\begin{pgfscope}%
\definecolor{textcolor}{rgb}{0.000000,0.000000,0.000000}%
\pgfsetstrokecolor{textcolor}%
\pgfsetfillcolor{textcolor}%
\pgftext[x=1.387022in, y=0.315885in, left, base,rotate=45.000000]{\color{textcolor}{\ifdefined\pdftexversion\else\setmainfont{NanumMyeongjo}\rmfamily\fi\fontsize{9.000000}{10.800000}\selectfont\catcode`\^=\active\def^{\ifmmode\sp\else\^{}\fi}\catcode`\%=\active\def%{\%}2015}}%
\end{pgfscope}%
\begin{pgfscope}%
\pgfsetbuttcap%
\pgfsetroundjoin%
\definecolor{currentfill}{rgb}{0.000000,0.000000,0.000000}%
\pgfsetfillcolor{currentfill}%
\pgfsetlinewidth{0.752812pt}%
\definecolor{currentstroke}{rgb}{0.000000,0.000000,0.000000}%
\pgfsetstrokecolor{currentstroke}%
\pgfsetdash{}{0pt}%
\pgfsys@defobject{currentmarker}{\pgfqpoint{0.000000in}{-0.013889in}}{\pgfqpoint{0.000000in}{0.000000in}}{%
\pgfpathmoveto{\pgfqpoint{0.000000in}{0.000000in}}%
\pgfpathlineto{\pgfqpoint{0.000000in}{-0.013889in}}%
\pgfusepath{stroke,fill}%
}%
\begin{pgfscope}%
\pgfsys@transformshift{1.815328in}{0.638889in}%
\pgfsys@useobject{currentmarker}{}%
\end{pgfscope}%
\end{pgfscope}%
\begin{pgfscope}%
\definecolor{textcolor}{rgb}{0.000000,0.000000,0.000000}%
\pgfsetstrokecolor{textcolor}%
\pgfsetfillcolor{textcolor}%
\pgftext[x=1.745454in, y=0.315885in, left, base,rotate=45.000000]{\color{textcolor}{\ifdefined\pdftexversion\else\setmainfont{NanumMyeongjo}\rmfamily\fi\fontsize{9.000000}{10.800000}\selectfont\catcode`\^=\active\def^{\ifmmode\sp\else\^{}\fi}\catcode`\%=\active\def%{\%}2016}}%
\end{pgfscope}%
\begin{pgfscope}%
\pgfsetbuttcap%
\pgfsetroundjoin%
\definecolor{currentfill}{rgb}{0.000000,0.000000,0.000000}%
\pgfsetfillcolor{currentfill}%
\pgfsetlinewidth{0.752812pt}%
\definecolor{currentstroke}{rgb}{0.000000,0.000000,0.000000}%
\pgfsetstrokecolor{currentstroke}%
\pgfsetdash{}{0pt}%
\pgfsys@defobject{currentmarker}{\pgfqpoint{0.000000in}{-0.013889in}}{\pgfqpoint{0.000000in}{0.000000in}}{%
\pgfpathmoveto{\pgfqpoint{0.000000in}{0.000000in}}%
\pgfpathlineto{\pgfqpoint{0.000000in}{-0.013889in}}%
\pgfusepath{stroke,fill}%
}%
\begin{pgfscope}%
\pgfsys@transformshift{2.173760in}{0.638889in}%
\pgfsys@useobject{currentmarker}{}%
\end{pgfscope}%
\end{pgfscope}%
\begin{pgfscope}%
\definecolor{textcolor}{rgb}{0.000000,0.000000,0.000000}%
\pgfsetstrokecolor{textcolor}%
\pgfsetfillcolor{textcolor}%
\pgftext[x=2.103887in, y=0.315885in, left, base,rotate=45.000000]{\color{textcolor}{\ifdefined\pdftexversion\else\setmainfont{NanumMyeongjo}\rmfamily\fi\fontsize{9.000000}{10.800000}\selectfont\catcode`\^=\active\def^{\ifmmode\sp\else\^{}\fi}\catcode`\%=\active\def%{\%}2017}}%
\end{pgfscope}%
\begin{pgfscope}%
\pgfsetbuttcap%
\pgfsetroundjoin%
\definecolor{currentfill}{rgb}{0.000000,0.000000,0.000000}%
\pgfsetfillcolor{currentfill}%
\pgfsetlinewidth{0.752812pt}%
\definecolor{currentstroke}{rgb}{0.000000,0.000000,0.000000}%
\pgfsetstrokecolor{currentstroke}%
\pgfsetdash{}{0pt}%
\pgfsys@defobject{currentmarker}{\pgfqpoint{0.000000in}{-0.013889in}}{\pgfqpoint{0.000000in}{0.000000in}}{%
\pgfpathmoveto{\pgfqpoint{0.000000in}{0.000000in}}%
\pgfpathlineto{\pgfqpoint{0.000000in}{-0.013889in}}%
\pgfusepath{stroke,fill}%
}%
\begin{pgfscope}%
\pgfsys@transformshift{2.532193in}{0.638889in}%
\pgfsys@useobject{currentmarker}{}%
\end{pgfscope}%
\end{pgfscope}%
\begin{pgfscope}%
\definecolor{textcolor}{rgb}{0.000000,0.000000,0.000000}%
\pgfsetstrokecolor{textcolor}%
\pgfsetfillcolor{textcolor}%
\pgftext[x=2.462319in, y=0.315885in, left, base,rotate=45.000000]{\color{textcolor}{\ifdefined\pdftexversion\else\setmainfont{NanumMyeongjo}\rmfamily\fi\fontsize{9.000000}{10.800000}\selectfont\catcode`\^=\active\def^{\ifmmode\sp\else\^{}\fi}\catcode`\%=\active\def%{\%}2018}}%
\end{pgfscope}%
\begin{pgfscope}%
\pgfsetbuttcap%
\pgfsetroundjoin%
\definecolor{currentfill}{rgb}{0.000000,0.000000,0.000000}%
\pgfsetfillcolor{currentfill}%
\pgfsetlinewidth{0.752812pt}%
\definecolor{currentstroke}{rgb}{0.000000,0.000000,0.000000}%
\pgfsetstrokecolor{currentstroke}%
\pgfsetdash{}{0pt}%
\pgfsys@defobject{currentmarker}{\pgfqpoint{0.000000in}{-0.013889in}}{\pgfqpoint{0.000000in}{0.000000in}}{%
\pgfpathmoveto{\pgfqpoint{0.000000in}{0.000000in}}%
\pgfpathlineto{\pgfqpoint{0.000000in}{-0.013889in}}%
\pgfusepath{stroke,fill}%
}%
\begin{pgfscope}%
\pgfsys@transformshift{2.890625in}{0.638889in}%
\pgfsys@useobject{currentmarker}{}%
\end{pgfscope}%
\end{pgfscope}%
\begin{pgfscope}%
\definecolor{textcolor}{rgb}{0.000000,0.000000,0.000000}%
\pgfsetstrokecolor{textcolor}%
\pgfsetfillcolor{textcolor}%
\pgftext[x=2.820752in, y=0.315885in, left, base,rotate=45.000000]{\color{textcolor}{\ifdefined\pdftexversion\else\setmainfont{NanumMyeongjo}\rmfamily\fi\fontsize{9.000000}{10.800000}\selectfont\catcode`\^=\active\def^{\ifmmode\sp\else\^{}\fi}\catcode`\%=\active\def%{\%}2019}}%
\end{pgfscope}%
\begin{pgfscope}%
\pgfsetbuttcap%
\pgfsetroundjoin%
\definecolor{currentfill}{rgb}{0.000000,0.000000,0.000000}%
\pgfsetfillcolor{currentfill}%
\pgfsetlinewidth{0.752812pt}%
\definecolor{currentstroke}{rgb}{0.000000,0.000000,0.000000}%
\pgfsetstrokecolor{currentstroke}%
\pgfsetdash{}{0pt}%
\pgfsys@defobject{currentmarker}{\pgfqpoint{0.000000in}{-0.013889in}}{\pgfqpoint{0.000000in}{0.000000in}}{%
\pgfpathmoveto{\pgfqpoint{0.000000in}{0.000000in}}%
\pgfpathlineto{\pgfqpoint{0.000000in}{-0.013889in}}%
\pgfusepath{stroke,fill}%
}%
\begin{pgfscope}%
\pgfsys@transformshift{3.249057in}{0.638889in}%
\pgfsys@useobject{currentmarker}{}%
\end{pgfscope}%
\end{pgfscope}%
\begin{pgfscope}%
\definecolor{textcolor}{rgb}{0.000000,0.000000,0.000000}%
\pgfsetstrokecolor{textcolor}%
\pgfsetfillcolor{textcolor}%
\pgftext[x=3.179184in, y=0.315885in, left, base,rotate=45.000000]{\color{textcolor}{\ifdefined\pdftexversion\else\setmainfont{NanumMyeongjo}\rmfamily\fi\fontsize{9.000000}{10.800000}\selectfont\catcode`\^=\active\def^{\ifmmode\sp\else\^{}\fi}\catcode`\%=\active\def%{\%}2020}}%
\end{pgfscope}%
\begin{pgfscope}%
\pgfsetbuttcap%
\pgfsetroundjoin%
\definecolor{currentfill}{rgb}{0.000000,0.000000,0.000000}%
\pgfsetfillcolor{currentfill}%
\pgfsetlinewidth{0.752812pt}%
\definecolor{currentstroke}{rgb}{0.000000,0.000000,0.000000}%
\pgfsetstrokecolor{currentstroke}%
\pgfsetdash{}{0pt}%
\pgfsys@defobject{currentmarker}{\pgfqpoint{0.000000in}{-0.013889in}}{\pgfqpoint{0.000000in}{0.000000in}}{%
\pgfpathmoveto{\pgfqpoint{0.000000in}{0.000000in}}%
\pgfpathlineto{\pgfqpoint{0.000000in}{-0.013889in}}%
\pgfusepath{stroke,fill}%
}%
\begin{pgfscope}%
\pgfsys@transformshift{3.607490in}{0.638889in}%
\pgfsys@useobject{currentmarker}{}%
\end{pgfscope}%
\end{pgfscope}%
\begin{pgfscope}%
\definecolor{textcolor}{rgb}{0.000000,0.000000,0.000000}%
\pgfsetstrokecolor{textcolor}%
\pgfsetfillcolor{textcolor}%
\pgftext[x=3.537617in, y=0.315885in, left, base,rotate=45.000000]{\color{textcolor}{\ifdefined\pdftexversion\else\setmainfont{NanumMyeongjo}\rmfamily\fi\fontsize{9.000000}{10.800000}\selectfont\catcode`\^=\active\def^{\ifmmode\sp\else\^{}\fi}\catcode`\%=\active\def%{\%}2021}}%
\end{pgfscope}%
\begin{pgfscope}%
\pgfsetbuttcap%
\pgfsetroundjoin%
\definecolor{currentfill}{rgb}{0.000000,0.000000,0.000000}%
\pgfsetfillcolor{currentfill}%
\pgfsetlinewidth{0.752812pt}%
\definecolor{currentstroke}{rgb}{0.000000,0.000000,0.000000}%
\pgfsetstrokecolor{currentstroke}%
\pgfsetdash{}{0pt}%
\pgfsys@defobject{currentmarker}{\pgfqpoint{0.000000in}{-0.013889in}}{\pgfqpoint{0.000000in}{0.000000in}}{%
\pgfpathmoveto{\pgfqpoint{0.000000in}{0.000000in}}%
\pgfpathlineto{\pgfqpoint{0.000000in}{-0.013889in}}%
\pgfusepath{stroke,fill}%
}%
\begin{pgfscope}%
\pgfsys@transformshift{3.965922in}{0.638889in}%
\pgfsys@useobject{currentmarker}{}%
\end{pgfscope}%
\end{pgfscope}%
\begin{pgfscope}%
\definecolor{textcolor}{rgb}{0.000000,0.000000,0.000000}%
\pgfsetstrokecolor{textcolor}%
\pgfsetfillcolor{textcolor}%
\pgftext[x=3.896049in, y=0.315885in, left, base,rotate=45.000000]{\color{textcolor}{\ifdefined\pdftexversion\else\setmainfont{NanumMyeongjo}\rmfamily\fi\fontsize{9.000000}{10.800000}\selectfont\catcode`\^=\active\def^{\ifmmode\sp\else\^{}\fi}\catcode`\%=\active\def%{\%}2022}}%
\end{pgfscope}%
\begin{pgfscope}%
\pgfsetbuttcap%
\pgfsetroundjoin%
\definecolor{currentfill}{rgb}{0.000000,0.000000,0.000000}%
\pgfsetfillcolor{currentfill}%
\pgfsetlinewidth{0.752812pt}%
\definecolor{currentstroke}{rgb}{0.000000,0.000000,0.000000}%
\pgfsetstrokecolor{currentstroke}%
\pgfsetdash{}{0pt}%
\pgfsys@defobject{currentmarker}{\pgfqpoint{0.000000in}{-0.013889in}}{\pgfqpoint{0.000000in}{0.000000in}}{%
\pgfpathmoveto{\pgfqpoint{0.000000in}{0.000000in}}%
\pgfpathlineto{\pgfqpoint{0.000000in}{-0.013889in}}%
\pgfusepath{stroke,fill}%
}%
\begin{pgfscope}%
\pgfsys@transformshift{4.324355in}{0.638889in}%
\pgfsys@useobject{currentmarker}{}%
\end{pgfscope}%
\end{pgfscope}%
\begin{pgfscope}%
\definecolor{textcolor}{rgb}{0.000000,0.000000,0.000000}%
\pgfsetstrokecolor{textcolor}%
\pgfsetfillcolor{textcolor}%
\pgftext[x=4.254481in, y=0.315885in, left, base,rotate=45.000000]{\color{textcolor}{\ifdefined\pdftexversion\else\setmainfont{NanumMyeongjo}\rmfamily\fi\fontsize{9.000000}{10.800000}\selectfont\catcode`\^=\active\def^{\ifmmode\sp\else\^{}\fi}\catcode`\%=\active\def%{\%}2023}}%
\end{pgfscope}%
\begin{pgfscope}%
\pgfsetbuttcap%
\pgfsetroundjoin%
\definecolor{currentfill}{rgb}{0.000000,0.000000,0.000000}%
\pgfsetfillcolor{currentfill}%
\pgfsetlinewidth{0.752812pt}%
\definecolor{currentstroke}{rgb}{0.000000,0.000000,0.000000}%
\pgfsetstrokecolor{currentstroke}%
\pgfsetdash{}{0pt}%
\pgfsys@defobject{currentmarker}{\pgfqpoint{0.000000in}{-0.013889in}}{\pgfqpoint{0.000000in}{0.000000in}}{%
\pgfpathmoveto{\pgfqpoint{0.000000in}{0.000000in}}%
\pgfpathlineto{\pgfqpoint{0.000000in}{-0.013889in}}%
\pgfusepath{stroke,fill}%
}%
\begin{pgfscope}%
\pgfsys@transformshift{4.682787in}{0.638889in}%
\pgfsys@useobject{currentmarker}{}%
\end{pgfscope}%
\end{pgfscope}%
\begin{pgfscope}%
\definecolor{textcolor}{rgb}{0.000000,0.000000,0.000000}%
\pgfsetstrokecolor{textcolor}%
\pgfsetfillcolor{textcolor}%
\pgftext[x=4.612914in, y=0.315885in, left, base,rotate=45.000000]{\color{textcolor}{\ifdefined\pdftexversion\else\setmainfont{NanumMyeongjo}\rmfamily\fi\fontsize{9.000000}{10.800000}\selectfont\catcode`\^=\active\def^{\ifmmode\sp\else\^{}\fi}\catcode`\%=\active\def%{\%}2024}}%
\end{pgfscope}%
\begin{pgfscope}%
\pgfpathrectangle{\pgfqpoint{0.781250in}{0.638889in}}{\pgfqpoint{4.218750in}{2.172222in}}%
\pgfusepath{clip}%
\pgfsetbuttcap%
\pgfsetroundjoin%
\pgfsetlinewidth{0.602250pt}%
\definecolor{currentstroke}{rgb}{0.690196,0.690196,0.690196}%
\pgfsetstrokecolor{currentstroke}%
\pgfsetstrokeopacity{0.450000}%
\pgfsetdash{{2.220000pt}{0.960000pt}}{0.000000pt}%
\pgfpathmoveto{\pgfqpoint{0.781250in}{0.638889in}}%
\pgfpathlineto{\pgfqpoint{5.000000in}{0.638889in}}%
\pgfusepath{stroke}%
\end{pgfscope}%
\begin{pgfscope}%
\pgfsetbuttcap%
\pgfsetroundjoin%
\definecolor{currentfill}{rgb}{0.000000,0.000000,0.000000}%
\pgfsetfillcolor{currentfill}%
\pgfsetlinewidth{0.752812pt}%
\definecolor{currentstroke}{rgb}{0.000000,0.000000,0.000000}%
\pgfsetstrokecolor{currentstroke}%
\pgfsetdash{}{0pt}%
\pgfsys@defobject{currentmarker}{\pgfqpoint{-0.013889in}{0.000000in}}{\pgfqpoint{-0.000000in}{0.000000in}}{%
\pgfpathmoveto{\pgfqpoint{-0.000000in}{0.000000in}}%
\pgfpathlineto{\pgfqpoint{-0.013889in}{0.000000in}}%
\pgfusepath{stroke,fill}%
}%
\begin{pgfscope}%
\pgfsys@transformshift{0.781250in}{0.638889in}%
\pgfsys@useobject{currentmarker}{}%
\end{pgfscope}%
\end{pgfscope}%
\begin{pgfscope}%
\definecolor{textcolor}{rgb}{0.000000,0.000000,0.000000}%
\pgfsetstrokecolor{textcolor}%
\pgfsetfillcolor{textcolor}%
\pgftext[x=0.651611in, y=0.588962in, left, base]{\color{textcolor}{\ifdefined\pdftexversion\else\setmainfont{NanumMyeongjo}\rmfamily\fi\fontsize{9.000000}{10.800000}\selectfont\catcode`\^=\active\def^{\ifmmode\sp\else\^{}\fi}\catcode`\%=\active\def%{\%}0}}%
\end{pgfscope}%
\begin{pgfscope}%
\pgfpathrectangle{\pgfqpoint{0.781250in}{0.638889in}}{\pgfqpoint{4.218750in}{2.172222in}}%
\pgfusepath{clip}%
\pgfsetbuttcap%
\pgfsetroundjoin%
\pgfsetlinewidth{0.602250pt}%
\definecolor{currentstroke}{rgb}{0.690196,0.690196,0.690196}%
\pgfsetstrokecolor{currentstroke}%
\pgfsetstrokeopacity{0.450000}%
\pgfsetdash{{2.220000pt}{0.960000pt}}{0.000000pt}%
\pgfpathmoveto{\pgfqpoint{0.781250in}{0.856111in}}%
\pgfpathlineto{\pgfqpoint{5.000000in}{0.856111in}}%
\pgfusepath{stroke}%
\end{pgfscope}%
\begin{pgfscope}%
\pgfsetbuttcap%
\pgfsetroundjoin%
\definecolor{currentfill}{rgb}{0.000000,0.000000,0.000000}%
\pgfsetfillcolor{currentfill}%
\pgfsetlinewidth{0.752812pt}%
\definecolor{currentstroke}{rgb}{0.000000,0.000000,0.000000}%
\pgfsetstrokecolor{currentstroke}%
\pgfsetdash{}{0pt}%
\pgfsys@defobject{currentmarker}{\pgfqpoint{-0.013889in}{0.000000in}}{\pgfqpoint{-0.000000in}{0.000000in}}{%
\pgfpathmoveto{\pgfqpoint{-0.000000in}{0.000000in}}%
\pgfpathlineto{\pgfqpoint{-0.013889in}{0.000000in}}%
\pgfusepath{stroke,fill}%
}%
\begin{pgfscope}%
\pgfsys@transformshift{0.781250in}{0.856111in}%
\pgfsys@useobject{currentmarker}{}%
\end{pgfscope}%
\end{pgfscope}%
\begin{pgfscope}%
\definecolor{textcolor}{rgb}{0.000000,0.000000,0.000000}%
\pgfsetstrokecolor{textcolor}%
\pgfsetfillcolor{textcolor}%
\pgftext[x=0.532837in, y=0.806184in, left, base]{\color{textcolor}{\ifdefined\pdftexversion\else\setmainfont{NanumMyeongjo}\rmfamily\fi\fontsize{9.000000}{10.800000}\selectfont\catcode`\^=\active\def^{\ifmmode\sp\else\^{}\fi}\catcode`\%=\active\def%{\%}1만}}%
\end{pgfscope}%
\begin{pgfscope}%
\pgfpathrectangle{\pgfqpoint{0.781250in}{0.638889in}}{\pgfqpoint{4.218750in}{2.172222in}}%
\pgfusepath{clip}%
\pgfsetbuttcap%
\pgfsetroundjoin%
\pgfsetlinewidth{0.602250pt}%
\definecolor{currentstroke}{rgb}{0.690196,0.690196,0.690196}%
\pgfsetstrokecolor{currentstroke}%
\pgfsetstrokeopacity{0.450000}%
\pgfsetdash{{2.220000pt}{0.960000pt}}{0.000000pt}%
\pgfpathmoveto{\pgfqpoint{0.781250in}{1.073333in}}%
\pgfpathlineto{\pgfqpoint{5.000000in}{1.073333in}}%
\pgfusepath{stroke}%
\end{pgfscope}%
\begin{pgfscope}%
\pgfsetbuttcap%
\pgfsetroundjoin%
\definecolor{currentfill}{rgb}{0.000000,0.000000,0.000000}%
\pgfsetfillcolor{currentfill}%
\pgfsetlinewidth{0.752812pt}%
\definecolor{currentstroke}{rgb}{0.000000,0.000000,0.000000}%
\pgfsetstrokecolor{currentstroke}%
\pgfsetdash{}{0pt}%
\pgfsys@defobject{currentmarker}{\pgfqpoint{-0.013889in}{0.000000in}}{\pgfqpoint{-0.000000in}{0.000000in}}{%
\pgfpathmoveto{\pgfqpoint{-0.000000in}{0.000000in}}%
\pgfpathlineto{\pgfqpoint{-0.013889in}{0.000000in}}%
\pgfusepath{stroke,fill}%
}%
\begin{pgfscope}%
\pgfsys@transformshift{0.781250in}{1.073333in}%
\pgfsys@useobject{currentmarker}{}%
\end{pgfscope}%
\end{pgfscope}%
\begin{pgfscope}%
\definecolor{textcolor}{rgb}{0.000000,0.000000,0.000000}%
\pgfsetstrokecolor{textcolor}%
\pgfsetfillcolor{textcolor}%
\pgftext[x=0.532837in, y=1.023407in, left, base]{\color{textcolor}{\ifdefined\pdftexversion\else\setmainfont{NanumMyeongjo}\rmfamily\fi\fontsize{9.000000}{10.800000}\selectfont\catcode`\^=\active\def^{\ifmmode\sp\else\^{}\fi}\catcode`\%=\active\def%{\%}2만}}%
\end{pgfscope}%
\begin{pgfscope}%
\pgfpathrectangle{\pgfqpoint{0.781250in}{0.638889in}}{\pgfqpoint{4.218750in}{2.172222in}}%
\pgfusepath{clip}%
\pgfsetbuttcap%
\pgfsetroundjoin%
\pgfsetlinewidth{0.602250pt}%
\definecolor{currentstroke}{rgb}{0.690196,0.690196,0.690196}%
\pgfsetstrokecolor{currentstroke}%
\pgfsetstrokeopacity{0.450000}%
\pgfsetdash{{2.220000pt}{0.960000pt}}{0.000000pt}%
\pgfpathmoveto{\pgfqpoint{0.781250in}{1.290556in}}%
\pgfpathlineto{\pgfqpoint{5.000000in}{1.290556in}}%
\pgfusepath{stroke}%
\end{pgfscope}%
\begin{pgfscope}%
\pgfsetbuttcap%
\pgfsetroundjoin%
\definecolor{currentfill}{rgb}{0.000000,0.000000,0.000000}%
\pgfsetfillcolor{currentfill}%
\pgfsetlinewidth{0.752812pt}%
\definecolor{currentstroke}{rgb}{0.000000,0.000000,0.000000}%
\pgfsetstrokecolor{currentstroke}%
\pgfsetdash{}{0pt}%
\pgfsys@defobject{currentmarker}{\pgfqpoint{-0.013889in}{0.000000in}}{\pgfqpoint{-0.000000in}{0.000000in}}{%
\pgfpathmoveto{\pgfqpoint{-0.000000in}{0.000000in}}%
\pgfpathlineto{\pgfqpoint{-0.013889in}{0.000000in}}%
\pgfusepath{stroke,fill}%
}%
\begin{pgfscope}%
\pgfsys@transformshift{0.781250in}{1.290556in}%
\pgfsys@useobject{currentmarker}{}%
\end{pgfscope}%
\end{pgfscope}%
\begin{pgfscope}%
\definecolor{textcolor}{rgb}{0.000000,0.000000,0.000000}%
\pgfsetstrokecolor{textcolor}%
\pgfsetfillcolor{textcolor}%
\pgftext[x=0.532837in, y=1.240629in, left, base]{\color{textcolor}{\ifdefined\pdftexversion\else\setmainfont{NanumMyeongjo}\rmfamily\fi\fontsize{9.000000}{10.800000}\selectfont\catcode`\^=\active\def^{\ifmmode\sp\else\^{}\fi}\catcode`\%=\active\def%{\%}3만}}%
\end{pgfscope}%
\begin{pgfscope}%
\pgfpathrectangle{\pgfqpoint{0.781250in}{0.638889in}}{\pgfqpoint{4.218750in}{2.172222in}}%
\pgfusepath{clip}%
\pgfsetbuttcap%
\pgfsetroundjoin%
\pgfsetlinewidth{0.602250pt}%
\definecolor{currentstroke}{rgb}{0.690196,0.690196,0.690196}%
\pgfsetstrokecolor{currentstroke}%
\pgfsetstrokeopacity{0.450000}%
\pgfsetdash{{2.220000pt}{0.960000pt}}{0.000000pt}%
\pgfpathmoveto{\pgfqpoint{0.781250in}{1.507778in}}%
\pgfpathlineto{\pgfqpoint{5.000000in}{1.507778in}}%
\pgfusepath{stroke}%
\end{pgfscope}%
\begin{pgfscope}%
\pgfsetbuttcap%
\pgfsetroundjoin%
\definecolor{currentfill}{rgb}{0.000000,0.000000,0.000000}%
\pgfsetfillcolor{currentfill}%
\pgfsetlinewidth{0.752812pt}%
\definecolor{currentstroke}{rgb}{0.000000,0.000000,0.000000}%
\pgfsetstrokecolor{currentstroke}%
\pgfsetdash{}{0pt}%
\pgfsys@defobject{currentmarker}{\pgfqpoint{-0.013889in}{0.000000in}}{\pgfqpoint{-0.000000in}{0.000000in}}{%
\pgfpathmoveto{\pgfqpoint{-0.000000in}{0.000000in}}%
\pgfpathlineto{\pgfqpoint{-0.013889in}{0.000000in}}%
\pgfusepath{stroke,fill}%
}%
\begin{pgfscope}%
\pgfsys@transformshift{0.781250in}{1.507778in}%
\pgfsys@useobject{currentmarker}{}%
\end{pgfscope}%
\end{pgfscope}%
\begin{pgfscope}%
\definecolor{textcolor}{rgb}{0.000000,0.000000,0.000000}%
\pgfsetstrokecolor{textcolor}%
\pgfsetfillcolor{textcolor}%
\pgftext[x=0.532837in, y=1.457851in, left, base]{\color{textcolor}{\ifdefined\pdftexversion\else\setmainfont{NanumMyeongjo}\rmfamily\fi\fontsize{9.000000}{10.800000}\selectfont\catcode`\^=\active\def^{\ifmmode\sp\else\^{}\fi}\catcode`\%=\active\def%{\%}4만}}%
\end{pgfscope}%
\begin{pgfscope}%
\pgfpathrectangle{\pgfqpoint{0.781250in}{0.638889in}}{\pgfqpoint{4.218750in}{2.172222in}}%
\pgfusepath{clip}%
\pgfsetbuttcap%
\pgfsetroundjoin%
\pgfsetlinewidth{0.602250pt}%
\definecolor{currentstroke}{rgb}{0.690196,0.690196,0.690196}%
\pgfsetstrokecolor{currentstroke}%
\pgfsetstrokeopacity{0.450000}%
\pgfsetdash{{2.220000pt}{0.960000pt}}{0.000000pt}%
\pgfpathmoveto{\pgfqpoint{0.781250in}{1.725000in}}%
\pgfpathlineto{\pgfqpoint{5.000000in}{1.725000in}}%
\pgfusepath{stroke}%
\end{pgfscope}%
\begin{pgfscope}%
\pgfsetbuttcap%
\pgfsetroundjoin%
\definecolor{currentfill}{rgb}{0.000000,0.000000,0.000000}%
\pgfsetfillcolor{currentfill}%
\pgfsetlinewidth{0.752812pt}%
\definecolor{currentstroke}{rgb}{0.000000,0.000000,0.000000}%
\pgfsetstrokecolor{currentstroke}%
\pgfsetdash{}{0pt}%
\pgfsys@defobject{currentmarker}{\pgfqpoint{-0.013889in}{0.000000in}}{\pgfqpoint{-0.000000in}{0.000000in}}{%
\pgfpathmoveto{\pgfqpoint{-0.000000in}{0.000000in}}%
\pgfpathlineto{\pgfqpoint{-0.013889in}{0.000000in}}%
\pgfusepath{stroke,fill}%
}%
\begin{pgfscope}%
\pgfsys@transformshift{0.781250in}{1.725000in}%
\pgfsys@useobject{currentmarker}{}%
\end{pgfscope}%
\end{pgfscope}%
\begin{pgfscope}%
\definecolor{textcolor}{rgb}{0.000000,0.000000,0.000000}%
\pgfsetstrokecolor{textcolor}%
\pgfsetfillcolor{textcolor}%
\pgftext[x=0.532837in, y=1.675073in, left, base]{\color{textcolor}{\ifdefined\pdftexversion\else\setmainfont{NanumMyeongjo}\rmfamily\fi\fontsize{9.000000}{10.800000}\selectfont\catcode`\^=\active\def^{\ifmmode\sp\else\^{}\fi}\catcode`\%=\active\def%{\%}5만}}%
\end{pgfscope}%
\begin{pgfscope}%
\pgfpathrectangle{\pgfqpoint{0.781250in}{0.638889in}}{\pgfqpoint{4.218750in}{2.172222in}}%
\pgfusepath{clip}%
\pgfsetbuttcap%
\pgfsetroundjoin%
\pgfsetlinewidth{0.602250pt}%
\definecolor{currentstroke}{rgb}{0.690196,0.690196,0.690196}%
\pgfsetstrokecolor{currentstroke}%
\pgfsetstrokeopacity{0.450000}%
\pgfsetdash{{2.220000pt}{0.960000pt}}{0.000000pt}%
\pgfpathmoveto{\pgfqpoint{0.781250in}{1.942222in}}%
\pgfpathlineto{\pgfqpoint{5.000000in}{1.942222in}}%
\pgfusepath{stroke}%
\end{pgfscope}%
\begin{pgfscope}%
\pgfsetbuttcap%
\pgfsetroundjoin%
\definecolor{currentfill}{rgb}{0.000000,0.000000,0.000000}%
\pgfsetfillcolor{currentfill}%
\pgfsetlinewidth{0.752812pt}%
\definecolor{currentstroke}{rgb}{0.000000,0.000000,0.000000}%
\pgfsetstrokecolor{currentstroke}%
\pgfsetdash{}{0pt}%
\pgfsys@defobject{currentmarker}{\pgfqpoint{-0.013889in}{0.000000in}}{\pgfqpoint{-0.000000in}{0.000000in}}{%
\pgfpathmoveto{\pgfqpoint{-0.000000in}{0.000000in}}%
\pgfpathlineto{\pgfqpoint{-0.013889in}{0.000000in}}%
\pgfusepath{stroke,fill}%
}%
\begin{pgfscope}%
\pgfsys@transformshift{0.781250in}{1.942222in}%
\pgfsys@useobject{currentmarker}{}%
\end{pgfscope}%
\end{pgfscope}%
\begin{pgfscope}%
\definecolor{textcolor}{rgb}{0.000000,0.000000,0.000000}%
\pgfsetstrokecolor{textcolor}%
\pgfsetfillcolor{textcolor}%
\pgftext[x=0.532837in, y=1.892295in, left, base]{\color{textcolor}{\ifdefined\pdftexversion\else\setmainfont{NanumMyeongjo}\rmfamily\fi\fontsize{9.000000}{10.800000}\selectfont\catcode`\^=\active\def^{\ifmmode\sp\else\^{}\fi}\catcode`\%=\active\def%{\%}6만}}%
\end{pgfscope}%
\begin{pgfscope}%
\pgfpathrectangle{\pgfqpoint{0.781250in}{0.638889in}}{\pgfqpoint{4.218750in}{2.172222in}}%
\pgfusepath{clip}%
\pgfsetbuttcap%
\pgfsetroundjoin%
\pgfsetlinewidth{0.602250pt}%
\definecolor{currentstroke}{rgb}{0.690196,0.690196,0.690196}%
\pgfsetstrokecolor{currentstroke}%
\pgfsetstrokeopacity{0.450000}%
\pgfsetdash{{2.220000pt}{0.960000pt}}{0.000000pt}%
\pgfpathmoveto{\pgfqpoint{0.781250in}{2.159444in}}%
\pgfpathlineto{\pgfqpoint{5.000000in}{2.159444in}}%
\pgfusepath{stroke}%
\end{pgfscope}%
\begin{pgfscope}%
\pgfsetbuttcap%
\pgfsetroundjoin%
\definecolor{currentfill}{rgb}{0.000000,0.000000,0.000000}%
\pgfsetfillcolor{currentfill}%
\pgfsetlinewidth{0.752812pt}%
\definecolor{currentstroke}{rgb}{0.000000,0.000000,0.000000}%
\pgfsetstrokecolor{currentstroke}%
\pgfsetdash{}{0pt}%
\pgfsys@defobject{currentmarker}{\pgfqpoint{-0.013889in}{0.000000in}}{\pgfqpoint{-0.000000in}{0.000000in}}{%
\pgfpathmoveto{\pgfqpoint{-0.000000in}{0.000000in}}%
\pgfpathlineto{\pgfqpoint{-0.013889in}{0.000000in}}%
\pgfusepath{stroke,fill}%
}%
\begin{pgfscope}%
\pgfsys@transformshift{0.781250in}{2.159444in}%
\pgfsys@useobject{currentmarker}{}%
\end{pgfscope}%
\end{pgfscope}%
\begin{pgfscope}%
\definecolor{textcolor}{rgb}{0.000000,0.000000,0.000000}%
\pgfsetstrokecolor{textcolor}%
\pgfsetfillcolor{textcolor}%
\pgftext[x=0.532837in, y=2.109518in, left, base]{\color{textcolor}{\ifdefined\pdftexversion\else\setmainfont{NanumMyeongjo}\rmfamily\fi\fontsize{9.000000}{10.800000}\selectfont\catcode`\^=\active\def^{\ifmmode\sp\else\^{}\fi}\catcode`\%=\active\def%{\%}7만}}%
\end{pgfscope}%
\begin{pgfscope}%
\pgfpathrectangle{\pgfqpoint{0.781250in}{0.638889in}}{\pgfqpoint{4.218750in}{2.172222in}}%
\pgfusepath{clip}%
\pgfsetbuttcap%
\pgfsetroundjoin%
\pgfsetlinewidth{0.602250pt}%
\definecolor{currentstroke}{rgb}{0.690196,0.690196,0.690196}%
\pgfsetstrokecolor{currentstroke}%
\pgfsetstrokeopacity{0.450000}%
\pgfsetdash{{2.220000pt}{0.960000pt}}{0.000000pt}%
\pgfpathmoveto{\pgfqpoint{0.781250in}{2.376667in}}%
\pgfpathlineto{\pgfqpoint{5.000000in}{2.376667in}}%
\pgfusepath{stroke}%
\end{pgfscope}%
\begin{pgfscope}%
\pgfsetbuttcap%
\pgfsetroundjoin%
\definecolor{currentfill}{rgb}{0.000000,0.000000,0.000000}%
\pgfsetfillcolor{currentfill}%
\pgfsetlinewidth{0.752812pt}%
\definecolor{currentstroke}{rgb}{0.000000,0.000000,0.000000}%
\pgfsetstrokecolor{currentstroke}%
\pgfsetdash{}{0pt}%
\pgfsys@defobject{currentmarker}{\pgfqpoint{-0.013889in}{0.000000in}}{\pgfqpoint{-0.000000in}{0.000000in}}{%
\pgfpathmoveto{\pgfqpoint{-0.000000in}{0.000000in}}%
\pgfpathlineto{\pgfqpoint{-0.013889in}{0.000000in}}%
\pgfusepath{stroke,fill}%
}%
\begin{pgfscope}%
\pgfsys@transformshift{0.781250in}{2.376667in}%
\pgfsys@useobject{currentmarker}{}%
\end{pgfscope}%
\end{pgfscope}%
\begin{pgfscope}%
\definecolor{textcolor}{rgb}{0.000000,0.000000,0.000000}%
\pgfsetstrokecolor{textcolor}%
\pgfsetfillcolor{textcolor}%
\pgftext[x=0.532837in, y=2.326740in, left, base]{\color{textcolor}{\ifdefined\pdftexversion\else\setmainfont{NanumMyeongjo}\rmfamily\fi\fontsize{9.000000}{10.800000}\selectfont\catcode`\^=\active\def^{\ifmmode\sp\else\^{}\fi}\catcode`\%=\active\def%{\%}8만}}%
\end{pgfscope}%
\begin{pgfscope}%
\pgfpathrectangle{\pgfqpoint{0.781250in}{0.638889in}}{\pgfqpoint{4.218750in}{2.172222in}}%
\pgfusepath{clip}%
\pgfsetbuttcap%
\pgfsetroundjoin%
\pgfsetlinewidth{0.602250pt}%
\definecolor{currentstroke}{rgb}{0.690196,0.690196,0.690196}%
\pgfsetstrokecolor{currentstroke}%
\pgfsetstrokeopacity{0.450000}%
\pgfsetdash{{2.220000pt}{0.960000pt}}{0.000000pt}%
\pgfpathmoveto{\pgfqpoint{0.781250in}{2.593889in}}%
\pgfpathlineto{\pgfqpoint{5.000000in}{2.593889in}}%
\pgfusepath{stroke}%
\end{pgfscope}%
\begin{pgfscope}%
\pgfsetbuttcap%
\pgfsetroundjoin%
\definecolor{currentfill}{rgb}{0.000000,0.000000,0.000000}%
\pgfsetfillcolor{currentfill}%
\pgfsetlinewidth{0.752812pt}%
\definecolor{currentstroke}{rgb}{0.000000,0.000000,0.000000}%
\pgfsetstrokecolor{currentstroke}%
\pgfsetdash{}{0pt}%
\pgfsys@defobject{currentmarker}{\pgfqpoint{-0.013889in}{0.000000in}}{\pgfqpoint{-0.000000in}{0.000000in}}{%
\pgfpathmoveto{\pgfqpoint{-0.000000in}{0.000000in}}%
\pgfpathlineto{\pgfqpoint{-0.013889in}{0.000000in}}%
\pgfusepath{stroke,fill}%
}%
\begin{pgfscope}%
\pgfsys@transformshift{0.781250in}{2.593889in}%
\pgfsys@useobject{currentmarker}{}%
\end{pgfscope}%
\end{pgfscope}%
\begin{pgfscope}%
\definecolor{textcolor}{rgb}{0.000000,0.000000,0.000000}%
\pgfsetstrokecolor{textcolor}%
\pgfsetfillcolor{textcolor}%
\pgftext[x=0.532837in, y=2.543962in, left, base]{\color{textcolor}{\ifdefined\pdftexversion\else\setmainfont{NanumMyeongjo}\rmfamily\fi\fontsize{9.000000}{10.800000}\selectfont\catcode`\^=\active\def^{\ifmmode\sp\else\^{}\fi}\catcode`\%=\active\def%{\%}9만}}%
\end{pgfscope}%
\begin{pgfscope}%
\pgfpathrectangle{\pgfqpoint{0.781250in}{0.638889in}}{\pgfqpoint{4.218750in}{2.172222in}}%
\pgfusepath{clip}%
\pgfsetbuttcap%
\pgfsetroundjoin%
\pgfsetlinewidth{0.602250pt}%
\definecolor{currentstroke}{rgb}{0.690196,0.690196,0.690196}%
\pgfsetstrokecolor{currentstroke}%
\pgfsetstrokeopacity{0.450000}%
\pgfsetdash{{2.220000pt}{0.960000pt}}{0.000000pt}%
\pgfpathmoveto{\pgfqpoint{0.781250in}{2.811111in}}%
\pgfpathlineto{\pgfqpoint{5.000000in}{2.811111in}}%
\pgfusepath{stroke}%
\end{pgfscope}%
\begin{pgfscope}%
\pgfsetbuttcap%
\pgfsetroundjoin%
\definecolor{currentfill}{rgb}{0.000000,0.000000,0.000000}%
\pgfsetfillcolor{currentfill}%
\pgfsetlinewidth{0.752812pt}%
\definecolor{currentstroke}{rgb}{0.000000,0.000000,0.000000}%
\pgfsetstrokecolor{currentstroke}%
\pgfsetdash{}{0pt}%
\pgfsys@defobject{currentmarker}{\pgfqpoint{-0.013889in}{0.000000in}}{\pgfqpoint{-0.000000in}{0.000000in}}{%
\pgfpathmoveto{\pgfqpoint{-0.000000in}{0.000000in}}%
\pgfpathlineto{\pgfqpoint{-0.013889in}{0.000000in}}%
\pgfusepath{stroke,fill}%
}%
\begin{pgfscope}%
\pgfsys@transformshift{0.781250in}{2.811111in}%
\pgfsys@useobject{currentmarker}{}%
\end{pgfscope}%
\end{pgfscope}%
\begin{pgfscope}%
\definecolor{textcolor}{rgb}{0.000000,0.000000,0.000000}%
\pgfsetstrokecolor{textcolor}%
\pgfsetfillcolor{textcolor}%
\pgftext[x=0.465698in, y=2.761184in, left, base]{\color{textcolor}{\ifdefined\pdftexversion\else\setmainfont{NanumMyeongjo}\rmfamily\fi\fontsize{9.000000}{10.800000}\selectfont\catcode`\^=\active\def^{\ifmmode\sp\else\^{}\fi}\catcode`\%=\active\def%{\%}10만}}%
\end{pgfscope}%
\begin{pgfscope}%
\pgfsetrectcap%
\pgfsetmiterjoin%
\pgfsetlinewidth{0.752812pt}%
\definecolor{currentstroke}{rgb}{0.000000,0.000000,0.000000}%
\pgfsetstrokecolor{currentstroke}%
\pgfsetdash{}{0pt}%
\pgfpathmoveto{\pgfqpoint{0.781250in}{0.638889in}}%
\pgfpathlineto{\pgfqpoint{0.781250in}{2.811111in}}%
\pgfusepath{stroke}%
\end{pgfscope}%
\begin{pgfscope}%
\pgfsetrectcap%
\pgfsetmiterjoin%
\pgfsetlinewidth{0.752812pt}%
\definecolor{currentstroke}{rgb}{0.000000,0.000000,0.000000}%
\pgfsetstrokecolor{currentstroke}%
\pgfsetdash{}{0pt}%
\pgfpathmoveto{\pgfqpoint{0.781250in}{0.638889in}}%
\pgfpathlineto{\pgfqpoint{5.000000in}{0.638889in}}%
\pgfusepath{stroke}%
\end{pgfscope}%
\begin{pgfscope}%
\pgfpathrectangle{\pgfqpoint{0.781250in}{0.638889in}}{\pgfqpoint{4.218750in}{2.172222in}}%
\pgfusepath{clip}%
\pgfsetbuttcap%
\pgfsetmiterjoin%
\definecolor{currentfill}{rgb}{0.337255,0.713725,0.627451}%
\pgfsetfillcolor{currentfill}%
\pgfsetlinewidth{1.003750pt}%
\definecolor{currentstroke}{rgb}{0.266667,0.266667,0.266667}%
\pgfsetstrokecolor{currentstroke}%
\pgfsetdash{}{0pt}%
\pgfpathmoveto{\pgfqpoint{0.973011in}{0.638889in}}%
\pgfpathlineto{\pgfqpoint{1.223914in}{0.638889in}}%
\pgfpathlineto{\pgfqpoint{1.223914in}{0.796245in}}%
\pgfpathlineto{\pgfqpoint{0.973011in}{0.796245in}}%
\pgfpathlineto{\pgfqpoint{0.973011in}{0.638889in}}%
\pgfpathclose%
\pgfusepath{stroke,fill}%
\end{pgfscope}%
\begin{pgfscope}%
\pgfpathrectangle{\pgfqpoint{0.781250in}{0.638889in}}{\pgfqpoint{4.218750in}{2.172222in}}%
\pgfusepath{clip}%
\pgfsetbuttcap%
\pgfsetmiterjoin%
\definecolor{currentfill}{rgb}{0.337255,0.713725,0.627451}%
\pgfsetfillcolor{currentfill}%
\pgfsetlinewidth{1.003750pt}%
\definecolor{currentstroke}{rgb}{0.266667,0.266667,0.266667}%
\pgfsetstrokecolor{currentstroke}%
\pgfsetdash{}{0pt}%
\pgfpathmoveto{\pgfqpoint{1.331444in}{0.638889in}}%
\pgfpathlineto{\pgfqpoint{1.582347in}{0.638889in}}%
\pgfpathlineto{\pgfqpoint{1.582347in}{0.751823in}}%
\pgfpathlineto{\pgfqpoint{1.331444in}{0.751823in}}%
\pgfpathlineto{\pgfqpoint{1.331444in}{0.638889in}}%
\pgfpathclose%
\pgfusepath{stroke,fill}%
\end{pgfscope}%
\begin{pgfscope}%
\pgfpathrectangle{\pgfqpoint{0.781250in}{0.638889in}}{\pgfqpoint{4.218750in}{2.172222in}}%
\pgfusepath{clip}%
\pgfsetbuttcap%
\pgfsetmiterjoin%
\definecolor{currentfill}{rgb}{0.337255,0.713725,0.627451}%
\pgfsetfillcolor{currentfill}%
\pgfsetlinewidth{1.003750pt}%
\definecolor{currentstroke}{rgb}{0.266667,0.266667,0.266667}%
\pgfsetstrokecolor{currentstroke}%
\pgfsetdash{}{0pt}%
\pgfpathmoveto{\pgfqpoint{1.689876in}{0.638889in}}%
\pgfpathlineto{\pgfqpoint{1.940779in}{0.638889in}}%
\pgfpathlineto{\pgfqpoint{1.940779in}{0.746436in}}%
\pgfpathlineto{\pgfqpoint{1.689876in}{0.746436in}}%
\pgfpathlineto{\pgfqpoint{1.689876in}{0.638889in}}%
\pgfpathclose%
\pgfusepath{stroke,fill}%
\end{pgfscope}%
\begin{pgfscope}%
\pgfpathrectangle{\pgfqpoint{0.781250in}{0.638889in}}{\pgfqpoint{4.218750in}{2.172222in}}%
\pgfusepath{clip}%
\pgfsetbuttcap%
\pgfsetmiterjoin%
\definecolor{currentfill}{rgb}{0.337255,0.713725,0.627451}%
\pgfsetfillcolor{currentfill}%
\pgfsetlinewidth{1.003750pt}%
\definecolor{currentstroke}{rgb}{0.266667,0.266667,0.266667}%
\pgfsetstrokecolor{currentstroke}%
\pgfsetdash{}{0pt}%
\pgfpathmoveto{\pgfqpoint{2.048309in}{0.638889in}}%
\pgfpathlineto{\pgfqpoint{2.299211in}{0.638889in}}%
\pgfpathlineto{\pgfqpoint{2.299211in}{0.770938in}}%
\pgfpathlineto{\pgfqpoint{2.048309in}{0.770938in}}%
\pgfpathlineto{\pgfqpoint{2.048309in}{0.638889in}}%
\pgfpathclose%
\pgfusepath{stroke,fill}%
\end{pgfscope}%
\begin{pgfscope}%
\pgfpathrectangle{\pgfqpoint{0.781250in}{0.638889in}}{\pgfqpoint{4.218750in}{2.172222in}}%
\pgfusepath{clip}%
\pgfsetbuttcap%
\pgfsetmiterjoin%
\definecolor{currentfill}{rgb}{0.337255,0.713725,0.627451}%
\pgfsetfillcolor{currentfill}%
\pgfsetlinewidth{1.003750pt}%
\definecolor{currentstroke}{rgb}{0.266667,0.266667,0.266667}%
\pgfsetstrokecolor{currentstroke}%
\pgfsetdash{}{0pt}%
\pgfpathmoveto{\pgfqpoint{2.406741in}{0.638889in}}%
\pgfpathlineto{\pgfqpoint{2.657644in}{0.638889in}}%
\pgfpathlineto{\pgfqpoint{2.657644in}{0.831847in}}%
\pgfpathlineto{\pgfqpoint{2.406741in}{0.831847in}}%
\pgfpathlineto{\pgfqpoint{2.406741in}{0.638889in}}%
\pgfpathclose%
\pgfusepath{stroke,fill}%
\end{pgfscope}%
\begin{pgfscope}%
\pgfpathrectangle{\pgfqpoint{0.781250in}{0.638889in}}{\pgfqpoint{4.218750in}{2.172222in}}%
\pgfusepath{clip}%
\pgfsetbuttcap%
\pgfsetmiterjoin%
\definecolor{currentfill}{rgb}{0.337255,0.713725,0.627451}%
\pgfsetfillcolor{currentfill}%
\pgfsetlinewidth{1.003750pt}%
\definecolor{currentstroke}{rgb}{0.266667,0.266667,0.266667}%
\pgfsetstrokecolor{currentstroke}%
\pgfsetdash{}{0pt}%
\pgfpathmoveto{\pgfqpoint{2.765174in}{0.638889in}}%
\pgfpathlineto{\pgfqpoint{3.016076in}{0.638889in}}%
\pgfpathlineto{\pgfqpoint{3.016076in}{0.909048in}}%
\pgfpathlineto{\pgfqpoint{2.765174in}{0.909048in}}%
\pgfpathlineto{\pgfqpoint{2.765174in}{0.638889in}}%
\pgfpathclose%
\pgfusepath{stroke,fill}%
\end{pgfscope}%
\begin{pgfscope}%
\pgfpathrectangle{\pgfqpoint{0.781250in}{0.638889in}}{\pgfqpoint{4.218750in}{2.172222in}}%
\pgfusepath{clip}%
\pgfsetbuttcap%
\pgfsetmiterjoin%
\definecolor{currentfill}{rgb}{0.337255,0.713725,0.627451}%
\pgfsetfillcolor{currentfill}%
\pgfsetlinewidth{1.003750pt}%
\definecolor{currentstroke}{rgb}{0.266667,0.266667,0.266667}%
\pgfsetstrokecolor{currentstroke}%
\pgfsetdash{}{0pt}%
\pgfpathmoveto{\pgfqpoint{3.123606in}{0.638889in}}%
\pgfpathlineto{\pgfqpoint{3.374509in}{0.638889in}}%
\pgfpathlineto{\pgfqpoint{3.374509in}{0.892474in}}%
\pgfpathlineto{\pgfqpoint{3.123606in}{0.892474in}}%
\pgfpathlineto{\pgfqpoint{3.123606in}{0.638889in}}%
\pgfpathclose%
\pgfusepath{stroke,fill}%
\end{pgfscope}%
\begin{pgfscope}%
\pgfpathrectangle{\pgfqpoint{0.781250in}{0.638889in}}{\pgfqpoint{4.218750in}{2.172222in}}%
\pgfusepath{clip}%
\pgfsetbuttcap%
\pgfsetmiterjoin%
\definecolor{currentfill}{rgb}{0.337255,0.713725,0.627451}%
\pgfsetfillcolor{currentfill}%
\pgfsetlinewidth{1.003750pt}%
\definecolor{currentstroke}{rgb}{0.266667,0.266667,0.266667}%
\pgfsetstrokecolor{currentstroke}%
\pgfsetdash{}{0pt}%
\pgfpathmoveto{\pgfqpoint{3.482039in}{0.638889in}}%
\pgfpathlineto{\pgfqpoint{3.732941in}{0.638889in}}%
\pgfpathlineto{\pgfqpoint{3.732941in}{0.860499in}}%
\pgfpathlineto{\pgfqpoint{3.482039in}{0.860499in}}%
\pgfpathlineto{\pgfqpoint{3.482039in}{0.638889in}}%
\pgfpathclose%
\pgfusepath{stroke,fill}%
\end{pgfscope}%
\begin{pgfscope}%
\pgfpathrectangle{\pgfqpoint{0.781250in}{0.638889in}}{\pgfqpoint{4.218750in}{2.172222in}}%
\pgfusepath{clip}%
\pgfsetbuttcap%
\pgfsetmiterjoin%
\definecolor{currentfill}{rgb}{0.337255,0.713725,0.627451}%
\pgfsetfillcolor{currentfill}%
\pgfsetlinewidth{1.003750pt}%
\definecolor{currentstroke}{rgb}{0.266667,0.266667,0.266667}%
\pgfsetstrokecolor{currentstroke}%
\pgfsetdash{}{0pt}%
\pgfpathmoveto{\pgfqpoint{3.840471in}{0.638889in}}%
\pgfpathlineto{\pgfqpoint{4.091374in}{0.638889in}}%
\pgfpathlineto{\pgfqpoint{4.091374in}{0.931031in}}%
\pgfpathlineto{\pgfqpoint{3.840471in}{0.931031in}}%
\pgfpathlineto{\pgfqpoint{3.840471in}{0.638889in}}%
\pgfpathclose%
\pgfusepath{stroke,fill}%
\end{pgfscope}%
\begin{pgfscope}%
\pgfpathrectangle{\pgfqpoint{0.781250in}{0.638889in}}{\pgfqpoint{4.218750in}{2.172222in}}%
\pgfusepath{clip}%
\pgfsetbuttcap%
\pgfsetmiterjoin%
\definecolor{currentfill}{rgb}{0.337255,0.713725,0.627451}%
\pgfsetfillcolor{currentfill}%
\pgfsetlinewidth{1.003750pt}%
\definecolor{currentstroke}{rgb}{0.266667,0.266667,0.266667}%
\pgfsetstrokecolor{currentstroke}%
\pgfsetdash{}{0pt}%
\pgfpathmoveto{\pgfqpoint{4.198903in}{0.638889in}}%
\pgfpathlineto{\pgfqpoint{4.449806in}{0.638889in}}%
\pgfpathlineto{\pgfqpoint{4.449806in}{1.020874in}}%
\pgfpathlineto{\pgfqpoint{4.198903in}{1.020874in}}%
\pgfpathlineto{\pgfqpoint{4.198903in}{0.638889in}}%
\pgfpathclose%
\pgfusepath{stroke,fill}%
\end{pgfscope}%
\begin{pgfscope}%
\pgfpathrectangle{\pgfqpoint{0.781250in}{0.638889in}}{\pgfqpoint{4.218750in}{2.172222in}}%
\pgfusepath{clip}%
\pgfsetbuttcap%
\pgfsetmiterjoin%
\definecolor{currentfill}{rgb}{0.337255,0.713725,0.627451}%
\pgfsetfillcolor{currentfill}%
\pgfsetlinewidth{1.003750pt}%
\definecolor{currentstroke}{rgb}{0.266667,0.266667,0.266667}%
\pgfsetstrokecolor{currentstroke}%
\pgfsetdash{}{0pt}%
\pgfpathmoveto{\pgfqpoint{4.557336in}{0.638889in}}%
\pgfpathlineto{\pgfqpoint{4.808239in}{0.638889in}}%
\pgfpathlineto{\pgfqpoint{4.808239in}{1.086823in}}%
\pgfpathlineto{\pgfqpoint{4.557336in}{1.086823in}}%
\pgfpathlineto{\pgfqpoint{4.557336in}{0.638889in}}%
\pgfpathclose%
\pgfusepath{stroke,fill}%
\end{pgfscope}%
\begin{pgfscope}%
\pgfpathrectangle{\pgfqpoint{0.781250in}{0.638889in}}{\pgfqpoint{4.218750in}{2.172222in}}%
\pgfusepath{clip}%
\pgfsetbuttcap%
\pgfsetmiterjoin%
\definecolor{currentfill}{rgb}{0.235294,0.490196,0.764706}%
\pgfsetfillcolor{currentfill}%
\pgfsetlinewidth{1.003750pt}%
\definecolor{currentstroke}{rgb}{0.266667,0.266667,0.266667}%
\pgfsetstrokecolor{currentstroke}%
\pgfsetdash{}{0pt}%
\pgfpathmoveto{\pgfqpoint{0.973011in}{0.796245in}}%
\pgfpathlineto{\pgfqpoint{1.223914in}{0.796245in}}%
\pgfpathlineto{\pgfqpoint{1.223914in}{1.126553in}}%
\pgfpathlineto{\pgfqpoint{0.973011in}{1.126553in}}%
\pgfpathlineto{\pgfqpoint{0.973011in}{0.796245in}}%
\pgfpathclose%
\pgfusepath{stroke,fill}%
\end{pgfscope}%
\begin{pgfscope}%
\pgfpathrectangle{\pgfqpoint{0.781250in}{0.638889in}}{\pgfqpoint{4.218750in}{2.172222in}}%
\pgfusepath{clip}%
\pgfsetbuttcap%
\pgfsetmiterjoin%
\definecolor{currentfill}{rgb}{0.235294,0.490196,0.764706}%
\pgfsetfillcolor{currentfill}%
\pgfsetlinewidth{1.003750pt}%
\definecolor{currentstroke}{rgb}{0.266667,0.266667,0.266667}%
\pgfsetstrokecolor{currentstroke}%
\pgfsetdash{}{0pt}%
\pgfpathmoveto{\pgfqpoint{1.331444in}{0.751823in}}%
\pgfpathlineto{\pgfqpoint{1.582347in}{0.751823in}}%
\pgfpathlineto{\pgfqpoint{1.582347in}{1.003735in}}%
\pgfpathlineto{\pgfqpoint{1.331444in}{1.003735in}}%
\pgfpathlineto{\pgfqpoint{1.331444in}{0.751823in}}%
\pgfpathclose%
\pgfusepath{stroke,fill}%
\end{pgfscope}%
\begin{pgfscope}%
\pgfpathrectangle{\pgfqpoint{0.781250in}{0.638889in}}{\pgfqpoint{4.218750in}{2.172222in}}%
\pgfusepath{clip}%
\pgfsetbuttcap%
\pgfsetmiterjoin%
\definecolor{currentfill}{rgb}{0.235294,0.490196,0.764706}%
\pgfsetfillcolor{currentfill}%
\pgfsetlinewidth{1.003750pt}%
\definecolor{currentstroke}{rgb}{0.266667,0.266667,0.266667}%
\pgfsetstrokecolor{currentstroke}%
\pgfsetdash{}{0pt}%
\pgfpathmoveto{\pgfqpoint{1.689876in}{0.746436in}}%
\pgfpathlineto{\pgfqpoint{1.940779in}{0.746436in}}%
\pgfpathlineto{\pgfqpoint{1.940779in}{0.961790in}}%
\pgfpathlineto{\pgfqpoint{1.689876in}{0.961790in}}%
\pgfpathlineto{\pgfqpoint{1.689876in}{0.746436in}}%
\pgfpathclose%
\pgfusepath{stroke,fill}%
\end{pgfscope}%
\begin{pgfscope}%
\pgfpathrectangle{\pgfqpoint{0.781250in}{0.638889in}}{\pgfqpoint{4.218750in}{2.172222in}}%
\pgfusepath{clip}%
\pgfsetbuttcap%
\pgfsetmiterjoin%
\definecolor{currentfill}{rgb}{0.235294,0.490196,0.764706}%
\pgfsetfillcolor{currentfill}%
\pgfsetlinewidth{1.003750pt}%
\definecolor{currentstroke}{rgb}{0.266667,0.266667,0.266667}%
\pgfsetstrokecolor{currentstroke}%
\pgfsetdash{}{0pt}%
\pgfpathmoveto{\pgfqpoint{2.048309in}{0.770938in}}%
\pgfpathlineto{\pgfqpoint{2.299211in}{0.770938in}}%
\pgfpathlineto{\pgfqpoint{2.299211in}{0.984489in}}%
\pgfpathlineto{\pgfqpoint{2.048309in}{0.984489in}}%
\pgfpathlineto{\pgfqpoint{2.048309in}{0.770938in}}%
\pgfpathclose%
\pgfusepath{stroke,fill}%
\end{pgfscope}%
\begin{pgfscope}%
\pgfpathrectangle{\pgfqpoint{0.781250in}{0.638889in}}{\pgfqpoint{4.218750in}{2.172222in}}%
\pgfusepath{clip}%
\pgfsetbuttcap%
\pgfsetmiterjoin%
\definecolor{currentfill}{rgb}{0.235294,0.490196,0.764706}%
\pgfsetfillcolor{currentfill}%
\pgfsetlinewidth{1.003750pt}%
\definecolor{currentstroke}{rgb}{0.266667,0.266667,0.266667}%
\pgfsetstrokecolor{currentstroke}%
\pgfsetdash{}{0pt}%
\pgfpathmoveto{\pgfqpoint{2.406741in}{0.831847in}}%
\pgfpathlineto{\pgfqpoint{2.657644in}{0.831847in}}%
\pgfpathlineto{\pgfqpoint{2.657644in}{1.060843in}}%
\pgfpathlineto{\pgfqpoint{2.406741in}{1.060843in}}%
\pgfpathlineto{\pgfqpoint{2.406741in}{0.831847in}}%
\pgfpathclose%
\pgfusepath{stroke,fill}%
\end{pgfscope}%
\begin{pgfscope}%
\pgfpathrectangle{\pgfqpoint{0.781250in}{0.638889in}}{\pgfqpoint{4.218750in}{2.172222in}}%
\pgfusepath{clip}%
\pgfsetbuttcap%
\pgfsetmiterjoin%
\definecolor{currentfill}{rgb}{0.235294,0.490196,0.764706}%
\pgfsetfillcolor{currentfill}%
\pgfsetlinewidth{1.003750pt}%
\definecolor{currentstroke}{rgb}{0.266667,0.266667,0.266667}%
\pgfsetstrokecolor{currentstroke}%
\pgfsetdash{}{0pt}%
\pgfpathmoveto{\pgfqpoint{2.765174in}{0.909048in}}%
\pgfpathlineto{\pgfqpoint{3.016076in}{0.909048in}}%
\pgfpathlineto{\pgfqpoint{3.016076in}{1.142823in}}%
\pgfpathlineto{\pgfqpoint{2.765174in}{1.142823in}}%
\pgfpathlineto{\pgfqpoint{2.765174in}{0.909048in}}%
\pgfpathclose%
\pgfusepath{stroke,fill}%
\end{pgfscope}%
\begin{pgfscope}%
\pgfpathrectangle{\pgfqpoint{0.781250in}{0.638889in}}{\pgfqpoint{4.218750in}{2.172222in}}%
\pgfusepath{clip}%
\pgfsetbuttcap%
\pgfsetmiterjoin%
\definecolor{currentfill}{rgb}{0.235294,0.490196,0.764706}%
\pgfsetfillcolor{currentfill}%
\pgfsetlinewidth{1.003750pt}%
\definecolor{currentstroke}{rgb}{0.266667,0.266667,0.266667}%
\pgfsetstrokecolor{currentstroke}%
\pgfsetdash{}{0pt}%
\pgfpathmoveto{\pgfqpoint{3.123606in}{0.892474in}}%
\pgfpathlineto{\pgfqpoint{3.374509in}{0.892474in}}%
\pgfpathlineto{\pgfqpoint{3.374509in}{1.112260in}}%
\pgfpathlineto{\pgfqpoint{3.123606in}{1.112260in}}%
\pgfpathlineto{\pgfqpoint{3.123606in}{0.892474in}}%
\pgfpathclose%
\pgfusepath{stroke,fill}%
\end{pgfscope}%
\begin{pgfscope}%
\pgfpathrectangle{\pgfqpoint{0.781250in}{0.638889in}}{\pgfqpoint{4.218750in}{2.172222in}}%
\pgfusepath{clip}%
\pgfsetbuttcap%
\pgfsetmiterjoin%
\definecolor{currentfill}{rgb}{0.235294,0.490196,0.764706}%
\pgfsetfillcolor{currentfill}%
\pgfsetlinewidth{1.003750pt}%
\definecolor{currentstroke}{rgb}{0.266667,0.266667,0.266667}%
\pgfsetstrokecolor{currentstroke}%
\pgfsetdash{}{0pt}%
\pgfpathmoveto{\pgfqpoint{3.482039in}{0.860499in}}%
\pgfpathlineto{\pgfqpoint{3.732941in}{0.860499in}}%
\pgfpathlineto{\pgfqpoint{3.732941in}{1.098423in}}%
\pgfpathlineto{\pgfqpoint{3.482039in}{1.098423in}}%
\pgfpathlineto{\pgfqpoint{3.482039in}{0.860499in}}%
\pgfpathclose%
\pgfusepath{stroke,fill}%
\end{pgfscope}%
\begin{pgfscope}%
\pgfpathrectangle{\pgfqpoint{0.781250in}{0.638889in}}{\pgfqpoint{4.218750in}{2.172222in}}%
\pgfusepath{clip}%
\pgfsetbuttcap%
\pgfsetmiterjoin%
\definecolor{currentfill}{rgb}{0.235294,0.490196,0.764706}%
\pgfsetfillcolor{currentfill}%
\pgfsetlinewidth{1.003750pt}%
\definecolor{currentstroke}{rgb}{0.266667,0.266667,0.266667}%
\pgfsetstrokecolor{currentstroke}%
\pgfsetdash{}{0pt}%
\pgfpathmoveto{\pgfqpoint{3.840471in}{0.931031in}}%
\pgfpathlineto{\pgfqpoint{4.091374in}{0.931031in}}%
\pgfpathlineto{\pgfqpoint{4.091374in}{1.210944in}}%
\pgfpathlineto{\pgfqpoint{3.840471in}{1.210944in}}%
\pgfpathlineto{\pgfqpoint{3.840471in}{0.931031in}}%
\pgfpathclose%
\pgfusepath{stroke,fill}%
\end{pgfscope}%
\begin{pgfscope}%
\pgfpathrectangle{\pgfqpoint{0.781250in}{0.638889in}}{\pgfqpoint{4.218750in}{2.172222in}}%
\pgfusepath{clip}%
\pgfsetbuttcap%
\pgfsetmiterjoin%
\definecolor{currentfill}{rgb}{0.235294,0.490196,0.764706}%
\pgfsetfillcolor{currentfill}%
\pgfsetlinewidth{1.003750pt}%
\definecolor{currentstroke}{rgb}{0.266667,0.266667,0.266667}%
\pgfsetstrokecolor{currentstroke}%
\pgfsetdash{}{0pt}%
\pgfpathmoveto{\pgfqpoint{4.198903in}{1.020874in}}%
\pgfpathlineto{\pgfqpoint{4.449806in}{1.020874in}}%
\pgfpathlineto{\pgfqpoint{4.449806in}{1.282909in}}%
\pgfpathlineto{\pgfqpoint{4.198903in}{1.282909in}}%
\pgfpathlineto{\pgfqpoint{4.198903in}{1.020874in}}%
\pgfpathclose%
\pgfusepath{stroke,fill}%
\end{pgfscope}%
\begin{pgfscope}%
\pgfpathrectangle{\pgfqpoint{0.781250in}{0.638889in}}{\pgfqpoint{4.218750in}{2.172222in}}%
\pgfusepath{clip}%
\pgfsetbuttcap%
\pgfsetmiterjoin%
\definecolor{currentfill}{rgb}{0.235294,0.490196,0.764706}%
\pgfsetfillcolor{currentfill}%
\pgfsetlinewidth{1.003750pt}%
\definecolor{currentstroke}{rgb}{0.266667,0.266667,0.266667}%
\pgfsetstrokecolor{currentstroke}%
\pgfsetdash{}{0pt}%
\pgfpathmoveto{\pgfqpoint{4.557336in}{1.086823in}}%
\pgfpathlineto{\pgfqpoint{4.808239in}{1.086823in}}%
\pgfpathlineto{\pgfqpoint{4.808239in}{1.372622in}}%
\pgfpathlineto{\pgfqpoint{4.557336in}{1.372622in}}%
\pgfpathlineto{\pgfqpoint{4.557336in}{1.086823in}}%
\pgfpathclose%
\pgfusepath{stroke,fill}%
\end{pgfscope}%
\begin{pgfscope}%
\pgfpathrectangle{\pgfqpoint{0.781250in}{0.638889in}}{\pgfqpoint{4.218750in}{2.172222in}}%
\pgfusepath{clip}%
\pgfsetbuttcap%
\pgfsetmiterjoin%
\definecolor{currentfill}{rgb}{0.725490,0.486275,0.164706}%
\pgfsetfillcolor{currentfill}%
\pgfsetlinewidth{1.003750pt}%
\definecolor{currentstroke}{rgb}{0.266667,0.266667,0.266667}%
\pgfsetstrokecolor{currentstroke}%
\pgfsetdash{}{0pt}%
\pgfpathmoveto{\pgfqpoint{0.973011in}{1.126553in}}%
\pgfpathlineto{\pgfqpoint{1.223914in}{1.126553in}}%
\pgfpathlineto{\pgfqpoint{1.223914in}{1.472914in}}%
\pgfpathlineto{\pgfqpoint{0.973011in}{1.472914in}}%
\pgfpathlineto{\pgfqpoint{0.973011in}{1.126553in}}%
\pgfpathclose%
\pgfusepath{stroke,fill}%
\end{pgfscope}%
\begin{pgfscope}%
\pgfpathrectangle{\pgfqpoint{0.781250in}{0.638889in}}{\pgfqpoint{4.218750in}{2.172222in}}%
\pgfusepath{clip}%
\pgfsetbuttcap%
\pgfsetmiterjoin%
\definecolor{currentfill}{rgb}{0.725490,0.486275,0.164706}%
\pgfsetfillcolor{currentfill}%
\pgfsetlinewidth{1.003750pt}%
\definecolor{currentstroke}{rgb}{0.266667,0.266667,0.266667}%
\pgfsetstrokecolor{currentstroke}%
\pgfsetdash{}{0pt}%
\pgfpathmoveto{\pgfqpoint{1.331444in}{1.003735in}}%
\pgfpathlineto{\pgfqpoint{1.582347in}{1.003735in}}%
\pgfpathlineto{\pgfqpoint{1.582347in}{1.294118in}}%
\pgfpathlineto{\pgfqpoint{1.331444in}{1.294118in}}%
\pgfpathlineto{\pgfqpoint{1.331444in}{1.003735in}}%
\pgfpathclose%
\pgfusepath{stroke,fill}%
\end{pgfscope}%
\begin{pgfscope}%
\pgfpathrectangle{\pgfqpoint{0.781250in}{0.638889in}}{\pgfqpoint{4.218750in}{2.172222in}}%
\pgfusepath{clip}%
\pgfsetbuttcap%
\pgfsetmiterjoin%
\definecolor{currentfill}{rgb}{0.725490,0.486275,0.164706}%
\pgfsetfillcolor{currentfill}%
\pgfsetlinewidth{1.003750pt}%
\definecolor{currentstroke}{rgb}{0.266667,0.266667,0.266667}%
\pgfsetstrokecolor{currentstroke}%
\pgfsetdash{}{0pt}%
\pgfpathmoveto{\pgfqpoint{1.689876in}{0.961790in}}%
\pgfpathlineto{\pgfqpoint{1.940779in}{0.961790in}}%
\pgfpathlineto{\pgfqpoint{1.940779in}{1.210270in}}%
\pgfpathlineto{\pgfqpoint{1.689876in}{1.210270in}}%
\pgfpathlineto{\pgfqpoint{1.689876in}{0.961790in}}%
\pgfpathclose%
\pgfusepath{stroke,fill}%
\end{pgfscope}%
\begin{pgfscope}%
\pgfpathrectangle{\pgfqpoint{0.781250in}{0.638889in}}{\pgfqpoint{4.218750in}{2.172222in}}%
\pgfusepath{clip}%
\pgfsetbuttcap%
\pgfsetmiterjoin%
\definecolor{currentfill}{rgb}{0.725490,0.486275,0.164706}%
\pgfsetfillcolor{currentfill}%
\pgfsetlinewidth{1.003750pt}%
\definecolor{currentstroke}{rgb}{0.266667,0.266667,0.266667}%
\pgfsetstrokecolor{currentstroke}%
\pgfsetdash{}{0pt}%
\pgfpathmoveto{\pgfqpoint{2.048309in}{0.984489in}}%
\pgfpathlineto{\pgfqpoint{2.299211in}{0.984489in}}%
\pgfpathlineto{\pgfqpoint{2.299211in}{1.209097in}}%
\pgfpathlineto{\pgfqpoint{2.048309in}{1.209097in}}%
\pgfpathlineto{\pgfqpoint{2.048309in}{0.984489in}}%
\pgfpathclose%
\pgfusepath{stroke,fill}%
\end{pgfscope}%
\begin{pgfscope}%
\pgfpathrectangle{\pgfqpoint{0.781250in}{0.638889in}}{\pgfqpoint{4.218750in}{2.172222in}}%
\pgfusepath{clip}%
\pgfsetbuttcap%
\pgfsetmiterjoin%
\definecolor{currentfill}{rgb}{0.725490,0.486275,0.164706}%
\pgfsetfillcolor{currentfill}%
\pgfsetlinewidth{1.003750pt}%
\definecolor{currentstroke}{rgb}{0.266667,0.266667,0.266667}%
\pgfsetstrokecolor{currentstroke}%
\pgfsetdash{}{0pt}%
\pgfpathmoveto{\pgfqpoint{2.406741in}{1.060843in}}%
\pgfpathlineto{\pgfqpoint{2.657644in}{1.060843in}}%
\pgfpathlineto{\pgfqpoint{2.657644in}{1.308129in}}%
\pgfpathlineto{\pgfqpoint{2.406741in}{1.308129in}}%
\pgfpathlineto{\pgfqpoint{2.406741in}{1.060843in}}%
\pgfpathclose%
\pgfusepath{stroke,fill}%
\end{pgfscope}%
\begin{pgfscope}%
\pgfpathrectangle{\pgfqpoint{0.781250in}{0.638889in}}{\pgfqpoint{4.218750in}{2.172222in}}%
\pgfusepath{clip}%
\pgfsetbuttcap%
\pgfsetmiterjoin%
\definecolor{currentfill}{rgb}{0.725490,0.486275,0.164706}%
\pgfsetfillcolor{currentfill}%
\pgfsetlinewidth{1.003750pt}%
\definecolor{currentstroke}{rgb}{0.266667,0.266667,0.266667}%
\pgfsetstrokecolor{currentstroke}%
\pgfsetdash{}{0pt}%
\pgfpathmoveto{\pgfqpoint{2.765174in}{1.142823in}}%
\pgfpathlineto{\pgfqpoint{3.016076in}{1.142823in}}%
\pgfpathlineto{\pgfqpoint{3.016076in}{1.407334in}}%
\pgfpathlineto{\pgfqpoint{2.765174in}{1.407334in}}%
\pgfpathlineto{\pgfqpoint{2.765174in}{1.142823in}}%
\pgfpathclose%
\pgfusepath{stroke,fill}%
\end{pgfscope}%
\begin{pgfscope}%
\pgfpathrectangle{\pgfqpoint{0.781250in}{0.638889in}}{\pgfqpoint{4.218750in}{2.172222in}}%
\pgfusepath{clip}%
\pgfsetbuttcap%
\pgfsetmiterjoin%
\definecolor{currentfill}{rgb}{0.725490,0.486275,0.164706}%
\pgfsetfillcolor{currentfill}%
\pgfsetlinewidth{1.003750pt}%
\definecolor{currentstroke}{rgb}{0.266667,0.266667,0.266667}%
\pgfsetstrokecolor{currentstroke}%
\pgfsetdash{}{0pt}%
\pgfpathmoveto{\pgfqpoint{3.123606in}{1.112260in}}%
\pgfpathlineto{\pgfqpoint{3.374509in}{1.112260in}}%
\pgfpathlineto{\pgfqpoint{3.374509in}{1.328157in}}%
\pgfpathlineto{\pgfqpoint{3.123606in}{1.328157in}}%
\pgfpathlineto{\pgfqpoint{3.123606in}{1.112260in}}%
\pgfpathclose%
\pgfusepath{stroke,fill}%
\end{pgfscope}%
\begin{pgfscope}%
\pgfpathrectangle{\pgfqpoint{0.781250in}{0.638889in}}{\pgfqpoint{4.218750in}{2.172222in}}%
\pgfusepath{clip}%
\pgfsetbuttcap%
\pgfsetmiterjoin%
\definecolor{currentfill}{rgb}{0.725490,0.486275,0.164706}%
\pgfsetfillcolor{currentfill}%
\pgfsetlinewidth{1.003750pt}%
\definecolor{currentstroke}{rgb}{0.266667,0.266667,0.266667}%
\pgfsetstrokecolor{currentstroke}%
\pgfsetdash{}{0pt}%
\pgfpathmoveto{\pgfqpoint{3.482039in}{1.098423in}}%
\pgfpathlineto{\pgfqpoint{3.732941in}{1.098423in}}%
\pgfpathlineto{\pgfqpoint{3.732941in}{1.302155in}}%
\pgfpathlineto{\pgfqpoint{3.482039in}{1.302155in}}%
\pgfpathlineto{\pgfqpoint{3.482039in}{1.098423in}}%
\pgfpathclose%
\pgfusepath{stroke,fill}%
\end{pgfscope}%
\begin{pgfscope}%
\pgfpathrectangle{\pgfqpoint{0.781250in}{0.638889in}}{\pgfqpoint{4.218750in}{2.172222in}}%
\pgfusepath{clip}%
\pgfsetbuttcap%
\pgfsetmiterjoin%
\definecolor{currentfill}{rgb}{0.725490,0.486275,0.164706}%
\pgfsetfillcolor{currentfill}%
\pgfsetlinewidth{1.003750pt}%
\definecolor{currentstroke}{rgb}{0.266667,0.266667,0.266667}%
\pgfsetstrokecolor{currentstroke}%
\pgfsetdash{}{0pt}%
\pgfpathmoveto{\pgfqpoint{3.840471in}{1.210944in}}%
\pgfpathlineto{\pgfqpoint{4.091374in}{1.210944in}}%
\pgfpathlineto{\pgfqpoint{4.091374in}{1.449280in}}%
\pgfpathlineto{\pgfqpoint{3.840471in}{1.449280in}}%
\pgfpathlineto{\pgfqpoint{3.840471in}{1.210944in}}%
\pgfpathclose%
\pgfusepath{stroke,fill}%
\end{pgfscope}%
\begin{pgfscope}%
\pgfpathrectangle{\pgfqpoint{0.781250in}{0.638889in}}{\pgfqpoint{4.218750in}{2.172222in}}%
\pgfusepath{clip}%
\pgfsetbuttcap%
\pgfsetmiterjoin%
\definecolor{currentfill}{rgb}{0.725490,0.486275,0.164706}%
\pgfsetfillcolor{currentfill}%
\pgfsetlinewidth{1.003750pt}%
\definecolor{currentstroke}{rgb}{0.266667,0.266667,0.266667}%
\pgfsetstrokecolor{currentstroke}%
\pgfsetdash{}{0pt}%
\pgfpathmoveto{\pgfqpoint{4.198903in}{1.282909in}}%
\pgfpathlineto{\pgfqpoint{4.449806in}{1.282909in}}%
\pgfpathlineto{\pgfqpoint{4.449806in}{1.514055in}}%
\pgfpathlineto{\pgfqpoint{4.198903in}{1.514055in}}%
\pgfpathlineto{\pgfqpoint{4.198903in}{1.282909in}}%
\pgfpathclose%
\pgfusepath{stroke,fill}%
\end{pgfscope}%
\begin{pgfscope}%
\pgfpathrectangle{\pgfqpoint{0.781250in}{0.638889in}}{\pgfqpoint{4.218750in}{2.172222in}}%
\pgfusepath{clip}%
\pgfsetbuttcap%
\pgfsetmiterjoin%
\definecolor{currentfill}{rgb}{0.725490,0.486275,0.164706}%
\pgfsetfillcolor{currentfill}%
\pgfsetlinewidth{1.003750pt}%
\definecolor{currentstroke}{rgb}{0.266667,0.266667,0.266667}%
\pgfsetstrokecolor{currentstroke}%
\pgfsetdash{}{0pt}%
\pgfpathmoveto{\pgfqpoint{4.557336in}{1.372622in}}%
\pgfpathlineto{\pgfqpoint{4.808239in}{1.372622in}}%
\pgfpathlineto{\pgfqpoint{4.808239in}{1.638893in}}%
\pgfpathlineto{\pgfqpoint{4.557336in}{1.638893in}}%
\pgfpathlineto{\pgfqpoint{4.557336in}{1.372622in}}%
\pgfpathclose%
\pgfusepath{stroke,fill}%
\end{pgfscope}%
\begin{pgfscope}%
\pgfpathrectangle{\pgfqpoint{0.781250in}{0.638889in}}{\pgfqpoint{4.218750in}{2.172222in}}%
\pgfusepath{clip}%
\pgfsetbuttcap%
\pgfsetmiterjoin%
\definecolor{currentfill}{rgb}{0.733333,0.321569,0.733333}%
\pgfsetfillcolor{currentfill}%
\pgfsetlinewidth{1.003750pt}%
\definecolor{currentstroke}{rgb}{0.266667,0.266667,0.266667}%
\pgfsetstrokecolor{currentstroke}%
\pgfsetdash{}{0pt}%
\pgfpathmoveto{\pgfqpoint{0.973011in}{1.472914in}}%
\pgfpathlineto{\pgfqpoint{1.223914in}{1.472914in}}%
\pgfpathlineto{\pgfqpoint{1.223914in}{1.651535in}}%
\pgfpathlineto{\pgfqpoint{0.973011in}{1.651535in}}%
\pgfpathlineto{\pgfqpoint{0.973011in}{1.472914in}}%
\pgfpathclose%
\pgfusepath{stroke,fill}%
\end{pgfscope}%
\begin{pgfscope}%
\pgfpathrectangle{\pgfqpoint{0.781250in}{0.638889in}}{\pgfqpoint{4.218750in}{2.172222in}}%
\pgfusepath{clip}%
\pgfsetbuttcap%
\pgfsetmiterjoin%
\definecolor{currentfill}{rgb}{0.733333,0.321569,0.733333}%
\pgfsetfillcolor{currentfill}%
\pgfsetlinewidth{1.003750pt}%
\definecolor{currentstroke}{rgb}{0.266667,0.266667,0.266667}%
\pgfsetstrokecolor{currentstroke}%
\pgfsetdash{}{0pt}%
\pgfpathmoveto{\pgfqpoint{1.331444in}{1.294118in}}%
\pgfpathlineto{\pgfqpoint{1.582347in}{1.294118in}}%
\pgfpathlineto{\pgfqpoint{1.582347in}{1.432575in}}%
\pgfpathlineto{\pgfqpoint{1.331444in}{1.432575in}}%
\pgfpathlineto{\pgfqpoint{1.331444in}{1.294118in}}%
\pgfpathclose%
\pgfusepath{stroke,fill}%
\end{pgfscope}%
\begin{pgfscope}%
\pgfpathrectangle{\pgfqpoint{0.781250in}{0.638889in}}{\pgfqpoint{4.218750in}{2.172222in}}%
\pgfusepath{clip}%
\pgfsetbuttcap%
\pgfsetmiterjoin%
\definecolor{currentfill}{rgb}{0.733333,0.321569,0.733333}%
\pgfsetfillcolor{currentfill}%
\pgfsetlinewidth{1.003750pt}%
\definecolor{currentstroke}{rgb}{0.266667,0.266667,0.266667}%
\pgfsetstrokecolor{currentstroke}%
\pgfsetdash{}{0pt}%
\pgfpathmoveto{\pgfqpoint{1.689876in}{1.210270in}}%
\pgfpathlineto{\pgfqpoint{1.940779in}{1.210270in}}%
\pgfpathlineto{\pgfqpoint{1.940779in}{1.324486in}}%
\pgfpathlineto{\pgfqpoint{1.689876in}{1.324486in}}%
\pgfpathlineto{\pgfqpoint{1.689876in}{1.210270in}}%
\pgfpathclose%
\pgfusepath{stroke,fill}%
\end{pgfscope}%
\begin{pgfscope}%
\pgfpathrectangle{\pgfqpoint{0.781250in}{0.638889in}}{\pgfqpoint{4.218750in}{2.172222in}}%
\pgfusepath{clip}%
\pgfsetbuttcap%
\pgfsetmiterjoin%
\definecolor{currentfill}{rgb}{0.733333,0.321569,0.733333}%
\pgfsetfillcolor{currentfill}%
\pgfsetlinewidth{1.003750pt}%
\definecolor{currentstroke}{rgb}{0.266667,0.266667,0.266667}%
\pgfsetstrokecolor{currentstroke}%
\pgfsetdash{}{0pt}%
\pgfpathmoveto{\pgfqpoint{2.048309in}{1.209097in}}%
\pgfpathlineto{\pgfqpoint{2.299211in}{1.209097in}}%
\pgfpathlineto{\pgfqpoint{2.299211in}{1.324942in}}%
\pgfpathlineto{\pgfqpoint{2.048309in}{1.324942in}}%
\pgfpathlineto{\pgfqpoint{2.048309in}{1.209097in}}%
\pgfpathclose%
\pgfusepath{stroke,fill}%
\end{pgfscope}%
\begin{pgfscope}%
\pgfpathrectangle{\pgfqpoint{0.781250in}{0.638889in}}{\pgfqpoint{4.218750in}{2.172222in}}%
\pgfusepath{clip}%
\pgfsetbuttcap%
\pgfsetmiterjoin%
\definecolor{currentfill}{rgb}{0.733333,0.321569,0.733333}%
\pgfsetfillcolor{currentfill}%
\pgfsetlinewidth{1.003750pt}%
\definecolor{currentstroke}{rgb}{0.266667,0.266667,0.266667}%
\pgfsetstrokecolor{currentstroke}%
\pgfsetdash{}{0pt}%
\pgfpathmoveto{\pgfqpoint{2.406741in}{1.308129in}}%
\pgfpathlineto{\pgfqpoint{2.657644in}{1.308129in}}%
\pgfpathlineto{\pgfqpoint{2.657644in}{1.437333in}}%
\pgfpathlineto{\pgfqpoint{2.406741in}{1.437333in}}%
\pgfpathlineto{\pgfqpoint{2.406741in}{1.308129in}}%
\pgfpathclose%
\pgfusepath{stroke,fill}%
\end{pgfscope}%
\begin{pgfscope}%
\pgfpathrectangle{\pgfqpoint{0.781250in}{0.638889in}}{\pgfqpoint{4.218750in}{2.172222in}}%
\pgfusepath{clip}%
\pgfsetbuttcap%
\pgfsetmiterjoin%
\definecolor{currentfill}{rgb}{0.733333,0.321569,0.733333}%
\pgfsetfillcolor{currentfill}%
\pgfsetlinewidth{1.003750pt}%
\definecolor{currentstroke}{rgb}{0.266667,0.266667,0.266667}%
\pgfsetstrokecolor{currentstroke}%
\pgfsetdash{}{0pt}%
\pgfpathmoveto{\pgfqpoint{2.765174in}{1.407334in}}%
\pgfpathlineto{\pgfqpoint{3.016076in}{1.407334in}}%
\pgfpathlineto{\pgfqpoint{3.016076in}{1.565190in}}%
\pgfpathlineto{\pgfqpoint{2.765174in}{1.565190in}}%
\pgfpathlineto{\pgfqpoint{2.765174in}{1.407334in}}%
\pgfpathclose%
\pgfusepath{stroke,fill}%
\end{pgfscope}%
\begin{pgfscope}%
\pgfpathrectangle{\pgfqpoint{0.781250in}{0.638889in}}{\pgfqpoint{4.218750in}{2.172222in}}%
\pgfusepath{clip}%
\pgfsetbuttcap%
\pgfsetmiterjoin%
\definecolor{currentfill}{rgb}{0.733333,0.321569,0.733333}%
\pgfsetfillcolor{currentfill}%
\pgfsetlinewidth{1.003750pt}%
\definecolor{currentstroke}{rgb}{0.266667,0.266667,0.266667}%
\pgfsetstrokecolor{currentstroke}%
\pgfsetdash{}{0pt}%
\pgfpathmoveto{\pgfqpoint{3.123606in}{1.328157in}}%
\pgfpathlineto{\pgfqpoint{3.374509in}{1.328157in}}%
\pgfpathlineto{\pgfqpoint{3.374509in}{1.477345in}}%
\pgfpathlineto{\pgfqpoint{3.123606in}{1.477345in}}%
\pgfpathlineto{\pgfqpoint{3.123606in}{1.328157in}}%
\pgfpathclose%
\pgfusepath{stroke,fill}%
\end{pgfscope}%
\begin{pgfscope}%
\pgfpathrectangle{\pgfqpoint{0.781250in}{0.638889in}}{\pgfqpoint{4.218750in}{2.172222in}}%
\pgfusepath{clip}%
\pgfsetbuttcap%
\pgfsetmiterjoin%
\definecolor{currentfill}{rgb}{0.733333,0.321569,0.733333}%
\pgfsetfillcolor{currentfill}%
\pgfsetlinewidth{1.003750pt}%
\definecolor{currentstroke}{rgb}{0.266667,0.266667,0.266667}%
\pgfsetstrokecolor{currentstroke}%
\pgfsetdash{}{0pt}%
\pgfpathmoveto{\pgfqpoint{3.482039in}{1.302155in}}%
\pgfpathlineto{\pgfqpoint{3.732941in}{1.302155in}}%
\pgfpathlineto{\pgfqpoint{3.732941in}{1.468634in}}%
\pgfpathlineto{\pgfqpoint{3.482039in}{1.468634in}}%
\pgfpathlineto{\pgfqpoint{3.482039in}{1.302155in}}%
\pgfpathclose%
\pgfusepath{stroke,fill}%
\end{pgfscope}%
\begin{pgfscope}%
\pgfpathrectangle{\pgfqpoint{0.781250in}{0.638889in}}{\pgfqpoint{4.218750in}{2.172222in}}%
\pgfusepath{clip}%
\pgfsetbuttcap%
\pgfsetmiterjoin%
\definecolor{currentfill}{rgb}{0.733333,0.321569,0.733333}%
\pgfsetfillcolor{currentfill}%
\pgfsetlinewidth{1.003750pt}%
\definecolor{currentstroke}{rgb}{0.266667,0.266667,0.266667}%
\pgfsetstrokecolor{currentstroke}%
\pgfsetdash{}{0pt}%
\pgfpathmoveto{\pgfqpoint{3.840471in}{1.449280in}}%
\pgfpathlineto{\pgfqpoint{4.091374in}{1.449280in}}%
\pgfpathlineto{\pgfqpoint{4.091374in}{1.643585in}}%
\pgfpathlineto{\pgfqpoint{3.840471in}{1.643585in}}%
\pgfpathlineto{\pgfqpoint{3.840471in}{1.449280in}}%
\pgfpathclose%
\pgfusepath{stroke,fill}%
\end{pgfscope}%
\begin{pgfscope}%
\pgfpathrectangle{\pgfqpoint{0.781250in}{0.638889in}}{\pgfqpoint{4.218750in}{2.172222in}}%
\pgfusepath{clip}%
\pgfsetbuttcap%
\pgfsetmiterjoin%
\definecolor{currentfill}{rgb}{0.733333,0.321569,0.733333}%
\pgfsetfillcolor{currentfill}%
\pgfsetlinewidth{1.003750pt}%
\definecolor{currentstroke}{rgb}{0.266667,0.266667,0.266667}%
\pgfsetstrokecolor{currentstroke}%
\pgfsetdash{}{0pt}%
\pgfpathmoveto{\pgfqpoint{4.198903in}{1.514055in}}%
\pgfpathlineto{\pgfqpoint{4.449806in}{1.514055in}}%
\pgfpathlineto{\pgfqpoint{4.449806in}{1.701627in}}%
\pgfpathlineto{\pgfqpoint{4.198903in}{1.701627in}}%
\pgfpathlineto{\pgfqpoint{4.198903in}{1.514055in}}%
\pgfpathclose%
\pgfusepath{stroke,fill}%
\end{pgfscope}%
\begin{pgfscope}%
\pgfpathrectangle{\pgfqpoint{0.781250in}{0.638889in}}{\pgfqpoint{4.218750in}{2.172222in}}%
\pgfusepath{clip}%
\pgfsetbuttcap%
\pgfsetmiterjoin%
\definecolor{currentfill}{rgb}{0.733333,0.321569,0.733333}%
\pgfsetfillcolor{currentfill}%
\pgfsetlinewidth{1.003750pt}%
\definecolor{currentstroke}{rgb}{0.266667,0.266667,0.266667}%
\pgfsetstrokecolor{currentstroke}%
\pgfsetdash{}{0pt}%
\pgfpathmoveto{\pgfqpoint{4.557336in}{1.638893in}}%
\pgfpathlineto{\pgfqpoint{4.808239in}{1.638893in}}%
\pgfpathlineto{\pgfqpoint{4.808239in}{1.844994in}}%
\pgfpathlineto{\pgfqpoint{4.557336in}{1.844994in}}%
\pgfpathlineto{\pgfqpoint{4.557336in}{1.638893in}}%
\pgfpathclose%
\pgfusepath{stroke,fill}%
\end{pgfscope}%
\begin{pgfscope}%
\pgfpathrectangle{\pgfqpoint{0.781250in}{0.638889in}}{\pgfqpoint{4.218750in}{2.172222in}}%
\pgfusepath{clip}%
\pgfsetbuttcap%
\pgfsetmiterjoin%
\definecolor{currentfill}{rgb}{0.549020,0.247059,0.121569}%
\pgfsetfillcolor{currentfill}%
\pgfsetlinewidth{1.003750pt}%
\definecolor{currentstroke}{rgb}{0.266667,0.266667,0.266667}%
\pgfsetstrokecolor{currentstroke}%
\pgfsetdash{}{0pt}%
\pgfpathmoveto{\pgfqpoint{0.973011in}{1.651535in}}%
\pgfpathlineto{\pgfqpoint{1.223914in}{1.651535in}}%
\pgfpathlineto{\pgfqpoint{1.223914in}{1.904252in}}%
\pgfpathlineto{\pgfqpoint{0.973011in}{1.904252in}}%
\pgfpathlineto{\pgfqpoint{0.973011in}{1.651535in}}%
\pgfpathclose%
\pgfusepath{stroke,fill}%
\end{pgfscope}%
\begin{pgfscope}%
\pgfpathrectangle{\pgfqpoint{0.781250in}{0.638889in}}{\pgfqpoint{4.218750in}{2.172222in}}%
\pgfusepath{clip}%
\pgfsetbuttcap%
\pgfsetmiterjoin%
\definecolor{currentfill}{rgb}{0.549020,0.247059,0.121569}%
\pgfsetfillcolor{currentfill}%
\pgfsetlinewidth{1.003750pt}%
\definecolor{currentstroke}{rgb}{0.266667,0.266667,0.266667}%
\pgfsetstrokecolor{currentstroke}%
\pgfsetdash{}{0pt}%
\pgfpathmoveto{\pgfqpoint{1.331444in}{1.432575in}}%
\pgfpathlineto{\pgfqpoint{1.582347in}{1.432575in}}%
\pgfpathlineto{\pgfqpoint{1.582347in}{1.623970in}}%
\pgfpathlineto{\pgfqpoint{1.331444in}{1.623970in}}%
\pgfpathlineto{\pgfqpoint{1.331444in}{1.432575in}}%
\pgfpathclose%
\pgfusepath{stroke,fill}%
\end{pgfscope}%
\begin{pgfscope}%
\pgfpathrectangle{\pgfqpoint{0.781250in}{0.638889in}}{\pgfqpoint{4.218750in}{2.172222in}}%
\pgfusepath{clip}%
\pgfsetbuttcap%
\pgfsetmiterjoin%
\definecolor{currentfill}{rgb}{0.549020,0.247059,0.121569}%
\pgfsetfillcolor{currentfill}%
\pgfsetlinewidth{1.003750pt}%
\definecolor{currentstroke}{rgb}{0.266667,0.266667,0.266667}%
\pgfsetstrokecolor{currentstroke}%
\pgfsetdash{}{0pt}%
\pgfpathmoveto{\pgfqpoint{1.689876in}{1.324486in}}%
\pgfpathlineto{\pgfqpoint{1.940779in}{1.324486in}}%
\pgfpathlineto{\pgfqpoint{1.940779in}{1.476433in}}%
\pgfpathlineto{\pgfqpoint{1.689876in}{1.476433in}}%
\pgfpathlineto{\pgfqpoint{1.689876in}{1.324486in}}%
\pgfpathclose%
\pgfusepath{stroke,fill}%
\end{pgfscope}%
\begin{pgfscope}%
\pgfpathrectangle{\pgfqpoint{0.781250in}{0.638889in}}{\pgfqpoint{4.218750in}{2.172222in}}%
\pgfusepath{clip}%
\pgfsetbuttcap%
\pgfsetmiterjoin%
\definecolor{currentfill}{rgb}{0.549020,0.247059,0.121569}%
\pgfsetfillcolor{currentfill}%
\pgfsetlinewidth{1.003750pt}%
\definecolor{currentstroke}{rgb}{0.266667,0.266667,0.266667}%
\pgfsetstrokecolor{currentstroke}%
\pgfsetdash{}{0pt}%
\pgfpathmoveto{\pgfqpoint{2.048309in}{1.324942in}}%
\pgfpathlineto{\pgfqpoint{2.299211in}{1.324942in}}%
\pgfpathlineto{\pgfqpoint{2.299211in}{1.468482in}}%
\pgfpathlineto{\pgfqpoint{2.048309in}{1.468482in}}%
\pgfpathlineto{\pgfqpoint{2.048309in}{1.324942in}}%
\pgfpathclose%
\pgfusepath{stroke,fill}%
\end{pgfscope}%
\begin{pgfscope}%
\pgfpathrectangle{\pgfqpoint{0.781250in}{0.638889in}}{\pgfqpoint{4.218750in}{2.172222in}}%
\pgfusepath{clip}%
\pgfsetbuttcap%
\pgfsetmiterjoin%
\definecolor{currentfill}{rgb}{0.549020,0.247059,0.121569}%
\pgfsetfillcolor{currentfill}%
\pgfsetlinewidth{1.003750pt}%
\definecolor{currentstroke}{rgb}{0.266667,0.266667,0.266667}%
\pgfsetstrokecolor{currentstroke}%
\pgfsetdash{}{0pt}%
\pgfpathmoveto{\pgfqpoint{2.406741in}{1.437333in}}%
\pgfpathlineto{\pgfqpoint{2.657644in}{1.437333in}}%
\pgfpathlineto{\pgfqpoint{2.657644in}{1.578744in}}%
\pgfpathlineto{\pgfqpoint{2.406741in}{1.578744in}}%
\pgfpathlineto{\pgfqpoint{2.406741in}{1.437333in}}%
\pgfpathclose%
\pgfusepath{stroke,fill}%
\end{pgfscope}%
\begin{pgfscope}%
\pgfpathrectangle{\pgfqpoint{0.781250in}{0.638889in}}{\pgfqpoint{4.218750in}{2.172222in}}%
\pgfusepath{clip}%
\pgfsetbuttcap%
\pgfsetmiterjoin%
\definecolor{currentfill}{rgb}{0.549020,0.247059,0.121569}%
\pgfsetfillcolor{currentfill}%
\pgfsetlinewidth{1.003750pt}%
\definecolor{currentstroke}{rgb}{0.266667,0.266667,0.266667}%
\pgfsetstrokecolor{currentstroke}%
\pgfsetdash{}{0pt}%
\pgfpathmoveto{\pgfqpoint{2.765174in}{1.565190in}}%
\pgfpathlineto{\pgfqpoint{3.016076in}{1.565190in}}%
\pgfpathlineto{\pgfqpoint{3.016076in}{1.725348in}}%
\pgfpathlineto{\pgfqpoint{2.765174in}{1.725348in}}%
\pgfpathlineto{\pgfqpoint{2.765174in}{1.565190in}}%
\pgfpathclose%
\pgfusepath{stroke,fill}%
\end{pgfscope}%
\begin{pgfscope}%
\pgfpathrectangle{\pgfqpoint{0.781250in}{0.638889in}}{\pgfqpoint{4.218750in}{2.172222in}}%
\pgfusepath{clip}%
\pgfsetbuttcap%
\pgfsetmiterjoin%
\definecolor{currentfill}{rgb}{0.549020,0.247059,0.121569}%
\pgfsetfillcolor{currentfill}%
\pgfsetlinewidth{1.003750pt}%
\definecolor{currentstroke}{rgb}{0.266667,0.266667,0.266667}%
\pgfsetstrokecolor{currentstroke}%
\pgfsetdash{}{0pt}%
\pgfpathmoveto{\pgfqpoint{3.123606in}{1.477345in}}%
\pgfpathlineto{\pgfqpoint{3.374509in}{1.477345in}}%
\pgfpathlineto{\pgfqpoint{3.374509in}{1.633354in}}%
\pgfpathlineto{\pgfqpoint{3.123606in}{1.633354in}}%
\pgfpathlineto{\pgfqpoint{3.123606in}{1.477345in}}%
\pgfpathclose%
\pgfusepath{stroke,fill}%
\end{pgfscope}%
\begin{pgfscope}%
\pgfpathrectangle{\pgfqpoint{0.781250in}{0.638889in}}{\pgfqpoint{4.218750in}{2.172222in}}%
\pgfusepath{clip}%
\pgfsetbuttcap%
\pgfsetmiterjoin%
\definecolor{currentfill}{rgb}{0.549020,0.247059,0.121569}%
\pgfsetfillcolor{currentfill}%
\pgfsetlinewidth{1.003750pt}%
\definecolor{currentstroke}{rgb}{0.266667,0.266667,0.266667}%
\pgfsetstrokecolor{currentstroke}%
\pgfsetdash{}{0pt}%
\pgfpathmoveto{\pgfqpoint{3.482039in}{1.468634in}}%
\pgfpathlineto{\pgfqpoint{3.732941in}{1.468634in}}%
\pgfpathlineto{\pgfqpoint{3.732941in}{1.602139in}}%
\pgfpathlineto{\pgfqpoint{3.482039in}{1.602139in}}%
\pgfpathlineto{\pgfqpoint{3.482039in}{1.468634in}}%
\pgfpathclose%
\pgfusepath{stroke,fill}%
\end{pgfscope}%
\begin{pgfscope}%
\pgfpathrectangle{\pgfqpoint{0.781250in}{0.638889in}}{\pgfqpoint{4.218750in}{2.172222in}}%
\pgfusepath{clip}%
\pgfsetbuttcap%
\pgfsetmiterjoin%
\definecolor{currentfill}{rgb}{0.549020,0.247059,0.121569}%
\pgfsetfillcolor{currentfill}%
\pgfsetlinewidth{1.003750pt}%
\definecolor{currentstroke}{rgb}{0.266667,0.266667,0.266667}%
\pgfsetstrokecolor{currentstroke}%
\pgfsetdash{}{0pt}%
\pgfpathmoveto{\pgfqpoint{3.840471in}{1.643585in}}%
\pgfpathlineto{\pgfqpoint{4.091374in}{1.643585in}}%
\pgfpathlineto{\pgfqpoint{4.091374in}{1.805198in}}%
\pgfpathlineto{\pgfqpoint{3.840471in}{1.805198in}}%
\pgfpathlineto{\pgfqpoint{3.840471in}{1.643585in}}%
\pgfpathclose%
\pgfusepath{stroke,fill}%
\end{pgfscope}%
\begin{pgfscope}%
\pgfpathrectangle{\pgfqpoint{0.781250in}{0.638889in}}{\pgfqpoint{4.218750in}{2.172222in}}%
\pgfusepath{clip}%
\pgfsetbuttcap%
\pgfsetmiterjoin%
\definecolor{currentfill}{rgb}{0.549020,0.247059,0.121569}%
\pgfsetfillcolor{currentfill}%
\pgfsetlinewidth{1.003750pt}%
\definecolor{currentstroke}{rgb}{0.266667,0.266667,0.266667}%
\pgfsetstrokecolor{currentstroke}%
\pgfsetdash{}{0pt}%
\pgfpathmoveto{\pgfqpoint{4.198903in}{1.701627in}}%
\pgfpathlineto{\pgfqpoint{4.449806in}{1.701627in}}%
\pgfpathlineto{\pgfqpoint{4.449806in}{1.857028in}}%
\pgfpathlineto{\pgfqpoint{4.198903in}{1.857028in}}%
\pgfpathlineto{\pgfqpoint{4.198903in}{1.701627in}}%
\pgfpathclose%
\pgfusepath{stroke,fill}%
\end{pgfscope}%
\begin{pgfscope}%
\pgfpathrectangle{\pgfqpoint{0.781250in}{0.638889in}}{\pgfqpoint{4.218750in}{2.172222in}}%
\pgfusepath{clip}%
\pgfsetbuttcap%
\pgfsetmiterjoin%
\definecolor{currentfill}{rgb}{0.549020,0.247059,0.121569}%
\pgfsetfillcolor{currentfill}%
\pgfsetlinewidth{1.003750pt}%
\definecolor{currentstroke}{rgb}{0.266667,0.266667,0.266667}%
\pgfsetstrokecolor{currentstroke}%
\pgfsetdash{}{0pt}%
\pgfpathmoveto{\pgfqpoint{4.557336in}{1.844994in}}%
\pgfpathlineto{\pgfqpoint{4.808239in}{1.844994in}}%
\pgfpathlineto{\pgfqpoint{4.808239in}{2.012428in}}%
\pgfpathlineto{\pgfqpoint{4.557336in}{2.012428in}}%
\pgfpathlineto{\pgfqpoint{4.557336in}{1.844994in}}%
\pgfpathclose%
\pgfusepath{stroke,fill}%
\end{pgfscope}%
\begin{pgfscope}%
\pgfpathrectangle{\pgfqpoint{0.781250in}{0.638889in}}{\pgfqpoint{4.218750in}{2.172222in}}%
\pgfusepath{clip}%
\pgfsetbuttcap%
\pgfsetmiterjoin%
\definecolor{currentfill}{rgb}{0.701961,0.760784,0.360784}%
\pgfsetfillcolor{currentfill}%
\pgfsetlinewidth{1.003750pt}%
\definecolor{currentstroke}{rgb}{0.266667,0.266667,0.266667}%
\pgfsetstrokecolor{currentstroke}%
\pgfsetdash{}{0pt}%
\pgfpathmoveto{\pgfqpoint{0.973011in}{1.904252in}}%
\pgfpathlineto{\pgfqpoint{1.223914in}{1.904252in}}%
\pgfpathlineto{\pgfqpoint{1.223914in}{2.077095in}}%
\pgfpathlineto{\pgfqpoint{0.973011in}{2.077095in}}%
\pgfpathlineto{\pgfqpoint{0.973011in}{1.904252in}}%
\pgfpathclose%
\pgfusepath{stroke,fill}%
\end{pgfscope}%
\begin{pgfscope}%
\pgfpathrectangle{\pgfqpoint{0.781250in}{0.638889in}}{\pgfqpoint{4.218750in}{2.172222in}}%
\pgfusepath{clip}%
\pgfsetbuttcap%
\pgfsetmiterjoin%
\definecolor{currentfill}{rgb}{0.701961,0.760784,0.360784}%
\pgfsetfillcolor{currentfill}%
\pgfsetlinewidth{1.003750pt}%
\definecolor{currentstroke}{rgb}{0.266667,0.266667,0.266667}%
\pgfsetstrokecolor{currentstroke}%
\pgfsetdash{}{0pt}%
\pgfpathmoveto{\pgfqpoint{1.331444in}{1.623970in}}%
\pgfpathlineto{\pgfqpoint{1.582347in}{1.623970in}}%
\pgfpathlineto{\pgfqpoint{1.582347in}{1.761819in}}%
\pgfpathlineto{\pgfqpoint{1.331444in}{1.761819in}}%
\pgfpathlineto{\pgfqpoint{1.331444in}{1.623970in}}%
\pgfpathclose%
\pgfusepath{stroke,fill}%
\end{pgfscope}%
\begin{pgfscope}%
\pgfpathrectangle{\pgfqpoint{0.781250in}{0.638889in}}{\pgfqpoint{4.218750in}{2.172222in}}%
\pgfusepath{clip}%
\pgfsetbuttcap%
\pgfsetmiterjoin%
\definecolor{currentfill}{rgb}{0.701961,0.760784,0.360784}%
\pgfsetfillcolor{currentfill}%
\pgfsetlinewidth{1.003750pt}%
\definecolor{currentstroke}{rgb}{0.266667,0.266667,0.266667}%
\pgfsetstrokecolor{currentstroke}%
\pgfsetdash{}{0pt}%
\pgfpathmoveto{\pgfqpoint{1.689876in}{1.476433in}}%
\pgfpathlineto{\pgfqpoint{1.940779in}{1.476433in}}%
\pgfpathlineto{\pgfqpoint{1.940779in}{1.596709in}}%
\pgfpathlineto{\pgfqpoint{1.689876in}{1.596709in}}%
\pgfpathlineto{\pgfqpoint{1.689876in}{1.476433in}}%
\pgfpathclose%
\pgfusepath{stroke,fill}%
\end{pgfscope}%
\begin{pgfscope}%
\pgfpathrectangle{\pgfqpoint{0.781250in}{0.638889in}}{\pgfqpoint{4.218750in}{2.172222in}}%
\pgfusepath{clip}%
\pgfsetbuttcap%
\pgfsetmiterjoin%
\definecolor{currentfill}{rgb}{0.701961,0.760784,0.360784}%
\pgfsetfillcolor{currentfill}%
\pgfsetlinewidth{1.003750pt}%
\definecolor{currentstroke}{rgb}{0.266667,0.266667,0.266667}%
\pgfsetstrokecolor{currentstroke}%
\pgfsetdash{}{0pt}%
\pgfpathmoveto{\pgfqpoint{2.048309in}{1.468482in}}%
\pgfpathlineto{\pgfqpoint{2.299211in}{1.468482in}}%
\pgfpathlineto{\pgfqpoint{2.299211in}{1.585174in}}%
\pgfpathlineto{\pgfqpoint{2.048309in}{1.585174in}}%
\pgfpathlineto{\pgfqpoint{2.048309in}{1.468482in}}%
\pgfpathclose%
\pgfusepath{stroke,fill}%
\end{pgfscope}%
\begin{pgfscope}%
\pgfpathrectangle{\pgfqpoint{0.781250in}{0.638889in}}{\pgfqpoint{4.218750in}{2.172222in}}%
\pgfusepath{clip}%
\pgfsetbuttcap%
\pgfsetmiterjoin%
\definecolor{currentfill}{rgb}{0.701961,0.760784,0.360784}%
\pgfsetfillcolor{currentfill}%
\pgfsetlinewidth{1.003750pt}%
\definecolor{currentstroke}{rgb}{0.266667,0.266667,0.266667}%
\pgfsetstrokecolor{currentstroke}%
\pgfsetdash{}{0pt}%
\pgfpathmoveto{\pgfqpoint{2.406741in}{1.578744in}}%
\pgfpathlineto{\pgfqpoint{2.657644in}{1.578744in}}%
\pgfpathlineto{\pgfqpoint{2.657644in}{1.694545in}}%
\pgfpathlineto{\pgfqpoint{2.406741in}{1.694545in}}%
\pgfpathlineto{\pgfqpoint{2.406741in}{1.578744in}}%
\pgfpathclose%
\pgfusepath{stroke,fill}%
\end{pgfscope}%
\begin{pgfscope}%
\pgfpathrectangle{\pgfqpoint{0.781250in}{0.638889in}}{\pgfqpoint{4.218750in}{2.172222in}}%
\pgfusepath{clip}%
\pgfsetbuttcap%
\pgfsetmiterjoin%
\definecolor{currentfill}{rgb}{0.701961,0.760784,0.360784}%
\pgfsetfillcolor{currentfill}%
\pgfsetlinewidth{1.003750pt}%
\definecolor{currentstroke}{rgb}{0.266667,0.266667,0.266667}%
\pgfsetstrokecolor{currentstroke}%
\pgfsetdash{}{0pt}%
\pgfpathmoveto{\pgfqpoint{2.765174in}{1.725348in}}%
\pgfpathlineto{\pgfqpoint{3.016076in}{1.725348in}}%
\pgfpathlineto{\pgfqpoint{3.016076in}{1.862958in}}%
\pgfpathlineto{\pgfqpoint{2.765174in}{1.862958in}}%
\pgfpathlineto{\pgfqpoint{2.765174in}{1.725348in}}%
\pgfpathclose%
\pgfusepath{stroke,fill}%
\end{pgfscope}%
\begin{pgfscope}%
\pgfpathrectangle{\pgfqpoint{0.781250in}{0.638889in}}{\pgfqpoint{4.218750in}{2.172222in}}%
\pgfusepath{clip}%
\pgfsetbuttcap%
\pgfsetmiterjoin%
\definecolor{currentfill}{rgb}{0.701961,0.760784,0.360784}%
\pgfsetfillcolor{currentfill}%
\pgfsetlinewidth{1.003750pt}%
\definecolor{currentstroke}{rgb}{0.266667,0.266667,0.266667}%
\pgfsetstrokecolor{currentstroke}%
\pgfsetdash{}{0pt}%
\pgfpathmoveto{\pgfqpoint{3.123606in}{1.633354in}}%
\pgfpathlineto{\pgfqpoint{3.374509in}{1.633354in}}%
\pgfpathlineto{\pgfqpoint{3.374509in}{1.767923in}}%
\pgfpathlineto{\pgfqpoint{3.123606in}{1.767923in}}%
\pgfpathlineto{\pgfqpoint{3.123606in}{1.633354in}}%
\pgfpathclose%
\pgfusepath{stroke,fill}%
\end{pgfscope}%
\begin{pgfscope}%
\pgfpathrectangle{\pgfqpoint{0.781250in}{0.638889in}}{\pgfqpoint{4.218750in}{2.172222in}}%
\pgfusepath{clip}%
\pgfsetbuttcap%
\pgfsetmiterjoin%
\definecolor{currentfill}{rgb}{0.701961,0.760784,0.360784}%
\pgfsetfillcolor{currentfill}%
\pgfsetlinewidth{1.003750pt}%
\definecolor{currentstroke}{rgb}{0.266667,0.266667,0.266667}%
\pgfsetstrokecolor{currentstroke}%
\pgfsetdash{}{0pt}%
\pgfpathmoveto{\pgfqpoint{3.482039in}{1.602139in}}%
\pgfpathlineto{\pgfqpoint{3.732941in}{1.602139in}}%
\pgfpathlineto{\pgfqpoint{3.732941in}{1.736274in}}%
\pgfpathlineto{\pgfqpoint{3.482039in}{1.736274in}}%
\pgfpathlineto{\pgfqpoint{3.482039in}{1.602139in}}%
\pgfpathclose%
\pgfusepath{stroke,fill}%
\end{pgfscope}%
\begin{pgfscope}%
\pgfpathrectangle{\pgfqpoint{0.781250in}{0.638889in}}{\pgfqpoint{4.218750in}{2.172222in}}%
\pgfusepath{clip}%
\pgfsetbuttcap%
\pgfsetmiterjoin%
\definecolor{currentfill}{rgb}{0.701961,0.760784,0.360784}%
\pgfsetfillcolor{currentfill}%
\pgfsetlinewidth{1.003750pt}%
\definecolor{currentstroke}{rgb}{0.266667,0.266667,0.266667}%
\pgfsetstrokecolor{currentstroke}%
\pgfsetdash{}{0pt}%
\pgfpathmoveto{\pgfqpoint{3.840471in}{1.805198in}}%
\pgfpathlineto{\pgfqpoint{4.091374in}{1.805198in}}%
\pgfpathlineto{\pgfqpoint{4.091374in}{1.942657in}}%
\pgfpathlineto{\pgfqpoint{3.840471in}{1.942657in}}%
\pgfpathlineto{\pgfqpoint{3.840471in}{1.805198in}}%
\pgfpathclose%
\pgfusepath{stroke,fill}%
\end{pgfscope}%
\begin{pgfscope}%
\pgfpathrectangle{\pgfqpoint{0.781250in}{0.638889in}}{\pgfqpoint{4.218750in}{2.172222in}}%
\pgfusepath{clip}%
\pgfsetbuttcap%
\pgfsetmiterjoin%
\definecolor{currentfill}{rgb}{0.701961,0.760784,0.360784}%
\pgfsetfillcolor{currentfill}%
\pgfsetlinewidth{1.003750pt}%
\definecolor{currentstroke}{rgb}{0.266667,0.266667,0.266667}%
\pgfsetstrokecolor{currentstroke}%
\pgfsetdash{}{0pt}%
\pgfpathmoveto{\pgfqpoint{4.198903in}{1.857028in}}%
\pgfpathlineto{\pgfqpoint{4.449806in}{1.857028in}}%
\pgfpathlineto{\pgfqpoint{4.449806in}{1.992574in}}%
\pgfpathlineto{\pgfqpoint{4.198903in}{1.992574in}}%
\pgfpathlineto{\pgfqpoint{4.198903in}{1.857028in}}%
\pgfpathclose%
\pgfusepath{stroke,fill}%
\end{pgfscope}%
\begin{pgfscope}%
\pgfpathrectangle{\pgfqpoint{0.781250in}{0.638889in}}{\pgfqpoint{4.218750in}{2.172222in}}%
\pgfusepath{clip}%
\pgfsetbuttcap%
\pgfsetmiterjoin%
\definecolor{currentfill}{rgb}{0.701961,0.760784,0.360784}%
\pgfsetfillcolor{currentfill}%
\pgfsetlinewidth{1.003750pt}%
\definecolor{currentstroke}{rgb}{0.266667,0.266667,0.266667}%
\pgfsetstrokecolor{currentstroke}%
\pgfsetdash{}{0pt}%
\pgfpathmoveto{\pgfqpoint{4.557336in}{2.012428in}}%
\pgfpathlineto{\pgfqpoint{4.808239in}{2.012428in}}%
\pgfpathlineto{\pgfqpoint{4.808239in}{2.148649in}}%
\pgfpathlineto{\pgfqpoint{4.557336in}{2.148649in}}%
\pgfpathlineto{\pgfqpoint{4.557336in}{2.012428in}}%
\pgfpathclose%
\pgfusepath{stroke,fill}%
\end{pgfscope}%
\begin{pgfscope}%
\pgfpathrectangle{\pgfqpoint{0.781250in}{0.638889in}}{\pgfqpoint{4.218750in}{2.172222in}}%
\pgfusepath{clip}%
\pgfsetbuttcap%
\pgfsetmiterjoin%
\definecolor{currentfill}{rgb}{0.447059,0.447059,0.447059}%
\pgfsetfillcolor{currentfill}%
\pgfsetlinewidth{1.003750pt}%
\definecolor{currentstroke}{rgb}{0.266667,0.266667,0.266667}%
\pgfsetstrokecolor{currentstroke}%
\pgfsetdash{}{0pt}%
\pgfpathmoveto{\pgfqpoint{0.973011in}{2.077095in}}%
\pgfpathlineto{\pgfqpoint{1.223914in}{2.077095in}}%
\pgfpathlineto{\pgfqpoint{1.223914in}{2.208298in}}%
\pgfpathlineto{\pgfqpoint{0.973011in}{2.208298in}}%
\pgfpathlineto{\pgfqpoint{0.973011in}{2.077095in}}%
\pgfpathclose%
\pgfusepath{stroke,fill}%
\end{pgfscope}%
\begin{pgfscope}%
\pgfpathrectangle{\pgfqpoint{0.781250in}{0.638889in}}{\pgfqpoint{4.218750in}{2.172222in}}%
\pgfusepath{clip}%
\pgfsetbuttcap%
\pgfsetmiterjoin%
\definecolor{currentfill}{rgb}{0.447059,0.447059,0.447059}%
\pgfsetfillcolor{currentfill}%
\pgfsetlinewidth{1.003750pt}%
\definecolor{currentstroke}{rgb}{0.266667,0.266667,0.266667}%
\pgfsetstrokecolor{currentstroke}%
\pgfsetdash{}{0pt}%
\pgfpathmoveto{\pgfqpoint{1.331444in}{1.761819in}}%
\pgfpathlineto{\pgfqpoint{1.582347in}{1.761819in}}%
\pgfpathlineto{\pgfqpoint{1.582347in}{1.872755in}}%
\pgfpathlineto{\pgfqpoint{1.331444in}{1.872755in}}%
\pgfpathlineto{\pgfqpoint{1.331444in}{1.761819in}}%
\pgfpathclose%
\pgfusepath{stroke,fill}%
\end{pgfscope}%
\begin{pgfscope}%
\pgfpathrectangle{\pgfqpoint{0.781250in}{0.638889in}}{\pgfqpoint{4.218750in}{2.172222in}}%
\pgfusepath{clip}%
\pgfsetbuttcap%
\pgfsetmiterjoin%
\definecolor{currentfill}{rgb}{0.447059,0.447059,0.447059}%
\pgfsetfillcolor{currentfill}%
\pgfsetlinewidth{1.003750pt}%
\definecolor{currentstroke}{rgb}{0.266667,0.266667,0.266667}%
\pgfsetstrokecolor{currentstroke}%
\pgfsetdash{}{0pt}%
\pgfpathmoveto{\pgfqpoint{1.689876in}{1.596709in}}%
\pgfpathlineto{\pgfqpoint{1.940779in}{1.596709in}}%
\pgfpathlineto{\pgfqpoint{1.940779in}{1.717962in}}%
\pgfpathlineto{\pgfqpoint{1.689876in}{1.717962in}}%
\pgfpathlineto{\pgfqpoint{1.689876in}{1.596709in}}%
\pgfpathclose%
\pgfusepath{stroke,fill}%
\end{pgfscope}%
\begin{pgfscope}%
\pgfpathrectangle{\pgfqpoint{0.781250in}{0.638889in}}{\pgfqpoint{4.218750in}{2.172222in}}%
\pgfusepath{clip}%
\pgfsetbuttcap%
\pgfsetmiterjoin%
\definecolor{currentfill}{rgb}{0.447059,0.447059,0.447059}%
\pgfsetfillcolor{currentfill}%
\pgfsetlinewidth{1.003750pt}%
\definecolor{currentstroke}{rgb}{0.266667,0.266667,0.266667}%
\pgfsetstrokecolor{currentstroke}%
\pgfsetdash{}{0pt}%
\pgfpathmoveto{\pgfqpoint{2.048309in}{1.585174in}}%
\pgfpathlineto{\pgfqpoint{2.299211in}{1.585174in}}%
\pgfpathlineto{\pgfqpoint{2.299211in}{1.663200in}}%
\pgfpathlineto{\pgfqpoint{2.048309in}{1.663200in}}%
\pgfpathlineto{\pgfqpoint{2.048309in}{1.585174in}}%
\pgfpathclose%
\pgfusepath{stroke,fill}%
\end{pgfscope}%
\begin{pgfscope}%
\pgfpathrectangle{\pgfqpoint{0.781250in}{0.638889in}}{\pgfqpoint{4.218750in}{2.172222in}}%
\pgfusepath{clip}%
\pgfsetbuttcap%
\pgfsetmiterjoin%
\definecolor{currentfill}{rgb}{0.447059,0.447059,0.447059}%
\pgfsetfillcolor{currentfill}%
\pgfsetlinewidth{1.003750pt}%
\definecolor{currentstroke}{rgb}{0.266667,0.266667,0.266667}%
\pgfsetstrokecolor{currentstroke}%
\pgfsetdash{}{0pt}%
\pgfpathmoveto{\pgfqpoint{2.406741in}{1.694545in}}%
\pgfpathlineto{\pgfqpoint{2.657644in}{1.694545in}}%
\pgfpathlineto{\pgfqpoint{2.657644in}{1.774223in}}%
\pgfpathlineto{\pgfqpoint{2.406741in}{1.774223in}}%
\pgfpathlineto{\pgfqpoint{2.406741in}{1.694545in}}%
\pgfpathclose%
\pgfusepath{stroke,fill}%
\end{pgfscope}%
\begin{pgfscope}%
\pgfpathrectangle{\pgfqpoint{0.781250in}{0.638889in}}{\pgfqpoint{4.218750in}{2.172222in}}%
\pgfusepath{clip}%
\pgfsetbuttcap%
\pgfsetmiterjoin%
\definecolor{currentfill}{rgb}{0.447059,0.447059,0.447059}%
\pgfsetfillcolor{currentfill}%
\pgfsetlinewidth{1.003750pt}%
\definecolor{currentstroke}{rgb}{0.266667,0.266667,0.266667}%
\pgfsetstrokecolor{currentstroke}%
\pgfsetdash{}{0pt}%
\pgfpathmoveto{\pgfqpoint{2.765174in}{1.862958in}}%
\pgfpathlineto{\pgfqpoint{3.016076in}{1.862958in}}%
\pgfpathlineto{\pgfqpoint{3.016076in}{1.950607in}}%
\pgfpathlineto{\pgfqpoint{2.765174in}{1.950607in}}%
\pgfpathlineto{\pgfqpoint{2.765174in}{1.862958in}}%
\pgfpathclose%
\pgfusepath{stroke,fill}%
\end{pgfscope}%
\begin{pgfscope}%
\pgfpathrectangle{\pgfqpoint{0.781250in}{0.638889in}}{\pgfqpoint{4.218750in}{2.172222in}}%
\pgfusepath{clip}%
\pgfsetbuttcap%
\pgfsetmiterjoin%
\definecolor{currentfill}{rgb}{0.447059,0.447059,0.447059}%
\pgfsetfillcolor{currentfill}%
\pgfsetlinewidth{1.003750pt}%
\definecolor{currentstroke}{rgb}{0.266667,0.266667,0.266667}%
\pgfsetstrokecolor{currentstroke}%
\pgfsetdash{}{0pt}%
\pgfpathmoveto{\pgfqpoint{3.123606in}{1.767923in}}%
\pgfpathlineto{\pgfqpoint{3.374509in}{1.767923in}}%
\pgfpathlineto{\pgfqpoint{3.374509in}{1.848817in}}%
\pgfpathlineto{\pgfqpoint{3.123606in}{1.848817in}}%
\pgfpathlineto{\pgfqpoint{3.123606in}{1.767923in}}%
\pgfpathclose%
\pgfusepath{stroke,fill}%
\end{pgfscope}%
\begin{pgfscope}%
\pgfpathrectangle{\pgfqpoint{0.781250in}{0.638889in}}{\pgfqpoint{4.218750in}{2.172222in}}%
\pgfusepath{clip}%
\pgfsetbuttcap%
\pgfsetmiterjoin%
\definecolor{currentfill}{rgb}{0.447059,0.447059,0.447059}%
\pgfsetfillcolor{currentfill}%
\pgfsetlinewidth{1.003750pt}%
\definecolor{currentstroke}{rgb}{0.266667,0.266667,0.266667}%
\pgfsetstrokecolor{currentstroke}%
\pgfsetdash{}{0pt}%
\pgfpathmoveto{\pgfqpoint{3.482039in}{1.736274in}}%
\pgfpathlineto{\pgfqpoint{3.732941in}{1.736274in}}%
\pgfpathlineto{\pgfqpoint{3.732941in}{1.839541in}}%
\pgfpathlineto{\pgfqpoint{3.482039in}{1.839541in}}%
\pgfpathlineto{\pgfqpoint{3.482039in}{1.736274in}}%
\pgfpathclose%
\pgfusepath{stroke,fill}%
\end{pgfscope}%
\begin{pgfscope}%
\pgfpathrectangle{\pgfqpoint{0.781250in}{0.638889in}}{\pgfqpoint{4.218750in}{2.172222in}}%
\pgfusepath{clip}%
\pgfsetbuttcap%
\pgfsetmiterjoin%
\definecolor{currentfill}{rgb}{0.447059,0.447059,0.447059}%
\pgfsetfillcolor{currentfill}%
\pgfsetlinewidth{1.003750pt}%
\definecolor{currentstroke}{rgb}{0.266667,0.266667,0.266667}%
\pgfsetstrokecolor{currentstroke}%
\pgfsetdash{}{0pt}%
\pgfpathmoveto{\pgfqpoint{3.840471in}{1.942657in}}%
\pgfpathlineto{\pgfqpoint{4.091374in}{1.942657in}}%
\pgfpathlineto{\pgfqpoint{4.091374in}{2.049964in}}%
\pgfpathlineto{\pgfqpoint{3.840471in}{2.049964in}}%
\pgfpathlineto{\pgfqpoint{3.840471in}{1.942657in}}%
\pgfpathclose%
\pgfusepath{stroke,fill}%
\end{pgfscope}%
\begin{pgfscope}%
\pgfpathrectangle{\pgfqpoint{0.781250in}{0.638889in}}{\pgfqpoint{4.218750in}{2.172222in}}%
\pgfusepath{clip}%
\pgfsetbuttcap%
\pgfsetmiterjoin%
\definecolor{currentfill}{rgb}{0.447059,0.447059,0.447059}%
\pgfsetfillcolor{currentfill}%
\pgfsetlinewidth{1.003750pt}%
\definecolor{currentstroke}{rgb}{0.266667,0.266667,0.266667}%
\pgfsetstrokecolor{currentstroke}%
\pgfsetdash{}{0pt}%
\pgfpathmoveto{\pgfqpoint{4.198903in}{1.992574in}}%
\pgfpathlineto{\pgfqpoint{4.449806in}{1.992574in}}%
\pgfpathlineto{\pgfqpoint{4.449806in}{2.107072in}}%
\pgfpathlineto{\pgfqpoint{4.198903in}{2.107072in}}%
\pgfpathlineto{\pgfqpoint{4.198903in}{1.992574in}}%
\pgfpathclose%
\pgfusepath{stroke,fill}%
\end{pgfscope}%
\begin{pgfscope}%
\pgfpathrectangle{\pgfqpoint{0.781250in}{0.638889in}}{\pgfqpoint{4.218750in}{2.172222in}}%
\pgfusepath{clip}%
\pgfsetbuttcap%
\pgfsetmiterjoin%
\definecolor{currentfill}{rgb}{0.447059,0.447059,0.447059}%
\pgfsetfillcolor{currentfill}%
\pgfsetlinewidth{1.003750pt}%
\definecolor{currentstroke}{rgb}{0.266667,0.266667,0.266667}%
\pgfsetstrokecolor{currentstroke}%
\pgfsetdash{}{0pt}%
\pgfpathmoveto{\pgfqpoint{4.557336in}{2.148648in}}%
\pgfpathlineto{\pgfqpoint{4.808239in}{2.148648in}}%
\pgfpathlineto{\pgfqpoint{4.808239in}{2.248158in}}%
\pgfpathlineto{\pgfqpoint{4.557336in}{2.248158in}}%
\pgfpathlineto{\pgfqpoint{4.557336in}{2.148648in}}%
\pgfpathclose%
\pgfusepath{stroke,fill}%
\end{pgfscope}%
\begin{pgfscope}%
\pgfpathrectangle{\pgfqpoint{0.781250in}{0.638889in}}{\pgfqpoint{4.218750in}{2.172222in}}%
\pgfusepath{clip}%
\pgfsetbuttcap%
\pgfsetmiterjoin%
\definecolor{currentfill}{rgb}{0.447059,0.447059,0.447059}%
\pgfsetfillcolor{currentfill}%
\pgfsetlinewidth{1.003750pt}%
\definecolor{currentstroke}{rgb}{0.266667,0.266667,0.266667}%
\pgfsetstrokecolor{currentstroke}%
\pgfsetdash{}{0pt}%
\pgfpathmoveto{\pgfqpoint{0.973011in}{2.208298in}}%
\pgfpathlineto{\pgfqpoint{1.223914in}{2.208298in}}%
\pgfpathlineto{\pgfqpoint{1.223914in}{2.396847in}}%
\pgfpathlineto{\pgfqpoint{0.973011in}{2.396847in}}%
\pgfpathlineto{\pgfqpoint{0.973011in}{2.208298in}}%
\pgfpathclose%
\pgfusepath{stroke,fill}%
\end{pgfscope}%
\begin{pgfscope}%
\pgfpathrectangle{\pgfqpoint{0.781250in}{0.638889in}}{\pgfqpoint{4.218750in}{2.172222in}}%
\pgfusepath{clip}%
\pgfsetbuttcap%
\pgfsetmiterjoin%
\definecolor{currentfill}{rgb}{0.447059,0.447059,0.447059}%
\pgfsetfillcolor{currentfill}%
\pgfsetlinewidth{1.003750pt}%
\definecolor{currentstroke}{rgb}{0.266667,0.266667,0.266667}%
\pgfsetstrokecolor{currentstroke}%
\pgfsetdash{}{0pt}%
\pgfpathmoveto{\pgfqpoint{1.331444in}{1.872755in}}%
\pgfpathlineto{\pgfqpoint{1.582347in}{1.872755in}}%
\pgfpathlineto{\pgfqpoint{1.582347in}{1.999200in}}%
\pgfpathlineto{\pgfqpoint{1.331444in}{1.999200in}}%
\pgfpathlineto{\pgfqpoint{1.331444in}{1.872755in}}%
\pgfpathclose%
\pgfusepath{stroke,fill}%
\end{pgfscope}%
\begin{pgfscope}%
\pgfpathrectangle{\pgfqpoint{0.781250in}{0.638889in}}{\pgfqpoint{4.218750in}{2.172222in}}%
\pgfusepath{clip}%
\pgfsetbuttcap%
\pgfsetmiterjoin%
\definecolor{currentfill}{rgb}{0.447059,0.447059,0.447059}%
\pgfsetfillcolor{currentfill}%
\pgfsetlinewidth{1.003750pt}%
\definecolor{currentstroke}{rgb}{0.266667,0.266667,0.266667}%
\pgfsetstrokecolor{currentstroke}%
\pgfsetdash{}{0pt}%
\pgfpathmoveto{\pgfqpoint{1.689876in}{1.717962in}}%
\pgfpathlineto{\pgfqpoint{1.940779in}{1.717962in}}%
\pgfpathlineto{\pgfqpoint{1.940779in}{1.838586in}}%
\pgfpathlineto{\pgfqpoint{1.689876in}{1.838586in}}%
\pgfpathlineto{\pgfqpoint{1.689876in}{1.717962in}}%
\pgfpathclose%
\pgfusepath{stroke,fill}%
\end{pgfscope}%
\begin{pgfscope}%
\pgfpathrectangle{\pgfqpoint{0.781250in}{0.638889in}}{\pgfqpoint{4.218750in}{2.172222in}}%
\pgfusepath{clip}%
\pgfsetbuttcap%
\pgfsetmiterjoin%
\definecolor{currentfill}{rgb}{0.447059,0.447059,0.447059}%
\pgfsetfillcolor{currentfill}%
\pgfsetlinewidth{1.003750pt}%
\definecolor{currentstroke}{rgb}{0.266667,0.266667,0.266667}%
\pgfsetstrokecolor{currentstroke}%
\pgfsetdash{}{0pt}%
\pgfpathmoveto{\pgfqpoint{2.048309in}{1.663200in}}%
\pgfpathlineto{\pgfqpoint{2.299211in}{1.663200in}}%
\pgfpathlineto{\pgfqpoint{2.299211in}{1.784758in}}%
\pgfpathlineto{\pgfqpoint{2.048309in}{1.784758in}}%
\pgfpathlineto{\pgfqpoint{2.048309in}{1.663200in}}%
\pgfpathclose%
\pgfusepath{stroke,fill}%
\end{pgfscope}%
\begin{pgfscope}%
\pgfpathrectangle{\pgfqpoint{0.781250in}{0.638889in}}{\pgfqpoint{4.218750in}{2.172222in}}%
\pgfusepath{clip}%
\pgfsetbuttcap%
\pgfsetmiterjoin%
\definecolor{currentfill}{rgb}{0.447059,0.447059,0.447059}%
\pgfsetfillcolor{currentfill}%
\pgfsetlinewidth{1.003750pt}%
\definecolor{currentstroke}{rgb}{0.266667,0.266667,0.266667}%
\pgfsetstrokecolor{currentstroke}%
\pgfsetdash{}{0pt}%
\pgfpathmoveto{\pgfqpoint{2.406741in}{1.774223in}}%
\pgfpathlineto{\pgfqpoint{2.657644in}{1.774223in}}%
\pgfpathlineto{\pgfqpoint{2.657644in}{1.886570in}}%
\pgfpathlineto{\pgfqpoint{2.406741in}{1.886570in}}%
\pgfpathlineto{\pgfqpoint{2.406741in}{1.774223in}}%
\pgfpathclose%
\pgfusepath{stroke,fill}%
\end{pgfscope}%
\begin{pgfscope}%
\pgfpathrectangle{\pgfqpoint{0.781250in}{0.638889in}}{\pgfqpoint{4.218750in}{2.172222in}}%
\pgfusepath{clip}%
\pgfsetbuttcap%
\pgfsetmiterjoin%
\definecolor{currentfill}{rgb}{0.447059,0.447059,0.447059}%
\pgfsetfillcolor{currentfill}%
\pgfsetlinewidth{1.003750pt}%
\definecolor{currentstroke}{rgb}{0.266667,0.266667,0.266667}%
\pgfsetstrokecolor{currentstroke}%
\pgfsetdash{}{0pt}%
\pgfpathmoveto{\pgfqpoint{2.765174in}{1.950607in}}%
\pgfpathlineto{\pgfqpoint{3.016076in}{1.950607in}}%
\pgfpathlineto{\pgfqpoint{3.016076in}{2.079311in}}%
\pgfpathlineto{\pgfqpoint{2.765174in}{2.079311in}}%
\pgfpathlineto{\pgfqpoint{2.765174in}{1.950607in}}%
\pgfpathclose%
\pgfusepath{stroke,fill}%
\end{pgfscope}%
\begin{pgfscope}%
\pgfpathrectangle{\pgfqpoint{0.781250in}{0.638889in}}{\pgfqpoint{4.218750in}{2.172222in}}%
\pgfusepath{clip}%
\pgfsetbuttcap%
\pgfsetmiterjoin%
\definecolor{currentfill}{rgb}{0.447059,0.447059,0.447059}%
\pgfsetfillcolor{currentfill}%
\pgfsetlinewidth{1.003750pt}%
\definecolor{currentstroke}{rgb}{0.266667,0.266667,0.266667}%
\pgfsetstrokecolor{currentstroke}%
\pgfsetdash{}{0pt}%
\pgfpathmoveto{\pgfqpoint{3.123606in}{1.848817in}}%
\pgfpathlineto{\pgfqpoint{3.374509in}{1.848817in}}%
\pgfpathlineto{\pgfqpoint{3.374509in}{1.977868in}}%
\pgfpathlineto{\pgfqpoint{3.123606in}{1.977868in}}%
\pgfpathlineto{\pgfqpoint{3.123606in}{1.848817in}}%
\pgfpathclose%
\pgfusepath{stroke,fill}%
\end{pgfscope}%
\begin{pgfscope}%
\pgfpathrectangle{\pgfqpoint{0.781250in}{0.638889in}}{\pgfqpoint{4.218750in}{2.172222in}}%
\pgfusepath{clip}%
\pgfsetbuttcap%
\pgfsetmiterjoin%
\definecolor{currentfill}{rgb}{0.447059,0.447059,0.447059}%
\pgfsetfillcolor{currentfill}%
\pgfsetlinewidth{1.003750pt}%
\definecolor{currentstroke}{rgb}{0.266667,0.266667,0.266667}%
\pgfsetstrokecolor{currentstroke}%
\pgfsetdash{}{0pt}%
\pgfpathmoveto{\pgfqpoint{3.482039in}{1.839541in}}%
\pgfpathlineto{\pgfqpoint{3.732941in}{1.839541in}}%
\pgfpathlineto{\pgfqpoint{3.732941in}{1.954126in}}%
\pgfpathlineto{\pgfqpoint{3.482039in}{1.954126in}}%
\pgfpathlineto{\pgfqpoint{3.482039in}{1.839541in}}%
\pgfpathclose%
\pgfusepath{stroke,fill}%
\end{pgfscope}%
\begin{pgfscope}%
\pgfpathrectangle{\pgfqpoint{0.781250in}{0.638889in}}{\pgfqpoint{4.218750in}{2.172222in}}%
\pgfusepath{clip}%
\pgfsetbuttcap%
\pgfsetmiterjoin%
\definecolor{currentfill}{rgb}{0.447059,0.447059,0.447059}%
\pgfsetfillcolor{currentfill}%
\pgfsetlinewidth{1.003750pt}%
\definecolor{currentstroke}{rgb}{0.266667,0.266667,0.266667}%
\pgfsetstrokecolor{currentstroke}%
\pgfsetdash{}{0pt}%
\pgfpathmoveto{\pgfqpoint{3.840471in}{2.049964in}}%
\pgfpathlineto{\pgfqpoint{4.091374in}{2.049964in}}%
\pgfpathlineto{\pgfqpoint{4.091374in}{2.174650in}}%
\pgfpathlineto{\pgfqpoint{3.840471in}{2.174650in}}%
\pgfpathlineto{\pgfqpoint{3.840471in}{2.049964in}}%
\pgfpathclose%
\pgfusepath{stroke,fill}%
\end{pgfscope}%
\begin{pgfscope}%
\pgfpathrectangle{\pgfqpoint{0.781250in}{0.638889in}}{\pgfqpoint{4.218750in}{2.172222in}}%
\pgfusepath{clip}%
\pgfsetbuttcap%
\pgfsetmiterjoin%
\definecolor{currentfill}{rgb}{0.447059,0.447059,0.447059}%
\pgfsetfillcolor{currentfill}%
\pgfsetlinewidth{1.003750pt}%
\definecolor{currentstroke}{rgb}{0.266667,0.266667,0.266667}%
\pgfsetstrokecolor{currentstroke}%
\pgfsetdash{}{0pt}%
\pgfpathmoveto{\pgfqpoint{4.198903in}{2.107072in}}%
\pgfpathlineto{\pgfqpoint{4.449806in}{2.107072in}}%
\pgfpathlineto{\pgfqpoint{4.449806in}{2.214966in}}%
\pgfpathlineto{\pgfqpoint{4.198903in}{2.214966in}}%
\pgfpathlineto{\pgfqpoint{4.198903in}{2.107072in}}%
\pgfpathclose%
\pgfusepath{stroke,fill}%
\end{pgfscope}%
\begin{pgfscope}%
\pgfpathrectangle{\pgfqpoint{0.781250in}{0.638889in}}{\pgfqpoint{4.218750in}{2.172222in}}%
\pgfusepath{clip}%
\pgfsetbuttcap%
\pgfsetmiterjoin%
\definecolor{currentfill}{rgb}{0.447059,0.447059,0.447059}%
\pgfsetfillcolor{currentfill}%
\pgfsetlinewidth{1.003750pt}%
\definecolor{currentstroke}{rgb}{0.266667,0.266667,0.266667}%
\pgfsetstrokecolor{currentstroke}%
\pgfsetdash{}{0pt}%
\pgfpathmoveto{\pgfqpoint{4.557336in}{2.248158in}}%
\pgfpathlineto{\pgfqpoint{4.808239in}{2.248158in}}%
\pgfpathlineto{\pgfqpoint{4.808239in}{2.361657in}}%
\pgfpathlineto{\pgfqpoint{4.557336in}{2.361657in}}%
\pgfpathlineto{\pgfqpoint{4.557336in}{2.248158in}}%
\pgfpathclose%
\pgfusepath{stroke,fill}%
\end{pgfscope}%
\begin{pgfscope}%
\pgfpathrectangle{\pgfqpoint{0.781250in}{0.638889in}}{\pgfqpoint{4.218750in}{2.172222in}}%
\pgfusepath{clip}%
\pgfsetbuttcap%
\pgfsetmiterjoin%
\definecolor{currentfill}{rgb}{0.447059,0.447059,0.447059}%
\pgfsetfillcolor{currentfill}%
\pgfsetlinewidth{1.003750pt}%
\definecolor{currentstroke}{rgb}{0.266667,0.266667,0.266667}%
\pgfsetstrokecolor{currentstroke}%
\pgfsetdash{}{0pt}%
\pgfpathmoveto{\pgfqpoint{0.973011in}{2.396847in}}%
\pgfpathlineto{\pgfqpoint{1.223914in}{2.396847in}}%
\pgfpathlineto{\pgfqpoint{1.223914in}{2.539105in}}%
\pgfpathlineto{\pgfqpoint{0.973011in}{2.539105in}}%
\pgfpathlineto{\pgfqpoint{0.973011in}{2.396847in}}%
\pgfpathclose%
\pgfusepath{stroke,fill}%
\end{pgfscope}%
\begin{pgfscope}%
\pgfpathrectangle{\pgfqpoint{0.781250in}{0.638889in}}{\pgfqpoint{4.218750in}{2.172222in}}%
\pgfusepath{clip}%
\pgfsetbuttcap%
\pgfsetmiterjoin%
\definecolor{currentfill}{rgb}{0.447059,0.447059,0.447059}%
\pgfsetfillcolor{currentfill}%
\pgfsetlinewidth{1.003750pt}%
\definecolor{currentstroke}{rgb}{0.266667,0.266667,0.266667}%
\pgfsetstrokecolor{currentstroke}%
\pgfsetdash{}{0pt}%
\pgfpathmoveto{\pgfqpoint{1.331444in}{1.999200in}}%
\pgfpathlineto{\pgfqpoint{1.582347in}{1.999200in}}%
\pgfpathlineto{\pgfqpoint{1.582347in}{2.112047in}}%
\pgfpathlineto{\pgfqpoint{1.331444in}{2.112047in}}%
\pgfpathlineto{\pgfqpoint{1.331444in}{1.999200in}}%
\pgfpathclose%
\pgfusepath{stroke,fill}%
\end{pgfscope}%
\begin{pgfscope}%
\pgfpathrectangle{\pgfqpoint{0.781250in}{0.638889in}}{\pgfqpoint{4.218750in}{2.172222in}}%
\pgfusepath{clip}%
\pgfsetbuttcap%
\pgfsetmiterjoin%
\definecolor{currentfill}{rgb}{0.447059,0.447059,0.447059}%
\pgfsetfillcolor{currentfill}%
\pgfsetlinewidth{1.003750pt}%
\definecolor{currentstroke}{rgb}{0.266667,0.266667,0.266667}%
\pgfsetstrokecolor{currentstroke}%
\pgfsetdash{}{0pt}%
\pgfpathmoveto{\pgfqpoint{1.689876in}{1.838586in}}%
\pgfpathlineto{\pgfqpoint{1.940779in}{1.838586in}}%
\pgfpathlineto{\pgfqpoint{1.940779in}{1.937661in}}%
\pgfpathlineto{\pgfqpoint{1.689876in}{1.937661in}}%
\pgfpathlineto{\pgfqpoint{1.689876in}{1.838586in}}%
\pgfpathclose%
\pgfusepath{stroke,fill}%
\end{pgfscope}%
\begin{pgfscope}%
\pgfpathrectangle{\pgfqpoint{0.781250in}{0.638889in}}{\pgfqpoint{4.218750in}{2.172222in}}%
\pgfusepath{clip}%
\pgfsetbuttcap%
\pgfsetmiterjoin%
\definecolor{currentfill}{rgb}{0.447059,0.447059,0.447059}%
\pgfsetfillcolor{currentfill}%
\pgfsetlinewidth{1.003750pt}%
\definecolor{currentstroke}{rgb}{0.266667,0.266667,0.266667}%
\pgfsetstrokecolor{currentstroke}%
\pgfsetdash{}{0pt}%
\pgfpathmoveto{\pgfqpoint{2.048309in}{1.784758in}}%
\pgfpathlineto{\pgfqpoint{2.299211in}{1.784758in}}%
\pgfpathlineto{\pgfqpoint{2.299211in}{1.874710in}}%
\pgfpathlineto{\pgfqpoint{2.048309in}{1.874710in}}%
\pgfpathlineto{\pgfqpoint{2.048309in}{1.784758in}}%
\pgfpathclose%
\pgfusepath{stroke,fill}%
\end{pgfscope}%
\begin{pgfscope}%
\pgfpathrectangle{\pgfqpoint{0.781250in}{0.638889in}}{\pgfqpoint{4.218750in}{2.172222in}}%
\pgfusepath{clip}%
\pgfsetbuttcap%
\pgfsetmiterjoin%
\definecolor{currentfill}{rgb}{0.447059,0.447059,0.447059}%
\pgfsetfillcolor{currentfill}%
\pgfsetlinewidth{1.003750pt}%
\definecolor{currentstroke}{rgb}{0.266667,0.266667,0.266667}%
\pgfsetstrokecolor{currentstroke}%
\pgfsetdash{}{0pt}%
\pgfpathmoveto{\pgfqpoint{2.406741in}{1.886570in}}%
\pgfpathlineto{\pgfqpoint{2.657644in}{1.886570in}}%
\pgfpathlineto{\pgfqpoint{2.657644in}{1.979432in}}%
\pgfpathlineto{\pgfqpoint{2.406741in}{1.979432in}}%
\pgfpathlineto{\pgfqpoint{2.406741in}{1.886570in}}%
\pgfpathclose%
\pgfusepath{stroke,fill}%
\end{pgfscope}%
\begin{pgfscope}%
\pgfpathrectangle{\pgfqpoint{0.781250in}{0.638889in}}{\pgfqpoint{4.218750in}{2.172222in}}%
\pgfusepath{clip}%
\pgfsetbuttcap%
\pgfsetmiterjoin%
\definecolor{currentfill}{rgb}{0.447059,0.447059,0.447059}%
\pgfsetfillcolor{currentfill}%
\pgfsetlinewidth{1.003750pt}%
\definecolor{currentstroke}{rgb}{0.266667,0.266667,0.266667}%
\pgfsetstrokecolor{currentstroke}%
\pgfsetdash{}{0pt}%
\pgfpathmoveto{\pgfqpoint{2.765174in}{2.079311in}}%
\pgfpathlineto{\pgfqpoint{3.016076in}{2.079311in}}%
\pgfpathlineto{\pgfqpoint{3.016076in}{2.169437in}}%
\pgfpathlineto{\pgfqpoint{2.765174in}{2.169437in}}%
\pgfpathlineto{\pgfqpoint{2.765174in}{2.079311in}}%
\pgfpathclose%
\pgfusepath{stroke,fill}%
\end{pgfscope}%
\begin{pgfscope}%
\pgfpathrectangle{\pgfqpoint{0.781250in}{0.638889in}}{\pgfqpoint{4.218750in}{2.172222in}}%
\pgfusepath{clip}%
\pgfsetbuttcap%
\pgfsetmiterjoin%
\definecolor{currentfill}{rgb}{0.447059,0.447059,0.447059}%
\pgfsetfillcolor{currentfill}%
\pgfsetlinewidth{1.003750pt}%
\definecolor{currentstroke}{rgb}{0.266667,0.266667,0.266667}%
\pgfsetstrokecolor{currentstroke}%
\pgfsetdash{}{0pt}%
\pgfpathmoveto{\pgfqpoint{3.123606in}{1.977868in}}%
\pgfpathlineto{\pgfqpoint{3.374509in}{1.977868in}}%
\pgfpathlineto{\pgfqpoint{3.374509in}{2.062607in}}%
\pgfpathlineto{\pgfqpoint{3.123606in}{2.062607in}}%
\pgfpathlineto{\pgfqpoint{3.123606in}{1.977868in}}%
\pgfpathclose%
\pgfusepath{stroke,fill}%
\end{pgfscope}%
\begin{pgfscope}%
\pgfpathrectangle{\pgfqpoint{0.781250in}{0.638889in}}{\pgfqpoint{4.218750in}{2.172222in}}%
\pgfusepath{clip}%
\pgfsetbuttcap%
\pgfsetmiterjoin%
\definecolor{currentfill}{rgb}{0.447059,0.447059,0.447059}%
\pgfsetfillcolor{currentfill}%
\pgfsetlinewidth{1.003750pt}%
\definecolor{currentstroke}{rgb}{0.266667,0.266667,0.266667}%
\pgfsetstrokecolor{currentstroke}%
\pgfsetdash{}{0pt}%
\pgfpathmoveto{\pgfqpoint{3.482039in}{1.954126in}}%
\pgfpathlineto{\pgfqpoint{3.732941in}{1.954126in}}%
\pgfpathlineto{\pgfqpoint{3.732941in}{2.031370in}}%
\pgfpathlineto{\pgfqpoint{3.482039in}{2.031370in}}%
\pgfpathlineto{\pgfqpoint{3.482039in}{1.954126in}}%
\pgfpathclose%
\pgfusepath{stroke,fill}%
\end{pgfscope}%
\begin{pgfscope}%
\pgfpathrectangle{\pgfqpoint{0.781250in}{0.638889in}}{\pgfqpoint{4.218750in}{2.172222in}}%
\pgfusepath{clip}%
\pgfsetbuttcap%
\pgfsetmiterjoin%
\definecolor{currentfill}{rgb}{0.447059,0.447059,0.447059}%
\pgfsetfillcolor{currentfill}%
\pgfsetlinewidth{1.003750pt}%
\definecolor{currentstroke}{rgb}{0.266667,0.266667,0.266667}%
\pgfsetstrokecolor{currentstroke}%
\pgfsetdash{}{0pt}%
\pgfpathmoveto{\pgfqpoint{3.840471in}{2.174650in}}%
\pgfpathlineto{\pgfqpoint{4.091374in}{2.174650in}}%
\pgfpathlineto{\pgfqpoint{4.091374in}{2.255956in}}%
\pgfpathlineto{\pgfqpoint{3.840471in}{2.255956in}}%
\pgfpathlineto{\pgfqpoint{3.840471in}{2.174650in}}%
\pgfpathclose%
\pgfusepath{stroke,fill}%
\end{pgfscope}%
\begin{pgfscope}%
\pgfpathrectangle{\pgfqpoint{0.781250in}{0.638889in}}{\pgfqpoint{4.218750in}{2.172222in}}%
\pgfusepath{clip}%
\pgfsetbuttcap%
\pgfsetmiterjoin%
\definecolor{currentfill}{rgb}{0.447059,0.447059,0.447059}%
\pgfsetfillcolor{currentfill}%
\pgfsetlinewidth{1.003750pt}%
\definecolor{currentstroke}{rgb}{0.266667,0.266667,0.266667}%
\pgfsetstrokecolor{currentstroke}%
\pgfsetdash{}{0pt}%
\pgfpathmoveto{\pgfqpoint{4.198903in}{2.214966in}}%
\pgfpathlineto{\pgfqpoint{4.449806in}{2.214966in}}%
\pgfpathlineto{\pgfqpoint{4.449806in}{2.295404in}}%
\pgfpathlineto{\pgfqpoint{4.198903in}{2.295404in}}%
\pgfpathlineto{\pgfqpoint{4.198903in}{2.214966in}}%
\pgfpathclose%
\pgfusepath{stroke,fill}%
\end{pgfscope}%
\begin{pgfscope}%
\pgfpathrectangle{\pgfqpoint{0.781250in}{0.638889in}}{\pgfqpoint{4.218750in}{2.172222in}}%
\pgfusepath{clip}%
\pgfsetbuttcap%
\pgfsetmiterjoin%
\definecolor{currentfill}{rgb}{0.447059,0.447059,0.447059}%
\pgfsetfillcolor{currentfill}%
\pgfsetlinewidth{1.003750pt}%
\definecolor{currentstroke}{rgb}{0.266667,0.266667,0.266667}%
\pgfsetstrokecolor{currentstroke}%
\pgfsetdash{}{0pt}%
\pgfpathmoveto{\pgfqpoint{4.557336in}{2.361657in}}%
\pgfpathlineto{\pgfqpoint{4.808239in}{2.361657in}}%
\pgfpathlineto{\pgfqpoint{4.808239in}{2.444092in}}%
\pgfpathlineto{\pgfqpoint{4.557336in}{2.444092in}}%
\pgfpathlineto{\pgfqpoint{4.557336in}{2.361657in}}%
\pgfpathclose%
\pgfusepath{stroke,fill}%
\end{pgfscope}%
\begin{pgfscope}%
\pgfpathrectangle{\pgfqpoint{0.781250in}{0.638889in}}{\pgfqpoint{4.218750in}{2.172222in}}%
\pgfusepath{clip}%
\pgfsetbuttcap%
\pgfsetmiterjoin%
\definecolor{currentfill}{rgb}{0.447059,0.447059,0.447059}%
\pgfsetfillcolor{currentfill}%
\pgfsetlinewidth{1.003750pt}%
\definecolor{currentstroke}{rgb}{0.266667,0.266667,0.266667}%
\pgfsetstrokecolor{currentstroke}%
\pgfsetdash{}{0pt}%
\pgfpathmoveto{\pgfqpoint{0.973011in}{2.539105in}}%
\pgfpathlineto{\pgfqpoint{1.223914in}{2.539105in}}%
\pgfpathlineto{\pgfqpoint{1.223914in}{2.551856in}}%
\pgfpathlineto{\pgfqpoint{0.973011in}{2.551856in}}%
\pgfpathlineto{\pgfqpoint{0.973011in}{2.539105in}}%
\pgfpathclose%
\pgfusepath{stroke,fill}%
\end{pgfscope}%
\begin{pgfscope}%
\pgfpathrectangle{\pgfqpoint{0.781250in}{0.638889in}}{\pgfqpoint{4.218750in}{2.172222in}}%
\pgfusepath{clip}%
\pgfsetbuttcap%
\pgfsetmiterjoin%
\definecolor{currentfill}{rgb}{0.447059,0.447059,0.447059}%
\pgfsetfillcolor{currentfill}%
\pgfsetlinewidth{1.003750pt}%
\definecolor{currentstroke}{rgb}{0.266667,0.266667,0.266667}%
\pgfsetstrokecolor{currentstroke}%
\pgfsetdash{}{0pt}%
\pgfpathmoveto{\pgfqpoint{1.331444in}{2.112047in}}%
\pgfpathlineto{\pgfqpoint{1.582347in}{2.112047in}}%
\pgfpathlineto{\pgfqpoint{1.582347in}{2.123690in}}%
\pgfpathlineto{\pgfqpoint{1.331444in}{2.123690in}}%
\pgfpathlineto{\pgfqpoint{1.331444in}{2.112047in}}%
\pgfpathclose%
\pgfusepath{stroke,fill}%
\end{pgfscope}%
\begin{pgfscope}%
\pgfpathrectangle{\pgfqpoint{0.781250in}{0.638889in}}{\pgfqpoint{4.218750in}{2.172222in}}%
\pgfusepath{clip}%
\pgfsetbuttcap%
\pgfsetmiterjoin%
\definecolor{currentfill}{rgb}{0.447059,0.447059,0.447059}%
\pgfsetfillcolor{currentfill}%
\pgfsetlinewidth{1.003750pt}%
\definecolor{currentstroke}{rgb}{0.266667,0.266667,0.266667}%
\pgfsetstrokecolor{currentstroke}%
\pgfsetdash{}{0pt}%
\pgfpathmoveto{\pgfqpoint{1.689876in}{1.937661in}}%
\pgfpathlineto{\pgfqpoint{1.940779in}{1.937661in}}%
\pgfpathlineto{\pgfqpoint{1.940779in}{1.943982in}}%
\pgfpathlineto{\pgfqpoint{1.689876in}{1.943982in}}%
\pgfpathlineto{\pgfqpoint{1.689876in}{1.937661in}}%
\pgfpathclose%
\pgfusepath{stroke,fill}%
\end{pgfscope}%
\begin{pgfscope}%
\pgfpathrectangle{\pgfqpoint{0.781250in}{0.638889in}}{\pgfqpoint{4.218750in}{2.172222in}}%
\pgfusepath{clip}%
\pgfsetbuttcap%
\pgfsetmiterjoin%
\definecolor{currentfill}{rgb}{0.447059,0.447059,0.447059}%
\pgfsetfillcolor{currentfill}%
\pgfsetlinewidth{1.003750pt}%
\definecolor{currentstroke}{rgb}{0.266667,0.266667,0.266667}%
\pgfsetstrokecolor{currentstroke}%
\pgfsetdash{}{0pt}%
\pgfpathmoveto{\pgfqpoint{2.048309in}{1.874710in}}%
\pgfpathlineto{\pgfqpoint{2.299211in}{1.874710in}}%
\pgfpathlineto{\pgfqpoint{2.299211in}{1.880879in}}%
\pgfpathlineto{\pgfqpoint{2.048309in}{1.880879in}}%
\pgfpathlineto{\pgfqpoint{2.048309in}{1.874710in}}%
\pgfpathclose%
\pgfusepath{stroke,fill}%
\end{pgfscope}%
\begin{pgfscope}%
\pgfpathrectangle{\pgfqpoint{0.781250in}{0.638889in}}{\pgfqpoint{4.218750in}{2.172222in}}%
\pgfusepath{clip}%
\pgfsetbuttcap%
\pgfsetmiterjoin%
\definecolor{currentfill}{rgb}{0.447059,0.447059,0.447059}%
\pgfsetfillcolor{currentfill}%
\pgfsetlinewidth{1.003750pt}%
\definecolor{currentstroke}{rgb}{0.266667,0.266667,0.266667}%
\pgfsetstrokecolor{currentstroke}%
\pgfsetdash{}{0pt}%
\pgfpathmoveto{\pgfqpoint{2.406741in}{1.979432in}}%
\pgfpathlineto{\pgfqpoint{2.657644in}{1.979432in}}%
\pgfpathlineto{\pgfqpoint{2.657644in}{1.985058in}}%
\pgfpathlineto{\pgfqpoint{2.406741in}{1.985058in}}%
\pgfpathlineto{\pgfqpoint{2.406741in}{1.979432in}}%
\pgfpathclose%
\pgfusepath{stroke,fill}%
\end{pgfscope}%
\begin{pgfscope}%
\pgfpathrectangle{\pgfqpoint{0.781250in}{0.638889in}}{\pgfqpoint{4.218750in}{2.172222in}}%
\pgfusepath{clip}%
\pgfsetbuttcap%
\pgfsetmiterjoin%
\definecolor{currentfill}{rgb}{0.447059,0.447059,0.447059}%
\pgfsetfillcolor{currentfill}%
\pgfsetlinewidth{1.003750pt}%
\definecolor{currentstroke}{rgb}{0.266667,0.266667,0.266667}%
\pgfsetstrokecolor{currentstroke}%
\pgfsetdash{}{0pt}%
\pgfpathmoveto{\pgfqpoint{2.765174in}{2.169437in}}%
\pgfpathlineto{\pgfqpoint{3.016076in}{2.169437in}}%
\pgfpathlineto{\pgfqpoint{3.016076in}{2.175562in}}%
\pgfpathlineto{\pgfqpoint{2.765174in}{2.175562in}}%
\pgfpathlineto{\pgfqpoint{2.765174in}{2.169437in}}%
\pgfpathclose%
\pgfusepath{stroke,fill}%
\end{pgfscope}%
\begin{pgfscope}%
\pgfpathrectangle{\pgfqpoint{0.781250in}{0.638889in}}{\pgfqpoint{4.218750in}{2.172222in}}%
\pgfusepath{clip}%
\pgfsetbuttcap%
\pgfsetmiterjoin%
\definecolor{currentfill}{rgb}{0.447059,0.447059,0.447059}%
\pgfsetfillcolor{currentfill}%
\pgfsetlinewidth{1.003750pt}%
\definecolor{currentstroke}{rgb}{0.266667,0.266667,0.266667}%
\pgfsetstrokecolor{currentstroke}%
\pgfsetdash{}{0pt}%
\pgfpathmoveto{\pgfqpoint{3.123606in}{2.062607in}}%
\pgfpathlineto{\pgfqpoint{3.374509in}{2.062607in}}%
\pgfpathlineto{\pgfqpoint{3.374509in}{2.070861in}}%
\pgfpathlineto{\pgfqpoint{3.123606in}{2.070861in}}%
\pgfpathlineto{\pgfqpoint{3.123606in}{2.062607in}}%
\pgfpathclose%
\pgfusepath{stroke,fill}%
\end{pgfscope}%
\begin{pgfscope}%
\pgfpathrectangle{\pgfqpoint{0.781250in}{0.638889in}}{\pgfqpoint{4.218750in}{2.172222in}}%
\pgfusepath{clip}%
\pgfsetbuttcap%
\pgfsetmiterjoin%
\definecolor{currentfill}{rgb}{0.447059,0.447059,0.447059}%
\pgfsetfillcolor{currentfill}%
\pgfsetlinewidth{1.003750pt}%
\definecolor{currentstroke}{rgb}{0.266667,0.266667,0.266667}%
\pgfsetstrokecolor{currentstroke}%
\pgfsetdash{}{0pt}%
\pgfpathmoveto{\pgfqpoint{3.482039in}{2.031370in}}%
\pgfpathlineto{\pgfqpoint{3.732941in}{2.031370in}}%
\pgfpathlineto{\pgfqpoint{3.732941in}{2.039342in}}%
\pgfpathlineto{\pgfqpoint{3.482039in}{2.039342in}}%
\pgfpathlineto{\pgfqpoint{3.482039in}{2.031370in}}%
\pgfpathclose%
\pgfusepath{stroke,fill}%
\end{pgfscope}%
\begin{pgfscope}%
\pgfpathrectangle{\pgfqpoint{0.781250in}{0.638889in}}{\pgfqpoint{4.218750in}{2.172222in}}%
\pgfusepath{clip}%
\pgfsetbuttcap%
\pgfsetmiterjoin%
\definecolor{currentfill}{rgb}{0.447059,0.447059,0.447059}%
\pgfsetfillcolor{currentfill}%
\pgfsetlinewidth{1.003750pt}%
\definecolor{currentstroke}{rgb}{0.266667,0.266667,0.266667}%
\pgfsetstrokecolor{currentstroke}%
\pgfsetdash{}{0pt}%
\pgfpathmoveto{\pgfqpoint{3.840471in}{2.255956in}}%
\pgfpathlineto{\pgfqpoint{4.091374in}{2.255956in}}%
\pgfpathlineto{\pgfqpoint{4.091374in}{2.264645in}}%
\pgfpathlineto{\pgfqpoint{3.840471in}{2.264645in}}%
\pgfpathlineto{\pgfqpoint{3.840471in}{2.255956in}}%
\pgfpathclose%
\pgfusepath{stroke,fill}%
\end{pgfscope}%
\begin{pgfscope}%
\pgfpathrectangle{\pgfqpoint{0.781250in}{0.638889in}}{\pgfqpoint{4.218750in}{2.172222in}}%
\pgfusepath{clip}%
\pgfsetbuttcap%
\pgfsetmiterjoin%
\definecolor{currentfill}{rgb}{0.447059,0.447059,0.447059}%
\pgfsetfillcolor{currentfill}%
\pgfsetlinewidth{1.003750pt}%
\definecolor{currentstroke}{rgb}{0.266667,0.266667,0.266667}%
\pgfsetstrokecolor{currentstroke}%
\pgfsetdash{}{0pt}%
\pgfpathmoveto{\pgfqpoint{4.198903in}{2.295404in}}%
\pgfpathlineto{\pgfqpoint{4.449806in}{2.295404in}}%
\pgfpathlineto{\pgfqpoint{4.449806in}{2.304071in}}%
\pgfpathlineto{\pgfqpoint{4.198903in}{2.304071in}}%
\pgfpathlineto{\pgfqpoint{4.198903in}{2.295404in}}%
\pgfpathclose%
\pgfusepath{stroke,fill}%
\end{pgfscope}%
\begin{pgfscope}%
\pgfpathrectangle{\pgfqpoint{0.781250in}{0.638889in}}{\pgfqpoint{4.218750in}{2.172222in}}%
\pgfusepath{clip}%
\pgfsetbuttcap%
\pgfsetmiterjoin%
\definecolor{currentfill}{rgb}{0.447059,0.447059,0.447059}%
\pgfsetfillcolor{currentfill}%
\pgfsetlinewidth{1.003750pt}%
\definecolor{currentstroke}{rgb}{0.266667,0.266667,0.266667}%
\pgfsetstrokecolor{currentstroke}%
\pgfsetdash{}{0pt}%
\pgfpathmoveto{\pgfqpoint{4.557336in}{2.444092in}}%
\pgfpathlineto{\pgfqpoint{4.808239in}{2.444092in}}%
\pgfpathlineto{\pgfqpoint{4.808239in}{2.453172in}}%
\pgfpathlineto{\pgfqpoint{4.557336in}{2.453172in}}%
\pgfpathlineto{\pgfqpoint{4.557336in}{2.444092in}}%
\pgfpathclose%
\pgfusepath{stroke,fill}%
\end{pgfscope}%
\begin{pgfscope}%
\pgfpathrectangle{\pgfqpoint{0.781250in}{0.638889in}}{\pgfqpoint{4.218750in}{2.172222in}}%
\pgfusepath{clip}%
\pgfsetbuttcap%
\pgfsetmiterjoin%
\definecolor{currentfill}{rgb}{0.447059,0.447059,0.447059}%
\pgfsetfillcolor{currentfill}%
\pgfsetlinewidth{1.003750pt}%
\definecolor{currentstroke}{rgb}{0.266667,0.266667,0.266667}%
\pgfsetstrokecolor{currentstroke}%
\pgfsetdash{}{0pt}%
\pgfpathmoveto{\pgfqpoint{0.973011in}{2.551856in}}%
\pgfpathlineto{\pgfqpoint{1.223914in}{2.551856in}}%
\pgfpathlineto{\pgfqpoint{1.223914in}{2.557982in}}%
\pgfpathlineto{\pgfqpoint{0.973011in}{2.557982in}}%
\pgfpathlineto{\pgfqpoint{0.973011in}{2.551856in}}%
\pgfpathclose%
\pgfusepath{stroke,fill}%
\end{pgfscope}%
\begin{pgfscope}%
\pgfpathrectangle{\pgfqpoint{0.781250in}{0.638889in}}{\pgfqpoint{4.218750in}{2.172222in}}%
\pgfusepath{clip}%
\pgfsetbuttcap%
\pgfsetmiterjoin%
\definecolor{currentfill}{rgb}{0.447059,0.447059,0.447059}%
\pgfsetfillcolor{currentfill}%
\pgfsetlinewidth{1.003750pt}%
\definecolor{currentstroke}{rgb}{0.266667,0.266667,0.266667}%
\pgfsetstrokecolor{currentstroke}%
\pgfsetdash{}{0pt}%
\pgfpathmoveto{\pgfqpoint{1.331444in}{2.123690in}}%
\pgfpathlineto{\pgfqpoint{1.582347in}{2.123690in}}%
\pgfpathlineto{\pgfqpoint{1.582347in}{2.128078in}}%
\pgfpathlineto{\pgfqpoint{1.331444in}{2.128078in}}%
\pgfpathlineto{\pgfqpoint{1.331444in}{2.123690in}}%
\pgfpathclose%
\pgfusepath{stroke,fill}%
\end{pgfscope}%
\begin{pgfscope}%
\pgfpathrectangle{\pgfqpoint{0.781250in}{0.638889in}}{\pgfqpoint{4.218750in}{2.172222in}}%
\pgfusepath{clip}%
\pgfsetbuttcap%
\pgfsetmiterjoin%
\definecolor{currentfill}{rgb}{0.447059,0.447059,0.447059}%
\pgfsetfillcolor{currentfill}%
\pgfsetlinewidth{1.003750pt}%
\definecolor{currentstroke}{rgb}{0.266667,0.266667,0.266667}%
\pgfsetstrokecolor{currentstroke}%
\pgfsetdash{}{0pt}%
\pgfpathmoveto{\pgfqpoint{1.689876in}{1.943982in}}%
\pgfpathlineto{\pgfqpoint{1.940779in}{1.943982in}}%
\pgfpathlineto{\pgfqpoint{1.940779in}{1.950020in}}%
\pgfpathlineto{\pgfqpoint{1.689876in}{1.950020in}}%
\pgfpathlineto{\pgfqpoint{1.689876in}{1.943982in}}%
\pgfpathclose%
\pgfusepath{stroke,fill}%
\end{pgfscope}%
\begin{pgfscope}%
\pgfpathrectangle{\pgfqpoint{0.781250in}{0.638889in}}{\pgfqpoint{4.218750in}{2.172222in}}%
\pgfusepath{clip}%
\pgfsetbuttcap%
\pgfsetmiterjoin%
\definecolor{currentfill}{rgb}{0.447059,0.447059,0.447059}%
\pgfsetfillcolor{currentfill}%
\pgfsetlinewidth{1.003750pt}%
\definecolor{currentstroke}{rgb}{0.266667,0.266667,0.266667}%
\pgfsetstrokecolor{currentstroke}%
\pgfsetdash{}{0pt}%
\pgfpathmoveto{\pgfqpoint{2.048309in}{1.880879in}}%
\pgfpathlineto{\pgfqpoint{2.299211in}{1.880879in}}%
\pgfpathlineto{\pgfqpoint{2.299211in}{1.884941in}}%
\pgfpathlineto{\pgfqpoint{2.048309in}{1.884941in}}%
\pgfpathlineto{\pgfqpoint{2.048309in}{1.880879in}}%
\pgfpathclose%
\pgfusepath{stroke,fill}%
\end{pgfscope}%
\begin{pgfscope}%
\pgfpathrectangle{\pgfqpoint{0.781250in}{0.638889in}}{\pgfqpoint{4.218750in}{2.172222in}}%
\pgfusepath{clip}%
\pgfsetbuttcap%
\pgfsetmiterjoin%
\definecolor{currentfill}{rgb}{0.447059,0.447059,0.447059}%
\pgfsetfillcolor{currentfill}%
\pgfsetlinewidth{1.003750pt}%
\definecolor{currentstroke}{rgb}{0.266667,0.266667,0.266667}%
\pgfsetstrokecolor{currentstroke}%
\pgfsetdash{}{0pt}%
\pgfpathmoveto{\pgfqpoint{2.406741in}{1.985058in}}%
\pgfpathlineto{\pgfqpoint{2.657644in}{1.985058in}}%
\pgfpathlineto{\pgfqpoint{2.657644in}{1.988708in}}%
\pgfpathlineto{\pgfqpoint{2.406741in}{1.988708in}}%
\pgfpathlineto{\pgfqpoint{2.406741in}{1.985058in}}%
\pgfpathclose%
\pgfusepath{stroke,fill}%
\end{pgfscope}%
\begin{pgfscope}%
\pgfpathrectangle{\pgfqpoint{0.781250in}{0.638889in}}{\pgfqpoint{4.218750in}{2.172222in}}%
\pgfusepath{clip}%
\pgfsetbuttcap%
\pgfsetmiterjoin%
\definecolor{currentfill}{rgb}{0.447059,0.447059,0.447059}%
\pgfsetfillcolor{currentfill}%
\pgfsetlinewidth{1.003750pt}%
\definecolor{currentstroke}{rgb}{0.266667,0.266667,0.266667}%
\pgfsetstrokecolor{currentstroke}%
\pgfsetdash{}{0pt}%
\pgfpathmoveto{\pgfqpoint{2.765174in}{2.175562in}}%
\pgfpathlineto{\pgfqpoint{3.016076in}{2.175562in}}%
\pgfpathlineto{\pgfqpoint{3.016076in}{2.179603in}}%
\pgfpathlineto{\pgfqpoint{2.765174in}{2.179603in}}%
\pgfpathlineto{\pgfqpoint{2.765174in}{2.175562in}}%
\pgfpathclose%
\pgfusepath{stroke,fill}%
\end{pgfscope}%
\begin{pgfscope}%
\pgfpathrectangle{\pgfqpoint{0.781250in}{0.638889in}}{\pgfqpoint{4.218750in}{2.172222in}}%
\pgfusepath{clip}%
\pgfsetbuttcap%
\pgfsetmiterjoin%
\definecolor{currentfill}{rgb}{0.447059,0.447059,0.447059}%
\pgfsetfillcolor{currentfill}%
\pgfsetlinewidth{1.003750pt}%
\definecolor{currentstroke}{rgb}{0.266667,0.266667,0.266667}%
\pgfsetstrokecolor{currentstroke}%
\pgfsetdash{}{0pt}%
\pgfpathmoveto{\pgfqpoint{3.123606in}{2.070861in}}%
\pgfpathlineto{\pgfqpoint{3.374509in}{2.070861in}}%
\pgfpathlineto{\pgfqpoint{3.374509in}{2.074424in}}%
\pgfpathlineto{\pgfqpoint{3.123606in}{2.074424in}}%
\pgfpathlineto{\pgfqpoint{3.123606in}{2.070861in}}%
\pgfpathclose%
\pgfusepath{stroke,fill}%
\end{pgfscope}%
\begin{pgfscope}%
\pgfpathrectangle{\pgfqpoint{0.781250in}{0.638889in}}{\pgfqpoint{4.218750in}{2.172222in}}%
\pgfusepath{clip}%
\pgfsetbuttcap%
\pgfsetmiterjoin%
\definecolor{currentfill}{rgb}{0.447059,0.447059,0.447059}%
\pgfsetfillcolor{currentfill}%
\pgfsetlinewidth{1.003750pt}%
\definecolor{currentstroke}{rgb}{0.266667,0.266667,0.266667}%
\pgfsetstrokecolor{currentstroke}%
\pgfsetdash{}{0pt}%
\pgfpathmoveto{\pgfqpoint{3.482039in}{2.039342in}}%
\pgfpathlineto{\pgfqpoint{3.732941in}{2.039342in}}%
\pgfpathlineto{\pgfqpoint{3.732941in}{2.042362in}}%
\pgfpathlineto{\pgfqpoint{3.482039in}{2.042362in}}%
\pgfpathlineto{\pgfqpoint{3.482039in}{2.039342in}}%
\pgfpathclose%
\pgfusepath{stroke,fill}%
\end{pgfscope}%
\begin{pgfscope}%
\pgfpathrectangle{\pgfqpoint{0.781250in}{0.638889in}}{\pgfqpoint{4.218750in}{2.172222in}}%
\pgfusepath{clip}%
\pgfsetbuttcap%
\pgfsetmiterjoin%
\definecolor{currentfill}{rgb}{0.447059,0.447059,0.447059}%
\pgfsetfillcolor{currentfill}%
\pgfsetlinewidth{1.003750pt}%
\definecolor{currentstroke}{rgb}{0.266667,0.266667,0.266667}%
\pgfsetstrokecolor{currentstroke}%
\pgfsetdash{}{0pt}%
\pgfpathmoveto{\pgfqpoint{3.840471in}{2.264645in}}%
\pgfpathlineto{\pgfqpoint{4.091374in}{2.264645in}}%
\pgfpathlineto{\pgfqpoint{4.091374in}{2.268599in}}%
\pgfpathlineto{\pgfqpoint{3.840471in}{2.268599in}}%
\pgfpathlineto{\pgfqpoint{3.840471in}{2.264645in}}%
\pgfpathclose%
\pgfusepath{stroke,fill}%
\end{pgfscope}%
\begin{pgfscope}%
\pgfpathrectangle{\pgfqpoint{0.781250in}{0.638889in}}{\pgfqpoint{4.218750in}{2.172222in}}%
\pgfusepath{clip}%
\pgfsetbuttcap%
\pgfsetmiterjoin%
\definecolor{currentfill}{rgb}{0.447059,0.447059,0.447059}%
\pgfsetfillcolor{currentfill}%
\pgfsetlinewidth{1.003750pt}%
\definecolor{currentstroke}{rgb}{0.266667,0.266667,0.266667}%
\pgfsetstrokecolor{currentstroke}%
\pgfsetdash{}{0pt}%
\pgfpathmoveto{\pgfqpoint{4.198903in}{2.304071in}}%
\pgfpathlineto{\pgfqpoint{4.449806in}{2.304071in}}%
\pgfpathlineto{\pgfqpoint{4.449806in}{2.310305in}}%
\pgfpathlineto{\pgfqpoint{4.198903in}{2.310305in}}%
\pgfpathlineto{\pgfqpoint{4.198903in}{2.304071in}}%
\pgfpathclose%
\pgfusepath{stroke,fill}%
\end{pgfscope}%
\begin{pgfscope}%
\pgfpathrectangle{\pgfqpoint{0.781250in}{0.638889in}}{\pgfqpoint{4.218750in}{2.172222in}}%
\pgfusepath{clip}%
\pgfsetbuttcap%
\pgfsetmiterjoin%
\definecolor{currentfill}{rgb}{0.447059,0.447059,0.447059}%
\pgfsetfillcolor{currentfill}%
\pgfsetlinewidth{1.003750pt}%
\definecolor{currentstroke}{rgb}{0.266667,0.266667,0.266667}%
\pgfsetstrokecolor{currentstroke}%
\pgfsetdash{}{0pt}%
\pgfpathmoveto{\pgfqpoint{4.557336in}{2.453172in}}%
\pgfpathlineto{\pgfqpoint{4.808239in}{2.453172in}}%
\pgfpathlineto{\pgfqpoint{4.808239in}{2.461057in}}%
\pgfpathlineto{\pgfqpoint{4.557336in}{2.461057in}}%
\pgfpathlineto{\pgfqpoint{4.557336in}{2.453172in}}%
\pgfpathclose%
\pgfusepath{stroke,fill}%
\end{pgfscope}%
\begin{pgfscope}%
\pgfpathrectangle{\pgfqpoint{0.781250in}{0.638889in}}{\pgfqpoint{4.218750in}{2.172222in}}%
\pgfusepath{clip}%
\pgfsetbuttcap%
\pgfsetmiterjoin%
\definecolor{currentfill}{rgb}{0.447059,0.447059,0.447059}%
\pgfsetfillcolor{currentfill}%
\pgfsetlinewidth{1.003750pt}%
\definecolor{currentstroke}{rgb}{0.266667,0.266667,0.266667}%
\pgfsetstrokecolor{currentstroke}%
\pgfsetdash{}{0pt}%
\pgfpathmoveto{\pgfqpoint{0.973011in}{2.557982in}}%
\pgfpathlineto{\pgfqpoint{1.223914in}{2.557982in}}%
\pgfpathlineto{\pgfqpoint{1.223914in}{2.563630in}}%
\pgfpathlineto{\pgfqpoint{0.973011in}{2.563630in}}%
\pgfpathlineto{\pgfqpoint{0.973011in}{2.557982in}}%
\pgfpathclose%
\pgfusepath{stroke,fill}%
\end{pgfscope}%
\begin{pgfscope}%
\pgfpathrectangle{\pgfqpoint{0.781250in}{0.638889in}}{\pgfqpoint{4.218750in}{2.172222in}}%
\pgfusepath{clip}%
\pgfsetbuttcap%
\pgfsetmiterjoin%
\definecolor{currentfill}{rgb}{0.447059,0.447059,0.447059}%
\pgfsetfillcolor{currentfill}%
\pgfsetlinewidth{1.003750pt}%
\definecolor{currentstroke}{rgb}{0.266667,0.266667,0.266667}%
\pgfsetstrokecolor{currentstroke}%
\pgfsetdash{}{0pt}%
\pgfpathmoveto{\pgfqpoint{1.331444in}{2.128078in}}%
\pgfpathlineto{\pgfqpoint{1.582347in}{2.128078in}}%
\pgfpathlineto{\pgfqpoint{1.582347in}{2.133008in}}%
\pgfpathlineto{\pgfqpoint{1.331444in}{2.133008in}}%
\pgfpathlineto{\pgfqpoint{1.331444in}{2.128078in}}%
\pgfpathclose%
\pgfusepath{stroke,fill}%
\end{pgfscope}%
\begin{pgfscope}%
\pgfpathrectangle{\pgfqpoint{0.781250in}{0.638889in}}{\pgfqpoint{4.218750in}{2.172222in}}%
\pgfusepath{clip}%
\pgfsetbuttcap%
\pgfsetmiterjoin%
\definecolor{currentfill}{rgb}{0.447059,0.447059,0.447059}%
\pgfsetfillcolor{currentfill}%
\pgfsetlinewidth{1.003750pt}%
\definecolor{currentstroke}{rgb}{0.266667,0.266667,0.266667}%
\pgfsetstrokecolor{currentstroke}%
\pgfsetdash{}{0pt}%
\pgfpathmoveto{\pgfqpoint{1.689876in}{1.950020in}}%
\pgfpathlineto{\pgfqpoint{1.940779in}{1.950020in}}%
\pgfpathlineto{\pgfqpoint{1.940779in}{1.955169in}}%
\pgfpathlineto{\pgfqpoint{1.689876in}{1.955169in}}%
\pgfpathlineto{\pgfqpoint{1.689876in}{1.950020in}}%
\pgfpathclose%
\pgfusepath{stroke,fill}%
\end{pgfscope}%
\begin{pgfscope}%
\pgfpathrectangle{\pgfqpoint{0.781250in}{0.638889in}}{\pgfqpoint{4.218750in}{2.172222in}}%
\pgfusepath{clip}%
\pgfsetbuttcap%
\pgfsetmiterjoin%
\definecolor{currentfill}{rgb}{0.447059,0.447059,0.447059}%
\pgfsetfillcolor{currentfill}%
\pgfsetlinewidth{1.003750pt}%
\definecolor{currentstroke}{rgb}{0.266667,0.266667,0.266667}%
\pgfsetstrokecolor{currentstroke}%
\pgfsetdash{}{0pt}%
\pgfpathmoveto{\pgfqpoint{2.048309in}{1.884941in}}%
\pgfpathlineto{\pgfqpoint{2.299211in}{1.884941in}}%
\pgfpathlineto{\pgfqpoint{2.299211in}{1.889633in}}%
\pgfpathlineto{\pgfqpoint{2.048309in}{1.889633in}}%
\pgfpathlineto{\pgfqpoint{2.048309in}{1.884941in}}%
\pgfpathclose%
\pgfusepath{stroke,fill}%
\end{pgfscope}%
\begin{pgfscope}%
\pgfpathrectangle{\pgfqpoint{0.781250in}{0.638889in}}{\pgfqpoint{4.218750in}{2.172222in}}%
\pgfusepath{clip}%
\pgfsetbuttcap%
\pgfsetmiterjoin%
\definecolor{currentfill}{rgb}{0.447059,0.447059,0.447059}%
\pgfsetfillcolor{currentfill}%
\pgfsetlinewidth{1.003750pt}%
\definecolor{currentstroke}{rgb}{0.266667,0.266667,0.266667}%
\pgfsetstrokecolor{currentstroke}%
\pgfsetdash{}{0pt}%
\pgfpathmoveto{\pgfqpoint{2.406741in}{1.988708in}}%
\pgfpathlineto{\pgfqpoint{2.657644in}{1.988708in}}%
\pgfpathlineto{\pgfqpoint{2.657644in}{1.993791in}}%
\pgfpathlineto{\pgfqpoint{2.406741in}{1.993791in}}%
\pgfpathlineto{\pgfqpoint{2.406741in}{1.988708in}}%
\pgfpathclose%
\pgfusepath{stroke,fill}%
\end{pgfscope}%
\begin{pgfscope}%
\pgfpathrectangle{\pgfqpoint{0.781250in}{0.638889in}}{\pgfqpoint{4.218750in}{2.172222in}}%
\pgfusepath{clip}%
\pgfsetbuttcap%
\pgfsetmiterjoin%
\definecolor{currentfill}{rgb}{0.447059,0.447059,0.447059}%
\pgfsetfillcolor{currentfill}%
\pgfsetlinewidth{1.003750pt}%
\definecolor{currentstroke}{rgb}{0.266667,0.266667,0.266667}%
\pgfsetstrokecolor{currentstroke}%
\pgfsetdash{}{0pt}%
\pgfpathmoveto{\pgfqpoint{2.765174in}{2.179603in}}%
\pgfpathlineto{\pgfqpoint{3.016076in}{2.179603in}}%
\pgfpathlineto{\pgfqpoint{3.016076in}{2.183882in}}%
\pgfpathlineto{\pgfqpoint{2.765174in}{2.183882in}}%
\pgfpathlineto{\pgfqpoint{2.765174in}{2.179603in}}%
\pgfpathclose%
\pgfusepath{stroke,fill}%
\end{pgfscope}%
\begin{pgfscope}%
\pgfpathrectangle{\pgfqpoint{0.781250in}{0.638889in}}{\pgfqpoint{4.218750in}{2.172222in}}%
\pgfusepath{clip}%
\pgfsetbuttcap%
\pgfsetmiterjoin%
\definecolor{currentfill}{rgb}{0.447059,0.447059,0.447059}%
\pgfsetfillcolor{currentfill}%
\pgfsetlinewidth{1.003750pt}%
\definecolor{currentstroke}{rgb}{0.266667,0.266667,0.266667}%
\pgfsetstrokecolor{currentstroke}%
\pgfsetdash{}{0pt}%
\pgfpathmoveto{\pgfqpoint{3.123606in}{2.074424in}}%
\pgfpathlineto{\pgfqpoint{3.374509in}{2.074424in}}%
\pgfpathlineto{\pgfqpoint{3.374509in}{2.078355in}}%
\pgfpathlineto{\pgfqpoint{3.123606in}{2.078355in}}%
\pgfpathlineto{\pgfqpoint{3.123606in}{2.074424in}}%
\pgfpathclose%
\pgfusepath{stroke,fill}%
\end{pgfscope}%
\begin{pgfscope}%
\pgfpathrectangle{\pgfqpoint{0.781250in}{0.638889in}}{\pgfqpoint{4.218750in}{2.172222in}}%
\pgfusepath{clip}%
\pgfsetbuttcap%
\pgfsetmiterjoin%
\definecolor{currentfill}{rgb}{0.447059,0.447059,0.447059}%
\pgfsetfillcolor{currentfill}%
\pgfsetlinewidth{1.003750pt}%
\definecolor{currentstroke}{rgb}{0.266667,0.266667,0.266667}%
\pgfsetstrokecolor{currentstroke}%
\pgfsetdash{}{0pt}%
\pgfpathmoveto{\pgfqpoint{3.482039in}{2.042362in}}%
\pgfpathlineto{\pgfqpoint{3.732941in}{2.042362in}}%
\pgfpathlineto{\pgfqpoint{3.732941in}{2.046098in}}%
\pgfpathlineto{\pgfqpoint{3.482039in}{2.046098in}}%
\pgfpathlineto{\pgfqpoint{3.482039in}{2.042362in}}%
\pgfpathclose%
\pgfusepath{stroke,fill}%
\end{pgfscope}%
\begin{pgfscope}%
\pgfpathrectangle{\pgfqpoint{0.781250in}{0.638889in}}{\pgfqpoint{4.218750in}{2.172222in}}%
\pgfusepath{clip}%
\pgfsetbuttcap%
\pgfsetmiterjoin%
\definecolor{currentfill}{rgb}{0.447059,0.447059,0.447059}%
\pgfsetfillcolor{currentfill}%
\pgfsetlinewidth{1.003750pt}%
\definecolor{currentstroke}{rgb}{0.266667,0.266667,0.266667}%
\pgfsetstrokecolor{currentstroke}%
\pgfsetdash{}{0pt}%
\pgfpathmoveto{\pgfqpoint{3.840471in}{2.268599in}}%
\pgfpathlineto{\pgfqpoint{4.091374in}{2.268599in}}%
\pgfpathlineto{\pgfqpoint{4.091374in}{2.272617in}}%
\pgfpathlineto{\pgfqpoint{3.840471in}{2.272617in}}%
\pgfpathlineto{\pgfqpoint{3.840471in}{2.268599in}}%
\pgfpathclose%
\pgfusepath{stroke,fill}%
\end{pgfscope}%
\begin{pgfscope}%
\pgfpathrectangle{\pgfqpoint{0.781250in}{0.638889in}}{\pgfqpoint{4.218750in}{2.172222in}}%
\pgfusepath{clip}%
\pgfsetbuttcap%
\pgfsetmiterjoin%
\definecolor{currentfill}{rgb}{0.447059,0.447059,0.447059}%
\pgfsetfillcolor{currentfill}%
\pgfsetlinewidth{1.003750pt}%
\definecolor{currentstroke}{rgb}{0.266667,0.266667,0.266667}%
\pgfsetstrokecolor{currentstroke}%
\pgfsetdash{}{0pt}%
\pgfpathmoveto{\pgfqpoint{4.198903in}{2.310305in}}%
\pgfpathlineto{\pgfqpoint{4.449806in}{2.310305in}}%
\pgfpathlineto{\pgfqpoint{4.449806in}{2.313998in}}%
\pgfpathlineto{\pgfqpoint{4.198903in}{2.313998in}}%
\pgfpathlineto{\pgfqpoint{4.198903in}{2.310305in}}%
\pgfpathclose%
\pgfusepath{stroke,fill}%
\end{pgfscope}%
\begin{pgfscope}%
\pgfpathrectangle{\pgfqpoint{0.781250in}{0.638889in}}{\pgfqpoint{4.218750in}{2.172222in}}%
\pgfusepath{clip}%
\pgfsetbuttcap%
\pgfsetmiterjoin%
\definecolor{currentfill}{rgb}{0.447059,0.447059,0.447059}%
\pgfsetfillcolor{currentfill}%
\pgfsetlinewidth{1.003750pt}%
\definecolor{currentstroke}{rgb}{0.266667,0.266667,0.266667}%
\pgfsetstrokecolor{currentstroke}%
\pgfsetdash{}{0pt}%
\pgfpathmoveto{\pgfqpoint{4.557336in}{2.461057in}}%
\pgfpathlineto{\pgfqpoint{4.808239in}{2.461057in}}%
\pgfpathlineto{\pgfqpoint{4.808239in}{2.464620in}}%
\pgfpathlineto{\pgfqpoint{4.557336in}{2.464620in}}%
\pgfpathlineto{\pgfqpoint{4.557336in}{2.461057in}}%
\pgfpathclose%
\pgfusepath{stroke,fill}%
\end{pgfscope}%
\begin{pgfscope}%
\pgfpathrectangle{\pgfqpoint{0.781250in}{0.638889in}}{\pgfqpoint{4.218750in}{2.172222in}}%
\pgfusepath{clip}%
\pgfsetbuttcap%
\pgfsetmiterjoin%
\definecolor{currentfill}{rgb}{0.447059,0.447059,0.447059}%
\pgfsetfillcolor{currentfill}%
\pgfsetlinewidth{1.003750pt}%
\definecolor{currentstroke}{rgb}{0.266667,0.266667,0.266667}%
\pgfsetstrokecolor{currentstroke}%
\pgfsetdash{}{0pt}%
\pgfpathmoveto{\pgfqpoint{0.973011in}{2.563630in}}%
\pgfpathlineto{\pgfqpoint{1.223914in}{2.563630in}}%
\pgfpathlineto{\pgfqpoint{1.223914in}{2.563630in}}%
\pgfpathlineto{\pgfqpoint{0.973011in}{2.563630in}}%
\pgfpathlineto{\pgfqpoint{0.973011in}{2.563630in}}%
\pgfpathclose%
\pgfusepath{stroke,fill}%
\end{pgfscope}%
\begin{pgfscope}%
\pgfpathrectangle{\pgfqpoint{0.781250in}{0.638889in}}{\pgfqpoint{4.218750in}{2.172222in}}%
\pgfusepath{clip}%
\pgfsetbuttcap%
\pgfsetmiterjoin%
\definecolor{currentfill}{rgb}{0.447059,0.447059,0.447059}%
\pgfsetfillcolor{currentfill}%
\pgfsetlinewidth{1.003750pt}%
\definecolor{currentstroke}{rgb}{0.266667,0.266667,0.266667}%
\pgfsetstrokecolor{currentstroke}%
\pgfsetdash{}{0pt}%
\pgfpathmoveto{\pgfqpoint{1.331444in}{2.133009in}}%
\pgfpathlineto{\pgfqpoint{1.582347in}{2.133009in}}%
\pgfpathlineto{\pgfqpoint{1.582347in}{2.133009in}}%
\pgfpathlineto{\pgfqpoint{1.331444in}{2.133009in}}%
\pgfpathlineto{\pgfqpoint{1.331444in}{2.133009in}}%
\pgfpathclose%
\pgfusepath{stroke,fill}%
\end{pgfscope}%
\begin{pgfscope}%
\pgfpathrectangle{\pgfqpoint{0.781250in}{0.638889in}}{\pgfqpoint{4.218750in}{2.172222in}}%
\pgfusepath{clip}%
\pgfsetbuttcap%
\pgfsetmiterjoin%
\definecolor{currentfill}{rgb}{0.447059,0.447059,0.447059}%
\pgfsetfillcolor{currentfill}%
\pgfsetlinewidth{1.003750pt}%
\definecolor{currentstroke}{rgb}{0.266667,0.266667,0.266667}%
\pgfsetstrokecolor{currentstroke}%
\pgfsetdash{}{0pt}%
\pgfpathmoveto{\pgfqpoint{1.689876in}{1.955169in}}%
\pgfpathlineto{\pgfqpoint{1.940779in}{1.955169in}}%
\pgfpathlineto{\pgfqpoint{1.940779in}{1.958731in}}%
\pgfpathlineto{\pgfqpoint{1.689876in}{1.958731in}}%
\pgfpathlineto{\pgfqpoint{1.689876in}{1.955169in}}%
\pgfpathclose%
\pgfusepath{stroke,fill}%
\end{pgfscope}%
\begin{pgfscope}%
\pgfpathrectangle{\pgfqpoint{0.781250in}{0.638889in}}{\pgfqpoint{4.218750in}{2.172222in}}%
\pgfusepath{clip}%
\pgfsetbuttcap%
\pgfsetmiterjoin%
\definecolor{currentfill}{rgb}{0.447059,0.447059,0.447059}%
\pgfsetfillcolor{currentfill}%
\pgfsetlinewidth{1.003750pt}%
\definecolor{currentstroke}{rgb}{0.266667,0.266667,0.266667}%
\pgfsetstrokecolor{currentstroke}%
\pgfsetdash{}{0pt}%
\pgfpathmoveto{\pgfqpoint{2.048309in}{1.889633in}}%
\pgfpathlineto{\pgfqpoint{2.299211in}{1.889633in}}%
\pgfpathlineto{\pgfqpoint{2.299211in}{1.893000in}}%
\pgfpathlineto{\pgfqpoint{2.048309in}{1.893000in}}%
\pgfpathlineto{\pgfqpoint{2.048309in}{1.889633in}}%
\pgfpathclose%
\pgfusepath{stroke,fill}%
\end{pgfscope}%
\begin{pgfscope}%
\pgfpathrectangle{\pgfqpoint{0.781250in}{0.638889in}}{\pgfqpoint{4.218750in}{2.172222in}}%
\pgfusepath{clip}%
\pgfsetbuttcap%
\pgfsetmiterjoin%
\definecolor{currentfill}{rgb}{0.447059,0.447059,0.447059}%
\pgfsetfillcolor{currentfill}%
\pgfsetlinewidth{1.003750pt}%
\definecolor{currentstroke}{rgb}{0.266667,0.266667,0.266667}%
\pgfsetstrokecolor{currentstroke}%
\pgfsetdash{}{0pt}%
\pgfpathmoveto{\pgfqpoint{2.406741in}{1.993791in}}%
\pgfpathlineto{\pgfqpoint{2.657644in}{1.993791in}}%
\pgfpathlineto{\pgfqpoint{2.657644in}{1.996376in}}%
\pgfpathlineto{\pgfqpoint{2.406741in}{1.996376in}}%
\pgfpathlineto{\pgfqpoint{2.406741in}{1.993791in}}%
\pgfpathclose%
\pgfusepath{stroke,fill}%
\end{pgfscope}%
\begin{pgfscope}%
\pgfpathrectangle{\pgfqpoint{0.781250in}{0.638889in}}{\pgfqpoint{4.218750in}{2.172222in}}%
\pgfusepath{clip}%
\pgfsetbuttcap%
\pgfsetmiterjoin%
\definecolor{currentfill}{rgb}{0.447059,0.447059,0.447059}%
\pgfsetfillcolor{currentfill}%
\pgfsetlinewidth{1.003750pt}%
\definecolor{currentstroke}{rgb}{0.266667,0.266667,0.266667}%
\pgfsetstrokecolor{currentstroke}%
\pgfsetdash{}{0pt}%
\pgfpathmoveto{\pgfqpoint{2.765174in}{2.183882in}}%
\pgfpathlineto{\pgfqpoint{3.016076in}{2.183882in}}%
\pgfpathlineto{\pgfqpoint{3.016076in}{2.189182in}}%
\pgfpathlineto{\pgfqpoint{2.765174in}{2.189182in}}%
\pgfpathlineto{\pgfqpoint{2.765174in}{2.183882in}}%
\pgfpathclose%
\pgfusepath{stroke,fill}%
\end{pgfscope}%
\begin{pgfscope}%
\pgfpathrectangle{\pgfqpoint{0.781250in}{0.638889in}}{\pgfqpoint{4.218750in}{2.172222in}}%
\pgfusepath{clip}%
\pgfsetbuttcap%
\pgfsetmiterjoin%
\definecolor{currentfill}{rgb}{0.447059,0.447059,0.447059}%
\pgfsetfillcolor{currentfill}%
\pgfsetlinewidth{1.003750pt}%
\definecolor{currentstroke}{rgb}{0.266667,0.266667,0.266667}%
\pgfsetstrokecolor{currentstroke}%
\pgfsetdash{}{0pt}%
\pgfpathmoveto{\pgfqpoint{3.123606in}{2.078355in}}%
\pgfpathlineto{\pgfqpoint{3.374509in}{2.078355in}}%
\pgfpathlineto{\pgfqpoint{3.374509in}{2.082874in}}%
\pgfpathlineto{\pgfqpoint{3.123606in}{2.082874in}}%
\pgfpathlineto{\pgfqpoint{3.123606in}{2.078355in}}%
\pgfpathclose%
\pgfusepath{stroke,fill}%
\end{pgfscope}%
\begin{pgfscope}%
\pgfpathrectangle{\pgfqpoint{0.781250in}{0.638889in}}{\pgfqpoint{4.218750in}{2.172222in}}%
\pgfusepath{clip}%
\pgfsetbuttcap%
\pgfsetmiterjoin%
\definecolor{currentfill}{rgb}{0.447059,0.447059,0.447059}%
\pgfsetfillcolor{currentfill}%
\pgfsetlinewidth{1.003750pt}%
\definecolor{currentstroke}{rgb}{0.266667,0.266667,0.266667}%
\pgfsetstrokecolor{currentstroke}%
\pgfsetdash{}{0pt}%
\pgfpathmoveto{\pgfqpoint{3.482039in}{2.046098in}}%
\pgfpathlineto{\pgfqpoint{3.732941in}{2.046098in}}%
\pgfpathlineto{\pgfqpoint{3.732941in}{2.050247in}}%
\pgfpathlineto{\pgfqpoint{3.482039in}{2.050247in}}%
\pgfpathlineto{\pgfqpoint{3.482039in}{2.046098in}}%
\pgfpathclose%
\pgfusepath{stroke,fill}%
\end{pgfscope}%
\begin{pgfscope}%
\pgfpathrectangle{\pgfqpoint{0.781250in}{0.638889in}}{\pgfqpoint{4.218750in}{2.172222in}}%
\pgfusepath{clip}%
\pgfsetbuttcap%
\pgfsetmiterjoin%
\definecolor{currentfill}{rgb}{0.447059,0.447059,0.447059}%
\pgfsetfillcolor{currentfill}%
\pgfsetlinewidth{1.003750pt}%
\definecolor{currentstroke}{rgb}{0.266667,0.266667,0.266667}%
\pgfsetstrokecolor{currentstroke}%
\pgfsetdash{}{0pt}%
\pgfpathmoveto{\pgfqpoint{3.840471in}{2.272617in}}%
\pgfpathlineto{\pgfqpoint{4.091374in}{2.272617in}}%
\pgfpathlineto{\pgfqpoint{4.091374in}{2.276158in}}%
\pgfpathlineto{\pgfqpoint{3.840471in}{2.276158in}}%
\pgfpathlineto{\pgfqpoint{3.840471in}{2.272617in}}%
\pgfpathclose%
\pgfusepath{stroke,fill}%
\end{pgfscope}%
\begin{pgfscope}%
\pgfpathrectangle{\pgfqpoint{0.781250in}{0.638889in}}{\pgfqpoint{4.218750in}{2.172222in}}%
\pgfusepath{clip}%
\pgfsetbuttcap%
\pgfsetmiterjoin%
\definecolor{currentfill}{rgb}{0.447059,0.447059,0.447059}%
\pgfsetfillcolor{currentfill}%
\pgfsetlinewidth{1.003750pt}%
\definecolor{currentstroke}{rgb}{0.266667,0.266667,0.266667}%
\pgfsetstrokecolor{currentstroke}%
\pgfsetdash{}{0pt}%
\pgfpathmoveto{\pgfqpoint{4.198903in}{2.313998in}}%
\pgfpathlineto{\pgfqpoint{4.449806in}{2.313998in}}%
\pgfpathlineto{\pgfqpoint{4.449806in}{2.317626in}}%
\pgfpathlineto{\pgfqpoint{4.198903in}{2.317626in}}%
\pgfpathlineto{\pgfqpoint{4.198903in}{2.313998in}}%
\pgfpathclose%
\pgfusepath{stroke,fill}%
\end{pgfscope}%
\begin{pgfscope}%
\pgfpathrectangle{\pgfqpoint{0.781250in}{0.638889in}}{\pgfqpoint{4.218750in}{2.172222in}}%
\pgfusepath{clip}%
\pgfsetbuttcap%
\pgfsetmiterjoin%
\definecolor{currentfill}{rgb}{0.447059,0.447059,0.447059}%
\pgfsetfillcolor{currentfill}%
\pgfsetlinewidth{1.003750pt}%
\definecolor{currentstroke}{rgb}{0.266667,0.266667,0.266667}%
\pgfsetstrokecolor{currentstroke}%
\pgfsetdash{}{0pt}%
\pgfpathmoveto{\pgfqpoint{4.557336in}{2.464620in}}%
\pgfpathlineto{\pgfqpoint{4.808239in}{2.464620in}}%
\pgfpathlineto{\pgfqpoint{4.808239in}{2.469073in}}%
\pgfpathlineto{\pgfqpoint{4.557336in}{2.469073in}}%
\pgfpathlineto{\pgfqpoint{4.557336in}{2.464620in}}%
\pgfpathclose%
\pgfusepath{stroke,fill}%
\end{pgfscope}%
\begin{pgfscope}%
\pgfpathrectangle{\pgfqpoint{0.781250in}{0.638889in}}{\pgfqpoint{4.218750in}{2.172222in}}%
\pgfusepath{clip}%
\pgfsetbuttcap%
\pgfsetmiterjoin%
\definecolor{currentfill}{rgb}{0.447059,0.447059,0.447059}%
\pgfsetfillcolor{currentfill}%
\pgfsetlinewidth{1.003750pt}%
\definecolor{currentstroke}{rgb}{0.266667,0.266667,0.266667}%
\pgfsetstrokecolor{currentstroke}%
\pgfsetdash{}{0pt}%
\pgfpathmoveto{\pgfqpoint{0.973011in}{2.563630in}}%
\pgfpathlineto{\pgfqpoint{1.223914in}{2.563630in}}%
\pgfpathlineto{\pgfqpoint{1.223914in}{2.569669in}}%
\pgfpathlineto{\pgfqpoint{0.973011in}{2.569669in}}%
\pgfpathlineto{\pgfqpoint{0.973011in}{2.563630in}}%
\pgfpathclose%
\pgfusepath{stroke,fill}%
\end{pgfscope}%
\begin{pgfscope}%
\pgfpathrectangle{\pgfqpoint{0.781250in}{0.638889in}}{\pgfqpoint{4.218750in}{2.172222in}}%
\pgfusepath{clip}%
\pgfsetbuttcap%
\pgfsetmiterjoin%
\definecolor{currentfill}{rgb}{0.447059,0.447059,0.447059}%
\pgfsetfillcolor{currentfill}%
\pgfsetlinewidth{1.003750pt}%
\definecolor{currentstroke}{rgb}{0.266667,0.266667,0.266667}%
\pgfsetstrokecolor{currentstroke}%
\pgfsetdash{}{0pt}%
\pgfpathmoveto{\pgfqpoint{1.331444in}{2.133009in}}%
\pgfpathlineto{\pgfqpoint{1.582347in}{2.133009in}}%
\pgfpathlineto{\pgfqpoint{1.582347in}{2.137874in}}%
\pgfpathlineto{\pgfqpoint{1.331444in}{2.137874in}}%
\pgfpathlineto{\pgfqpoint{1.331444in}{2.133009in}}%
\pgfpathclose%
\pgfusepath{stroke,fill}%
\end{pgfscope}%
\begin{pgfscope}%
\pgfpathrectangle{\pgfqpoint{0.781250in}{0.638889in}}{\pgfqpoint{4.218750in}{2.172222in}}%
\pgfusepath{clip}%
\pgfsetbuttcap%
\pgfsetmiterjoin%
\definecolor{currentfill}{rgb}{0.447059,0.447059,0.447059}%
\pgfsetfillcolor{currentfill}%
\pgfsetlinewidth{1.003750pt}%
\definecolor{currentstroke}{rgb}{0.266667,0.266667,0.266667}%
\pgfsetstrokecolor{currentstroke}%
\pgfsetdash{}{0pt}%
\pgfpathmoveto{\pgfqpoint{1.689876in}{1.958731in}}%
\pgfpathlineto{\pgfqpoint{1.940779in}{1.958731in}}%
\pgfpathlineto{\pgfqpoint{1.940779in}{1.961968in}}%
\pgfpathlineto{\pgfqpoint{1.689876in}{1.961968in}}%
\pgfpathlineto{\pgfqpoint{1.689876in}{1.958731in}}%
\pgfpathclose%
\pgfusepath{stroke,fill}%
\end{pgfscope}%
\begin{pgfscope}%
\pgfpathrectangle{\pgfqpoint{0.781250in}{0.638889in}}{\pgfqpoint{4.218750in}{2.172222in}}%
\pgfusepath{clip}%
\pgfsetbuttcap%
\pgfsetmiterjoin%
\definecolor{currentfill}{rgb}{0.447059,0.447059,0.447059}%
\pgfsetfillcolor{currentfill}%
\pgfsetlinewidth{1.003750pt}%
\definecolor{currentstroke}{rgb}{0.266667,0.266667,0.266667}%
\pgfsetstrokecolor{currentstroke}%
\pgfsetdash{}{0pt}%
\pgfpathmoveto{\pgfqpoint{2.048309in}{1.893000in}}%
\pgfpathlineto{\pgfqpoint{2.299211in}{1.893000in}}%
\pgfpathlineto{\pgfqpoint{2.299211in}{1.896236in}}%
\pgfpathlineto{\pgfqpoint{2.048309in}{1.896236in}}%
\pgfpathlineto{\pgfqpoint{2.048309in}{1.893000in}}%
\pgfpathclose%
\pgfusepath{stroke,fill}%
\end{pgfscope}%
\begin{pgfscope}%
\pgfpathrectangle{\pgfqpoint{0.781250in}{0.638889in}}{\pgfqpoint{4.218750in}{2.172222in}}%
\pgfusepath{clip}%
\pgfsetbuttcap%
\pgfsetmiterjoin%
\definecolor{currentfill}{rgb}{0.447059,0.447059,0.447059}%
\pgfsetfillcolor{currentfill}%
\pgfsetlinewidth{1.003750pt}%
\definecolor{currentstroke}{rgb}{0.266667,0.266667,0.266667}%
\pgfsetstrokecolor{currentstroke}%
\pgfsetdash{}{0pt}%
\pgfpathmoveto{\pgfqpoint{2.406741in}{1.996376in}}%
\pgfpathlineto{\pgfqpoint{2.657644in}{1.996376in}}%
\pgfpathlineto{\pgfqpoint{2.657644in}{1.999287in}}%
\pgfpathlineto{\pgfqpoint{2.406741in}{1.999287in}}%
\pgfpathlineto{\pgfqpoint{2.406741in}{1.996376in}}%
\pgfpathclose%
\pgfusepath{stroke,fill}%
\end{pgfscope}%
\begin{pgfscope}%
\pgfpathrectangle{\pgfqpoint{0.781250in}{0.638889in}}{\pgfqpoint{4.218750in}{2.172222in}}%
\pgfusepath{clip}%
\pgfsetbuttcap%
\pgfsetmiterjoin%
\definecolor{currentfill}{rgb}{0.447059,0.447059,0.447059}%
\pgfsetfillcolor{currentfill}%
\pgfsetlinewidth{1.003750pt}%
\definecolor{currentstroke}{rgb}{0.266667,0.266667,0.266667}%
\pgfsetstrokecolor{currentstroke}%
\pgfsetdash{}{0pt}%
\pgfpathmoveto{\pgfqpoint{2.765174in}{2.189182in}}%
\pgfpathlineto{\pgfqpoint{3.016076in}{2.189182in}}%
\pgfpathlineto{\pgfqpoint{3.016076in}{2.191897in}}%
\pgfpathlineto{\pgfqpoint{2.765174in}{2.191897in}}%
\pgfpathlineto{\pgfqpoint{2.765174in}{2.189182in}}%
\pgfpathclose%
\pgfusepath{stroke,fill}%
\end{pgfscope}%
\begin{pgfscope}%
\pgfpathrectangle{\pgfqpoint{0.781250in}{0.638889in}}{\pgfqpoint{4.218750in}{2.172222in}}%
\pgfusepath{clip}%
\pgfsetbuttcap%
\pgfsetmiterjoin%
\definecolor{currentfill}{rgb}{0.447059,0.447059,0.447059}%
\pgfsetfillcolor{currentfill}%
\pgfsetlinewidth{1.003750pt}%
\definecolor{currentstroke}{rgb}{0.266667,0.266667,0.266667}%
\pgfsetstrokecolor{currentstroke}%
\pgfsetdash{}{0pt}%
\pgfpathmoveto{\pgfqpoint{3.123606in}{2.082874in}}%
\pgfpathlineto{\pgfqpoint{3.374509in}{2.082874in}}%
\pgfpathlineto{\pgfqpoint{3.374509in}{2.085285in}}%
\pgfpathlineto{\pgfqpoint{3.123606in}{2.085285in}}%
\pgfpathlineto{\pgfqpoint{3.123606in}{2.082874in}}%
\pgfpathclose%
\pgfusepath{stroke,fill}%
\end{pgfscope}%
\begin{pgfscope}%
\pgfpathrectangle{\pgfqpoint{0.781250in}{0.638889in}}{\pgfqpoint{4.218750in}{2.172222in}}%
\pgfusepath{clip}%
\pgfsetbuttcap%
\pgfsetmiterjoin%
\definecolor{currentfill}{rgb}{0.447059,0.447059,0.447059}%
\pgfsetfillcolor{currentfill}%
\pgfsetlinewidth{1.003750pt}%
\definecolor{currentstroke}{rgb}{0.266667,0.266667,0.266667}%
\pgfsetstrokecolor{currentstroke}%
\pgfsetdash{}{0pt}%
\pgfpathmoveto{\pgfqpoint{3.482039in}{2.050247in}}%
\pgfpathlineto{\pgfqpoint{3.732941in}{2.050247in}}%
\pgfpathlineto{\pgfqpoint{3.732941in}{2.052767in}}%
\pgfpathlineto{\pgfqpoint{3.482039in}{2.052767in}}%
\pgfpathlineto{\pgfqpoint{3.482039in}{2.050247in}}%
\pgfpathclose%
\pgfusepath{stroke,fill}%
\end{pgfscope}%
\begin{pgfscope}%
\pgfpathrectangle{\pgfqpoint{0.781250in}{0.638889in}}{\pgfqpoint{4.218750in}{2.172222in}}%
\pgfusepath{clip}%
\pgfsetbuttcap%
\pgfsetmiterjoin%
\definecolor{currentfill}{rgb}{0.447059,0.447059,0.447059}%
\pgfsetfillcolor{currentfill}%
\pgfsetlinewidth{1.003750pt}%
\definecolor{currentstroke}{rgb}{0.266667,0.266667,0.266667}%
\pgfsetstrokecolor{currentstroke}%
\pgfsetdash{}{0pt}%
\pgfpathmoveto{\pgfqpoint{3.840471in}{2.276158in}}%
\pgfpathlineto{\pgfqpoint{4.091374in}{2.276158in}}%
\pgfpathlineto{\pgfqpoint{4.091374in}{2.278743in}}%
\pgfpathlineto{\pgfqpoint{3.840471in}{2.278743in}}%
\pgfpathlineto{\pgfqpoint{3.840471in}{2.276158in}}%
\pgfpathclose%
\pgfusepath{stroke,fill}%
\end{pgfscope}%
\begin{pgfscope}%
\pgfpathrectangle{\pgfqpoint{0.781250in}{0.638889in}}{\pgfqpoint{4.218750in}{2.172222in}}%
\pgfusepath{clip}%
\pgfsetbuttcap%
\pgfsetmiterjoin%
\definecolor{currentfill}{rgb}{0.447059,0.447059,0.447059}%
\pgfsetfillcolor{currentfill}%
\pgfsetlinewidth{1.003750pt}%
\definecolor{currentstroke}{rgb}{0.266667,0.266667,0.266667}%
\pgfsetstrokecolor{currentstroke}%
\pgfsetdash{}{0pt}%
\pgfpathmoveto{\pgfqpoint{4.198903in}{2.317626in}}%
\pgfpathlineto{\pgfqpoint{4.449806in}{2.317626in}}%
\pgfpathlineto{\pgfqpoint{4.449806in}{2.320276in}}%
\pgfpathlineto{\pgfqpoint{4.198903in}{2.320276in}}%
\pgfpathlineto{\pgfqpoint{4.198903in}{2.317626in}}%
\pgfpathclose%
\pgfusepath{stroke,fill}%
\end{pgfscope}%
\begin{pgfscope}%
\pgfpathrectangle{\pgfqpoint{0.781250in}{0.638889in}}{\pgfqpoint{4.218750in}{2.172222in}}%
\pgfusepath{clip}%
\pgfsetbuttcap%
\pgfsetmiterjoin%
\definecolor{currentfill}{rgb}{0.447059,0.447059,0.447059}%
\pgfsetfillcolor{currentfill}%
\pgfsetlinewidth{1.003750pt}%
\definecolor{currentstroke}{rgb}{0.266667,0.266667,0.266667}%
\pgfsetstrokecolor{currentstroke}%
\pgfsetdash{}{0pt}%
\pgfpathmoveto{\pgfqpoint{4.557336in}{2.469073in}}%
\pgfpathlineto{\pgfqpoint{4.808239in}{2.469073in}}%
\pgfpathlineto{\pgfqpoint{4.808239in}{2.477197in}}%
\pgfpathlineto{\pgfqpoint{4.557336in}{2.477197in}}%
\pgfpathlineto{\pgfqpoint{4.557336in}{2.469073in}}%
\pgfpathclose%
\pgfusepath{stroke,fill}%
\end{pgfscope}%
\begin{pgfscope}%
\pgfpathrectangle{\pgfqpoint{0.781250in}{0.638889in}}{\pgfqpoint{4.218750in}{2.172222in}}%
\pgfusepath{clip}%
\pgfsetbuttcap%
\pgfsetmiterjoin%
\definecolor{currentfill}{rgb}{0.447059,0.447059,0.447059}%
\pgfsetfillcolor{currentfill}%
\pgfsetlinewidth{1.003750pt}%
\definecolor{currentstroke}{rgb}{0.266667,0.266667,0.266667}%
\pgfsetstrokecolor{currentstroke}%
\pgfsetdash{}{0pt}%
\pgfpathmoveto{\pgfqpoint{0.973011in}{2.569669in}}%
\pgfpathlineto{\pgfqpoint{1.223914in}{2.569669in}}%
\pgfpathlineto{\pgfqpoint{1.223914in}{2.573752in}}%
\pgfpathlineto{\pgfqpoint{0.973011in}{2.573752in}}%
\pgfpathlineto{\pgfqpoint{0.973011in}{2.569669in}}%
\pgfpathclose%
\pgfusepath{stroke,fill}%
\end{pgfscope}%
\begin{pgfscope}%
\pgfpathrectangle{\pgfqpoint{0.781250in}{0.638889in}}{\pgfqpoint{4.218750in}{2.172222in}}%
\pgfusepath{clip}%
\pgfsetbuttcap%
\pgfsetmiterjoin%
\definecolor{currentfill}{rgb}{0.447059,0.447059,0.447059}%
\pgfsetfillcolor{currentfill}%
\pgfsetlinewidth{1.003750pt}%
\definecolor{currentstroke}{rgb}{0.266667,0.266667,0.266667}%
\pgfsetstrokecolor{currentstroke}%
\pgfsetdash{}{0pt}%
\pgfpathmoveto{\pgfqpoint{1.331444in}{2.137874in}}%
\pgfpathlineto{\pgfqpoint{1.582347in}{2.137874in}}%
\pgfpathlineto{\pgfqpoint{1.582347in}{2.140937in}}%
\pgfpathlineto{\pgfqpoint{1.331444in}{2.140937in}}%
\pgfpathlineto{\pgfqpoint{1.331444in}{2.137874in}}%
\pgfpathclose%
\pgfusepath{stroke,fill}%
\end{pgfscope}%
\begin{pgfscope}%
\pgfpathrectangle{\pgfqpoint{0.781250in}{0.638889in}}{\pgfqpoint{4.218750in}{2.172222in}}%
\pgfusepath{clip}%
\pgfsetbuttcap%
\pgfsetmiterjoin%
\definecolor{currentfill}{rgb}{0.447059,0.447059,0.447059}%
\pgfsetfillcolor{currentfill}%
\pgfsetlinewidth{1.003750pt}%
\definecolor{currentstroke}{rgb}{0.266667,0.266667,0.266667}%
\pgfsetstrokecolor{currentstroke}%
\pgfsetdash{}{0pt}%
\pgfpathmoveto{\pgfqpoint{1.689876in}{1.961968in}}%
\pgfpathlineto{\pgfqpoint{1.940779in}{1.961968in}}%
\pgfpathlineto{\pgfqpoint{1.940779in}{1.964726in}}%
\pgfpathlineto{\pgfqpoint{1.689876in}{1.964726in}}%
\pgfpathlineto{\pgfqpoint{1.689876in}{1.961968in}}%
\pgfpathclose%
\pgfusepath{stroke,fill}%
\end{pgfscope}%
\begin{pgfscope}%
\pgfpathrectangle{\pgfqpoint{0.781250in}{0.638889in}}{\pgfqpoint{4.218750in}{2.172222in}}%
\pgfusepath{clip}%
\pgfsetbuttcap%
\pgfsetmiterjoin%
\definecolor{currentfill}{rgb}{0.447059,0.447059,0.447059}%
\pgfsetfillcolor{currentfill}%
\pgfsetlinewidth{1.003750pt}%
\definecolor{currentstroke}{rgb}{0.266667,0.266667,0.266667}%
\pgfsetstrokecolor{currentstroke}%
\pgfsetdash{}{0pt}%
\pgfpathmoveto{\pgfqpoint{2.048309in}{1.896236in}}%
\pgfpathlineto{\pgfqpoint{2.299211in}{1.896236in}}%
\pgfpathlineto{\pgfqpoint{2.299211in}{1.898148in}}%
\pgfpathlineto{\pgfqpoint{2.048309in}{1.898148in}}%
\pgfpathlineto{\pgfqpoint{2.048309in}{1.896236in}}%
\pgfpathclose%
\pgfusepath{stroke,fill}%
\end{pgfscope}%
\begin{pgfscope}%
\pgfpathrectangle{\pgfqpoint{0.781250in}{0.638889in}}{\pgfqpoint{4.218750in}{2.172222in}}%
\pgfusepath{clip}%
\pgfsetbuttcap%
\pgfsetmiterjoin%
\definecolor{currentfill}{rgb}{0.447059,0.447059,0.447059}%
\pgfsetfillcolor{currentfill}%
\pgfsetlinewidth{1.003750pt}%
\definecolor{currentstroke}{rgb}{0.266667,0.266667,0.266667}%
\pgfsetstrokecolor{currentstroke}%
\pgfsetdash{}{0pt}%
\pgfpathmoveto{\pgfqpoint{2.406741in}{1.999287in}}%
\pgfpathlineto{\pgfqpoint{2.657644in}{1.999287in}}%
\pgfpathlineto{\pgfqpoint{2.657644in}{2.000916in}}%
\pgfpathlineto{\pgfqpoint{2.406741in}{2.000916in}}%
\pgfpathlineto{\pgfqpoint{2.406741in}{1.999287in}}%
\pgfpathclose%
\pgfusepath{stroke,fill}%
\end{pgfscope}%
\begin{pgfscope}%
\pgfpathrectangle{\pgfqpoint{0.781250in}{0.638889in}}{\pgfqpoint{4.218750in}{2.172222in}}%
\pgfusepath{clip}%
\pgfsetbuttcap%
\pgfsetmiterjoin%
\definecolor{currentfill}{rgb}{0.447059,0.447059,0.447059}%
\pgfsetfillcolor{currentfill}%
\pgfsetlinewidth{1.003750pt}%
\definecolor{currentstroke}{rgb}{0.266667,0.266667,0.266667}%
\pgfsetstrokecolor{currentstroke}%
\pgfsetdash{}{0pt}%
\pgfpathmoveto{\pgfqpoint{2.765174in}{2.191897in}}%
\pgfpathlineto{\pgfqpoint{3.016076in}{2.191897in}}%
\pgfpathlineto{\pgfqpoint{3.016076in}{2.193657in}}%
\pgfpathlineto{\pgfqpoint{2.765174in}{2.193657in}}%
\pgfpathlineto{\pgfqpoint{2.765174in}{2.191897in}}%
\pgfpathclose%
\pgfusepath{stroke,fill}%
\end{pgfscope}%
\begin{pgfscope}%
\pgfpathrectangle{\pgfqpoint{0.781250in}{0.638889in}}{\pgfqpoint{4.218750in}{2.172222in}}%
\pgfusepath{clip}%
\pgfsetbuttcap%
\pgfsetmiterjoin%
\definecolor{currentfill}{rgb}{0.447059,0.447059,0.447059}%
\pgfsetfillcolor{currentfill}%
\pgfsetlinewidth{1.003750pt}%
\definecolor{currentstroke}{rgb}{0.266667,0.266667,0.266667}%
\pgfsetstrokecolor{currentstroke}%
\pgfsetdash{}{0pt}%
\pgfpathmoveto{\pgfqpoint{3.123606in}{2.085285in}}%
\pgfpathlineto{\pgfqpoint{3.374509in}{2.085285in}}%
\pgfpathlineto{\pgfqpoint{3.374509in}{2.086892in}}%
\pgfpathlineto{\pgfqpoint{3.123606in}{2.086892in}}%
\pgfpathlineto{\pgfqpoint{3.123606in}{2.085285in}}%
\pgfpathclose%
\pgfusepath{stroke,fill}%
\end{pgfscope}%
\begin{pgfscope}%
\pgfpathrectangle{\pgfqpoint{0.781250in}{0.638889in}}{\pgfqpoint{4.218750in}{2.172222in}}%
\pgfusepath{clip}%
\pgfsetbuttcap%
\pgfsetmiterjoin%
\definecolor{currentfill}{rgb}{0.447059,0.447059,0.447059}%
\pgfsetfillcolor{currentfill}%
\pgfsetlinewidth{1.003750pt}%
\definecolor{currentstroke}{rgb}{0.266667,0.266667,0.266667}%
\pgfsetstrokecolor{currentstroke}%
\pgfsetdash{}{0pt}%
\pgfpathmoveto{\pgfqpoint{3.482039in}{2.052767in}}%
\pgfpathlineto{\pgfqpoint{3.732941in}{2.052767in}}%
\pgfpathlineto{\pgfqpoint{3.732941in}{2.055243in}}%
\pgfpathlineto{\pgfqpoint{3.482039in}{2.055243in}}%
\pgfpathlineto{\pgfqpoint{3.482039in}{2.052767in}}%
\pgfpathclose%
\pgfusepath{stroke,fill}%
\end{pgfscope}%
\begin{pgfscope}%
\pgfpathrectangle{\pgfqpoint{0.781250in}{0.638889in}}{\pgfqpoint{4.218750in}{2.172222in}}%
\pgfusepath{clip}%
\pgfsetbuttcap%
\pgfsetmiterjoin%
\definecolor{currentfill}{rgb}{0.447059,0.447059,0.447059}%
\pgfsetfillcolor{currentfill}%
\pgfsetlinewidth{1.003750pt}%
\definecolor{currentstroke}{rgb}{0.266667,0.266667,0.266667}%
\pgfsetstrokecolor{currentstroke}%
\pgfsetdash{}{0pt}%
\pgfpathmoveto{\pgfqpoint{3.840471in}{2.278743in}}%
\pgfpathlineto{\pgfqpoint{4.091374in}{2.278743in}}%
\pgfpathlineto{\pgfqpoint{4.091374in}{2.280807in}}%
\pgfpathlineto{\pgfqpoint{3.840471in}{2.280807in}}%
\pgfpathlineto{\pgfqpoint{3.840471in}{2.278743in}}%
\pgfpathclose%
\pgfusepath{stroke,fill}%
\end{pgfscope}%
\begin{pgfscope}%
\pgfpathrectangle{\pgfqpoint{0.781250in}{0.638889in}}{\pgfqpoint{4.218750in}{2.172222in}}%
\pgfusepath{clip}%
\pgfsetbuttcap%
\pgfsetmiterjoin%
\definecolor{currentfill}{rgb}{0.447059,0.447059,0.447059}%
\pgfsetfillcolor{currentfill}%
\pgfsetlinewidth{1.003750pt}%
\definecolor{currentstroke}{rgb}{0.266667,0.266667,0.266667}%
\pgfsetstrokecolor{currentstroke}%
\pgfsetdash{}{0pt}%
\pgfpathmoveto{\pgfqpoint{4.198903in}{2.320276in}}%
\pgfpathlineto{\pgfqpoint{4.449806in}{2.320276in}}%
\pgfpathlineto{\pgfqpoint{4.449806in}{2.322166in}}%
\pgfpathlineto{\pgfqpoint{4.198903in}{2.322166in}}%
\pgfpathlineto{\pgfqpoint{4.198903in}{2.320276in}}%
\pgfpathclose%
\pgfusepath{stroke,fill}%
\end{pgfscope}%
\begin{pgfscope}%
\pgfpathrectangle{\pgfqpoint{0.781250in}{0.638889in}}{\pgfqpoint{4.218750in}{2.172222in}}%
\pgfusepath{clip}%
\pgfsetbuttcap%
\pgfsetmiterjoin%
\definecolor{currentfill}{rgb}{0.447059,0.447059,0.447059}%
\pgfsetfillcolor{currentfill}%
\pgfsetlinewidth{1.003750pt}%
\definecolor{currentstroke}{rgb}{0.266667,0.266667,0.266667}%
\pgfsetstrokecolor{currentstroke}%
\pgfsetdash{}{0pt}%
\pgfpathmoveto{\pgfqpoint{4.557336in}{2.477197in}}%
\pgfpathlineto{\pgfqpoint{4.808239in}{2.477197in}}%
\pgfpathlineto{\pgfqpoint{4.808239in}{2.478522in}}%
\pgfpathlineto{\pgfqpoint{4.557336in}{2.478522in}}%
\pgfpathlineto{\pgfqpoint{4.557336in}{2.477197in}}%
\pgfpathclose%
\pgfusepath{stroke,fill}%
\end{pgfscope}%
\begin{pgfscope}%
\pgfpathrectangle{\pgfqpoint{0.781250in}{0.638889in}}{\pgfqpoint{4.218750in}{2.172222in}}%
\pgfusepath{clip}%
\pgfsetbuttcap%
\pgfsetmiterjoin%
\definecolor{currentfill}{rgb}{0.447059,0.447059,0.447059}%
\pgfsetfillcolor{currentfill}%
\pgfsetlinewidth{1.003750pt}%
\definecolor{currentstroke}{rgb}{0.266667,0.266667,0.266667}%
\pgfsetstrokecolor{currentstroke}%
\pgfsetdash{}{0pt}%
\pgfpathmoveto{\pgfqpoint{0.973011in}{2.573752in}}%
\pgfpathlineto{\pgfqpoint{1.223914in}{2.573752in}}%
\pgfpathlineto{\pgfqpoint{1.223914in}{2.575099in}}%
\pgfpathlineto{\pgfqpoint{0.973011in}{2.575099in}}%
\pgfpathlineto{\pgfqpoint{0.973011in}{2.573752in}}%
\pgfpathclose%
\pgfusepath{stroke,fill}%
\end{pgfscope}%
\begin{pgfscope}%
\pgfpathrectangle{\pgfqpoint{0.781250in}{0.638889in}}{\pgfqpoint{4.218750in}{2.172222in}}%
\pgfusepath{clip}%
\pgfsetbuttcap%
\pgfsetmiterjoin%
\definecolor{currentfill}{rgb}{0.447059,0.447059,0.447059}%
\pgfsetfillcolor{currentfill}%
\pgfsetlinewidth{1.003750pt}%
\definecolor{currentstroke}{rgb}{0.266667,0.266667,0.266667}%
\pgfsetstrokecolor{currentstroke}%
\pgfsetdash{}{0pt}%
\pgfpathmoveto{\pgfqpoint{1.331444in}{2.140937in}}%
\pgfpathlineto{\pgfqpoint{1.582347in}{2.140937in}}%
\pgfpathlineto{\pgfqpoint{1.582347in}{2.142349in}}%
\pgfpathlineto{\pgfqpoint{1.331444in}{2.142349in}}%
\pgfpathlineto{\pgfqpoint{1.331444in}{2.140937in}}%
\pgfpathclose%
\pgfusepath{stroke,fill}%
\end{pgfscope}%
\begin{pgfscope}%
\pgfpathrectangle{\pgfqpoint{0.781250in}{0.638889in}}{\pgfqpoint{4.218750in}{2.172222in}}%
\pgfusepath{clip}%
\pgfsetbuttcap%
\pgfsetmiterjoin%
\definecolor{currentfill}{rgb}{0.447059,0.447059,0.447059}%
\pgfsetfillcolor{currentfill}%
\pgfsetlinewidth{1.003750pt}%
\definecolor{currentstroke}{rgb}{0.266667,0.266667,0.266667}%
\pgfsetstrokecolor{currentstroke}%
\pgfsetdash{}{0pt}%
\pgfpathmoveto{\pgfqpoint{1.689876in}{1.964726in}}%
\pgfpathlineto{\pgfqpoint{1.940779in}{1.964726in}}%
\pgfpathlineto{\pgfqpoint{1.940779in}{1.966030in}}%
\pgfpathlineto{\pgfqpoint{1.689876in}{1.966030in}}%
\pgfpathlineto{\pgfqpoint{1.689876in}{1.964726in}}%
\pgfpathclose%
\pgfusepath{stroke,fill}%
\end{pgfscope}%
\begin{pgfscope}%
\pgfpathrectangle{\pgfqpoint{0.781250in}{0.638889in}}{\pgfqpoint{4.218750in}{2.172222in}}%
\pgfusepath{clip}%
\pgfsetbuttcap%
\pgfsetmiterjoin%
\definecolor{currentfill}{rgb}{0.447059,0.447059,0.447059}%
\pgfsetfillcolor{currentfill}%
\pgfsetlinewidth{1.003750pt}%
\definecolor{currentstroke}{rgb}{0.266667,0.266667,0.266667}%
\pgfsetstrokecolor{currentstroke}%
\pgfsetdash{}{0pt}%
\pgfpathmoveto{\pgfqpoint{2.048309in}{1.898148in}}%
\pgfpathlineto{\pgfqpoint{2.299211in}{1.898148in}}%
\pgfpathlineto{\pgfqpoint{2.299211in}{1.899495in}}%
\pgfpathlineto{\pgfqpoint{2.048309in}{1.899495in}}%
\pgfpathlineto{\pgfqpoint{2.048309in}{1.898148in}}%
\pgfpathclose%
\pgfusepath{stroke,fill}%
\end{pgfscope}%
\begin{pgfscope}%
\pgfpathrectangle{\pgfqpoint{0.781250in}{0.638889in}}{\pgfqpoint{4.218750in}{2.172222in}}%
\pgfusepath{clip}%
\pgfsetbuttcap%
\pgfsetmiterjoin%
\definecolor{currentfill}{rgb}{0.447059,0.447059,0.447059}%
\pgfsetfillcolor{currentfill}%
\pgfsetlinewidth{1.003750pt}%
\definecolor{currentstroke}{rgb}{0.266667,0.266667,0.266667}%
\pgfsetstrokecolor{currentstroke}%
\pgfsetdash{}{0pt}%
\pgfpathmoveto{\pgfqpoint{2.406741in}{2.000916in}}%
\pgfpathlineto{\pgfqpoint{2.657644in}{2.000916in}}%
\pgfpathlineto{\pgfqpoint{2.657644in}{2.003479in}}%
\pgfpathlineto{\pgfqpoint{2.406741in}{2.003479in}}%
\pgfpathlineto{\pgfqpoint{2.406741in}{2.000916in}}%
\pgfpathclose%
\pgfusepath{stroke,fill}%
\end{pgfscope}%
\begin{pgfscope}%
\pgfpathrectangle{\pgfqpoint{0.781250in}{0.638889in}}{\pgfqpoint{4.218750in}{2.172222in}}%
\pgfusepath{clip}%
\pgfsetbuttcap%
\pgfsetmiterjoin%
\definecolor{currentfill}{rgb}{0.447059,0.447059,0.447059}%
\pgfsetfillcolor{currentfill}%
\pgfsetlinewidth{1.003750pt}%
\definecolor{currentstroke}{rgb}{0.266667,0.266667,0.266667}%
\pgfsetstrokecolor{currentstroke}%
\pgfsetdash{}{0pt}%
\pgfpathmoveto{\pgfqpoint{2.765174in}{2.193657in}}%
\pgfpathlineto{\pgfqpoint{3.016076in}{2.193657in}}%
\pgfpathlineto{\pgfqpoint{3.016076in}{2.195786in}}%
\pgfpathlineto{\pgfqpoint{2.765174in}{2.195786in}}%
\pgfpathlineto{\pgfqpoint{2.765174in}{2.193657in}}%
\pgfpathclose%
\pgfusepath{stroke,fill}%
\end{pgfscope}%
\begin{pgfscope}%
\pgfpathrectangle{\pgfqpoint{0.781250in}{0.638889in}}{\pgfqpoint{4.218750in}{2.172222in}}%
\pgfusepath{clip}%
\pgfsetbuttcap%
\pgfsetmiterjoin%
\definecolor{currentfill}{rgb}{0.447059,0.447059,0.447059}%
\pgfsetfillcolor{currentfill}%
\pgfsetlinewidth{1.003750pt}%
\definecolor{currentstroke}{rgb}{0.266667,0.266667,0.266667}%
\pgfsetstrokecolor{currentstroke}%
\pgfsetdash{}{0pt}%
\pgfpathmoveto{\pgfqpoint{3.123606in}{2.086892in}}%
\pgfpathlineto{\pgfqpoint{3.374509in}{2.086892in}}%
\pgfpathlineto{\pgfqpoint{3.374509in}{2.088456in}}%
\pgfpathlineto{\pgfqpoint{3.123606in}{2.088456in}}%
\pgfpathlineto{\pgfqpoint{3.123606in}{2.086892in}}%
\pgfpathclose%
\pgfusepath{stroke,fill}%
\end{pgfscope}%
\begin{pgfscope}%
\pgfpathrectangle{\pgfqpoint{0.781250in}{0.638889in}}{\pgfqpoint{4.218750in}{2.172222in}}%
\pgfusepath{clip}%
\pgfsetbuttcap%
\pgfsetmiterjoin%
\definecolor{currentfill}{rgb}{0.447059,0.447059,0.447059}%
\pgfsetfillcolor{currentfill}%
\pgfsetlinewidth{1.003750pt}%
\definecolor{currentstroke}{rgb}{0.266667,0.266667,0.266667}%
\pgfsetstrokecolor{currentstroke}%
\pgfsetdash{}{0pt}%
\pgfpathmoveto{\pgfqpoint{3.482039in}{2.055243in}}%
\pgfpathlineto{\pgfqpoint{3.732941in}{2.055243in}}%
\pgfpathlineto{\pgfqpoint{3.732941in}{2.056525in}}%
\pgfpathlineto{\pgfqpoint{3.482039in}{2.056525in}}%
\pgfpathlineto{\pgfqpoint{3.482039in}{2.055243in}}%
\pgfpathclose%
\pgfusepath{stroke,fill}%
\end{pgfscope}%
\begin{pgfscope}%
\pgfpathrectangle{\pgfqpoint{0.781250in}{0.638889in}}{\pgfqpoint{4.218750in}{2.172222in}}%
\pgfusepath{clip}%
\pgfsetbuttcap%
\pgfsetmiterjoin%
\definecolor{currentfill}{rgb}{0.447059,0.447059,0.447059}%
\pgfsetfillcolor{currentfill}%
\pgfsetlinewidth{1.003750pt}%
\definecolor{currentstroke}{rgb}{0.266667,0.266667,0.266667}%
\pgfsetstrokecolor{currentstroke}%
\pgfsetdash{}{0pt}%
\pgfpathmoveto{\pgfqpoint{3.840471in}{2.280807in}}%
\pgfpathlineto{\pgfqpoint{4.091374in}{2.280807in}}%
\pgfpathlineto{\pgfqpoint{4.091374in}{2.281936in}}%
\pgfpathlineto{\pgfqpoint{3.840471in}{2.281936in}}%
\pgfpathlineto{\pgfqpoint{3.840471in}{2.280807in}}%
\pgfpathclose%
\pgfusepath{stroke,fill}%
\end{pgfscope}%
\begin{pgfscope}%
\pgfpathrectangle{\pgfqpoint{0.781250in}{0.638889in}}{\pgfqpoint{4.218750in}{2.172222in}}%
\pgfusepath{clip}%
\pgfsetbuttcap%
\pgfsetmiterjoin%
\definecolor{currentfill}{rgb}{0.447059,0.447059,0.447059}%
\pgfsetfillcolor{currentfill}%
\pgfsetlinewidth{1.003750pt}%
\definecolor{currentstroke}{rgb}{0.266667,0.266667,0.266667}%
\pgfsetstrokecolor{currentstroke}%
\pgfsetdash{}{0pt}%
\pgfpathmoveto{\pgfqpoint{4.198903in}{2.322166in}}%
\pgfpathlineto{\pgfqpoint{4.449806in}{2.322166in}}%
\pgfpathlineto{\pgfqpoint{4.449806in}{2.323143in}}%
\pgfpathlineto{\pgfqpoint{4.198903in}{2.323143in}}%
\pgfpathlineto{\pgfqpoint{4.198903in}{2.322166in}}%
\pgfpathclose%
\pgfusepath{stroke,fill}%
\end{pgfscope}%
\begin{pgfscope}%
\pgfpathrectangle{\pgfqpoint{0.781250in}{0.638889in}}{\pgfqpoint{4.218750in}{2.172222in}}%
\pgfusepath{clip}%
\pgfsetbuttcap%
\pgfsetmiterjoin%
\definecolor{currentfill}{rgb}{0.447059,0.447059,0.447059}%
\pgfsetfillcolor{currentfill}%
\pgfsetlinewidth{1.003750pt}%
\definecolor{currentstroke}{rgb}{0.266667,0.266667,0.266667}%
\pgfsetstrokecolor{currentstroke}%
\pgfsetdash{}{0pt}%
\pgfpathmoveto{\pgfqpoint{4.557336in}{2.478522in}}%
\pgfpathlineto{\pgfqpoint{4.808239in}{2.478522in}}%
\pgfpathlineto{\pgfqpoint{4.808239in}{2.479478in}}%
\pgfpathlineto{\pgfqpoint{4.557336in}{2.479478in}}%
\pgfpathlineto{\pgfqpoint{4.557336in}{2.478522in}}%
\pgfpathclose%
\pgfusepath{stroke,fill}%
\end{pgfscope}%
\begin{pgfscope}%
\pgfpathrectangle{\pgfqpoint{0.781250in}{0.638889in}}{\pgfqpoint{4.218750in}{2.172222in}}%
\pgfusepath{clip}%
\pgfsetbuttcap%
\pgfsetmiterjoin%
\definecolor{currentfill}{rgb}{0.447059,0.447059,0.447059}%
\pgfsetfillcolor{currentfill}%
\pgfsetlinewidth{1.003750pt}%
\definecolor{currentstroke}{rgb}{0.266667,0.266667,0.266667}%
\pgfsetstrokecolor{currentstroke}%
\pgfsetdash{}{0pt}%
\pgfpathmoveto{\pgfqpoint{0.973011in}{2.575099in}}%
\pgfpathlineto{\pgfqpoint{1.223914in}{2.575099in}}%
\pgfpathlineto{\pgfqpoint{1.223914in}{2.575773in}}%
\pgfpathlineto{\pgfqpoint{0.973011in}{2.575773in}}%
\pgfpathlineto{\pgfqpoint{0.973011in}{2.575099in}}%
\pgfpathclose%
\pgfusepath{stroke,fill}%
\end{pgfscope}%
\begin{pgfscope}%
\pgfpathrectangle{\pgfqpoint{0.781250in}{0.638889in}}{\pgfqpoint{4.218750in}{2.172222in}}%
\pgfusepath{clip}%
\pgfsetbuttcap%
\pgfsetmiterjoin%
\definecolor{currentfill}{rgb}{0.447059,0.447059,0.447059}%
\pgfsetfillcolor{currentfill}%
\pgfsetlinewidth{1.003750pt}%
\definecolor{currentstroke}{rgb}{0.266667,0.266667,0.266667}%
\pgfsetstrokecolor{currentstroke}%
\pgfsetdash{}{0pt}%
\pgfpathmoveto{\pgfqpoint{1.331444in}{2.142349in}}%
\pgfpathlineto{\pgfqpoint{1.582347in}{2.142349in}}%
\pgfpathlineto{\pgfqpoint{1.582347in}{2.142653in}}%
\pgfpathlineto{\pgfqpoint{1.331444in}{2.142653in}}%
\pgfpathlineto{\pgfqpoint{1.331444in}{2.142349in}}%
\pgfpathclose%
\pgfusepath{stroke,fill}%
\end{pgfscope}%
\begin{pgfscope}%
\pgfpathrectangle{\pgfqpoint{0.781250in}{0.638889in}}{\pgfqpoint{4.218750in}{2.172222in}}%
\pgfusepath{clip}%
\pgfsetbuttcap%
\pgfsetmiterjoin%
\definecolor{currentfill}{rgb}{0.447059,0.447059,0.447059}%
\pgfsetfillcolor{currentfill}%
\pgfsetlinewidth{1.003750pt}%
\definecolor{currentstroke}{rgb}{0.266667,0.266667,0.266667}%
\pgfsetstrokecolor{currentstroke}%
\pgfsetdash{}{0pt}%
\pgfpathmoveto{\pgfqpoint{1.689876in}{1.966030in}}%
\pgfpathlineto{\pgfqpoint{1.940779in}{1.966030in}}%
\pgfpathlineto{\pgfqpoint{1.940779in}{1.966073in}}%
\pgfpathlineto{\pgfqpoint{1.689876in}{1.966073in}}%
\pgfpathlineto{\pgfqpoint{1.689876in}{1.966030in}}%
\pgfpathclose%
\pgfusepath{stroke,fill}%
\end{pgfscope}%
\begin{pgfscope}%
\pgfpathrectangle{\pgfqpoint{0.781250in}{0.638889in}}{\pgfqpoint{4.218750in}{2.172222in}}%
\pgfusepath{clip}%
\pgfsetbuttcap%
\pgfsetmiterjoin%
\definecolor{currentfill}{rgb}{0.447059,0.447059,0.447059}%
\pgfsetfillcolor{currentfill}%
\pgfsetlinewidth{1.003750pt}%
\definecolor{currentstroke}{rgb}{0.266667,0.266667,0.266667}%
\pgfsetstrokecolor{currentstroke}%
\pgfsetdash{}{0pt}%
\pgfpathmoveto{\pgfqpoint{2.048309in}{1.899495in}}%
\pgfpathlineto{\pgfqpoint{2.299211in}{1.899495in}}%
\pgfpathlineto{\pgfqpoint{2.299211in}{1.899647in}}%
\pgfpathlineto{\pgfqpoint{2.048309in}{1.899647in}}%
\pgfpathlineto{\pgfqpoint{2.048309in}{1.899495in}}%
\pgfpathclose%
\pgfusepath{stroke,fill}%
\end{pgfscope}%
\begin{pgfscope}%
\pgfpathrectangle{\pgfqpoint{0.781250in}{0.638889in}}{\pgfqpoint{4.218750in}{2.172222in}}%
\pgfusepath{clip}%
\pgfsetbuttcap%
\pgfsetmiterjoin%
\definecolor{currentfill}{rgb}{0.447059,0.447059,0.447059}%
\pgfsetfillcolor{currentfill}%
\pgfsetlinewidth{1.003750pt}%
\definecolor{currentstroke}{rgb}{0.266667,0.266667,0.266667}%
\pgfsetstrokecolor{currentstroke}%
\pgfsetdash{}{0pt}%
\pgfpathmoveto{\pgfqpoint{2.406741in}{2.003479in}}%
\pgfpathlineto{\pgfqpoint{2.657644in}{2.003479in}}%
\pgfpathlineto{\pgfqpoint{2.657644in}{2.003609in}}%
\pgfpathlineto{\pgfqpoint{2.406741in}{2.003609in}}%
\pgfpathlineto{\pgfqpoint{2.406741in}{2.003479in}}%
\pgfpathclose%
\pgfusepath{stroke,fill}%
\end{pgfscope}%
\begin{pgfscope}%
\pgfpathrectangle{\pgfqpoint{0.781250in}{0.638889in}}{\pgfqpoint{4.218750in}{2.172222in}}%
\pgfusepath{clip}%
\pgfsetbuttcap%
\pgfsetmiterjoin%
\definecolor{currentfill}{rgb}{0.447059,0.447059,0.447059}%
\pgfsetfillcolor{currentfill}%
\pgfsetlinewidth{1.003750pt}%
\definecolor{currentstroke}{rgb}{0.266667,0.266667,0.266667}%
\pgfsetstrokecolor{currentstroke}%
\pgfsetdash{}{0pt}%
\pgfpathmoveto{\pgfqpoint{2.765174in}{2.195786in}}%
\pgfpathlineto{\pgfqpoint{3.016076in}{2.195786in}}%
\pgfpathlineto{\pgfqpoint{3.016076in}{2.195873in}}%
\pgfpathlineto{\pgfqpoint{2.765174in}{2.195873in}}%
\pgfpathlineto{\pgfqpoint{2.765174in}{2.195786in}}%
\pgfpathclose%
\pgfusepath{stroke,fill}%
\end{pgfscope}%
\begin{pgfscope}%
\pgfpathrectangle{\pgfqpoint{0.781250in}{0.638889in}}{\pgfqpoint{4.218750in}{2.172222in}}%
\pgfusepath{clip}%
\pgfsetbuttcap%
\pgfsetmiterjoin%
\definecolor{currentfill}{rgb}{0.447059,0.447059,0.447059}%
\pgfsetfillcolor{currentfill}%
\pgfsetlinewidth{1.003750pt}%
\definecolor{currentstroke}{rgb}{0.266667,0.266667,0.266667}%
\pgfsetstrokecolor{currentstroke}%
\pgfsetdash{}{0pt}%
\pgfpathmoveto{\pgfqpoint{3.123606in}{2.088456in}}%
\pgfpathlineto{\pgfqpoint{3.374509in}{2.088456in}}%
\pgfpathlineto{\pgfqpoint{3.374509in}{2.088543in}}%
\pgfpathlineto{\pgfqpoint{3.123606in}{2.088543in}}%
\pgfpathlineto{\pgfqpoint{3.123606in}{2.088456in}}%
\pgfpathclose%
\pgfusepath{stroke,fill}%
\end{pgfscope}%
\begin{pgfscope}%
\pgfpathrectangle{\pgfqpoint{0.781250in}{0.638889in}}{\pgfqpoint{4.218750in}{2.172222in}}%
\pgfusepath{clip}%
\pgfsetbuttcap%
\pgfsetmiterjoin%
\definecolor{currentfill}{rgb}{0.447059,0.447059,0.447059}%
\pgfsetfillcolor{currentfill}%
\pgfsetlinewidth{1.003750pt}%
\definecolor{currentstroke}{rgb}{0.266667,0.266667,0.266667}%
\pgfsetstrokecolor{currentstroke}%
\pgfsetdash{}{0pt}%
\pgfpathmoveto{\pgfqpoint{3.482039in}{2.056525in}}%
\pgfpathlineto{\pgfqpoint{3.732941in}{2.056525in}}%
\pgfpathlineto{\pgfqpoint{3.732941in}{2.056742in}}%
\pgfpathlineto{\pgfqpoint{3.482039in}{2.056742in}}%
\pgfpathlineto{\pgfqpoint{3.482039in}{2.056525in}}%
\pgfpathclose%
\pgfusepath{stroke,fill}%
\end{pgfscope}%
\begin{pgfscope}%
\pgfpathrectangle{\pgfqpoint{0.781250in}{0.638889in}}{\pgfqpoint{4.218750in}{2.172222in}}%
\pgfusepath{clip}%
\pgfsetbuttcap%
\pgfsetmiterjoin%
\definecolor{currentfill}{rgb}{0.447059,0.447059,0.447059}%
\pgfsetfillcolor{currentfill}%
\pgfsetlinewidth{1.003750pt}%
\definecolor{currentstroke}{rgb}{0.266667,0.266667,0.266667}%
\pgfsetstrokecolor{currentstroke}%
\pgfsetdash{}{0pt}%
\pgfpathmoveto{\pgfqpoint{3.840471in}{2.281936in}}%
\pgfpathlineto{\pgfqpoint{4.091374in}{2.281936in}}%
\pgfpathlineto{\pgfqpoint{4.091374in}{2.282088in}}%
\pgfpathlineto{\pgfqpoint{3.840471in}{2.282088in}}%
\pgfpathlineto{\pgfqpoint{3.840471in}{2.281936in}}%
\pgfpathclose%
\pgfusepath{stroke,fill}%
\end{pgfscope}%
\begin{pgfscope}%
\pgfpathrectangle{\pgfqpoint{0.781250in}{0.638889in}}{\pgfqpoint{4.218750in}{2.172222in}}%
\pgfusepath{clip}%
\pgfsetbuttcap%
\pgfsetmiterjoin%
\definecolor{currentfill}{rgb}{0.447059,0.447059,0.447059}%
\pgfsetfillcolor{currentfill}%
\pgfsetlinewidth{1.003750pt}%
\definecolor{currentstroke}{rgb}{0.266667,0.266667,0.266667}%
\pgfsetstrokecolor{currentstroke}%
\pgfsetdash{}{0pt}%
\pgfpathmoveto{\pgfqpoint{4.198903in}{2.323143in}}%
\pgfpathlineto{\pgfqpoint{4.449806in}{2.323143in}}%
\pgfpathlineto{\pgfqpoint{4.449806in}{2.323273in}}%
\pgfpathlineto{\pgfqpoint{4.198903in}{2.323273in}}%
\pgfpathlineto{\pgfqpoint{4.198903in}{2.323143in}}%
\pgfpathclose%
\pgfusepath{stroke,fill}%
\end{pgfscope}%
\begin{pgfscope}%
\pgfpathrectangle{\pgfqpoint{0.781250in}{0.638889in}}{\pgfqpoint{4.218750in}{2.172222in}}%
\pgfusepath{clip}%
\pgfsetbuttcap%
\pgfsetmiterjoin%
\definecolor{currentfill}{rgb}{0.447059,0.447059,0.447059}%
\pgfsetfillcolor{currentfill}%
\pgfsetlinewidth{1.003750pt}%
\definecolor{currentstroke}{rgb}{0.266667,0.266667,0.266667}%
\pgfsetstrokecolor{currentstroke}%
\pgfsetdash{}{0pt}%
\pgfpathmoveto{\pgfqpoint{4.557336in}{2.479478in}}%
\pgfpathlineto{\pgfqpoint{4.808239in}{2.479478in}}%
\pgfpathlineto{\pgfqpoint{4.808239in}{2.479782in}}%
\pgfpathlineto{\pgfqpoint{4.557336in}{2.479782in}}%
\pgfpathlineto{\pgfqpoint{4.557336in}{2.479478in}}%
\pgfpathclose%
\pgfusepath{stroke,fill}%
\end{pgfscope}%
\begin{pgfscope}%
\definecolor{textcolor}{rgb}{0.000000,0.000000,0.000000}%
\pgfsetstrokecolor{textcolor}%
\pgfsetfillcolor{textcolor}%
\pgftext[x=1.098463in,y=2.603550in,,bottom]{\color{textcolor}{\ifdefined\pdftexversion\else\setmainfont{NanumMyeongjo}\rmfamily\fi\fontsize{5.000000}{6.000000}\selectfont\catcode`\^=\active\def^{\ifmmode\sp\else\^{}\fi}\catcode`\%=\active\def%{\%}89,166}}%
\end{pgfscope}%
\begin{pgfscope}%
\definecolor{textcolor}{rgb}{0.000000,0.000000,0.000000}%
\pgfsetstrokecolor{textcolor}%
\pgfsetfillcolor{textcolor}%
\pgftext[x=1.456895in,y=2.170431in,,bottom]{\color{textcolor}{\ifdefined\pdftexversion\else\setmainfont{NanumMyeongjo}\rmfamily\fi\fontsize{5.000000}{6.000000}\selectfont\catcode`\^=\active\def^{\ifmmode\sp\else\^{}\fi}\catcode`\%=\active\def%{\%}69,227}}%
\end{pgfscope}%
\begin{pgfscope}%
\definecolor{textcolor}{rgb}{0.000000,0.000000,0.000000}%
\pgfsetstrokecolor{textcolor}%
\pgfsetfillcolor{textcolor}%
\pgftext[x=1.815328in,y=1.993851in,,bottom]{\color{textcolor}{\ifdefined\pdftexversion\else\setmainfont{NanumMyeongjo}\rmfamily\fi\fontsize{5.000000}{6.000000}\selectfont\catcode`\^=\active\def^{\ifmmode\sp\else\^{}\fi}\catcode`\%=\active\def%{\%}61,098}}%
\end{pgfscope}%
\begin{pgfscope}%
\definecolor{textcolor}{rgb}{0.000000,0.000000,0.000000}%
\pgfsetstrokecolor{textcolor}%
\pgfsetfillcolor{textcolor}%
\pgftext[x=2.173760in,y=1.927424in,,bottom]{\color{textcolor}{\ifdefined\pdftexversion\else\setmainfont{NanumMyeongjo}\rmfamily\fi\fontsize{5.000000}{6.000000}\selectfont\catcode`\^=\active\def^{\ifmmode\sp\else\^{}\fi}\catcode`\%=\active\def%{\%}58,040}}%
\end{pgfscope}%
\begin{pgfscope}%
\definecolor{textcolor}{rgb}{0.000000,0.000000,0.000000}%
\pgfsetstrokecolor{textcolor}%
\pgfsetfillcolor{textcolor}%
\pgftext[x=2.532193in,y=2.031387in,,bottom]{\color{textcolor}{\ifdefined\pdftexversion\else\setmainfont{NanumMyeongjo}\rmfamily\fi\fontsize{5.000000}{6.000000}\selectfont\catcode`\^=\active\def^{\ifmmode\sp\else\^{}\fi}\catcode`\%=\active\def%{\%}62,826}}%
\end{pgfscope}%
\begin{pgfscope}%
\definecolor{textcolor}{rgb}{0.000000,0.000000,0.000000}%
\pgfsetstrokecolor{textcolor}%
\pgfsetfillcolor{textcolor}%
\pgftext[x=2.890625in,y=2.223650in,,bottom]{\color{textcolor}{\ifdefined\pdftexversion\else\setmainfont{NanumMyeongjo}\rmfamily\fi\fontsize{5.000000}{6.000000}\selectfont\catcode`\^=\active\def^{\ifmmode\sp\else\^{}\fi}\catcode`\%=\active\def%{\%}71,677}}%
\end{pgfscope}%
\begin{pgfscope}%
\definecolor{textcolor}{rgb}{0.000000,0.000000,0.000000}%
\pgfsetstrokecolor{textcolor}%
\pgfsetfillcolor{textcolor}%
\pgftext[x=3.249057in,y=2.116321in,,bottom]{\color{textcolor}{\ifdefined\pdftexversion\else\setmainfont{NanumMyeongjo}\rmfamily\fi\fontsize{5.000000}{6.000000}\selectfont\catcode`\^=\active\def^{\ifmmode\sp\else\^{}\fi}\catcode`\%=\active\def%{\%}66,736}}%
\end{pgfscope}%
\begin{pgfscope}%
\definecolor{textcolor}{rgb}{0.000000,0.000000,0.000000}%
\pgfsetstrokecolor{textcolor}%
\pgfsetfillcolor{textcolor}%
\pgftext[x=3.607490in,y=2.084520in,,bottom]{\color{textcolor}{\ifdefined\pdftexversion\else\setmainfont{NanumMyeongjo}\rmfamily\fi\fontsize{5.000000}{6.000000}\selectfont\catcode`\^=\active\def^{\ifmmode\sp\else\^{}\fi}\catcode`\%=\active\def%{\%}65,272}}%
\end{pgfscope}%
\begin{pgfscope}%
\definecolor{textcolor}{rgb}{0.000000,0.000000,0.000000}%
\pgfsetstrokecolor{textcolor}%
\pgfsetfillcolor{textcolor}%
\pgftext[x=3.965922in,y=2.309866in,,bottom]{\color{textcolor}{\ifdefined\pdftexversion\else\setmainfont{NanumMyeongjo}\rmfamily\fi\fontsize{5.000000}{6.000000}\selectfont\catcode`\^=\active\def^{\ifmmode\sp\else\^{}\fi}\catcode`\%=\active\def%{\%}75,646}}%
\end{pgfscope}%
\begin{pgfscope}%
\definecolor{textcolor}{rgb}{0.000000,0.000000,0.000000}%
\pgfsetstrokecolor{textcolor}%
\pgfsetfillcolor{textcolor}%
\pgftext[x=4.324355in,y=2.351051in,,bottom]{\color{textcolor}{\ifdefined\pdftexversion\else\setmainfont{NanumMyeongjo}\rmfamily\fi\fontsize{5.000000}{6.000000}\selectfont\catcode`\^=\active\def^{\ifmmode\sp\else\^{}\fi}\catcode`\%=\active\def%{\%}77,542}}%
\end{pgfscope}%
\begin{pgfscope}%
\definecolor{textcolor}{rgb}{0.000000,0.000000,0.000000}%
\pgfsetstrokecolor{textcolor}%
\pgfsetfillcolor{textcolor}%
\pgftext[x=4.682787in,y=2.507560in,,bottom]{\color{textcolor}{\ifdefined\pdftexversion\else\setmainfont{NanumMyeongjo}\rmfamily\fi\fontsize{5.000000}{6.000000}\selectfont\catcode`\^=\active\def^{\ifmmode\sp\else\^{}\fi}\catcode`\%=\active\def%{\%}84,747}}%
\end{pgfscope}%
\begin{pgfscope}%
\definecolor{textcolor}{rgb}{1.000000,1.000000,1.000000}%
\pgfsetstrokecolor{textcolor}%
\pgfsetfillcolor{textcolor}%
\pgftext[x=1.098463in,y=0.731078in,,]{\color{textcolor}{\ifdefined\pdftexversion\else\setmainfont{NanumMyeongjo}\rmfamily\fi\fontsize{5.000000}{6.000000}\selectfont\catcode`\^=\active\def^{\ifmmode\sp\else\^{}\fi}\catcode`\%=\active\def%{\%}7,244}}%
\end{pgfscope}%
\begin{pgfscope}%
\definecolor{textcolor}{rgb}{1.000000,1.000000,1.000000}%
\pgfsetstrokecolor{textcolor}%
\pgfsetfillcolor{textcolor}%
\pgftext[x=1.456895in,y=0.686656in,,]{\color{textcolor}{\ifdefined\pdftexversion\else\setmainfont{NanumMyeongjo}\rmfamily\fi\fontsize{5.000000}{6.000000}\selectfont\catcode`\^=\active\def^{\ifmmode\sp\else\^{}\fi}\catcode`\%=\active\def%{\%}5,199}}%
\end{pgfscope}%
\begin{pgfscope}%
\definecolor{textcolor}{rgb}{1.000000,1.000000,1.000000}%
\pgfsetstrokecolor{textcolor}%
\pgfsetfillcolor{textcolor}%
\pgftext[x=1.815328in,y=0.681269in,,]{\color{textcolor}{\ifdefined\pdftexversion\else\setmainfont{NanumMyeongjo}\rmfamily\fi\fontsize{5.000000}{6.000000}\selectfont\catcode`\^=\active\def^{\ifmmode\sp\else\^{}\fi}\catcode`\%=\active\def%{\%}4,951}}%
\end{pgfscope}%
\begin{pgfscope}%
\definecolor{textcolor}{rgb}{1.000000,1.000000,1.000000}%
\pgfsetstrokecolor{textcolor}%
\pgfsetfillcolor{textcolor}%
\pgftext[x=2.173760in,y=0.705772in,,]{\color{textcolor}{\ifdefined\pdftexversion\else\setmainfont{NanumMyeongjo}\rmfamily\fi\fontsize{5.000000}{6.000000}\selectfont\catcode`\^=\active\def^{\ifmmode\sp\else\^{}\fi}\catcode`\%=\active\def%{\%}6,079}}%
\end{pgfscope}%
\begin{pgfscope}%
\definecolor{textcolor}{rgb}{1.000000,1.000000,1.000000}%
\pgfsetstrokecolor{textcolor}%
\pgfsetfillcolor{textcolor}%
\pgftext[x=2.532193in,y=0.766681in,,]{\color{textcolor}{\ifdefined\pdftexversion\else\setmainfont{NanumMyeongjo}\rmfamily\fi\fontsize{5.000000}{6.000000}\selectfont\catcode`\^=\active\def^{\ifmmode\sp\else\^{}\fi}\catcode`\%=\active\def%{\%}8,883}}%
\end{pgfscope}%
\begin{pgfscope}%
\definecolor{textcolor}{rgb}{1.000000,1.000000,1.000000}%
\pgfsetstrokecolor{textcolor}%
\pgfsetfillcolor{textcolor}%
\pgftext[x=2.890625in,y=0.843882in,,]{\color{textcolor}{\ifdefined\pdftexversion\else\setmainfont{NanumMyeongjo}\rmfamily\fi\fontsize{5.000000}{6.000000}\selectfont\catcode`\^=\active\def^{\ifmmode\sp\else\^{}\fi}\catcode`\%=\active\def%{\%}12,437}}%
\end{pgfscope}%
\begin{pgfscope}%
\definecolor{textcolor}{rgb}{1.000000,1.000000,1.000000}%
\pgfsetstrokecolor{textcolor}%
\pgfsetfillcolor{textcolor}%
\pgftext[x=3.249057in,y=0.827307in,,]{\color{textcolor}{\ifdefined\pdftexversion\else\setmainfont{NanumMyeongjo}\rmfamily\fi\fontsize{5.000000}{6.000000}\selectfont\catcode`\^=\active\def^{\ifmmode\sp\else\^{}\fi}\catcode`\%=\active\def%{\%}11,674}}%
\end{pgfscope}%
\begin{pgfscope}%
\definecolor{textcolor}{rgb}{1.000000,1.000000,1.000000}%
\pgfsetstrokecolor{textcolor}%
\pgfsetfillcolor{textcolor}%
\pgftext[x=3.607490in,y=0.795332in,,]{\color{textcolor}{\ifdefined\pdftexversion\else\setmainfont{NanumMyeongjo}\rmfamily\fi\fontsize{5.000000}{6.000000}\selectfont\catcode`\^=\active\def^{\ifmmode\sp\else\^{}\fi}\catcode`\%=\active\def%{\%}10,202}}%
\end{pgfscope}%
\begin{pgfscope}%
\definecolor{textcolor}{rgb}{1.000000,1.000000,1.000000}%
\pgfsetstrokecolor{textcolor}%
\pgfsetfillcolor{textcolor}%
\pgftext[x=3.965922in,y=0.865864in,,]{\color{textcolor}{\ifdefined\pdftexversion\else\setmainfont{NanumMyeongjo}\rmfamily\fi\fontsize{5.000000}{6.000000}\selectfont\catcode`\^=\active\def^{\ifmmode\sp\else\^{}\fi}\catcode`\%=\active\def%{\%}13,449}}%
\end{pgfscope}%
\begin{pgfscope}%
\definecolor{textcolor}{rgb}{1.000000,1.000000,1.000000}%
\pgfsetstrokecolor{textcolor}%
\pgfsetfillcolor{textcolor}%
\pgftext[x=4.324355in,y=0.955708in,,]{\color{textcolor}{\ifdefined\pdftexversion\else\setmainfont{NanumMyeongjo}\rmfamily\fi\fontsize{5.000000}{6.000000}\selectfont\catcode`\^=\active\def^{\ifmmode\sp\else\^{}\fi}\catcode`\%=\active\def%{\%}17,585}}%
\end{pgfscope}%
\begin{pgfscope}%
\definecolor{textcolor}{rgb}{1.000000,1.000000,1.000000}%
\pgfsetstrokecolor{textcolor}%
\pgfsetfillcolor{textcolor}%
\pgftext[x=4.682787in,y=1.021656in,,]{\color{textcolor}{\ifdefined\pdftexversion\else\setmainfont{NanumMyeongjo}\rmfamily\fi\fontsize{5.000000}{6.000000}\selectfont\catcode`\^=\active\def^{\ifmmode\sp\else\^{}\fi}\catcode`\%=\active\def%{\%}20,621}}%
\end{pgfscope}%
\begin{pgfscope}%
\definecolor{textcolor}{rgb}{1.000000,1.000000,1.000000}%
\pgfsetstrokecolor{textcolor}%
\pgfsetfillcolor{textcolor}%
\pgftext[x=1.098463in,y=1.061386in,,]{\color{textcolor}{\ifdefined\pdftexversion\else\setmainfont{NanumMyeongjo}\rmfamily\fi\fontsize{5.000000}{6.000000}\selectfont\catcode`\^=\active\def^{\ifmmode\sp\else\^{}\fi}\catcode`\%=\active\def%{\%}15,206}}%
\end{pgfscope}%
\begin{pgfscope}%
\definecolor{textcolor}{rgb}{1.000000,1.000000,1.000000}%
\pgfsetstrokecolor{textcolor}%
\pgfsetfillcolor{textcolor}%
\pgftext[x=1.456895in,y=0.938569in,,]{\color{textcolor}{\ifdefined\pdftexversion\else\setmainfont{NanumMyeongjo}\rmfamily\fi\fontsize{5.000000}{6.000000}\selectfont\catcode`\^=\active\def^{\ifmmode\sp\else\^{}\fi}\catcode`\%=\active\def%{\%}11,597}}%
\end{pgfscope}%
\begin{pgfscope}%
\definecolor{textcolor}{rgb}{1.000000,1.000000,1.000000}%
\pgfsetstrokecolor{textcolor}%
\pgfsetfillcolor{textcolor}%
\pgftext[x=1.815328in,y=0.896623in,,]{\color{textcolor}{\ifdefined\pdftexversion\else\setmainfont{NanumMyeongjo}\rmfamily\fi\fontsize{5.000000}{6.000000}\selectfont\catcode`\^=\active\def^{\ifmmode\sp\else\^{}\fi}\catcode`\%=\active\def%{\%}9,914}}%
\end{pgfscope}%
\begin{pgfscope}%
\definecolor{textcolor}{rgb}{1.000000,1.000000,1.000000}%
\pgfsetstrokecolor{textcolor}%
\pgfsetfillcolor{textcolor}%
\pgftext[x=2.173760in,y=0.919323in,,]{\color{textcolor}{\ifdefined\pdftexversion\else\setmainfont{NanumMyeongjo}\rmfamily\fi\fontsize{5.000000}{6.000000}\selectfont\catcode`\^=\active\def^{\ifmmode\sp\else\^{}\fi}\catcode`\%=\active\def%{\%}9,831}}%
\end{pgfscope}%
\begin{pgfscope}%
\definecolor{textcolor}{rgb}{1.000000,1.000000,1.000000}%
\pgfsetstrokecolor{textcolor}%
\pgfsetfillcolor{textcolor}%
\pgftext[x=2.532193in,y=0.995676in,,]{\color{textcolor}{\ifdefined\pdftexversion\else\setmainfont{NanumMyeongjo}\rmfamily\fi\fontsize{5.000000}{6.000000}\selectfont\catcode`\^=\active\def^{\ifmmode\sp\else\^{}\fi}\catcode`\%=\active\def%{\%}10,542}}%
\end{pgfscope}%
\begin{pgfscope}%
\definecolor{textcolor}{rgb}{1.000000,1.000000,1.000000}%
\pgfsetstrokecolor{textcolor}%
\pgfsetfillcolor{textcolor}%
\pgftext[x=2.890625in,y=1.077656in,,]{\color{textcolor}{\ifdefined\pdftexversion\else\setmainfont{NanumMyeongjo}\rmfamily\fi\fontsize{5.000000}{6.000000}\selectfont\catcode`\^=\active\def^{\ifmmode\sp\else\^{}\fi}\catcode`\%=\active\def%{\%}10,762}}%
\end{pgfscope}%
\begin{pgfscope}%
\definecolor{textcolor}{rgb}{1.000000,1.000000,1.000000}%
\pgfsetstrokecolor{textcolor}%
\pgfsetfillcolor{textcolor}%
\pgftext[x=3.249057in,y=1.047093in,,]{\color{textcolor}{\ifdefined\pdftexversion\else\setmainfont{NanumMyeongjo}\rmfamily\fi\fontsize{5.000000}{6.000000}\selectfont\catcode`\^=\active\def^{\ifmmode\sp\else\^{}\fi}\catcode`\%=\active\def%{\%}10,118}}%
\end{pgfscope}%
\begin{pgfscope}%
\definecolor{textcolor}{rgb}{1.000000,1.000000,1.000000}%
\pgfsetstrokecolor{textcolor}%
\pgfsetfillcolor{textcolor}%
\pgftext[x=3.607490in,y=1.033256in,,]{\color{textcolor}{\ifdefined\pdftexversion\else\setmainfont{NanumMyeongjo}\rmfamily\fi\fontsize{5.000000}{6.000000}\selectfont\catcode`\^=\active\def^{\ifmmode\sp\else\^{}\fi}\catcode`\%=\active\def%{\%}10,953}}%
\end{pgfscope}%
\begin{pgfscope}%
\definecolor{textcolor}{rgb}{1.000000,1.000000,1.000000}%
\pgfsetstrokecolor{textcolor}%
\pgfsetfillcolor{textcolor}%
\pgftext[x=3.965922in,y=1.145777in,,]{\color{textcolor}{\ifdefined\pdftexversion\else\setmainfont{NanumMyeongjo}\rmfamily\fi\fontsize{5.000000}{6.000000}\selectfont\catcode`\^=\active\def^{\ifmmode\sp\else\^{}\fi}\catcode`\%=\active\def%{\%}12,886}}%
\end{pgfscope}%
\begin{pgfscope}%
\definecolor{textcolor}{rgb}{1.000000,1.000000,1.000000}%
\pgfsetstrokecolor{textcolor}%
\pgfsetfillcolor{textcolor}%
\pgftext[x=4.324355in,y=1.217743in,,]{\color{textcolor}{\ifdefined\pdftexversion\else\setmainfont{NanumMyeongjo}\rmfamily\fi\fontsize{5.000000}{6.000000}\selectfont\catcode`\^=\active\def^{\ifmmode\sp\else\^{}\fi}\catcode`\%=\active\def%{\%}12,063}}%
\end{pgfscope}%
\begin{pgfscope}%
\definecolor{textcolor}{rgb}{1.000000,1.000000,1.000000}%
\pgfsetstrokecolor{textcolor}%
\pgfsetfillcolor{textcolor}%
\pgftext[x=4.682787in,y=1.307455in,,]{\color{textcolor}{\ifdefined\pdftexversion\else\setmainfont{NanumMyeongjo}\rmfamily\fi\fontsize{5.000000}{6.000000}\selectfont\catcode`\^=\active\def^{\ifmmode\sp\else\^{}\fi}\catcode`\%=\active\def%{\%}13,157}}%
\end{pgfscope}%
\begin{pgfscope}%
\definecolor{textcolor}{rgb}{1.000000,1.000000,1.000000}%
\pgfsetstrokecolor{textcolor}%
\pgfsetfillcolor{textcolor}%
\pgftext[x=1.098463in,y=1.407747in,,]{\color{textcolor}{\ifdefined\pdftexversion\else\setmainfont{NanumMyeongjo}\rmfamily\fi\fontsize{5.000000}{6.000000}\selectfont\catcode`\^=\active\def^{\ifmmode\sp\else\^{}\fi}\catcode`\%=\active\def%{\%}15,945}}%
\end{pgfscope}%
\begin{pgfscope}%
\definecolor{textcolor}{rgb}{1.000000,1.000000,1.000000}%
\pgfsetstrokecolor{textcolor}%
\pgfsetfillcolor{textcolor}%
\pgftext[x=1.456895in,y=1.228951in,,]{\color{textcolor}{\ifdefined\pdftexversion\else\setmainfont{NanumMyeongjo}\rmfamily\fi\fontsize{5.000000}{6.000000}\selectfont\catcode`\^=\active\def^{\ifmmode\sp\else\^{}\fi}\catcode`\%=\active\def%{\%}13,368}}%
\end{pgfscope}%
\begin{pgfscope}%
\definecolor{textcolor}{rgb}{1.000000,1.000000,1.000000}%
\pgfsetstrokecolor{textcolor}%
\pgfsetfillcolor{textcolor}%
\pgftext[x=1.815328in,y=1.145104in,,]{\color{textcolor}{\ifdefined\pdftexversion\else\setmainfont{NanumMyeongjo}\rmfamily\fi\fontsize{5.000000}{6.000000}\selectfont\catcode`\^=\active\def^{\ifmmode\sp\else\^{}\fi}\catcode`\%=\active\def%{\%}11,439}}%
\end{pgfscope}%
\begin{pgfscope}%
\definecolor{textcolor}{rgb}{1.000000,1.000000,1.000000}%
\pgfsetstrokecolor{textcolor}%
\pgfsetfillcolor{textcolor}%
\pgftext[x=2.173760in,y=1.143931in,,]{\color{textcolor}{\ifdefined\pdftexversion\else\setmainfont{NanumMyeongjo}\rmfamily\fi\fontsize{5.000000}{6.000000}\selectfont\catcode`\^=\active\def^{\ifmmode\sp\else\^{}\fi}\catcode`\%=\active\def%{\%}10,340}}%
\end{pgfscope}%
\begin{pgfscope}%
\definecolor{textcolor}{rgb}{1.000000,1.000000,1.000000}%
\pgfsetstrokecolor{textcolor}%
\pgfsetfillcolor{textcolor}%
\pgftext[x=2.532193in,y=1.242962in,,]{\color{textcolor}{\ifdefined\pdftexversion\else\setmainfont{NanumMyeongjo}\rmfamily\fi\fontsize{5.000000}{6.000000}\selectfont\catcode`\^=\active\def^{\ifmmode\sp\else\^{}\fi}\catcode`\%=\active\def%{\%}11,384}}%
\end{pgfscope}%
\begin{pgfscope}%
\definecolor{textcolor}{rgb}{1.000000,1.000000,1.000000}%
\pgfsetstrokecolor{textcolor}%
\pgfsetfillcolor{textcolor}%
\pgftext[x=2.890625in,y=1.342168in,,]{\color{textcolor}{\ifdefined\pdftexversion\else\setmainfont{NanumMyeongjo}\rmfamily\fi\fontsize{5.000000}{6.000000}\selectfont\catcode`\^=\active\def^{\ifmmode\sp\else\^{}\fi}\catcode`\%=\active\def%{\%}12,177}}%
\end{pgfscope}%
\begin{pgfscope}%
\definecolor{textcolor}{rgb}{1.000000,1.000000,1.000000}%
\pgfsetstrokecolor{textcolor}%
\pgfsetfillcolor{textcolor}%
\pgftext[x=3.249057in,y=1.262990in,,]{\color{textcolor}{\ifdefined\pdftexversion\else\setmainfont{NanumMyeongjo}\rmfamily\fi\fontsize{5.000000}{6.000000}\selectfont\catcode`\^=\active\def^{\ifmmode\sp\else\^{}\fi}\catcode`\%=\active\def%{\%}9,939}}%
\end{pgfscope}%
\begin{pgfscope}%
\definecolor{textcolor}{rgb}{1.000000,1.000000,1.000000}%
\pgfsetstrokecolor{textcolor}%
\pgfsetfillcolor{textcolor}%
\pgftext[x=3.607490in,y=1.236989in,,]{\color{textcolor}{\ifdefined\pdftexversion\else\setmainfont{NanumMyeongjo}\rmfamily\fi\fontsize{5.000000}{6.000000}\selectfont\catcode`\^=\active\def^{\ifmmode\sp\else\^{}\fi}\catcode`\%=\active\def%{\%}9,379}}%
\end{pgfscope}%
\begin{pgfscope}%
\definecolor{textcolor}{rgb}{1.000000,1.000000,1.000000}%
\pgfsetstrokecolor{textcolor}%
\pgfsetfillcolor{textcolor}%
\pgftext[x=3.965922in,y=1.384113in,,]{\color{textcolor}{\ifdefined\pdftexversion\else\setmainfont{NanumMyeongjo}\rmfamily\fi\fontsize{5.000000}{6.000000}\selectfont\catcode`\^=\active\def^{\ifmmode\sp\else\^{}\fi}\catcode`\%=\active\def%{\%}10,972}}%
\end{pgfscope}%
\begin{pgfscope}%
\definecolor{textcolor}{rgb}{1.000000,1.000000,1.000000}%
\pgfsetstrokecolor{textcolor}%
\pgfsetfillcolor{textcolor}%
\pgftext[x=4.324355in,y=1.448889in,,]{\color{textcolor}{\ifdefined\pdftexversion\else\setmainfont{NanumMyeongjo}\rmfamily\fi\fontsize{5.000000}{6.000000}\selectfont\catcode`\^=\active\def^{\ifmmode\sp\else\^{}\fi}\catcode`\%=\active\def%{\%}10,641}}%
\end{pgfscope}%
\begin{pgfscope}%
\definecolor{textcolor}{rgb}{1.000000,1.000000,1.000000}%
\pgfsetstrokecolor{textcolor}%
\pgfsetfillcolor{textcolor}%
\pgftext[x=4.682787in,y=1.573726in,,]{\color{textcolor}{\ifdefined\pdftexversion\else\setmainfont{NanumMyeongjo}\rmfamily\fi\fontsize{5.000000}{6.000000}\selectfont\catcode`\^=\active\def^{\ifmmode\sp\else\^{}\fi}\catcode`\%=\active\def%{\%}12,258}}%
\end{pgfscope}%
\begin{pgfscope}%
\pgfsetbuttcap%
\pgfsetmiterjoin%
\definecolor{currentfill}{rgb}{0.337255,0.713725,0.627451}%
\pgfsetfillcolor{currentfill}%
\pgfsetlinewidth{1.003750pt}%
\definecolor{currentstroke}{rgb}{0.266667,0.266667,0.266667}%
\pgfsetstrokecolor{currentstroke}%
\pgfsetdash{}{0pt}%
\pgfpathmoveto{\pgfqpoint{5.028125in}{2.598758in}}%
\pgfpathlineto{\pgfqpoint{5.278125in}{2.598758in}}%
\pgfpathlineto{\pgfqpoint{5.278125in}{2.686258in}}%
\pgfpathlineto{\pgfqpoint{5.028125in}{2.686258in}}%
\pgfpathlineto{\pgfqpoint{5.028125in}{2.598758in}}%
\pgfpathclose%
\pgfusepath{stroke,fill}%
\end{pgfscope}%
\begin{pgfscope}%
\definecolor{textcolor}{rgb}{0.000000,0.000000,0.000000}%
\pgfsetstrokecolor{textcolor}%
\pgfsetfillcolor{textcolor}%
\pgftext[x=5.378125in,y=2.598758in,left,base]{\color{textcolor}{\ifdefined\pdftexversion\else\setmainfont{NanumMyeongjo}\rmfamily\fi\fontsize{9.000000}{10.800000}\selectfont\catcode`\^=\active\def^{\ifmmode\sp\else\^{}\fi}\catcode`\%=\active\def%{\%}전라북도}}%
\end{pgfscope}%
\begin{pgfscope}%
\pgfsetbuttcap%
\pgfsetmiterjoin%
\definecolor{currentfill}{rgb}{0.235294,0.490196,0.764706}%
\pgfsetfillcolor{currentfill}%
\pgfsetlinewidth{1.003750pt}%
\definecolor{currentstroke}{rgb}{0.266667,0.266667,0.266667}%
\pgfsetstrokecolor{currentstroke}%
\pgfsetdash{}{0pt}%
\pgfpathmoveto{\pgfqpoint{5.028125in}{2.407474in}}%
\pgfpathlineto{\pgfqpoint{5.278125in}{2.407474in}}%
\pgfpathlineto{\pgfqpoint{5.278125in}{2.494974in}}%
\pgfpathlineto{\pgfqpoint{5.028125in}{2.494974in}}%
\pgfpathlineto{\pgfqpoint{5.028125in}{2.407474in}}%
\pgfpathclose%
\pgfusepath{stroke,fill}%
\end{pgfscope}%
\begin{pgfscope}%
\definecolor{textcolor}{rgb}{0.000000,0.000000,0.000000}%
\pgfsetstrokecolor{textcolor}%
\pgfsetfillcolor{textcolor}%
\pgftext[x=5.378125in,y=2.407474in,left,base]{\color{textcolor}{\ifdefined\pdftexversion\else\setmainfont{NanumMyeongjo}\rmfamily\fi\fontsize{9.000000}{10.800000}\selectfont\catcode`\^=\active\def^{\ifmmode\sp\else\^{}\fi}\catcode`\%=\active\def%{\%}경상북도}}%
\end{pgfscope}%
\begin{pgfscope}%
\pgfsetbuttcap%
\pgfsetmiterjoin%
\definecolor{currentfill}{rgb}{0.725490,0.486275,0.164706}%
\pgfsetfillcolor{currentfill}%
\pgfsetlinewidth{1.003750pt}%
\definecolor{currentstroke}{rgb}{0.266667,0.266667,0.266667}%
\pgfsetstrokecolor{currentstroke}%
\pgfsetdash{}{0pt}%
\pgfpathmoveto{\pgfqpoint{5.028125in}{2.216190in}}%
\pgfpathlineto{\pgfqpoint{5.278125in}{2.216190in}}%
\pgfpathlineto{\pgfqpoint{5.278125in}{2.303690in}}%
\pgfpathlineto{\pgfqpoint{5.028125in}{2.303690in}}%
\pgfpathlineto{\pgfqpoint{5.028125in}{2.216190in}}%
\pgfpathclose%
\pgfusepath{stroke,fill}%
\end{pgfscope}%
\begin{pgfscope}%
\definecolor{textcolor}{rgb}{0.000000,0.000000,0.000000}%
\pgfsetstrokecolor{textcolor}%
\pgfsetfillcolor{textcolor}%
\pgftext[x=5.378125in,y=2.216190in,left,base]{\color{textcolor}{\ifdefined\pdftexversion\else\setmainfont{NanumMyeongjo}\rmfamily\fi\fontsize{9.000000}{10.800000}\selectfont\catcode`\^=\active\def^{\ifmmode\sp\else\^{}\fi}\catcode`\%=\active\def%{\%}전라남도}}%
\end{pgfscope}%
\begin{pgfscope}%
\pgfsetbuttcap%
\pgfsetmiterjoin%
\definecolor{currentfill}{rgb}{0.733333,0.321569,0.733333}%
\pgfsetfillcolor{currentfill}%
\pgfsetlinewidth{1.003750pt}%
\definecolor{currentstroke}{rgb}{0.266667,0.266667,0.266667}%
\pgfsetstrokecolor{currentstroke}%
\pgfsetdash{}{0pt}%
\pgfpathmoveto{\pgfqpoint{5.028125in}{2.024906in}}%
\pgfpathlineto{\pgfqpoint{5.278125in}{2.024906in}}%
\pgfpathlineto{\pgfqpoint{5.278125in}{2.112406in}}%
\pgfpathlineto{\pgfqpoint{5.028125in}{2.112406in}}%
\pgfpathlineto{\pgfqpoint{5.028125in}{2.024906in}}%
\pgfpathclose%
\pgfusepath{stroke,fill}%
\end{pgfscope}%
\begin{pgfscope}%
\definecolor{textcolor}{rgb}{0.000000,0.000000,0.000000}%
\pgfsetstrokecolor{textcolor}%
\pgfsetfillcolor{textcolor}%
\pgftext[x=5.378125in,y=2.024906in,left,base]{\color{textcolor}{\ifdefined\pdftexversion\else\setmainfont{NanumMyeongjo}\rmfamily\fi\fontsize{9.000000}{10.800000}\selectfont\catcode`\^=\active\def^{\ifmmode\sp\else\^{}\fi}\catcode`\%=\active\def%{\%}충청남도}}%
\end{pgfscope}%
\begin{pgfscope}%
\pgfsetbuttcap%
\pgfsetmiterjoin%
\definecolor{currentfill}{rgb}{0.549020,0.247059,0.121569}%
\pgfsetfillcolor{currentfill}%
\pgfsetlinewidth{1.003750pt}%
\definecolor{currentstroke}{rgb}{0.266667,0.266667,0.266667}%
\pgfsetstrokecolor{currentstroke}%
\pgfsetdash{}{0pt}%
\pgfpathmoveto{\pgfqpoint{5.028125in}{1.833622in}}%
\pgfpathlineto{\pgfqpoint{5.278125in}{1.833622in}}%
\pgfpathlineto{\pgfqpoint{5.278125in}{1.921122in}}%
\pgfpathlineto{\pgfqpoint{5.028125in}{1.921122in}}%
\pgfpathlineto{\pgfqpoint{5.028125in}{1.833622in}}%
\pgfpathclose%
\pgfusepath{stroke,fill}%
\end{pgfscope}%
\begin{pgfscope}%
\definecolor{textcolor}{rgb}{0.000000,0.000000,0.000000}%
\pgfsetstrokecolor{textcolor}%
\pgfsetfillcolor{textcolor}%
\pgftext[x=5.378125in,y=1.833622in,left,base]{\color{textcolor}{\ifdefined\pdftexversion\else\setmainfont{NanumMyeongjo}\rmfamily\fi\fontsize{9.000000}{10.800000}\selectfont\catcode`\^=\active\def^{\ifmmode\sp\else\^{}\fi}\catcode`\%=\active\def%{\%}충청북도}}%
\end{pgfscope}%
\begin{pgfscope}%
\pgfsetbuttcap%
\pgfsetmiterjoin%
\definecolor{currentfill}{rgb}{0.701961,0.760784,0.360784}%
\pgfsetfillcolor{currentfill}%
\pgfsetlinewidth{1.003750pt}%
\definecolor{currentstroke}{rgb}{0.266667,0.266667,0.266667}%
\pgfsetstrokecolor{currentstroke}%
\pgfsetdash{}{0pt}%
\pgfpathmoveto{\pgfqpoint{5.028125in}{1.642338in}}%
\pgfpathlineto{\pgfqpoint{5.278125in}{1.642338in}}%
\pgfpathlineto{\pgfqpoint{5.278125in}{1.729837in}}%
\pgfpathlineto{\pgfqpoint{5.028125in}{1.729837in}}%
\pgfpathlineto{\pgfqpoint{5.028125in}{1.642338in}}%
\pgfpathclose%
\pgfusepath{stroke,fill}%
\end{pgfscope}%
\begin{pgfscope}%
\definecolor{textcolor}{rgb}{0.000000,0.000000,0.000000}%
\pgfsetstrokecolor{textcolor}%
\pgfsetfillcolor{textcolor}%
\pgftext[x=5.378125in,y=1.642337in,left,base]{\color{textcolor}{\ifdefined\pdftexversion\else\setmainfont{NanumMyeongjo}\rmfamily\fi\fontsize{9.000000}{10.800000}\selectfont\catcode`\^=\active\def^{\ifmmode\sp\else\^{}\fi}\catcode`\%=\active\def%{\%}경기도}}%
\end{pgfscope}%
\begin{pgfscope}%
\pgfsetbuttcap%
\pgfsetmiterjoin%
\definecolor{currentfill}{rgb}{0.447059,0.447059,0.447059}%
\pgfsetfillcolor{currentfill}%
\pgfsetlinewidth{1.003750pt}%
\definecolor{currentstroke}{rgb}{0.266667,0.266667,0.266667}%
\pgfsetstrokecolor{currentstroke}%
\pgfsetdash{}{0pt}%
\pgfpathmoveto{\pgfqpoint{5.028125in}{1.451053in}}%
\pgfpathlineto{\pgfqpoint{5.278125in}{1.451053in}}%
\pgfpathlineto{\pgfqpoint{5.278125in}{1.538553in}}%
\pgfpathlineto{\pgfqpoint{5.028125in}{1.538553in}}%
\pgfpathlineto{\pgfqpoint{5.028125in}{1.451053in}}%
\pgfpathclose%
\pgfusepath{stroke,fill}%
\end{pgfscope}%
\begin{pgfscope}%
\definecolor{textcolor}{rgb}{0.000000,0.000000,0.000000}%
\pgfsetstrokecolor{textcolor}%
\pgfsetfillcolor{textcolor}%
\pgftext[x=5.378125in,y=1.451053in,left,base]{\color{textcolor}{\ifdefined\pdftexversion\else\setmainfont{NanumMyeongjo}\rmfamily\fi\fontsize{9.000000}{10.800000}\selectfont\catcode`\^=\active\def^{\ifmmode\sp\else\^{}\fi}\catcode`\%=\active\def%{\%}기타}}%
\end{pgfscope}%
\begin{pgfscope}%
\definecolor{textcolor}{rgb}{0.333333,0.333333,0.333333}%
\pgfsetstrokecolor{textcolor}%
\pgfsetfillcolor{textcolor}%
\pgftext[x=2.062500in,y=0.159722in,,top]{\color{textcolor}{\ifdefined\pdftexversion\else\setmainfont{NanumMyeongjo}\rmfamily\fi\fontsize{9.000000}{10.800000}\selectfont\catcode`\^=\active\def^{\ifmmode\sp\else\^{}\fi}\catcode`\%=\active\def%{\%}출처: 국가농식품통계서비스(KASS) 자료 기반 저자 작성}}%
\end{pgfscope}%
\begin{pgfscope}%
\definecolor{textcolor}{rgb}{0.333333,0.333333,0.333333}%
\pgfsetstrokecolor{textcolor}%
\pgfsetfillcolor{textcolor}%
\pgftext[x=4.687500in,y=3.034722in,,top]{\color{textcolor}{\ifdefined\pdftexversion\else\setmainfont{NanumMyeongjo}\rmfamily\fi\fontsize{9.000000}{10.800000}\selectfont\catcode`\^=\active\def^{\ifmmode\sp\else\^{}\fi}\catcode`\%=\active\def%{\%}(단위: ha)}}%
\end{pgfscope}%
\end{pgfpicture}%
\makeatother%
\endgroup%
}
\end{center}
}
