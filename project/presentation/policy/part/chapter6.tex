
\intro
{\chaptersix}

\slide
{\maintitle}
{\chaptersix}
{제도상의 미비점 - 관련 기사}{
\vspace{10pt}

\begin{center}
\includegraphics[width=0.9\textwidth]{asset/전업농신문_제도뒷전.png}
\end{center}
\vspace{-10pt}
{\small ※ 전업농신문}

\begin{itemize}
    \item 콩 재해보험 약관 "병해충 피해는 제외" \\
    올해 논콩 피해의 대부분이 \\
    습해에 의한 자주무늬병, 곰팡이병
    \item 전북, 경북, 충북지역 \\
    농업재해법상  인정되어 복구비 지급 하지만\\ 
    보험가입농가는 '이중 보조'라는 이유로 제외
\end{itemize}

}

\slide
{\maintitle}
{\chaptersix}
{기사 내용 검토 - 보험 약관}{
\vspace{10pt}

\begin{center}
\includegraphics[width=0.8\textwidth]{asset/콩_병충해_제외.png}
\end{center}

\begin{center}
\includegraphics[width=0.8\textwidth]{asset/미보상감수량.png}
\end{center}
\vspace{-10pt}
{\small ※ 농작물재해보럼 밭작물 약관}
}


\slide
{\maintitle}
{\chaptersix}
{기사 내용 검토 - 농어업재해대책법}{
\begin{center}
\includegraphics[width=1\textwidth]{asset/농어업재해대책법.png}
\end{center}
\begin{center}
\includegraphics[width=1\textwidth]{asset/대책법_보상항목.png}
\end{center}
\vspace{-10pt}
{\small ※ 농어업재해대책법}

}


\slide
{\maintitle}
{\chaptersix}
{소결}{

\begin{tcolorbox}[colback=white, colframe=black, boxrule=0.8pt, rounded corners]
\begin{itemize}
    \item 논은 일반적으로 지대가 낮고 물빠짐이 좋지 않기에 \\ 적지에 논콩을 재배하는 것이 중요함
    \item 우리나라는 콩의 생장에 중요한 늦여름에 \\강수량이 몰려있기에 배수 관리에 각별히 신경써야 함
    \item 최근 발생한 초여름과 초가을 장마로 \\논콩 침수 피해가 큰 규모로 발생함  
    \item 논콩 생산 농가는 기상이변에 철저히 대비해야하며,\\ 정부 또한 재해보장제도를 완비해야 함
\end{itemize}
\end{tcolorbox}

}
