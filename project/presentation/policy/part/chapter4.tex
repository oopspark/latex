
\intro
{\chapterfour}


\slide
{\maintitle}
{\chapterfour}
{주요 재해 대응 제도}{

\begin{itemize}
    \item 「재난 및 안전관리 기본법」상 \textbf{특별재난지역 선포}:\\ 시설·운전 자금의 상환 기한 연기 및 그 이자 감면 등
\vspace{5pt}
    \item 「농어업재해대책법」(이하 농업재해법)상 \textbf{농업재해 인정}: \\ 농경지 복구비 및 농약, 종묘, 비료 대금 지원
\vspace{5pt}
    \item 「농어업재해보험법」상 \textbf{재해보험 가입}: \\ 보험료 지원(정부지원 33\textasciitilde 60\%, 지방자치단체 30\textasciitilde 50\%)
\vspace{5pt}
    \item NH농협손해보험에서 출시한 \textbf{농업수입안정보험 가입}: \\기준수입 보장; 정부 및 지자체에서 보험료 일부 지원
\end{itemize}

}


\slide
{\maintitle}
{\chapterfour}
{제도상의 미비점 - 관련 기사}{
\vspace{10pt}

\begin{center}
\includegraphics[width=0.9\textwidth]{asset/전업농신문_제도뒷전.png}
\end{center}
\vspace{-10pt}
{\small ※ 전업농신문}

\begin{itemize}
    \item 콩 재해보험 약관 "병해충 피해는 제외" \\
    올해 논콩 피해의 대부분이 \\
    습해에 의한 자주무늬병, 곰팡이병
    \item 전북, 경북, 충북지역 \\
    농어업재해대책법상 농업재해가 인정되어 복구비 지급 \\ 
    하지만 재해보험 가입농가는 '이중 보조'라는 이유로 제외
\end{itemize}

}

\slide
{\maintitle}
{\chapterfour}
{기사 내용 검토 - 보험 약관}{
\vspace{10pt}

\begin{center}
\includegraphics[width=0.8\textwidth]{asset/콩_병충해_제외.png}
\end{center}

\begin{center}
\includegraphics[width=0.8\textwidth]{asset/미보상감수량.png}
\end{center}
\vspace{-10pt}
{\small ※ 농작물재해보럼 밭작물 약관}
\begin{itemize}
    \item 콩은 재해보험 약관상 병충해 보상 불가
    \item 병해충 피해 감수량 제외시 보험금 과소 책정 가능
\end{itemize}

}


\slide
{\maintitle}
{\chapterfour}
{기사 내용 검토 - 농업재해법}{
\vspace{0pt}
\begin{center}
\includegraphics[width=1\textwidth]{asset/농어업재해대책법.png}
\end{center}
\begin{center}
\includegraphics[width=0.8\textwidth]{asset/대책법_보상항목.png}
\end{center}
\vspace{-10pt}
{\small ※ 농어업재해대책법}

\vspace{10pt}
\begin{itemize}
    \item 재해보험으로 수령한 보험금이 \\농업재해대책법 상 지원액보다 적은 경우 차액 지원 가능
    \item 하지만 재량행위로 규정되어 있어 실제 지원여부 불확실
\end{itemize}
}


\slide
{\maintitle}
{\chapterfour}
{미비점 개선방안 고찰 1}{
\vspace{10pt}


\begin{itemize}
    \item 습해에 이은 병충해는 논콩의 대표적인 수확량 감소 원인임
    \item 만일, 재해보험 보장대상으로 포함시킬 경우 \\\textbf{보험금 지급 과다 및 도덕적 해이 발생}이 우려됨
    \item 따라서, \textbf{배수 시설 확충 등 재해 예방을 지원}하고 정부 지도에 적극 협조한 농가에 보험료 할인 등의 혜택을 주는 등 \textbf{경제적 유인책}을 쓰는 것이 바람직함
\end{itemize}

}


\slide
{\maintitle}
{\chapterfour}
{미비점 개선방안 고찰 2}{
\vspace{10pt}


\begin{itemize}
    \item 재해보험 가입 농가를 농업재해법 상 지원농가에서 분리하는 것은 \textbf{재해보험가입 동기 및 효용을 감소}시킴
    \item 농업재해법에 의한 지원은 재해보험 가입여부에 상관없이 지급하는 것이 \textbf{소득 하한선을 보장}하는데 적절함
    \item 동일 항목에 대한 보상금 이중수령 문제는 \textbf{재해보험 가입자의 예측가능성 증진}을 위해 농업재해법이 아니라 재해보험 상품 차원에서 교정되어야 함 
    \item 장기적으로 \textbf{최저수준 농업재해 보장방식을 일원화}할 필요성이 있음
\end{itemize}

}

\slide
{\maintitle}
{\chapterfour}
{소결}{

\begin{tcolorbox}[colback=white, colframe=black, boxrule=0.8pt, rounded corners]
\begin{itemize}
    \item 농업재해에 대응하는 제도는 특별재난지역 선포, \\정부에 의한 농업재해 인정, \\재해보험 및 농업수입안정보험 등이 있음
    \item 논콩의 경우 병해충 피해가 많으나 \\재해보험 보장대상이 아님
    \item 농업재해가 인정되어도 재해보험 가입자는 \\그에 따른 지원 여부 불확실함
    \item 재해보험 가입 농가의 도덕적 해이를 방지하면서도 \\소득 하한선을 보장하고 \\예측가능성을 증진시킬 방안이 요구됨
\end{itemize}
\end{tcolorbox}

}
