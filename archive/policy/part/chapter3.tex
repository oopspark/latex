

\intro
{\chapterthree}

\slide
{\maintitle}
{\chapterthree}
{국내 콩 생산량}{
\vspace{-10pt}
\begin{center}
    \hspace*{-40pt}{%% Creator: Matplotlib, PGF backend
%%
%% To include the figure in your LaTeX document, write
%%   \input{<filename>.pgf}
%%
%% Make sure the required packages are loaded in your preamble
%%   \usepackage{pgf}
%%
%% Also ensure that all the required font packages are loaded; for instance,
%% the lmodern package is sometimes necessary when using math font.
%%   \usepackage{lmodern}
%%
%% Figures using additional raster images can only be included by \input if
%% they are in the same directory as the main LaTeX file. For loading figures
%% from other directories you can use the `import` package
%%   \usepackage{import}
%%
%% and then include the figures with
%%   \import{<path to file>}{<filename>.pgf}
%%
%% Matplotlib used the following preamble
%%   \def\mathdefault#1{#1}
%%   \everymath=\expandafter{\the\everymath\displaystyle}
%%   \IfFileExists{scrextend.sty}{
%%     \usepackage[fontsize=5.000000pt]{scrextend}
%%   }{
%%     \renewcommand{\normalsize}{\fontsize{5.000000}{6.000000}\selectfont}
%%     \normalsize
%%   }
%%   
%%   \ifdefined\pdftexversion\else  % non-pdftex case.
%%     \usepackage{fontspec}
%%     \setmainfont{DejaVuSerif.ttf}[Path=\detokenize{/home/user/.cache/pypoetry/virtualenvs/graph-KASAOWVd-py3.12/lib/python3.12/site-packages/matplotlib/mpl-data/fonts/ttf/}]
%%     \setsansfont{DejaVuSans.ttf}[Path=\detokenize{/home/user/.cache/pypoetry/virtualenvs/graph-KASAOWVd-py3.12/lib/python3.12/site-packages/matplotlib/mpl-data/fonts/ttf/}]
%%     \setmonofont{DejaVuSansMono.ttf}[Path=\detokenize{/home/user/.cache/pypoetry/virtualenvs/graph-KASAOWVd-py3.12/lib/python3.12/site-packages/matplotlib/mpl-data/fonts/ttf/}]
%%   \fi
%%   \makeatletter\@ifpackageloaded{underscore}{}{\usepackage[strings]{underscore}}\makeatother
%%
\begingroup%
\makeatletter%
\begin{pgfpicture}%
\pgfpathrectangle{\pgfpointorigin}{\pgfqpoint{6.944444in}{2.777778in}}%
\pgfusepath{use as bounding box, clip}%
\begin{pgfscope}%
\pgfsetbuttcap%
\pgfsetmiterjoin%
\definecolor{currentfill}{rgb}{1.000000,1.000000,1.000000}%
\pgfsetfillcolor{currentfill}%
\pgfsetlinewidth{0.000000pt}%
\definecolor{currentstroke}{rgb}{1.000000,1.000000,1.000000}%
\pgfsetstrokecolor{currentstroke}%
\pgfsetdash{}{0pt}%
\pgfpathmoveto{\pgfqpoint{0.000000in}{0.000000in}}%
\pgfpathlineto{\pgfqpoint{6.944444in}{0.000000in}}%
\pgfpathlineto{\pgfqpoint{6.944444in}{2.777778in}}%
\pgfpathlineto{\pgfqpoint{0.000000in}{2.777778in}}%
\pgfpathlineto{\pgfqpoint{0.000000in}{0.000000in}}%
\pgfpathclose%
\pgfusepath{fill}%
\end{pgfscope}%
\begin{pgfscope}%
\pgfsetbuttcap%
\pgfsetmiterjoin%
\definecolor{currentfill}{rgb}{1.000000,1.000000,1.000000}%
\pgfsetfillcolor{currentfill}%
\pgfsetlinewidth{0.000000pt}%
\definecolor{currentstroke}{rgb}{0.000000,0.000000,0.000000}%
\pgfsetstrokecolor{currentstroke}%
\pgfsetstrokeopacity{0.000000}%
\pgfsetdash{}{0pt}%
\pgfpathmoveto{\pgfqpoint{0.868056in}{0.555556in}}%
\pgfpathlineto{\pgfqpoint{4.861111in}{0.555556in}}%
\pgfpathlineto{\pgfqpoint{4.861111in}{2.444444in}}%
\pgfpathlineto{\pgfqpoint{0.868056in}{2.444444in}}%
\pgfpathlineto{\pgfqpoint{0.868056in}{0.555556in}}%
\pgfpathclose%
\pgfusepath{fill}%
\end{pgfscope}%
\begin{pgfscope}%
\pgfsetbuttcap%
\pgfsetroundjoin%
\definecolor{currentfill}{rgb}{0.000000,0.000000,0.000000}%
\pgfsetfillcolor{currentfill}%
\pgfsetlinewidth{0.752812pt}%
\definecolor{currentstroke}{rgb}{0.000000,0.000000,0.000000}%
\pgfsetstrokecolor{currentstroke}%
\pgfsetdash{}{0pt}%
\pgfsys@defobject{currentmarker}{\pgfqpoint{0.000000in}{-0.013889in}}{\pgfqpoint{0.000000in}{0.000000in}}{%
\pgfpathmoveto{\pgfqpoint{0.000000in}{0.000000in}}%
\pgfpathlineto{\pgfqpoint{0.000000in}{-0.013889in}}%
\pgfusepath{stroke,fill}%
}%
\begin{pgfscope}%
\pgfsys@transformshift{1.168298in}{0.555556in}%
\pgfsys@useobject{currentmarker}{}%
\end{pgfscope}%
\end{pgfscope}%
\begin{pgfscope}%
\definecolor{textcolor}{rgb}{0.000000,0.000000,0.000000}%
\pgfsetstrokecolor{textcolor}%
\pgfsetfillcolor{textcolor}%
\pgftext[x=1.168298in,y=0.493056in,,top]{\color{textcolor}{\ifdefined\pdftexversion\else\setmainfont{NanumMyeongjo}\rmfamily\fi\fontsize{5.000000}{6.000000}\selectfont\catcode`\^=\active\def^{\ifmmode\sp\else\^{}\fi}\catcode`\%=\active\def%{\%}2014}}%
\end{pgfscope}%
\begin{pgfscope}%
\pgfsetbuttcap%
\pgfsetroundjoin%
\definecolor{currentfill}{rgb}{0.000000,0.000000,0.000000}%
\pgfsetfillcolor{currentfill}%
\pgfsetlinewidth{0.752812pt}%
\definecolor{currentstroke}{rgb}{0.000000,0.000000,0.000000}%
\pgfsetstrokecolor{currentstroke}%
\pgfsetdash{}{0pt}%
\pgfsys@defobject{currentmarker}{\pgfqpoint{0.000000in}{-0.013889in}}{\pgfqpoint{0.000000in}{0.000000in}}{%
\pgfpathmoveto{\pgfqpoint{0.000000in}{0.000000in}}%
\pgfpathlineto{\pgfqpoint{0.000000in}{-0.013889in}}%
\pgfusepath{stroke,fill}%
}%
\begin{pgfscope}%
\pgfsys@transformshift{1.507555in}{0.555556in}%
\pgfsys@useobject{currentmarker}{}%
\end{pgfscope}%
\end{pgfscope}%
\begin{pgfscope}%
\definecolor{textcolor}{rgb}{0.000000,0.000000,0.000000}%
\pgfsetstrokecolor{textcolor}%
\pgfsetfillcolor{textcolor}%
\pgftext[x=1.507555in,y=0.493056in,,top]{\color{textcolor}{\ifdefined\pdftexversion\else\setmainfont{NanumMyeongjo}\rmfamily\fi\fontsize{5.000000}{6.000000}\selectfont\catcode`\^=\active\def^{\ifmmode\sp\else\^{}\fi}\catcode`\%=\active\def%{\%}2015}}%
\end{pgfscope}%
\begin{pgfscope}%
\pgfsetbuttcap%
\pgfsetroundjoin%
\definecolor{currentfill}{rgb}{0.000000,0.000000,0.000000}%
\pgfsetfillcolor{currentfill}%
\pgfsetlinewidth{0.752812pt}%
\definecolor{currentstroke}{rgb}{0.000000,0.000000,0.000000}%
\pgfsetstrokecolor{currentstroke}%
\pgfsetdash{}{0pt}%
\pgfsys@defobject{currentmarker}{\pgfqpoint{0.000000in}{-0.013889in}}{\pgfqpoint{0.000000in}{0.000000in}}{%
\pgfpathmoveto{\pgfqpoint{0.000000in}{0.000000in}}%
\pgfpathlineto{\pgfqpoint{0.000000in}{-0.013889in}}%
\pgfusepath{stroke,fill}%
}%
\begin{pgfscope}%
\pgfsys@transformshift{1.846812in}{0.555556in}%
\pgfsys@useobject{currentmarker}{}%
\end{pgfscope}%
\end{pgfscope}%
\begin{pgfscope}%
\definecolor{textcolor}{rgb}{0.000000,0.000000,0.000000}%
\pgfsetstrokecolor{textcolor}%
\pgfsetfillcolor{textcolor}%
\pgftext[x=1.846812in,y=0.493056in,,top]{\color{textcolor}{\ifdefined\pdftexversion\else\setmainfont{NanumMyeongjo}\rmfamily\fi\fontsize{5.000000}{6.000000}\selectfont\catcode`\^=\active\def^{\ifmmode\sp\else\^{}\fi}\catcode`\%=\active\def%{\%}2016}}%
\end{pgfscope}%
\begin{pgfscope}%
\pgfsetbuttcap%
\pgfsetroundjoin%
\definecolor{currentfill}{rgb}{0.000000,0.000000,0.000000}%
\pgfsetfillcolor{currentfill}%
\pgfsetlinewidth{0.752812pt}%
\definecolor{currentstroke}{rgb}{0.000000,0.000000,0.000000}%
\pgfsetstrokecolor{currentstroke}%
\pgfsetdash{}{0pt}%
\pgfsys@defobject{currentmarker}{\pgfqpoint{0.000000in}{-0.013889in}}{\pgfqpoint{0.000000in}{0.000000in}}{%
\pgfpathmoveto{\pgfqpoint{0.000000in}{0.000000in}}%
\pgfpathlineto{\pgfqpoint{0.000000in}{-0.013889in}}%
\pgfusepath{stroke,fill}%
}%
\begin{pgfscope}%
\pgfsys@transformshift{2.186069in}{0.555556in}%
\pgfsys@useobject{currentmarker}{}%
\end{pgfscope}%
\end{pgfscope}%
\begin{pgfscope}%
\definecolor{textcolor}{rgb}{0.000000,0.000000,0.000000}%
\pgfsetstrokecolor{textcolor}%
\pgfsetfillcolor{textcolor}%
\pgftext[x=2.186069in,y=0.493056in,,top]{\color{textcolor}{\ifdefined\pdftexversion\else\setmainfont{NanumMyeongjo}\rmfamily\fi\fontsize{5.000000}{6.000000}\selectfont\catcode`\^=\active\def^{\ifmmode\sp\else\^{}\fi}\catcode`\%=\active\def%{\%}2017}}%
\end{pgfscope}%
\begin{pgfscope}%
\pgfsetbuttcap%
\pgfsetroundjoin%
\definecolor{currentfill}{rgb}{0.000000,0.000000,0.000000}%
\pgfsetfillcolor{currentfill}%
\pgfsetlinewidth{0.752812pt}%
\definecolor{currentstroke}{rgb}{0.000000,0.000000,0.000000}%
\pgfsetstrokecolor{currentstroke}%
\pgfsetdash{}{0pt}%
\pgfsys@defobject{currentmarker}{\pgfqpoint{0.000000in}{-0.013889in}}{\pgfqpoint{0.000000in}{0.000000in}}{%
\pgfpathmoveto{\pgfqpoint{0.000000in}{0.000000in}}%
\pgfpathlineto{\pgfqpoint{0.000000in}{-0.013889in}}%
\pgfusepath{stroke,fill}%
}%
\begin{pgfscope}%
\pgfsys@transformshift{2.525326in}{0.555556in}%
\pgfsys@useobject{currentmarker}{}%
\end{pgfscope}%
\end{pgfscope}%
\begin{pgfscope}%
\definecolor{textcolor}{rgb}{0.000000,0.000000,0.000000}%
\pgfsetstrokecolor{textcolor}%
\pgfsetfillcolor{textcolor}%
\pgftext[x=2.525326in,y=0.493056in,,top]{\color{textcolor}{\ifdefined\pdftexversion\else\setmainfont{NanumMyeongjo}\rmfamily\fi\fontsize{5.000000}{6.000000}\selectfont\catcode`\^=\active\def^{\ifmmode\sp\else\^{}\fi}\catcode`\%=\active\def%{\%}2018}}%
\end{pgfscope}%
\begin{pgfscope}%
\pgfsetbuttcap%
\pgfsetroundjoin%
\definecolor{currentfill}{rgb}{0.000000,0.000000,0.000000}%
\pgfsetfillcolor{currentfill}%
\pgfsetlinewidth{0.752812pt}%
\definecolor{currentstroke}{rgb}{0.000000,0.000000,0.000000}%
\pgfsetstrokecolor{currentstroke}%
\pgfsetdash{}{0pt}%
\pgfsys@defobject{currentmarker}{\pgfqpoint{0.000000in}{-0.013889in}}{\pgfqpoint{0.000000in}{0.000000in}}{%
\pgfpathmoveto{\pgfqpoint{0.000000in}{0.000000in}}%
\pgfpathlineto{\pgfqpoint{0.000000in}{-0.013889in}}%
\pgfusepath{stroke,fill}%
}%
\begin{pgfscope}%
\pgfsys@transformshift{2.864583in}{0.555556in}%
\pgfsys@useobject{currentmarker}{}%
\end{pgfscope}%
\end{pgfscope}%
\begin{pgfscope}%
\definecolor{textcolor}{rgb}{0.000000,0.000000,0.000000}%
\pgfsetstrokecolor{textcolor}%
\pgfsetfillcolor{textcolor}%
\pgftext[x=2.864583in,y=0.493056in,,top]{\color{textcolor}{\ifdefined\pdftexversion\else\setmainfont{NanumMyeongjo}\rmfamily\fi\fontsize{5.000000}{6.000000}\selectfont\catcode`\^=\active\def^{\ifmmode\sp\else\^{}\fi}\catcode`\%=\active\def%{\%}2019}}%
\end{pgfscope}%
\begin{pgfscope}%
\pgfsetbuttcap%
\pgfsetroundjoin%
\definecolor{currentfill}{rgb}{0.000000,0.000000,0.000000}%
\pgfsetfillcolor{currentfill}%
\pgfsetlinewidth{0.752812pt}%
\definecolor{currentstroke}{rgb}{0.000000,0.000000,0.000000}%
\pgfsetstrokecolor{currentstroke}%
\pgfsetdash{}{0pt}%
\pgfsys@defobject{currentmarker}{\pgfqpoint{0.000000in}{-0.013889in}}{\pgfqpoint{0.000000in}{0.000000in}}{%
\pgfpathmoveto{\pgfqpoint{0.000000in}{0.000000in}}%
\pgfpathlineto{\pgfqpoint{0.000000in}{-0.013889in}}%
\pgfusepath{stroke,fill}%
}%
\begin{pgfscope}%
\pgfsys@transformshift{3.203840in}{0.555556in}%
\pgfsys@useobject{currentmarker}{}%
\end{pgfscope}%
\end{pgfscope}%
\begin{pgfscope}%
\definecolor{textcolor}{rgb}{0.000000,0.000000,0.000000}%
\pgfsetstrokecolor{textcolor}%
\pgfsetfillcolor{textcolor}%
\pgftext[x=3.203840in,y=0.493056in,,top]{\color{textcolor}{\ifdefined\pdftexversion\else\setmainfont{NanumMyeongjo}\rmfamily\fi\fontsize{5.000000}{6.000000}\selectfont\catcode`\^=\active\def^{\ifmmode\sp\else\^{}\fi}\catcode`\%=\active\def%{\%}2020}}%
\end{pgfscope}%
\begin{pgfscope}%
\pgfsetbuttcap%
\pgfsetroundjoin%
\definecolor{currentfill}{rgb}{0.000000,0.000000,0.000000}%
\pgfsetfillcolor{currentfill}%
\pgfsetlinewidth{0.752812pt}%
\definecolor{currentstroke}{rgb}{0.000000,0.000000,0.000000}%
\pgfsetstrokecolor{currentstroke}%
\pgfsetdash{}{0pt}%
\pgfsys@defobject{currentmarker}{\pgfqpoint{0.000000in}{-0.013889in}}{\pgfqpoint{0.000000in}{0.000000in}}{%
\pgfpathmoveto{\pgfqpoint{0.000000in}{0.000000in}}%
\pgfpathlineto{\pgfqpoint{0.000000in}{-0.013889in}}%
\pgfusepath{stroke,fill}%
}%
\begin{pgfscope}%
\pgfsys@transformshift{3.543097in}{0.555556in}%
\pgfsys@useobject{currentmarker}{}%
\end{pgfscope}%
\end{pgfscope}%
\begin{pgfscope}%
\definecolor{textcolor}{rgb}{0.000000,0.000000,0.000000}%
\pgfsetstrokecolor{textcolor}%
\pgfsetfillcolor{textcolor}%
\pgftext[x=3.543097in,y=0.493056in,,top]{\color{textcolor}{\ifdefined\pdftexversion\else\setmainfont{NanumMyeongjo}\rmfamily\fi\fontsize{5.000000}{6.000000}\selectfont\catcode`\^=\active\def^{\ifmmode\sp\else\^{}\fi}\catcode`\%=\active\def%{\%}2021}}%
\end{pgfscope}%
\begin{pgfscope}%
\pgfsetbuttcap%
\pgfsetroundjoin%
\definecolor{currentfill}{rgb}{0.000000,0.000000,0.000000}%
\pgfsetfillcolor{currentfill}%
\pgfsetlinewidth{0.752812pt}%
\definecolor{currentstroke}{rgb}{0.000000,0.000000,0.000000}%
\pgfsetstrokecolor{currentstroke}%
\pgfsetdash{}{0pt}%
\pgfsys@defobject{currentmarker}{\pgfqpoint{0.000000in}{-0.013889in}}{\pgfqpoint{0.000000in}{0.000000in}}{%
\pgfpathmoveto{\pgfqpoint{0.000000in}{0.000000in}}%
\pgfpathlineto{\pgfqpoint{0.000000in}{-0.013889in}}%
\pgfusepath{stroke,fill}%
}%
\begin{pgfscope}%
\pgfsys@transformshift{3.882355in}{0.555556in}%
\pgfsys@useobject{currentmarker}{}%
\end{pgfscope}%
\end{pgfscope}%
\begin{pgfscope}%
\definecolor{textcolor}{rgb}{0.000000,0.000000,0.000000}%
\pgfsetstrokecolor{textcolor}%
\pgfsetfillcolor{textcolor}%
\pgftext[x=3.882355in,y=0.493056in,,top]{\color{textcolor}{\ifdefined\pdftexversion\else\setmainfont{NanumMyeongjo}\rmfamily\fi\fontsize{5.000000}{6.000000}\selectfont\catcode`\^=\active\def^{\ifmmode\sp\else\^{}\fi}\catcode`\%=\active\def%{\%}2022}}%
\end{pgfscope}%
\begin{pgfscope}%
\pgfsetbuttcap%
\pgfsetroundjoin%
\definecolor{currentfill}{rgb}{0.000000,0.000000,0.000000}%
\pgfsetfillcolor{currentfill}%
\pgfsetlinewidth{0.752812pt}%
\definecolor{currentstroke}{rgb}{0.000000,0.000000,0.000000}%
\pgfsetstrokecolor{currentstroke}%
\pgfsetdash{}{0pt}%
\pgfsys@defobject{currentmarker}{\pgfqpoint{0.000000in}{-0.013889in}}{\pgfqpoint{0.000000in}{0.000000in}}{%
\pgfpathmoveto{\pgfqpoint{0.000000in}{0.000000in}}%
\pgfpathlineto{\pgfqpoint{0.000000in}{-0.013889in}}%
\pgfusepath{stroke,fill}%
}%
\begin{pgfscope}%
\pgfsys@transformshift{4.221612in}{0.555556in}%
\pgfsys@useobject{currentmarker}{}%
\end{pgfscope}%
\end{pgfscope}%
\begin{pgfscope}%
\definecolor{textcolor}{rgb}{0.000000,0.000000,0.000000}%
\pgfsetstrokecolor{textcolor}%
\pgfsetfillcolor{textcolor}%
\pgftext[x=4.221612in,y=0.493056in,,top]{\color{textcolor}{\ifdefined\pdftexversion\else\setmainfont{NanumMyeongjo}\rmfamily\fi\fontsize{5.000000}{6.000000}\selectfont\catcode`\^=\active\def^{\ifmmode\sp\else\^{}\fi}\catcode`\%=\active\def%{\%}2023}}%
\end{pgfscope}%
\begin{pgfscope}%
\pgfsetbuttcap%
\pgfsetroundjoin%
\definecolor{currentfill}{rgb}{0.000000,0.000000,0.000000}%
\pgfsetfillcolor{currentfill}%
\pgfsetlinewidth{0.752812pt}%
\definecolor{currentstroke}{rgb}{0.000000,0.000000,0.000000}%
\pgfsetstrokecolor{currentstroke}%
\pgfsetdash{}{0pt}%
\pgfsys@defobject{currentmarker}{\pgfqpoint{0.000000in}{-0.013889in}}{\pgfqpoint{0.000000in}{0.000000in}}{%
\pgfpathmoveto{\pgfqpoint{0.000000in}{0.000000in}}%
\pgfpathlineto{\pgfqpoint{0.000000in}{-0.013889in}}%
\pgfusepath{stroke,fill}%
}%
\begin{pgfscope}%
\pgfsys@transformshift{4.560869in}{0.555556in}%
\pgfsys@useobject{currentmarker}{}%
\end{pgfscope}%
\end{pgfscope}%
\begin{pgfscope}%
\definecolor{textcolor}{rgb}{0.000000,0.000000,0.000000}%
\pgfsetstrokecolor{textcolor}%
\pgfsetfillcolor{textcolor}%
\pgftext[x=4.560869in,y=0.493056in,,top]{\color{textcolor}{\ifdefined\pdftexversion\else\setmainfont{NanumMyeongjo}\rmfamily\fi\fontsize{5.000000}{6.000000}\selectfont\catcode`\^=\active\def^{\ifmmode\sp\else\^{}\fi}\catcode`\%=\active\def%{\%}2024}}%
\end{pgfscope}%
\begin{pgfscope}%
\pgfpathrectangle{\pgfqpoint{0.868056in}{0.555556in}}{\pgfqpoint{3.993056in}{1.888889in}}%
\pgfusepath{clip}%
\pgfsetbuttcap%
\pgfsetroundjoin%
\pgfsetlinewidth{0.602250pt}%
\definecolor{currentstroke}{rgb}{0.690196,0.690196,0.690196}%
\pgfsetstrokecolor{currentstroke}%
\pgfsetstrokeopacity{0.450000}%
\pgfsetdash{{2.220000pt}{0.960000pt}}{0.000000pt}%
\pgfpathmoveto{\pgfqpoint{0.868056in}{0.555556in}}%
\pgfpathlineto{\pgfqpoint{4.861111in}{0.555556in}}%
\pgfusepath{stroke}%
\end{pgfscope}%
\begin{pgfscope}%
\pgfsetbuttcap%
\pgfsetroundjoin%
\definecolor{currentfill}{rgb}{0.000000,0.000000,0.000000}%
\pgfsetfillcolor{currentfill}%
\pgfsetlinewidth{0.752812pt}%
\definecolor{currentstroke}{rgb}{0.000000,0.000000,0.000000}%
\pgfsetstrokecolor{currentstroke}%
\pgfsetdash{}{0pt}%
\pgfsys@defobject{currentmarker}{\pgfqpoint{-0.013889in}{0.000000in}}{\pgfqpoint{-0.000000in}{0.000000in}}{%
\pgfpathmoveto{\pgfqpoint{-0.000000in}{0.000000in}}%
\pgfpathlineto{\pgfqpoint{-0.013889in}{0.000000in}}%
\pgfusepath{stroke,fill}%
}%
\begin{pgfscope}%
\pgfsys@transformshift{0.868056in}{0.555556in}%
\pgfsys@useobject{currentmarker}{}%
\end{pgfscope}%
\end{pgfscope}%
\begin{pgfscope}%
\definecolor{textcolor}{rgb}{0.000000,0.000000,0.000000}%
\pgfsetstrokecolor{textcolor}%
\pgfsetfillcolor{textcolor}%
\pgftext[x=0.768256in, y=0.527818in, left, base]{\color{textcolor}{\ifdefined\pdftexversion\else\setmainfont{NanumMyeongjo}\rmfamily\fi\fontsize{5.000000}{6.000000}\selectfont\catcode`\^=\active\def^{\ifmmode\sp\else\^{}\fi}\catcode`\%=\active\def%{\%}0}}%
\end{pgfscope}%
\begin{pgfscope}%
\pgfpathrectangle{\pgfqpoint{0.868056in}{0.555556in}}{\pgfqpoint{3.993056in}{1.888889in}}%
\pgfusepath{clip}%
\pgfsetbuttcap%
\pgfsetroundjoin%
\pgfsetlinewidth{0.602250pt}%
\definecolor{currentstroke}{rgb}{0.690196,0.690196,0.690196}%
\pgfsetstrokecolor{currentstroke}%
\pgfsetstrokeopacity{0.450000}%
\pgfsetdash{{2.220000pt}{0.960000pt}}{0.000000pt}%
\pgfpathmoveto{\pgfqpoint{0.868056in}{0.762577in}}%
\pgfpathlineto{\pgfqpoint{4.861111in}{0.762577in}}%
\pgfusepath{stroke}%
\end{pgfscope}%
\begin{pgfscope}%
\pgfsetbuttcap%
\pgfsetroundjoin%
\definecolor{currentfill}{rgb}{0.000000,0.000000,0.000000}%
\pgfsetfillcolor{currentfill}%
\pgfsetlinewidth{0.752812pt}%
\definecolor{currentstroke}{rgb}{0.000000,0.000000,0.000000}%
\pgfsetstrokecolor{currentstroke}%
\pgfsetdash{}{0pt}%
\pgfsys@defobject{currentmarker}{\pgfqpoint{-0.013889in}{0.000000in}}{\pgfqpoint{-0.000000in}{0.000000in}}{%
\pgfpathmoveto{\pgfqpoint{-0.000000in}{0.000000in}}%
\pgfpathlineto{\pgfqpoint{-0.013889in}{0.000000in}}%
\pgfusepath{stroke,fill}%
}%
\begin{pgfscope}%
\pgfsys@transformshift{0.868056in}{0.762577in}%
\pgfsys@useobject{currentmarker}{}%
\end{pgfscope}%
\end{pgfscope}%
\begin{pgfscope}%
\definecolor{textcolor}{rgb}{0.000000,0.000000,0.000000}%
\pgfsetstrokecolor{textcolor}%
\pgfsetfillcolor{textcolor}%
\pgftext[x=0.702271in, y=0.734840in, left, base]{\color{textcolor}{\ifdefined\pdftexversion\else\setmainfont{NanumMyeongjo}\rmfamily\fi\fontsize{5.000000}{6.000000}\selectfont\catcode`\^=\active\def^{\ifmmode\sp\else\^{}\fi}\catcode`\%=\active\def%{\%}2만}}%
\end{pgfscope}%
\begin{pgfscope}%
\pgfpathrectangle{\pgfqpoint{0.868056in}{0.555556in}}{\pgfqpoint{3.993056in}{1.888889in}}%
\pgfusepath{clip}%
\pgfsetbuttcap%
\pgfsetroundjoin%
\pgfsetlinewidth{0.602250pt}%
\definecolor{currentstroke}{rgb}{0.690196,0.690196,0.690196}%
\pgfsetstrokecolor{currentstroke}%
\pgfsetstrokeopacity{0.450000}%
\pgfsetdash{{2.220000pt}{0.960000pt}}{0.000000pt}%
\pgfpathmoveto{\pgfqpoint{0.868056in}{0.969598in}}%
\pgfpathlineto{\pgfqpoint{4.861111in}{0.969598in}}%
\pgfusepath{stroke}%
\end{pgfscope}%
\begin{pgfscope}%
\pgfsetbuttcap%
\pgfsetroundjoin%
\definecolor{currentfill}{rgb}{0.000000,0.000000,0.000000}%
\pgfsetfillcolor{currentfill}%
\pgfsetlinewidth{0.752812pt}%
\definecolor{currentstroke}{rgb}{0.000000,0.000000,0.000000}%
\pgfsetstrokecolor{currentstroke}%
\pgfsetdash{}{0pt}%
\pgfsys@defobject{currentmarker}{\pgfqpoint{-0.013889in}{0.000000in}}{\pgfqpoint{-0.000000in}{0.000000in}}{%
\pgfpathmoveto{\pgfqpoint{-0.000000in}{0.000000in}}%
\pgfpathlineto{\pgfqpoint{-0.013889in}{0.000000in}}%
\pgfusepath{stroke,fill}%
}%
\begin{pgfscope}%
\pgfsys@transformshift{0.868056in}{0.969598in}%
\pgfsys@useobject{currentmarker}{}%
\end{pgfscope}%
\end{pgfscope}%
\begin{pgfscope}%
\definecolor{textcolor}{rgb}{0.000000,0.000000,0.000000}%
\pgfsetstrokecolor{textcolor}%
\pgfsetfillcolor{textcolor}%
\pgftext[x=0.702271in, y=0.941861in, left, base]{\color{textcolor}{\ifdefined\pdftexversion\else\setmainfont{NanumMyeongjo}\rmfamily\fi\fontsize{5.000000}{6.000000}\selectfont\catcode`\^=\active\def^{\ifmmode\sp\else\^{}\fi}\catcode`\%=\active\def%{\%}4만}}%
\end{pgfscope}%
\begin{pgfscope}%
\pgfpathrectangle{\pgfqpoint{0.868056in}{0.555556in}}{\pgfqpoint{3.993056in}{1.888889in}}%
\pgfusepath{clip}%
\pgfsetbuttcap%
\pgfsetroundjoin%
\pgfsetlinewidth{0.602250pt}%
\definecolor{currentstroke}{rgb}{0.690196,0.690196,0.690196}%
\pgfsetstrokecolor{currentstroke}%
\pgfsetstrokeopacity{0.450000}%
\pgfsetdash{{2.220000pt}{0.960000pt}}{0.000000pt}%
\pgfpathmoveto{\pgfqpoint{0.868056in}{1.176619in}}%
\pgfpathlineto{\pgfqpoint{4.861111in}{1.176619in}}%
\pgfusepath{stroke}%
\end{pgfscope}%
\begin{pgfscope}%
\pgfsetbuttcap%
\pgfsetroundjoin%
\definecolor{currentfill}{rgb}{0.000000,0.000000,0.000000}%
\pgfsetfillcolor{currentfill}%
\pgfsetlinewidth{0.752812pt}%
\definecolor{currentstroke}{rgb}{0.000000,0.000000,0.000000}%
\pgfsetstrokecolor{currentstroke}%
\pgfsetdash{}{0pt}%
\pgfsys@defobject{currentmarker}{\pgfqpoint{-0.013889in}{0.000000in}}{\pgfqpoint{-0.000000in}{0.000000in}}{%
\pgfpathmoveto{\pgfqpoint{-0.000000in}{0.000000in}}%
\pgfpathlineto{\pgfqpoint{-0.013889in}{0.000000in}}%
\pgfusepath{stroke,fill}%
}%
\begin{pgfscope}%
\pgfsys@transformshift{0.868056in}{1.176619in}%
\pgfsys@useobject{currentmarker}{}%
\end{pgfscope}%
\end{pgfscope}%
\begin{pgfscope}%
\definecolor{textcolor}{rgb}{0.000000,0.000000,0.000000}%
\pgfsetstrokecolor{textcolor}%
\pgfsetfillcolor{textcolor}%
\pgftext[x=0.702271in, y=1.148882in, left, base]{\color{textcolor}{\ifdefined\pdftexversion\else\setmainfont{NanumMyeongjo}\rmfamily\fi\fontsize{5.000000}{6.000000}\selectfont\catcode`\^=\active\def^{\ifmmode\sp\else\^{}\fi}\catcode`\%=\active\def%{\%}6만}}%
\end{pgfscope}%
\begin{pgfscope}%
\pgfpathrectangle{\pgfqpoint{0.868056in}{0.555556in}}{\pgfqpoint{3.993056in}{1.888889in}}%
\pgfusepath{clip}%
\pgfsetbuttcap%
\pgfsetroundjoin%
\pgfsetlinewidth{0.602250pt}%
\definecolor{currentstroke}{rgb}{0.690196,0.690196,0.690196}%
\pgfsetstrokecolor{currentstroke}%
\pgfsetstrokeopacity{0.450000}%
\pgfsetdash{{2.220000pt}{0.960000pt}}{0.000000pt}%
\pgfpathmoveto{\pgfqpoint{0.868056in}{1.383640in}}%
\pgfpathlineto{\pgfqpoint{4.861111in}{1.383640in}}%
\pgfusepath{stroke}%
\end{pgfscope}%
\begin{pgfscope}%
\pgfsetbuttcap%
\pgfsetroundjoin%
\definecolor{currentfill}{rgb}{0.000000,0.000000,0.000000}%
\pgfsetfillcolor{currentfill}%
\pgfsetlinewidth{0.752812pt}%
\definecolor{currentstroke}{rgb}{0.000000,0.000000,0.000000}%
\pgfsetstrokecolor{currentstroke}%
\pgfsetdash{}{0pt}%
\pgfsys@defobject{currentmarker}{\pgfqpoint{-0.013889in}{0.000000in}}{\pgfqpoint{-0.000000in}{0.000000in}}{%
\pgfpathmoveto{\pgfqpoint{-0.000000in}{0.000000in}}%
\pgfpathlineto{\pgfqpoint{-0.013889in}{0.000000in}}%
\pgfusepath{stroke,fill}%
}%
\begin{pgfscope}%
\pgfsys@transformshift{0.868056in}{1.383640in}%
\pgfsys@useobject{currentmarker}{}%
\end{pgfscope}%
\end{pgfscope}%
\begin{pgfscope}%
\definecolor{textcolor}{rgb}{0.000000,0.000000,0.000000}%
\pgfsetstrokecolor{textcolor}%
\pgfsetfillcolor{textcolor}%
\pgftext[x=0.702271in, y=1.355903in, left, base]{\color{textcolor}{\ifdefined\pdftexversion\else\setmainfont{NanumMyeongjo}\rmfamily\fi\fontsize{5.000000}{6.000000}\selectfont\catcode`\^=\active\def^{\ifmmode\sp\else\^{}\fi}\catcode`\%=\active\def%{\%}8만}}%
\end{pgfscope}%
\begin{pgfscope}%
\pgfpathrectangle{\pgfqpoint{0.868056in}{0.555556in}}{\pgfqpoint{3.993056in}{1.888889in}}%
\pgfusepath{clip}%
\pgfsetbuttcap%
\pgfsetroundjoin%
\pgfsetlinewidth{0.602250pt}%
\definecolor{currentstroke}{rgb}{0.690196,0.690196,0.690196}%
\pgfsetstrokecolor{currentstroke}%
\pgfsetstrokeopacity{0.450000}%
\pgfsetdash{{2.220000pt}{0.960000pt}}{0.000000pt}%
\pgfpathmoveto{\pgfqpoint{0.868056in}{1.590662in}}%
\pgfpathlineto{\pgfqpoint{4.861111in}{1.590662in}}%
\pgfusepath{stroke}%
\end{pgfscope}%
\begin{pgfscope}%
\pgfsetbuttcap%
\pgfsetroundjoin%
\definecolor{currentfill}{rgb}{0.000000,0.000000,0.000000}%
\pgfsetfillcolor{currentfill}%
\pgfsetlinewidth{0.752812pt}%
\definecolor{currentstroke}{rgb}{0.000000,0.000000,0.000000}%
\pgfsetstrokecolor{currentstroke}%
\pgfsetdash{}{0pt}%
\pgfsys@defobject{currentmarker}{\pgfqpoint{-0.013889in}{0.000000in}}{\pgfqpoint{-0.000000in}{0.000000in}}{%
\pgfpathmoveto{\pgfqpoint{-0.000000in}{0.000000in}}%
\pgfpathlineto{\pgfqpoint{-0.013889in}{0.000000in}}%
\pgfusepath{stroke,fill}%
}%
\begin{pgfscope}%
\pgfsys@transformshift{0.868056in}{1.590662in}%
\pgfsys@useobject{currentmarker}{}%
\end{pgfscope}%
\end{pgfscope}%
\begin{pgfscope}%
\definecolor{textcolor}{rgb}{0.000000,0.000000,0.000000}%
\pgfsetstrokecolor{textcolor}%
\pgfsetfillcolor{textcolor}%
\pgftext[x=0.664971in, y=1.562924in, left, base]{\color{textcolor}{\ifdefined\pdftexversion\else\setmainfont{NanumMyeongjo}\rmfamily\fi\fontsize{5.000000}{6.000000}\selectfont\catcode`\^=\active\def^{\ifmmode\sp\else\^{}\fi}\catcode`\%=\active\def%{\%}10만}}%
\end{pgfscope}%
\begin{pgfscope}%
\pgfpathrectangle{\pgfqpoint{0.868056in}{0.555556in}}{\pgfqpoint{3.993056in}{1.888889in}}%
\pgfusepath{clip}%
\pgfsetbuttcap%
\pgfsetroundjoin%
\pgfsetlinewidth{0.602250pt}%
\definecolor{currentstroke}{rgb}{0.690196,0.690196,0.690196}%
\pgfsetstrokecolor{currentstroke}%
\pgfsetstrokeopacity{0.450000}%
\pgfsetdash{{2.220000pt}{0.960000pt}}{0.000000pt}%
\pgfpathmoveto{\pgfqpoint{0.868056in}{1.797683in}}%
\pgfpathlineto{\pgfqpoint{4.861111in}{1.797683in}}%
\pgfusepath{stroke}%
\end{pgfscope}%
\begin{pgfscope}%
\pgfsetbuttcap%
\pgfsetroundjoin%
\definecolor{currentfill}{rgb}{0.000000,0.000000,0.000000}%
\pgfsetfillcolor{currentfill}%
\pgfsetlinewidth{0.752812pt}%
\definecolor{currentstroke}{rgb}{0.000000,0.000000,0.000000}%
\pgfsetstrokecolor{currentstroke}%
\pgfsetdash{}{0pt}%
\pgfsys@defobject{currentmarker}{\pgfqpoint{-0.013889in}{0.000000in}}{\pgfqpoint{-0.000000in}{0.000000in}}{%
\pgfpathmoveto{\pgfqpoint{-0.000000in}{0.000000in}}%
\pgfpathlineto{\pgfqpoint{-0.013889in}{0.000000in}}%
\pgfusepath{stroke,fill}%
}%
\begin{pgfscope}%
\pgfsys@transformshift{0.868056in}{1.797683in}%
\pgfsys@useobject{currentmarker}{}%
\end{pgfscope}%
\end{pgfscope}%
\begin{pgfscope}%
\definecolor{textcolor}{rgb}{0.000000,0.000000,0.000000}%
\pgfsetstrokecolor{textcolor}%
\pgfsetfillcolor{textcolor}%
\pgftext[x=0.664971in, y=1.769946in, left, base]{\color{textcolor}{\ifdefined\pdftexversion\else\setmainfont{NanumMyeongjo}\rmfamily\fi\fontsize{5.000000}{6.000000}\selectfont\catcode`\^=\active\def^{\ifmmode\sp\else\^{}\fi}\catcode`\%=\active\def%{\%}12만}}%
\end{pgfscope}%
\begin{pgfscope}%
\pgfpathrectangle{\pgfqpoint{0.868056in}{0.555556in}}{\pgfqpoint{3.993056in}{1.888889in}}%
\pgfusepath{clip}%
\pgfsetbuttcap%
\pgfsetroundjoin%
\pgfsetlinewidth{0.602250pt}%
\definecolor{currentstroke}{rgb}{0.690196,0.690196,0.690196}%
\pgfsetstrokecolor{currentstroke}%
\pgfsetstrokeopacity{0.450000}%
\pgfsetdash{{2.220000pt}{0.960000pt}}{0.000000pt}%
\pgfpathmoveto{\pgfqpoint{0.868056in}{2.004704in}}%
\pgfpathlineto{\pgfqpoint{4.861111in}{2.004704in}}%
\pgfusepath{stroke}%
\end{pgfscope}%
\begin{pgfscope}%
\pgfsetbuttcap%
\pgfsetroundjoin%
\definecolor{currentfill}{rgb}{0.000000,0.000000,0.000000}%
\pgfsetfillcolor{currentfill}%
\pgfsetlinewidth{0.752812pt}%
\definecolor{currentstroke}{rgb}{0.000000,0.000000,0.000000}%
\pgfsetstrokecolor{currentstroke}%
\pgfsetdash{}{0pt}%
\pgfsys@defobject{currentmarker}{\pgfqpoint{-0.013889in}{0.000000in}}{\pgfqpoint{-0.000000in}{0.000000in}}{%
\pgfpathmoveto{\pgfqpoint{-0.000000in}{0.000000in}}%
\pgfpathlineto{\pgfqpoint{-0.013889in}{0.000000in}}%
\pgfusepath{stroke,fill}%
}%
\begin{pgfscope}%
\pgfsys@transformshift{0.868056in}{2.004704in}%
\pgfsys@useobject{currentmarker}{}%
\end{pgfscope}%
\end{pgfscope}%
\begin{pgfscope}%
\definecolor{textcolor}{rgb}{0.000000,0.000000,0.000000}%
\pgfsetstrokecolor{textcolor}%
\pgfsetfillcolor{textcolor}%
\pgftext[x=0.664971in, y=1.976967in, left, base]{\color{textcolor}{\ifdefined\pdftexversion\else\setmainfont{NanumMyeongjo}\rmfamily\fi\fontsize{5.000000}{6.000000}\selectfont\catcode`\^=\active\def^{\ifmmode\sp\else\^{}\fi}\catcode`\%=\active\def%{\%}14만}}%
\end{pgfscope}%
\begin{pgfscope}%
\pgfpathrectangle{\pgfqpoint{0.868056in}{0.555556in}}{\pgfqpoint{3.993056in}{1.888889in}}%
\pgfusepath{clip}%
\pgfsetbuttcap%
\pgfsetroundjoin%
\pgfsetlinewidth{0.602250pt}%
\definecolor{currentstroke}{rgb}{0.690196,0.690196,0.690196}%
\pgfsetstrokecolor{currentstroke}%
\pgfsetstrokeopacity{0.450000}%
\pgfsetdash{{2.220000pt}{0.960000pt}}{0.000000pt}%
\pgfpathmoveto{\pgfqpoint{0.868056in}{2.211725in}}%
\pgfpathlineto{\pgfqpoint{4.861111in}{2.211725in}}%
\pgfusepath{stroke}%
\end{pgfscope}%
\begin{pgfscope}%
\pgfsetbuttcap%
\pgfsetroundjoin%
\definecolor{currentfill}{rgb}{0.000000,0.000000,0.000000}%
\pgfsetfillcolor{currentfill}%
\pgfsetlinewidth{0.752812pt}%
\definecolor{currentstroke}{rgb}{0.000000,0.000000,0.000000}%
\pgfsetstrokecolor{currentstroke}%
\pgfsetdash{}{0pt}%
\pgfsys@defobject{currentmarker}{\pgfqpoint{-0.013889in}{0.000000in}}{\pgfqpoint{-0.000000in}{0.000000in}}{%
\pgfpathmoveto{\pgfqpoint{-0.000000in}{0.000000in}}%
\pgfpathlineto{\pgfqpoint{-0.013889in}{0.000000in}}%
\pgfusepath{stroke,fill}%
}%
\begin{pgfscope}%
\pgfsys@transformshift{0.868056in}{2.211725in}%
\pgfsys@useobject{currentmarker}{}%
\end{pgfscope}%
\end{pgfscope}%
\begin{pgfscope}%
\definecolor{textcolor}{rgb}{0.000000,0.000000,0.000000}%
\pgfsetstrokecolor{textcolor}%
\pgfsetfillcolor{textcolor}%
\pgftext[x=0.664971in, y=2.183988in, left, base]{\color{textcolor}{\ifdefined\pdftexversion\else\setmainfont{NanumMyeongjo}\rmfamily\fi\fontsize{5.000000}{6.000000}\selectfont\catcode`\^=\active\def^{\ifmmode\sp\else\^{}\fi}\catcode`\%=\active\def%{\%}16만}}%
\end{pgfscope}%
\begin{pgfscope}%
\pgfpathrectangle{\pgfqpoint{0.868056in}{0.555556in}}{\pgfqpoint{3.993056in}{1.888889in}}%
\pgfusepath{clip}%
\pgfsetbuttcap%
\pgfsetroundjoin%
\pgfsetlinewidth{0.602250pt}%
\definecolor{currentstroke}{rgb}{0.690196,0.690196,0.690196}%
\pgfsetstrokecolor{currentstroke}%
\pgfsetstrokeopacity{0.450000}%
\pgfsetdash{{2.220000pt}{0.960000pt}}{0.000000pt}%
\pgfpathmoveto{\pgfqpoint{0.868056in}{2.418746in}}%
\pgfpathlineto{\pgfqpoint{4.861111in}{2.418746in}}%
\pgfusepath{stroke}%
\end{pgfscope}%
\begin{pgfscope}%
\pgfsetbuttcap%
\pgfsetroundjoin%
\definecolor{currentfill}{rgb}{0.000000,0.000000,0.000000}%
\pgfsetfillcolor{currentfill}%
\pgfsetlinewidth{0.752812pt}%
\definecolor{currentstroke}{rgb}{0.000000,0.000000,0.000000}%
\pgfsetstrokecolor{currentstroke}%
\pgfsetdash{}{0pt}%
\pgfsys@defobject{currentmarker}{\pgfqpoint{-0.013889in}{0.000000in}}{\pgfqpoint{-0.000000in}{0.000000in}}{%
\pgfpathmoveto{\pgfqpoint{-0.000000in}{0.000000in}}%
\pgfpathlineto{\pgfqpoint{-0.013889in}{0.000000in}}%
\pgfusepath{stroke,fill}%
}%
\begin{pgfscope}%
\pgfsys@transformshift{0.868056in}{2.418746in}%
\pgfsys@useobject{currentmarker}{}%
\end{pgfscope}%
\end{pgfscope}%
\begin{pgfscope}%
\definecolor{textcolor}{rgb}{0.000000,0.000000,0.000000}%
\pgfsetstrokecolor{textcolor}%
\pgfsetfillcolor{textcolor}%
\pgftext[x=0.664971in, y=2.391009in, left, base]{\color{textcolor}{\ifdefined\pdftexversion\else\setmainfont{NanumMyeongjo}\rmfamily\fi\fontsize{5.000000}{6.000000}\selectfont\catcode`\^=\active\def^{\ifmmode\sp\else\^{}\fi}\catcode`\%=\active\def%{\%}18만}}%
\end{pgfscope}%
\begin{pgfscope}%
\pgfsetrectcap%
\pgfsetmiterjoin%
\pgfsetlinewidth{0.752812pt}%
\definecolor{currentstroke}{rgb}{0.000000,0.000000,0.000000}%
\pgfsetstrokecolor{currentstroke}%
\pgfsetdash{}{0pt}%
\pgfpathmoveto{\pgfqpoint{0.868056in}{0.555556in}}%
\pgfpathlineto{\pgfqpoint{0.868056in}{2.444444in}}%
\pgfusepath{stroke}%
\end{pgfscope}%
\begin{pgfscope}%
\pgfsetrectcap%
\pgfsetmiterjoin%
\pgfsetlinewidth{0.752812pt}%
\definecolor{currentstroke}{rgb}{0.000000,0.000000,0.000000}%
\pgfsetstrokecolor{currentstroke}%
\pgfsetdash{}{0pt}%
\pgfpathmoveto{\pgfqpoint{0.868056in}{0.555556in}}%
\pgfpathlineto{\pgfqpoint{4.861111in}{0.555556in}}%
\pgfusepath{stroke}%
\end{pgfscope}%
\begin{pgfscope}%
\pgfpathrectangle{\pgfqpoint{0.868056in}{0.555556in}}{\pgfqpoint{3.993056in}{1.888889in}}%
\pgfusepath{clip}%
\pgfsetbuttcap%
\pgfsetmiterjoin%
\definecolor{currentfill}{rgb}{0.227451,0.192157,0.427451}%
\pgfsetfillcolor{currentfill}%
\pgfsetlinewidth{1.003750pt}%
\definecolor{currentstroke}{rgb}{0.266667,0.266667,0.266667}%
\pgfsetstrokecolor{currentstroke}%
\pgfsetdash{}{0pt}%
\pgfpathmoveto{\pgfqpoint{1.049558in}{0.555556in}}%
\pgfpathlineto{\pgfqpoint{1.287038in}{0.555556in}}%
\pgfpathlineto{\pgfqpoint{1.287038in}{2.197058in}}%
\pgfpathlineto{\pgfqpoint{1.049558in}{2.197058in}}%
\pgfpathlineto{\pgfqpoint{1.049558in}{0.555556in}}%
\pgfpathclose%
\pgfusepath{stroke,fill}%
\end{pgfscope}%
\begin{pgfscope}%
\pgfpathrectangle{\pgfqpoint{0.868056in}{0.555556in}}{\pgfqpoint{3.993056in}{1.888889in}}%
\pgfusepath{clip}%
\pgfsetbuttcap%
\pgfsetmiterjoin%
\definecolor{currentfill}{rgb}{0.227451,0.192157,0.427451}%
\pgfsetfillcolor{currentfill}%
\pgfsetlinewidth{1.003750pt}%
\definecolor{currentstroke}{rgb}{0.266667,0.266667,0.266667}%
\pgfsetstrokecolor{currentstroke}%
\pgfsetdash{}{0pt}%
\pgfpathmoveto{\pgfqpoint{1.388815in}{0.555556in}}%
\pgfpathlineto{\pgfqpoint{1.626295in}{0.555556in}}%
\pgfpathlineto{\pgfqpoint{1.626295in}{1.788160in}}%
\pgfpathlineto{\pgfqpoint{1.388815in}{1.788160in}}%
\pgfpathlineto{\pgfqpoint{1.388815in}{0.555556in}}%
\pgfpathclose%
\pgfusepath{stroke,fill}%
\end{pgfscope}%
\begin{pgfscope}%
\pgfpathrectangle{\pgfqpoint{0.868056in}{0.555556in}}{\pgfqpoint{3.993056in}{1.888889in}}%
\pgfusepath{clip}%
\pgfsetbuttcap%
\pgfsetmiterjoin%
\definecolor{currentfill}{rgb}{0.227451,0.192157,0.427451}%
\pgfsetfillcolor{currentfill}%
\pgfsetlinewidth{1.003750pt}%
\definecolor{currentstroke}{rgb}{0.266667,0.266667,0.266667}%
\pgfsetstrokecolor{currentstroke}%
\pgfsetdash{}{0pt}%
\pgfpathmoveto{\pgfqpoint{1.728072in}{0.555556in}}%
\pgfpathlineto{\pgfqpoint{1.965552in}{0.555556in}}%
\pgfpathlineto{\pgfqpoint{1.965552in}{1.494593in}}%
\pgfpathlineto{\pgfqpoint{1.728072in}{1.494593in}}%
\pgfpathlineto{\pgfqpoint{1.728072in}{0.555556in}}%
\pgfpathclose%
\pgfusepath{stroke,fill}%
\end{pgfscope}%
\begin{pgfscope}%
\pgfpathrectangle{\pgfqpoint{0.868056in}{0.555556in}}{\pgfqpoint{3.993056in}{1.888889in}}%
\pgfusepath{clip}%
\pgfsetbuttcap%
\pgfsetmiterjoin%
\definecolor{currentfill}{rgb}{0.227451,0.192157,0.427451}%
\pgfsetfillcolor{currentfill}%
\pgfsetlinewidth{1.003750pt}%
\definecolor{currentstroke}{rgb}{0.266667,0.266667,0.266667}%
\pgfsetstrokecolor{currentstroke}%
\pgfsetdash{}{0pt}%
\pgfpathmoveto{\pgfqpoint{2.067329in}{0.555556in}}%
\pgfpathlineto{\pgfqpoint{2.304809in}{0.555556in}}%
\pgfpathlineto{\pgfqpoint{2.304809in}{1.602731in}}%
\pgfpathlineto{\pgfqpoint{2.067329in}{1.602731in}}%
\pgfpathlineto{\pgfqpoint{2.067329in}{0.555556in}}%
\pgfpathclose%
\pgfusepath{stroke,fill}%
\end{pgfscope}%
\begin{pgfscope}%
\pgfpathrectangle{\pgfqpoint{0.868056in}{0.555556in}}{\pgfqpoint{3.993056in}{1.888889in}}%
\pgfusepath{clip}%
\pgfsetbuttcap%
\pgfsetmiterjoin%
\definecolor{currentfill}{rgb}{0.227451,0.192157,0.427451}%
\pgfsetfillcolor{currentfill}%
\pgfsetlinewidth{1.003750pt}%
\definecolor{currentstroke}{rgb}{0.266667,0.266667,0.266667}%
\pgfsetstrokecolor{currentstroke}%
\pgfsetdash{}{0pt}%
\pgfpathmoveto{\pgfqpoint{2.406586in}{0.555556in}}%
\pgfpathlineto{\pgfqpoint{2.644066in}{0.555556in}}%
\pgfpathlineto{\pgfqpoint{2.644066in}{1.653617in}}%
\pgfpathlineto{\pgfqpoint{2.406586in}{1.653617in}}%
\pgfpathlineto{\pgfqpoint{2.406586in}{0.555556in}}%
\pgfpathclose%
\pgfusepath{stroke,fill}%
\end{pgfscope}%
\begin{pgfscope}%
\pgfpathrectangle{\pgfqpoint{0.868056in}{0.555556in}}{\pgfqpoint{3.993056in}{1.888889in}}%
\pgfusepath{clip}%
\pgfsetbuttcap%
\pgfsetmiterjoin%
\definecolor{currentfill}{rgb}{0.227451,0.192157,0.427451}%
\pgfsetfillcolor{currentfill}%
\pgfsetlinewidth{1.003750pt}%
\definecolor{currentstroke}{rgb}{0.266667,0.266667,0.266667}%
\pgfsetstrokecolor{currentstroke}%
\pgfsetdash{}{0pt}%
\pgfpathmoveto{\pgfqpoint{2.745843in}{0.555556in}}%
\pgfpathlineto{\pgfqpoint{2.983323in}{0.555556in}}%
\pgfpathlineto{\pgfqpoint{2.983323in}{1.851001in}}%
\pgfpathlineto{\pgfqpoint{2.745843in}{1.851001in}}%
\pgfpathlineto{\pgfqpoint{2.745843in}{0.555556in}}%
\pgfpathclose%
\pgfusepath{stroke,fill}%
\end{pgfscope}%
\begin{pgfscope}%
\pgfpathrectangle{\pgfqpoint{0.868056in}{0.555556in}}{\pgfqpoint{3.993056in}{1.888889in}}%
\pgfusepath{clip}%
\pgfsetbuttcap%
\pgfsetmiterjoin%
\definecolor{currentfill}{rgb}{0.227451,0.192157,0.427451}%
\pgfsetfillcolor{currentfill}%
\pgfsetlinewidth{1.003750pt}%
\definecolor{currentstroke}{rgb}{0.266667,0.266667,0.266667}%
\pgfsetstrokecolor{currentstroke}%
\pgfsetdash{}{0pt}%
\pgfpathmoveto{\pgfqpoint{3.085100in}{0.555556in}}%
\pgfpathlineto{\pgfqpoint{3.322580in}{0.555556in}}%
\pgfpathlineto{\pgfqpoint{3.322580in}{1.576967in}}%
\pgfpathlineto{\pgfqpoint{3.085100in}{1.576967in}}%
\pgfpathlineto{\pgfqpoint{3.085100in}{0.555556in}}%
\pgfpathclose%
\pgfusepath{stroke,fill}%
\end{pgfscope}%
\begin{pgfscope}%
\pgfpathrectangle{\pgfqpoint{0.868056in}{0.555556in}}{\pgfqpoint{3.993056in}{1.888889in}}%
\pgfusepath{clip}%
\pgfsetbuttcap%
\pgfsetmiterjoin%
\definecolor{currentfill}{rgb}{0.227451,0.192157,0.427451}%
\pgfsetfillcolor{currentfill}%
\pgfsetlinewidth{1.003750pt}%
\definecolor{currentstroke}{rgb}{0.266667,0.266667,0.266667}%
\pgfsetstrokecolor{currentstroke}%
\pgfsetdash{}{0pt}%
\pgfpathmoveto{\pgfqpoint{3.424357in}{0.555556in}}%
\pgfpathlineto{\pgfqpoint{3.661837in}{0.555556in}}%
\pgfpathlineto{\pgfqpoint{3.661837in}{1.881185in}}%
\pgfpathlineto{\pgfqpoint{3.424357in}{1.881185in}}%
\pgfpathlineto{\pgfqpoint{3.424357in}{0.555556in}}%
\pgfpathclose%
\pgfusepath{stroke,fill}%
\end{pgfscope}%
\begin{pgfscope}%
\pgfpathrectangle{\pgfqpoint{0.868056in}{0.555556in}}{\pgfqpoint{3.993056in}{1.888889in}}%
\pgfusepath{clip}%
\pgfsetbuttcap%
\pgfsetmiterjoin%
\definecolor{currentfill}{rgb}{0.227451,0.192157,0.427451}%
\pgfsetfillcolor{currentfill}%
\pgfsetlinewidth{1.003750pt}%
\definecolor{currentstroke}{rgb}{0.266667,0.266667,0.266667}%
\pgfsetstrokecolor{currentstroke}%
\pgfsetdash{}{0pt}%
\pgfpathmoveto{\pgfqpoint{3.763615in}{0.555556in}}%
\pgfpathlineto{\pgfqpoint{4.001094in}{0.555556in}}%
\pgfpathlineto{\pgfqpoint{4.001094in}{2.104602in}}%
\pgfpathlineto{\pgfqpoint{3.763615in}{2.104602in}}%
\pgfpathlineto{\pgfqpoint{3.763615in}{0.555556in}}%
\pgfpathclose%
\pgfusepath{stroke,fill}%
\end{pgfscope}%
\begin{pgfscope}%
\pgfpathrectangle{\pgfqpoint{0.868056in}{0.555556in}}{\pgfqpoint{3.993056in}{1.888889in}}%
\pgfusepath{clip}%
\pgfsetbuttcap%
\pgfsetmiterjoin%
\definecolor{currentfill}{rgb}{0.227451,0.192157,0.427451}%
\pgfsetfillcolor{currentfill}%
\pgfsetlinewidth{1.003750pt}%
\definecolor{currentstroke}{rgb}{0.266667,0.266667,0.266667}%
\pgfsetstrokecolor{currentstroke}%
\pgfsetdash{}{0pt}%
\pgfpathmoveto{\pgfqpoint{4.102872in}{0.555556in}}%
\pgfpathlineto{\pgfqpoint{4.340352in}{0.555556in}}%
\pgfpathlineto{\pgfqpoint{4.340352in}{2.185195in}}%
\pgfpathlineto{\pgfqpoint{4.102872in}{2.185195in}}%
\pgfpathlineto{\pgfqpoint{4.102872in}{0.555556in}}%
\pgfpathclose%
\pgfusepath{stroke,fill}%
\end{pgfscope}%
\begin{pgfscope}%
\pgfpathrectangle{\pgfqpoint{0.868056in}{0.555556in}}{\pgfqpoint{3.993056in}{1.888889in}}%
\pgfusepath{clip}%
\pgfsetbuttcap%
\pgfsetmiterjoin%
\definecolor{currentfill}{rgb}{0.227451,0.192157,0.427451}%
\pgfsetfillcolor{currentfill}%
\pgfsetlinewidth{1.003750pt}%
\definecolor{currentstroke}{rgb}{0.266667,0.266667,0.266667}%
\pgfsetstrokecolor{currentstroke}%
\pgfsetdash{}{0pt}%
\pgfpathmoveto{\pgfqpoint{4.442129in}{0.555556in}}%
\pgfpathlineto{\pgfqpoint{4.679609in}{0.555556in}}%
\pgfpathlineto{\pgfqpoint{4.679609in}{2.354497in}}%
\pgfpathlineto{\pgfqpoint{4.442129in}{2.354497in}}%
\pgfpathlineto{\pgfqpoint{4.442129in}{0.555556in}}%
\pgfpathclose%
\pgfusepath{stroke,fill}%
\end{pgfscope}%
\begin{pgfscope}%
\definecolor{textcolor}{rgb}{0.000000,0.000000,0.000000}%
\pgfsetstrokecolor{textcolor}%
\pgfsetfillcolor{textcolor}%
\pgftext[x=1.168298in,y=2.224836in,,bottom]{\color{textcolor}{\ifdefined\pdftexversion\else\setmainfont{NanumMyeongjo}\rmfamily\fi\fontsize{5.000000}{6.000000}\bfseries\selectfont\catcode`\^=\active\def^{\ifmmode\sp\else\^{}\fi}\catcode`\%=\active\def%{\%}158,583}}%
\end{pgfscope}%
\begin{pgfscope}%
\definecolor{textcolor}{rgb}{0.000000,0.000000,0.000000}%
\pgfsetstrokecolor{textcolor}%
\pgfsetfillcolor{textcolor}%
\pgftext[x=1.507555in,y=1.815938in,,bottom]{\color{textcolor}{\ifdefined\pdftexversion\else\setmainfont{NanumMyeongjo}\rmfamily\fi\fontsize{5.000000}{6.000000}\bfseries\selectfont\catcode`\^=\active\def^{\ifmmode\sp\else\^{}\fi}\catcode`\%=\active\def%{\%}119,080}}%
\end{pgfscope}%
\begin{pgfscope}%
\definecolor{textcolor}{rgb}{0.000000,0.000000,0.000000}%
\pgfsetstrokecolor{textcolor}%
\pgfsetfillcolor{textcolor}%
\pgftext[x=1.846812in,y=1.522371in,,bottom]{\color{textcolor}{\ifdefined\pdftexversion\else\setmainfont{NanumMyeongjo}\rmfamily\fi\fontsize{5.000000}{6.000000}\bfseries\selectfont\catcode`\^=\active\def^{\ifmmode\sp\else\^{}\fi}\catcode`\%=\active\def%{\%}90,719}}%
\end{pgfscope}%
\begin{pgfscope}%
\definecolor{textcolor}{rgb}{0.000000,0.000000,0.000000}%
\pgfsetstrokecolor{textcolor}%
\pgfsetfillcolor{textcolor}%
\pgftext[x=2.186069in,y=1.630509in,,bottom]{\color{textcolor}{\ifdefined\pdftexversion\else\setmainfont{NanumMyeongjo}\rmfamily\fi\fontsize{5.000000}{6.000000}\bfseries\selectfont\catcode`\^=\active\def^{\ifmmode\sp\else\^{}\fi}\catcode`\%=\active\def%{\%}101,166}}%
\end{pgfscope}%
\begin{pgfscope}%
\definecolor{textcolor}{rgb}{0.000000,0.000000,0.000000}%
\pgfsetstrokecolor{textcolor}%
\pgfsetfillcolor{textcolor}%
\pgftext[x=2.525326in,y=1.681394in,,bottom]{\color{textcolor}{\ifdefined\pdftexversion\else\setmainfont{NanumMyeongjo}\rmfamily\fi\fontsize{5.000000}{6.000000}\bfseries\selectfont\catcode`\^=\active\def^{\ifmmode\sp\else\^{}\fi}\catcode`\%=\active\def%{\%}106,082}}%
\end{pgfscope}%
\begin{pgfscope}%
\definecolor{textcolor}{rgb}{0.000000,0.000000,0.000000}%
\pgfsetstrokecolor{textcolor}%
\pgfsetfillcolor{textcolor}%
\pgftext[x=2.864583in,y=1.878779in,,bottom]{\color{textcolor}{\ifdefined\pdftexversion\else\setmainfont{NanumMyeongjo}\rmfamily\fi\fontsize{5.000000}{6.000000}\bfseries\selectfont\catcode`\^=\active\def^{\ifmmode\sp\else\^{}\fi}\catcode`\%=\active\def%{\%}125,151}}%
\end{pgfscope}%
\begin{pgfscope}%
\definecolor{textcolor}{rgb}{0.000000,0.000000,0.000000}%
\pgfsetstrokecolor{textcolor}%
\pgfsetfillcolor{textcolor}%
\pgftext[x=3.203840in,y=1.604745in,,bottom]{\color{textcolor}{\ifdefined\pdftexversion\else\setmainfont{NanumMyeongjo}\rmfamily\fi\fontsize{5.000000}{6.000000}\bfseries\selectfont\catcode`\^=\active\def^{\ifmmode\sp\else\^{}\fi}\catcode`\%=\active\def%{\%}98,677}}%
\end{pgfscope}%
\begin{pgfscope}%
\definecolor{textcolor}{rgb}{0.000000,0.000000,0.000000}%
\pgfsetstrokecolor{textcolor}%
\pgfsetfillcolor{textcolor}%
\pgftext[x=3.543097in,y=1.908963in,,bottom]{\color{textcolor}{\ifdefined\pdftexversion\else\setmainfont{NanumMyeongjo}\rmfamily\fi\fontsize{5.000000}{6.000000}\bfseries\selectfont\catcode`\^=\active\def^{\ifmmode\sp\else\^{}\fi}\catcode`\%=\active\def%{\%}128,067}}%
\end{pgfscope}%
\begin{pgfscope}%
\definecolor{textcolor}{rgb}{0.000000,0.000000,0.000000}%
\pgfsetstrokecolor{textcolor}%
\pgfsetfillcolor{textcolor}%
\pgftext[x=3.882355in,y=2.132380in,,bottom]{\color{textcolor}{\ifdefined\pdftexversion\else\setmainfont{NanumMyeongjo}\rmfamily\fi\fontsize{5.000000}{6.000000}\bfseries\selectfont\catcode`\^=\active\def^{\ifmmode\sp\else\^{}\fi}\catcode`\%=\active\def%{\%}149,651}}%
\end{pgfscope}%
\begin{pgfscope}%
\definecolor{textcolor}{rgb}{0.000000,0.000000,0.000000}%
\pgfsetstrokecolor{textcolor}%
\pgfsetfillcolor{textcolor}%
\pgftext[x=4.221612in,y=2.212973in,,bottom]{\color{textcolor}{\ifdefined\pdftexversion\else\setmainfont{NanumMyeongjo}\rmfamily\fi\fontsize{5.000000}{6.000000}\bfseries\selectfont\catcode`\^=\active\def^{\ifmmode\sp\else\^{}\fi}\catcode`\%=\active\def%{\%}157,437}}%
\end{pgfscope}%
\begin{pgfscope}%
\definecolor{textcolor}{rgb}{0.000000,0.000000,0.000000}%
\pgfsetstrokecolor{textcolor}%
\pgfsetfillcolor{textcolor}%
\pgftext[x=4.560869in,y=2.382275in,,bottom]{\color{textcolor}{\ifdefined\pdftexversion\else\setmainfont{NanumMyeongjo}\rmfamily\fi\fontsize{5.000000}{6.000000}\bfseries\selectfont\catcode`\^=\active\def^{\ifmmode\sp\else\^{}\fi}\catcode`\%=\active\def%{\%}173,793}}%
\end{pgfscope}%
\begin{pgfscope}%
\definecolor{textcolor}{rgb}{0.333333,0.333333,0.333333}%
\pgfsetstrokecolor{textcolor}%
\pgfsetfillcolor{textcolor}%
\pgftext[x=1.736111in,y=0.333333in,,top]{\color{textcolor}{\ifdefined\pdftexversion\else\setmainfont{NanumMyeongjo}\rmfamily\fi\fontsize{5.000000}{6.000000}\selectfont\catcode`\^=\active\def^{\ifmmode\sp\else\^{}\fi}\catcode`\%=\active\def%{\%}출처: 국가농식품통계서비스(KASS) 자료 기반 저자 작성}}%
\end{pgfscope}%
\begin{pgfscope}%
\definecolor{textcolor}{rgb}{0.333333,0.333333,0.333333}%
\pgfsetstrokecolor{textcolor}%
\pgfsetfillcolor{textcolor}%
\pgftext[x=4.513889in,y=2.583333in,,top]{\color{textcolor}{\ifdefined\pdftexversion\else\setmainfont{NanumMyeongjo}\rmfamily\fi\fontsize{5.000000}{6.000000}\selectfont\catcode`\^=\active\def^{\ifmmode\sp\else\^{}\fi}\catcode`\%=\active\def%{\%}(단위: 톤)}}%
\end{pgfscope}%
\end{pgfpicture}%
\makeatother%
\endgroup%
}
\end{center}
}



\slide
{\maintitle}
{\chapterthree}
{국내 콩 생산량}{
\vspace{-10pt}
\begin{center}
    \hspace*{-40pt}{%% Creator: Matplotlib, PGF backend
%%
%% To include the figure in your LaTeX document, write
%%   \input{<filename>.pgf}
%%
%% Make sure the required packages are loaded in your preamble
%%   \usepackage{pgf}
%%
%% Also ensure that all the required font packages are loaded; for instance,
%% the lmodern package is sometimes necessary when using math font.
%%   \usepackage{lmodern}
%%
%% Figures using additional raster images can only be included by \input if
%% they are in the same directory as the main LaTeX file. For loading figures
%% from other directories you can use the `import` package
%%   \usepackage{import}
%%
%% and then include the figures with
%%   \import{<path to file>}{<filename>.pgf}
%%
%% Matplotlib used the following preamble
%%   \def\mathdefault#1{#1}
%%   \everymath=\expandafter{\the\everymath\displaystyle}
%%   \IfFileExists{scrextend.sty}{
%%     \usepackage[fontsize=5.000000pt]{scrextend}
%%   }{
%%     \renewcommand{\normalsize}{\fontsize{5.000000}{6.000000}\selectfont}
%%     \normalsize
%%   }
%%   
%%   \ifdefined\pdftexversion\else  % non-pdftex case.
%%     \usepackage{fontspec}
%%     \setmainfont{DejaVuSerif.ttf}[Path=\detokenize{/home/user/.cache/pypoetry/virtualenvs/graph-KASAOWVd-py3.12/lib/python3.12/site-packages/matplotlib/mpl-data/fonts/ttf/}]
%%     \setsansfont{DejaVuSans.ttf}[Path=\detokenize{/home/user/.cache/pypoetry/virtualenvs/graph-KASAOWVd-py3.12/lib/python3.12/site-packages/matplotlib/mpl-data/fonts/ttf/}]
%%     \setmonofont{DejaVuSansMono.ttf}[Path=\detokenize{/home/user/.cache/pypoetry/virtualenvs/graph-KASAOWVd-py3.12/lib/python3.12/site-packages/matplotlib/mpl-data/fonts/ttf/}]
%%   \fi
%%   \makeatletter\@ifpackageloaded{underscore}{}{\usepackage[strings]{underscore}}\makeatother
%%
\begingroup%
\makeatletter%
\begin{pgfpicture}%
\pgfpathrectangle{\pgfpointorigin}{\pgfqpoint{6.944444in}{2.777778in}}%
\pgfusepath{use as bounding box, clip}%
\begin{pgfscope}%
\pgfsetbuttcap%
\pgfsetmiterjoin%
\definecolor{currentfill}{rgb}{1.000000,1.000000,1.000000}%
\pgfsetfillcolor{currentfill}%
\pgfsetlinewidth{0.000000pt}%
\definecolor{currentstroke}{rgb}{1.000000,1.000000,1.000000}%
\pgfsetstrokecolor{currentstroke}%
\pgfsetdash{}{0pt}%
\pgfpathmoveto{\pgfqpoint{0.000000in}{0.000000in}}%
\pgfpathlineto{\pgfqpoint{6.944444in}{0.000000in}}%
\pgfpathlineto{\pgfqpoint{6.944444in}{2.777778in}}%
\pgfpathlineto{\pgfqpoint{0.000000in}{2.777778in}}%
\pgfpathlineto{\pgfqpoint{0.000000in}{0.000000in}}%
\pgfpathclose%
\pgfusepath{fill}%
\end{pgfscope}%
\begin{pgfscope}%
\pgfsetbuttcap%
\pgfsetmiterjoin%
\definecolor{currentfill}{rgb}{1.000000,1.000000,1.000000}%
\pgfsetfillcolor{currentfill}%
\pgfsetlinewidth{0.000000pt}%
\definecolor{currentstroke}{rgb}{0.000000,0.000000,0.000000}%
\pgfsetstrokecolor{currentstroke}%
\pgfsetstrokeopacity{0.000000}%
\pgfsetdash{}{0pt}%
\pgfpathmoveto{\pgfqpoint{0.868056in}{0.555556in}}%
\pgfpathlineto{\pgfqpoint{4.861111in}{0.555556in}}%
\pgfpathlineto{\pgfqpoint{4.861111in}{2.444444in}}%
\pgfpathlineto{\pgfqpoint{0.868056in}{2.444444in}}%
\pgfpathlineto{\pgfqpoint{0.868056in}{0.555556in}}%
\pgfpathclose%
\pgfusepath{fill}%
\end{pgfscope}%
\begin{pgfscope}%
\pgfsetbuttcap%
\pgfsetroundjoin%
\definecolor{currentfill}{rgb}{0.000000,0.000000,0.000000}%
\pgfsetfillcolor{currentfill}%
\pgfsetlinewidth{0.752812pt}%
\definecolor{currentstroke}{rgb}{0.000000,0.000000,0.000000}%
\pgfsetstrokecolor{currentstroke}%
\pgfsetdash{}{0pt}%
\pgfsys@defobject{currentmarker}{\pgfqpoint{0.000000in}{-0.013889in}}{\pgfqpoint{0.000000in}{0.000000in}}{%
\pgfpathmoveto{\pgfqpoint{0.000000in}{0.000000in}}%
\pgfpathlineto{\pgfqpoint{0.000000in}{-0.013889in}}%
\pgfusepath{stroke,fill}%
}%
\begin{pgfscope}%
\pgfsys@transformshift{1.168298in}{0.555556in}%
\pgfsys@useobject{currentmarker}{}%
\end{pgfscope}%
\end{pgfscope}%
\begin{pgfscope}%
\definecolor{textcolor}{rgb}{0.000000,0.000000,0.000000}%
\pgfsetstrokecolor{textcolor}%
\pgfsetfillcolor{textcolor}%
\pgftext[x=1.168298in,y=0.493056in,,top]{\color{textcolor}{\ifdefined\pdftexversion\else\setmainfont{NanumMyeongjo}\rmfamily\fi\fontsize{5.000000}{6.000000}\selectfont\catcode`\^=\active\def^{\ifmmode\sp\else\^{}\fi}\catcode`\%=\active\def%{\%}2014}}%
\end{pgfscope}%
\begin{pgfscope}%
\pgfsetbuttcap%
\pgfsetroundjoin%
\definecolor{currentfill}{rgb}{0.000000,0.000000,0.000000}%
\pgfsetfillcolor{currentfill}%
\pgfsetlinewidth{0.752812pt}%
\definecolor{currentstroke}{rgb}{0.000000,0.000000,0.000000}%
\pgfsetstrokecolor{currentstroke}%
\pgfsetdash{}{0pt}%
\pgfsys@defobject{currentmarker}{\pgfqpoint{0.000000in}{-0.013889in}}{\pgfqpoint{0.000000in}{0.000000in}}{%
\pgfpathmoveto{\pgfqpoint{0.000000in}{0.000000in}}%
\pgfpathlineto{\pgfqpoint{0.000000in}{-0.013889in}}%
\pgfusepath{stroke,fill}%
}%
\begin{pgfscope}%
\pgfsys@transformshift{1.507555in}{0.555556in}%
\pgfsys@useobject{currentmarker}{}%
\end{pgfscope}%
\end{pgfscope}%
\begin{pgfscope}%
\definecolor{textcolor}{rgb}{0.000000,0.000000,0.000000}%
\pgfsetstrokecolor{textcolor}%
\pgfsetfillcolor{textcolor}%
\pgftext[x=1.507555in,y=0.493056in,,top]{\color{textcolor}{\ifdefined\pdftexversion\else\setmainfont{NanumMyeongjo}\rmfamily\fi\fontsize{5.000000}{6.000000}\selectfont\catcode`\^=\active\def^{\ifmmode\sp\else\^{}\fi}\catcode`\%=\active\def%{\%}2015}}%
\end{pgfscope}%
\begin{pgfscope}%
\pgfsetbuttcap%
\pgfsetroundjoin%
\definecolor{currentfill}{rgb}{0.000000,0.000000,0.000000}%
\pgfsetfillcolor{currentfill}%
\pgfsetlinewidth{0.752812pt}%
\definecolor{currentstroke}{rgb}{0.000000,0.000000,0.000000}%
\pgfsetstrokecolor{currentstroke}%
\pgfsetdash{}{0pt}%
\pgfsys@defobject{currentmarker}{\pgfqpoint{0.000000in}{-0.013889in}}{\pgfqpoint{0.000000in}{0.000000in}}{%
\pgfpathmoveto{\pgfqpoint{0.000000in}{0.000000in}}%
\pgfpathlineto{\pgfqpoint{0.000000in}{-0.013889in}}%
\pgfusepath{stroke,fill}%
}%
\begin{pgfscope}%
\pgfsys@transformshift{1.846812in}{0.555556in}%
\pgfsys@useobject{currentmarker}{}%
\end{pgfscope}%
\end{pgfscope}%
\begin{pgfscope}%
\definecolor{textcolor}{rgb}{0.000000,0.000000,0.000000}%
\pgfsetstrokecolor{textcolor}%
\pgfsetfillcolor{textcolor}%
\pgftext[x=1.846812in,y=0.493056in,,top]{\color{textcolor}{\ifdefined\pdftexversion\else\setmainfont{NanumMyeongjo}\rmfamily\fi\fontsize{5.000000}{6.000000}\selectfont\catcode`\^=\active\def^{\ifmmode\sp\else\^{}\fi}\catcode`\%=\active\def%{\%}2016}}%
\end{pgfscope}%
\begin{pgfscope}%
\pgfsetbuttcap%
\pgfsetroundjoin%
\definecolor{currentfill}{rgb}{0.000000,0.000000,0.000000}%
\pgfsetfillcolor{currentfill}%
\pgfsetlinewidth{0.752812pt}%
\definecolor{currentstroke}{rgb}{0.000000,0.000000,0.000000}%
\pgfsetstrokecolor{currentstroke}%
\pgfsetdash{}{0pt}%
\pgfsys@defobject{currentmarker}{\pgfqpoint{0.000000in}{-0.013889in}}{\pgfqpoint{0.000000in}{0.000000in}}{%
\pgfpathmoveto{\pgfqpoint{0.000000in}{0.000000in}}%
\pgfpathlineto{\pgfqpoint{0.000000in}{-0.013889in}}%
\pgfusepath{stroke,fill}%
}%
\begin{pgfscope}%
\pgfsys@transformshift{2.186069in}{0.555556in}%
\pgfsys@useobject{currentmarker}{}%
\end{pgfscope}%
\end{pgfscope}%
\begin{pgfscope}%
\definecolor{textcolor}{rgb}{0.000000,0.000000,0.000000}%
\pgfsetstrokecolor{textcolor}%
\pgfsetfillcolor{textcolor}%
\pgftext[x=2.186069in,y=0.493056in,,top]{\color{textcolor}{\ifdefined\pdftexversion\else\setmainfont{NanumMyeongjo}\rmfamily\fi\fontsize{5.000000}{6.000000}\selectfont\catcode`\^=\active\def^{\ifmmode\sp\else\^{}\fi}\catcode`\%=\active\def%{\%}2017}}%
\end{pgfscope}%
\begin{pgfscope}%
\pgfsetbuttcap%
\pgfsetroundjoin%
\definecolor{currentfill}{rgb}{0.000000,0.000000,0.000000}%
\pgfsetfillcolor{currentfill}%
\pgfsetlinewidth{0.752812pt}%
\definecolor{currentstroke}{rgb}{0.000000,0.000000,0.000000}%
\pgfsetstrokecolor{currentstroke}%
\pgfsetdash{}{0pt}%
\pgfsys@defobject{currentmarker}{\pgfqpoint{0.000000in}{-0.013889in}}{\pgfqpoint{0.000000in}{0.000000in}}{%
\pgfpathmoveto{\pgfqpoint{0.000000in}{0.000000in}}%
\pgfpathlineto{\pgfqpoint{0.000000in}{-0.013889in}}%
\pgfusepath{stroke,fill}%
}%
\begin{pgfscope}%
\pgfsys@transformshift{2.525326in}{0.555556in}%
\pgfsys@useobject{currentmarker}{}%
\end{pgfscope}%
\end{pgfscope}%
\begin{pgfscope}%
\definecolor{textcolor}{rgb}{0.000000,0.000000,0.000000}%
\pgfsetstrokecolor{textcolor}%
\pgfsetfillcolor{textcolor}%
\pgftext[x=2.525326in,y=0.493056in,,top]{\color{textcolor}{\ifdefined\pdftexversion\else\setmainfont{NanumMyeongjo}\rmfamily\fi\fontsize{5.000000}{6.000000}\selectfont\catcode`\^=\active\def^{\ifmmode\sp\else\^{}\fi}\catcode`\%=\active\def%{\%}2018}}%
\end{pgfscope}%
\begin{pgfscope}%
\pgfsetbuttcap%
\pgfsetroundjoin%
\definecolor{currentfill}{rgb}{0.000000,0.000000,0.000000}%
\pgfsetfillcolor{currentfill}%
\pgfsetlinewidth{0.752812pt}%
\definecolor{currentstroke}{rgb}{0.000000,0.000000,0.000000}%
\pgfsetstrokecolor{currentstroke}%
\pgfsetdash{}{0pt}%
\pgfsys@defobject{currentmarker}{\pgfqpoint{0.000000in}{-0.013889in}}{\pgfqpoint{0.000000in}{0.000000in}}{%
\pgfpathmoveto{\pgfqpoint{0.000000in}{0.000000in}}%
\pgfpathlineto{\pgfqpoint{0.000000in}{-0.013889in}}%
\pgfusepath{stroke,fill}%
}%
\begin{pgfscope}%
\pgfsys@transformshift{2.864583in}{0.555556in}%
\pgfsys@useobject{currentmarker}{}%
\end{pgfscope}%
\end{pgfscope}%
\begin{pgfscope}%
\definecolor{textcolor}{rgb}{0.000000,0.000000,0.000000}%
\pgfsetstrokecolor{textcolor}%
\pgfsetfillcolor{textcolor}%
\pgftext[x=2.864583in,y=0.493056in,,top]{\color{textcolor}{\ifdefined\pdftexversion\else\setmainfont{NanumMyeongjo}\rmfamily\fi\fontsize{5.000000}{6.000000}\selectfont\catcode`\^=\active\def^{\ifmmode\sp\else\^{}\fi}\catcode`\%=\active\def%{\%}2019}}%
\end{pgfscope}%
\begin{pgfscope}%
\pgfsetbuttcap%
\pgfsetroundjoin%
\definecolor{currentfill}{rgb}{0.000000,0.000000,0.000000}%
\pgfsetfillcolor{currentfill}%
\pgfsetlinewidth{0.752812pt}%
\definecolor{currentstroke}{rgb}{0.000000,0.000000,0.000000}%
\pgfsetstrokecolor{currentstroke}%
\pgfsetdash{}{0pt}%
\pgfsys@defobject{currentmarker}{\pgfqpoint{0.000000in}{-0.013889in}}{\pgfqpoint{0.000000in}{0.000000in}}{%
\pgfpathmoveto{\pgfqpoint{0.000000in}{0.000000in}}%
\pgfpathlineto{\pgfqpoint{0.000000in}{-0.013889in}}%
\pgfusepath{stroke,fill}%
}%
\begin{pgfscope}%
\pgfsys@transformshift{3.203840in}{0.555556in}%
\pgfsys@useobject{currentmarker}{}%
\end{pgfscope}%
\end{pgfscope}%
\begin{pgfscope}%
\definecolor{textcolor}{rgb}{0.000000,0.000000,0.000000}%
\pgfsetstrokecolor{textcolor}%
\pgfsetfillcolor{textcolor}%
\pgftext[x=3.203840in,y=0.493056in,,top]{\color{textcolor}{\ifdefined\pdftexversion\else\setmainfont{NanumMyeongjo}\rmfamily\fi\fontsize{5.000000}{6.000000}\selectfont\catcode`\^=\active\def^{\ifmmode\sp\else\^{}\fi}\catcode`\%=\active\def%{\%}2020}}%
\end{pgfscope}%
\begin{pgfscope}%
\pgfsetbuttcap%
\pgfsetroundjoin%
\definecolor{currentfill}{rgb}{0.000000,0.000000,0.000000}%
\pgfsetfillcolor{currentfill}%
\pgfsetlinewidth{0.752812pt}%
\definecolor{currentstroke}{rgb}{0.000000,0.000000,0.000000}%
\pgfsetstrokecolor{currentstroke}%
\pgfsetdash{}{0pt}%
\pgfsys@defobject{currentmarker}{\pgfqpoint{0.000000in}{-0.013889in}}{\pgfqpoint{0.000000in}{0.000000in}}{%
\pgfpathmoveto{\pgfqpoint{0.000000in}{0.000000in}}%
\pgfpathlineto{\pgfqpoint{0.000000in}{-0.013889in}}%
\pgfusepath{stroke,fill}%
}%
\begin{pgfscope}%
\pgfsys@transformshift{3.543097in}{0.555556in}%
\pgfsys@useobject{currentmarker}{}%
\end{pgfscope}%
\end{pgfscope}%
\begin{pgfscope}%
\definecolor{textcolor}{rgb}{0.000000,0.000000,0.000000}%
\pgfsetstrokecolor{textcolor}%
\pgfsetfillcolor{textcolor}%
\pgftext[x=3.543097in,y=0.493056in,,top]{\color{textcolor}{\ifdefined\pdftexversion\else\setmainfont{NanumMyeongjo}\rmfamily\fi\fontsize{5.000000}{6.000000}\selectfont\catcode`\^=\active\def^{\ifmmode\sp\else\^{}\fi}\catcode`\%=\active\def%{\%}2021}}%
\end{pgfscope}%
\begin{pgfscope}%
\pgfsetbuttcap%
\pgfsetroundjoin%
\definecolor{currentfill}{rgb}{0.000000,0.000000,0.000000}%
\pgfsetfillcolor{currentfill}%
\pgfsetlinewidth{0.752812pt}%
\definecolor{currentstroke}{rgb}{0.000000,0.000000,0.000000}%
\pgfsetstrokecolor{currentstroke}%
\pgfsetdash{}{0pt}%
\pgfsys@defobject{currentmarker}{\pgfqpoint{0.000000in}{-0.013889in}}{\pgfqpoint{0.000000in}{0.000000in}}{%
\pgfpathmoveto{\pgfqpoint{0.000000in}{0.000000in}}%
\pgfpathlineto{\pgfqpoint{0.000000in}{-0.013889in}}%
\pgfusepath{stroke,fill}%
}%
\begin{pgfscope}%
\pgfsys@transformshift{3.882355in}{0.555556in}%
\pgfsys@useobject{currentmarker}{}%
\end{pgfscope}%
\end{pgfscope}%
\begin{pgfscope}%
\definecolor{textcolor}{rgb}{0.000000,0.000000,0.000000}%
\pgfsetstrokecolor{textcolor}%
\pgfsetfillcolor{textcolor}%
\pgftext[x=3.882355in,y=0.493056in,,top]{\color{textcolor}{\ifdefined\pdftexversion\else\setmainfont{NanumMyeongjo}\rmfamily\fi\fontsize{5.000000}{6.000000}\selectfont\catcode`\^=\active\def^{\ifmmode\sp\else\^{}\fi}\catcode`\%=\active\def%{\%}2022}}%
\end{pgfscope}%
\begin{pgfscope}%
\pgfsetbuttcap%
\pgfsetroundjoin%
\definecolor{currentfill}{rgb}{0.000000,0.000000,0.000000}%
\pgfsetfillcolor{currentfill}%
\pgfsetlinewidth{0.752812pt}%
\definecolor{currentstroke}{rgb}{0.000000,0.000000,0.000000}%
\pgfsetstrokecolor{currentstroke}%
\pgfsetdash{}{0pt}%
\pgfsys@defobject{currentmarker}{\pgfqpoint{0.000000in}{-0.013889in}}{\pgfqpoint{0.000000in}{0.000000in}}{%
\pgfpathmoveto{\pgfqpoint{0.000000in}{0.000000in}}%
\pgfpathlineto{\pgfqpoint{0.000000in}{-0.013889in}}%
\pgfusepath{stroke,fill}%
}%
\begin{pgfscope}%
\pgfsys@transformshift{4.221612in}{0.555556in}%
\pgfsys@useobject{currentmarker}{}%
\end{pgfscope}%
\end{pgfscope}%
\begin{pgfscope}%
\definecolor{textcolor}{rgb}{0.000000,0.000000,0.000000}%
\pgfsetstrokecolor{textcolor}%
\pgfsetfillcolor{textcolor}%
\pgftext[x=4.221612in,y=0.493056in,,top]{\color{textcolor}{\ifdefined\pdftexversion\else\setmainfont{NanumMyeongjo}\rmfamily\fi\fontsize{5.000000}{6.000000}\selectfont\catcode`\^=\active\def^{\ifmmode\sp\else\^{}\fi}\catcode`\%=\active\def%{\%}2023}}%
\end{pgfscope}%
\begin{pgfscope}%
\pgfsetbuttcap%
\pgfsetroundjoin%
\definecolor{currentfill}{rgb}{0.000000,0.000000,0.000000}%
\pgfsetfillcolor{currentfill}%
\pgfsetlinewidth{0.752812pt}%
\definecolor{currentstroke}{rgb}{0.000000,0.000000,0.000000}%
\pgfsetstrokecolor{currentstroke}%
\pgfsetdash{}{0pt}%
\pgfsys@defobject{currentmarker}{\pgfqpoint{0.000000in}{-0.013889in}}{\pgfqpoint{0.000000in}{0.000000in}}{%
\pgfpathmoveto{\pgfqpoint{0.000000in}{0.000000in}}%
\pgfpathlineto{\pgfqpoint{0.000000in}{-0.013889in}}%
\pgfusepath{stroke,fill}%
}%
\begin{pgfscope}%
\pgfsys@transformshift{4.560869in}{0.555556in}%
\pgfsys@useobject{currentmarker}{}%
\end{pgfscope}%
\end{pgfscope}%
\begin{pgfscope}%
\definecolor{textcolor}{rgb}{0.000000,0.000000,0.000000}%
\pgfsetstrokecolor{textcolor}%
\pgfsetfillcolor{textcolor}%
\pgftext[x=4.560869in,y=0.493056in,,top]{\color{textcolor}{\ifdefined\pdftexversion\else\setmainfont{NanumMyeongjo}\rmfamily\fi\fontsize{5.000000}{6.000000}\selectfont\catcode`\^=\active\def^{\ifmmode\sp\else\^{}\fi}\catcode`\%=\active\def%{\%}2024}}%
\end{pgfscope}%
\begin{pgfscope}%
\pgfpathrectangle{\pgfqpoint{0.868056in}{0.555556in}}{\pgfqpoint{3.993056in}{1.888889in}}%
\pgfusepath{clip}%
\pgfsetbuttcap%
\pgfsetroundjoin%
\pgfsetlinewidth{0.602250pt}%
\definecolor{currentstroke}{rgb}{0.690196,0.690196,0.690196}%
\pgfsetstrokecolor{currentstroke}%
\pgfsetstrokeopacity{0.400000}%
\pgfsetdash{{2.220000pt}{0.960000pt}}{0.000000pt}%
\pgfpathmoveto{\pgfqpoint{0.868056in}{0.555556in}}%
\pgfpathlineto{\pgfqpoint{4.861111in}{0.555556in}}%
\pgfusepath{stroke}%
\end{pgfscope}%
\begin{pgfscope}%
\pgfsetbuttcap%
\pgfsetroundjoin%
\definecolor{currentfill}{rgb}{0.000000,0.000000,0.000000}%
\pgfsetfillcolor{currentfill}%
\pgfsetlinewidth{0.752812pt}%
\definecolor{currentstroke}{rgb}{0.000000,0.000000,0.000000}%
\pgfsetstrokecolor{currentstroke}%
\pgfsetdash{}{0pt}%
\pgfsys@defobject{currentmarker}{\pgfqpoint{-0.013889in}{0.000000in}}{\pgfqpoint{-0.000000in}{0.000000in}}{%
\pgfpathmoveto{\pgfqpoint{-0.000000in}{0.000000in}}%
\pgfpathlineto{\pgfqpoint{-0.013889in}{0.000000in}}%
\pgfusepath{stroke,fill}%
}%
\begin{pgfscope}%
\pgfsys@transformshift{0.868056in}{0.555556in}%
\pgfsys@useobject{currentmarker}{}%
\end{pgfscope}%
\end{pgfscope}%
\begin{pgfscope}%
\definecolor{textcolor}{rgb}{0.000000,0.000000,0.000000}%
\pgfsetstrokecolor{textcolor}%
\pgfsetfillcolor{textcolor}%
\pgftext[x=0.768256in, y=0.527818in, left, base]{\color{textcolor}{\ifdefined\pdftexversion\else\setmainfont{NanumMyeongjo}\rmfamily\fi\fontsize{5.000000}{6.000000}\selectfont\catcode`\^=\active\def^{\ifmmode\sp\else\^{}\fi}\catcode`\%=\active\def%{\%}0}}%
\end{pgfscope}%
\begin{pgfscope}%
\pgfpathrectangle{\pgfqpoint{0.868056in}{0.555556in}}{\pgfqpoint{3.993056in}{1.888889in}}%
\pgfusepath{clip}%
\pgfsetbuttcap%
\pgfsetroundjoin%
\pgfsetlinewidth{0.602250pt}%
\definecolor{currentstroke}{rgb}{0.690196,0.690196,0.690196}%
\pgfsetstrokecolor{currentstroke}%
\pgfsetstrokeopacity{0.400000}%
\pgfsetdash{{2.220000pt}{0.960000pt}}{0.000000pt}%
\pgfpathmoveto{\pgfqpoint{0.868056in}{0.762577in}}%
\pgfpathlineto{\pgfqpoint{4.861111in}{0.762577in}}%
\pgfusepath{stroke}%
\end{pgfscope}%
\begin{pgfscope}%
\pgfsetbuttcap%
\pgfsetroundjoin%
\definecolor{currentfill}{rgb}{0.000000,0.000000,0.000000}%
\pgfsetfillcolor{currentfill}%
\pgfsetlinewidth{0.752812pt}%
\definecolor{currentstroke}{rgb}{0.000000,0.000000,0.000000}%
\pgfsetstrokecolor{currentstroke}%
\pgfsetdash{}{0pt}%
\pgfsys@defobject{currentmarker}{\pgfqpoint{-0.013889in}{0.000000in}}{\pgfqpoint{-0.000000in}{0.000000in}}{%
\pgfpathmoveto{\pgfqpoint{-0.000000in}{0.000000in}}%
\pgfpathlineto{\pgfqpoint{-0.013889in}{0.000000in}}%
\pgfusepath{stroke,fill}%
}%
\begin{pgfscope}%
\pgfsys@transformshift{0.868056in}{0.762577in}%
\pgfsys@useobject{currentmarker}{}%
\end{pgfscope}%
\end{pgfscope}%
\begin{pgfscope}%
\definecolor{textcolor}{rgb}{0.000000,0.000000,0.000000}%
\pgfsetstrokecolor{textcolor}%
\pgfsetfillcolor{textcolor}%
\pgftext[x=0.702271in, y=0.734840in, left, base]{\color{textcolor}{\ifdefined\pdftexversion\else\setmainfont{NanumMyeongjo}\rmfamily\fi\fontsize{5.000000}{6.000000}\selectfont\catcode`\^=\active\def^{\ifmmode\sp\else\^{}\fi}\catcode`\%=\active\def%{\%}2만}}%
\end{pgfscope}%
\begin{pgfscope}%
\pgfpathrectangle{\pgfqpoint{0.868056in}{0.555556in}}{\pgfqpoint{3.993056in}{1.888889in}}%
\pgfusepath{clip}%
\pgfsetbuttcap%
\pgfsetroundjoin%
\pgfsetlinewidth{0.602250pt}%
\definecolor{currentstroke}{rgb}{0.690196,0.690196,0.690196}%
\pgfsetstrokecolor{currentstroke}%
\pgfsetstrokeopacity{0.400000}%
\pgfsetdash{{2.220000pt}{0.960000pt}}{0.000000pt}%
\pgfpathmoveto{\pgfqpoint{0.868056in}{0.969598in}}%
\pgfpathlineto{\pgfqpoint{4.861111in}{0.969598in}}%
\pgfusepath{stroke}%
\end{pgfscope}%
\begin{pgfscope}%
\pgfsetbuttcap%
\pgfsetroundjoin%
\definecolor{currentfill}{rgb}{0.000000,0.000000,0.000000}%
\pgfsetfillcolor{currentfill}%
\pgfsetlinewidth{0.752812pt}%
\definecolor{currentstroke}{rgb}{0.000000,0.000000,0.000000}%
\pgfsetstrokecolor{currentstroke}%
\pgfsetdash{}{0pt}%
\pgfsys@defobject{currentmarker}{\pgfqpoint{-0.013889in}{0.000000in}}{\pgfqpoint{-0.000000in}{0.000000in}}{%
\pgfpathmoveto{\pgfqpoint{-0.000000in}{0.000000in}}%
\pgfpathlineto{\pgfqpoint{-0.013889in}{0.000000in}}%
\pgfusepath{stroke,fill}%
}%
\begin{pgfscope}%
\pgfsys@transformshift{0.868056in}{0.969598in}%
\pgfsys@useobject{currentmarker}{}%
\end{pgfscope}%
\end{pgfscope}%
\begin{pgfscope}%
\definecolor{textcolor}{rgb}{0.000000,0.000000,0.000000}%
\pgfsetstrokecolor{textcolor}%
\pgfsetfillcolor{textcolor}%
\pgftext[x=0.702271in, y=0.941861in, left, base]{\color{textcolor}{\ifdefined\pdftexversion\else\setmainfont{NanumMyeongjo}\rmfamily\fi\fontsize{5.000000}{6.000000}\selectfont\catcode`\^=\active\def^{\ifmmode\sp\else\^{}\fi}\catcode`\%=\active\def%{\%}4만}}%
\end{pgfscope}%
\begin{pgfscope}%
\pgfpathrectangle{\pgfqpoint{0.868056in}{0.555556in}}{\pgfqpoint{3.993056in}{1.888889in}}%
\pgfusepath{clip}%
\pgfsetbuttcap%
\pgfsetroundjoin%
\pgfsetlinewidth{0.602250pt}%
\definecolor{currentstroke}{rgb}{0.690196,0.690196,0.690196}%
\pgfsetstrokecolor{currentstroke}%
\pgfsetstrokeopacity{0.400000}%
\pgfsetdash{{2.220000pt}{0.960000pt}}{0.000000pt}%
\pgfpathmoveto{\pgfqpoint{0.868056in}{1.176619in}}%
\pgfpathlineto{\pgfqpoint{4.861111in}{1.176619in}}%
\pgfusepath{stroke}%
\end{pgfscope}%
\begin{pgfscope}%
\pgfsetbuttcap%
\pgfsetroundjoin%
\definecolor{currentfill}{rgb}{0.000000,0.000000,0.000000}%
\pgfsetfillcolor{currentfill}%
\pgfsetlinewidth{0.752812pt}%
\definecolor{currentstroke}{rgb}{0.000000,0.000000,0.000000}%
\pgfsetstrokecolor{currentstroke}%
\pgfsetdash{}{0pt}%
\pgfsys@defobject{currentmarker}{\pgfqpoint{-0.013889in}{0.000000in}}{\pgfqpoint{-0.000000in}{0.000000in}}{%
\pgfpathmoveto{\pgfqpoint{-0.000000in}{0.000000in}}%
\pgfpathlineto{\pgfqpoint{-0.013889in}{0.000000in}}%
\pgfusepath{stroke,fill}%
}%
\begin{pgfscope}%
\pgfsys@transformshift{0.868056in}{1.176619in}%
\pgfsys@useobject{currentmarker}{}%
\end{pgfscope}%
\end{pgfscope}%
\begin{pgfscope}%
\definecolor{textcolor}{rgb}{0.000000,0.000000,0.000000}%
\pgfsetstrokecolor{textcolor}%
\pgfsetfillcolor{textcolor}%
\pgftext[x=0.702271in, y=1.148882in, left, base]{\color{textcolor}{\ifdefined\pdftexversion\else\setmainfont{NanumMyeongjo}\rmfamily\fi\fontsize{5.000000}{6.000000}\selectfont\catcode`\^=\active\def^{\ifmmode\sp\else\^{}\fi}\catcode`\%=\active\def%{\%}6만}}%
\end{pgfscope}%
\begin{pgfscope}%
\pgfpathrectangle{\pgfqpoint{0.868056in}{0.555556in}}{\pgfqpoint{3.993056in}{1.888889in}}%
\pgfusepath{clip}%
\pgfsetbuttcap%
\pgfsetroundjoin%
\pgfsetlinewidth{0.602250pt}%
\definecolor{currentstroke}{rgb}{0.690196,0.690196,0.690196}%
\pgfsetstrokecolor{currentstroke}%
\pgfsetstrokeopacity{0.400000}%
\pgfsetdash{{2.220000pt}{0.960000pt}}{0.000000pt}%
\pgfpathmoveto{\pgfqpoint{0.868056in}{1.383640in}}%
\pgfpathlineto{\pgfqpoint{4.861111in}{1.383640in}}%
\pgfusepath{stroke}%
\end{pgfscope}%
\begin{pgfscope}%
\pgfsetbuttcap%
\pgfsetroundjoin%
\definecolor{currentfill}{rgb}{0.000000,0.000000,0.000000}%
\pgfsetfillcolor{currentfill}%
\pgfsetlinewidth{0.752812pt}%
\definecolor{currentstroke}{rgb}{0.000000,0.000000,0.000000}%
\pgfsetstrokecolor{currentstroke}%
\pgfsetdash{}{0pt}%
\pgfsys@defobject{currentmarker}{\pgfqpoint{-0.013889in}{0.000000in}}{\pgfqpoint{-0.000000in}{0.000000in}}{%
\pgfpathmoveto{\pgfqpoint{-0.000000in}{0.000000in}}%
\pgfpathlineto{\pgfqpoint{-0.013889in}{0.000000in}}%
\pgfusepath{stroke,fill}%
}%
\begin{pgfscope}%
\pgfsys@transformshift{0.868056in}{1.383640in}%
\pgfsys@useobject{currentmarker}{}%
\end{pgfscope}%
\end{pgfscope}%
\begin{pgfscope}%
\definecolor{textcolor}{rgb}{0.000000,0.000000,0.000000}%
\pgfsetstrokecolor{textcolor}%
\pgfsetfillcolor{textcolor}%
\pgftext[x=0.702271in, y=1.355903in, left, base]{\color{textcolor}{\ifdefined\pdftexversion\else\setmainfont{NanumMyeongjo}\rmfamily\fi\fontsize{5.000000}{6.000000}\selectfont\catcode`\^=\active\def^{\ifmmode\sp\else\^{}\fi}\catcode`\%=\active\def%{\%}8만}}%
\end{pgfscope}%
\begin{pgfscope}%
\pgfpathrectangle{\pgfqpoint{0.868056in}{0.555556in}}{\pgfqpoint{3.993056in}{1.888889in}}%
\pgfusepath{clip}%
\pgfsetbuttcap%
\pgfsetroundjoin%
\pgfsetlinewidth{0.602250pt}%
\definecolor{currentstroke}{rgb}{0.690196,0.690196,0.690196}%
\pgfsetstrokecolor{currentstroke}%
\pgfsetstrokeopacity{0.400000}%
\pgfsetdash{{2.220000pt}{0.960000pt}}{0.000000pt}%
\pgfpathmoveto{\pgfqpoint{0.868056in}{1.590662in}}%
\pgfpathlineto{\pgfqpoint{4.861111in}{1.590662in}}%
\pgfusepath{stroke}%
\end{pgfscope}%
\begin{pgfscope}%
\pgfsetbuttcap%
\pgfsetroundjoin%
\definecolor{currentfill}{rgb}{0.000000,0.000000,0.000000}%
\pgfsetfillcolor{currentfill}%
\pgfsetlinewidth{0.752812pt}%
\definecolor{currentstroke}{rgb}{0.000000,0.000000,0.000000}%
\pgfsetstrokecolor{currentstroke}%
\pgfsetdash{}{0pt}%
\pgfsys@defobject{currentmarker}{\pgfqpoint{-0.013889in}{0.000000in}}{\pgfqpoint{-0.000000in}{0.000000in}}{%
\pgfpathmoveto{\pgfqpoint{-0.000000in}{0.000000in}}%
\pgfpathlineto{\pgfqpoint{-0.013889in}{0.000000in}}%
\pgfusepath{stroke,fill}%
}%
\begin{pgfscope}%
\pgfsys@transformshift{0.868056in}{1.590662in}%
\pgfsys@useobject{currentmarker}{}%
\end{pgfscope}%
\end{pgfscope}%
\begin{pgfscope}%
\definecolor{textcolor}{rgb}{0.000000,0.000000,0.000000}%
\pgfsetstrokecolor{textcolor}%
\pgfsetfillcolor{textcolor}%
\pgftext[x=0.664971in, y=1.562924in, left, base]{\color{textcolor}{\ifdefined\pdftexversion\else\setmainfont{NanumMyeongjo}\rmfamily\fi\fontsize{5.000000}{6.000000}\selectfont\catcode`\^=\active\def^{\ifmmode\sp\else\^{}\fi}\catcode`\%=\active\def%{\%}10만}}%
\end{pgfscope}%
\begin{pgfscope}%
\pgfpathrectangle{\pgfqpoint{0.868056in}{0.555556in}}{\pgfqpoint{3.993056in}{1.888889in}}%
\pgfusepath{clip}%
\pgfsetbuttcap%
\pgfsetroundjoin%
\pgfsetlinewidth{0.602250pt}%
\definecolor{currentstroke}{rgb}{0.690196,0.690196,0.690196}%
\pgfsetstrokecolor{currentstroke}%
\pgfsetstrokeopacity{0.400000}%
\pgfsetdash{{2.220000pt}{0.960000pt}}{0.000000pt}%
\pgfpathmoveto{\pgfqpoint{0.868056in}{1.797683in}}%
\pgfpathlineto{\pgfqpoint{4.861111in}{1.797683in}}%
\pgfusepath{stroke}%
\end{pgfscope}%
\begin{pgfscope}%
\pgfsetbuttcap%
\pgfsetroundjoin%
\definecolor{currentfill}{rgb}{0.000000,0.000000,0.000000}%
\pgfsetfillcolor{currentfill}%
\pgfsetlinewidth{0.752812pt}%
\definecolor{currentstroke}{rgb}{0.000000,0.000000,0.000000}%
\pgfsetstrokecolor{currentstroke}%
\pgfsetdash{}{0pt}%
\pgfsys@defobject{currentmarker}{\pgfqpoint{-0.013889in}{0.000000in}}{\pgfqpoint{-0.000000in}{0.000000in}}{%
\pgfpathmoveto{\pgfqpoint{-0.000000in}{0.000000in}}%
\pgfpathlineto{\pgfqpoint{-0.013889in}{0.000000in}}%
\pgfusepath{stroke,fill}%
}%
\begin{pgfscope}%
\pgfsys@transformshift{0.868056in}{1.797683in}%
\pgfsys@useobject{currentmarker}{}%
\end{pgfscope}%
\end{pgfscope}%
\begin{pgfscope}%
\definecolor{textcolor}{rgb}{0.000000,0.000000,0.000000}%
\pgfsetstrokecolor{textcolor}%
\pgfsetfillcolor{textcolor}%
\pgftext[x=0.664971in, y=1.769946in, left, base]{\color{textcolor}{\ifdefined\pdftexversion\else\setmainfont{NanumMyeongjo}\rmfamily\fi\fontsize{5.000000}{6.000000}\selectfont\catcode`\^=\active\def^{\ifmmode\sp\else\^{}\fi}\catcode`\%=\active\def%{\%}12만}}%
\end{pgfscope}%
\begin{pgfscope}%
\pgfpathrectangle{\pgfqpoint{0.868056in}{0.555556in}}{\pgfqpoint{3.993056in}{1.888889in}}%
\pgfusepath{clip}%
\pgfsetbuttcap%
\pgfsetroundjoin%
\pgfsetlinewidth{0.602250pt}%
\definecolor{currentstroke}{rgb}{0.690196,0.690196,0.690196}%
\pgfsetstrokecolor{currentstroke}%
\pgfsetstrokeopacity{0.400000}%
\pgfsetdash{{2.220000pt}{0.960000pt}}{0.000000pt}%
\pgfpathmoveto{\pgfqpoint{0.868056in}{2.004704in}}%
\pgfpathlineto{\pgfqpoint{4.861111in}{2.004704in}}%
\pgfusepath{stroke}%
\end{pgfscope}%
\begin{pgfscope}%
\pgfsetbuttcap%
\pgfsetroundjoin%
\definecolor{currentfill}{rgb}{0.000000,0.000000,0.000000}%
\pgfsetfillcolor{currentfill}%
\pgfsetlinewidth{0.752812pt}%
\definecolor{currentstroke}{rgb}{0.000000,0.000000,0.000000}%
\pgfsetstrokecolor{currentstroke}%
\pgfsetdash{}{0pt}%
\pgfsys@defobject{currentmarker}{\pgfqpoint{-0.013889in}{0.000000in}}{\pgfqpoint{-0.000000in}{0.000000in}}{%
\pgfpathmoveto{\pgfqpoint{-0.000000in}{0.000000in}}%
\pgfpathlineto{\pgfqpoint{-0.013889in}{0.000000in}}%
\pgfusepath{stroke,fill}%
}%
\begin{pgfscope}%
\pgfsys@transformshift{0.868056in}{2.004704in}%
\pgfsys@useobject{currentmarker}{}%
\end{pgfscope}%
\end{pgfscope}%
\begin{pgfscope}%
\definecolor{textcolor}{rgb}{0.000000,0.000000,0.000000}%
\pgfsetstrokecolor{textcolor}%
\pgfsetfillcolor{textcolor}%
\pgftext[x=0.664971in, y=1.976967in, left, base]{\color{textcolor}{\ifdefined\pdftexversion\else\setmainfont{NanumMyeongjo}\rmfamily\fi\fontsize{5.000000}{6.000000}\selectfont\catcode`\^=\active\def^{\ifmmode\sp\else\^{}\fi}\catcode`\%=\active\def%{\%}14만}}%
\end{pgfscope}%
\begin{pgfscope}%
\pgfpathrectangle{\pgfqpoint{0.868056in}{0.555556in}}{\pgfqpoint{3.993056in}{1.888889in}}%
\pgfusepath{clip}%
\pgfsetbuttcap%
\pgfsetroundjoin%
\pgfsetlinewidth{0.602250pt}%
\definecolor{currentstroke}{rgb}{0.690196,0.690196,0.690196}%
\pgfsetstrokecolor{currentstroke}%
\pgfsetstrokeopacity{0.400000}%
\pgfsetdash{{2.220000pt}{0.960000pt}}{0.000000pt}%
\pgfpathmoveto{\pgfqpoint{0.868056in}{2.211725in}}%
\pgfpathlineto{\pgfqpoint{4.861111in}{2.211725in}}%
\pgfusepath{stroke}%
\end{pgfscope}%
\begin{pgfscope}%
\pgfsetbuttcap%
\pgfsetroundjoin%
\definecolor{currentfill}{rgb}{0.000000,0.000000,0.000000}%
\pgfsetfillcolor{currentfill}%
\pgfsetlinewidth{0.752812pt}%
\definecolor{currentstroke}{rgb}{0.000000,0.000000,0.000000}%
\pgfsetstrokecolor{currentstroke}%
\pgfsetdash{}{0pt}%
\pgfsys@defobject{currentmarker}{\pgfqpoint{-0.013889in}{0.000000in}}{\pgfqpoint{-0.000000in}{0.000000in}}{%
\pgfpathmoveto{\pgfqpoint{-0.000000in}{0.000000in}}%
\pgfpathlineto{\pgfqpoint{-0.013889in}{0.000000in}}%
\pgfusepath{stroke,fill}%
}%
\begin{pgfscope}%
\pgfsys@transformshift{0.868056in}{2.211725in}%
\pgfsys@useobject{currentmarker}{}%
\end{pgfscope}%
\end{pgfscope}%
\begin{pgfscope}%
\definecolor{textcolor}{rgb}{0.000000,0.000000,0.000000}%
\pgfsetstrokecolor{textcolor}%
\pgfsetfillcolor{textcolor}%
\pgftext[x=0.664971in, y=2.183988in, left, base]{\color{textcolor}{\ifdefined\pdftexversion\else\setmainfont{NanumMyeongjo}\rmfamily\fi\fontsize{5.000000}{6.000000}\selectfont\catcode`\^=\active\def^{\ifmmode\sp\else\^{}\fi}\catcode`\%=\active\def%{\%}16만}}%
\end{pgfscope}%
\begin{pgfscope}%
\pgfpathrectangle{\pgfqpoint{0.868056in}{0.555556in}}{\pgfqpoint{3.993056in}{1.888889in}}%
\pgfusepath{clip}%
\pgfsetbuttcap%
\pgfsetroundjoin%
\pgfsetlinewidth{0.602250pt}%
\definecolor{currentstroke}{rgb}{0.690196,0.690196,0.690196}%
\pgfsetstrokecolor{currentstroke}%
\pgfsetstrokeopacity{0.400000}%
\pgfsetdash{{2.220000pt}{0.960000pt}}{0.000000pt}%
\pgfpathmoveto{\pgfqpoint{0.868056in}{2.418746in}}%
\pgfpathlineto{\pgfqpoint{4.861111in}{2.418746in}}%
\pgfusepath{stroke}%
\end{pgfscope}%
\begin{pgfscope}%
\pgfsetbuttcap%
\pgfsetroundjoin%
\definecolor{currentfill}{rgb}{0.000000,0.000000,0.000000}%
\pgfsetfillcolor{currentfill}%
\pgfsetlinewidth{0.752812pt}%
\definecolor{currentstroke}{rgb}{0.000000,0.000000,0.000000}%
\pgfsetstrokecolor{currentstroke}%
\pgfsetdash{}{0pt}%
\pgfsys@defobject{currentmarker}{\pgfqpoint{-0.013889in}{0.000000in}}{\pgfqpoint{-0.000000in}{0.000000in}}{%
\pgfpathmoveto{\pgfqpoint{-0.000000in}{0.000000in}}%
\pgfpathlineto{\pgfqpoint{-0.013889in}{0.000000in}}%
\pgfusepath{stroke,fill}%
}%
\begin{pgfscope}%
\pgfsys@transformshift{0.868056in}{2.418746in}%
\pgfsys@useobject{currentmarker}{}%
\end{pgfscope}%
\end{pgfscope}%
\begin{pgfscope}%
\definecolor{textcolor}{rgb}{0.000000,0.000000,0.000000}%
\pgfsetstrokecolor{textcolor}%
\pgfsetfillcolor{textcolor}%
\pgftext[x=0.664971in, y=2.391009in, left, base]{\color{textcolor}{\ifdefined\pdftexversion\else\setmainfont{NanumMyeongjo}\rmfamily\fi\fontsize{5.000000}{6.000000}\selectfont\catcode`\^=\active\def^{\ifmmode\sp\else\^{}\fi}\catcode`\%=\active\def%{\%}18만}}%
\end{pgfscope}%
\begin{pgfscope}%
\pgfsetrectcap%
\pgfsetmiterjoin%
\pgfsetlinewidth{0.752812pt}%
\definecolor{currentstroke}{rgb}{0.000000,0.000000,0.000000}%
\pgfsetstrokecolor{currentstroke}%
\pgfsetdash{}{0pt}%
\pgfpathmoveto{\pgfqpoint{0.868056in}{0.555556in}}%
\pgfpathlineto{\pgfqpoint{0.868056in}{2.444444in}}%
\pgfusepath{stroke}%
\end{pgfscope}%
\begin{pgfscope}%
\pgfsetrectcap%
\pgfsetmiterjoin%
\pgfsetlinewidth{0.752812pt}%
\definecolor{currentstroke}{rgb}{0.000000,0.000000,0.000000}%
\pgfsetstrokecolor{currentstroke}%
\pgfsetdash{}{0pt}%
\pgfpathmoveto{\pgfqpoint{0.868056in}{0.555556in}}%
\pgfpathlineto{\pgfqpoint{4.861111in}{0.555556in}}%
\pgfusepath{stroke}%
\end{pgfscope}%
\begin{pgfscope}%
\pgfpathrectangle{\pgfqpoint{0.868056in}{0.555556in}}{\pgfqpoint{3.993056in}{1.888889in}}%
\pgfusepath{clip}%
\pgfsetbuttcap%
\pgfsetmiterjoin%
\definecolor{currentfill}{rgb}{0.337255,0.713725,0.627451}%
\pgfsetfillcolor{currentfill}%
\pgfsetlinewidth{1.003750pt}%
\definecolor{currentstroke}{rgb}{0.266667,0.266667,0.266667}%
\pgfsetstrokecolor{currentstroke}%
\pgfsetdash{}{0pt}%
\pgfpathmoveto{\pgfqpoint{1.049558in}{0.555556in}}%
\pgfpathlineto{\pgfqpoint{1.287038in}{0.555556in}}%
\pgfpathlineto{\pgfqpoint{1.287038in}{0.710221in}}%
\pgfpathlineto{\pgfqpoint{1.049558in}{0.710221in}}%
\pgfpathlineto{\pgfqpoint{1.049558in}{0.555556in}}%
\pgfpathclose%
\pgfusepath{stroke,fill}%
\end{pgfscope}%
\begin{pgfscope}%
\pgfpathrectangle{\pgfqpoint{0.868056in}{0.555556in}}{\pgfqpoint{3.993056in}{1.888889in}}%
\pgfusepath{clip}%
\pgfsetbuttcap%
\pgfsetmiterjoin%
\definecolor{currentfill}{rgb}{0.337255,0.713725,0.627451}%
\pgfsetfillcolor{currentfill}%
\pgfsetlinewidth{1.003750pt}%
\definecolor{currentstroke}{rgb}{0.266667,0.266667,0.266667}%
\pgfsetstrokecolor{currentstroke}%
\pgfsetdash{}{0pt}%
\pgfpathmoveto{\pgfqpoint{1.388815in}{0.555556in}}%
\pgfpathlineto{\pgfqpoint{1.626295in}{0.555556in}}%
\pgfpathlineto{\pgfqpoint{1.626295in}{0.670535in}}%
\pgfpathlineto{\pgfqpoint{1.388815in}{0.670535in}}%
\pgfpathlineto{\pgfqpoint{1.388815in}{0.555556in}}%
\pgfpathclose%
\pgfusepath{stroke,fill}%
\end{pgfscope}%
\begin{pgfscope}%
\pgfpathrectangle{\pgfqpoint{0.868056in}{0.555556in}}{\pgfqpoint{3.993056in}{1.888889in}}%
\pgfusepath{clip}%
\pgfsetbuttcap%
\pgfsetmiterjoin%
\definecolor{currentfill}{rgb}{0.337255,0.713725,0.627451}%
\pgfsetfillcolor{currentfill}%
\pgfsetlinewidth{1.003750pt}%
\definecolor{currentstroke}{rgb}{0.266667,0.266667,0.266667}%
\pgfsetstrokecolor{currentstroke}%
\pgfsetdash{}{0pt}%
\pgfpathmoveto{\pgfqpoint{1.728072in}{0.555556in}}%
\pgfpathlineto{\pgfqpoint{1.965552in}{0.555556in}}%
\pgfpathlineto{\pgfqpoint{1.965552in}{0.631284in}}%
\pgfpathlineto{\pgfqpoint{1.728072in}{0.631284in}}%
\pgfpathlineto{\pgfqpoint{1.728072in}{0.555556in}}%
\pgfpathclose%
\pgfusepath{stroke,fill}%
\end{pgfscope}%
\begin{pgfscope}%
\pgfpathrectangle{\pgfqpoint{0.868056in}{0.555556in}}{\pgfqpoint{3.993056in}{1.888889in}}%
\pgfusepath{clip}%
\pgfsetbuttcap%
\pgfsetmiterjoin%
\definecolor{currentfill}{rgb}{0.337255,0.713725,0.627451}%
\pgfsetfillcolor{currentfill}%
\pgfsetlinewidth{1.003750pt}%
\definecolor{currentstroke}{rgb}{0.266667,0.266667,0.266667}%
\pgfsetstrokecolor{currentstroke}%
\pgfsetdash{}{0pt}%
\pgfpathmoveto{\pgfqpoint{2.067329in}{0.555556in}}%
\pgfpathlineto{\pgfqpoint{2.304809in}{0.555556in}}%
\pgfpathlineto{\pgfqpoint{2.304809in}{0.686952in}}%
\pgfpathlineto{\pgfqpoint{2.067329in}{0.686952in}}%
\pgfpathlineto{\pgfqpoint{2.067329in}{0.555556in}}%
\pgfpathclose%
\pgfusepath{stroke,fill}%
\end{pgfscope}%
\begin{pgfscope}%
\pgfpathrectangle{\pgfqpoint{0.868056in}{0.555556in}}{\pgfqpoint{3.993056in}{1.888889in}}%
\pgfusepath{clip}%
\pgfsetbuttcap%
\pgfsetmiterjoin%
\definecolor{currentfill}{rgb}{0.337255,0.713725,0.627451}%
\pgfsetfillcolor{currentfill}%
\pgfsetlinewidth{1.003750pt}%
\definecolor{currentstroke}{rgb}{0.266667,0.266667,0.266667}%
\pgfsetstrokecolor{currentstroke}%
\pgfsetdash{}{0pt}%
\pgfpathmoveto{\pgfqpoint{2.406586in}{0.555556in}}%
\pgfpathlineto{\pgfqpoint{2.644066in}{0.555556in}}%
\pgfpathlineto{\pgfqpoint{2.644066in}{0.746574in}}%
\pgfpathlineto{\pgfqpoint{2.406586in}{0.746574in}}%
\pgfpathlineto{\pgfqpoint{2.406586in}{0.555556in}}%
\pgfpathclose%
\pgfusepath{stroke,fill}%
\end{pgfscope}%
\begin{pgfscope}%
\pgfpathrectangle{\pgfqpoint{0.868056in}{0.555556in}}{\pgfqpoint{3.993056in}{1.888889in}}%
\pgfusepath{clip}%
\pgfsetbuttcap%
\pgfsetmiterjoin%
\definecolor{currentfill}{rgb}{0.337255,0.713725,0.627451}%
\pgfsetfillcolor{currentfill}%
\pgfsetlinewidth{1.003750pt}%
\definecolor{currentstroke}{rgb}{0.266667,0.266667,0.266667}%
\pgfsetstrokecolor{currentstroke}%
\pgfsetdash{}{0pt}%
\pgfpathmoveto{\pgfqpoint{2.745843in}{0.555556in}}%
\pgfpathlineto{\pgfqpoint{2.983323in}{0.555556in}}%
\pgfpathlineto{\pgfqpoint{2.983323in}{0.811713in}}%
\pgfpathlineto{\pgfqpoint{2.745843in}{0.811713in}}%
\pgfpathlineto{\pgfqpoint{2.745843in}{0.555556in}}%
\pgfpathclose%
\pgfusepath{stroke,fill}%
\end{pgfscope}%
\begin{pgfscope}%
\pgfpathrectangle{\pgfqpoint{0.868056in}{0.555556in}}{\pgfqpoint{3.993056in}{1.888889in}}%
\pgfusepath{clip}%
\pgfsetbuttcap%
\pgfsetmiterjoin%
\definecolor{currentfill}{rgb}{0.337255,0.713725,0.627451}%
\pgfsetfillcolor{currentfill}%
\pgfsetlinewidth{1.003750pt}%
\definecolor{currentstroke}{rgb}{0.266667,0.266667,0.266667}%
\pgfsetstrokecolor{currentstroke}%
\pgfsetdash{}{0pt}%
\pgfpathmoveto{\pgfqpoint{3.085100in}{0.555556in}}%
\pgfpathlineto{\pgfqpoint{3.322580in}{0.555556in}}%
\pgfpathlineto{\pgfqpoint{3.322580in}{0.720614in}}%
\pgfpathlineto{\pgfqpoint{3.085100in}{0.720614in}}%
\pgfpathlineto{\pgfqpoint{3.085100in}{0.555556in}}%
\pgfpathclose%
\pgfusepath{stroke,fill}%
\end{pgfscope}%
\begin{pgfscope}%
\pgfpathrectangle{\pgfqpoint{0.868056in}{0.555556in}}{\pgfqpoint{3.993056in}{1.888889in}}%
\pgfusepath{clip}%
\pgfsetbuttcap%
\pgfsetmiterjoin%
\definecolor{currentfill}{rgb}{0.337255,0.713725,0.627451}%
\pgfsetfillcolor{currentfill}%
\pgfsetlinewidth{1.003750pt}%
\definecolor{currentstroke}{rgb}{0.266667,0.266667,0.266667}%
\pgfsetstrokecolor{currentstroke}%
\pgfsetdash{}{0pt}%
\pgfpathmoveto{\pgfqpoint{3.424357in}{0.555556in}}%
\pgfpathlineto{\pgfqpoint{3.661837in}{0.555556in}}%
\pgfpathlineto{\pgfqpoint{3.661837in}{0.808618in}}%
\pgfpathlineto{\pgfqpoint{3.424357in}{0.808618in}}%
\pgfpathlineto{\pgfqpoint{3.424357in}{0.555556in}}%
\pgfpathclose%
\pgfusepath{stroke,fill}%
\end{pgfscope}%
\begin{pgfscope}%
\pgfpathrectangle{\pgfqpoint{0.868056in}{0.555556in}}{\pgfqpoint{3.993056in}{1.888889in}}%
\pgfusepath{clip}%
\pgfsetbuttcap%
\pgfsetmiterjoin%
\definecolor{currentfill}{rgb}{0.337255,0.713725,0.627451}%
\pgfsetfillcolor{currentfill}%
\pgfsetlinewidth{1.003750pt}%
\definecolor{currentstroke}{rgb}{0.266667,0.266667,0.266667}%
\pgfsetstrokecolor{currentstroke}%
\pgfsetdash{}{0pt}%
\pgfpathmoveto{\pgfqpoint{3.763615in}{0.555556in}}%
\pgfpathlineto{\pgfqpoint{4.001094in}{0.555556in}}%
\pgfpathlineto{\pgfqpoint{4.001094in}{0.916394in}}%
\pgfpathlineto{\pgfqpoint{3.763615in}{0.916394in}}%
\pgfpathlineto{\pgfqpoint{3.763615in}{0.555556in}}%
\pgfpathclose%
\pgfusepath{stroke,fill}%
\end{pgfscope}%
\begin{pgfscope}%
\pgfpathrectangle{\pgfqpoint{0.868056in}{0.555556in}}{\pgfqpoint{3.993056in}{1.888889in}}%
\pgfusepath{clip}%
\pgfsetbuttcap%
\pgfsetmiterjoin%
\definecolor{currentfill}{rgb}{0.337255,0.713725,0.627451}%
\pgfsetfillcolor{currentfill}%
\pgfsetlinewidth{1.003750pt}%
\definecolor{currentstroke}{rgb}{0.266667,0.266667,0.266667}%
\pgfsetstrokecolor{currentstroke}%
\pgfsetdash{}{0pt}%
\pgfpathmoveto{\pgfqpoint{4.102872in}{0.555556in}}%
\pgfpathlineto{\pgfqpoint{4.340352in}{0.555556in}}%
\pgfpathlineto{\pgfqpoint{4.340352in}{0.965023in}}%
\pgfpathlineto{\pgfqpoint{4.102872in}{0.965023in}}%
\pgfpathlineto{\pgfqpoint{4.102872in}{0.555556in}}%
\pgfpathclose%
\pgfusepath{stroke,fill}%
\end{pgfscope}%
\begin{pgfscope}%
\pgfpathrectangle{\pgfqpoint{0.868056in}{0.555556in}}{\pgfqpoint{3.993056in}{1.888889in}}%
\pgfusepath{clip}%
\pgfsetbuttcap%
\pgfsetmiterjoin%
\definecolor{currentfill}{rgb}{0.337255,0.713725,0.627451}%
\pgfsetfillcolor{currentfill}%
\pgfsetlinewidth{1.003750pt}%
\definecolor{currentstroke}{rgb}{0.266667,0.266667,0.266667}%
\pgfsetstrokecolor{currentstroke}%
\pgfsetdash{}{0pt}%
\pgfpathmoveto{\pgfqpoint{4.442129in}{0.555556in}}%
\pgfpathlineto{\pgfqpoint{4.679609in}{0.555556in}}%
\pgfpathlineto{\pgfqpoint{4.679609in}{1.095591in}}%
\pgfpathlineto{\pgfqpoint{4.442129in}{1.095591in}}%
\pgfpathlineto{\pgfqpoint{4.442129in}{0.555556in}}%
\pgfpathclose%
\pgfusepath{stroke,fill}%
\end{pgfscope}%
\begin{pgfscope}%
\pgfpathrectangle{\pgfqpoint{0.868056in}{0.555556in}}{\pgfqpoint{3.993056in}{1.888889in}}%
\pgfusepath{clip}%
\pgfsetbuttcap%
\pgfsetmiterjoin%
\definecolor{currentfill}{rgb}{0.235294,0.490196,0.764706}%
\pgfsetfillcolor{currentfill}%
\pgfsetlinewidth{1.003750pt}%
\definecolor{currentstroke}{rgb}{0.266667,0.266667,0.266667}%
\pgfsetstrokecolor{currentstroke}%
\pgfsetdash{}{0pt}%
\pgfpathmoveto{\pgfqpoint{1.049558in}{0.710221in}}%
\pgfpathlineto{\pgfqpoint{1.287038in}{0.710221in}}%
\pgfpathlineto{\pgfqpoint{1.287038in}{1.003425in}}%
\pgfpathlineto{\pgfqpoint{1.049558in}{1.003425in}}%
\pgfpathlineto{\pgfqpoint{1.049558in}{0.710221in}}%
\pgfpathclose%
\pgfusepath{stroke,fill}%
\end{pgfscope}%
\begin{pgfscope}%
\pgfpathrectangle{\pgfqpoint{0.868056in}{0.555556in}}{\pgfqpoint{3.993056in}{1.888889in}}%
\pgfusepath{clip}%
\pgfsetbuttcap%
\pgfsetmiterjoin%
\definecolor{currentfill}{rgb}{0.235294,0.490196,0.764706}%
\pgfsetfillcolor{currentfill}%
\pgfsetlinewidth{1.003750pt}%
\definecolor{currentstroke}{rgb}{0.266667,0.266667,0.266667}%
\pgfsetstrokecolor{currentstroke}%
\pgfsetdash{}{0pt}%
\pgfpathmoveto{\pgfqpoint{1.388815in}{0.670535in}}%
\pgfpathlineto{\pgfqpoint{1.626295in}{0.670535in}}%
\pgfpathlineto{\pgfqpoint{1.626295in}{0.877670in}}%
\pgfpathlineto{\pgfqpoint{1.388815in}{0.877670in}}%
\pgfpathlineto{\pgfqpoint{1.388815in}{0.670535in}}%
\pgfpathclose%
\pgfusepath{stroke,fill}%
\end{pgfscope}%
\begin{pgfscope}%
\pgfpathrectangle{\pgfqpoint{0.868056in}{0.555556in}}{\pgfqpoint{3.993056in}{1.888889in}}%
\pgfusepath{clip}%
\pgfsetbuttcap%
\pgfsetmiterjoin%
\definecolor{currentfill}{rgb}{0.235294,0.490196,0.764706}%
\pgfsetfillcolor{currentfill}%
\pgfsetlinewidth{1.003750pt}%
\definecolor{currentstroke}{rgb}{0.266667,0.266667,0.266667}%
\pgfsetstrokecolor{currentstroke}%
\pgfsetdash{}{0pt}%
\pgfpathmoveto{\pgfqpoint{1.728072in}{0.631284in}}%
\pgfpathlineto{\pgfqpoint{1.965552in}{0.631284in}}%
\pgfpathlineto{\pgfqpoint{1.965552in}{0.770040in}}%
\pgfpathlineto{\pgfqpoint{1.728072in}{0.770040in}}%
\pgfpathlineto{\pgfqpoint{1.728072in}{0.631284in}}%
\pgfpathclose%
\pgfusepath{stroke,fill}%
\end{pgfscope}%
\begin{pgfscope}%
\pgfpathrectangle{\pgfqpoint{0.868056in}{0.555556in}}{\pgfqpoint{3.993056in}{1.888889in}}%
\pgfusepath{clip}%
\pgfsetbuttcap%
\pgfsetmiterjoin%
\definecolor{currentfill}{rgb}{0.235294,0.490196,0.764706}%
\pgfsetfillcolor{currentfill}%
\pgfsetlinewidth{1.003750pt}%
\definecolor{currentstroke}{rgb}{0.266667,0.266667,0.266667}%
\pgfsetstrokecolor{currentstroke}%
\pgfsetdash{}{0pt}%
\pgfpathmoveto{\pgfqpoint{2.067329in}{0.686952in}}%
\pgfpathlineto{\pgfqpoint{2.304809in}{0.686952in}}%
\pgfpathlineto{\pgfqpoint{2.304809in}{0.868209in}}%
\pgfpathlineto{\pgfqpoint{2.067329in}{0.868209in}}%
\pgfpathlineto{\pgfqpoint{2.067329in}{0.686952in}}%
\pgfpathclose%
\pgfusepath{stroke,fill}%
\end{pgfscope}%
\begin{pgfscope}%
\pgfpathrectangle{\pgfqpoint{0.868056in}{0.555556in}}{\pgfqpoint{3.993056in}{1.888889in}}%
\pgfusepath{clip}%
\pgfsetbuttcap%
\pgfsetmiterjoin%
\definecolor{currentfill}{rgb}{0.235294,0.490196,0.764706}%
\pgfsetfillcolor{currentfill}%
\pgfsetlinewidth{1.003750pt}%
\definecolor{currentstroke}{rgb}{0.266667,0.266667,0.266667}%
\pgfsetstrokecolor{currentstroke}%
\pgfsetdash{}{0pt}%
\pgfpathmoveto{\pgfqpoint{2.406586in}{0.746574in}}%
\pgfpathlineto{\pgfqpoint{2.644066in}{0.746574in}}%
\pgfpathlineto{\pgfqpoint{2.644066in}{0.940926in}}%
\pgfpathlineto{\pgfqpoint{2.406586in}{0.940926in}}%
\pgfpathlineto{\pgfqpoint{2.406586in}{0.746574in}}%
\pgfpathclose%
\pgfusepath{stroke,fill}%
\end{pgfscope}%
\begin{pgfscope}%
\pgfpathrectangle{\pgfqpoint{0.868056in}{0.555556in}}{\pgfqpoint{3.993056in}{1.888889in}}%
\pgfusepath{clip}%
\pgfsetbuttcap%
\pgfsetmiterjoin%
\definecolor{currentfill}{rgb}{0.235294,0.490196,0.764706}%
\pgfsetfillcolor{currentfill}%
\pgfsetlinewidth{1.003750pt}%
\definecolor{currentstroke}{rgb}{0.266667,0.266667,0.266667}%
\pgfsetstrokecolor{currentstroke}%
\pgfsetdash{}{0pt}%
\pgfpathmoveto{\pgfqpoint{2.745843in}{0.811713in}}%
\pgfpathlineto{\pgfqpoint{2.983323in}{0.811713in}}%
\pgfpathlineto{\pgfqpoint{2.983323in}{1.031818in}}%
\pgfpathlineto{\pgfqpoint{2.745843in}{1.031818in}}%
\pgfpathlineto{\pgfqpoint{2.745843in}{0.811713in}}%
\pgfpathclose%
\pgfusepath{stroke,fill}%
\end{pgfscope}%
\begin{pgfscope}%
\pgfpathrectangle{\pgfqpoint{0.868056in}{0.555556in}}{\pgfqpoint{3.993056in}{1.888889in}}%
\pgfusepath{clip}%
\pgfsetbuttcap%
\pgfsetmiterjoin%
\definecolor{currentfill}{rgb}{0.235294,0.490196,0.764706}%
\pgfsetfillcolor{currentfill}%
\pgfsetlinewidth{1.003750pt}%
\definecolor{currentstroke}{rgb}{0.266667,0.266667,0.266667}%
\pgfsetstrokecolor{currentstroke}%
\pgfsetdash{}{0pt}%
\pgfpathmoveto{\pgfqpoint{3.085100in}{0.720614in}}%
\pgfpathlineto{\pgfqpoint{3.322580in}{0.720614in}}%
\pgfpathlineto{\pgfqpoint{3.322580in}{0.899252in}}%
\pgfpathlineto{\pgfqpoint{3.085100in}{0.899252in}}%
\pgfpathlineto{\pgfqpoint{3.085100in}{0.720614in}}%
\pgfpathclose%
\pgfusepath{stroke,fill}%
\end{pgfscope}%
\begin{pgfscope}%
\pgfpathrectangle{\pgfqpoint{0.868056in}{0.555556in}}{\pgfqpoint{3.993056in}{1.888889in}}%
\pgfusepath{clip}%
\pgfsetbuttcap%
\pgfsetmiterjoin%
\definecolor{currentfill}{rgb}{0.235294,0.490196,0.764706}%
\pgfsetfillcolor{currentfill}%
\pgfsetlinewidth{1.003750pt}%
\definecolor{currentstroke}{rgb}{0.266667,0.266667,0.266667}%
\pgfsetstrokecolor{currentstroke}%
\pgfsetdash{}{0pt}%
\pgfpathmoveto{\pgfqpoint{3.424357in}{0.808618in}}%
\pgfpathlineto{\pgfqpoint{3.661837in}{0.808618in}}%
\pgfpathlineto{\pgfqpoint{3.661837in}{1.067198in}}%
\pgfpathlineto{\pgfqpoint{3.424357in}{1.067198in}}%
\pgfpathlineto{\pgfqpoint{3.424357in}{0.808618in}}%
\pgfpathclose%
\pgfusepath{stroke,fill}%
\end{pgfscope}%
\begin{pgfscope}%
\pgfpathrectangle{\pgfqpoint{0.868056in}{0.555556in}}{\pgfqpoint{3.993056in}{1.888889in}}%
\pgfusepath{clip}%
\pgfsetbuttcap%
\pgfsetmiterjoin%
\definecolor{currentfill}{rgb}{0.235294,0.490196,0.764706}%
\pgfsetfillcolor{currentfill}%
\pgfsetlinewidth{1.003750pt}%
\definecolor{currentstroke}{rgb}{0.266667,0.266667,0.266667}%
\pgfsetstrokecolor{currentstroke}%
\pgfsetdash{}{0pt}%
\pgfpathmoveto{\pgfqpoint{3.763615in}{0.916394in}}%
\pgfpathlineto{\pgfqpoint{4.001094in}{0.916394in}}%
\pgfpathlineto{\pgfqpoint{4.001094in}{1.174187in}}%
\pgfpathlineto{\pgfqpoint{3.763615in}{1.174187in}}%
\pgfpathlineto{\pgfqpoint{3.763615in}{0.916394in}}%
\pgfpathclose%
\pgfusepath{stroke,fill}%
\end{pgfscope}%
\begin{pgfscope}%
\pgfpathrectangle{\pgfqpoint{0.868056in}{0.555556in}}{\pgfqpoint{3.993056in}{1.888889in}}%
\pgfusepath{clip}%
\pgfsetbuttcap%
\pgfsetmiterjoin%
\definecolor{currentfill}{rgb}{0.235294,0.490196,0.764706}%
\pgfsetfillcolor{currentfill}%
\pgfsetlinewidth{1.003750pt}%
\definecolor{currentstroke}{rgb}{0.266667,0.266667,0.266667}%
\pgfsetstrokecolor{currentstroke}%
\pgfsetdash{}{0pt}%
\pgfpathmoveto{\pgfqpoint{4.102872in}{0.965023in}}%
\pgfpathlineto{\pgfqpoint{4.340352in}{0.965023in}}%
\pgfpathlineto{\pgfqpoint{4.340352in}{1.201545in}}%
\pgfpathlineto{\pgfqpoint{4.102872in}{1.201545in}}%
\pgfpathlineto{\pgfqpoint{4.102872in}{0.965023in}}%
\pgfpathclose%
\pgfusepath{stroke,fill}%
\end{pgfscope}%
\begin{pgfscope}%
\pgfpathrectangle{\pgfqpoint{0.868056in}{0.555556in}}{\pgfqpoint{3.993056in}{1.888889in}}%
\pgfusepath{clip}%
\pgfsetbuttcap%
\pgfsetmiterjoin%
\definecolor{currentfill}{rgb}{0.235294,0.490196,0.764706}%
\pgfsetfillcolor{currentfill}%
\pgfsetlinewidth{1.003750pt}%
\definecolor{currentstroke}{rgb}{0.266667,0.266667,0.266667}%
\pgfsetstrokecolor{currentstroke}%
\pgfsetdash{}{0pt}%
\pgfpathmoveto{\pgfqpoint{4.442129in}{1.095591in}}%
\pgfpathlineto{\pgfqpoint{4.679609in}{1.095591in}}%
\pgfpathlineto{\pgfqpoint{4.679609in}{1.352608in}}%
\pgfpathlineto{\pgfqpoint{4.442129in}{1.352608in}}%
\pgfpathlineto{\pgfqpoint{4.442129in}{1.095591in}}%
\pgfpathclose%
\pgfusepath{stroke,fill}%
\end{pgfscope}%
\begin{pgfscope}%
\pgfpathrectangle{\pgfqpoint{0.868056in}{0.555556in}}{\pgfqpoint{3.993056in}{1.888889in}}%
\pgfusepath{clip}%
\pgfsetbuttcap%
\pgfsetmiterjoin%
\definecolor{currentfill}{rgb}{0.725490,0.486275,0.164706}%
\pgfsetfillcolor{currentfill}%
\pgfsetlinewidth{1.003750pt}%
\definecolor{currentstroke}{rgb}{0.266667,0.266667,0.266667}%
\pgfsetstrokecolor{currentstroke}%
\pgfsetdash{}{0pt}%
\pgfpathmoveto{\pgfqpoint{1.049558in}{1.003425in}}%
\pgfpathlineto{\pgfqpoint{1.287038in}{1.003425in}}%
\pgfpathlineto{\pgfqpoint{1.287038in}{1.215819in}}%
\pgfpathlineto{\pgfqpoint{1.049558in}{1.215819in}}%
\pgfpathlineto{\pgfqpoint{1.049558in}{1.003425in}}%
\pgfpathclose%
\pgfusepath{stroke,fill}%
\end{pgfscope}%
\begin{pgfscope}%
\pgfpathrectangle{\pgfqpoint{0.868056in}{0.555556in}}{\pgfqpoint{3.993056in}{1.888889in}}%
\pgfusepath{clip}%
\pgfsetbuttcap%
\pgfsetmiterjoin%
\definecolor{currentfill}{rgb}{0.725490,0.486275,0.164706}%
\pgfsetfillcolor{currentfill}%
\pgfsetlinewidth{1.003750pt}%
\definecolor{currentstroke}{rgb}{0.266667,0.266667,0.266667}%
\pgfsetstrokecolor{currentstroke}%
\pgfsetdash{}{0pt}%
\pgfpathmoveto{\pgfqpoint{1.388815in}{0.877670in}}%
\pgfpathlineto{\pgfqpoint{1.626295in}{0.877670in}}%
\pgfpathlineto{\pgfqpoint{1.626295in}{1.087434in}}%
\pgfpathlineto{\pgfqpoint{1.388815in}{1.087434in}}%
\pgfpathlineto{\pgfqpoint{1.388815in}{0.877670in}}%
\pgfpathclose%
\pgfusepath{stroke,fill}%
\end{pgfscope}%
\begin{pgfscope}%
\pgfpathrectangle{\pgfqpoint{0.868056in}{0.555556in}}{\pgfqpoint{3.993056in}{1.888889in}}%
\pgfusepath{clip}%
\pgfsetbuttcap%
\pgfsetmiterjoin%
\definecolor{currentfill}{rgb}{0.725490,0.486275,0.164706}%
\pgfsetfillcolor{currentfill}%
\pgfsetlinewidth{1.003750pt}%
\definecolor{currentstroke}{rgb}{0.266667,0.266667,0.266667}%
\pgfsetstrokecolor{currentstroke}%
\pgfsetdash{}{0pt}%
\pgfpathmoveto{\pgfqpoint{1.728072in}{0.770040in}}%
\pgfpathlineto{\pgfqpoint{1.965552in}{0.770040in}}%
\pgfpathlineto{\pgfqpoint{1.965552in}{0.937354in}}%
\pgfpathlineto{\pgfqpoint{1.728072in}{0.937354in}}%
\pgfpathlineto{\pgfqpoint{1.728072in}{0.770040in}}%
\pgfpathclose%
\pgfusepath{stroke,fill}%
\end{pgfscope}%
\begin{pgfscope}%
\pgfpathrectangle{\pgfqpoint{0.868056in}{0.555556in}}{\pgfqpoint{3.993056in}{1.888889in}}%
\pgfusepath{clip}%
\pgfsetbuttcap%
\pgfsetmiterjoin%
\definecolor{currentfill}{rgb}{0.725490,0.486275,0.164706}%
\pgfsetfillcolor{currentfill}%
\pgfsetlinewidth{1.003750pt}%
\definecolor{currentstroke}{rgb}{0.266667,0.266667,0.266667}%
\pgfsetstrokecolor{currentstroke}%
\pgfsetdash{}{0pt}%
\pgfpathmoveto{\pgfqpoint{2.067329in}{0.868209in}}%
\pgfpathlineto{\pgfqpoint{2.304809in}{0.868209in}}%
\pgfpathlineto{\pgfqpoint{2.304809in}{1.037449in}}%
\pgfpathlineto{\pgfqpoint{2.067329in}{1.037449in}}%
\pgfpathlineto{\pgfqpoint{2.067329in}{0.868209in}}%
\pgfpathclose%
\pgfusepath{stroke,fill}%
\end{pgfscope}%
\begin{pgfscope}%
\pgfpathrectangle{\pgfqpoint{0.868056in}{0.555556in}}{\pgfqpoint{3.993056in}{1.888889in}}%
\pgfusepath{clip}%
\pgfsetbuttcap%
\pgfsetmiterjoin%
\definecolor{currentfill}{rgb}{0.725490,0.486275,0.164706}%
\pgfsetfillcolor{currentfill}%
\pgfsetlinewidth{1.003750pt}%
\definecolor{currentstroke}{rgb}{0.266667,0.266667,0.266667}%
\pgfsetstrokecolor{currentstroke}%
\pgfsetdash{}{0pt}%
\pgfpathmoveto{\pgfqpoint{2.406586in}{0.940926in}}%
\pgfpathlineto{\pgfqpoint{2.644066in}{0.940926in}}%
\pgfpathlineto{\pgfqpoint{2.644066in}{1.105476in}}%
\pgfpathlineto{\pgfqpoint{2.406586in}{1.105476in}}%
\pgfpathlineto{\pgfqpoint{2.406586in}{0.940926in}}%
\pgfpathclose%
\pgfusepath{stroke,fill}%
\end{pgfscope}%
\begin{pgfscope}%
\pgfpathrectangle{\pgfqpoint{0.868056in}{0.555556in}}{\pgfqpoint{3.993056in}{1.888889in}}%
\pgfusepath{clip}%
\pgfsetbuttcap%
\pgfsetmiterjoin%
\definecolor{currentfill}{rgb}{0.725490,0.486275,0.164706}%
\pgfsetfillcolor{currentfill}%
\pgfsetlinewidth{1.003750pt}%
\definecolor{currentstroke}{rgb}{0.266667,0.266667,0.266667}%
\pgfsetstrokecolor{currentstroke}%
\pgfsetdash{}{0pt}%
\pgfpathmoveto{\pgfqpoint{2.745843in}{1.031818in}}%
\pgfpathlineto{\pgfqpoint{2.983323in}{1.031818in}}%
\pgfpathlineto{\pgfqpoint{2.983323in}{1.207310in}}%
\pgfpathlineto{\pgfqpoint{2.745843in}{1.207310in}}%
\pgfpathlineto{\pgfqpoint{2.745843in}{1.031818in}}%
\pgfpathclose%
\pgfusepath{stroke,fill}%
\end{pgfscope}%
\begin{pgfscope}%
\pgfpathrectangle{\pgfqpoint{0.868056in}{0.555556in}}{\pgfqpoint{3.993056in}{1.888889in}}%
\pgfusepath{clip}%
\pgfsetbuttcap%
\pgfsetmiterjoin%
\definecolor{currentfill}{rgb}{0.725490,0.486275,0.164706}%
\pgfsetfillcolor{currentfill}%
\pgfsetlinewidth{1.003750pt}%
\definecolor{currentstroke}{rgb}{0.266667,0.266667,0.266667}%
\pgfsetstrokecolor{currentstroke}%
\pgfsetdash{}{0pt}%
\pgfpathmoveto{\pgfqpoint{3.085100in}{0.899252in}}%
\pgfpathlineto{\pgfqpoint{3.322580in}{0.899252in}}%
\pgfpathlineto{\pgfqpoint{3.322580in}{1.026974in}}%
\pgfpathlineto{\pgfqpoint{3.085100in}{1.026974in}}%
\pgfpathlineto{\pgfqpoint{3.085100in}{0.899252in}}%
\pgfpathclose%
\pgfusepath{stroke,fill}%
\end{pgfscope}%
\begin{pgfscope}%
\pgfpathrectangle{\pgfqpoint{0.868056in}{0.555556in}}{\pgfqpoint{3.993056in}{1.888889in}}%
\pgfusepath{clip}%
\pgfsetbuttcap%
\pgfsetmiterjoin%
\definecolor{currentfill}{rgb}{0.725490,0.486275,0.164706}%
\pgfsetfillcolor{currentfill}%
\pgfsetlinewidth{1.003750pt}%
\definecolor{currentstroke}{rgb}{0.266667,0.266667,0.266667}%
\pgfsetstrokecolor{currentstroke}%
\pgfsetdash{}{0pt}%
\pgfpathmoveto{\pgfqpoint{3.424357in}{1.067198in}}%
\pgfpathlineto{\pgfqpoint{3.661837in}{1.067198in}}%
\pgfpathlineto{\pgfqpoint{3.661837in}{1.231966in}}%
\pgfpathlineto{\pgfqpoint{3.424357in}{1.231966in}}%
\pgfpathlineto{\pgfqpoint{3.424357in}{1.067198in}}%
\pgfpathclose%
\pgfusepath{stroke,fill}%
\end{pgfscope}%
\begin{pgfscope}%
\pgfpathrectangle{\pgfqpoint{0.868056in}{0.555556in}}{\pgfqpoint{3.993056in}{1.888889in}}%
\pgfusepath{clip}%
\pgfsetbuttcap%
\pgfsetmiterjoin%
\definecolor{currentfill}{rgb}{0.725490,0.486275,0.164706}%
\pgfsetfillcolor{currentfill}%
\pgfsetlinewidth{1.003750pt}%
\definecolor{currentstroke}{rgb}{0.266667,0.266667,0.266667}%
\pgfsetstrokecolor{currentstroke}%
\pgfsetdash{}{0pt}%
\pgfpathmoveto{\pgfqpoint{3.763615in}{1.174187in}}%
\pgfpathlineto{\pgfqpoint{4.001094in}{1.174187in}}%
\pgfpathlineto{\pgfqpoint{4.001094in}{1.388184in}}%
\pgfpathlineto{\pgfqpoint{3.763615in}{1.388184in}}%
\pgfpathlineto{\pgfqpoint{3.763615in}{1.174187in}}%
\pgfpathclose%
\pgfusepath{stroke,fill}%
\end{pgfscope}%
\begin{pgfscope}%
\pgfpathrectangle{\pgfqpoint{0.868056in}{0.555556in}}{\pgfqpoint{3.993056in}{1.888889in}}%
\pgfusepath{clip}%
\pgfsetbuttcap%
\pgfsetmiterjoin%
\definecolor{currentfill}{rgb}{0.725490,0.486275,0.164706}%
\pgfsetfillcolor{currentfill}%
\pgfsetlinewidth{1.003750pt}%
\definecolor{currentstroke}{rgb}{0.266667,0.266667,0.266667}%
\pgfsetstrokecolor{currentstroke}%
\pgfsetdash{}{0pt}%
\pgfpathmoveto{\pgfqpoint{4.102872in}{1.201545in}}%
\pgfpathlineto{\pgfqpoint{4.340352in}{1.201545in}}%
\pgfpathlineto{\pgfqpoint{4.340352in}{1.414010in}}%
\pgfpathlineto{\pgfqpoint{4.102872in}{1.414010in}}%
\pgfpathlineto{\pgfqpoint{4.102872in}{1.201545in}}%
\pgfpathclose%
\pgfusepath{stroke,fill}%
\end{pgfscope}%
\begin{pgfscope}%
\pgfpathrectangle{\pgfqpoint{0.868056in}{0.555556in}}{\pgfqpoint{3.993056in}{1.888889in}}%
\pgfusepath{clip}%
\pgfsetbuttcap%
\pgfsetmiterjoin%
\definecolor{currentfill}{rgb}{0.725490,0.486275,0.164706}%
\pgfsetfillcolor{currentfill}%
\pgfsetlinewidth{1.003750pt}%
\definecolor{currentstroke}{rgb}{0.266667,0.266667,0.266667}%
\pgfsetstrokecolor{currentstroke}%
\pgfsetdash{}{0pt}%
\pgfpathmoveto{\pgfqpoint{4.442129in}{1.352608in}}%
\pgfpathlineto{\pgfqpoint{4.679609in}{1.352608in}}%
\pgfpathlineto{\pgfqpoint{4.679609in}{1.580838in}}%
\pgfpathlineto{\pgfqpoint{4.442129in}{1.580838in}}%
\pgfpathlineto{\pgfqpoint{4.442129in}{1.352608in}}%
\pgfpathclose%
\pgfusepath{stroke,fill}%
\end{pgfscope}%
\begin{pgfscope}%
\pgfpathrectangle{\pgfqpoint{0.868056in}{0.555556in}}{\pgfqpoint{3.993056in}{1.888889in}}%
\pgfusepath{clip}%
\pgfsetbuttcap%
\pgfsetmiterjoin%
\definecolor{currentfill}{rgb}{0.733333,0.321569,0.733333}%
\pgfsetfillcolor{currentfill}%
\pgfsetlinewidth{1.003750pt}%
\definecolor{currentstroke}{rgb}{0.266667,0.266667,0.266667}%
\pgfsetstrokecolor{currentstroke}%
\pgfsetdash{}{0pt}%
\pgfpathmoveto{\pgfqpoint{1.049558in}{1.215819in}}%
\pgfpathlineto{\pgfqpoint{1.287038in}{1.215819in}}%
\pgfpathlineto{\pgfqpoint{1.287038in}{1.368497in}}%
\pgfpathlineto{\pgfqpoint{1.049558in}{1.368497in}}%
\pgfpathlineto{\pgfqpoint{1.049558in}{1.215819in}}%
\pgfpathclose%
\pgfusepath{stroke,fill}%
\end{pgfscope}%
\begin{pgfscope}%
\pgfpathrectangle{\pgfqpoint{0.868056in}{0.555556in}}{\pgfqpoint{3.993056in}{1.888889in}}%
\pgfusepath{clip}%
\pgfsetbuttcap%
\pgfsetmiterjoin%
\definecolor{currentfill}{rgb}{0.733333,0.321569,0.733333}%
\pgfsetfillcolor{currentfill}%
\pgfsetlinewidth{1.003750pt}%
\definecolor{currentstroke}{rgb}{0.266667,0.266667,0.266667}%
\pgfsetstrokecolor{currentstroke}%
\pgfsetdash{}{0pt}%
\pgfpathmoveto{\pgfqpoint{1.388815in}{1.087434in}}%
\pgfpathlineto{\pgfqpoint{1.626295in}{1.087434in}}%
\pgfpathlineto{\pgfqpoint{1.626295in}{1.193729in}}%
\pgfpathlineto{\pgfqpoint{1.388815in}{1.193729in}}%
\pgfpathlineto{\pgfqpoint{1.388815in}{1.087434in}}%
\pgfpathclose%
\pgfusepath{stroke,fill}%
\end{pgfscope}%
\begin{pgfscope}%
\pgfpathrectangle{\pgfqpoint{0.868056in}{0.555556in}}{\pgfqpoint{3.993056in}{1.888889in}}%
\pgfusepath{clip}%
\pgfsetbuttcap%
\pgfsetmiterjoin%
\definecolor{currentfill}{rgb}{0.733333,0.321569,0.733333}%
\pgfsetfillcolor{currentfill}%
\pgfsetlinewidth{1.003750pt}%
\definecolor{currentstroke}{rgb}{0.266667,0.266667,0.266667}%
\pgfsetstrokecolor{currentstroke}%
\pgfsetdash{}{0pt}%
\pgfpathmoveto{\pgfqpoint{1.728072in}{0.937354in}}%
\pgfpathlineto{\pgfqpoint{1.965552in}{0.937354in}}%
\pgfpathlineto{\pgfqpoint{1.965552in}{1.024873in}}%
\pgfpathlineto{\pgfqpoint{1.728072in}{1.024873in}}%
\pgfpathlineto{\pgfqpoint{1.728072in}{0.937354in}}%
\pgfpathclose%
\pgfusepath{stroke,fill}%
\end{pgfscope}%
\begin{pgfscope}%
\pgfpathrectangle{\pgfqpoint{0.868056in}{0.555556in}}{\pgfqpoint{3.993056in}{1.888889in}}%
\pgfusepath{clip}%
\pgfsetbuttcap%
\pgfsetmiterjoin%
\definecolor{currentfill}{rgb}{0.733333,0.321569,0.733333}%
\pgfsetfillcolor{currentfill}%
\pgfsetlinewidth{1.003750pt}%
\definecolor{currentstroke}{rgb}{0.266667,0.266667,0.266667}%
\pgfsetstrokecolor{currentstroke}%
\pgfsetdash{}{0pt}%
\pgfpathmoveto{\pgfqpoint{2.067329in}{1.037449in}}%
\pgfpathlineto{\pgfqpoint{2.304809in}{1.037449in}}%
\pgfpathlineto{\pgfqpoint{2.304809in}{1.134107in}}%
\pgfpathlineto{\pgfqpoint{2.067329in}{1.134107in}}%
\pgfpathlineto{\pgfqpoint{2.067329in}{1.037449in}}%
\pgfpathclose%
\pgfusepath{stroke,fill}%
\end{pgfscope}%
\begin{pgfscope}%
\pgfpathrectangle{\pgfqpoint{0.868056in}{0.555556in}}{\pgfqpoint{3.993056in}{1.888889in}}%
\pgfusepath{clip}%
\pgfsetbuttcap%
\pgfsetmiterjoin%
\definecolor{currentfill}{rgb}{0.733333,0.321569,0.733333}%
\pgfsetfillcolor{currentfill}%
\pgfsetlinewidth{1.003750pt}%
\definecolor{currentstroke}{rgb}{0.266667,0.266667,0.266667}%
\pgfsetstrokecolor{currentstroke}%
\pgfsetdash{}{0pt}%
\pgfpathmoveto{\pgfqpoint{2.406586in}{1.105476in}}%
\pgfpathlineto{\pgfqpoint{2.644066in}{1.105476in}}%
\pgfpathlineto{\pgfqpoint{2.644066in}{1.205757in}}%
\pgfpathlineto{\pgfqpoint{2.406586in}{1.205757in}}%
\pgfpathlineto{\pgfqpoint{2.406586in}{1.105476in}}%
\pgfpathclose%
\pgfusepath{stroke,fill}%
\end{pgfscope}%
\begin{pgfscope}%
\pgfpathrectangle{\pgfqpoint{0.868056in}{0.555556in}}{\pgfqpoint{3.993056in}{1.888889in}}%
\pgfusepath{clip}%
\pgfsetbuttcap%
\pgfsetmiterjoin%
\definecolor{currentfill}{rgb}{0.733333,0.321569,0.733333}%
\pgfsetfillcolor{currentfill}%
\pgfsetlinewidth{1.003750pt}%
\definecolor{currentstroke}{rgb}{0.266667,0.266667,0.266667}%
\pgfsetstrokecolor{currentstroke}%
\pgfsetdash{}{0pt}%
\pgfpathmoveto{\pgfqpoint{2.745843in}{1.207310in}}%
\pgfpathlineto{\pgfqpoint{2.983323in}{1.207310in}}%
\pgfpathlineto{\pgfqpoint{2.983323in}{1.328552in}}%
\pgfpathlineto{\pgfqpoint{2.745843in}{1.328552in}}%
\pgfpathlineto{\pgfqpoint{2.745843in}{1.207310in}}%
\pgfpathclose%
\pgfusepath{stroke,fill}%
\end{pgfscope}%
\begin{pgfscope}%
\pgfpathrectangle{\pgfqpoint{0.868056in}{0.555556in}}{\pgfqpoint{3.993056in}{1.888889in}}%
\pgfusepath{clip}%
\pgfsetbuttcap%
\pgfsetmiterjoin%
\definecolor{currentfill}{rgb}{0.733333,0.321569,0.733333}%
\pgfsetfillcolor{currentfill}%
\pgfsetlinewidth{1.003750pt}%
\definecolor{currentstroke}{rgb}{0.266667,0.266667,0.266667}%
\pgfsetstrokecolor{currentstroke}%
\pgfsetdash{}{0pt}%
\pgfpathmoveto{\pgfqpoint{3.085100in}{1.026974in}}%
\pgfpathlineto{\pgfqpoint{3.322580in}{1.026974in}}%
\pgfpathlineto{\pgfqpoint{3.322580in}{1.132834in}}%
\pgfpathlineto{\pgfqpoint{3.085100in}{1.132834in}}%
\pgfpathlineto{\pgfqpoint{3.085100in}{1.026974in}}%
\pgfpathclose%
\pgfusepath{stroke,fill}%
\end{pgfscope}%
\begin{pgfscope}%
\pgfpathrectangle{\pgfqpoint{0.868056in}{0.555556in}}{\pgfqpoint{3.993056in}{1.888889in}}%
\pgfusepath{clip}%
\pgfsetbuttcap%
\pgfsetmiterjoin%
\definecolor{currentfill}{rgb}{0.733333,0.321569,0.733333}%
\pgfsetfillcolor{currentfill}%
\pgfsetlinewidth{1.003750pt}%
\definecolor{currentstroke}{rgb}{0.266667,0.266667,0.266667}%
\pgfsetstrokecolor{currentstroke}%
\pgfsetdash{}{0pt}%
\pgfpathmoveto{\pgfqpoint{3.424357in}{1.231966in}}%
\pgfpathlineto{\pgfqpoint{3.661837in}{1.231966in}}%
\pgfpathlineto{\pgfqpoint{3.661837in}{1.404270in}}%
\pgfpathlineto{\pgfqpoint{3.424357in}{1.404270in}}%
\pgfpathlineto{\pgfqpoint{3.424357in}{1.231966in}}%
\pgfpathclose%
\pgfusepath{stroke,fill}%
\end{pgfscope}%
\begin{pgfscope}%
\pgfpathrectangle{\pgfqpoint{0.868056in}{0.555556in}}{\pgfqpoint{3.993056in}{1.888889in}}%
\pgfusepath{clip}%
\pgfsetbuttcap%
\pgfsetmiterjoin%
\definecolor{currentfill}{rgb}{0.733333,0.321569,0.733333}%
\pgfsetfillcolor{currentfill}%
\pgfsetlinewidth{1.003750pt}%
\definecolor{currentstroke}{rgb}{0.266667,0.266667,0.266667}%
\pgfsetstrokecolor{currentstroke}%
\pgfsetdash{}{0pt}%
\pgfpathmoveto{\pgfqpoint{3.763615in}{1.388184in}}%
\pgfpathlineto{\pgfqpoint{4.001094in}{1.388184in}}%
\pgfpathlineto{\pgfqpoint{4.001094in}{1.577909in}}%
\pgfpathlineto{\pgfqpoint{3.763615in}{1.577909in}}%
\pgfpathlineto{\pgfqpoint{3.763615in}{1.388184in}}%
\pgfpathclose%
\pgfusepath{stroke,fill}%
\end{pgfscope}%
\begin{pgfscope}%
\pgfpathrectangle{\pgfqpoint{0.868056in}{0.555556in}}{\pgfqpoint{3.993056in}{1.888889in}}%
\pgfusepath{clip}%
\pgfsetbuttcap%
\pgfsetmiterjoin%
\definecolor{currentfill}{rgb}{0.733333,0.321569,0.733333}%
\pgfsetfillcolor{currentfill}%
\pgfsetlinewidth{1.003750pt}%
\definecolor{currentstroke}{rgb}{0.266667,0.266667,0.266667}%
\pgfsetstrokecolor{currentstroke}%
\pgfsetdash{}{0pt}%
\pgfpathmoveto{\pgfqpoint{4.102872in}{1.414010in}}%
\pgfpathlineto{\pgfqpoint{4.340352in}{1.414010in}}%
\pgfpathlineto{\pgfqpoint{4.340352in}{1.615100in}}%
\pgfpathlineto{\pgfqpoint{4.102872in}{1.615100in}}%
\pgfpathlineto{\pgfqpoint{4.102872in}{1.414010in}}%
\pgfpathclose%
\pgfusepath{stroke,fill}%
\end{pgfscope}%
\begin{pgfscope}%
\pgfpathrectangle{\pgfqpoint{0.868056in}{0.555556in}}{\pgfqpoint{3.993056in}{1.888889in}}%
\pgfusepath{clip}%
\pgfsetbuttcap%
\pgfsetmiterjoin%
\definecolor{currentfill}{rgb}{0.733333,0.321569,0.733333}%
\pgfsetfillcolor{currentfill}%
\pgfsetlinewidth{1.003750pt}%
\definecolor{currentstroke}{rgb}{0.266667,0.266667,0.266667}%
\pgfsetstrokecolor{currentstroke}%
\pgfsetdash{}{0pt}%
\pgfpathmoveto{\pgfqpoint{4.442129in}{1.580838in}}%
\pgfpathlineto{\pgfqpoint{4.679609in}{1.580838in}}%
\pgfpathlineto{\pgfqpoint{4.679609in}{1.788708in}}%
\pgfpathlineto{\pgfqpoint{4.442129in}{1.788708in}}%
\pgfpathlineto{\pgfqpoint{4.442129in}{1.580838in}}%
\pgfpathclose%
\pgfusepath{stroke,fill}%
\end{pgfscope}%
\begin{pgfscope}%
\pgfpathrectangle{\pgfqpoint{0.868056in}{0.555556in}}{\pgfqpoint{3.993056in}{1.888889in}}%
\pgfusepath{clip}%
\pgfsetbuttcap%
\pgfsetmiterjoin%
\definecolor{currentfill}{rgb}{0.549020,0.247059,0.121569}%
\pgfsetfillcolor{currentfill}%
\pgfsetlinewidth{1.003750pt}%
\definecolor{currentstroke}{rgb}{0.266667,0.266667,0.266667}%
\pgfsetstrokecolor{currentstroke}%
\pgfsetdash{}{0pt}%
\pgfpathmoveto{\pgfqpoint{1.049558in}{1.368497in}}%
\pgfpathlineto{\pgfqpoint{1.287038in}{1.368497in}}%
\pgfpathlineto{\pgfqpoint{1.287038in}{1.623071in}}%
\pgfpathlineto{\pgfqpoint{1.049558in}{1.623071in}}%
\pgfpathlineto{\pgfqpoint{1.049558in}{1.368497in}}%
\pgfpathclose%
\pgfusepath{stroke,fill}%
\end{pgfscope}%
\begin{pgfscope}%
\pgfpathrectangle{\pgfqpoint{0.868056in}{0.555556in}}{\pgfqpoint{3.993056in}{1.888889in}}%
\pgfusepath{clip}%
\pgfsetbuttcap%
\pgfsetmiterjoin%
\definecolor{currentfill}{rgb}{0.549020,0.247059,0.121569}%
\pgfsetfillcolor{currentfill}%
\pgfsetlinewidth{1.003750pt}%
\definecolor{currentstroke}{rgb}{0.266667,0.266667,0.266667}%
\pgfsetstrokecolor{currentstroke}%
\pgfsetdash{}{0pt}%
\pgfpathmoveto{\pgfqpoint{1.388815in}{1.193729in}}%
\pgfpathlineto{\pgfqpoint{1.626295in}{1.193729in}}%
\pgfpathlineto{\pgfqpoint{1.626295in}{1.369159in}}%
\pgfpathlineto{\pgfqpoint{1.388815in}{1.369159in}}%
\pgfpathlineto{\pgfqpoint{1.388815in}{1.193729in}}%
\pgfpathclose%
\pgfusepath{stroke,fill}%
\end{pgfscope}%
\begin{pgfscope}%
\pgfpathrectangle{\pgfqpoint{0.868056in}{0.555556in}}{\pgfqpoint{3.993056in}{1.888889in}}%
\pgfusepath{clip}%
\pgfsetbuttcap%
\pgfsetmiterjoin%
\definecolor{currentfill}{rgb}{0.549020,0.247059,0.121569}%
\pgfsetfillcolor{currentfill}%
\pgfsetlinewidth{1.003750pt}%
\definecolor{currentstroke}{rgb}{0.266667,0.266667,0.266667}%
\pgfsetstrokecolor{currentstroke}%
\pgfsetdash{}{0pt}%
\pgfpathmoveto{\pgfqpoint{1.728072in}{1.024873in}}%
\pgfpathlineto{\pgfqpoint{1.965552in}{1.024873in}}%
\pgfpathlineto{\pgfqpoint{1.965552in}{1.145411in}}%
\pgfpathlineto{\pgfqpoint{1.728072in}{1.145411in}}%
\pgfpathlineto{\pgfqpoint{1.728072in}{1.024873in}}%
\pgfpathclose%
\pgfusepath{stroke,fill}%
\end{pgfscope}%
\begin{pgfscope}%
\pgfpathrectangle{\pgfqpoint{0.868056in}{0.555556in}}{\pgfqpoint{3.993056in}{1.888889in}}%
\pgfusepath{clip}%
\pgfsetbuttcap%
\pgfsetmiterjoin%
\definecolor{currentfill}{rgb}{0.549020,0.247059,0.121569}%
\pgfsetfillcolor{currentfill}%
\pgfsetlinewidth{1.003750pt}%
\definecolor{currentstroke}{rgb}{0.266667,0.266667,0.266667}%
\pgfsetstrokecolor{currentstroke}%
\pgfsetdash{}{0pt}%
\pgfpathmoveto{\pgfqpoint{2.067329in}{1.134107in}}%
\pgfpathlineto{\pgfqpoint{2.304809in}{1.134107in}}%
\pgfpathlineto{\pgfqpoint{2.304809in}{1.266715in}}%
\pgfpathlineto{\pgfqpoint{2.067329in}{1.266715in}}%
\pgfpathlineto{\pgfqpoint{2.067329in}{1.134107in}}%
\pgfpathclose%
\pgfusepath{stroke,fill}%
\end{pgfscope}%
\begin{pgfscope}%
\pgfpathrectangle{\pgfqpoint{0.868056in}{0.555556in}}{\pgfqpoint{3.993056in}{1.888889in}}%
\pgfusepath{clip}%
\pgfsetbuttcap%
\pgfsetmiterjoin%
\definecolor{currentfill}{rgb}{0.549020,0.247059,0.121569}%
\pgfsetfillcolor{currentfill}%
\pgfsetlinewidth{1.003750pt}%
\definecolor{currentstroke}{rgb}{0.266667,0.266667,0.266667}%
\pgfsetstrokecolor{currentstroke}%
\pgfsetdash{}{0pt}%
\pgfpathmoveto{\pgfqpoint{2.406586in}{1.205757in}}%
\pgfpathlineto{\pgfqpoint{2.644066in}{1.205757in}}%
\pgfpathlineto{\pgfqpoint{2.644066in}{1.337671in}}%
\pgfpathlineto{\pgfqpoint{2.406586in}{1.337671in}}%
\pgfpathlineto{\pgfqpoint{2.406586in}{1.205757in}}%
\pgfpathclose%
\pgfusepath{stroke,fill}%
\end{pgfscope}%
\begin{pgfscope}%
\pgfpathrectangle{\pgfqpoint{0.868056in}{0.555556in}}{\pgfqpoint{3.993056in}{1.888889in}}%
\pgfusepath{clip}%
\pgfsetbuttcap%
\pgfsetmiterjoin%
\definecolor{currentfill}{rgb}{0.549020,0.247059,0.121569}%
\pgfsetfillcolor{currentfill}%
\pgfsetlinewidth{1.003750pt}%
\definecolor{currentstroke}{rgb}{0.266667,0.266667,0.266667}%
\pgfsetstrokecolor{currentstroke}%
\pgfsetdash{}{0pt}%
\pgfpathmoveto{\pgfqpoint{2.745843in}{1.328552in}}%
\pgfpathlineto{\pgfqpoint{2.983323in}{1.328552in}}%
\pgfpathlineto{\pgfqpoint{2.983323in}{1.502926in}}%
\pgfpathlineto{\pgfqpoint{2.745843in}{1.502926in}}%
\pgfpathlineto{\pgfqpoint{2.745843in}{1.328552in}}%
\pgfpathclose%
\pgfusepath{stroke,fill}%
\end{pgfscope}%
\begin{pgfscope}%
\pgfpathrectangle{\pgfqpoint{0.868056in}{0.555556in}}{\pgfqpoint{3.993056in}{1.888889in}}%
\pgfusepath{clip}%
\pgfsetbuttcap%
\pgfsetmiterjoin%
\definecolor{currentfill}{rgb}{0.549020,0.247059,0.121569}%
\pgfsetfillcolor{currentfill}%
\pgfsetlinewidth{1.003750pt}%
\definecolor{currentstroke}{rgb}{0.266667,0.266667,0.266667}%
\pgfsetstrokecolor{currentstroke}%
\pgfsetdash{}{0pt}%
\pgfpathmoveto{\pgfqpoint{3.085100in}{1.132834in}}%
\pgfpathlineto{\pgfqpoint{3.322580in}{1.132834in}}%
\pgfpathlineto{\pgfqpoint{3.322580in}{1.266239in}}%
\pgfpathlineto{\pgfqpoint{3.085100in}{1.266239in}}%
\pgfpathlineto{\pgfqpoint{3.085100in}{1.132834in}}%
\pgfpathclose%
\pgfusepath{stroke,fill}%
\end{pgfscope}%
\begin{pgfscope}%
\pgfpathrectangle{\pgfqpoint{0.868056in}{0.555556in}}{\pgfqpoint{3.993056in}{1.888889in}}%
\pgfusepath{clip}%
\pgfsetbuttcap%
\pgfsetmiterjoin%
\definecolor{currentfill}{rgb}{0.549020,0.247059,0.121569}%
\pgfsetfillcolor{currentfill}%
\pgfsetlinewidth{1.003750pt}%
\definecolor{currentstroke}{rgb}{0.266667,0.266667,0.266667}%
\pgfsetstrokecolor{currentstroke}%
\pgfsetdash{}{0pt}%
\pgfpathmoveto{\pgfqpoint{3.424357in}{1.404270in}}%
\pgfpathlineto{\pgfqpoint{3.661837in}{1.404270in}}%
\pgfpathlineto{\pgfqpoint{3.661837in}{1.517324in}}%
\pgfpathlineto{\pgfqpoint{3.424357in}{1.517324in}}%
\pgfpathlineto{\pgfqpoint{3.424357in}{1.404270in}}%
\pgfpathclose%
\pgfusepath{stroke,fill}%
\end{pgfscope}%
\begin{pgfscope}%
\pgfpathrectangle{\pgfqpoint{0.868056in}{0.555556in}}{\pgfqpoint{3.993056in}{1.888889in}}%
\pgfusepath{clip}%
\pgfsetbuttcap%
\pgfsetmiterjoin%
\definecolor{currentfill}{rgb}{0.549020,0.247059,0.121569}%
\pgfsetfillcolor{currentfill}%
\pgfsetlinewidth{1.003750pt}%
\definecolor{currentstroke}{rgb}{0.266667,0.266667,0.266667}%
\pgfsetstrokecolor{currentstroke}%
\pgfsetdash{}{0pt}%
\pgfpathmoveto{\pgfqpoint{3.763615in}{1.577909in}}%
\pgfpathlineto{\pgfqpoint{4.001094in}{1.577909in}}%
\pgfpathlineto{\pgfqpoint{4.001094in}{1.728341in}}%
\pgfpathlineto{\pgfqpoint{3.763615in}{1.728341in}}%
\pgfpathlineto{\pgfqpoint{3.763615in}{1.577909in}}%
\pgfpathclose%
\pgfusepath{stroke,fill}%
\end{pgfscope}%
\begin{pgfscope}%
\pgfpathrectangle{\pgfqpoint{0.868056in}{0.555556in}}{\pgfqpoint{3.993056in}{1.888889in}}%
\pgfusepath{clip}%
\pgfsetbuttcap%
\pgfsetmiterjoin%
\definecolor{currentfill}{rgb}{0.549020,0.247059,0.121569}%
\pgfsetfillcolor{currentfill}%
\pgfsetlinewidth{1.003750pt}%
\definecolor{currentstroke}{rgb}{0.266667,0.266667,0.266667}%
\pgfsetstrokecolor{currentstroke}%
\pgfsetdash{}{0pt}%
\pgfpathmoveto{\pgfqpoint{4.102872in}{1.615100in}}%
\pgfpathlineto{\pgfqpoint{4.340352in}{1.615100in}}%
\pgfpathlineto{\pgfqpoint{4.340352in}{1.781204in}}%
\pgfpathlineto{\pgfqpoint{4.102872in}{1.781204in}}%
\pgfpathlineto{\pgfqpoint{4.102872in}{1.615100in}}%
\pgfpathclose%
\pgfusepath{stroke,fill}%
\end{pgfscope}%
\begin{pgfscope}%
\pgfpathrectangle{\pgfqpoint{0.868056in}{0.555556in}}{\pgfqpoint{3.993056in}{1.888889in}}%
\pgfusepath{clip}%
\pgfsetbuttcap%
\pgfsetmiterjoin%
\definecolor{currentfill}{rgb}{0.549020,0.247059,0.121569}%
\pgfsetfillcolor{currentfill}%
\pgfsetlinewidth{1.003750pt}%
\definecolor{currentstroke}{rgb}{0.266667,0.266667,0.266667}%
\pgfsetstrokecolor{currentstroke}%
\pgfsetdash{}{0pt}%
\pgfpathmoveto{\pgfqpoint{4.442129in}{1.788708in}}%
\pgfpathlineto{\pgfqpoint{4.679609in}{1.788708in}}%
\pgfpathlineto{\pgfqpoint{4.679609in}{1.949605in}}%
\pgfpathlineto{\pgfqpoint{4.442129in}{1.949605in}}%
\pgfpathlineto{\pgfqpoint{4.442129in}{1.788708in}}%
\pgfpathclose%
\pgfusepath{stroke,fill}%
\end{pgfscope}%
\begin{pgfscope}%
\pgfpathrectangle{\pgfqpoint{0.868056in}{0.555556in}}{\pgfqpoint{3.993056in}{1.888889in}}%
\pgfusepath{clip}%
\pgfsetbuttcap%
\pgfsetmiterjoin%
\definecolor{currentfill}{rgb}{0.701961,0.760784,0.360784}%
\pgfsetfillcolor{currentfill}%
\pgfsetlinewidth{1.003750pt}%
\definecolor{currentstroke}{rgb}{0.266667,0.266667,0.266667}%
\pgfsetstrokecolor{currentstroke}%
\pgfsetdash{}{0pt}%
\pgfpathmoveto{\pgfqpoint{1.049558in}{1.623071in}}%
\pgfpathlineto{\pgfqpoint{1.287038in}{1.623071in}}%
\pgfpathlineto{\pgfqpoint{1.287038in}{1.798014in}}%
\pgfpathlineto{\pgfqpoint{1.049558in}{1.798014in}}%
\pgfpathlineto{\pgfqpoint{1.049558in}{1.623071in}}%
\pgfpathclose%
\pgfusepath{stroke,fill}%
\end{pgfscope}%
\begin{pgfscope}%
\pgfpathrectangle{\pgfqpoint{0.868056in}{0.555556in}}{\pgfqpoint{3.993056in}{1.888889in}}%
\pgfusepath{clip}%
\pgfsetbuttcap%
\pgfsetmiterjoin%
\definecolor{currentfill}{rgb}{0.701961,0.760784,0.360784}%
\pgfsetfillcolor{currentfill}%
\pgfsetlinewidth{1.003750pt}%
\definecolor{currentstroke}{rgb}{0.266667,0.266667,0.266667}%
\pgfsetstrokecolor{currentstroke}%
\pgfsetdash{}{0pt}%
\pgfpathmoveto{\pgfqpoint{1.388815in}{1.369159in}}%
\pgfpathlineto{\pgfqpoint{1.626295in}{1.369159in}}%
\pgfpathlineto{\pgfqpoint{1.626295in}{1.461294in}}%
\pgfpathlineto{\pgfqpoint{1.388815in}{1.461294in}}%
\pgfpathlineto{\pgfqpoint{1.388815in}{1.369159in}}%
\pgfpathclose%
\pgfusepath{stroke,fill}%
\end{pgfscope}%
\begin{pgfscope}%
\pgfpathrectangle{\pgfqpoint{0.868056in}{0.555556in}}{\pgfqpoint{3.993056in}{1.888889in}}%
\pgfusepath{clip}%
\pgfsetbuttcap%
\pgfsetmiterjoin%
\definecolor{currentfill}{rgb}{0.701961,0.760784,0.360784}%
\pgfsetfillcolor{currentfill}%
\pgfsetlinewidth{1.003750pt}%
\definecolor{currentstroke}{rgb}{0.266667,0.266667,0.266667}%
\pgfsetstrokecolor{currentstroke}%
\pgfsetdash{}{0pt}%
\pgfpathmoveto{\pgfqpoint{1.728072in}{1.145411in}}%
\pgfpathlineto{\pgfqpoint{1.965552in}{1.145411in}}%
\pgfpathlineto{\pgfqpoint{1.965552in}{1.213707in}}%
\pgfpathlineto{\pgfqpoint{1.728072in}{1.213707in}}%
\pgfpathlineto{\pgfqpoint{1.728072in}{1.145411in}}%
\pgfpathclose%
\pgfusepath{stroke,fill}%
\end{pgfscope}%
\begin{pgfscope}%
\pgfpathrectangle{\pgfqpoint{0.868056in}{0.555556in}}{\pgfqpoint{3.993056in}{1.888889in}}%
\pgfusepath{clip}%
\pgfsetbuttcap%
\pgfsetmiterjoin%
\definecolor{currentfill}{rgb}{0.701961,0.760784,0.360784}%
\pgfsetfillcolor{currentfill}%
\pgfsetlinewidth{1.003750pt}%
\definecolor{currentstroke}{rgb}{0.266667,0.266667,0.266667}%
\pgfsetstrokecolor{currentstroke}%
\pgfsetdash{}{0pt}%
\pgfpathmoveto{\pgfqpoint{2.067329in}{1.266715in}}%
\pgfpathlineto{\pgfqpoint{2.304809in}{1.266715in}}%
\pgfpathlineto{\pgfqpoint{2.304809in}{1.350786in}}%
\pgfpathlineto{\pgfqpoint{2.067329in}{1.350786in}}%
\pgfpathlineto{\pgfqpoint{2.067329in}{1.266715in}}%
\pgfpathclose%
\pgfusepath{stroke,fill}%
\end{pgfscope}%
\begin{pgfscope}%
\pgfpathrectangle{\pgfqpoint{0.868056in}{0.555556in}}{\pgfqpoint{3.993056in}{1.888889in}}%
\pgfusepath{clip}%
\pgfsetbuttcap%
\pgfsetmiterjoin%
\definecolor{currentfill}{rgb}{0.701961,0.760784,0.360784}%
\pgfsetfillcolor{currentfill}%
\pgfsetlinewidth{1.003750pt}%
\definecolor{currentstroke}{rgb}{0.266667,0.266667,0.266667}%
\pgfsetstrokecolor{currentstroke}%
\pgfsetdash{}{0pt}%
\pgfpathmoveto{\pgfqpoint{2.406586in}{1.337671in}}%
\pgfpathlineto{\pgfqpoint{2.644066in}{1.337671in}}%
\pgfpathlineto{\pgfqpoint{2.644066in}{1.423927in}}%
\pgfpathlineto{\pgfqpoint{2.406586in}{1.423927in}}%
\pgfpathlineto{\pgfqpoint{2.406586in}{1.337671in}}%
\pgfpathclose%
\pgfusepath{stroke,fill}%
\end{pgfscope}%
\begin{pgfscope}%
\pgfpathrectangle{\pgfqpoint{0.868056in}{0.555556in}}{\pgfqpoint{3.993056in}{1.888889in}}%
\pgfusepath{clip}%
\pgfsetbuttcap%
\pgfsetmiterjoin%
\definecolor{currentfill}{rgb}{0.701961,0.760784,0.360784}%
\pgfsetfillcolor{currentfill}%
\pgfsetlinewidth{1.003750pt}%
\definecolor{currentstroke}{rgb}{0.266667,0.266667,0.266667}%
\pgfsetstrokecolor{currentstroke}%
\pgfsetdash{}{0pt}%
\pgfpathmoveto{\pgfqpoint{2.745843in}{1.502926in}}%
\pgfpathlineto{\pgfqpoint{2.983323in}{1.502926in}}%
\pgfpathlineto{\pgfqpoint{2.983323in}{1.610256in}}%
\pgfpathlineto{\pgfqpoint{2.745843in}{1.610256in}}%
\pgfpathlineto{\pgfqpoint{2.745843in}{1.502926in}}%
\pgfpathclose%
\pgfusepath{stroke,fill}%
\end{pgfscope}%
\begin{pgfscope}%
\pgfpathrectangle{\pgfqpoint{0.868056in}{0.555556in}}{\pgfqpoint{3.993056in}{1.888889in}}%
\pgfusepath{clip}%
\pgfsetbuttcap%
\pgfsetmiterjoin%
\definecolor{currentfill}{rgb}{0.701961,0.760784,0.360784}%
\pgfsetfillcolor{currentfill}%
\pgfsetlinewidth{1.003750pt}%
\definecolor{currentstroke}{rgb}{0.266667,0.266667,0.266667}%
\pgfsetstrokecolor{currentstroke}%
\pgfsetdash{}{0pt}%
\pgfpathmoveto{\pgfqpoint{3.085100in}{1.266239in}}%
\pgfpathlineto{\pgfqpoint{3.322580in}{1.266239in}}%
\pgfpathlineto{\pgfqpoint{3.322580in}{1.356852in}}%
\pgfpathlineto{\pgfqpoint{3.085100in}{1.356852in}}%
\pgfpathlineto{\pgfqpoint{3.085100in}{1.266239in}}%
\pgfpathclose%
\pgfusepath{stroke,fill}%
\end{pgfscope}%
\begin{pgfscope}%
\pgfpathrectangle{\pgfqpoint{0.868056in}{0.555556in}}{\pgfqpoint{3.993056in}{1.888889in}}%
\pgfusepath{clip}%
\pgfsetbuttcap%
\pgfsetmiterjoin%
\definecolor{currentfill}{rgb}{0.701961,0.760784,0.360784}%
\pgfsetfillcolor{currentfill}%
\pgfsetlinewidth{1.003750pt}%
\definecolor{currentstroke}{rgb}{0.266667,0.266667,0.266667}%
\pgfsetstrokecolor{currentstroke}%
\pgfsetdash{}{0pt}%
\pgfpathmoveto{\pgfqpoint{3.424357in}{1.517324in}}%
\pgfpathlineto{\pgfqpoint{3.661837in}{1.517324in}}%
\pgfpathlineto{\pgfqpoint{3.661837in}{1.626373in}}%
\pgfpathlineto{\pgfqpoint{3.424357in}{1.626373in}}%
\pgfpathlineto{\pgfqpoint{3.424357in}{1.517324in}}%
\pgfpathclose%
\pgfusepath{stroke,fill}%
\end{pgfscope}%
\begin{pgfscope}%
\pgfpathrectangle{\pgfqpoint{0.868056in}{0.555556in}}{\pgfqpoint{3.993056in}{1.888889in}}%
\pgfusepath{clip}%
\pgfsetbuttcap%
\pgfsetmiterjoin%
\definecolor{currentfill}{rgb}{0.701961,0.760784,0.360784}%
\pgfsetfillcolor{currentfill}%
\pgfsetlinewidth{1.003750pt}%
\definecolor{currentstroke}{rgb}{0.266667,0.266667,0.266667}%
\pgfsetstrokecolor{currentstroke}%
\pgfsetdash{}{0pt}%
\pgfpathmoveto{\pgfqpoint{3.763615in}{1.728341in}}%
\pgfpathlineto{\pgfqpoint{4.001094in}{1.728341in}}%
\pgfpathlineto{\pgfqpoint{4.001094in}{1.843455in}}%
\pgfpathlineto{\pgfqpoint{3.763615in}{1.843455in}}%
\pgfpathlineto{\pgfqpoint{3.763615in}{1.728341in}}%
\pgfpathclose%
\pgfusepath{stroke,fill}%
\end{pgfscope}%
\begin{pgfscope}%
\pgfpathrectangle{\pgfqpoint{0.868056in}{0.555556in}}{\pgfqpoint{3.993056in}{1.888889in}}%
\pgfusepath{clip}%
\pgfsetbuttcap%
\pgfsetmiterjoin%
\definecolor{currentfill}{rgb}{0.701961,0.760784,0.360784}%
\pgfsetfillcolor{currentfill}%
\pgfsetlinewidth{1.003750pt}%
\definecolor{currentstroke}{rgb}{0.266667,0.266667,0.266667}%
\pgfsetstrokecolor{currentstroke}%
\pgfsetdash{}{0pt}%
\pgfpathmoveto{\pgfqpoint{4.102872in}{1.781204in}}%
\pgfpathlineto{\pgfqpoint{4.340352in}{1.781204in}}%
\pgfpathlineto{\pgfqpoint{4.340352in}{1.906731in}}%
\pgfpathlineto{\pgfqpoint{4.102872in}{1.906731in}}%
\pgfpathlineto{\pgfqpoint{4.102872in}{1.781204in}}%
\pgfpathclose%
\pgfusepath{stroke,fill}%
\end{pgfscope}%
\begin{pgfscope}%
\pgfpathrectangle{\pgfqpoint{0.868056in}{0.555556in}}{\pgfqpoint{3.993056in}{1.888889in}}%
\pgfusepath{clip}%
\pgfsetbuttcap%
\pgfsetmiterjoin%
\definecolor{currentfill}{rgb}{0.701961,0.760784,0.360784}%
\pgfsetfillcolor{currentfill}%
\pgfsetlinewidth{1.003750pt}%
\definecolor{currentstroke}{rgb}{0.266667,0.266667,0.266667}%
\pgfsetstrokecolor{currentstroke}%
\pgfsetdash{}{0pt}%
\pgfpathmoveto{\pgfqpoint{4.442129in}{1.949605in}}%
\pgfpathlineto{\pgfqpoint{4.679609in}{1.949605in}}%
\pgfpathlineto{\pgfqpoint{4.679609in}{2.086767in}}%
\pgfpathlineto{\pgfqpoint{4.442129in}{2.086767in}}%
\pgfpathlineto{\pgfqpoint{4.442129in}{1.949605in}}%
\pgfpathclose%
\pgfusepath{stroke,fill}%
\end{pgfscope}%
\begin{pgfscope}%
\pgfpathrectangle{\pgfqpoint{0.868056in}{0.555556in}}{\pgfqpoint{3.993056in}{1.888889in}}%
\pgfusepath{clip}%
\pgfsetbuttcap%
\pgfsetmiterjoin%
\definecolor{currentfill}{rgb}{0.447059,0.447059,0.447059}%
\pgfsetfillcolor{currentfill}%
\pgfsetlinewidth{1.003750pt}%
\definecolor{currentstroke}{rgb}{0.266667,0.266667,0.266667}%
\pgfsetstrokecolor{currentstroke}%
\pgfsetdash{}{0pt}%
\pgfpathmoveto{\pgfqpoint{1.049558in}{1.798014in}}%
\pgfpathlineto{\pgfqpoint{1.287038in}{1.798014in}}%
\pgfpathlineto{\pgfqpoint{1.287038in}{1.972730in}}%
\pgfpathlineto{\pgfqpoint{1.049558in}{1.972730in}}%
\pgfpathlineto{\pgfqpoint{1.049558in}{1.798014in}}%
\pgfpathclose%
\pgfusepath{stroke,fill}%
\end{pgfscope}%
\begin{pgfscope}%
\pgfpathrectangle{\pgfqpoint{0.868056in}{0.555556in}}{\pgfqpoint{3.993056in}{1.888889in}}%
\pgfusepath{clip}%
\pgfsetbuttcap%
\pgfsetmiterjoin%
\definecolor{currentfill}{rgb}{0.447059,0.447059,0.447059}%
\pgfsetfillcolor{currentfill}%
\pgfsetlinewidth{1.003750pt}%
\definecolor{currentstroke}{rgb}{0.266667,0.266667,0.266667}%
\pgfsetstrokecolor{currentstroke}%
\pgfsetdash{}{0pt}%
\pgfpathmoveto{\pgfqpoint{1.388815in}{1.461294in}}%
\pgfpathlineto{\pgfqpoint{1.626295in}{1.461294in}}%
\pgfpathlineto{\pgfqpoint{1.626295in}{1.577816in}}%
\pgfpathlineto{\pgfqpoint{1.388815in}{1.577816in}}%
\pgfpathlineto{\pgfqpoint{1.388815in}{1.461294in}}%
\pgfpathclose%
\pgfusepath{stroke,fill}%
\end{pgfscope}%
\begin{pgfscope}%
\pgfpathrectangle{\pgfqpoint{0.868056in}{0.555556in}}{\pgfqpoint{3.993056in}{1.888889in}}%
\pgfusepath{clip}%
\pgfsetbuttcap%
\pgfsetmiterjoin%
\definecolor{currentfill}{rgb}{0.447059,0.447059,0.447059}%
\pgfsetfillcolor{currentfill}%
\pgfsetlinewidth{1.003750pt}%
\definecolor{currentstroke}{rgb}{0.266667,0.266667,0.266667}%
\pgfsetstrokecolor{currentstroke}%
\pgfsetdash{}{0pt}%
\pgfpathmoveto{\pgfqpoint{1.728072in}{1.213707in}}%
\pgfpathlineto{\pgfqpoint{1.965552in}{1.213707in}}%
\pgfpathlineto{\pgfqpoint{1.965552in}{1.317300in}}%
\pgfpathlineto{\pgfqpoint{1.728072in}{1.317300in}}%
\pgfpathlineto{\pgfqpoint{1.728072in}{1.213707in}}%
\pgfpathclose%
\pgfusepath{stroke,fill}%
\end{pgfscope}%
\begin{pgfscope}%
\pgfpathrectangle{\pgfqpoint{0.868056in}{0.555556in}}{\pgfqpoint{3.993056in}{1.888889in}}%
\pgfusepath{clip}%
\pgfsetbuttcap%
\pgfsetmiterjoin%
\definecolor{currentfill}{rgb}{0.447059,0.447059,0.447059}%
\pgfsetfillcolor{currentfill}%
\pgfsetlinewidth{1.003750pt}%
\definecolor{currentstroke}{rgb}{0.266667,0.266667,0.266667}%
\pgfsetstrokecolor{currentstroke}%
\pgfsetdash{}{0pt}%
\pgfpathmoveto{\pgfqpoint{2.067329in}{1.350786in}}%
\pgfpathlineto{\pgfqpoint{2.304809in}{1.350786in}}%
\pgfpathlineto{\pgfqpoint{2.304809in}{1.430903in}}%
\pgfpathlineto{\pgfqpoint{2.067329in}{1.430903in}}%
\pgfpathlineto{\pgfqpoint{2.067329in}{1.350786in}}%
\pgfpathclose%
\pgfusepath{stroke,fill}%
\end{pgfscope}%
\begin{pgfscope}%
\pgfpathrectangle{\pgfqpoint{0.868056in}{0.555556in}}{\pgfqpoint{3.993056in}{1.888889in}}%
\pgfusepath{clip}%
\pgfsetbuttcap%
\pgfsetmiterjoin%
\definecolor{currentfill}{rgb}{0.447059,0.447059,0.447059}%
\pgfsetfillcolor{currentfill}%
\pgfsetlinewidth{1.003750pt}%
\definecolor{currentstroke}{rgb}{0.266667,0.266667,0.266667}%
\pgfsetstrokecolor{currentstroke}%
\pgfsetdash{}{0pt}%
\pgfpathmoveto{\pgfqpoint{2.406586in}{1.423927in}}%
\pgfpathlineto{\pgfqpoint{2.644066in}{1.423927in}}%
\pgfpathlineto{\pgfqpoint{2.644066in}{1.517583in}}%
\pgfpathlineto{\pgfqpoint{2.406586in}{1.517583in}}%
\pgfpathlineto{\pgfqpoint{2.406586in}{1.423927in}}%
\pgfpathclose%
\pgfusepath{stroke,fill}%
\end{pgfscope}%
\begin{pgfscope}%
\pgfpathrectangle{\pgfqpoint{0.868056in}{0.555556in}}{\pgfqpoint{3.993056in}{1.888889in}}%
\pgfusepath{clip}%
\pgfsetbuttcap%
\pgfsetmiterjoin%
\definecolor{currentfill}{rgb}{0.447059,0.447059,0.447059}%
\pgfsetfillcolor{currentfill}%
\pgfsetlinewidth{1.003750pt}%
\definecolor{currentstroke}{rgb}{0.266667,0.266667,0.266667}%
\pgfsetstrokecolor{currentstroke}%
\pgfsetdash{}{0pt}%
\pgfpathmoveto{\pgfqpoint{2.745843in}{1.610256in}}%
\pgfpathlineto{\pgfqpoint{2.983323in}{1.610256in}}%
\pgfpathlineto{\pgfqpoint{2.983323in}{1.723621in}}%
\pgfpathlineto{\pgfqpoint{2.745843in}{1.723621in}}%
\pgfpathlineto{\pgfqpoint{2.745843in}{1.610256in}}%
\pgfpathclose%
\pgfusepath{stroke,fill}%
\end{pgfscope}%
\begin{pgfscope}%
\pgfpathrectangle{\pgfqpoint{0.868056in}{0.555556in}}{\pgfqpoint{3.993056in}{1.888889in}}%
\pgfusepath{clip}%
\pgfsetbuttcap%
\pgfsetmiterjoin%
\definecolor{currentfill}{rgb}{0.447059,0.447059,0.447059}%
\pgfsetfillcolor{currentfill}%
\pgfsetlinewidth{1.003750pt}%
\definecolor{currentstroke}{rgb}{0.266667,0.266667,0.266667}%
\pgfsetstrokecolor{currentstroke}%
\pgfsetdash{}{0pt}%
\pgfpathmoveto{\pgfqpoint{3.085100in}{1.356852in}}%
\pgfpathlineto{\pgfqpoint{3.322580in}{1.356852in}}%
\pgfpathlineto{\pgfqpoint{3.322580in}{1.455808in}}%
\pgfpathlineto{\pgfqpoint{3.085100in}{1.455808in}}%
\pgfpathlineto{\pgfqpoint{3.085100in}{1.356852in}}%
\pgfpathclose%
\pgfusepath{stroke,fill}%
\end{pgfscope}%
\begin{pgfscope}%
\pgfpathrectangle{\pgfqpoint{0.868056in}{0.555556in}}{\pgfqpoint{3.993056in}{1.888889in}}%
\pgfusepath{clip}%
\pgfsetbuttcap%
\pgfsetmiterjoin%
\definecolor{currentfill}{rgb}{0.447059,0.447059,0.447059}%
\pgfsetfillcolor{currentfill}%
\pgfsetlinewidth{1.003750pt}%
\definecolor{currentstroke}{rgb}{0.266667,0.266667,0.266667}%
\pgfsetstrokecolor{currentstroke}%
\pgfsetdash{}{0pt}%
\pgfpathmoveto{\pgfqpoint{3.424357in}{1.626373in}}%
\pgfpathlineto{\pgfqpoint{3.661837in}{1.626373in}}%
\pgfpathlineto{\pgfqpoint{3.661837in}{1.739189in}}%
\pgfpathlineto{\pgfqpoint{3.424357in}{1.739189in}}%
\pgfpathlineto{\pgfqpoint{3.424357in}{1.626373in}}%
\pgfpathclose%
\pgfusepath{stroke,fill}%
\end{pgfscope}%
\begin{pgfscope}%
\pgfpathrectangle{\pgfqpoint{0.868056in}{0.555556in}}{\pgfqpoint{3.993056in}{1.888889in}}%
\pgfusepath{clip}%
\pgfsetbuttcap%
\pgfsetmiterjoin%
\definecolor{currentfill}{rgb}{0.447059,0.447059,0.447059}%
\pgfsetfillcolor{currentfill}%
\pgfsetlinewidth{1.003750pt}%
\definecolor{currentstroke}{rgb}{0.266667,0.266667,0.266667}%
\pgfsetstrokecolor{currentstroke}%
\pgfsetdash{}{0pt}%
\pgfpathmoveto{\pgfqpoint{3.763615in}{1.843455in}}%
\pgfpathlineto{\pgfqpoint{4.001094in}{1.843455in}}%
\pgfpathlineto{\pgfqpoint{4.001094in}{1.947525in}}%
\pgfpathlineto{\pgfqpoint{3.763615in}{1.947525in}}%
\pgfpathlineto{\pgfqpoint{3.763615in}{1.843455in}}%
\pgfpathclose%
\pgfusepath{stroke,fill}%
\end{pgfscope}%
\begin{pgfscope}%
\pgfpathrectangle{\pgfqpoint{0.868056in}{0.555556in}}{\pgfqpoint{3.993056in}{1.888889in}}%
\pgfusepath{clip}%
\pgfsetbuttcap%
\pgfsetmiterjoin%
\definecolor{currentfill}{rgb}{0.447059,0.447059,0.447059}%
\pgfsetfillcolor{currentfill}%
\pgfsetlinewidth{1.003750pt}%
\definecolor{currentstroke}{rgb}{0.266667,0.266667,0.266667}%
\pgfsetstrokecolor{currentstroke}%
\pgfsetdash{}{0pt}%
\pgfpathmoveto{\pgfqpoint{4.102872in}{1.906731in}}%
\pgfpathlineto{\pgfqpoint{4.340352in}{1.906731in}}%
\pgfpathlineto{\pgfqpoint{4.340352in}{2.014672in}}%
\pgfpathlineto{\pgfqpoint{4.102872in}{2.014672in}}%
\pgfpathlineto{\pgfqpoint{4.102872in}{1.906731in}}%
\pgfpathclose%
\pgfusepath{stroke,fill}%
\end{pgfscope}%
\begin{pgfscope}%
\pgfpathrectangle{\pgfqpoint{0.868056in}{0.555556in}}{\pgfqpoint{3.993056in}{1.888889in}}%
\pgfusepath{clip}%
\pgfsetbuttcap%
\pgfsetmiterjoin%
\definecolor{currentfill}{rgb}{0.447059,0.447059,0.447059}%
\pgfsetfillcolor{currentfill}%
\pgfsetlinewidth{1.003750pt}%
\definecolor{currentstroke}{rgb}{0.266667,0.266667,0.266667}%
\pgfsetstrokecolor{currentstroke}%
\pgfsetdash{}{0pt}%
\pgfpathmoveto{\pgfqpoint{4.442129in}{2.086767in}}%
\pgfpathlineto{\pgfqpoint{4.679609in}{2.086767in}}%
\pgfpathlineto{\pgfqpoint{4.679609in}{2.193725in}}%
\pgfpathlineto{\pgfqpoint{4.442129in}{2.193725in}}%
\pgfpathlineto{\pgfqpoint{4.442129in}{2.086767in}}%
\pgfpathclose%
\pgfusepath{stroke,fill}%
\end{pgfscope}%
\begin{pgfscope}%
\pgfpathrectangle{\pgfqpoint{0.868056in}{0.555556in}}{\pgfqpoint{3.993056in}{1.888889in}}%
\pgfusepath{clip}%
\pgfsetbuttcap%
\pgfsetmiterjoin%
\definecolor{currentfill}{rgb}{0.447059,0.447059,0.447059}%
\pgfsetfillcolor{currentfill}%
\pgfsetlinewidth{1.003750pt}%
\definecolor{currentstroke}{rgb}{0.266667,0.266667,0.266667}%
\pgfsetstrokecolor{currentstroke}%
\pgfsetdash{}{0pt}%
\pgfpathmoveto{\pgfqpoint{1.049558in}{1.972730in}}%
\pgfpathlineto{\pgfqpoint{1.287038in}{1.972730in}}%
\pgfpathlineto{\pgfqpoint{1.287038in}{2.063529in}}%
\pgfpathlineto{\pgfqpoint{1.049558in}{2.063529in}}%
\pgfpathlineto{\pgfqpoint{1.049558in}{1.972730in}}%
\pgfpathclose%
\pgfusepath{stroke,fill}%
\end{pgfscope}%
\begin{pgfscope}%
\pgfpathrectangle{\pgfqpoint{0.868056in}{0.555556in}}{\pgfqpoint{3.993056in}{1.888889in}}%
\pgfusepath{clip}%
\pgfsetbuttcap%
\pgfsetmiterjoin%
\definecolor{currentfill}{rgb}{0.447059,0.447059,0.447059}%
\pgfsetfillcolor{currentfill}%
\pgfsetlinewidth{1.003750pt}%
\definecolor{currentstroke}{rgb}{0.266667,0.266667,0.266667}%
\pgfsetstrokecolor{currentstroke}%
\pgfsetdash{}{0pt}%
\pgfpathmoveto{\pgfqpoint{1.388815in}{1.577816in}}%
\pgfpathlineto{\pgfqpoint{1.626295in}{1.577816in}}%
\pgfpathlineto{\pgfqpoint{1.626295in}{1.665407in}}%
\pgfpathlineto{\pgfqpoint{1.388815in}{1.665407in}}%
\pgfpathlineto{\pgfqpoint{1.388815in}{1.577816in}}%
\pgfpathclose%
\pgfusepath{stroke,fill}%
\end{pgfscope}%
\begin{pgfscope}%
\pgfpathrectangle{\pgfqpoint{0.868056in}{0.555556in}}{\pgfqpoint{3.993056in}{1.888889in}}%
\pgfusepath{clip}%
\pgfsetbuttcap%
\pgfsetmiterjoin%
\definecolor{currentfill}{rgb}{0.447059,0.447059,0.447059}%
\pgfsetfillcolor{currentfill}%
\pgfsetlinewidth{1.003750pt}%
\definecolor{currentstroke}{rgb}{0.266667,0.266667,0.266667}%
\pgfsetstrokecolor{currentstroke}%
\pgfsetdash{}{0pt}%
\pgfpathmoveto{\pgfqpoint{1.728072in}{1.317300in}}%
\pgfpathlineto{\pgfqpoint{1.965552in}{1.317300in}}%
\pgfpathlineto{\pgfqpoint{1.965552in}{1.412986in}}%
\pgfpathlineto{\pgfqpoint{1.728072in}{1.412986in}}%
\pgfpathlineto{\pgfqpoint{1.728072in}{1.317300in}}%
\pgfpathclose%
\pgfusepath{stroke,fill}%
\end{pgfscope}%
\begin{pgfscope}%
\pgfpathrectangle{\pgfqpoint{0.868056in}{0.555556in}}{\pgfqpoint{3.993056in}{1.888889in}}%
\pgfusepath{clip}%
\pgfsetbuttcap%
\pgfsetmiterjoin%
\definecolor{currentfill}{rgb}{0.447059,0.447059,0.447059}%
\pgfsetfillcolor{currentfill}%
\pgfsetlinewidth{1.003750pt}%
\definecolor{currentstroke}{rgb}{0.266667,0.266667,0.266667}%
\pgfsetstrokecolor{currentstroke}%
\pgfsetdash{}{0pt}%
\pgfpathmoveto{\pgfqpoint{2.067329in}{1.430903in}}%
\pgfpathlineto{\pgfqpoint{2.304809in}{1.430903in}}%
\pgfpathlineto{\pgfqpoint{2.304809in}{1.502150in}}%
\pgfpathlineto{\pgfqpoint{2.067329in}{1.502150in}}%
\pgfpathlineto{\pgfqpoint{2.067329in}{1.430903in}}%
\pgfpathclose%
\pgfusepath{stroke,fill}%
\end{pgfscope}%
\begin{pgfscope}%
\pgfpathrectangle{\pgfqpoint{0.868056in}{0.555556in}}{\pgfqpoint{3.993056in}{1.888889in}}%
\pgfusepath{clip}%
\pgfsetbuttcap%
\pgfsetmiterjoin%
\definecolor{currentfill}{rgb}{0.447059,0.447059,0.447059}%
\pgfsetfillcolor{currentfill}%
\pgfsetlinewidth{1.003750pt}%
\definecolor{currentstroke}{rgb}{0.266667,0.266667,0.266667}%
\pgfsetstrokecolor{currentstroke}%
\pgfsetdash{}{0pt}%
\pgfpathmoveto{\pgfqpoint{2.406586in}{1.517583in}}%
\pgfpathlineto{\pgfqpoint{2.644066in}{1.517583in}}%
\pgfpathlineto{\pgfqpoint{2.644066in}{1.572299in}}%
\pgfpathlineto{\pgfqpoint{2.406586in}{1.572299in}}%
\pgfpathlineto{\pgfqpoint{2.406586in}{1.517583in}}%
\pgfpathclose%
\pgfusepath{stroke,fill}%
\end{pgfscope}%
\begin{pgfscope}%
\pgfpathrectangle{\pgfqpoint{0.868056in}{0.555556in}}{\pgfqpoint{3.993056in}{1.888889in}}%
\pgfusepath{clip}%
\pgfsetbuttcap%
\pgfsetmiterjoin%
\definecolor{currentfill}{rgb}{0.447059,0.447059,0.447059}%
\pgfsetfillcolor{currentfill}%
\pgfsetlinewidth{1.003750pt}%
\definecolor{currentstroke}{rgb}{0.266667,0.266667,0.266667}%
\pgfsetstrokecolor{currentstroke}%
\pgfsetdash{}{0pt}%
\pgfpathmoveto{\pgfqpoint{2.745843in}{1.723621in}}%
\pgfpathlineto{\pgfqpoint{2.983323in}{1.723621in}}%
\pgfpathlineto{\pgfqpoint{2.983323in}{1.766205in}}%
\pgfpathlineto{\pgfqpoint{2.745843in}{1.766205in}}%
\pgfpathlineto{\pgfqpoint{2.745843in}{1.723621in}}%
\pgfpathclose%
\pgfusepath{stroke,fill}%
\end{pgfscope}%
\begin{pgfscope}%
\pgfpathrectangle{\pgfqpoint{0.868056in}{0.555556in}}{\pgfqpoint{3.993056in}{1.888889in}}%
\pgfusepath{clip}%
\pgfsetbuttcap%
\pgfsetmiterjoin%
\definecolor{currentfill}{rgb}{0.447059,0.447059,0.447059}%
\pgfsetfillcolor{currentfill}%
\pgfsetlinewidth{1.003750pt}%
\definecolor{currentstroke}{rgb}{0.266667,0.266667,0.266667}%
\pgfsetstrokecolor{currentstroke}%
\pgfsetdash{}{0pt}%
\pgfpathmoveto{\pgfqpoint{3.085100in}{1.455808in}}%
\pgfpathlineto{\pgfqpoint{3.322580in}{1.455808in}}%
\pgfpathlineto{\pgfqpoint{3.322580in}{1.496643in}}%
\pgfpathlineto{\pgfqpoint{3.085100in}{1.496643in}}%
\pgfpathlineto{\pgfqpoint{3.085100in}{1.455808in}}%
\pgfpathclose%
\pgfusepath{stroke,fill}%
\end{pgfscope}%
\begin{pgfscope}%
\pgfpathrectangle{\pgfqpoint{0.868056in}{0.555556in}}{\pgfqpoint{3.993056in}{1.888889in}}%
\pgfusepath{clip}%
\pgfsetbuttcap%
\pgfsetmiterjoin%
\definecolor{currentfill}{rgb}{0.447059,0.447059,0.447059}%
\pgfsetfillcolor{currentfill}%
\pgfsetlinewidth{1.003750pt}%
\definecolor{currentstroke}{rgb}{0.266667,0.266667,0.266667}%
\pgfsetstrokecolor{currentstroke}%
\pgfsetdash{}{0pt}%
\pgfpathmoveto{\pgfqpoint{3.424357in}{1.739189in}}%
\pgfpathlineto{\pgfqpoint{3.661837in}{1.739189in}}%
\pgfpathlineto{\pgfqpoint{3.661837in}{1.798583in}}%
\pgfpathlineto{\pgfqpoint{3.424357in}{1.798583in}}%
\pgfpathlineto{\pgfqpoint{3.424357in}{1.739189in}}%
\pgfpathclose%
\pgfusepath{stroke,fill}%
\end{pgfscope}%
\begin{pgfscope}%
\pgfpathrectangle{\pgfqpoint{0.868056in}{0.555556in}}{\pgfqpoint{3.993056in}{1.888889in}}%
\pgfusepath{clip}%
\pgfsetbuttcap%
\pgfsetmiterjoin%
\definecolor{currentfill}{rgb}{0.447059,0.447059,0.447059}%
\pgfsetfillcolor{currentfill}%
\pgfsetlinewidth{1.003750pt}%
\definecolor{currentstroke}{rgb}{0.266667,0.266667,0.266667}%
\pgfsetstrokecolor{currentstroke}%
\pgfsetdash{}{0pt}%
\pgfpathmoveto{\pgfqpoint{3.763615in}{1.947525in}}%
\pgfpathlineto{\pgfqpoint{4.001094in}{1.947525in}}%
\pgfpathlineto{\pgfqpoint{4.001094in}{2.016390in}}%
\pgfpathlineto{\pgfqpoint{3.763615in}{2.016390in}}%
\pgfpathlineto{\pgfqpoint{3.763615in}{1.947525in}}%
\pgfpathclose%
\pgfusepath{stroke,fill}%
\end{pgfscope}%
\begin{pgfscope}%
\pgfpathrectangle{\pgfqpoint{0.868056in}{0.555556in}}{\pgfqpoint{3.993056in}{1.888889in}}%
\pgfusepath{clip}%
\pgfsetbuttcap%
\pgfsetmiterjoin%
\definecolor{currentfill}{rgb}{0.447059,0.447059,0.447059}%
\pgfsetfillcolor{currentfill}%
\pgfsetlinewidth{1.003750pt}%
\definecolor{currentstroke}{rgb}{0.266667,0.266667,0.266667}%
\pgfsetstrokecolor{currentstroke}%
\pgfsetdash{}{0pt}%
\pgfpathmoveto{\pgfqpoint{4.102872in}{2.014672in}}%
\pgfpathlineto{\pgfqpoint{4.340352in}{2.014672in}}%
\pgfpathlineto{\pgfqpoint{4.340352in}{2.092388in}}%
\pgfpathlineto{\pgfqpoint{4.102872in}{2.092388in}}%
\pgfpathlineto{\pgfqpoint{4.102872in}{2.014672in}}%
\pgfpathclose%
\pgfusepath{stroke,fill}%
\end{pgfscope}%
\begin{pgfscope}%
\pgfpathrectangle{\pgfqpoint{0.868056in}{0.555556in}}{\pgfqpoint{3.993056in}{1.888889in}}%
\pgfusepath{clip}%
\pgfsetbuttcap%
\pgfsetmiterjoin%
\definecolor{currentfill}{rgb}{0.447059,0.447059,0.447059}%
\pgfsetfillcolor{currentfill}%
\pgfsetlinewidth{1.003750pt}%
\definecolor{currentstroke}{rgb}{0.266667,0.266667,0.266667}%
\pgfsetstrokecolor{currentstroke}%
\pgfsetdash{}{0pt}%
\pgfpathmoveto{\pgfqpoint{4.442129in}{2.193725in}}%
\pgfpathlineto{\pgfqpoint{4.679609in}{2.193725in}}%
\pgfpathlineto{\pgfqpoint{4.679609in}{2.256297in}}%
\pgfpathlineto{\pgfqpoint{4.442129in}{2.256297in}}%
\pgfpathlineto{\pgfqpoint{4.442129in}{2.193725in}}%
\pgfpathclose%
\pgfusepath{stroke,fill}%
\end{pgfscope}%
\begin{pgfscope}%
\pgfpathrectangle{\pgfqpoint{0.868056in}{0.555556in}}{\pgfqpoint{3.993056in}{1.888889in}}%
\pgfusepath{clip}%
\pgfsetbuttcap%
\pgfsetmiterjoin%
\definecolor{currentfill}{rgb}{0.447059,0.447059,0.447059}%
\pgfsetfillcolor{currentfill}%
\pgfsetlinewidth{1.003750pt}%
\definecolor{currentstroke}{rgb}{0.266667,0.266667,0.266667}%
\pgfsetstrokecolor{currentstroke}%
\pgfsetdash{}{0pt}%
\pgfpathmoveto{\pgfqpoint{1.049558in}{2.063529in}}%
\pgfpathlineto{\pgfqpoint{1.287038in}{2.063529in}}%
\pgfpathlineto{\pgfqpoint{1.287038in}{2.166491in}}%
\pgfpathlineto{\pgfqpoint{1.049558in}{2.166491in}}%
\pgfpathlineto{\pgfqpoint{1.049558in}{2.063529in}}%
\pgfpathclose%
\pgfusepath{stroke,fill}%
\end{pgfscope}%
\begin{pgfscope}%
\pgfpathrectangle{\pgfqpoint{0.868056in}{0.555556in}}{\pgfqpoint{3.993056in}{1.888889in}}%
\pgfusepath{clip}%
\pgfsetbuttcap%
\pgfsetmiterjoin%
\definecolor{currentfill}{rgb}{0.447059,0.447059,0.447059}%
\pgfsetfillcolor{currentfill}%
\pgfsetlinewidth{1.003750pt}%
\definecolor{currentstroke}{rgb}{0.266667,0.266667,0.266667}%
\pgfsetstrokecolor{currentstroke}%
\pgfsetdash{}{0pt}%
\pgfpathmoveto{\pgfqpoint{1.388815in}{1.665407in}}%
\pgfpathlineto{\pgfqpoint{1.626295in}{1.665407in}}%
\pgfpathlineto{\pgfqpoint{1.626295in}{1.769217in}}%
\pgfpathlineto{\pgfqpoint{1.388815in}{1.769217in}}%
\pgfpathlineto{\pgfqpoint{1.388815in}{1.665407in}}%
\pgfpathclose%
\pgfusepath{stroke,fill}%
\end{pgfscope}%
\begin{pgfscope}%
\pgfpathrectangle{\pgfqpoint{0.868056in}{0.555556in}}{\pgfqpoint{3.993056in}{1.888889in}}%
\pgfusepath{clip}%
\pgfsetbuttcap%
\pgfsetmiterjoin%
\definecolor{currentfill}{rgb}{0.447059,0.447059,0.447059}%
\pgfsetfillcolor{currentfill}%
\pgfsetlinewidth{1.003750pt}%
\definecolor{currentstroke}{rgb}{0.266667,0.266667,0.266667}%
\pgfsetstrokecolor{currentstroke}%
\pgfsetdash{}{0pt}%
\pgfpathmoveto{\pgfqpoint{1.728072in}{1.412986in}}%
\pgfpathlineto{\pgfqpoint{1.965552in}{1.412986in}}%
\pgfpathlineto{\pgfqpoint{1.965552in}{1.476127in}}%
\pgfpathlineto{\pgfqpoint{1.728072in}{1.476127in}}%
\pgfpathlineto{\pgfqpoint{1.728072in}{1.412986in}}%
\pgfpathclose%
\pgfusepath{stroke,fill}%
\end{pgfscope}%
\begin{pgfscope}%
\pgfpathrectangle{\pgfqpoint{0.868056in}{0.555556in}}{\pgfqpoint{3.993056in}{1.888889in}}%
\pgfusepath{clip}%
\pgfsetbuttcap%
\pgfsetmiterjoin%
\definecolor{currentfill}{rgb}{0.447059,0.447059,0.447059}%
\pgfsetfillcolor{currentfill}%
\pgfsetlinewidth{1.003750pt}%
\definecolor{currentstroke}{rgb}{0.266667,0.266667,0.266667}%
\pgfsetstrokecolor{currentstroke}%
\pgfsetdash{}{0pt}%
\pgfpathmoveto{\pgfqpoint{2.067329in}{1.502150in}}%
\pgfpathlineto{\pgfqpoint{2.304809in}{1.502150in}}%
\pgfpathlineto{\pgfqpoint{2.304809in}{1.584627in}}%
\pgfpathlineto{\pgfqpoint{2.067329in}{1.584627in}}%
\pgfpathlineto{\pgfqpoint{2.067329in}{1.502150in}}%
\pgfpathclose%
\pgfusepath{stroke,fill}%
\end{pgfscope}%
\begin{pgfscope}%
\pgfpathrectangle{\pgfqpoint{0.868056in}{0.555556in}}{\pgfqpoint{3.993056in}{1.888889in}}%
\pgfusepath{clip}%
\pgfsetbuttcap%
\pgfsetmiterjoin%
\definecolor{currentfill}{rgb}{0.447059,0.447059,0.447059}%
\pgfsetfillcolor{currentfill}%
\pgfsetlinewidth{1.003750pt}%
\definecolor{currentstroke}{rgb}{0.266667,0.266667,0.266667}%
\pgfsetstrokecolor{currentstroke}%
\pgfsetdash{}{0pt}%
\pgfpathmoveto{\pgfqpoint{2.406586in}{1.572299in}}%
\pgfpathlineto{\pgfqpoint{2.644066in}{1.572299in}}%
\pgfpathlineto{\pgfqpoint{2.644066in}{1.635958in}}%
\pgfpathlineto{\pgfqpoint{2.406586in}{1.635958in}}%
\pgfpathlineto{\pgfqpoint{2.406586in}{1.572299in}}%
\pgfpathclose%
\pgfusepath{stroke,fill}%
\end{pgfscope}%
\begin{pgfscope}%
\pgfpathrectangle{\pgfqpoint{0.868056in}{0.555556in}}{\pgfqpoint{3.993056in}{1.888889in}}%
\pgfusepath{clip}%
\pgfsetbuttcap%
\pgfsetmiterjoin%
\definecolor{currentfill}{rgb}{0.447059,0.447059,0.447059}%
\pgfsetfillcolor{currentfill}%
\pgfsetlinewidth{1.003750pt}%
\definecolor{currentstroke}{rgb}{0.266667,0.266667,0.266667}%
\pgfsetstrokecolor{currentstroke}%
\pgfsetdash{}{0pt}%
\pgfpathmoveto{\pgfqpoint{2.745843in}{1.766205in}}%
\pgfpathlineto{\pgfqpoint{2.983323in}{1.766205in}}%
\pgfpathlineto{\pgfqpoint{2.983323in}{1.831251in}}%
\pgfpathlineto{\pgfqpoint{2.745843in}{1.831251in}}%
\pgfpathlineto{\pgfqpoint{2.745843in}{1.766205in}}%
\pgfpathclose%
\pgfusepath{stroke,fill}%
\end{pgfscope}%
\begin{pgfscope}%
\pgfpathrectangle{\pgfqpoint{0.868056in}{0.555556in}}{\pgfqpoint{3.993056in}{1.888889in}}%
\pgfusepath{clip}%
\pgfsetbuttcap%
\pgfsetmiterjoin%
\definecolor{currentfill}{rgb}{0.447059,0.447059,0.447059}%
\pgfsetfillcolor{currentfill}%
\pgfsetlinewidth{1.003750pt}%
\definecolor{currentstroke}{rgb}{0.266667,0.266667,0.266667}%
\pgfsetstrokecolor{currentstroke}%
\pgfsetdash{}{0pt}%
\pgfpathmoveto{\pgfqpoint{3.085100in}{1.496643in}}%
\pgfpathlineto{\pgfqpoint{3.322580in}{1.496643in}}%
\pgfpathlineto{\pgfqpoint{3.322580in}{1.558832in}}%
\pgfpathlineto{\pgfqpoint{3.085100in}{1.558832in}}%
\pgfpathlineto{\pgfqpoint{3.085100in}{1.496643in}}%
\pgfpathclose%
\pgfusepath{stroke,fill}%
\end{pgfscope}%
\begin{pgfscope}%
\pgfpathrectangle{\pgfqpoint{0.868056in}{0.555556in}}{\pgfqpoint{3.993056in}{1.888889in}}%
\pgfusepath{clip}%
\pgfsetbuttcap%
\pgfsetmiterjoin%
\definecolor{currentfill}{rgb}{0.447059,0.447059,0.447059}%
\pgfsetfillcolor{currentfill}%
\pgfsetlinewidth{1.003750pt}%
\definecolor{currentstroke}{rgb}{0.266667,0.266667,0.266667}%
\pgfsetstrokecolor{currentstroke}%
\pgfsetdash{}{0pt}%
\pgfpathmoveto{\pgfqpoint{3.424357in}{1.798583in}}%
\pgfpathlineto{\pgfqpoint{3.661837in}{1.798583in}}%
\pgfpathlineto{\pgfqpoint{3.661837in}{1.860058in}}%
\pgfpathlineto{\pgfqpoint{3.424357in}{1.860058in}}%
\pgfpathlineto{\pgfqpoint{3.424357in}{1.798583in}}%
\pgfpathclose%
\pgfusepath{stroke,fill}%
\end{pgfscope}%
\begin{pgfscope}%
\pgfpathrectangle{\pgfqpoint{0.868056in}{0.555556in}}{\pgfqpoint{3.993056in}{1.888889in}}%
\pgfusepath{clip}%
\pgfsetbuttcap%
\pgfsetmiterjoin%
\definecolor{currentfill}{rgb}{0.447059,0.447059,0.447059}%
\pgfsetfillcolor{currentfill}%
\pgfsetlinewidth{1.003750pt}%
\definecolor{currentstroke}{rgb}{0.266667,0.266667,0.266667}%
\pgfsetstrokecolor{currentstroke}%
\pgfsetdash{}{0pt}%
\pgfpathmoveto{\pgfqpoint{3.763615in}{2.016390in}}%
\pgfpathlineto{\pgfqpoint{4.001094in}{2.016390in}}%
\pgfpathlineto{\pgfqpoint{4.001094in}{2.082730in}}%
\pgfpathlineto{\pgfqpoint{3.763615in}{2.082730in}}%
\pgfpathlineto{\pgfqpoint{3.763615in}{2.016390in}}%
\pgfpathclose%
\pgfusepath{stroke,fill}%
\end{pgfscope}%
\begin{pgfscope}%
\pgfpathrectangle{\pgfqpoint{0.868056in}{0.555556in}}{\pgfqpoint{3.993056in}{1.888889in}}%
\pgfusepath{clip}%
\pgfsetbuttcap%
\pgfsetmiterjoin%
\definecolor{currentfill}{rgb}{0.447059,0.447059,0.447059}%
\pgfsetfillcolor{currentfill}%
\pgfsetlinewidth{1.003750pt}%
\definecolor{currentstroke}{rgb}{0.266667,0.266667,0.266667}%
\pgfsetstrokecolor{currentstroke}%
\pgfsetdash{}{0pt}%
\pgfpathmoveto{\pgfqpoint{4.102872in}{2.092388in}}%
\pgfpathlineto{\pgfqpoint{4.340352in}{2.092388in}}%
\pgfpathlineto{\pgfqpoint{4.340352in}{2.161678in}}%
\pgfpathlineto{\pgfqpoint{4.102872in}{2.161678in}}%
\pgfpathlineto{\pgfqpoint{4.102872in}{2.092388in}}%
\pgfpathclose%
\pgfusepath{stroke,fill}%
\end{pgfscope}%
\begin{pgfscope}%
\pgfpathrectangle{\pgfqpoint{0.868056in}{0.555556in}}{\pgfqpoint{3.993056in}{1.888889in}}%
\pgfusepath{clip}%
\pgfsetbuttcap%
\pgfsetmiterjoin%
\definecolor{currentfill}{rgb}{0.447059,0.447059,0.447059}%
\pgfsetfillcolor{currentfill}%
\pgfsetlinewidth{1.003750pt}%
\definecolor{currentstroke}{rgb}{0.266667,0.266667,0.266667}%
\pgfsetstrokecolor{currentstroke}%
\pgfsetdash{}{0pt}%
\pgfpathmoveto{\pgfqpoint{4.442129in}{2.256297in}}%
\pgfpathlineto{\pgfqpoint{4.679609in}{2.256297in}}%
\pgfpathlineto{\pgfqpoint{4.679609in}{2.324541in}}%
\pgfpathlineto{\pgfqpoint{4.442129in}{2.324541in}}%
\pgfpathlineto{\pgfqpoint{4.442129in}{2.256297in}}%
\pgfpathclose%
\pgfusepath{stroke,fill}%
\end{pgfscope}%
\begin{pgfscope}%
\pgfpathrectangle{\pgfqpoint{0.868056in}{0.555556in}}{\pgfqpoint{3.993056in}{1.888889in}}%
\pgfusepath{clip}%
\pgfsetbuttcap%
\pgfsetmiterjoin%
\definecolor{currentfill}{rgb}{0.447059,0.447059,0.447059}%
\pgfsetfillcolor{currentfill}%
\pgfsetlinewidth{1.003750pt}%
\definecolor{currentstroke}{rgb}{0.266667,0.266667,0.266667}%
\pgfsetstrokecolor{currentstroke}%
\pgfsetdash{}{0pt}%
\pgfpathmoveto{\pgfqpoint{1.049558in}{2.166491in}}%
\pgfpathlineto{\pgfqpoint{1.287038in}{2.166491in}}%
\pgfpathlineto{\pgfqpoint{1.287038in}{2.180807in}}%
\pgfpathlineto{\pgfqpoint{1.049558in}{2.180807in}}%
\pgfpathlineto{\pgfqpoint{1.049558in}{2.166491in}}%
\pgfpathclose%
\pgfusepath{stroke,fill}%
\end{pgfscope}%
\begin{pgfscope}%
\pgfpathrectangle{\pgfqpoint{0.868056in}{0.555556in}}{\pgfqpoint{3.993056in}{1.888889in}}%
\pgfusepath{clip}%
\pgfsetbuttcap%
\pgfsetmiterjoin%
\definecolor{currentfill}{rgb}{0.447059,0.447059,0.447059}%
\pgfsetfillcolor{currentfill}%
\pgfsetlinewidth{1.003750pt}%
\definecolor{currentstroke}{rgb}{0.266667,0.266667,0.266667}%
\pgfsetstrokecolor{currentstroke}%
\pgfsetdash{}{0pt}%
\pgfpathmoveto{\pgfqpoint{1.388815in}{1.769217in}}%
\pgfpathlineto{\pgfqpoint{1.626295in}{1.769217in}}%
\pgfpathlineto{\pgfqpoint{1.626295in}{1.773710in}}%
\pgfpathlineto{\pgfqpoint{1.388815in}{1.773710in}}%
\pgfpathlineto{\pgfqpoint{1.388815in}{1.769217in}}%
\pgfpathclose%
\pgfusepath{stroke,fill}%
\end{pgfscope}%
\begin{pgfscope}%
\pgfpathrectangle{\pgfqpoint{0.868056in}{0.555556in}}{\pgfqpoint{3.993056in}{1.888889in}}%
\pgfusepath{clip}%
\pgfsetbuttcap%
\pgfsetmiterjoin%
\definecolor{currentfill}{rgb}{0.447059,0.447059,0.447059}%
\pgfsetfillcolor{currentfill}%
\pgfsetlinewidth{1.003750pt}%
\definecolor{currentstroke}{rgb}{0.266667,0.266667,0.266667}%
\pgfsetstrokecolor{currentstroke}%
\pgfsetdash{}{0pt}%
\pgfpathmoveto{\pgfqpoint{1.728072in}{1.476127in}}%
\pgfpathlineto{\pgfqpoint{1.965552in}{1.476127in}}%
\pgfpathlineto{\pgfqpoint{1.965552in}{1.479036in}}%
\pgfpathlineto{\pgfqpoint{1.728072in}{1.479036in}}%
\pgfpathlineto{\pgfqpoint{1.728072in}{1.476127in}}%
\pgfpathclose%
\pgfusepath{stroke,fill}%
\end{pgfscope}%
\begin{pgfscope}%
\pgfpathrectangle{\pgfqpoint{0.868056in}{0.555556in}}{\pgfqpoint{3.993056in}{1.888889in}}%
\pgfusepath{clip}%
\pgfsetbuttcap%
\pgfsetmiterjoin%
\definecolor{currentfill}{rgb}{0.447059,0.447059,0.447059}%
\pgfsetfillcolor{currentfill}%
\pgfsetlinewidth{1.003750pt}%
\definecolor{currentstroke}{rgb}{0.266667,0.266667,0.266667}%
\pgfsetstrokecolor{currentstroke}%
\pgfsetdash{}{0pt}%
\pgfpathmoveto{\pgfqpoint{2.067329in}{1.584627in}}%
\pgfpathlineto{\pgfqpoint{2.304809in}{1.584627in}}%
\pgfpathlineto{\pgfqpoint{2.304809in}{1.588591in}}%
\pgfpathlineto{\pgfqpoint{2.067329in}{1.588591in}}%
\pgfpathlineto{\pgfqpoint{2.067329in}{1.584627in}}%
\pgfpathclose%
\pgfusepath{stroke,fill}%
\end{pgfscope}%
\begin{pgfscope}%
\pgfpathrectangle{\pgfqpoint{0.868056in}{0.555556in}}{\pgfqpoint{3.993056in}{1.888889in}}%
\pgfusepath{clip}%
\pgfsetbuttcap%
\pgfsetmiterjoin%
\definecolor{currentfill}{rgb}{0.447059,0.447059,0.447059}%
\pgfsetfillcolor{currentfill}%
\pgfsetlinewidth{1.003750pt}%
\definecolor{currentstroke}{rgb}{0.266667,0.266667,0.266667}%
\pgfsetstrokecolor{currentstroke}%
\pgfsetdash{}{0pt}%
\pgfpathmoveto{\pgfqpoint{2.406586in}{1.635958in}}%
\pgfpathlineto{\pgfqpoint{2.644066in}{1.635958in}}%
\pgfpathlineto{\pgfqpoint{2.644066in}{1.639591in}}%
\pgfpathlineto{\pgfqpoint{2.406586in}{1.639591in}}%
\pgfpathlineto{\pgfqpoint{2.406586in}{1.635958in}}%
\pgfpathclose%
\pgfusepath{stroke,fill}%
\end{pgfscope}%
\begin{pgfscope}%
\pgfpathrectangle{\pgfqpoint{0.868056in}{0.555556in}}{\pgfqpoint{3.993056in}{1.888889in}}%
\pgfusepath{clip}%
\pgfsetbuttcap%
\pgfsetmiterjoin%
\definecolor{currentfill}{rgb}{0.447059,0.447059,0.447059}%
\pgfsetfillcolor{currentfill}%
\pgfsetlinewidth{1.003750pt}%
\definecolor{currentstroke}{rgb}{0.266667,0.266667,0.266667}%
\pgfsetstrokecolor{currentstroke}%
\pgfsetdash{}{0pt}%
\pgfpathmoveto{\pgfqpoint{2.745843in}{1.831251in}}%
\pgfpathlineto{\pgfqpoint{2.983323in}{1.831251in}}%
\pgfpathlineto{\pgfqpoint{2.983323in}{1.835495in}}%
\pgfpathlineto{\pgfqpoint{2.745843in}{1.835495in}}%
\pgfpathlineto{\pgfqpoint{2.745843in}{1.831251in}}%
\pgfpathclose%
\pgfusepath{stroke,fill}%
\end{pgfscope}%
\begin{pgfscope}%
\pgfpathrectangle{\pgfqpoint{0.868056in}{0.555556in}}{\pgfqpoint{3.993056in}{1.888889in}}%
\pgfusepath{clip}%
\pgfsetbuttcap%
\pgfsetmiterjoin%
\definecolor{currentfill}{rgb}{0.447059,0.447059,0.447059}%
\pgfsetfillcolor{currentfill}%
\pgfsetlinewidth{1.003750pt}%
\definecolor{currentstroke}{rgb}{0.266667,0.266667,0.266667}%
\pgfsetstrokecolor{currentstroke}%
\pgfsetdash{}{0pt}%
\pgfpathmoveto{\pgfqpoint{3.085100in}{1.558832in}}%
\pgfpathlineto{\pgfqpoint{3.322580in}{1.558832in}}%
\pgfpathlineto{\pgfqpoint{3.322580in}{1.564815in}}%
\pgfpathlineto{\pgfqpoint{3.085100in}{1.564815in}}%
\pgfpathlineto{\pgfqpoint{3.085100in}{1.558832in}}%
\pgfpathclose%
\pgfusepath{stroke,fill}%
\end{pgfscope}%
\begin{pgfscope}%
\pgfpathrectangle{\pgfqpoint{0.868056in}{0.555556in}}{\pgfqpoint{3.993056in}{1.888889in}}%
\pgfusepath{clip}%
\pgfsetbuttcap%
\pgfsetmiterjoin%
\definecolor{currentfill}{rgb}{0.447059,0.447059,0.447059}%
\pgfsetfillcolor{currentfill}%
\pgfsetlinewidth{1.003750pt}%
\definecolor{currentstroke}{rgb}{0.266667,0.266667,0.266667}%
\pgfsetstrokecolor{currentstroke}%
\pgfsetdash{}{0pt}%
\pgfpathmoveto{\pgfqpoint{3.424357in}{1.860058in}}%
\pgfpathlineto{\pgfqpoint{3.661837in}{1.860058in}}%
\pgfpathlineto{\pgfqpoint{3.661837in}{1.866466in}}%
\pgfpathlineto{\pgfqpoint{3.424357in}{1.866466in}}%
\pgfpathlineto{\pgfqpoint{3.424357in}{1.860058in}}%
\pgfpathclose%
\pgfusepath{stroke,fill}%
\end{pgfscope}%
\begin{pgfscope}%
\pgfpathrectangle{\pgfqpoint{0.868056in}{0.555556in}}{\pgfqpoint{3.993056in}{1.888889in}}%
\pgfusepath{clip}%
\pgfsetbuttcap%
\pgfsetmiterjoin%
\definecolor{currentfill}{rgb}{0.447059,0.447059,0.447059}%
\pgfsetfillcolor{currentfill}%
\pgfsetlinewidth{1.003750pt}%
\definecolor{currentstroke}{rgb}{0.266667,0.266667,0.266667}%
\pgfsetstrokecolor{currentstroke}%
\pgfsetdash{}{0pt}%
\pgfpathmoveto{\pgfqpoint{3.763615in}{2.082730in}}%
\pgfpathlineto{\pgfqpoint{4.001094in}{2.082730in}}%
\pgfpathlineto{\pgfqpoint{4.001094in}{2.089779in}}%
\pgfpathlineto{\pgfqpoint{3.763615in}{2.089779in}}%
\pgfpathlineto{\pgfqpoint{3.763615in}{2.082730in}}%
\pgfpathclose%
\pgfusepath{stroke,fill}%
\end{pgfscope}%
\begin{pgfscope}%
\pgfpathrectangle{\pgfqpoint{0.868056in}{0.555556in}}{\pgfqpoint{3.993056in}{1.888889in}}%
\pgfusepath{clip}%
\pgfsetbuttcap%
\pgfsetmiterjoin%
\definecolor{currentfill}{rgb}{0.447059,0.447059,0.447059}%
\pgfsetfillcolor{currentfill}%
\pgfsetlinewidth{1.003750pt}%
\definecolor{currentstroke}{rgb}{0.266667,0.266667,0.266667}%
\pgfsetstrokecolor{currentstroke}%
\pgfsetdash{}{0pt}%
\pgfpathmoveto{\pgfqpoint{4.102872in}{2.161678in}}%
\pgfpathlineto{\pgfqpoint{4.340352in}{2.161678in}}%
\pgfpathlineto{\pgfqpoint{4.340352in}{2.169462in}}%
\pgfpathlineto{\pgfqpoint{4.102872in}{2.169462in}}%
\pgfpathlineto{\pgfqpoint{4.102872in}{2.161678in}}%
\pgfpathclose%
\pgfusepath{stroke,fill}%
\end{pgfscope}%
\begin{pgfscope}%
\pgfpathrectangle{\pgfqpoint{0.868056in}{0.555556in}}{\pgfqpoint{3.993056in}{1.888889in}}%
\pgfusepath{clip}%
\pgfsetbuttcap%
\pgfsetmiterjoin%
\definecolor{currentfill}{rgb}{0.447059,0.447059,0.447059}%
\pgfsetfillcolor{currentfill}%
\pgfsetlinewidth{1.003750pt}%
\definecolor{currentstroke}{rgb}{0.266667,0.266667,0.266667}%
\pgfsetstrokecolor{currentstroke}%
\pgfsetdash{}{0pt}%
\pgfpathmoveto{\pgfqpoint{4.442129in}{2.324541in}}%
\pgfpathlineto{\pgfqpoint{4.679609in}{2.324541in}}%
\pgfpathlineto{\pgfqpoint{4.679609in}{2.331383in}}%
\pgfpathlineto{\pgfqpoint{4.442129in}{2.331383in}}%
\pgfpathlineto{\pgfqpoint{4.442129in}{2.324541in}}%
\pgfpathclose%
\pgfusepath{stroke,fill}%
\end{pgfscope}%
\begin{pgfscope}%
\pgfpathrectangle{\pgfqpoint{0.868056in}{0.555556in}}{\pgfqpoint{3.993056in}{1.888889in}}%
\pgfusepath{clip}%
\pgfsetbuttcap%
\pgfsetmiterjoin%
\definecolor{currentfill}{rgb}{0.447059,0.447059,0.447059}%
\pgfsetfillcolor{currentfill}%
\pgfsetlinewidth{1.003750pt}%
\definecolor{currentstroke}{rgb}{0.266667,0.266667,0.266667}%
\pgfsetstrokecolor{currentstroke}%
\pgfsetdash{}{0pt}%
\pgfpathmoveto{\pgfqpoint{1.049558in}{2.180807in}}%
\pgfpathlineto{\pgfqpoint{1.287038in}{2.180807in}}%
\pgfpathlineto{\pgfqpoint{1.287038in}{2.184233in}}%
\pgfpathlineto{\pgfqpoint{1.049558in}{2.184233in}}%
\pgfpathlineto{\pgfqpoint{1.049558in}{2.180807in}}%
\pgfpathclose%
\pgfusepath{stroke,fill}%
\end{pgfscope}%
\begin{pgfscope}%
\pgfpathrectangle{\pgfqpoint{0.868056in}{0.555556in}}{\pgfqpoint{3.993056in}{1.888889in}}%
\pgfusepath{clip}%
\pgfsetbuttcap%
\pgfsetmiterjoin%
\definecolor{currentfill}{rgb}{0.447059,0.447059,0.447059}%
\pgfsetfillcolor{currentfill}%
\pgfsetlinewidth{1.003750pt}%
\definecolor{currentstroke}{rgb}{0.266667,0.266667,0.266667}%
\pgfsetstrokecolor{currentstroke}%
\pgfsetdash{}{0pt}%
\pgfpathmoveto{\pgfqpoint{1.388815in}{1.773710in}}%
\pgfpathlineto{\pgfqpoint{1.626295in}{1.773710in}}%
\pgfpathlineto{\pgfqpoint{1.626295in}{1.776712in}}%
\pgfpathlineto{\pgfqpoint{1.388815in}{1.776712in}}%
\pgfpathlineto{\pgfqpoint{1.388815in}{1.773710in}}%
\pgfpathclose%
\pgfusepath{stroke,fill}%
\end{pgfscope}%
\begin{pgfscope}%
\pgfpathrectangle{\pgfqpoint{0.868056in}{0.555556in}}{\pgfqpoint{3.993056in}{1.888889in}}%
\pgfusepath{clip}%
\pgfsetbuttcap%
\pgfsetmiterjoin%
\definecolor{currentfill}{rgb}{0.447059,0.447059,0.447059}%
\pgfsetfillcolor{currentfill}%
\pgfsetlinewidth{1.003750pt}%
\definecolor{currentstroke}{rgb}{0.266667,0.266667,0.266667}%
\pgfsetstrokecolor{currentstroke}%
\pgfsetdash{}{0pt}%
\pgfpathmoveto{\pgfqpoint{1.728072in}{1.479036in}}%
\pgfpathlineto{\pgfqpoint{1.965552in}{1.479036in}}%
\pgfpathlineto{\pgfqpoint{1.965552in}{1.483249in}}%
\pgfpathlineto{\pgfqpoint{1.728072in}{1.483249in}}%
\pgfpathlineto{\pgfqpoint{1.728072in}{1.479036in}}%
\pgfpathclose%
\pgfusepath{stroke,fill}%
\end{pgfscope}%
\begin{pgfscope}%
\pgfpathrectangle{\pgfqpoint{0.868056in}{0.555556in}}{\pgfqpoint{3.993056in}{1.888889in}}%
\pgfusepath{clip}%
\pgfsetbuttcap%
\pgfsetmiterjoin%
\definecolor{currentfill}{rgb}{0.447059,0.447059,0.447059}%
\pgfsetfillcolor{currentfill}%
\pgfsetlinewidth{1.003750pt}%
\definecolor{currentstroke}{rgb}{0.266667,0.266667,0.266667}%
\pgfsetstrokecolor{currentstroke}%
\pgfsetdash{}{0pt}%
\pgfpathmoveto{\pgfqpoint{2.067329in}{1.588591in}}%
\pgfpathlineto{\pgfqpoint{2.304809in}{1.588591in}}%
\pgfpathlineto{\pgfqpoint{2.304809in}{1.591510in}}%
\pgfpathlineto{\pgfqpoint{2.067329in}{1.591510in}}%
\pgfpathlineto{\pgfqpoint{2.067329in}{1.588591in}}%
\pgfpathclose%
\pgfusepath{stroke,fill}%
\end{pgfscope}%
\begin{pgfscope}%
\pgfpathrectangle{\pgfqpoint{0.868056in}{0.555556in}}{\pgfqpoint{3.993056in}{1.888889in}}%
\pgfusepath{clip}%
\pgfsetbuttcap%
\pgfsetmiterjoin%
\definecolor{currentfill}{rgb}{0.447059,0.447059,0.447059}%
\pgfsetfillcolor{currentfill}%
\pgfsetlinewidth{1.003750pt}%
\definecolor{currentstroke}{rgb}{0.266667,0.266667,0.266667}%
\pgfsetstrokecolor{currentstroke}%
\pgfsetdash{}{0pt}%
\pgfpathmoveto{\pgfqpoint{2.406586in}{1.639591in}}%
\pgfpathlineto{\pgfqpoint{2.644066in}{1.639591in}}%
\pgfpathlineto{\pgfqpoint{2.644066in}{1.642282in}}%
\pgfpathlineto{\pgfqpoint{2.406586in}{1.642282in}}%
\pgfpathlineto{\pgfqpoint{2.406586in}{1.639591in}}%
\pgfpathclose%
\pgfusepath{stroke,fill}%
\end{pgfscope}%
\begin{pgfscope}%
\pgfpathrectangle{\pgfqpoint{0.868056in}{0.555556in}}{\pgfqpoint{3.993056in}{1.888889in}}%
\pgfusepath{clip}%
\pgfsetbuttcap%
\pgfsetmiterjoin%
\definecolor{currentfill}{rgb}{0.447059,0.447059,0.447059}%
\pgfsetfillcolor{currentfill}%
\pgfsetlinewidth{1.003750pt}%
\definecolor{currentstroke}{rgb}{0.266667,0.266667,0.266667}%
\pgfsetstrokecolor{currentstroke}%
\pgfsetdash{}{0pt}%
\pgfpathmoveto{\pgfqpoint{2.745843in}{1.835495in}}%
\pgfpathlineto{\pgfqpoint{2.983323in}{1.835495in}}%
\pgfpathlineto{\pgfqpoint{2.983323in}{1.838528in}}%
\pgfpathlineto{\pgfqpoint{2.745843in}{1.838528in}}%
\pgfpathlineto{\pgfqpoint{2.745843in}{1.835495in}}%
\pgfpathclose%
\pgfusepath{stroke,fill}%
\end{pgfscope}%
\begin{pgfscope}%
\pgfpathrectangle{\pgfqpoint{0.868056in}{0.555556in}}{\pgfqpoint{3.993056in}{1.888889in}}%
\pgfusepath{clip}%
\pgfsetbuttcap%
\pgfsetmiterjoin%
\definecolor{currentfill}{rgb}{0.447059,0.447059,0.447059}%
\pgfsetfillcolor{currentfill}%
\pgfsetlinewidth{1.003750pt}%
\definecolor{currentstroke}{rgb}{0.266667,0.266667,0.266667}%
\pgfsetstrokecolor{currentstroke}%
\pgfsetdash{}{0pt}%
\pgfpathmoveto{\pgfqpoint{3.085100in}{1.564815in}}%
\pgfpathlineto{\pgfqpoint{3.322580in}{1.564815in}}%
\pgfpathlineto{\pgfqpoint{3.322580in}{1.567817in}}%
\pgfpathlineto{\pgfqpoint{3.085100in}{1.567817in}}%
\pgfpathlineto{\pgfqpoint{3.085100in}{1.564815in}}%
\pgfpathclose%
\pgfusepath{stroke,fill}%
\end{pgfscope}%
\begin{pgfscope}%
\pgfpathrectangle{\pgfqpoint{0.868056in}{0.555556in}}{\pgfqpoint{3.993056in}{1.888889in}}%
\pgfusepath{clip}%
\pgfsetbuttcap%
\pgfsetmiterjoin%
\definecolor{currentfill}{rgb}{0.447059,0.447059,0.447059}%
\pgfsetfillcolor{currentfill}%
\pgfsetlinewidth{1.003750pt}%
\definecolor{currentstroke}{rgb}{0.266667,0.266667,0.266667}%
\pgfsetstrokecolor{currentstroke}%
\pgfsetdash{}{0pt}%
\pgfpathmoveto{\pgfqpoint{3.424357in}{1.866466in}}%
\pgfpathlineto{\pgfqpoint{3.661837in}{1.866466in}}%
\pgfpathlineto{\pgfqpoint{3.661837in}{1.869799in}}%
\pgfpathlineto{\pgfqpoint{3.424357in}{1.869799in}}%
\pgfpathlineto{\pgfqpoint{3.424357in}{1.866466in}}%
\pgfpathclose%
\pgfusepath{stroke,fill}%
\end{pgfscope}%
\begin{pgfscope}%
\pgfpathrectangle{\pgfqpoint{0.868056in}{0.555556in}}{\pgfqpoint{3.993056in}{1.888889in}}%
\pgfusepath{clip}%
\pgfsetbuttcap%
\pgfsetmiterjoin%
\definecolor{currentfill}{rgb}{0.447059,0.447059,0.447059}%
\pgfsetfillcolor{currentfill}%
\pgfsetlinewidth{1.003750pt}%
\definecolor{currentstroke}{rgb}{0.266667,0.266667,0.266667}%
\pgfsetstrokecolor{currentstroke}%
\pgfsetdash{}{0pt}%
\pgfpathmoveto{\pgfqpoint{3.763615in}{2.089779in}}%
\pgfpathlineto{\pgfqpoint{4.001094in}{2.089779in}}%
\pgfpathlineto{\pgfqpoint{4.001094in}{2.093733in}}%
\pgfpathlineto{\pgfqpoint{3.763615in}{2.093733in}}%
\pgfpathlineto{\pgfqpoint{3.763615in}{2.089779in}}%
\pgfpathclose%
\pgfusepath{stroke,fill}%
\end{pgfscope}%
\begin{pgfscope}%
\pgfpathrectangle{\pgfqpoint{0.868056in}{0.555556in}}{\pgfqpoint{3.993056in}{1.888889in}}%
\pgfusepath{clip}%
\pgfsetbuttcap%
\pgfsetmiterjoin%
\definecolor{currentfill}{rgb}{0.447059,0.447059,0.447059}%
\pgfsetfillcolor{currentfill}%
\pgfsetlinewidth{1.003750pt}%
\definecolor{currentstroke}{rgb}{0.266667,0.266667,0.266667}%
\pgfsetstrokecolor{currentstroke}%
\pgfsetdash{}{0pt}%
\pgfpathmoveto{\pgfqpoint{4.102872in}{2.169462in}}%
\pgfpathlineto{\pgfqpoint{4.340352in}{2.169462in}}%
\pgfpathlineto{\pgfqpoint{4.340352in}{2.175321in}}%
\pgfpathlineto{\pgfqpoint{4.102872in}{2.175321in}}%
\pgfpathlineto{\pgfqpoint{4.102872in}{2.169462in}}%
\pgfpathclose%
\pgfusepath{stroke,fill}%
\end{pgfscope}%
\begin{pgfscope}%
\pgfpathrectangle{\pgfqpoint{0.868056in}{0.555556in}}{\pgfqpoint{3.993056in}{1.888889in}}%
\pgfusepath{clip}%
\pgfsetbuttcap%
\pgfsetmiterjoin%
\definecolor{currentfill}{rgb}{0.447059,0.447059,0.447059}%
\pgfsetfillcolor{currentfill}%
\pgfsetlinewidth{1.003750pt}%
\definecolor{currentstroke}{rgb}{0.266667,0.266667,0.266667}%
\pgfsetstrokecolor{currentstroke}%
\pgfsetdash{}{0pt}%
\pgfpathmoveto{\pgfqpoint{4.442129in}{2.331383in}}%
\pgfpathlineto{\pgfqpoint{4.679609in}{2.331383in}}%
\pgfpathlineto{\pgfqpoint{4.679609in}{2.339923in}}%
\pgfpathlineto{\pgfqpoint{4.442129in}{2.339923in}}%
\pgfpathlineto{\pgfqpoint{4.442129in}{2.331383in}}%
\pgfpathclose%
\pgfusepath{stroke,fill}%
\end{pgfscope}%
\begin{pgfscope}%
\pgfpathrectangle{\pgfqpoint{0.868056in}{0.555556in}}{\pgfqpoint{3.993056in}{1.888889in}}%
\pgfusepath{clip}%
\pgfsetbuttcap%
\pgfsetmiterjoin%
\definecolor{currentfill}{rgb}{0.447059,0.447059,0.447059}%
\pgfsetfillcolor{currentfill}%
\pgfsetlinewidth{1.003750pt}%
\definecolor{currentstroke}{rgb}{0.266667,0.266667,0.266667}%
\pgfsetstrokecolor{currentstroke}%
\pgfsetdash{}{0pt}%
\pgfpathmoveto{\pgfqpoint{1.049558in}{2.184233in}}%
\pgfpathlineto{\pgfqpoint{1.287038in}{2.184233in}}%
\pgfpathlineto{\pgfqpoint{1.287038in}{2.184233in}}%
\pgfpathlineto{\pgfqpoint{1.049558in}{2.184233in}}%
\pgfpathlineto{\pgfqpoint{1.049558in}{2.184233in}}%
\pgfpathclose%
\pgfusepath{stroke,fill}%
\end{pgfscope}%
\begin{pgfscope}%
\pgfpathrectangle{\pgfqpoint{0.868056in}{0.555556in}}{\pgfqpoint{3.993056in}{1.888889in}}%
\pgfusepath{clip}%
\pgfsetbuttcap%
\pgfsetmiterjoin%
\definecolor{currentfill}{rgb}{0.447059,0.447059,0.447059}%
\pgfsetfillcolor{currentfill}%
\pgfsetlinewidth{1.003750pt}%
\definecolor{currentstroke}{rgb}{0.266667,0.266667,0.266667}%
\pgfsetstrokecolor{currentstroke}%
\pgfsetdash{}{0pt}%
\pgfpathmoveto{\pgfqpoint{1.388815in}{1.776712in}}%
\pgfpathlineto{\pgfqpoint{1.626295in}{1.776712in}}%
\pgfpathlineto{\pgfqpoint{1.626295in}{1.776712in}}%
\pgfpathlineto{\pgfqpoint{1.388815in}{1.776712in}}%
\pgfpathlineto{\pgfqpoint{1.388815in}{1.776712in}}%
\pgfpathclose%
\pgfusepath{stroke,fill}%
\end{pgfscope}%
\begin{pgfscope}%
\pgfpathrectangle{\pgfqpoint{0.868056in}{0.555556in}}{\pgfqpoint{3.993056in}{1.888889in}}%
\pgfusepath{clip}%
\pgfsetbuttcap%
\pgfsetmiterjoin%
\definecolor{currentfill}{rgb}{0.447059,0.447059,0.447059}%
\pgfsetfillcolor{currentfill}%
\pgfsetlinewidth{1.003750pt}%
\definecolor{currentstroke}{rgb}{0.266667,0.266667,0.266667}%
\pgfsetstrokecolor{currentstroke}%
\pgfsetdash{}{0pt}%
\pgfpathmoveto{\pgfqpoint{1.728072in}{1.483249in}}%
\pgfpathlineto{\pgfqpoint{1.965552in}{1.483249in}}%
\pgfpathlineto{\pgfqpoint{1.965552in}{1.486489in}}%
\pgfpathlineto{\pgfqpoint{1.728072in}{1.486489in}}%
\pgfpathlineto{\pgfqpoint{1.728072in}{1.483249in}}%
\pgfpathclose%
\pgfusepath{stroke,fill}%
\end{pgfscope}%
\begin{pgfscope}%
\pgfpathrectangle{\pgfqpoint{0.868056in}{0.555556in}}{\pgfqpoint{3.993056in}{1.888889in}}%
\pgfusepath{clip}%
\pgfsetbuttcap%
\pgfsetmiterjoin%
\definecolor{currentfill}{rgb}{0.447059,0.447059,0.447059}%
\pgfsetfillcolor{currentfill}%
\pgfsetlinewidth{1.003750pt}%
\definecolor{currentstroke}{rgb}{0.266667,0.266667,0.266667}%
\pgfsetstrokecolor{currentstroke}%
\pgfsetdash{}{0pt}%
\pgfpathmoveto{\pgfqpoint{2.067329in}{1.591510in}}%
\pgfpathlineto{\pgfqpoint{2.304809in}{1.591510in}}%
\pgfpathlineto{\pgfqpoint{2.304809in}{1.594088in}}%
\pgfpathlineto{\pgfqpoint{2.067329in}{1.594088in}}%
\pgfpathlineto{\pgfqpoint{2.067329in}{1.591510in}}%
\pgfpathclose%
\pgfusepath{stroke,fill}%
\end{pgfscope}%
\begin{pgfscope}%
\pgfpathrectangle{\pgfqpoint{0.868056in}{0.555556in}}{\pgfqpoint{3.993056in}{1.888889in}}%
\pgfusepath{clip}%
\pgfsetbuttcap%
\pgfsetmiterjoin%
\definecolor{currentfill}{rgb}{0.447059,0.447059,0.447059}%
\pgfsetfillcolor{currentfill}%
\pgfsetlinewidth{1.003750pt}%
\definecolor{currentstroke}{rgb}{0.266667,0.266667,0.266667}%
\pgfsetstrokecolor{currentstroke}%
\pgfsetdash{}{0pt}%
\pgfpathmoveto{\pgfqpoint{2.406586in}{1.642282in}}%
\pgfpathlineto{\pgfqpoint{2.644066in}{1.642282in}}%
\pgfpathlineto{\pgfqpoint{2.644066in}{1.644166in}}%
\pgfpathlineto{\pgfqpoint{2.406586in}{1.644166in}}%
\pgfpathlineto{\pgfqpoint{2.406586in}{1.642282in}}%
\pgfpathclose%
\pgfusepath{stroke,fill}%
\end{pgfscope}%
\begin{pgfscope}%
\pgfpathrectangle{\pgfqpoint{0.868056in}{0.555556in}}{\pgfqpoint{3.993056in}{1.888889in}}%
\pgfusepath{clip}%
\pgfsetbuttcap%
\pgfsetmiterjoin%
\definecolor{currentfill}{rgb}{0.447059,0.447059,0.447059}%
\pgfsetfillcolor{currentfill}%
\pgfsetlinewidth{1.003750pt}%
\definecolor{currentstroke}{rgb}{0.266667,0.266667,0.266667}%
\pgfsetstrokecolor{currentstroke}%
\pgfsetdash{}{0pt}%
\pgfpathmoveto{\pgfqpoint{2.745843in}{1.838528in}}%
\pgfpathlineto{\pgfqpoint{2.983323in}{1.838528in}}%
\pgfpathlineto{\pgfqpoint{2.983323in}{1.842762in}}%
\pgfpathlineto{\pgfqpoint{2.745843in}{1.842762in}}%
\pgfpathlineto{\pgfqpoint{2.745843in}{1.838528in}}%
\pgfpathclose%
\pgfusepath{stroke,fill}%
\end{pgfscope}%
\begin{pgfscope}%
\pgfpathrectangle{\pgfqpoint{0.868056in}{0.555556in}}{\pgfqpoint{3.993056in}{1.888889in}}%
\pgfusepath{clip}%
\pgfsetbuttcap%
\pgfsetmiterjoin%
\definecolor{currentfill}{rgb}{0.447059,0.447059,0.447059}%
\pgfsetfillcolor{currentfill}%
\pgfsetlinewidth{1.003750pt}%
\definecolor{currentstroke}{rgb}{0.266667,0.266667,0.266667}%
\pgfsetstrokecolor{currentstroke}%
\pgfsetdash{}{0pt}%
\pgfpathmoveto{\pgfqpoint{3.085100in}{1.567817in}}%
\pgfpathlineto{\pgfqpoint{3.322580in}{1.567817in}}%
\pgfpathlineto{\pgfqpoint{3.322580in}{1.570467in}}%
\pgfpathlineto{\pgfqpoint{3.085100in}{1.570467in}}%
\pgfpathlineto{\pgfqpoint{3.085100in}{1.567817in}}%
\pgfpathclose%
\pgfusepath{stroke,fill}%
\end{pgfscope}%
\begin{pgfscope}%
\pgfpathrectangle{\pgfqpoint{0.868056in}{0.555556in}}{\pgfqpoint{3.993056in}{1.888889in}}%
\pgfusepath{clip}%
\pgfsetbuttcap%
\pgfsetmiterjoin%
\definecolor{currentfill}{rgb}{0.447059,0.447059,0.447059}%
\pgfsetfillcolor{currentfill}%
\pgfsetlinewidth{1.003750pt}%
\definecolor{currentstroke}{rgb}{0.266667,0.266667,0.266667}%
\pgfsetstrokecolor{currentstroke}%
\pgfsetdash{}{0pt}%
\pgfpathmoveto{\pgfqpoint{3.424357in}{1.869799in}}%
\pgfpathlineto{\pgfqpoint{3.661837in}{1.869799in}}%
\pgfpathlineto{\pgfqpoint{3.661837in}{1.873194in}}%
\pgfpathlineto{\pgfqpoint{3.424357in}{1.873194in}}%
\pgfpathlineto{\pgfqpoint{3.424357in}{1.869799in}}%
\pgfpathclose%
\pgfusepath{stroke,fill}%
\end{pgfscope}%
\begin{pgfscope}%
\pgfpathrectangle{\pgfqpoint{0.868056in}{0.555556in}}{\pgfqpoint{3.993056in}{1.888889in}}%
\pgfusepath{clip}%
\pgfsetbuttcap%
\pgfsetmiterjoin%
\definecolor{currentfill}{rgb}{0.447059,0.447059,0.447059}%
\pgfsetfillcolor{currentfill}%
\pgfsetlinewidth{1.003750pt}%
\definecolor{currentstroke}{rgb}{0.266667,0.266667,0.266667}%
\pgfsetstrokecolor{currentstroke}%
\pgfsetdash{}{0pt}%
\pgfpathmoveto{\pgfqpoint{3.763615in}{2.093733in}}%
\pgfpathlineto{\pgfqpoint{4.001094in}{2.093733in}}%
\pgfpathlineto{\pgfqpoint{4.001094in}{2.097222in}}%
\pgfpathlineto{\pgfqpoint{3.763615in}{2.097222in}}%
\pgfpathlineto{\pgfqpoint{3.763615in}{2.093733in}}%
\pgfpathclose%
\pgfusepath{stroke,fill}%
\end{pgfscope}%
\begin{pgfscope}%
\pgfpathrectangle{\pgfqpoint{0.868056in}{0.555556in}}{\pgfqpoint{3.993056in}{1.888889in}}%
\pgfusepath{clip}%
\pgfsetbuttcap%
\pgfsetmiterjoin%
\definecolor{currentfill}{rgb}{0.447059,0.447059,0.447059}%
\pgfsetfillcolor{currentfill}%
\pgfsetlinewidth{1.003750pt}%
\definecolor{currentstroke}{rgb}{0.266667,0.266667,0.266667}%
\pgfsetstrokecolor{currentstroke}%
\pgfsetdash{}{0pt}%
\pgfpathmoveto{\pgfqpoint{4.102872in}{2.175321in}}%
\pgfpathlineto{\pgfqpoint{4.340352in}{2.175321in}}%
\pgfpathlineto{\pgfqpoint{4.340352in}{2.178001in}}%
\pgfpathlineto{\pgfqpoint{4.102872in}{2.178001in}}%
\pgfpathlineto{\pgfqpoint{4.102872in}{2.175321in}}%
\pgfpathclose%
\pgfusepath{stroke,fill}%
\end{pgfscope}%
\begin{pgfscope}%
\pgfpathrectangle{\pgfqpoint{0.868056in}{0.555556in}}{\pgfqpoint{3.993056in}{1.888889in}}%
\pgfusepath{clip}%
\pgfsetbuttcap%
\pgfsetmiterjoin%
\definecolor{currentfill}{rgb}{0.447059,0.447059,0.447059}%
\pgfsetfillcolor{currentfill}%
\pgfsetlinewidth{1.003750pt}%
\definecolor{currentstroke}{rgb}{0.266667,0.266667,0.266667}%
\pgfsetstrokecolor{currentstroke}%
\pgfsetdash{}{0pt}%
\pgfpathmoveto{\pgfqpoint{4.442129in}{2.339923in}}%
\pgfpathlineto{\pgfqpoint{4.679609in}{2.339923in}}%
\pgfpathlineto{\pgfqpoint{4.679609in}{2.343587in}}%
\pgfpathlineto{\pgfqpoint{4.442129in}{2.343587in}}%
\pgfpathlineto{\pgfqpoint{4.442129in}{2.339923in}}%
\pgfpathclose%
\pgfusepath{stroke,fill}%
\end{pgfscope}%
\begin{pgfscope}%
\pgfpathrectangle{\pgfqpoint{0.868056in}{0.555556in}}{\pgfqpoint{3.993056in}{1.888889in}}%
\pgfusepath{clip}%
\pgfsetbuttcap%
\pgfsetmiterjoin%
\definecolor{currentfill}{rgb}{0.447059,0.447059,0.447059}%
\pgfsetfillcolor{currentfill}%
\pgfsetlinewidth{1.003750pt}%
\definecolor{currentstroke}{rgb}{0.266667,0.266667,0.266667}%
\pgfsetstrokecolor{currentstroke}%
\pgfsetdash{}{0pt}%
\pgfpathmoveto{\pgfqpoint{1.049558in}{2.184233in}}%
\pgfpathlineto{\pgfqpoint{1.287038in}{2.184233in}}%
\pgfpathlineto{\pgfqpoint{1.287038in}{2.188197in}}%
\pgfpathlineto{\pgfqpoint{1.049558in}{2.188197in}}%
\pgfpathlineto{\pgfqpoint{1.049558in}{2.184233in}}%
\pgfpathclose%
\pgfusepath{stroke,fill}%
\end{pgfscope}%
\begin{pgfscope}%
\pgfpathrectangle{\pgfqpoint{0.868056in}{0.555556in}}{\pgfqpoint{3.993056in}{1.888889in}}%
\pgfusepath{clip}%
\pgfsetbuttcap%
\pgfsetmiterjoin%
\definecolor{currentfill}{rgb}{0.447059,0.447059,0.447059}%
\pgfsetfillcolor{currentfill}%
\pgfsetlinewidth{1.003750pt}%
\definecolor{currentstroke}{rgb}{0.266667,0.266667,0.266667}%
\pgfsetstrokecolor{currentstroke}%
\pgfsetdash{}{0pt}%
\pgfpathmoveto{\pgfqpoint{1.388815in}{1.776712in}}%
\pgfpathlineto{\pgfqpoint{1.626295in}{1.776712in}}%
\pgfpathlineto{\pgfqpoint{1.626295in}{1.780893in}}%
\pgfpathlineto{\pgfqpoint{1.388815in}{1.780893in}}%
\pgfpathlineto{\pgfqpoint{1.388815in}{1.776712in}}%
\pgfpathclose%
\pgfusepath{stroke,fill}%
\end{pgfscope}%
\begin{pgfscope}%
\pgfpathrectangle{\pgfqpoint{0.868056in}{0.555556in}}{\pgfqpoint{3.993056in}{1.888889in}}%
\pgfusepath{clip}%
\pgfsetbuttcap%
\pgfsetmiterjoin%
\definecolor{currentfill}{rgb}{0.447059,0.447059,0.447059}%
\pgfsetfillcolor{currentfill}%
\pgfsetlinewidth{1.003750pt}%
\definecolor{currentstroke}{rgb}{0.266667,0.266667,0.266667}%
\pgfsetstrokecolor{currentstroke}%
\pgfsetdash{}{0pt}%
\pgfpathmoveto{\pgfqpoint{1.728072in}{1.486489in}}%
\pgfpathlineto{\pgfqpoint{1.965552in}{1.486489in}}%
\pgfpathlineto{\pgfqpoint{1.965552in}{1.489873in}}%
\pgfpathlineto{\pgfqpoint{1.728072in}{1.489873in}}%
\pgfpathlineto{\pgfqpoint{1.728072in}{1.486489in}}%
\pgfpathclose%
\pgfusepath{stroke,fill}%
\end{pgfscope}%
\begin{pgfscope}%
\pgfpathrectangle{\pgfqpoint{0.868056in}{0.555556in}}{\pgfqpoint{3.993056in}{1.888889in}}%
\pgfusepath{clip}%
\pgfsetbuttcap%
\pgfsetmiterjoin%
\definecolor{currentfill}{rgb}{0.447059,0.447059,0.447059}%
\pgfsetfillcolor{currentfill}%
\pgfsetlinewidth{1.003750pt}%
\definecolor{currentstroke}{rgb}{0.266667,0.266667,0.266667}%
\pgfsetstrokecolor{currentstroke}%
\pgfsetdash{}{0pt}%
\pgfpathmoveto{\pgfqpoint{2.067329in}{1.594088in}}%
\pgfpathlineto{\pgfqpoint{2.304809in}{1.594088in}}%
\pgfpathlineto{\pgfqpoint{2.304809in}{1.597048in}}%
\pgfpathlineto{\pgfqpoint{2.067329in}{1.597048in}}%
\pgfpathlineto{\pgfqpoint{2.067329in}{1.594088in}}%
\pgfpathclose%
\pgfusepath{stroke,fill}%
\end{pgfscope}%
\begin{pgfscope}%
\pgfpathrectangle{\pgfqpoint{0.868056in}{0.555556in}}{\pgfqpoint{3.993056in}{1.888889in}}%
\pgfusepath{clip}%
\pgfsetbuttcap%
\pgfsetmiterjoin%
\definecolor{currentfill}{rgb}{0.447059,0.447059,0.447059}%
\pgfsetfillcolor{currentfill}%
\pgfsetlinewidth{1.003750pt}%
\definecolor{currentstroke}{rgb}{0.266667,0.266667,0.266667}%
\pgfsetstrokecolor{currentstroke}%
\pgfsetdash{}{0pt}%
\pgfpathmoveto{\pgfqpoint{2.406586in}{1.644166in}}%
\pgfpathlineto{\pgfqpoint{2.644066in}{1.644166in}}%
\pgfpathlineto{\pgfqpoint{2.644066in}{1.647468in}}%
\pgfpathlineto{\pgfqpoint{2.406586in}{1.647468in}}%
\pgfpathlineto{\pgfqpoint{2.406586in}{1.644166in}}%
\pgfpathclose%
\pgfusepath{stroke,fill}%
\end{pgfscope}%
\begin{pgfscope}%
\pgfpathrectangle{\pgfqpoint{0.868056in}{0.555556in}}{\pgfqpoint{3.993056in}{1.888889in}}%
\pgfusepath{clip}%
\pgfsetbuttcap%
\pgfsetmiterjoin%
\definecolor{currentfill}{rgb}{0.447059,0.447059,0.447059}%
\pgfsetfillcolor{currentfill}%
\pgfsetlinewidth{1.003750pt}%
\definecolor{currentstroke}{rgb}{0.266667,0.266667,0.266667}%
\pgfsetstrokecolor{currentstroke}%
\pgfsetdash{}{0pt}%
\pgfpathmoveto{\pgfqpoint{2.745843in}{1.842762in}}%
\pgfpathlineto{\pgfqpoint{2.983323in}{1.842762in}}%
\pgfpathlineto{\pgfqpoint{2.983323in}{1.845784in}}%
\pgfpathlineto{\pgfqpoint{2.745843in}{1.845784in}}%
\pgfpathlineto{\pgfqpoint{2.745843in}{1.842762in}}%
\pgfpathclose%
\pgfusepath{stroke,fill}%
\end{pgfscope}%
\begin{pgfscope}%
\pgfpathrectangle{\pgfqpoint{0.868056in}{0.555556in}}{\pgfqpoint{3.993056in}{1.888889in}}%
\pgfusepath{clip}%
\pgfsetbuttcap%
\pgfsetmiterjoin%
\definecolor{currentfill}{rgb}{0.447059,0.447059,0.447059}%
\pgfsetfillcolor{currentfill}%
\pgfsetlinewidth{1.003750pt}%
\definecolor{currentstroke}{rgb}{0.266667,0.266667,0.266667}%
\pgfsetstrokecolor{currentstroke}%
\pgfsetdash{}{0pt}%
\pgfpathmoveto{\pgfqpoint{3.085100in}{1.570467in}}%
\pgfpathlineto{\pgfqpoint{3.322580in}{1.570467in}}%
\pgfpathlineto{\pgfqpoint{3.322580in}{1.573261in}}%
\pgfpathlineto{\pgfqpoint{3.085100in}{1.573261in}}%
\pgfpathlineto{\pgfqpoint{3.085100in}{1.570467in}}%
\pgfpathclose%
\pgfusepath{stroke,fill}%
\end{pgfscope}%
\begin{pgfscope}%
\pgfpathrectangle{\pgfqpoint{0.868056in}{0.555556in}}{\pgfqpoint{3.993056in}{1.888889in}}%
\pgfusepath{clip}%
\pgfsetbuttcap%
\pgfsetmiterjoin%
\definecolor{currentfill}{rgb}{0.447059,0.447059,0.447059}%
\pgfsetfillcolor{currentfill}%
\pgfsetlinewidth{1.003750pt}%
\definecolor{currentstroke}{rgb}{0.266667,0.266667,0.266667}%
\pgfsetstrokecolor{currentstroke}%
\pgfsetdash{}{0pt}%
\pgfpathmoveto{\pgfqpoint{3.424357in}{1.873194in}}%
\pgfpathlineto{\pgfqpoint{3.661837in}{1.873194in}}%
\pgfpathlineto{\pgfqpoint{3.661837in}{1.875999in}}%
\pgfpathlineto{\pgfqpoint{3.424357in}{1.875999in}}%
\pgfpathlineto{\pgfqpoint{3.424357in}{1.873194in}}%
\pgfpathclose%
\pgfusepath{stroke,fill}%
\end{pgfscope}%
\begin{pgfscope}%
\pgfpathrectangle{\pgfqpoint{0.868056in}{0.555556in}}{\pgfqpoint{3.993056in}{1.888889in}}%
\pgfusepath{clip}%
\pgfsetbuttcap%
\pgfsetmiterjoin%
\definecolor{currentfill}{rgb}{0.447059,0.447059,0.447059}%
\pgfsetfillcolor{currentfill}%
\pgfsetlinewidth{1.003750pt}%
\definecolor{currentstroke}{rgb}{0.266667,0.266667,0.266667}%
\pgfsetstrokecolor{currentstroke}%
\pgfsetdash{}{0pt}%
\pgfpathmoveto{\pgfqpoint{3.763615in}{2.097222in}}%
\pgfpathlineto{\pgfqpoint{4.001094in}{2.097222in}}%
\pgfpathlineto{\pgfqpoint{4.001094in}{2.099602in}}%
\pgfpathlineto{\pgfqpoint{3.763615in}{2.099602in}}%
\pgfpathlineto{\pgfqpoint{3.763615in}{2.097222in}}%
\pgfpathclose%
\pgfusepath{stroke,fill}%
\end{pgfscope}%
\begin{pgfscope}%
\pgfpathrectangle{\pgfqpoint{0.868056in}{0.555556in}}{\pgfqpoint{3.993056in}{1.888889in}}%
\pgfusepath{clip}%
\pgfsetbuttcap%
\pgfsetmiterjoin%
\definecolor{currentfill}{rgb}{0.447059,0.447059,0.447059}%
\pgfsetfillcolor{currentfill}%
\pgfsetlinewidth{1.003750pt}%
\definecolor{currentstroke}{rgb}{0.266667,0.266667,0.266667}%
\pgfsetstrokecolor{currentstroke}%
\pgfsetdash{}{0pt}%
\pgfpathmoveto{\pgfqpoint{4.102872in}{2.178001in}}%
\pgfpathlineto{\pgfqpoint{4.340352in}{2.178001in}}%
\pgfpathlineto{\pgfqpoint{4.340352in}{2.180610in}}%
\pgfpathlineto{\pgfqpoint{4.102872in}{2.180610in}}%
\pgfpathlineto{\pgfqpoint{4.102872in}{2.178001in}}%
\pgfpathclose%
\pgfusepath{stroke,fill}%
\end{pgfscope}%
\begin{pgfscope}%
\pgfpathrectangle{\pgfqpoint{0.868056in}{0.555556in}}{\pgfqpoint{3.993056in}{1.888889in}}%
\pgfusepath{clip}%
\pgfsetbuttcap%
\pgfsetmiterjoin%
\definecolor{currentfill}{rgb}{0.447059,0.447059,0.447059}%
\pgfsetfillcolor{currentfill}%
\pgfsetlinewidth{1.003750pt}%
\definecolor{currentstroke}{rgb}{0.266667,0.266667,0.266667}%
\pgfsetstrokecolor{currentstroke}%
\pgfsetdash{}{0pt}%
\pgfpathmoveto{\pgfqpoint{4.442129in}{2.343587in}}%
\pgfpathlineto{\pgfqpoint{4.679609in}{2.343587in}}%
\pgfpathlineto{\pgfqpoint{4.679609in}{2.345513in}}%
\pgfpathlineto{\pgfqpoint{4.442129in}{2.345513in}}%
\pgfpathlineto{\pgfqpoint{4.442129in}{2.343587in}}%
\pgfpathclose%
\pgfusepath{stroke,fill}%
\end{pgfscope}%
\begin{pgfscope}%
\pgfpathrectangle{\pgfqpoint{0.868056in}{0.555556in}}{\pgfqpoint{3.993056in}{1.888889in}}%
\pgfusepath{clip}%
\pgfsetbuttcap%
\pgfsetmiterjoin%
\definecolor{currentfill}{rgb}{0.447059,0.447059,0.447059}%
\pgfsetfillcolor{currentfill}%
\pgfsetlinewidth{1.003750pt}%
\definecolor{currentstroke}{rgb}{0.266667,0.266667,0.266667}%
\pgfsetstrokecolor{currentstroke}%
\pgfsetdash{}{0pt}%
\pgfpathmoveto{\pgfqpoint{1.049558in}{2.188197in}}%
\pgfpathlineto{\pgfqpoint{1.287038in}{2.188197in}}%
\pgfpathlineto{\pgfqpoint{1.287038in}{2.191613in}}%
\pgfpathlineto{\pgfqpoint{1.049558in}{2.191613in}}%
\pgfpathlineto{\pgfqpoint{1.049558in}{2.188197in}}%
\pgfpathclose%
\pgfusepath{stroke,fill}%
\end{pgfscope}%
\begin{pgfscope}%
\pgfpathrectangle{\pgfqpoint{0.868056in}{0.555556in}}{\pgfqpoint{3.993056in}{1.888889in}}%
\pgfusepath{clip}%
\pgfsetbuttcap%
\pgfsetmiterjoin%
\definecolor{currentfill}{rgb}{0.447059,0.447059,0.447059}%
\pgfsetfillcolor{currentfill}%
\pgfsetlinewidth{1.003750pt}%
\definecolor{currentstroke}{rgb}{0.266667,0.266667,0.266667}%
\pgfsetstrokecolor{currentstroke}%
\pgfsetdash{}{0pt}%
\pgfpathmoveto{\pgfqpoint{1.388815in}{1.780893in}}%
\pgfpathlineto{\pgfqpoint{1.626295in}{1.780893in}}%
\pgfpathlineto{\pgfqpoint{1.626295in}{1.783284in}}%
\pgfpathlineto{\pgfqpoint{1.388815in}{1.783284in}}%
\pgfpathlineto{\pgfqpoint{1.388815in}{1.780893in}}%
\pgfpathclose%
\pgfusepath{stroke,fill}%
\end{pgfscope}%
\begin{pgfscope}%
\pgfpathrectangle{\pgfqpoint{0.868056in}{0.555556in}}{\pgfqpoint{3.993056in}{1.888889in}}%
\pgfusepath{clip}%
\pgfsetbuttcap%
\pgfsetmiterjoin%
\definecolor{currentfill}{rgb}{0.447059,0.447059,0.447059}%
\pgfsetfillcolor{currentfill}%
\pgfsetlinewidth{1.003750pt}%
\definecolor{currentstroke}{rgb}{0.266667,0.266667,0.266667}%
\pgfsetstrokecolor{currentstroke}%
\pgfsetdash{}{0pt}%
\pgfpathmoveto{\pgfqpoint{1.728072in}{1.489873in}}%
\pgfpathlineto{\pgfqpoint{1.965552in}{1.489873in}}%
\pgfpathlineto{\pgfqpoint{1.965552in}{1.491664in}}%
\pgfpathlineto{\pgfqpoint{1.728072in}{1.491664in}}%
\pgfpathlineto{\pgfqpoint{1.728072in}{1.489873in}}%
\pgfpathclose%
\pgfusepath{stroke,fill}%
\end{pgfscope}%
\begin{pgfscope}%
\pgfpathrectangle{\pgfqpoint{0.868056in}{0.555556in}}{\pgfqpoint{3.993056in}{1.888889in}}%
\pgfusepath{clip}%
\pgfsetbuttcap%
\pgfsetmiterjoin%
\definecolor{currentfill}{rgb}{0.447059,0.447059,0.447059}%
\pgfsetfillcolor{currentfill}%
\pgfsetlinewidth{1.003750pt}%
\definecolor{currentstroke}{rgb}{0.266667,0.266667,0.266667}%
\pgfsetstrokecolor{currentstroke}%
\pgfsetdash{}{0pt}%
\pgfpathmoveto{\pgfqpoint{2.067329in}{1.597048in}}%
\pgfpathlineto{\pgfqpoint{2.304809in}{1.597048in}}%
\pgfpathlineto{\pgfqpoint{2.304809in}{1.598466in}}%
\pgfpathlineto{\pgfqpoint{2.067329in}{1.598466in}}%
\pgfpathlineto{\pgfqpoint{2.067329in}{1.597048in}}%
\pgfpathclose%
\pgfusepath{stroke,fill}%
\end{pgfscope}%
\begin{pgfscope}%
\pgfpathrectangle{\pgfqpoint{0.868056in}{0.555556in}}{\pgfqpoint{3.993056in}{1.888889in}}%
\pgfusepath{clip}%
\pgfsetbuttcap%
\pgfsetmiterjoin%
\definecolor{currentfill}{rgb}{0.447059,0.447059,0.447059}%
\pgfsetfillcolor{currentfill}%
\pgfsetlinewidth{1.003750pt}%
\definecolor{currentstroke}{rgb}{0.266667,0.266667,0.266667}%
\pgfsetstrokecolor{currentstroke}%
\pgfsetdash{}{0pt}%
\pgfpathmoveto{\pgfqpoint{2.406586in}{1.647468in}}%
\pgfpathlineto{\pgfqpoint{2.644066in}{1.647468in}}%
\pgfpathlineto{\pgfqpoint{2.644066in}{1.648659in}}%
\pgfpathlineto{\pgfqpoint{2.406586in}{1.648659in}}%
\pgfpathlineto{\pgfqpoint{2.406586in}{1.647468in}}%
\pgfpathclose%
\pgfusepath{stroke,fill}%
\end{pgfscope}%
\begin{pgfscope}%
\pgfpathrectangle{\pgfqpoint{0.868056in}{0.555556in}}{\pgfqpoint{3.993056in}{1.888889in}}%
\pgfusepath{clip}%
\pgfsetbuttcap%
\pgfsetmiterjoin%
\definecolor{currentfill}{rgb}{0.447059,0.447059,0.447059}%
\pgfsetfillcolor{currentfill}%
\pgfsetlinewidth{1.003750pt}%
\definecolor{currentstroke}{rgb}{0.266667,0.266667,0.266667}%
\pgfsetstrokecolor{currentstroke}%
\pgfsetdash{}{0pt}%
\pgfpathmoveto{\pgfqpoint{2.745843in}{1.845784in}}%
\pgfpathlineto{\pgfqpoint{2.983323in}{1.845784in}}%
\pgfpathlineto{\pgfqpoint{2.983323in}{1.847057in}}%
\pgfpathlineto{\pgfqpoint{2.745843in}{1.847057in}}%
\pgfpathlineto{\pgfqpoint{2.745843in}{1.845784in}}%
\pgfpathclose%
\pgfusepath{stroke,fill}%
\end{pgfscope}%
\begin{pgfscope}%
\pgfpathrectangle{\pgfqpoint{0.868056in}{0.555556in}}{\pgfqpoint{3.993056in}{1.888889in}}%
\pgfusepath{clip}%
\pgfsetbuttcap%
\pgfsetmiterjoin%
\definecolor{currentfill}{rgb}{0.447059,0.447059,0.447059}%
\pgfsetfillcolor{currentfill}%
\pgfsetlinewidth{1.003750pt}%
\definecolor{currentstroke}{rgb}{0.266667,0.266667,0.266667}%
\pgfsetstrokecolor{currentstroke}%
\pgfsetdash{}{0pt}%
\pgfpathmoveto{\pgfqpoint{3.085100in}{1.573261in}}%
\pgfpathlineto{\pgfqpoint{3.322580in}{1.573261in}}%
\pgfpathlineto{\pgfqpoint{3.322580in}{1.574193in}}%
\pgfpathlineto{\pgfqpoint{3.085100in}{1.574193in}}%
\pgfpathlineto{\pgfqpoint{3.085100in}{1.573261in}}%
\pgfpathclose%
\pgfusepath{stroke,fill}%
\end{pgfscope}%
\begin{pgfscope}%
\pgfpathrectangle{\pgfqpoint{0.868056in}{0.555556in}}{\pgfqpoint{3.993056in}{1.888889in}}%
\pgfusepath{clip}%
\pgfsetbuttcap%
\pgfsetmiterjoin%
\definecolor{currentfill}{rgb}{0.447059,0.447059,0.447059}%
\pgfsetfillcolor{currentfill}%
\pgfsetlinewidth{1.003750pt}%
\definecolor{currentstroke}{rgb}{0.266667,0.266667,0.266667}%
\pgfsetstrokecolor{currentstroke}%
\pgfsetdash{}{0pt}%
\pgfpathmoveto{\pgfqpoint{3.424357in}{1.875999in}}%
\pgfpathlineto{\pgfqpoint{3.661837in}{1.875999in}}%
\pgfpathlineto{\pgfqpoint{3.661837in}{1.878038in}}%
\pgfpathlineto{\pgfqpoint{3.424357in}{1.878038in}}%
\pgfpathlineto{\pgfqpoint{3.424357in}{1.875999in}}%
\pgfpathclose%
\pgfusepath{stroke,fill}%
\end{pgfscope}%
\begin{pgfscope}%
\pgfpathrectangle{\pgfqpoint{0.868056in}{0.555556in}}{\pgfqpoint{3.993056in}{1.888889in}}%
\pgfusepath{clip}%
\pgfsetbuttcap%
\pgfsetmiterjoin%
\definecolor{currentfill}{rgb}{0.447059,0.447059,0.447059}%
\pgfsetfillcolor{currentfill}%
\pgfsetlinewidth{1.003750pt}%
\definecolor{currentstroke}{rgb}{0.266667,0.266667,0.266667}%
\pgfsetstrokecolor{currentstroke}%
\pgfsetdash{}{0pt}%
\pgfpathmoveto{\pgfqpoint{3.763615in}{2.099602in}}%
\pgfpathlineto{\pgfqpoint{4.001094in}{2.099602in}}%
\pgfpathlineto{\pgfqpoint{4.001094in}{2.102076in}}%
\pgfpathlineto{\pgfqpoint{3.763615in}{2.102076in}}%
\pgfpathlineto{\pgfqpoint{3.763615in}{2.099602in}}%
\pgfpathclose%
\pgfusepath{stroke,fill}%
\end{pgfscope}%
\begin{pgfscope}%
\pgfpathrectangle{\pgfqpoint{0.868056in}{0.555556in}}{\pgfqpoint{3.993056in}{1.888889in}}%
\pgfusepath{clip}%
\pgfsetbuttcap%
\pgfsetmiterjoin%
\definecolor{currentfill}{rgb}{0.447059,0.447059,0.447059}%
\pgfsetfillcolor{currentfill}%
\pgfsetlinewidth{1.003750pt}%
\definecolor{currentstroke}{rgb}{0.266667,0.266667,0.266667}%
\pgfsetstrokecolor{currentstroke}%
\pgfsetdash{}{0pt}%
\pgfpathmoveto{\pgfqpoint{4.102872in}{2.180610in}}%
\pgfpathlineto{\pgfqpoint{4.340352in}{2.180610in}}%
\pgfpathlineto{\pgfqpoint{4.340352in}{2.182659in}}%
\pgfpathlineto{\pgfqpoint{4.102872in}{2.182659in}}%
\pgfpathlineto{\pgfqpoint{4.102872in}{2.180610in}}%
\pgfpathclose%
\pgfusepath{stroke,fill}%
\end{pgfscope}%
\begin{pgfscope}%
\pgfpathrectangle{\pgfqpoint{0.868056in}{0.555556in}}{\pgfqpoint{3.993056in}{1.888889in}}%
\pgfusepath{clip}%
\pgfsetbuttcap%
\pgfsetmiterjoin%
\definecolor{currentfill}{rgb}{0.447059,0.447059,0.447059}%
\pgfsetfillcolor{currentfill}%
\pgfsetlinewidth{1.003750pt}%
\definecolor{currentstroke}{rgb}{0.266667,0.266667,0.266667}%
\pgfsetstrokecolor{currentstroke}%
\pgfsetdash{}{0pt}%
\pgfpathmoveto{\pgfqpoint{4.442129in}{2.345513in}}%
\pgfpathlineto{\pgfqpoint{4.679609in}{2.345513in}}%
\pgfpathlineto{\pgfqpoint{4.679609in}{2.346693in}}%
\pgfpathlineto{\pgfqpoint{4.442129in}{2.346693in}}%
\pgfpathlineto{\pgfqpoint{4.442129in}{2.345513in}}%
\pgfpathclose%
\pgfusepath{stroke,fill}%
\end{pgfscope}%
\begin{pgfscope}%
\pgfpathrectangle{\pgfqpoint{0.868056in}{0.555556in}}{\pgfqpoint{3.993056in}{1.888889in}}%
\pgfusepath{clip}%
\pgfsetbuttcap%
\pgfsetmiterjoin%
\definecolor{currentfill}{rgb}{0.447059,0.447059,0.447059}%
\pgfsetfillcolor{currentfill}%
\pgfsetlinewidth{1.003750pt}%
\definecolor{currentstroke}{rgb}{0.266667,0.266667,0.266667}%
\pgfsetstrokecolor{currentstroke}%
\pgfsetdash{}{0pt}%
\pgfpathmoveto{\pgfqpoint{1.049558in}{2.191613in}}%
\pgfpathlineto{\pgfqpoint{1.287038in}{2.191613in}}%
\pgfpathlineto{\pgfqpoint{1.287038in}{2.195308in}}%
\pgfpathlineto{\pgfqpoint{1.049558in}{2.195308in}}%
\pgfpathlineto{\pgfqpoint{1.049558in}{2.191613in}}%
\pgfpathclose%
\pgfusepath{stroke,fill}%
\end{pgfscope}%
\begin{pgfscope}%
\pgfpathrectangle{\pgfqpoint{0.868056in}{0.555556in}}{\pgfqpoint{3.993056in}{1.888889in}}%
\pgfusepath{clip}%
\pgfsetbuttcap%
\pgfsetmiterjoin%
\definecolor{currentfill}{rgb}{0.447059,0.447059,0.447059}%
\pgfsetfillcolor{currentfill}%
\pgfsetlinewidth{1.003750pt}%
\definecolor{currentstroke}{rgb}{0.266667,0.266667,0.266667}%
\pgfsetstrokecolor{currentstroke}%
\pgfsetdash{}{0pt}%
\pgfpathmoveto{\pgfqpoint{1.388815in}{1.783284in}}%
\pgfpathlineto{\pgfqpoint{1.626295in}{1.783284in}}%
\pgfpathlineto{\pgfqpoint{1.626295in}{1.786617in}}%
\pgfpathlineto{\pgfqpoint{1.388815in}{1.786617in}}%
\pgfpathlineto{\pgfqpoint{1.388815in}{1.783284in}}%
\pgfpathclose%
\pgfusepath{stroke,fill}%
\end{pgfscope}%
\begin{pgfscope}%
\pgfpathrectangle{\pgfqpoint{0.868056in}{0.555556in}}{\pgfqpoint{3.993056in}{1.888889in}}%
\pgfusepath{clip}%
\pgfsetbuttcap%
\pgfsetmiterjoin%
\definecolor{currentfill}{rgb}{0.447059,0.447059,0.447059}%
\pgfsetfillcolor{currentfill}%
\pgfsetlinewidth{1.003750pt}%
\definecolor{currentstroke}{rgb}{0.266667,0.266667,0.266667}%
\pgfsetstrokecolor{currentstroke}%
\pgfsetdash{}{0pt}%
\pgfpathmoveto{\pgfqpoint{1.728072in}{1.491664in}}%
\pgfpathlineto{\pgfqpoint{1.965552in}{1.491664in}}%
\pgfpathlineto{\pgfqpoint{1.965552in}{1.493507in}}%
\pgfpathlineto{\pgfqpoint{1.728072in}{1.493507in}}%
\pgfpathlineto{\pgfqpoint{1.728072in}{1.491664in}}%
\pgfpathclose%
\pgfusepath{stroke,fill}%
\end{pgfscope}%
\begin{pgfscope}%
\pgfpathrectangle{\pgfqpoint{0.868056in}{0.555556in}}{\pgfqpoint{3.993056in}{1.888889in}}%
\pgfusepath{clip}%
\pgfsetbuttcap%
\pgfsetmiterjoin%
\definecolor{currentfill}{rgb}{0.447059,0.447059,0.447059}%
\pgfsetfillcolor{currentfill}%
\pgfsetlinewidth{1.003750pt}%
\definecolor{currentstroke}{rgb}{0.266667,0.266667,0.266667}%
\pgfsetstrokecolor{currentstroke}%
\pgfsetdash{}{0pt}%
\pgfpathmoveto{\pgfqpoint{2.067329in}{1.598466in}}%
\pgfpathlineto{\pgfqpoint{2.304809in}{1.598466in}}%
\pgfpathlineto{\pgfqpoint{2.304809in}{1.601385in}}%
\pgfpathlineto{\pgfqpoint{2.067329in}{1.601385in}}%
\pgfpathlineto{\pgfqpoint{2.067329in}{1.598466in}}%
\pgfpathclose%
\pgfusepath{stroke,fill}%
\end{pgfscope}%
\begin{pgfscope}%
\pgfpathrectangle{\pgfqpoint{0.868056in}{0.555556in}}{\pgfqpoint{3.993056in}{1.888889in}}%
\pgfusepath{clip}%
\pgfsetbuttcap%
\pgfsetmiterjoin%
\definecolor{currentfill}{rgb}{0.447059,0.447059,0.447059}%
\pgfsetfillcolor{currentfill}%
\pgfsetlinewidth{1.003750pt}%
\definecolor{currentstroke}{rgb}{0.266667,0.266667,0.266667}%
\pgfsetstrokecolor{currentstroke}%
\pgfsetdash{}{0pt}%
\pgfpathmoveto{\pgfqpoint{2.406586in}{1.648659in}}%
\pgfpathlineto{\pgfqpoint{2.644066in}{1.648659in}}%
\pgfpathlineto{\pgfqpoint{2.644066in}{1.651019in}}%
\pgfpathlineto{\pgfqpoint{2.406586in}{1.651019in}}%
\pgfpathlineto{\pgfqpoint{2.406586in}{1.648659in}}%
\pgfpathclose%
\pgfusepath{stroke,fill}%
\end{pgfscope}%
\begin{pgfscope}%
\pgfpathrectangle{\pgfqpoint{0.868056in}{0.555556in}}{\pgfqpoint{3.993056in}{1.888889in}}%
\pgfusepath{clip}%
\pgfsetbuttcap%
\pgfsetmiterjoin%
\definecolor{currentfill}{rgb}{0.447059,0.447059,0.447059}%
\pgfsetfillcolor{currentfill}%
\pgfsetlinewidth{1.003750pt}%
\definecolor{currentstroke}{rgb}{0.266667,0.266667,0.266667}%
\pgfsetstrokecolor{currentstroke}%
\pgfsetdash{}{0pt}%
\pgfpathmoveto{\pgfqpoint{2.745843in}{1.847057in}}%
\pgfpathlineto{\pgfqpoint{2.983323in}{1.847057in}}%
\pgfpathlineto{\pgfqpoint{2.983323in}{1.849272in}}%
\pgfpathlineto{\pgfqpoint{2.745843in}{1.849272in}}%
\pgfpathlineto{\pgfqpoint{2.745843in}{1.847057in}}%
\pgfpathclose%
\pgfusepath{stroke,fill}%
\end{pgfscope}%
\begin{pgfscope}%
\pgfpathrectangle{\pgfqpoint{0.868056in}{0.555556in}}{\pgfqpoint{3.993056in}{1.888889in}}%
\pgfusepath{clip}%
\pgfsetbuttcap%
\pgfsetmiterjoin%
\definecolor{currentfill}{rgb}{0.447059,0.447059,0.447059}%
\pgfsetfillcolor{currentfill}%
\pgfsetlinewidth{1.003750pt}%
\definecolor{currentstroke}{rgb}{0.266667,0.266667,0.266667}%
\pgfsetstrokecolor{currentstroke}%
\pgfsetdash{}{0pt}%
\pgfpathmoveto{\pgfqpoint{3.085100in}{1.574193in}}%
\pgfpathlineto{\pgfqpoint{3.322580in}{1.574193in}}%
\pgfpathlineto{\pgfqpoint{3.322580in}{1.575611in}}%
\pgfpathlineto{\pgfqpoint{3.085100in}{1.575611in}}%
\pgfpathlineto{\pgfqpoint{3.085100in}{1.574193in}}%
\pgfpathclose%
\pgfusepath{stroke,fill}%
\end{pgfscope}%
\begin{pgfscope}%
\pgfpathrectangle{\pgfqpoint{0.868056in}{0.555556in}}{\pgfqpoint{3.993056in}{1.888889in}}%
\pgfusepath{clip}%
\pgfsetbuttcap%
\pgfsetmiterjoin%
\definecolor{currentfill}{rgb}{0.447059,0.447059,0.447059}%
\pgfsetfillcolor{currentfill}%
\pgfsetlinewidth{1.003750pt}%
\definecolor{currentstroke}{rgb}{0.266667,0.266667,0.266667}%
\pgfsetstrokecolor{currentstroke}%
\pgfsetdash{}{0pt}%
\pgfpathmoveto{\pgfqpoint{3.424357in}{1.878038in}}%
\pgfpathlineto{\pgfqpoint{3.661837in}{1.878038in}}%
\pgfpathlineto{\pgfqpoint{3.661837in}{1.879829in}}%
\pgfpathlineto{\pgfqpoint{3.424357in}{1.879829in}}%
\pgfpathlineto{\pgfqpoint{3.424357in}{1.878038in}}%
\pgfpathclose%
\pgfusepath{stroke,fill}%
\end{pgfscope}%
\begin{pgfscope}%
\pgfpathrectangle{\pgfqpoint{0.868056in}{0.555556in}}{\pgfqpoint{3.993056in}{1.888889in}}%
\pgfusepath{clip}%
\pgfsetbuttcap%
\pgfsetmiterjoin%
\definecolor{currentfill}{rgb}{0.447059,0.447059,0.447059}%
\pgfsetfillcolor{currentfill}%
\pgfsetlinewidth{1.003750pt}%
\definecolor{currentstroke}{rgb}{0.266667,0.266667,0.266667}%
\pgfsetstrokecolor{currentstroke}%
\pgfsetdash{}{0pt}%
\pgfpathmoveto{\pgfqpoint{3.763615in}{2.102076in}}%
\pgfpathlineto{\pgfqpoint{4.001094in}{2.102076in}}%
\pgfpathlineto{\pgfqpoint{4.001094in}{2.103505in}}%
\pgfpathlineto{\pgfqpoint{3.763615in}{2.103505in}}%
\pgfpathlineto{\pgfqpoint{3.763615in}{2.102076in}}%
\pgfpathclose%
\pgfusepath{stroke,fill}%
\end{pgfscope}%
\begin{pgfscope}%
\pgfpathrectangle{\pgfqpoint{0.868056in}{0.555556in}}{\pgfqpoint{3.993056in}{1.888889in}}%
\pgfusepath{clip}%
\pgfsetbuttcap%
\pgfsetmiterjoin%
\definecolor{currentfill}{rgb}{0.447059,0.447059,0.447059}%
\pgfsetfillcolor{currentfill}%
\pgfsetlinewidth{1.003750pt}%
\definecolor{currentstroke}{rgb}{0.266667,0.266667,0.266667}%
\pgfsetstrokecolor{currentstroke}%
\pgfsetdash{}{0pt}%
\pgfpathmoveto{\pgfqpoint{4.102872in}{2.182659in}}%
\pgfpathlineto{\pgfqpoint{4.340352in}{2.182659in}}%
\pgfpathlineto{\pgfqpoint{4.340352in}{2.184243in}}%
\pgfpathlineto{\pgfqpoint{4.102872in}{2.184243in}}%
\pgfpathlineto{\pgfqpoint{4.102872in}{2.182659in}}%
\pgfpathclose%
\pgfusepath{stroke,fill}%
\end{pgfscope}%
\begin{pgfscope}%
\pgfpathrectangle{\pgfqpoint{0.868056in}{0.555556in}}{\pgfqpoint{3.993056in}{1.888889in}}%
\pgfusepath{clip}%
\pgfsetbuttcap%
\pgfsetmiterjoin%
\definecolor{currentfill}{rgb}{0.447059,0.447059,0.447059}%
\pgfsetfillcolor{currentfill}%
\pgfsetlinewidth{1.003750pt}%
\definecolor{currentstroke}{rgb}{0.266667,0.266667,0.266667}%
\pgfsetstrokecolor{currentstroke}%
\pgfsetdash{}{0pt}%
\pgfpathmoveto{\pgfqpoint{4.442129in}{2.346693in}}%
\pgfpathlineto{\pgfqpoint{4.679609in}{2.346693in}}%
\pgfpathlineto{\pgfqpoint{4.679609in}{2.353028in}}%
\pgfpathlineto{\pgfqpoint{4.442129in}{2.353028in}}%
\pgfpathlineto{\pgfqpoint{4.442129in}{2.346693in}}%
\pgfpathclose%
\pgfusepath{stroke,fill}%
\end{pgfscope}%
\begin{pgfscope}%
\pgfpathrectangle{\pgfqpoint{0.868056in}{0.555556in}}{\pgfqpoint{3.993056in}{1.888889in}}%
\pgfusepath{clip}%
\pgfsetbuttcap%
\pgfsetmiterjoin%
\definecolor{currentfill}{rgb}{0.447059,0.447059,0.447059}%
\pgfsetfillcolor{currentfill}%
\pgfsetlinewidth{1.003750pt}%
\definecolor{currentstroke}{rgb}{0.266667,0.266667,0.266667}%
\pgfsetstrokecolor{currentstroke}%
\pgfsetdash{}{0pt}%
\pgfpathmoveto{\pgfqpoint{1.049558in}{2.195308in}}%
\pgfpathlineto{\pgfqpoint{1.287038in}{2.195308in}}%
\pgfpathlineto{\pgfqpoint{1.287038in}{2.196592in}}%
\pgfpathlineto{\pgfqpoint{1.049558in}{2.196592in}}%
\pgfpathlineto{\pgfqpoint{1.049558in}{2.195308in}}%
\pgfpathclose%
\pgfusepath{stroke,fill}%
\end{pgfscope}%
\begin{pgfscope}%
\pgfpathrectangle{\pgfqpoint{0.868056in}{0.555556in}}{\pgfqpoint{3.993056in}{1.888889in}}%
\pgfusepath{clip}%
\pgfsetbuttcap%
\pgfsetmiterjoin%
\definecolor{currentfill}{rgb}{0.447059,0.447059,0.447059}%
\pgfsetfillcolor{currentfill}%
\pgfsetlinewidth{1.003750pt}%
\definecolor{currentstroke}{rgb}{0.266667,0.266667,0.266667}%
\pgfsetstrokecolor{currentstroke}%
\pgfsetdash{}{0pt}%
\pgfpathmoveto{\pgfqpoint{1.388815in}{1.786617in}}%
\pgfpathlineto{\pgfqpoint{1.626295in}{1.786617in}}%
\pgfpathlineto{\pgfqpoint{1.626295in}{1.787984in}}%
\pgfpathlineto{\pgfqpoint{1.388815in}{1.787984in}}%
\pgfpathlineto{\pgfqpoint{1.388815in}{1.786617in}}%
\pgfpathclose%
\pgfusepath{stroke,fill}%
\end{pgfscope}%
\begin{pgfscope}%
\pgfpathrectangle{\pgfqpoint{0.868056in}{0.555556in}}{\pgfqpoint{3.993056in}{1.888889in}}%
\pgfusepath{clip}%
\pgfsetbuttcap%
\pgfsetmiterjoin%
\definecolor{currentfill}{rgb}{0.447059,0.447059,0.447059}%
\pgfsetfillcolor{currentfill}%
\pgfsetlinewidth{1.003750pt}%
\definecolor{currentstroke}{rgb}{0.266667,0.266667,0.266667}%
\pgfsetstrokecolor{currentstroke}%
\pgfsetdash{}{0pt}%
\pgfpathmoveto{\pgfqpoint{1.728072in}{1.493507in}}%
\pgfpathlineto{\pgfqpoint{1.965552in}{1.493507in}}%
\pgfpathlineto{\pgfqpoint{1.965552in}{1.494562in}}%
\pgfpathlineto{\pgfqpoint{1.728072in}{1.494562in}}%
\pgfpathlineto{\pgfqpoint{1.728072in}{1.493507in}}%
\pgfpathclose%
\pgfusepath{stroke,fill}%
\end{pgfscope}%
\begin{pgfscope}%
\pgfpathrectangle{\pgfqpoint{0.868056in}{0.555556in}}{\pgfqpoint{3.993056in}{1.888889in}}%
\pgfusepath{clip}%
\pgfsetbuttcap%
\pgfsetmiterjoin%
\definecolor{currentfill}{rgb}{0.447059,0.447059,0.447059}%
\pgfsetfillcolor{currentfill}%
\pgfsetlinewidth{1.003750pt}%
\definecolor{currentstroke}{rgb}{0.266667,0.266667,0.266667}%
\pgfsetstrokecolor{currentstroke}%
\pgfsetdash{}{0pt}%
\pgfpathmoveto{\pgfqpoint{2.067329in}{1.601385in}}%
\pgfpathlineto{\pgfqpoint{2.304809in}{1.601385in}}%
\pgfpathlineto{\pgfqpoint{2.304809in}{1.602638in}}%
\pgfpathlineto{\pgfqpoint{2.067329in}{1.602638in}}%
\pgfpathlineto{\pgfqpoint{2.067329in}{1.601385in}}%
\pgfpathclose%
\pgfusepath{stroke,fill}%
\end{pgfscope}%
\begin{pgfscope}%
\pgfpathrectangle{\pgfqpoint{0.868056in}{0.555556in}}{\pgfqpoint{3.993056in}{1.888889in}}%
\pgfusepath{clip}%
\pgfsetbuttcap%
\pgfsetmiterjoin%
\definecolor{currentfill}{rgb}{0.447059,0.447059,0.447059}%
\pgfsetfillcolor{currentfill}%
\pgfsetlinewidth{1.003750pt}%
\definecolor{currentstroke}{rgb}{0.266667,0.266667,0.266667}%
\pgfsetstrokecolor{currentstroke}%
\pgfsetdash{}{0pt}%
\pgfpathmoveto{\pgfqpoint{2.406586in}{1.651019in}}%
\pgfpathlineto{\pgfqpoint{2.644066in}{1.651019in}}%
\pgfpathlineto{\pgfqpoint{2.644066in}{1.653472in}}%
\pgfpathlineto{\pgfqpoint{2.406586in}{1.653472in}}%
\pgfpathlineto{\pgfqpoint{2.406586in}{1.651019in}}%
\pgfpathclose%
\pgfusepath{stroke,fill}%
\end{pgfscope}%
\begin{pgfscope}%
\pgfpathrectangle{\pgfqpoint{0.868056in}{0.555556in}}{\pgfqpoint{3.993056in}{1.888889in}}%
\pgfusepath{clip}%
\pgfsetbuttcap%
\pgfsetmiterjoin%
\definecolor{currentfill}{rgb}{0.447059,0.447059,0.447059}%
\pgfsetfillcolor{currentfill}%
\pgfsetlinewidth{1.003750pt}%
\definecolor{currentstroke}{rgb}{0.266667,0.266667,0.266667}%
\pgfsetstrokecolor{currentstroke}%
\pgfsetdash{}{0pt}%
\pgfpathmoveto{\pgfqpoint{2.745843in}{1.849272in}}%
\pgfpathlineto{\pgfqpoint{2.983323in}{1.849272in}}%
\pgfpathlineto{\pgfqpoint{2.983323in}{1.850887in}}%
\pgfpathlineto{\pgfqpoint{2.745843in}{1.850887in}}%
\pgfpathlineto{\pgfqpoint{2.745843in}{1.849272in}}%
\pgfpathclose%
\pgfusepath{stroke,fill}%
\end{pgfscope}%
\begin{pgfscope}%
\pgfpathrectangle{\pgfqpoint{0.868056in}{0.555556in}}{\pgfqpoint{3.993056in}{1.888889in}}%
\pgfusepath{clip}%
\pgfsetbuttcap%
\pgfsetmiterjoin%
\definecolor{currentfill}{rgb}{0.447059,0.447059,0.447059}%
\pgfsetfillcolor{currentfill}%
\pgfsetlinewidth{1.003750pt}%
\definecolor{currentstroke}{rgb}{0.266667,0.266667,0.266667}%
\pgfsetstrokecolor{currentstroke}%
\pgfsetdash{}{0pt}%
\pgfpathmoveto{\pgfqpoint{3.085100in}{1.575611in}}%
\pgfpathlineto{\pgfqpoint{3.322580in}{1.575611in}}%
\pgfpathlineto{\pgfqpoint{3.322580in}{1.576884in}}%
\pgfpathlineto{\pgfqpoint{3.085100in}{1.576884in}}%
\pgfpathlineto{\pgfqpoint{3.085100in}{1.575611in}}%
\pgfpathclose%
\pgfusepath{stroke,fill}%
\end{pgfscope}%
\begin{pgfscope}%
\pgfpathrectangle{\pgfqpoint{0.868056in}{0.555556in}}{\pgfqpoint{3.993056in}{1.888889in}}%
\pgfusepath{clip}%
\pgfsetbuttcap%
\pgfsetmiterjoin%
\definecolor{currentfill}{rgb}{0.447059,0.447059,0.447059}%
\pgfsetfillcolor{currentfill}%
\pgfsetlinewidth{1.003750pt}%
\definecolor{currentstroke}{rgb}{0.266667,0.266667,0.266667}%
\pgfsetstrokecolor{currentstroke}%
\pgfsetdash{}{0pt}%
\pgfpathmoveto{\pgfqpoint{3.424357in}{1.879829in}}%
\pgfpathlineto{\pgfqpoint{3.661837in}{1.879829in}}%
\pgfpathlineto{\pgfqpoint{3.661837in}{1.880967in}}%
\pgfpathlineto{\pgfqpoint{3.424357in}{1.880967in}}%
\pgfpathlineto{\pgfqpoint{3.424357in}{1.879829in}}%
\pgfpathclose%
\pgfusepath{stroke,fill}%
\end{pgfscope}%
\begin{pgfscope}%
\pgfpathrectangle{\pgfqpoint{0.868056in}{0.555556in}}{\pgfqpoint{3.993056in}{1.888889in}}%
\pgfusepath{clip}%
\pgfsetbuttcap%
\pgfsetmiterjoin%
\definecolor{currentfill}{rgb}{0.447059,0.447059,0.447059}%
\pgfsetfillcolor{currentfill}%
\pgfsetlinewidth{1.003750pt}%
\definecolor{currentstroke}{rgb}{0.266667,0.266667,0.266667}%
\pgfsetstrokecolor{currentstroke}%
\pgfsetdash{}{0pt}%
\pgfpathmoveto{\pgfqpoint{3.763615in}{2.103505in}}%
\pgfpathlineto{\pgfqpoint{4.001094in}{2.103505in}}%
\pgfpathlineto{\pgfqpoint{4.001094in}{2.104488in}}%
\pgfpathlineto{\pgfqpoint{3.763615in}{2.104488in}}%
\pgfpathlineto{\pgfqpoint{3.763615in}{2.103505in}}%
\pgfpathclose%
\pgfusepath{stroke,fill}%
\end{pgfscope}%
\begin{pgfscope}%
\pgfpathrectangle{\pgfqpoint{0.868056in}{0.555556in}}{\pgfqpoint{3.993056in}{1.888889in}}%
\pgfusepath{clip}%
\pgfsetbuttcap%
\pgfsetmiterjoin%
\definecolor{currentfill}{rgb}{0.447059,0.447059,0.447059}%
\pgfsetfillcolor{currentfill}%
\pgfsetlinewidth{1.003750pt}%
\definecolor{currentstroke}{rgb}{0.266667,0.266667,0.266667}%
\pgfsetstrokecolor{currentstroke}%
\pgfsetdash{}{0pt}%
\pgfpathmoveto{\pgfqpoint{4.102872in}{2.184243in}}%
\pgfpathlineto{\pgfqpoint{4.340352in}{2.184243in}}%
\pgfpathlineto{\pgfqpoint{4.340352in}{2.185113in}}%
\pgfpathlineto{\pgfqpoint{4.102872in}{2.185113in}}%
\pgfpathlineto{\pgfqpoint{4.102872in}{2.184243in}}%
\pgfpathclose%
\pgfusepath{stroke,fill}%
\end{pgfscope}%
\begin{pgfscope}%
\pgfpathrectangle{\pgfqpoint{0.868056in}{0.555556in}}{\pgfqpoint{3.993056in}{1.888889in}}%
\pgfusepath{clip}%
\pgfsetbuttcap%
\pgfsetmiterjoin%
\definecolor{currentfill}{rgb}{0.447059,0.447059,0.447059}%
\pgfsetfillcolor{currentfill}%
\pgfsetlinewidth{1.003750pt}%
\definecolor{currentstroke}{rgb}{0.266667,0.266667,0.266667}%
\pgfsetstrokecolor{currentstroke}%
\pgfsetdash{}{0pt}%
\pgfpathmoveto{\pgfqpoint{4.442129in}{2.353028in}}%
\pgfpathlineto{\pgfqpoint{4.679609in}{2.353028in}}%
\pgfpathlineto{\pgfqpoint{4.679609in}{2.354063in}}%
\pgfpathlineto{\pgfqpoint{4.442129in}{2.354063in}}%
\pgfpathlineto{\pgfqpoint{4.442129in}{2.353028in}}%
\pgfpathclose%
\pgfusepath{stroke,fill}%
\end{pgfscope}%
\begin{pgfscope}%
\pgfpathrectangle{\pgfqpoint{0.868056in}{0.555556in}}{\pgfqpoint{3.993056in}{1.888889in}}%
\pgfusepath{clip}%
\pgfsetbuttcap%
\pgfsetmiterjoin%
\definecolor{currentfill}{rgb}{0.447059,0.447059,0.447059}%
\pgfsetfillcolor{currentfill}%
\pgfsetlinewidth{1.003750pt}%
\definecolor{currentstroke}{rgb}{0.266667,0.266667,0.266667}%
\pgfsetstrokecolor{currentstroke}%
\pgfsetdash{}{0pt}%
\pgfpathmoveto{\pgfqpoint{1.049558in}{2.196592in}}%
\pgfpathlineto{\pgfqpoint{1.287038in}{2.196592in}}%
\pgfpathlineto{\pgfqpoint{1.287038in}{2.197037in}}%
\pgfpathlineto{\pgfqpoint{1.049558in}{2.197037in}}%
\pgfpathlineto{\pgfqpoint{1.049558in}{2.196592in}}%
\pgfpathclose%
\pgfusepath{stroke,fill}%
\end{pgfscope}%
\begin{pgfscope}%
\pgfpathrectangle{\pgfqpoint{0.868056in}{0.555556in}}{\pgfqpoint{3.993056in}{1.888889in}}%
\pgfusepath{clip}%
\pgfsetbuttcap%
\pgfsetmiterjoin%
\definecolor{currentfill}{rgb}{0.447059,0.447059,0.447059}%
\pgfsetfillcolor{currentfill}%
\pgfsetlinewidth{1.003750pt}%
\definecolor{currentstroke}{rgb}{0.266667,0.266667,0.266667}%
\pgfsetstrokecolor{currentstroke}%
\pgfsetdash{}{0pt}%
\pgfpathmoveto{\pgfqpoint{1.388815in}{1.787984in}}%
\pgfpathlineto{\pgfqpoint{1.626295in}{1.787984in}}%
\pgfpathlineto{\pgfqpoint{1.626295in}{1.788149in}}%
\pgfpathlineto{\pgfqpoint{1.388815in}{1.788149in}}%
\pgfpathlineto{\pgfqpoint{1.388815in}{1.787984in}}%
\pgfpathclose%
\pgfusepath{stroke,fill}%
\end{pgfscope}%
\begin{pgfscope}%
\pgfpathrectangle{\pgfqpoint{0.868056in}{0.555556in}}{\pgfqpoint{3.993056in}{1.888889in}}%
\pgfusepath{clip}%
\pgfsetbuttcap%
\pgfsetmiterjoin%
\definecolor{currentfill}{rgb}{0.447059,0.447059,0.447059}%
\pgfsetfillcolor{currentfill}%
\pgfsetlinewidth{1.003750pt}%
\definecolor{currentstroke}{rgb}{0.266667,0.266667,0.266667}%
\pgfsetstrokecolor{currentstroke}%
\pgfsetdash{}{0pt}%
\pgfpathmoveto{\pgfqpoint{1.728072in}{1.494562in}}%
\pgfpathlineto{\pgfqpoint{1.965552in}{1.494562in}}%
\pgfpathlineto{\pgfqpoint{1.965552in}{1.494593in}}%
\pgfpathlineto{\pgfqpoint{1.728072in}{1.494593in}}%
\pgfpathlineto{\pgfqpoint{1.728072in}{1.494562in}}%
\pgfpathclose%
\pgfusepath{stroke,fill}%
\end{pgfscope}%
\begin{pgfscope}%
\pgfpathrectangle{\pgfqpoint{0.868056in}{0.555556in}}{\pgfqpoint{3.993056in}{1.888889in}}%
\pgfusepath{clip}%
\pgfsetbuttcap%
\pgfsetmiterjoin%
\definecolor{currentfill}{rgb}{0.447059,0.447059,0.447059}%
\pgfsetfillcolor{currentfill}%
\pgfsetlinewidth{1.003750pt}%
\definecolor{currentstroke}{rgb}{0.266667,0.266667,0.266667}%
\pgfsetstrokecolor{currentstroke}%
\pgfsetdash{}{0pt}%
\pgfpathmoveto{\pgfqpoint{2.067329in}{1.602638in}}%
\pgfpathlineto{\pgfqpoint{2.304809in}{1.602638in}}%
\pgfpathlineto{\pgfqpoint{2.304809in}{1.602731in}}%
\pgfpathlineto{\pgfqpoint{2.067329in}{1.602731in}}%
\pgfpathlineto{\pgfqpoint{2.067329in}{1.602638in}}%
\pgfpathclose%
\pgfusepath{stroke,fill}%
\end{pgfscope}%
\begin{pgfscope}%
\pgfpathrectangle{\pgfqpoint{0.868056in}{0.555556in}}{\pgfqpoint{3.993056in}{1.888889in}}%
\pgfusepath{clip}%
\pgfsetbuttcap%
\pgfsetmiterjoin%
\definecolor{currentfill}{rgb}{0.447059,0.447059,0.447059}%
\pgfsetfillcolor{currentfill}%
\pgfsetlinewidth{1.003750pt}%
\definecolor{currentstroke}{rgb}{0.266667,0.266667,0.266667}%
\pgfsetstrokecolor{currentstroke}%
\pgfsetdash{}{0pt}%
\pgfpathmoveto{\pgfqpoint{2.406586in}{1.653472in}}%
\pgfpathlineto{\pgfqpoint{2.644066in}{1.653472in}}%
\pgfpathlineto{\pgfqpoint{2.644066in}{1.653575in}}%
\pgfpathlineto{\pgfqpoint{2.406586in}{1.653575in}}%
\pgfpathlineto{\pgfqpoint{2.406586in}{1.653472in}}%
\pgfpathclose%
\pgfusepath{stroke,fill}%
\end{pgfscope}%
\begin{pgfscope}%
\pgfpathrectangle{\pgfqpoint{0.868056in}{0.555556in}}{\pgfqpoint{3.993056in}{1.888889in}}%
\pgfusepath{clip}%
\pgfsetbuttcap%
\pgfsetmiterjoin%
\definecolor{currentfill}{rgb}{0.447059,0.447059,0.447059}%
\pgfsetfillcolor{currentfill}%
\pgfsetlinewidth{1.003750pt}%
\definecolor{currentstroke}{rgb}{0.266667,0.266667,0.266667}%
\pgfsetstrokecolor{currentstroke}%
\pgfsetdash{}{0pt}%
\pgfpathmoveto{\pgfqpoint{2.745843in}{1.850887in}}%
\pgfpathlineto{\pgfqpoint{2.983323in}{1.850887in}}%
\pgfpathlineto{\pgfqpoint{2.983323in}{1.850991in}}%
\pgfpathlineto{\pgfqpoint{2.745843in}{1.850991in}}%
\pgfpathlineto{\pgfqpoint{2.745843in}{1.850887in}}%
\pgfpathclose%
\pgfusepath{stroke,fill}%
\end{pgfscope}%
\begin{pgfscope}%
\pgfpathrectangle{\pgfqpoint{0.868056in}{0.555556in}}{\pgfqpoint{3.993056in}{1.888889in}}%
\pgfusepath{clip}%
\pgfsetbuttcap%
\pgfsetmiterjoin%
\definecolor{currentfill}{rgb}{0.447059,0.447059,0.447059}%
\pgfsetfillcolor{currentfill}%
\pgfsetlinewidth{1.003750pt}%
\definecolor{currentstroke}{rgb}{0.266667,0.266667,0.266667}%
\pgfsetstrokecolor{currentstroke}%
\pgfsetdash{}{0pt}%
\pgfpathmoveto{\pgfqpoint{3.085100in}{1.576884in}}%
\pgfpathlineto{\pgfqpoint{3.322580in}{1.576884in}}%
\pgfpathlineto{\pgfqpoint{3.322580in}{1.576977in}}%
\pgfpathlineto{\pgfqpoint{3.085100in}{1.576977in}}%
\pgfpathlineto{\pgfqpoint{3.085100in}{1.576884in}}%
\pgfpathclose%
\pgfusepath{stroke,fill}%
\end{pgfscope}%
\begin{pgfscope}%
\pgfpathrectangle{\pgfqpoint{0.868056in}{0.555556in}}{\pgfqpoint{3.993056in}{1.888889in}}%
\pgfusepath{clip}%
\pgfsetbuttcap%
\pgfsetmiterjoin%
\definecolor{currentfill}{rgb}{0.447059,0.447059,0.447059}%
\pgfsetfillcolor{currentfill}%
\pgfsetlinewidth{1.003750pt}%
\definecolor{currentstroke}{rgb}{0.266667,0.266667,0.266667}%
\pgfsetstrokecolor{currentstroke}%
\pgfsetdash{}{0pt}%
\pgfpathmoveto{\pgfqpoint{3.424357in}{1.880967in}}%
\pgfpathlineto{\pgfqpoint{3.661837in}{1.880967in}}%
\pgfpathlineto{\pgfqpoint{3.661837in}{1.881174in}}%
\pgfpathlineto{\pgfqpoint{3.424357in}{1.881174in}}%
\pgfpathlineto{\pgfqpoint{3.424357in}{1.880967in}}%
\pgfpathclose%
\pgfusepath{stroke,fill}%
\end{pgfscope}%
\begin{pgfscope}%
\pgfpathrectangle{\pgfqpoint{0.868056in}{0.555556in}}{\pgfqpoint{3.993056in}{1.888889in}}%
\pgfusepath{clip}%
\pgfsetbuttcap%
\pgfsetmiterjoin%
\definecolor{currentfill}{rgb}{0.447059,0.447059,0.447059}%
\pgfsetfillcolor{currentfill}%
\pgfsetlinewidth{1.003750pt}%
\definecolor{currentstroke}{rgb}{0.266667,0.266667,0.266667}%
\pgfsetstrokecolor{currentstroke}%
\pgfsetdash{}{0pt}%
\pgfpathmoveto{\pgfqpoint{3.763615in}{2.104488in}}%
\pgfpathlineto{\pgfqpoint{4.001094in}{2.104488in}}%
\pgfpathlineto{\pgfqpoint{4.001094in}{2.104612in}}%
\pgfpathlineto{\pgfqpoint{3.763615in}{2.104612in}}%
\pgfpathlineto{\pgfqpoint{3.763615in}{2.104488in}}%
\pgfpathclose%
\pgfusepath{stroke,fill}%
\end{pgfscope}%
\begin{pgfscope}%
\pgfpathrectangle{\pgfqpoint{0.868056in}{0.555556in}}{\pgfqpoint{3.993056in}{1.888889in}}%
\pgfusepath{clip}%
\pgfsetbuttcap%
\pgfsetmiterjoin%
\definecolor{currentfill}{rgb}{0.447059,0.447059,0.447059}%
\pgfsetfillcolor{currentfill}%
\pgfsetlinewidth{1.003750pt}%
\definecolor{currentstroke}{rgb}{0.266667,0.266667,0.266667}%
\pgfsetstrokecolor{currentstroke}%
\pgfsetdash{}{0pt}%
\pgfpathmoveto{\pgfqpoint{4.102872in}{2.185113in}}%
\pgfpathlineto{\pgfqpoint{4.340352in}{2.185113in}}%
\pgfpathlineto{\pgfqpoint{4.340352in}{2.185195in}}%
\pgfpathlineto{\pgfqpoint{4.102872in}{2.185195in}}%
\pgfpathlineto{\pgfqpoint{4.102872in}{2.185113in}}%
\pgfpathclose%
\pgfusepath{stroke,fill}%
\end{pgfscope}%
\begin{pgfscope}%
\pgfpathrectangle{\pgfqpoint{0.868056in}{0.555556in}}{\pgfqpoint{3.993056in}{1.888889in}}%
\pgfusepath{clip}%
\pgfsetbuttcap%
\pgfsetmiterjoin%
\definecolor{currentfill}{rgb}{0.447059,0.447059,0.447059}%
\pgfsetfillcolor{currentfill}%
\pgfsetlinewidth{1.003750pt}%
\definecolor{currentstroke}{rgb}{0.266667,0.266667,0.266667}%
\pgfsetstrokecolor{currentstroke}%
\pgfsetdash{}{0pt}%
\pgfpathmoveto{\pgfqpoint{4.442129in}{2.354063in}}%
\pgfpathlineto{\pgfqpoint{4.679609in}{2.354063in}}%
\pgfpathlineto{\pgfqpoint{4.679609in}{2.354497in}}%
\pgfpathlineto{\pgfqpoint{4.442129in}{2.354497in}}%
\pgfpathlineto{\pgfqpoint{4.442129in}{2.354063in}}%
\pgfpathclose%
\pgfusepath{stroke,fill}%
\end{pgfscope}%
\begin{pgfscope}%
\definecolor{textcolor}{rgb}{0.000000,0.000000,0.000000}%
\pgfsetstrokecolor{textcolor}%
\pgfsetfillcolor{textcolor}%
\pgftext[x=1.168298in,y=2.224815in,,bottom]{\color{textcolor}{\ifdefined\pdftexversion\else\setmainfont{NanumMyeongjo}\rmfamily\fi\fontsize{5.000000}{6.000000}\selectfont\catcode`\^=\active\def^{\ifmmode\sp\else\^{}\fi}\catcode`\%=\active\def%{\%}158,581}}%
\end{pgfscope}%
\begin{pgfscope}%
\definecolor{textcolor}{rgb}{0.000000,0.000000,0.000000}%
\pgfsetstrokecolor{textcolor}%
\pgfsetfillcolor{textcolor}%
\pgftext[x=1.507555in,y=1.815927in,,bottom]{\color{textcolor}{\ifdefined\pdftexversion\else\setmainfont{NanumMyeongjo}\rmfamily\fi\fontsize{5.000000}{6.000000}\selectfont\catcode`\^=\active\def^{\ifmmode\sp\else\^{}\fi}\catcode`\%=\active\def%{\%}119,079}}%
\end{pgfscope}%
\begin{pgfscope}%
\definecolor{textcolor}{rgb}{0.000000,0.000000,0.000000}%
\pgfsetstrokecolor{textcolor}%
\pgfsetfillcolor{textcolor}%
\pgftext[x=1.846812in,y=1.522371in,,bottom]{\color{textcolor}{\ifdefined\pdftexversion\else\setmainfont{NanumMyeongjo}\rmfamily\fi\fontsize{5.000000}{6.000000}\selectfont\catcode`\^=\active\def^{\ifmmode\sp\else\^{}\fi}\catcode`\%=\active\def%{\%}90,719}}%
\end{pgfscope}%
\begin{pgfscope}%
\definecolor{textcolor}{rgb}{0.000000,0.000000,0.000000}%
\pgfsetstrokecolor{textcolor}%
\pgfsetfillcolor{textcolor}%
\pgftext[x=2.186069in,y=1.630509in,,bottom]{\color{textcolor}{\ifdefined\pdftexversion\else\setmainfont{NanumMyeongjo}\rmfamily\fi\fontsize{5.000000}{6.000000}\selectfont\catcode`\^=\active\def^{\ifmmode\sp\else\^{}\fi}\catcode`\%=\active\def%{\%}101,166}}%
\end{pgfscope}%
\begin{pgfscope}%
\definecolor{textcolor}{rgb}{0.000000,0.000000,0.000000}%
\pgfsetstrokecolor{textcolor}%
\pgfsetfillcolor{textcolor}%
\pgftext[x=2.525326in,y=1.681353in,,bottom]{\color{textcolor}{\ifdefined\pdftexversion\else\setmainfont{NanumMyeongjo}\rmfamily\fi\fontsize{5.000000}{6.000000}\selectfont\catcode`\^=\active\def^{\ifmmode\sp\else\^{}\fi}\catcode`\%=\active\def%{\%}106,078}}%
\end{pgfscope}%
\begin{pgfscope}%
\definecolor{textcolor}{rgb}{0.000000,0.000000,0.000000}%
\pgfsetstrokecolor{textcolor}%
\pgfsetfillcolor{textcolor}%
\pgftext[x=2.864583in,y=1.878769in,,bottom]{\color{textcolor}{\ifdefined\pdftexversion\else\setmainfont{NanumMyeongjo}\rmfamily\fi\fontsize{5.000000}{6.000000}\selectfont\catcode`\^=\active\def^{\ifmmode\sp\else\^{}\fi}\catcode`\%=\active\def%{\%}125,150}}%
\end{pgfscope}%
\begin{pgfscope}%
\definecolor{textcolor}{rgb}{0.000000,0.000000,0.000000}%
\pgfsetstrokecolor{textcolor}%
\pgfsetfillcolor{textcolor}%
\pgftext[x=3.203840in,y=1.604755in,,bottom]{\color{textcolor}{\ifdefined\pdftexversion\else\setmainfont{NanumMyeongjo}\rmfamily\fi\fontsize{5.000000}{6.000000}\selectfont\catcode`\^=\active\def^{\ifmmode\sp\else\^{}\fi}\catcode`\%=\active\def%{\%}98,678}}%
\end{pgfscope}%
\begin{pgfscope}%
\definecolor{textcolor}{rgb}{0.000000,0.000000,0.000000}%
\pgfsetstrokecolor{textcolor}%
\pgfsetfillcolor{textcolor}%
\pgftext[x=3.543097in,y=1.908952in,,bottom]{\color{textcolor}{\ifdefined\pdftexversion\else\setmainfont{NanumMyeongjo}\rmfamily\fi\fontsize{5.000000}{6.000000}\selectfont\catcode`\^=\active\def^{\ifmmode\sp\else\^{}\fi}\catcode`\%=\active\def%{\%}128,066}}%
\end{pgfscope}%
\begin{pgfscope}%
\definecolor{textcolor}{rgb}{0.000000,0.000000,0.000000}%
\pgfsetstrokecolor{textcolor}%
\pgfsetfillcolor{textcolor}%
\pgftext[x=3.882355in,y=2.132390in,,bottom]{\color{textcolor}{\ifdefined\pdftexversion\else\setmainfont{NanumMyeongjo}\rmfamily\fi\fontsize{5.000000}{6.000000}\selectfont\catcode`\^=\active\def^{\ifmmode\sp\else\^{}\fi}\catcode`\%=\active\def%{\%}149,652}}%
\end{pgfscope}%
\begin{pgfscope}%
\definecolor{textcolor}{rgb}{0.000000,0.000000,0.000000}%
\pgfsetstrokecolor{textcolor}%
\pgfsetfillcolor{textcolor}%
\pgftext[x=4.221612in,y=2.212973in,,bottom]{\color{textcolor}{\ifdefined\pdftexversion\else\setmainfont{NanumMyeongjo}\rmfamily\fi\fontsize{5.000000}{6.000000}\selectfont\catcode`\^=\active\def^{\ifmmode\sp\else\^{}\fi}\catcode`\%=\active\def%{\%}157,437}}%
\end{pgfscope}%
\begin{pgfscope}%
\definecolor{textcolor}{rgb}{0.000000,0.000000,0.000000}%
\pgfsetstrokecolor{textcolor}%
\pgfsetfillcolor{textcolor}%
\pgftext[x=4.560869in,y=2.382275in,,bottom]{\color{textcolor}{\ifdefined\pdftexversion\else\setmainfont{NanumMyeongjo}\rmfamily\fi\fontsize{5.000000}{6.000000}\selectfont\catcode`\^=\active\def^{\ifmmode\sp\else\^{}\fi}\catcode`\%=\active\def%{\%}173,793}}%
\end{pgfscope}%
\begin{pgfscope}%
\definecolor{textcolor}{rgb}{1.000000,1.000000,1.000000}%
\pgfsetstrokecolor{textcolor}%
\pgfsetfillcolor{textcolor}%
\pgftext[x=1.168298in,y=0.627413in,,]{\color{textcolor}{\ifdefined\pdftexversion\else\setmainfont{NanumMyeongjo}\rmfamily\fi\fontsize{5.000000}{6.000000}\selectfont\catcode`\^=\active\def^{\ifmmode\sp\else\^{}\fi}\catcode`\%=\active\def%{\%}14,942}}%
\end{pgfscope}%
\begin{pgfscope}%
\definecolor{textcolor}{rgb}{1.000000,1.000000,1.000000}%
\pgfsetstrokecolor{textcolor}%
\pgfsetfillcolor{textcolor}%
\pgftext[x=1.507555in,y=0.587727in,,]{\color{textcolor}{\ifdefined\pdftexversion\else\setmainfont{NanumMyeongjo}\rmfamily\fi\fontsize{5.000000}{6.000000}\selectfont\catcode`\^=\active\def^{\ifmmode\sp\else\^{}\fi}\catcode`\%=\active\def%{\%}11,108}}%
\end{pgfscope}%
\begin{pgfscope}%
\definecolor{textcolor}{rgb}{1.000000,1.000000,1.000000}%
\pgfsetstrokecolor{textcolor}%
\pgfsetfillcolor{textcolor}%
\pgftext[x=2.186069in,y=0.604143in,,]{\color{textcolor}{\ifdefined\pdftexversion\else\setmainfont{NanumMyeongjo}\rmfamily\fi\fontsize{5.000000}{6.000000}\selectfont\catcode`\^=\active\def^{\ifmmode\sp\else\^{}\fi}\catcode`\%=\active\def%{\%}12,694}}%
\end{pgfscope}%
\begin{pgfscope}%
\definecolor{textcolor}{rgb}{1.000000,1.000000,1.000000}%
\pgfsetstrokecolor{textcolor}%
\pgfsetfillcolor{textcolor}%
\pgftext[x=2.525326in,y=0.663766in,,]{\color{textcolor}{\ifdefined\pdftexversion\else\setmainfont{NanumMyeongjo}\rmfamily\fi\fontsize{5.000000}{6.000000}\selectfont\catcode`\^=\active\def^{\ifmmode\sp\else\^{}\fi}\catcode`\%=\active\def%{\%}18,454}}%
\end{pgfscope}%
\begin{pgfscope}%
\definecolor{textcolor}{rgb}{1.000000,1.000000,1.000000}%
\pgfsetstrokecolor{textcolor}%
\pgfsetfillcolor{textcolor}%
\pgftext[x=2.864583in,y=0.728905in,,]{\color{textcolor}{\ifdefined\pdftexversion\else\setmainfont{NanumMyeongjo}\rmfamily\fi\fontsize{5.000000}{6.000000}\selectfont\catcode`\^=\active\def^{\ifmmode\sp\else\^{}\fi}\catcode`\%=\active\def%{\%}24,747}}%
\end{pgfscope}%
\begin{pgfscope}%
\definecolor{textcolor}{rgb}{1.000000,1.000000,1.000000}%
\pgfsetstrokecolor{textcolor}%
\pgfsetfillcolor{textcolor}%
\pgftext[x=3.203840in,y=0.637805in,,]{\color{textcolor}{\ifdefined\pdftexversion\else\setmainfont{NanumMyeongjo}\rmfamily\fi\fontsize{5.000000}{6.000000}\selectfont\catcode`\^=\active\def^{\ifmmode\sp\else\^{}\fi}\catcode`\%=\active\def%{\%}15,946}}%
\end{pgfscope}%
\begin{pgfscope}%
\definecolor{textcolor}{rgb}{1.000000,1.000000,1.000000}%
\pgfsetstrokecolor{textcolor}%
\pgfsetfillcolor{textcolor}%
\pgftext[x=3.543097in,y=0.725810in,,]{\color{textcolor}{\ifdefined\pdftexversion\else\setmainfont{NanumMyeongjo}\rmfamily\fi\fontsize{5.000000}{6.000000}\selectfont\catcode`\^=\active\def^{\ifmmode\sp\else\^{}\fi}\catcode`\%=\active\def%{\%}24,448}}%
\end{pgfscope}%
\begin{pgfscope}%
\definecolor{textcolor}{rgb}{1.000000,1.000000,1.000000}%
\pgfsetstrokecolor{textcolor}%
\pgfsetfillcolor{textcolor}%
\pgftext[x=3.882355in,y=0.833585in,,]{\color{textcolor}{\ifdefined\pdftexversion\else\setmainfont{NanumMyeongjo}\rmfamily\fi\fontsize{5.000000}{6.000000}\selectfont\catcode`\^=\active\def^{\ifmmode\sp\else\^{}\fi}\catcode`\%=\active\def%{\%}34,860}}%
\end{pgfscope}%
\begin{pgfscope}%
\definecolor{textcolor}{rgb}{1.000000,1.000000,1.000000}%
\pgfsetstrokecolor{textcolor}%
\pgfsetfillcolor{textcolor}%
\pgftext[x=4.221612in,y=0.882214in,,]{\color{textcolor}{\ifdefined\pdftexversion\else\setmainfont{NanumMyeongjo}\rmfamily\fi\fontsize{5.000000}{6.000000}\selectfont\catcode`\^=\active\def^{\ifmmode\sp\else\^{}\fi}\catcode`\%=\active\def%{\%}39,558}}%
\end{pgfscope}%
\begin{pgfscope}%
\definecolor{textcolor}{rgb}{1.000000,1.000000,1.000000}%
\pgfsetstrokecolor{textcolor}%
\pgfsetfillcolor{textcolor}%
\pgftext[x=4.560869in,y=1.012783in,,]{\color{textcolor}{\ifdefined\pdftexversion\else\setmainfont{NanumMyeongjo}\rmfamily\fi\fontsize{5.000000}{6.000000}\selectfont\catcode`\^=\active\def^{\ifmmode\sp\else\^{}\fi}\catcode`\%=\active\def%{\%}52,172}}%
\end{pgfscope}%
\begin{pgfscope}%
\definecolor{textcolor}{rgb}{1.000000,1.000000,1.000000}%
\pgfsetstrokecolor{textcolor}%
\pgfsetfillcolor{textcolor}%
\pgftext[x=1.168298in,y=0.920617in,,]{\color{textcolor}{\ifdefined\pdftexversion\else\setmainfont{NanumMyeongjo}\rmfamily\fi\fontsize{5.000000}{6.000000}\selectfont\catcode`\^=\active\def^{\ifmmode\sp\else\^{}\fi}\catcode`\%=\active\def%{\%}28,326}}%
\end{pgfscope}%
\begin{pgfscope}%
\definecolor{textcolor}{rgb}{1.000000,1.000000,1.000000}%
\pgfsetstrokecolor{textcolor}%
\pgfsetfillcolor{textcolor}%
\pgftext[x=1.507555in,y=0.794862in,,]{\color{textcolor}{\ifdefined\pdftexversion\else\setmainfont{NanumMyeongjo}\rmfamily\fi\fontsize{5.000000}{6.000000}\selectfont\catcode`\^=\active\def^{\ifmmode\sp\else\^{}\fi}\catcode`\%=\active\def%{\%}20,011}}%
\end{pgfscope}%
\begin{pgfscope}%
\definecolor{textcolor}{rgb}{1.000000,1.000000,1.000000}%
\pgfsetstrokecolor{textcolor}%
\pgfsetfillcolor{textcolor}%
\pgftext[x=1.846812in,y=0.687231in,,]{\color{textcolor}{\ifdefined\pdftexversion\else\setmainfont{NanumMyeongjo}\rmfamily\fi\fontsize{5.000000}{6.000000}\selectfont\catcode`\^=\active\def^{\ifmmode\sp\else\^{}\fi}\catcode`\%=\active\def%{\%}13,405}}%
\end{pgfscope}%
\begin{pgfscope}%
\definecolor{textcolor}{rgb}{1.000000,1.000000,1.000000}%
\pgfsetstrokecolor{textcolor}%
\pgfsetfillcolor{textcolor}%
\pgftext[x=2.186069in,y=0.785401in,,]{\color{textcolor}{\ifdefined\pdftexversion\else\setmainfont{NanumMyeongjo}\rmfamily\fi\fontsize{5.000000}{6.000000}\selectfont\catcode`\^=\active\def^{\ifmmode\sp\else\^{}\fi}\catcode`\%=\active\def%{\%}17,511}}%
\end{pgfscope}%
\begin{pgfscope}%
\definecolor{textcolor}{rgb}{1.000000,1.000000,1.000000}%
\pgfsetstrokecolor{textcolor}%
\pgfsetfillcolor{textcolor}%
\pgftext[x=2.525326in,y=0.858117in,,]{\color{textcolor}{\ifdefined\pdftexversion\else\setmainfont{NanumMyeongjo}\rmfamily\fi\fontsize{5.000000}{6.000000}\selectfont\catcode`\^=\active\def^{\ifmmode\sp\else\^{}\fi}\catcode`\%=\active\def%{\%}18,776}}%
\end{pgfscope}%
\begin{pgfscope}%
\definecolor{textcolor}{rgb}{1.000000,1.000000,1.000000}%
\pgfsetstrokecolor{textcolor}%
\pgfsetfillcolor{textcolor}%
\pgftext[x=2.864583in,y=0.949010in,,]{\color{textcolor}{\ifdefined\pdftexversion\else\setmainfont{NanumMyeongjo}\rmfamily\fi\fontsize{5.000000}{6.000000}\selectfont\catcode`\^=\active\def^{\ifmmode\sp\else\^{}\fi}\catcode`\%=\active\def%{\%}21,264}}%
\end{pgfscope}%
\begin{pgfscope}%
\definecolor{textcolor}{rgb}{1.000000,1.000000,1.000000}%
\pgfsetstrokecolor{textcolor}%
\pgfsetfillcolor{textcolor}%
\pgftext[x=3.203840in,y=0.816444in,,]{\color{textcolor}{\ifdefined\pdftexversion\else\setmainfont{NanumMyeongjo}\rmfamily\fi\fontsize{5.000000}{6.000000}\selectfont\catcode`\^=\active\def^{\ifmmode\sp\else\^{}\fi}\catcode`\%=\active\def%{\%}17,258}}%
\end{pgfscope}%
\begin{pgfscope}%
\definecolor{textcolor}{rgb}{1.000000,1.000000,1.000000}%
\pgfsetstrokecolor{textcolor}%
\pgfsetfillcolor{textcolor}%
\pgftext[x=3.543097in,y=0.984390in,,]{\color{textcolor}{\ifdefined\pdftexversion\else\setmainfont{NanumMyeongjo}\rmfamily\fi\fontsize{5.000000}{6.000000}\selectfont\catcode`\^=\active\def^{\ifmmode\sp\else\^{}\fi}\catcode`\%=\active\def%{\%}24,981}}%
\end{pgfscope}%
\begin{pgfscope}%
\definecolor{textcolor}{rgb}{1.000000,1.000000,1.000000}%
\pgfsetstrokecolor{textcolor}%
\pgfsetfillcolor{textcolor}%
\pgftext[x=3.882355in,y=1.091378in,,]{\color{textcolor}{\ifdefined\pdftexversion\else\setmainfont{NanumMyeongjo}\rmfamily\fi\fontsize{5.000000}{6.000000}\selectfont\catcode`\^=\active\def^{\ifmmode\sp\else\^{}\fi}\catcode`\%=\active\def%{\%}24,905}}%
\end{pgfscope}%
\begin{pgfscope}%
\definecolor{textcolor}{rgb}{1.000000,1.000000,1.000000}%
\pgfsetstrokecolor{textcolor}%
\pgfsetfillcolor{textcolor}%
\pgftext[x=4.221612in,y=1.118736in,,]{\color{textcolor}{\ifdefined\pdftexversion\else\setmainfont{NanumMyeongjo}\rmfamily\fi\fontsize{5.000000}{6.000000}\selectfont\catcode`\^=\active\def^{\ifmmode\sp\else\^{}\fi}\catcode`\%=\active\def%{\%}22,850}}%
\end{pgfscope}%
\begin{pgfscope}%
\definecolor{textcolor}{rgb}{1.000000,1.000000,1.000000}%
\pgfsetstrokecolor{textcolor}%
\pgfsetfillcolor{textcolor}%
\pgftext[x=4.560869in,y=1.269799in,,]{\color{textcolor}{\ifdefined\pdftexversion\else\setmainfont{NanumMyeongjo}\rmfamily\fi\fontsize{5.000000}{6.000000}\selectfont\catcode`\^=\active\def^{\ifmmode\sp\else\^{}\fi}\catcode`\%=\active\def%{\%}24,830}}%
\end{pgfscope}%
\begin{pgfscope}%
\definecolor{textcolor}{rgb}{1.000000,1.000000,1.000000}%
\pgfsetstrokecolor{textcolor}%
\pgfsetfillcolor{textcolor}%
\pgftext[x=1.168298in,y=1.133010in,,]{\color{textcolor}{\ifdefined\pdftexversion\else\setmainfont{NanumMyeongjo}\rmfamily\fi\fontsize{5.000000}{6.000000}\selectfont\catcode`\^=\active\def^{\ifmmode\sp\else\^{}\fi}\catcode`\%=\active\def%{\%}20,519}}%
\end{pgfscope}%
\begin{pgfscope}%
\definecolor{textcolor}{rgb}{1.000000,1.000000,1.000000}%
\pgfsetstrokecolor{textcolor}%
\pgfsetfillcolor{textcolor}%
\pgftext[x=1.507555in,y=1.004626in,,]{\color{textcolor}{\ifdefined\pdftexversion\else\setmainfont{NanumMyeongjo}\rmfamily\fi\fontsize{5.000000}{6.000000}\selectfont\catcode`\^=\active\def^{\ifmmode\sp\else\^{}\fi}\catcode`\%=\active\def%{\%}20,265}}%
\end{pgfscope}%
\begin{pgfscope}%
\definecolor{textcolor}{rgb}{1.000000,1.000000,1.000000}%
\pgfsetstrokecolor{textcolor}%
\pgfsetfillcolor{textcolor}%
\pgftext[x=1.846812in,y=0.854546in,,]{\color{textcolor}{\ifdefined\pdftexversion\else\setmainfont{NanumMyeongjo}\rmfamily\fi\fontsize{5.000000}{6.000000}\selectfont\catcode`\^=\active\def^{\ifmmode\sp\else\^{}\fi}\catcode`\%=\active\def%{\%}16,164}}%
\end{pgfscope}%
\begin{pgfscope}%
\definecolor{textcolor}{rgb}{1.000000,1.000000,1.000000}%
\pgfsetstrokecolor{textcolor}%
\pgfsetfillcolor{textcolor}%
\pgftext[x=2.186069in,y=0.954641in,,]{\color{textcolor}{\ifdefined\pdftexversion\else\setmainfont{NanumMyeongjo}\rmfamily\fi\fontsize{5.000000}{6.000000}\selectfont\catcode`\^=\active\def^{\ifmmode\sp\else\^{}\fi}\catcode`\%=\active\def%{\%}16,350}}%
\end{pgfscope}%
\begin{pgfscope}%
\definecolor{textcolor}{rgb}{1.000000,1.000000,1.000000}%
\pgfsetstrokecolor{textcolor}%
\pgfsetfillcolor{textcolor}%
\pgftext[x=2.525326in,y=1.022668in,,]{\color{textcolor}{\ifdefined\pdftexversion\else\setmainfont{NanumMyeongjo}\rmfamily\fi\fontsize{5.000000}{6.000000}\selectfont\catcode`\^=\active\def^{\ifmmode\sp\else\^{}\fi}\catcode`\%=\active\def%{\%}15,897}}%
\end{pgfscope}%
\begin{pgfscope}%
\definecolor{textcolor}{rgb}{1.000000,1.000000,1.000000}%
\pgfsetstrokecolor{textcolor}%
\pgfsetfillcolor{textcolor}%
\pgftext[x=2.864583in,y=1.124502in,,]{\color{textcolor}{\ifdefined\pdftexversion\else\setmainfont{NanumMyeongjo}\rmfamily\fi\fontsize{5.000000}{6.000000}\selectfont\catcode`\^=\active\def^{\ifmmode\sp\else\^{}\fi}\catcode`\%=\active\def%{\%}16,954}}%
\end{pgfscope}%
\begin{pgfscope}%
\definecolor{textcolor}{rgb}{1.000000,1.000000,1.000000}%
\pgfsetstrokecolor{textcolor}%
\pgfsetfillcolor{textcolor}%
\pgftext[x=3.203840in,y=0.944165in,,]{\color{textcolor}{\ifdefined\pdftexversion\else\setmainfont{NanumMyeongjo}\rmfamily\fi\fontsize{5.000000}{6.000000}\selectfont\catcode`\^=\active\def^{\ifmmode\sp\else\^{}\fi}\catcode`\%=\active\def%{\%}12,339}}%
\end{pgfscope}%
\begin{pgfscope}%
\definecolor{textcolor}{rgb}{1.000000,1.000000,1.000000}%
\pgfsetstrokecolor{textcolor}%
\pgfsetfillcolor{textcolor}%
\pgftext[x=3.543097in,y=1.149158in,,]{\color{textcolor}{\ifdefined\pdftexversion\else\setmainfont{NanumMyeongjo}\rmfamily\fi\fontsize{5.000000}{6.000000}\selectfont\catcode`\^=\active\def^{\ifmmode\sp\else\^{}\fi}\catcode`\%=\active\def%{\%}15,918}}%
\end{pgfscope}%
\begin{pgfscope}%
\definecolor{textcolor}{rgb}{1.000000,1.000000,1.000000}%
\pgfsetstrokecolor{textcolor}%
\pgfsetfillcolor{textcolor}%
\pgftext[x=3.882355in,y=1.305376in,,]{\color{textcolor}{\ifdefined\pdftexversion\else\setmainfont{NanumMyeongjo}\rmfamily\fi\fontsize{5.000000}{6.000000}\selectfont\catcode`\^=\active\def^{\ifmmode\sp\else\^{}\fi}\catcode`\%=\active\def%{\%}20,674}}%
\end{pgfscope}%
\begin{pgfscope}%
\definecolor{textcolor}{rgb}{1.000000,1.000000,1.000000}%
\pgfsetstrokecolor{textcolor}%
\pgfsetfillcolor{textcolor}%
\pgftext[x=4.221612in,y=1.331202in,,]{\color{textcolor}{\ifdefined\pdftexversion\else\setmainfont{NanumMyeongjo}\rmfamily\fi\fontsize{5.000000}{6.000000}\selectfont\catcode`\^=\active\def^{\ifmmode\sp\else\^{}\fi}\catcode`\%=\active\def%{\%}20,526}}%
\end{pgfscope}%
\begin{pgfscope}%
\definecolor{textcolor}{rgb}{1.000000,1.000000,1.000000}%
\pgfsetstrokecolor{textcolor}%
\pgfsetfillcolor{textcolor}%
\pgftext[x=4.560869in,y=1.498030in,,]{\color{textcolor}{\ifdefined\pdftexversion\else\setmainfont{NanumMyeongjo}\rmfamily\fi\fontsize{5.000000}{6.000000}\selectfont\catcode`\^=\active\def^{\ifmmode\sp\else\^{}\fi}\catcode`\%=\active\def%{\%}22,049}}%
\end{pgfscope}%
\begin{pgfscope}%
\pgfsetbuttcap%
\pgfsetmiterjoin%
\definecolor{currentfill}{rgb}{0.337255,0.713725,0.627451}%
\pgfsetfillcolor{currentfill}%
\pgfsetlinewidth{1.003750pt}%
\definecolor{currentstroke}{rgb}{0.266667,0.266667,0.266667}%
\pgfsetstrokecolor{currentstroke}%
\pgfsetdash{}{0pt}%
\pgfpathmoveto{\pgfqpoint{4.843750in}{2.326470in}}%
\pgfpathlineto{\pgfqpoint{4.982639in}{2.326470in}}%
\pgfpathlineto{\pgfqpoint{4.982639in}{2.375081in}}%
\pgfpathlineto{\pgfqpoint{4.843750in}{2.375081in}}%
\pgfpathlineto{\pgfqpoint{4.843750in}{2.326470in}}%
\pgfpathclose%
\pgfusepath{stroke,fill}%
\end{pgfscope}%
\begin{pgfscope}%
\definecolor{textcolor}{rgb}{0.000000,0.000000,0.000000}%
\pgfsetstrokecolor{textcolor}%
\pgfsetfillcolor{textcolor}%
\pgftext[x=5.038194in,y=2.326470in,left,base]{\color{textcolor}{\ifdefined\pdftexversion\else\setmainfont{NanumMyeongjo}\rmfamily\fi\fontsize{5.000000}{6.000000}\selectfont\catcode`\^=\active\def^{\ifmmode\sp\else\^{}\fi}\catcode`\%=\active\def%{\%}전라북도}}%
\end{pgfscope}%
\begin{pgfscope}%
\pgfsetbuttcap%
\pgfsetmiterjoin%
\definecolor{currentfill}{rgb}{0.235294,0.490196,0.764706}%
\pgfsetfillcolor{currentfill}%
\pgfsetlinewidth{1.003750pt}%
\definecolor{currentstroke}{rgb}{0.266667,0.266667,0.266667}%
\pgfsetstrokecolor{currentstroke}%
\pgfsetdash{}{0pt}%
\pgfpathmoveto{\pgfqpoint{4.843750in}{2.220202in}}%
\pgfpathlineto{\pgfqpoint{4.982639in}{2.220202in}}%
\pgfpathlineto{\pgfqpoint{4.982639in}{2.268813in}}%
\pgfpathlineto{\pgfqpoint{4.843750in}{2.268813in}}%
\pgfpathlineto{\pgfqpoint{4.843750in}{2.220202in}}%
\pgfpathclose%
\pgfusepath{stroke,fill}%
\end{pgfscope}%
\begin{pgfscope}%
\definecolor{textcolor}{rgb}{0.000000,0.000000,0.000000}%
\pgfsetstrokecolor{textcolor}%
\pgfsetfillcolor{textcolor}%
\pgftext[x=5.038194in,y=2.220202in,left,base]{\color{textcolor}{\ifdefined\pdftexversion\else\setmainfont{NanumMyeongjo}\rmfamily\fi\fontsize{5.000000}{6.000000}\selectfont\catcode`\^=\active\def^{\ifmmode\sp\else\^{}\fi}\catcode`\%=\active\def%{\%}경상북도}}%
\end{pgfscope}%
\begin{pgfscope}%
\pgfsetbuttcap%
\pgfsetmiterjoin%
\definecolor{currentfill}{rgb}{0.725490,0.486275,0.164706}%
\pgfsetfillcolor{currentfill}%
\pgfsetlinewidth{1.003750pt}%
\definecolor{currentstroke}{rgb}{0.266667,0.266667,0.266667}%
\pgfsetstrokecolor{currentstroke}%
\pgfsetdash{}{0pt}%
\pgfpathmoveto{\pgfqpoint{4.843750in}{2.113933in}}%
\pgfpathlineto{\pgfqpoint{4.982639in}{2.113933in}}%
\pgfpathlineto{\pgfqpoint{4.982639in}{2.162544in}}%
\pgfpathlineto{\pgfqpoint{4.843750in}{2.162544in}}%
\pgfpathlineto{\pgfqpoint{4.843750in}{2.113933in}}%
\pgfpathclose%
\pgfusepath{stroke,fill}%
\end{pgfscope}%
\begin{pgfscope}%
\definecolor{textcolor}{rgb}{0.000000,0.000000,0.000000}%
\pgfsetstrokecolor{textcolor}%
\pgfsetfillcolor{textcolor}%
\pgftext[x=5.038194in,y=2.113933in,left,base]{\color{textcolor}{\ifdefined\pdftexversion\else\setmainfont{NanumMyeongjo}\rmfamily\fi\fontsize{5.000000}{6.000000}\selectfont\catcode`\^=\active\def^{\ifmmode\sp\else\^{}\fi}\catcode`\%=\active\def%{\%}전라남도}}%
\end{pgfscope}%
\begin{pgfscope}%
\pgfsetbuttcap%
\pgfsetmiterjoin%
\definecolor{currentfill}{rgb}{0.733333,0.321569,0.733333}%
\pgfsetfillcolor{currentfill}%
\pgfsetlinewidth{1.003750pt}%
\definecolor{currentstroke}{rgb}{0.266667,0.266667,0.266667}%
\pgfsetstrokecolor{currentstroke}%
\pgfsetdash{}{0pt}%
\pgfpathmoveto{\pgfqpoint{4.843750in}{2.007664in}}%
\pgfpathlineto{\pgfqpoint{4.982639in}{2.007664in}}%
\pgfpathlineto{\pgfqpoint{4.982639in}{2.056275in}}%
\pgfpathlineto{\pgfqpoint{4.843750in}{2.056275in}}%
\pgfpathlineto{\pgfqpoint{4.843750in}{2.007664in}}%
\pgfpathclose%
\pgfusepath{stroke,fill}%
\end{pgfscope}%
\begin{pgfscope}%
\definecolor{textcolor}{rgb}{0.000000,0.000000,0.000000}%
\pgfsetstrokecolor{textcolor}%
\pgfsetfillcolor{textcolor}%
\pgftext[x=5.038194in,y=2.007664in,left,base]{\color{textcolor}{\ifdefined\pdftexversion\else\setmainfont{NanumMyeongjo}\rmfamily\fi\fontsize{5.000000}{6.000000}\selectfont\catcode`\^=\active\def^{\ifmmode\sp\else\^{}\fi}\catcode`\%=\active\def%{\%}충청남도}}%
\end{pgfscope}%
\begin{pgfscope}%
\pgfsetbuttcap%
\pgfsetmiterjoin%
\definecolor{currentfill}{rgb}{0.549020,0.247059,0.121569}%
\pgfsetfillcolor{currentfill}%
\pgfsetlinewidth{1.003750pt}%
\definecolor{currentstroke}{rgb}{0.266667,0.266667,0.266667}%
\pgfsetstrokecolor{currentstroke}%
\pgfsetdash{}{0pt}%
\pgfpathmoveto{\pgfqpoint{4.843750in}{1.901395in}}%
\pgfpathlineto{\pgfqpoint{4.982639in}{1.901395in}}%
\pgfpathlineto{\pgfqpoint{4.982639in}{1.950006in}}%
\pgfpathlineto{\pgfqpoint{4.843750in}{1.950006in}}%
\pgfpathlineto{\pgfqpoint{4.843750in}{1.901395in}}%
\pgfpathclose%
\pgfusepath{stroke,fill}%
\end{pgfscope}%
\begin{pgfscope}%
\definecolor{textcolor}{rgb}{0.000000,0.000000,0.000000}%
\pgfsetstrokecolor{textcolor}%
\pgfsetfillcolor{textcolor}%
\pgftext[x=5.038194in,y=1.901395in,left,base]{\color{textcolor}{\ifdefined\pdftexversion\else\setmainfont{NanumMyeongjo}\rmfamily\fi\fontsize{5.000000}{6.000000}\selectfont\catcode`\^=\active\def^{\ifmmode\sp\else\^{}\fi}\catcode`\%=\active\def%{\%}충청북도}}%
\end{pgfscope}%
\begin{pgfscope}%
\pgfsetbuttcap%
\pgfsetmiterjoin%
\definecolor{currentfill}{rgb}{0.701961,0.760784,0.360784}%
\pgfsetfillcolor{currentfill}%
\pgfsetlinewidth{1.003750pt}%
\definecolor{currentstroke}{rgb}{0.266667,0.266667,0.266667}%
\pgfsetstrokecolor{currentstroke}%
\pgfsetdash{}{0pt}%
\pgfpathmoveto{\pgfqpoint{4.843750in}{1.795127in}}%
\pgfpathlineto{\pgfqpoint{4.982639in}{1.795127in}}%
\pgfpathlineto{\pgfqpoint{4.982639in}{1.843738in}}%
\pgfpathlineto{\pgfqpoint{4.843750in}{1.843738in}}%
\pgfpathlineto{\pgfqpoint{4.843750in}{1.795127in}}%
\pgfpathclose%
\pgfusepath{stroke,fill}%
\end{pgfscope}%
\begin{pgfscope}%
\definecolor{textcolor}{rgb}{0.000000,0.000000,0.000000}%
\pgfsetstrokecolor{textcolor}%
\pgfsetfillcolor{textcolor}%
\pgftext[x=5.038194in,y=1.795127in,left,base]{\color{textcolor}{\ifdefined\pdftexversion\else\setmainfont{NanumMyeongjo}\rmfamily\fi\fontsize{5.000000}{6.000000}\selectfont\catcode`\^=\active\def^{\ifmmode\sp\else\^{}\fi}\catcode`\%=\active\def%{\%}경기도}}%
\end{pgfscope}%
\begin{pgfscope}%
\pgfsetbuttcap%
\pgfsetmiterjoin%
\definecolor{currentfill}{rgb}{0.447059,0.447059,0.447059}%
\pgfsetfillcolor{currentfill}%
\pgfsetlinewidth{1.003750pt}%
\definecolor{currentstroke}{rgb}{0.266667,0.266667,0.266667}%
\pgfsetstrokecolor{currentstroke}%
\pgfsetdash{}{0pt}%
\pgfpathmoveto{\pgfqpoint{4.843750in}{1.688858in}}%
\pgfpathlineto{\pgfqpoint{4.982639in}{1.688858in}}%
\pgfpathlineto{\pgfqpoint{4.982639in}{1.737469in}}%
\pgfpathlineto{\pgfqpoint{4.843750in}{1.737469in}}%
\pgfpathlineto{\pgfqpoint{4.843750in}{1.688858in}}%
\pgfpathclose%
\pgfusepath{stroke,fill}%
\end{pgfscope}%
\begin{pgfscope}%
\definecolor{textcolor}{rgb}{0.000000,0.000000,0.000000}%
\pgfsetstrokecolor{textcolor}%
\pgfsetfillcolor{textcolor}%
\pgftext[x=5.038194in,y=1.688858in,left,base]{\color{textcolor}{\ifdefined\pdftexversion\else\setmainfont{NanumMyeongjo}\rmfamily\fi\fontsize{5.000000}{6.000000}\selectfont\catcode`\^=\active\def^{\ifmmode\sp\else\^{}\fi}\catcode`\%=\active\def%{\%}기타}}%
\end{pgfscope}%
\begin{pgfscope}%
\definecolor{textcolor}{rgb}{0.333333,0.333333,0.333333}%
\pgfsetstrokecolor{textcolor}%
\pgfsetfillcolor{textcolor}%
\pgftext[x=1.736111in,y=0.333333in,,top]{\color{textcolor}{\ifdefined\pdftexversion\else\setmainfont{NanumMyeongjo}\rmfamily\fi\fontsize{5.000000}{6.000000}\selectfont\catcode`\^=\active\def^{\ifmmode\sp\else\^{}\fi}\catcode`\%=\active\def%{\%}출처: 국가농식품통계서비스(KASS) 자료 기반 저자 작성}}%
\end{pgfscope}%
\begin{pgfscope}%
\definecolor{textcolor}{rgb}{0.333333,0.333333,0.333333}%
\pgfsetstrokecolor{textcolor}%
\pgfsetfillcolor{textcolor}%
\pgftext[x=4.513889in,y=2.583333in,,top]{\color{textcolor}{\ifdefined\pdftexversion\else\setmainfont{NanumMyeongjo}\rmfamily\fi\fontsize{5.000000}{6.000000}\selectfont\catcode`\^=\active\def^{\ifmmode\sp\else\^{}\fi}\catcode`\%=\active\def%{\%}(단위: 톤)}}%
\end{pgfscope}%
\end{pgfpicture}%
\makeatother%
\endgroup%
}
\end{center}
}


\slide
{\maintitle}
{\chapterthree}
{국내 콩 생산면적}{
\vspace{-10pt}
\begin{center}
    \hspace*{-40pt}{%% Creator: Matplotlib, PGF backend
%%
%% To include the figure in your LaTeX document, write
%%   \input{<filename>.pgf}
%%
%% Make sure the required packages are loaded in your preamble
%%   \usepackage{pgf}
%%
%% Also ensure that all the required font packages are loaded; for instance,
%% the lmodern package is sometimes necessary when using math font.
%%   \usepackage{lmodern}
%%
%% Figures using additional raster images can only be included by \input if
%% they are in the same directory as the main LaTeX file. For loading figures
%% from other directories you can use the `import` package
%%   \usepackage{import}
%%
%% and then include the figures with
%%   \import{<path to file>}{<filename>.pgf}
%%
%% Matplotlib used the following preamble
%%   \def\mathdefault#1{#1}
%%   \everymath=\expandafter{\the\everymath\displaystyle}
%%   \IfFileExists{scrextend.sty}{
%%     \usepackage[fontsize=9.000000pt]{scrextend}
%%   }{
%%     \renewcommand{\normalsize}{\fontsize{9.000000}{10.800000}\selectfont}
%%     \normalsize
%%   }
%%   
%%   \ifdefined\pdftexversion\else  % non-pdftex case.
%%     \usepackage{fontspec}
%%     \setmainfont{DejaVuSerif.ttf}[Path=\detokenize{/home/user/.cache/pypoetry/virtualenvs/graph-KASAOWVd-py3.12/lib/python3.12/site-packages/matplotlib/mpl-data/fonts/ttf/}]
%%     \setsansfont{DejaVuSans.ttf}[Path=\detokenize{/home/user/.cache/pypoetry/virtualenvs/graph-KASAOWVd-py3.12/lib/python3.12/site-packages/matplotlib/mpl-data/fonts/ttf/}]
%%     \setmonofont{DejaVuSansMono.ttf}[Path=\detokenize{/home/user/.cache/pypoetry/virtualenvs/graph-KASAOWVd-py3.12/lib/python3.12/site-packages/matplotlib/mpl-data/fonts/ttf/}]
%%   \fi
%%   \makeatletter\@ifpackageloaded{underscore}{}{\usepackage[strings]{underscore}}\makeatother
%%
\begingroup%
\makeatletter%
\begin{pgfpicture}%
\pgfpathrectangle{\pgfpointorigin}{\pgfqpoint{6.250000in}{3.194444in}}%
\pgfusepath{use as bounding box, clip}%
\begin{pgfscope}%
\pgfsetbuttcap%
\pgfsetmiterjoin%
\definecolor{currentfill}{rgb}{1.000000,1.000000,1.000000}%
\pgfsetfillcolor{currentfill}%
\pgfsetlinewidth{0.000000pt}%
\definecolor{currentstroke}{rgb}{1.000000,1.000000,1.000000}%
\pgfsetstrokecolor{currentstroke}%
\pgfsetdash{}{0pt}%
\pgfpathmoveto{\pgfqpoint{0.000000in}{0.000000in}}%
\pgfpathlineto{\pgfqpoint{6.250000in}{0.000000in}}%
\pgfpathlineto{\pgfqpoint{6.250000in}{3.194444in}}%
\pgfpathlineto{\pgfqpoint{0.000000in}{3.194444in}}%
\pgfpathlineto{\pgfqpoint{0.000000in}{0.000000in}}%
\pgfpathclose%
\pgfusepath{fill}%
\end{pgfscope}%
\begin{pgfscope}%
\pgfsetbuttcap%
\pgfsetmiterjoin%
\definecolor{currentfill}{rgb}{1.000000,1.000000,1.000000}%
\pgfsetfillcolor{currentfill}%
\pgfsetlinewidth{0.000000pt}%
\definecolor{currentstroke}{rgb}{0.000000,0.000000,0.000000}%
\pgfsetstrokecolor{currentstroke}%
\pgfsetstrokeopacity{0.000000}%
\pgfsetdash{}{0pt}%
\pgfpathmoveto{\pgfqpoint{0.781250in}{0.638889in}}%
\pgfpathlineto{\pgfqpoint{5.000000in}{0.638889in}}%
\pgfpathlineto{\pgfqpoint{5.000000in}{2.811111in}}%
\pgfpathlineto{\pgfqpoint{0.781250in}{2.811111in}}%
\pgfpathlineto{\pgfqpoint{0.781250in}{0.638889in}}%
\pgfpathclose%
\pgfusepath{fill}%
\end{pgfscope}%
\begin{pgfscope}%
\pgfsetbuttcap%
\pgfsetroundjoin%
\definecolor{currentfill}{rgb}{0.000000,0.000000,0.000000}%
\pgfsetfillcolor{currentfill}%
\pgfsetlinewidth{0.752812pt}%
\definecolor{currentstroke}{rgb}{0.000000,0.000000,0.000000}%
\pgfsetstrokecolor{currentstroke}%
\pgfsetdash{}{0pt}%
\pgfsys@defobject{currentmarker}{\pgfqpoint{0.000000in}{-0.013889in}}{\pgfqpoint{0.000000in}{0.000000in}}{%
\pgfpathmoveto{\pgfqpoint{0.000000in}{0.000000in}}%
\pgfpathlineto{\pgfqpoint{0.000000in}{-0.013889in}}%
\pgfusepath{stroke,fill}%
}%
\begin{pgfscope}%
\pgfsys@transformshift{1.098463in}{0.638889in}%
\pgfsys@useobject{currentmarker}{}%
\end{pgfscope}%
\end{pgfscope}%
\begin{pgfscope}%
\definecolor{textcolor}{rgb}{0.000000,0.000000,0.000000}%
\pgfsetstrokecolor{textcolor}%
\pgfsetfillcolor{textcolor}%
\pgftext[x=1.028589in, y=0.315885in, left, base,rotate=45.000000]{\color{textcolor}{\ifdefined\pdftexversion\else\setmainfont{NanumMyeongjo}\rmfamily\fi\fontsize{9.000000}{10.800000}\selectfont\catcode`\^=\active\def^{\ifmmode\sp\else\^{}\fi}\catcode`\%=\active\def%{\%}2014}}%
\end{pgfscope}%
\begin{pgfscope}%
\pgfsetbuttcap%
\pgfsetroundjoin%
\definecolor{currentfill}{rgb}{0.000000,0.000000,0.000000}%
\pgfsetfillcolor{currentfill}%
\pgfsetlinewidth{0.752812pt}%
\definecolor{currentstroke}{rgb}{0.000000,0.000000,0.000000}%
\pgfsetstrokecolor{currentstroke}%
\pgfsetdash{}{0pt}%
\pgfsys@defobject{currentmarker}{\pgfqpoint{0.000000in}{-0.013889in}}{\pgfqpoint{0.000000in}{0.000000in}}{%
\pgfpathmoveto{\pgfqpoint{0.000000in}{0.000000in}}%
\pgfpathlineto{\pgfqpoint{0.000000in}{-0.013889in}}%
\pgfusepath{stroke,fill}%
}%
\begin{pgfscope}%
\pgfsys@transformshift{1.456895in}{0.638889in}%
\pgfsys@useobject{currentmarker}{}%
\end{pgfscope}%
\end{pgfscope}%
\begin{pgfscope}%
\definecolor{textcolor}{rgb}{0.000000,0.000000,0.000000}%
\pgfsetstrokecolor{textcolor}%
\pgfsetfillcolor{textcolor}%
\pgftext[x=1.387022in, y=0.315885in, left, base,rotate=45.000000]{\color{textcolor}{\ifdefined\pdftexversion\else\setmainfont{NanumMyeongjo}\rmfamily\fi\fontsize{9.000000}{10.800000}\selectfont\catcode`\^=\active\def^{\ifmmode\sp\else\^{}\fi}\catcode`\%=\active\def%{\%}2015}}%
\end{pgfscope}%
\begin{pgfscope}%
\pgfsetbuttcap%
\pgfsetroundjoin%
\definecolor{currentfill}{rgb}{0.000000,0.000000,0.000000}%
\pgfsetfillcolor{currentfill}%
\pgfsetlinewidth{0.752812pt}%
\definecolor{currentstroke}{rgb}{0.000000,0.000000,0.000000}%
\pgfsetstrokecolor{currentstroke}%
\pgfsetdash{}{0pt}%
\pgfsys@defobject{currentmarker}{\pgfqpoint{0.000000in}{-0.013889in}}{\pgfqpoint{0.000000in}{0.000000in}}{%
\pgfpathmoveto{\pgfqpoint{0.000000in}{0.000000in}}%
\pgfpathlineto{\pgfqpoint{0.000000in}{-0.013889in}}%
\pgfusepath{stroke,fill}%
}%
\begin{pgfscope}%
\pgfsys@transformshift{1.815328in}{0.638889in}%
\pgfsys@useobject{currentmarker}{}%
\end{pgfscope}%
\end{pgfscope}%
\begin{pgfscope}%
\definecolor{textcolor}{rgb}{0.000000,0.000000,0.000000}%
\pgfsetstrokecolor{textcolor}%
\pgfsetfillcolor{textcolor}%
\pgftext[x=1.745454in, y=0.315885in, left, base,rotate=45.000000]{\color{textcolor}{\ifdefined\pdftexversion\else\setmainfont{NanumMyeongjo}\rmfamily\fi\fontsize{9.000000}{10.800000}\selectfont\catcode`\^=\active\def^{\ifmmode\sp\else\^{}\fi}\catcode`\%=\active\def%{\%}2016}}%
\end{pgfscope}%
\begin{pgfscope}%
\pgfsetbuttcap%
\pgfsetroundjoin%
\definecolor{currentfill}{rgb}{0.000000,0.000000,0.000000}%
\pgfsetfillcolor{currentfill}%
\pgfsetlinewidth{0.752812pt}%
\definecolor{currentstroke}{rgb}{0.000000,0.000000,0.000000}%
\pgfsetstrokecolor{currentstroke}%
\pgfsetdash{}{0pt}%
\pgfsys@defobject{currentmarker}{\pgfqpoint{0.000000in}{-0.013889in}}{\pgfqpoint{0.000000in}{0.000000in}}{%
\pgfpathmoveto{\pgfqpoint{0.000000in}{0.000000in}}%
\pgfpathlineto{\pgfqpoint{0.000000in}{-0.013889in}}%
\pgfusepath{stroke,fill}%
}%
\begin{pgfscope}%
\pgfsys@transformshift{2.173760in}{0.638889in}%
\pgfsys@useobject{currentmarker}{}%
\end{pgfscope}%
\end{pgfscope}%
\begin{pgfscope}%
\definecolor{textcolor}{rgb}{0.000000,0.000000,0.000000}%
\pgfsetstrokecolor{textcolor}%
\pgfsetfillcolor{textcolor}%
\pgftext[x=2.103887in, y=0.315885in, left, base,rotate=45.000000]{\color{textcolor}{\ifdefined\pdftexversion\else\setmainfont{NanumMyeongjo}\rmfamily\fi\fontsize{9.000000}{10.800000}\selectfont\catcode`\^=\active\def^{\ifmmode\sp\else\^{}\fi}\catcode`\%=\active\def%{\%}2017}}%
\end{pgfscope}%
\begin{pgfscope}%
\pgfsetbuttcap%
\pgfsetroundjoin%
\definecolor{currentfill}{rgb}{0.000000,0.000000,0.000000}%
\pgfsetfillcolor{currentfill}%
\pgfsetlinewidth{0.752812pt}%
\definecolor{currentstroke}{rgb}{0.000000,0.000000,0.000000}%
\pgfsetstrokecolor{currentstroke}%
\pgfsetdash{}{0pt}%
\pgfsys@defobject{currentmarker}{\pgfqpoint{0.000000in}{-0.013889in}}{\pgfqpoint{0.000000in}{0.000000in}}{%
\pgfpathmoveto{\pgfqpoint{0.000000in}{0.000000in}}%
\pgfpathlineto{\pgfqpoint{0.000000in}{-0.013889in}}%
\pgfusepath{stroke,fill}%
}%
\begin{pgfscope}%
\pgfsys@transformshift{2.532193in}{0.638889in}%
\pgfsys@useobject{currentmarker}{}%
\end{pgfscope}%
\end{pgfscope}%
\begin{pgfscope}%
\definecolor{textcolor}{rgb}{0.000000,0.000000,0.000000}%
\pgfsetstrokecolor{textcolor}%
\pgfsetfillcolor{textcolor}%
\pgftext[x=2.462319in, y=0.315885in, left, base,rotate=45.000000]{\color{textcolor}{\ifdefined\pdftexversion\else\setmainfont{NanumMyeongjo}\rmfamily\fi\fontsize{9.000000}{10.800000}\selectfont\catcode`\^=\active\def^{\ifmmode\sp\else\^{}\fi}\catcode`\%=\active\def%{\%}2018}}%
\end{pgfscope}%
\begin{pgfscope}%
\pgfsetbuttcap%
\pgfsetroundjoin%
\definecolor{currentfill}{rgb}{0.000000,0.000000,0.000000}%
\pgfsetfillcolor{currentfill}%
\pgfsetlinewidth{0.752812pt}%
\definecolor{currentstroke}{rgb}{0.000000,0.000000,0.000000}%
\pgfsetstrokecolor{currentstroke}%
\pgfsetdash{}{0pt}%
\pgfsys@defobject{currentmarker}{\pgfqpoint{0.000000in}{-0.013889in}}{\pgfqpoint{0.000000in}{0.000000in}}{%
\pgfpathmoveto{\pgfqpoint{0.000000in}{0.000000in}}%
\pgfpathlineto{\pgfqpoint{0.000000in}{-0.013889in}}%
\pgfusepath{stroke,fill}%
}%
\begin{pgfscope}%
\pgfsys@transformshift{2.890625in}{0.638889in}%
\pgfsys@useobject{currentmarker}{}%
\end{pgfscope}%
\end{pgfscope}%
\begin{pgfscope}%
\definecolor{textcolor}{rgb}{0.000000,0.000000,0.000000}%
\pgfsetstrokecolor{textcolor}%
\pgfsetfillcolor{textcolor}%
\pgftext[x=2.820752in, y=0.315885in, left, base,rotate=45.000000]{\color{textcolor}{\ifdefined\pdftexversion\else\setmainfont{NanumMyeongjo}\rmfamily\fi\fontsize{9.000000}{10.800000}\selectfont\catcode`\^=\active\def^{\ifmmode\sp\else\^{}\fi}\catcode`\%=\active\def%{\%}2019}}%
\end{pgfscope}%
\begin{pgfscope}%
\pgfsetbuttcap%
\pgfsetroundjoin%
\definecolor{currentfill}{rgb}{0.000000,0.000000,0.000000}%
\pgfsetfillcolor{currentfill}%
\pgfsetlinewidth{0.752812pt}%
\definecolor{currentstroke}{rgb}{0.000000,0.000000,0.000000}%
\pgfsetstrokecolor{currentstroke}%
\pgfsetdash{}{0pt}%
\pgfsys@defobject{currentmarker}{\pgfqpoint{0.000000in}{-0.013889in}}{\pgfqpoint{0.000000in}{0.000000in}}{%
\pgfpathmoveto{\pgfqpoint{0.000000in}{0.000000in}}%
\pgfpathlineto{\pgfqpoint{0.000000in}{-0.013889in}}%
\pgfusepath{stroke,fill}%
}%
\begin{pgfscope}%
\pgfsys@transformshift{3.249057in}{0.638889in}%
\pgfsys@useobject{currentmarker}{}%
\end{pgfscope}%
\end{pgfscope}%
\begin{pgfscope}%
\definecolor{textcolor}{rgb}{0.000000,0.000000,0.000000}%
\pgfsetstrokecolor{textcolor}%
\pgfsetfillcolor{textcolor}%
\pgftext[x=3.179184in, y=0.315885in, left, base,rotate=45.000000]{\color{textcolor}{\ifdefined\pdftexversion\else\setmainfont{NanumMyeongjo}\rmfamily\fi\fontsize{9.000000}{10.800000}\selectfont\catcode`\^=\active\def^{\ifmmode\sp\else\^{}\fi}\catcode`\%=\active\def%{\%}2020}}%
\end{pgfscope}%
\begin{pgfscope}%
\pgfsetbuttcap%
\pgfsetroundjoin%
\definecolor{currentfill}{rgb}{0.000000,0.000000,0.000000}%
\pgfsetfillcolor{currentfill}%
\pgfsetlinewidth{0.752812pt}%
\definecolor{currentstroke}{rgb}{0.000000,0.000000,0.000000}%
\pgfsetstrokecolor{currentstroke}%
\pgfsetdash{}{0pt}%
\pgfsys@defobject{currentmarker}{\pgfqpoint{0.000000in}{-0.013889in}}{\pgfqpoint{0.000000in}{0.000000in}}{%
\pgfpathmoveto{\pgfqpoint{0.000000in}{0.000000in}}%
\pgfpathlineto{\pgfqpoint{0.000000in}{-0.013889in}}%
\pgfusepath{stroke,fill}%
}%
\begin{pgfscope}%
\pgfsys@transformshift{3.607490in}{0.638889in}%
\pgfsys@useobject{currentmarker}{}%
\end{pgfscope}%
\end{pgfscope}%
\begin{pgfscope}%
\definecolor{textcolor}{rgb}{0.000000,0.000000,0.000000}%
\pgfsetstrokecolor{textcolor}%
\pgfsetfillcolor{textcolor}%
\pgftext[x=3.537617in, y=0.315885in, left, base,rotate=45.000000]{\color{textcolor}{\ifdefined\pdftexversion\else\setmainfont{NanumMyeongjo}\rmfamily\fi\fontsize{9.000000}{10.800000}\selectfont\catcode`\^=\active\def^{\ifmmode\sp\else\^{}\fi}\catcode`\%=\active\def%{\%}2021}}%
\end{pgfscope}%
\begin{pgfscope}%
\pgfsetbuttcap%
\pgfsetroundjoin%
\definecolor{currentfill}{rgb}{0.000000,0.000000,0.000000}%
\pgfsetfillcolor{currentfill}%
\pgfsetlinewidth{0.752812pt}%
\definecolor{currentstroke}{rgb}{0.000000,0.000000,0.000000}%
\pgfsetstrokecolor{currentstroke}%
\pgfsetdash{}{0pt}%
\pgfsys@defobject{currentmarker}{\pgfqpoint{0.000000in}{-0.013889in}}{\pgfqpoint{0.000000in}{0.000000in}}{%
\pgfpathmoveto{\pgfqpoint{0.000000in}{0.000000in}}%
\pgfpathlineto{\pgfqpoint{0.000000in}{-0.013889in}}%
\pgfusepath{stroke,fill}%
}%
\begin{pgfscope}%
\pgfsys@transformshift{3.965922in}{0.638889in}%
\pgfsys@useobject{currentmarker}{}%
\end{pgfscope}%
\end{pgfscope}%
\begin{pgfscope}%
\definecolor{textcolor}{rgb}{0.000000,0.000000,0.000000}%
\pgfsetstrokecolor{textcolor}%
\pgfsetfillcolor{textcolor}%
\pgftext[x=3.896049in, y=0.315885in, left, base,rotate=45.000000]{\color{textcolor}{\ifdefined\pdftexversion\else\setmainfont{NanumMyeongjo}\rmfamily\fi\fontsize{9.000000}{10.800000}\selectfont\catcode`\^=\active\def^{\ifmmode\sp\else\^{}\fi}\catcode`\%=\active\def%{\%}2022}}%
\end{pgfscope}%
\begin{pgfscope}%
\pgfsetbuttcap%
\pgfsetroundjoin%
\definecolor{currentfill}{rgb}{0.000000,0.000000,0.000000}%
\pgfsetfillcolor{currentfill}%
\pgfsetlinewidth{0.752812pt}%
\definecolor{currentstroke}{rgb}{0.000000,0.000000,0.000000}%
\pgfsetstrokecolor{currentstroke}%
\pgfsetdash{}{0pt}%
\pgfsys@defobject{currentmarker}{\pgfqpoint{0.000000in}{-0.013889in}}{\pgfqpoint{0.000000in}{0.000000in}}{%
\pgfpathmoveto{\pgfqpoint{0.000000in}{0.000000in}}%
\pgfpathlineto{\pgfqpoint{0.000000in}{-0.013889in}}%
\pgfusepath{stroke,fill}%
}%
\begin{pgfscope}%
\pgfsys@transformshift{4.324355in}{0.638889in}%
\pgfsys@useobject{currentmarker}{}%
\end{pgfscope}%
\end{pgfscope}%
\begin{pgfscope}%
\definecolor{textcolor}{rgb}{0.000000,0.000000,0.000000}%
\pgfsetstrokecolor{textcolor}%
\pgfsetfillcolor{textcolor}%
\pgftext[x=4.254481in, y=0.315885in, left, base,rotate=45.000000]{\color{textcolor}{\ifdefined\pdftexversion\else\setmainfont{NanumMyeongjo}\rmfamily\fi\fontsize{9.000000}{10.800000}\selectfont\catcode`\^=\active\def^{\ifmmode\sp\else\^{}\fi}\catcode`\%=\active\def%{\%}2023}}%
\end{pgfscope}%
\begin{pgfscope}%
\pgfsetbuttcap%
\pgfsetroundjoin%
\definecolor{currentfill}{rgb}{0.000000,0.000000,0.000000}%
\pgfsetfillcolor{currentfill}%
\pgfsetlinewidth{0.752812pt}%
\definecolor{currentstroke}{rgb}{0.000000,0.000000,0.000000}%
\pgfsetstrokecolor{currentstroke}%
\pgfsetdash{}{0pt}%
\pgfsys@defobject{currentmarker}{\pgfqpoint{0.000000in}{-0.013889in}}{\pgfqpoint{0.000000in}{0.000000in}}{%
\pgfpathmoveto{\pgfqpoint{0.000000in}{0.000000in}}%
\pgfpathlineto{\pgfqpoint{0.000000in}{-0.013889in}}%
\pgfusepath{stroke,fill}%
}%
\begin{pgfscope}%
\pgfsys@transformshift{4.682787in}{0.638889in}%
\pgfsys@useobject{currentmarker}{}%
\end{pgfscope}%
\end{pgfscope}%
\begin{pgfscope}%
\definecolor{textcolor}{rgb}{0.000000,0.000000,0.000000}%
\pgfsetstrokecolor{textcolor}%
\pgfsetfillcolor{textcolor}%
\pgftext[x=4.612914in, y=0.315885in, left, base,rotate=45.000000]{\color{textcolor}{\ifdefined\pdftexversion\else\setmainfont{NanumMyeongjo}\rmfamily\fi\fontsize{9.000000}{10.800000}\selectfont\catcode`\^=\active\def^{\ifmmode\sp\else\^{}\fi}\catcode`\%=\active\def%{\%}2024}}%
\end{pgfscope}%
\begin{pgfscope}%
\pgfpathrectangle{\pgfqpoint{0.781250in}{0.638889in}}{\pgfqpoint{4.218750in}{2.172222in}}%
\pgfusepath{clip}%
\pgfsetbuttcap%
\pgfsetroundjoin%
\pgfsetlinewidth{0.602250pt}%
\definecolor{currentstroke}{rgb}{0.690196,0.690196,0.690196}%
\pgfsetstrokecolor{currentstroke}%
\pgfsetstrokeopacity{0.450000}%
\pgfsetdash{{2.220000pt}{0.960000pt}}{0.000000pt}%
\pgfpathmoveto{\pgfqpoint{0.781250in}{0.638889in}}%
\pgfpathlineto{\pgfqpoint{5.000000in}{0.638889in}}%
\pgfusepath{stroke}%
\end{pgfscope}%
\begin{pgfscope}%
\pgfsetbuttcap%
\pgfsetroundjoin%
\definecolor{currentfill}{rgb}{0.000000,0.000000,0.000000}%
\pgfsetfillcolor{currentfill}%
\pgfsetlinewidth{0.752812pt}%
\definecolor{currentstroke}{rgb}{0.000000,0.000000,0.000000}%
\pgfsetstrokecolor{currentstroke}%
\pgfsetdash{}{0pt}%
\pgfsys@defobject{currentmarker}{\pgfqpoint{-0.013889in}{0.000000in}}{\pgfqpoint{-0.000000in}{0.000000in}}{%
\pgfpathmoveto{\pgfqpoint{-0.000000in}{0.000000in}}%
\pgfpathlineto{\pgfqpoint{-0.013889in}{0.000000in}}%
\pgfusepath{stroke,fill}%
}%
\begin{pgfscope}%
\pgfsys@transformshift{0.781250in}{0.638889in}%
\pgfsys@useobject{currentmarker}{}%
\end{pgfscope}%
\end{pgfscope}%
\begin{pgfscope}%
\definecolor{textcolor}{rgb}{0.000000,0.000000,0.000000}%
\pgfsetstrokecolor{textcolor}%
\pgfsetfillcolor{textcolor}%
\pgftext[x=0.651611in, y=0.588962in, left, base]{\color{textcolor}{\ifdefined\pdftexversion\else\setmainfont{NanumMyeongjo}\rmfamily\fi\fontsize{9.000000}{10.800000}\selectfont\catcode`\^=\active\def^{\ifmmode\sp\else\^{}\fi}\catcode`\%=\active\def%{\%}0}}%
\end{pgfscope}%
\begin{pgfscope}%
\pgfpathrectangle{\pgfqpoint{0.781250in}{0.638889in}}{\pgfqpoint{4.218750in}{2.172222in}}%
\pgfusepath{clip}%
\pgfsetbuttcap%
\pgfsetroundjoin%
\pgfsetlinewidth{0.602250pt}%
\definecolor{currentstroke}{rgb}{0.690196,0.690196,0.690196}%
\pgfsetstrokecolor{currentstroke}%
\pgfsetstrokeopacity{0.450000}%
\pgfsetdash{{2.220000pt}{0.960000pt}}{0.000000pt}%
\pgfpathmoveto{\pgfqpoint{0.781250in}{0.856111in}}%
\pgfpathlineto{\pgfqpoint{5.000000in}{0.856111in}}%
\pgfusepath{stroke}%
\end{pgfscope}%
\begin{pgfscope}%
\pgfsetbuttcap%
\pgfsetroundjoin%
\definecolor{currentfill}{rgb}{0.000000,0.000000,0.000000}%
\pgfsetfillcolor{currentfill}%
\pgfsetlinewidth{0.752812pt}%
\definecolor{currentstroke}{rgb}{0.000000,0.000000,0.000000}%
\pgfsetstrokecolor{currentstroke}%
\pgfsetdash{}{0pt}%
\pgfsys@defobject{currentmarker}{\pgfqpoint{-0.013889in}{0.000000in}}{\pgfqpoint{-0.000000in}{0.000000in}}{%
\pgfpathmoveto{\pgfqpoint{-0.000000in}{0.000000in}}%
\pgfpathlineto{\pgfqpoint{-0.013889in}{0.000000in}}%
\pgfusepath{stroke,fill}%
}%
\begin{pgfscope}%
\pgfsys@transformshift{0.781250in}{0.856111in}%
\pgfsys@useobject{currentmarker}{}%
\end{pgfscope}%
\end{pgfscope}%
\begin{pgfscope}%
\definecolor{textcolor}{rgb}{0.000000,0.000000,0.000000}%
\pgfsetstrokecolor{textcolor}%
\pgfsetfillcolor{textcolor}%
\pgftext[x=0.532837in, y=0.806184in, left, base]{\color{textcolor}{\ifdefined\pdftexversion\else\setmainfont{NanumMyeongjo}\rmfamily\fi\fontsize{9.000000}{10.800000}\selectfont\catcode`\^=\active\def^{\ifmmode\sp\else\^{}\fi}\catcode`\%=\active\def%{\%}1만}}%
\end{pgfscope}%
\begin{pgfscope}%
\pgfpathrectangle{\pgfqpoint{0.781250in}{0.638889in}}{\pgfqpoint{4.218750in}{2.172222in}}%
\pgfusepath{clip}%
\pgfsetbuttcap%
\pgfsetroundjoin%
\pgfsetlinewidth{0.602250pt}%
\definecolor{currentstroke}{rgb}{0.690196,0.690196,0.690196}%
\pgfsetstrokecolor{currentstroke}%
\pgfsetstrokeopacity{0.450000}%
\pgfsetdash{{2.220000pt}{0.960000pt}}{0.000000pt}%
\pgfpathmoveto{\pgfqpoint{0.781250in}{1.073333in}}%
\pgfpathlineto{\pgfqpoint{5.000000in}{1.073333in}}%
\pgfusepath{stroke}%
\end{pgfscope}%
\begin{pgfscope}%
\pgfsetbuttcap%
\pgfsetroundjoin%
\definecolor{currentfill}{rgb}{0.000000,0.000000,0.000000}%
\pgfsetfillcolor{currentfill}%
\pgfsetlinewidth{0.752812pt}%
\definecolor{currentstroke}{rgb}{0.000000,0.000000,0.000000}%
\pgfsetstrokecolor{currentstroke}%
\pgfsetdash{}{0pt}%
\pgfsys@defobject{currentmarker}{\pgfqpoint{-0.013889in}{0.000000in}}{\pgfqpoint{-0.000000in}{0.000000in}}{%
\pgfpathmoveto{\pgfqpoint{-0.000000in}{0.000000in}}%
\pgfpathlineto{\pgfqpoint{-0.013889in}{0.000000in}}%
\pgfusepath{stroke,fill}%
}%
\begin{pgfscope}%
\pgfsys@transformshift{0.781250in}{1.073333in}%
\pgfsys@useobject{currentmarker}{}%
\end{pgfscope}%
\end{pgfscope}%
\begin{pgfscope}%
\definecolor{textcolor}{rgb}{0.000000,0.000000,0.000000}%
\pgfsetstrokecolor{textcolor}%
\pgfsetfillcolor{textcolor}%
\pgftext[x=0.532837in, y=1.023407in, left, base]{\color{textcolor}{\ifdefined\pdftexversion\else\setmainfont{NanumMyeongjo}\rmfamily\fi\fontsize{9.000000}{10.800000}\selectfont\catcode`\^=\active\def^{\ifmmode\sp\else\^{}\fi}\catcode`\%=\active\def%{\%}2만}}%
\end{pgfscope}%
\begin{pgfscope}%
\pgfpathrectangle{\pgfqpoint{0.781250in}{0.638889in}}{\pgfqpoint{4.218750in}{2.172222in}}%
\pgfusepath{clip}%
\pgfsetbuttcap%
\pgfsetroundjoin%
\pgfsetlinewidth{0.602250pt}%
\definecolor{currentstroke}{rgb}{0.690196,0.690196,0.690196}%
\pgfsetstrokecolor{currentstroke}%
\pgfsetstrokeopacity{0.450000}%
\pgfsetdash{{2.220000pt}{0.960000pt}}{0.000000pt}%
\pgfpathmoveto{\pgfqpoint{0.781250in}{1.290556in}}%
\pgfpathlineto{\pgfqpoint{5.000000in}{1.290556in}}%
\pgfusepath{stroke}%
\end{pgfscope}%
\begin{pgfscope}%
\pgfsetbuttcap%
\pgfsetroundjoin%
\definecolor{currentfill}{rgb}{0.000000,0.000000,0.000000}%
\pgfsetfillcolor{currentfill}%
\pgfsetlinewidth{0.752812pt}%
\definecolor{currentstroke}{rgb}{0.000000,0.000000,0.000000}%
\pgfsetstrokecolor{currentstroke}%
\pgfsetdash{}{0pt}%
\pgfsys@defobject{currentmarker}{\pgfqpoint{-0.013889in}{0.000000in}}{\pgfqpoint{-0.000000in}{0.000000in}}{%
\pgfpathmoveto{\pgfqpoint{-0.000000in}{0.000000in}}%
\pgfpathlineto{\pgfqpoint{-0.013889in}{0.000000in}}%
\pgfusepath{stroke,fill}%
}%
\begin{pgfscope}%
\pgfsys@transformshift{0.781250in}{1.290556in}%
\pgfsys@useobject{currentmarker}{}%
\end{pgfscope}%
\end{pgfscope}%
\begin{pgfscope}%
\definecolor{textcolor}{rgb}{0.000000,0.000000,0.000000}%
\pgfsetstrokecolor{textcolor}%
\pgfsetfillcolor{textcolor}%
\pgftext[x=0.532837in, y=1.240629in, left, base]{\color{textcolor}{\ifdefined\pdftexversion\else\setmainfont{NanumMyeongjo}\rmfamily\fi\fontsize{9.000000}{10.800000}\selectfont\catcode`\^=\active\def^{\ifmmode\sp\else\^{}\fi}\catcode`\%=\active\def%{\%}3만}}%
\end{pgfscope}%
\begin{pgfscope}%
\pgfpathrectangle{\pgfqpoint{0.781250in}{0.638889in}}{\pgfqpoint{4.218750in}{2.172222in}}%
\pgfusepath{clip}%
\pgfsetbuttcap%
\pgfsetroundjoin%
\pgfsetlinewidth{0.602250pt}%
\definecolor{currentstroke}{rgb}{0.690196,0.690196,0.690196}%
\pgfsetstrokecolor{currentstroke}%
\pgfsetstrokeopacity{0.450000}%
\pgfsetdash{{2.220000pt}{0.960000pt}}{0.000000pt}%
\pgfpathmoveto{\pgfqpoint{0.781250in}{1.507778in}}%
\pgfpathlineto{\pgfqpoint{5.000000in}{1.507778in}}%
\pgfusepath{stroke}%
\end{pgfscope}%
\begin{pgfscope}%
\pgfsetbuttcap%
\pgfsetroundjoin%
\definecolor{currentfill}{rgb}{0.000000,0.000000,0.000000}%
\pgfsetfillcolor{currentfill}%
\pgfsetlinewidth{0.752812pt}%
\definecolor{currentstroke}{rgb}{0.000000,0.000000,0.000000}%
\pgfsetstrokecolor{currentstroke}%
\pgfsetdash{}{0pt}%
\pgfsys@defobject{currentmarker}{\pgfqpoint{-0.013889in}{0.000000in}}{\pgfqpoint{-0.000000in}{0.000000in}}{%
\pgfpathmoveto{\pgfqpoint{-0.000000in}{0.000000in}}%
\pgfpathlineto{\pgfqpoint{-0.013889in}{0.000000in}}%
\pgfusepath{stroke,fill}%
}%
\begin{pgfscope}%
\pgfsys@transformshift{0.781250in}{1.507778in}%
\pgfsys@useobject{currentmarker}{}%
\end{pgfscope}%
\end{pgfscope}%
\begin{pgfscope}%
\definecolor{textcolor}{rgb}{0.000000,0.000000,0.000000}%
\pgfsetstrokecolor{textcolor}%
\pgfsetfillcolor{textcolor}%
\pgftext[x=0.532837in, y=1.457851in, left, base]{\color{textcolor}{\ifdefined\pdftexversion\else\setmainfont{NanumMyeongjo}\rmfamily\fi\fontsize{9.000000}{10.800000}\selectfont\catcode`\^=\active\def^{\ifmmode\sp\else\^{}\fi}\catcode`\%=\active\def%{\%}4만}}%
\end{pgfscope}%
\begin{pgfscope}%
\pgfpathrectangle{\pgfqpoint{0.781250in}{0.638889in}}{\pgfqpoint{4.218750in}{2.172222in}}%
\pgfusepath{clip}%
\pgfsetbuttcap%
\pgfsetroundjoin%
\pgfsetlinewidth{0.602250pt}%
\definecolor{currentstroke}{rgb}{0.690196,0.690196,0.690196}%
\pgfsetstrokecolor{currentstroke}%
\pgfsetstrokeopacity{0.450000}%
\pgfsetdash{{2.220000pt}{0.960000pt}}{0.000000pt}%
\pgfpathmoveto{\pgfqpoint{0.781250in}{1.725000in}}%
\pgfpathlineto{\pgfqpoint{5.000000in}{1.725000in}}%
\pgfusepath{stroke}%
\end{pgfscope}%
\begin{pgfscope}%
\pgfsetbuttcap%
\pgfsetroundjoin%
\definecolor{currentfill}{rgb}{0.000000,0.000000,0.000000}%
\pgfsetfillcolor{currentfill}%
\pgfsetlinewidth{0.752812pt}%
\definecolor{currentstroke}{rgb}{0.000000,0.000000,0.000000}%
\pgfsetstrokecolor{currentstroke}%
\pgfsetdash{}{0pt}%
\pgfsys@defobject{currentmarker}{\pgfqpoint{-0.013889in}{0.000000in}}{\pgfqpoint{-0.000000in}{0.000000in}}{%
\pgfpathmoveto{\pgfqpoint{-0.000000in}{0.000000in}}%
\pgfpathlineto{\pgfqpoint{-0.013889in}{0.000000in}}%
\pgfusepath{stroke,fill}%
}%
\begin{pgfscope}%
\pgfsys@transformshift{0.781250in}{1.725000in}%
\pgfsys@useobject{currentmarker}{}%
\end{pgfscope}%
\end{pgfscope}%
\begin{pgfscope}%
\definecolor{textcolor}{rgb}{0.000000,0.000000,0.000000}%
\pgfsetstrokecolor{textcolor}%
\pgfsetfillcolor{textcolor}%
\pgftext[x=0.532837in, y=1.675073in, left, base]{\color{textcolor}{\ifdefined\pdftexversion\else\setmainfont{NanumMyeongjo}\rmfamily\fi\fontsize{9.000000}{10.800000}\selectfont\catcode`\^=\active\def^{\ifmmode\sp\else\^{}\fi}\catcode`\%=\active\def%{\%}5만}}%
\end{pgfscope}%
\begin{pgfscope}%
\pgfpathrectangle{\pgfqpoint{0.781250in}{0.638889in}}{\pgfqpoint{4.218750in}{2.172222in}}%
\pgfusepath{clip}%
\pgfsetbuttcap%
\pgfsetroundjoin%
\pgfsetlinewidth{0.602250pt}%
\definecolor{currentstroke}{rgb}{0.690196,0.690196,0.690196}%
\pgfsetstrokecolor{currentstroke}%
\pgfsetstrokeopacity{0.450000}%
\pgfsetdash{{2.220000pt}{0.960000pt}}{0.000000pt}%
\pgfpathmoveto{\pgfqpoint{0.781250in}{1.942222in}}%
\pgfpathlineto{\pgfqpoint{5.000000in}{1.942222in}}%
\pgfusepath{stroke}%
\end{pgfscope}%
\begin{pgfscope}%
\pgfsetbuttcap%
\pgfsetroundjoin%
\definecolor{currentfill}{rgb}{0.000000,0.000000,0.000000}%
\pgfsetfillcolor{currentfill}%
\pgfsetlinewidth{0.752812pt}%
\definecolor{currentstroke}{rgb}{0.000000,0.000000,0.000000}%
\pgfsetstrokecolor{currentstroke}%
\pgfsetdash{}{0pt}%
\pgfsys@defobject{currentmarker}{\pgfqpoint{-0.013889in}{0.000000in}}{\pgfqpoint{-0.000000in}{0.000000in}}{%
\pgfpathmoveto{\pgfqpoint{-0.000000in}{0.000000in}}%
\pgfpathlineto{\pgfqpoint{-0.013889in}{0.000000in}}%
\pgfusepath{stroke,fill}%
}%
\begin{pgfscope}%
\pgfsys@transformshift{0.781250in}{1.942222in}%
\pgfsys@useobject{currentmarker}{}%
\end{pgfscope}%
\end{pgfscope}%
\begin{pgfscope}%
\definecolor{textcolor}{rgb}{0.000000,0.000000,0.000000}%
\pgfsetstrokecolor{textcolor}%
\pgfsetfillcolor{textcolor}%
\pgftext[x=0.532837in, y=1.892295in, left, base]{\color{textcolor}{\ifdefined\pdftexversion\else\setmainfont{NanumMyeongjo}\rmfamily\fi\fontsize{9.000000}{10.800000}\selectfont\catcode`\^=\active\def^{\ifmmode\sp\else\^{}\fi}\catcode`\%=\active\def%{\%}6만}}%
\end{pgfscope}%
\begin{pgfscope}%
\pgfpathrectangle{\pgfqpoint{0.781250in}{0.638889in}}{\pgfqpoint{4.218750in}{2.172222in}}%
\pgfusepath{clip}%
\pgfsetbuttcap%
\pgfsetroundjoin%
\pgfsetlinewidth{0.602250pt}%
\definecolor{currentstroke}{rgb}{0.690196,0.690196,0.690196}%
\pgfsetstrokecolor{currentstroke}%
\pgfsetstrokeopacity{0.450000}%
\pgfsetdash{{2.220000pt}{0.960000pt}}{0.000000pt}%
\pgfpathmoveto{\pgfqpoint{0.781250in}{2.159444in}}%
\pgfpathlineto{\pgfqpoint{5.000000in}{2.159444in}}%
\pgfusepath{stroke}%
\end{pgfscope}%
\begin{pgfscope}%
\pgfsetbuttcap%
\pgfsetroundjoin%
\definecolor{currentfill}{rgb}{0.000000,0.000000,0.000000}%
\pgfsetfillcolor{currentfill}%
\pgfsetlinewidth{0.752812pt}%
\definecolor{currentstroke}{rgb}{0.000000,0.000000,0.000000}%
\pgfsetstrokecolor{currentstroke}%
\pgfsetdash{}{0pt}%
\pgfsys@defobject{currentmarker}{\pgfqpoint{-0.013889in}{0.000000in}}{\pgfqpoint{-0.000000in}{0.000000in}}{%
\pgfpathmoveto{\pgfqpoint{-0.000000in}{0.000000in}}%
\pgfpathlineto{\pgfqpoint{-0.013889in}{0.000000in}}%
\pgfusepath{stroke,fill}%
}%
\begin{pgfscope}%
\pgfsys@transformshift{0.781250in}{2.159444in}%
\pgfsys@useobject{currentmarker}{}%
\end{pgfscope}%
\end{pgfscope}%
\begin{pgfscope}%
\definecolor{textcolor}{rgb}{0.000000,0.000000,0.000000}%
\pgfsetstrokecolor{textcolor}%
\pgfsetfillcolor{textcolor}%
\pgftext[x=0.532837in, y=2.109518in, left, base]{\color{textcolor}{\ifdefined\pdftexversion\else\setmainfont{NanumMyeongjo}\rmfamily\fi\fontsize{9.000000}{10.800000}\selectfont\catcode`\^=\active\def^{\ifmmode\sp\else\^{}\fi}\catcode`\%=\active\def%{\%}7만}}%
\end{pgfscope}%
\begin{pgfscope}%
\pgfpathrectangle{\pgfqpoint{0.781250in}{0.638889in}}{\pgfqpoint{4.218750in}{2.172222in}}%
\pgfusepath{clip}%
\pgfsetbuttcap%
\pgfsetroundjoin%
\pgfsetlinewidth{0.602250pt}%
\definecolor{currentstroke}{rgb}{0.690196,0.690196,0.690196}%
\pgfsetstrokecolor{currentstroke}%
\pgfsetstrokeopacity{0.450000}%
\pgfsetdash{{2.220000pt}{0.960000pt}}{0.000000pt}%
\pgfpathmoveto{\pgfqpoint{0.781250in}{2.376667in}}%
\pgfpathlineto{\pgfqpoint{5.000000in}{2.376667in}}%
\pgfusepath{stroke}%
\end{pgfscope}%
\begin{pgfscope}%
\pgfsetbuttcap%
\pgfsetroundjoin%
\definecolor{currentfill}{rgb}{0.000000,0.000000,0.000000}%
\pgfsetfillcolor{currentfill}%
\pgfsetlinewidth{0.752812pt}%
\definecolor{currentstroke}{rgb}{0.000000,0.000000,0.000000}%
\pgfsetstrokecolor{currentstroke}%
\pgfsetdash{}{0pt}%
\pgfsys@defobject{currentmarker}{\pgfqpoint{-0.013889in}{0.000000in}}{\pgfqpoint{-0.000000in}{0.000000in}}{%
\pgfpathmoveto{\pgfqpoint{-0.000000in}{0.000000in}}%
\pgfpathlineto{\pgfqpoint{-0.013889in}{0.000000in}}%
\pgfusepath{stroke,fill}%
}%
\begin{pgfscope}%
\pgfsys@transformshift{0.781250in}{2.376667in}%
\pgfsys@useobject{currentmarker}{}%
\end{pgfscope}%
\end{pgfscope}%
\begin{pgfscope}%
\definecolor{textcolor}{rgb}{0.000000,0.000000,0.000000}%
\pgfsetstrokecolor{textcolor}%
\pgfsetfillcolor{textcolor}%
\pgftext[x=0.532837in, y=2.326740in, left, base]{\color{textcolor}{\ifdefined\pdftexversion\else\setmainfont{NanumMyeongjo}\rmfamily\fi\fontsize{9.000000}{10.800000}\selectfont\catcode`\^=\active\def^{\ifmmode\sp\else\^{}\fi}\catcode`\%=\active\def%{\%}8만}}%
\end{pgfscope}%
\begin{pgfscope}%
\pgfpathrectangle{\pgfqpoint{0.781250in}{0.638889in}}{\pgfqpoint{4.218750in}{2.172222in}}%
\pgfusepath{clip}%
\pgfsetbuttcap%
\pgfsetroundjoin%
\pgfsetlinewidth{0.602250pt}%
\definecolor{currentstroke}{rgb}{0.690196,0.690196,0.690196}%
\pgfsetstrokecolor{currentstroke}%
\pgfsetstrokeopacity{0.450000}%
\pgfsetdash{{2.220000pt}{0.960000pt}}{0.000000pt}%
\pgfpathmoveto{\pgfqpoint{0.781250in}{2.593889in}}%
\pgfpathlineto{\pgfqpoint{5.000000in}{2.593889in}}%
\pgfusepath{stroke}%
\end{pgfscope}%
\begin{pgfscope}%
\pgfsetbuttcap%
\pgfsetroundjoin%
\definecolor{currentfill}{rgb}{0.000000,0.000000,0.000000}%
\pgfsetfillcolor{currentfill}%
\pgfsetlinewidth{0.752812pt}%
\definecolor{currentstroke}{rgb}{0.000000,0.000000,0.000000}%
\pgfsetstrokecolor{currentstroke}%
\pgfsetdash{}{0pt}%
\pgfsys@defobject{currentmarker}{\pgfqpoint{-0.013889in}{0.000000in}}{\pgfqpoint{-0.000000in}{0.000000in}}{%
\pgfpathmoveto{\pgfqpoint{-0.000000in}{0.000000in}}%
\pgfpathlineto{\pgfqpoint{-0.013889in}{0.000000in}}%
\pgfusepath{stroke,fill}%
}%
\begin{pgfscope}%
\pgfsys@transformshift{0.781250in}{2.593889in}%
\pgfsys@useobject{currentmarker}{}%
\end{pgfscope}%
\end{pgfscope}%
\begin{pgfscope}%
\definecolor{textcolor}{rgb}{0.000000,0.000000,0.000000}%
\pgfsetstrokecolor{textcolor}%
\pgfsetfillcolor{textcolor}%
\pgftext[x=0.532837in, y=2.543962in, left, base]{\color{textcolor}{\ifdefined\pdftexversion\else\setmainfont{NanumMyeongjo}\rmfamily\fi\fontsize{9.000000}{10.800000}\selectfont\catcode`\^=\active\def^{\ifmmode\sp\else\^{}\fi}\catcode`\%=\active\def%{\%}9만}}%
\end{pgfscope}%
\begin{pgfscope}%
\pgfpathrectangle{\pgfqpoint{0.781250in}{0.638889in}}{\pgfqpoint{4.218750in}{2.172222in}}%
\pgfusepath{clip}%
\pgfsetbuttcap%
\pgfsetroundjoin%
\pgfsetlinewidth{0.602250pt}%
\definecolor{currentstroke}{rgb}{0.690196,0.690196,0.690196}%
\pgfsetstrokecolor{currentstroke}%
\pgfsetstrokeopacity{0.450000}%
\pgfsetdash{{2.220000pt}{0.960000pt}}{0.000000pt}%
\pgfpathmoveto{\pgfqpoint{0.781250in}{2.811111in}}%
\pgfpathlineto{\pgfqpoint{5.000000in}{2.811111in}}%
\pgfusepath{stroke}%
\end{pgfscope}%
\begin{pgfscope}%
\pgfsetbuttcap%
\pgfsetroundjoin%
\definecolor{currentfill}{rgb}{0.000000,0.000000,0.000000}%
\pgfsetfillcolor{currentfill}%
\pgfsetlinewidth{0.752812pt}%
\definecolor{currentstroke}{rgb}{0.000000,0.000000,0.000000}%
\pgfsetstrokecolor{currentstroke}%
\pgfsetdash{}{0pt}%
\pgfsys@defobject{currentmarker}{\pgfqpoint{-0.013889in}{0.000000in}}{\pgfqpoint{-0.000000in}{0.000000in}}{%
\pgfpathmoveto{\pgfqpoint{-0.000000in}{0.000000in}}%
\pgfpathlineto{\pgfqpoint{-0.013889in}{0.000000in}}%
\pgfusepath{stroke,fill}%
}%
\begin{pgfscope}%
\pgfsys@transformshift{0.781250in}{2.811111in}%
\pgfsys@useobject{currentmarker}{}%
\end{pgfscope}%
\end{pgfscope}%
\begin{pgfscope}%
\definecolor{textcolor}{rgb}{0.000000,0.000000,0.000000}%
\pgfsetstrokecolor{textcolor}%
\pgfsetfillcolor{textcolor}%
\pgftext[x=0.465698in, y=2.761184in, left, base]{\color{textcolor}{\ifdefined\pdftexversion\else\setmainfont{NanumMyeongjo}\rmfamily\fi\fontsize{9.000000}{10.800000}\selectfont\catcode`\^=\active\def^{\ifmmode\sp\else\^{}\fi}\catcode`\%=\active\def%{\%}10만}}%
\end{pgfscope}%
\begin{pgfscope}%
\pgfsetrectcap%
\pgfsetmiterjoin%
\pgfsetlinewidth{0.752812pt}%
\definecolor{currentstroke}{rgb}{0.000000,0.000000,0.000000}%
\pgfsetstrokecolor{currentstroke}%
\pgfsetdash{}{0pt}%
\pgfpathmoveto{\pgfqpoint{0.781250in}{0.638889in}}%
\pgfpathlineto{\pgfqpoint{0.781250in}{2.811111in}}%
\pgfusepath{stroke}%
\end{pgfscope}%
\begin{pgfscope}%
\pgfsetrectcap%
\pgfsetmiterjoin%
\pgfsetlinewidth{0.752812pt}%
\definecolor{currentstroke}{rgb}{0.000000,0.000000,0.000000}%
\pgfsetstrokecolor{currentstroke}%
\pgfsetdash{}{0pt}%
\pgfpathmoveto{\pgfqpoint{0.781250in}{0.638889in}}%
\pgfpathlineto{\pgfqpoint{5.000000in}{0.638889in}}%
\pgfusepath{stroke}%
\end{pgfscope}%
\begin{pgfscope}%
\pgfpathrectangle{\pgfqpoint{0.781250in}{0.638889in}}{\pgfqpoint{4.218750in}{2.172222in}}%
\pgfusepath{clip}%
\pgfsetbuttcap%
\pgfsetmiterjoin%
\definecolor{currentfill}{rgb}{0.337255,0.713725,0.627451}%
\pgfsetfillcolor{currentfill}%
\pgfsetlinewidth{1.003750pt}%
\definecolor{currentstroke}{rgb}{0.266667,0.266667,0.266667}%
\pgfsetstrokecolor{currentstroke}%
\pgfsetdash{}{0pt}%
\pgfpathmoveto{\pgfqpoint{0.973011in}{0.638889in}}%
\pgfpathlineto{\pgfqpoint{1.223914in}{0.638889in}}%
\pgfpathlineto{\pgfqpoint{1.223914in}{0.796245in}}%
\pgfpathlineto{\pgfqpoint{0.973011in}{0.796245in}}%
\pgfpathlineto{\pgfqpoint{0.973011in}{0.638889in}}%
\pgfpathclose%
\pgfusepath{stroke,fill}%
\end{pgfscope}%
\begin{pgfscope}%
\pgfpathrectangle{\pgfqpoint{0.781250in}{0.638889in}}{\pgfqpoint{4.218750in}{2.172222in}}%
\pgfusepath{clip}%
\pgfsetbuttcap%
\pgfsetmiterjoin%
\definecolor{currentfill}{rgb}{0.337255,0.713725,0.627451}%
\pgfsetfillcolor{currentfill}%
\pgfsetlinewidth{1.003750pt}%
\definecolor{currentstroke}{rgb}{0.266667,0.266667,0.266667}%
\pgfsetstrokecolor{currentstroke}%
\pgfsetdash{}{0pt}%
\pgfpathmoveto{\pgfqpoint{1.331444in}{0.638889in}}%
\pgfpathlineto{\pgfqpoint{1.582347in}{0.638889in}}%
\pgfpathlineto{\pgfqpoint{1.582347in}{0.751823in}}%
\pgfpathlineto{\pgfqpoint{1.331444in}{0.751823in}}%
\pgfpathlineto{\pgfqpoint{1.331444in}{0.638889in}}%
\pgfpathclose%
\pgfusepath{stroke,fill}%
\end{pgfscope}%
\begin{pgfscope}%
\pgfpathrectangle{\pgfqpoint{0.781250in}{0.638889in}}{\pgfqpoint{4.218750in}{2.172222in}}%
\pgfusepath{clip}%
\pgfsetbuttcap%
\pgfsetmiterjoin%
\definecolor{currentfill}{rgb}{0.337255,0.713725,0.627451}%
\pgfsetfillcolor{currentfill}%
\pgfsetlinewidth{1.003750pt}%
\definecolor{currentstroke}{rgb}{0.266667,0.266667,0.266667}%
\pgfsetstrokecolor{currentstroke}%
\pgfsetdash{}{0pt}%
\pgfpathmoveto{\pgfqpoint{1.689876in}{0.638889in}}%
\pgfpathlineto{\pgfqpoint{1.940779in}{0.638889in}}%
\pgfpathlineto{\pgfqpoint{1.940779in}{0.746436in}}%
\pgfpathlineto{\pgfqpoint{1.689876in}{0.746436in}}%
\pgfpathlineto{\pgfqpoint{1.689876in}{0.638889in}}%
\pgfpathclose%
\pgfusepath{stroke,fill}%
\end{pgfscope}%
\begin{pgfscope}%
\pgfpathrectangle{\pgfqpoint{0.781250in}{0.638889in}}{\pgfqpoint{4.218750in}{2.172222in}}%
\pgfusepath{clip}%
\pgfsetbuttcap%
\pgfsetmiterjoin%
\definecolor{currentfill}{rgb}{0.337255,0.713725,0.627451}%
\pgfsetfillcolor{currentfill}%
\pgfsetlinewidth{1.003750pt}%
\definecolor{currentstroke}{rgb}{0.266667,0.266667,0.266667}%
\pgfsetstrokecolor{currentstroke}%
\pgfsetdash{}{0pt}%
\pgfpathmoveto{\pgfqpoint{2.048309in}{0.638889in}}%
\pgfpathlineto{\pgfqpoint{2.299211in}{0.638889in}}%
\pgfpathlineto{\pgfqpoint{2.299211in}{0.770938in}}%
\pgfpathlineto{\pgfqpoint{2.048309in}{0.770938in}}%
\pgfpathlineto{\pgfqpoint{2.048309in}{0.638889in}}%
\pgfpathclose%
\pgfusepath{stroke,fill}%
\end{pgfscope}%
\begin{pgfscope}%
\pgfpathrectangle{\pgfqpoint{0.781250in}{0.638889in}}{\pgfqpoint{4.218750in}{2.172222in}}%
\pgfusepath{clip}%
\pgfsetbuttcap%
\pgfsetmiterjoin%
\definecolor{currentfill}{rgb}{0.337255,0.713725,0.627451}%
\pgfsetfillcolor{currentfill}%
\pgfsetlinewidth{1.003750pt}%
\definecolor{currentstroke}{rgb}{0.266667,0.266667,0.266667}%
\pgfsetstrokecolor{currentstroke}%
\pgfsetdash{}{0pt}%
\pgfpathmoveto{\pgfqpoint{2.406741in}{0.638889in}}%
\pgfpathlineto{\pgfqpoint{2.657644in}{0.638889in}}%
\pgfpathlineto{\pgfqpoint{2.657644in}{0.831847in}}%
\pgfpathlineto{\pgfqpoint{2.406741in}{0.831847in}}%
\pgfpathlineto{\pgfqpoint{2.406741in}{0.638889in}}%
\pgfpathclose%
\pgfusepath{stroke,fill}%
\end{pgfscope}%
\begin{pgfscope}%
\pgfpathrectangle{\pgfqpoint{0.781250in}{0.638889in}}{\pgfqpoint{4.218750in}{2.172222in}}%
\pgfusepath{clip}%
\pgfsetbuttcap%
\pgfsetmiterjoin%
\definecolor{currentfill}{rgb}{0.337255,0.713725,0.627451}%
\pgfsetfillcolor{currentfill}%
\pgfsetlinewidth{1.003750pt}%
\definecolor{currentstroke}{rgb}{0.266667,0.266667,0.266667}%
\pgfsetstrokecolor{currentstroke}%
\pgfsetdash{}{0pt}%
\pgfpathmoveto{\pgfqpoint{2.765174in}{0.638889in}}%
\pgfpathlineto{\pgfqpoint{3.016076in}{0.638889in}}%
\pgfpathlineto{\pgfqpoint{3.016076in}{0.909048in}}%
\pgfpathlineto{\pgfqpoint{2.765174in}{0.909048in}}%
\pgfpathlineto{\pgfqpoint{2.765174in}{0.638889in}}%
\pgfpathclose%
\pgfusepath{stroke,fill}%
\end{pgfscope}%
\begin{pgfscope}%
\pgfpathrectangle{\pgfqpoint{0.781250in}{0.638889in}}{\pgfqpoint{4.218750in}{2.172222in}}%
\pgfusepath{clip}%
\pgfsetbuttcap%
\pgfsetmiterjoin%
\definecolor{currentfill}{rgb}{0.337255,0.713725,0.627451}%
\pgfsetfillcolor{currentfill}%
\pgfsetlinewidth{1.003750pt}%
\definecolor{currentstroke}{rgb}{0.266667,0.266667,0.266667}%
\pgfsetstrokecolor{currentstroke}%
\pgfsetdash{}{0pt}%
\pgfpathmoveto{\pgfqpoint{3.123606in}{0.638889in}}%
\pgfpathlineto{\pgfqpoint{3.374509in}{0.638889in}}%
\pgfpathlineto{\pgfqpoint{3.374509in}{0.892474in}}%
\pgfpathlineto{\pgfqpoint{3.123606in}{0.892474in}}%
\pgfpathlineto{\pgfqpoint{3.123606in}{0.638889in}}%
\pgfpathclose%
\pgfusepath{stroke,fill}%
\end{pgfscope}%
\begin{pgfscope}%
\pgfpathrectangle{\pgfqpoint{0.781250in}{0.638889in}}{\pgfqpoint{4.218750in}{2.172222in}}%
\pgfusepath{clip}%
\pgfsetbuttcap%
\pgfsetmiterjoin%
\definecolor{currentfill}{rgb}{0.337255,0.713725,0.627451}%
\pgfsetfillcolor{currentfill}%
\pgfsetlinewidth{1.003750pt}%
\definecolor{currentstroke}{rgb}{0.266667,0.266667,0.266667}%
\pgfsetstrokecolor{currentstroke}%
\pgfsetdash{}{0pt}%
\pgfpathmoveto{\pgfqpoint{3.482039in}{0.638889in}}%
\pgfpathlineto{\pgfqpoint{3.732941in}{0.638889in}}%
\pgfpathlineto{\pgfqpoint{3.732941in}{0.860499in}}%
\pgfpathlineto{\pgfqpoint{3.482039in}{0.860499in}}%
\pgfpathlineto{\pgfqpoint{3.482039in}{0.638889in}}%
\pgfpathclose%
\pgfusepath{stroke,fill}%
\end{pgfscope}%
\begin{pgfscope}%
\pgfpathrectangle{\pgfqpoint{0.781250in}{0.638889in}}{\pgfqpoint{4.218750in}{2.172222in}}%
\pgfusepath{clip}%
\pgfsetbuttcap%
\pgfsetmiterjoin%
\definecolor{currentfill}{rgb}{0.337255,0.713725,0.627451}%
\pgfsetfillcolor{currentfill}%
\pgfsetlinewidth{1.003750pt}%
\definecolor{currentstroke}{rgb}{0.266667,0.266667,0.266667}%
\pgfsetstrokecolor{currentstroke}%
\pgfsetdash{}{0pt}%
\pgfpathmoveto{\pgfqpoint{3.840471in}{0.638889in}}%
\pgfpathlineto{\pgfqpoint{4.091374in}{0.638889in}}%
\pgfpathlineto{\pgfqpoint{4.091374in}{0.931031in}}%
\pgfpathlineto{\pgfqpoint{3.840471in}{0.931031in}}%
\pgfpathlineto{\pgfqpoint{3.840471in}{0.638889in}}%
\pgfpathclose%
\pgfusepath{stroke,fill}%
\end{pgfscope}%
\begin{pgfscope}%
\pgfpathrectangle{\pgfqpoint{0.781250in}{0.638889in}}{\pgfqpoint{4.218750in}{2.172222in}}%
\pgfusepath{clip}%
\pgfsetbuttcap%
\pgfsetmiterjoin%
\definecolor{currentfill}{rgb}{0.337255,0.713725,0.627451}%
\pgfsetfillcolor{currentfill}%
\pgfsetlinewidth{1.003750pt}%
\definecolor{currentstroke}{rgb}{0.266667,0.266667,0.266667}%
\pgfsetstrokecolor{currentstroke}%
\pgfsetdash{}{0pt}%
\pgfpathmoveto{\pgfqpoint{4.198903in}{0.638889in}}%
\pgfpathlineto{\pgfqpoint{4.449806in}{0.638889in}}%
\pgfpathlineto{\pgfqpoint{4.449806in}{1.020874in}}%
\pgfpathlineto{\pgfqpoint{4.198903in}{1.020874in}}%
\pgfpathlineto{\pgfqpoint{4.198903in}{0.638889in}}%
\pgfpathclose%
\pgfusepath{stroke,fill}%
\end{pgfscope}%
\begin{pgfscope}%
\pgfpathrectangle{\pgfqpoint{0.781250in}{0.638889in}}{\pgfqpoint{4.218750in}{2.172222in}}%
\pgfusepath{clip}%
\pgfsetbuttcap%
\pgfsetmiterjoin%
\definecolor{currentfill}{rgb}{0.337255,0.713725,0.627451}%
\pgfsetfillcolor{currentfill}%
\pgfsetlinewidth{1.003750pt}%
\definecolor{currentstroke}{rgb}{0.266667,0.266667,0.266667}%
\pgfsetstrokecolor{currentstroke}%
\pgfsetdash{}{0pt}%
\pgfpathmoveto{\pgfqpoint{4.557336in}{0.638889in}}%
\pgfpathlineto{\pgfqpoint{4.808239in}{0.638889in}}%
\pgfpathlineto{\pgfqpoint{4.808239in}{1.086823in}}%
\pgfpathlineto{\pgfqpoint{4.557336in}{1.086823in}}%
\pgfpathlineto{\pgfqpoint{4.557336in}{0.638889in}}%
\pgfpathclose%
\pgfusepath{stroke,fill}%
\end{pgfscope}%
\begin{pgfscope}%
\pgfpathrectangle{\pgfqpoint{0.781250in}{0.638889in}}{\pgfqpoint{4.218750in}{2.172222in}}%
\pgfusepath{clip}%
\pgfsetbuttcap%
\pgfsetmiterjoin%
\definecolor{currentfill}{rgb}{0.235294,0.490196,0.764706}%
\pgfsetfillcolor{currentfill}%
\pgfsetlinewidth{1.003750pt}%
\definecolor{currentstroke}{rgb}{0.266667,0.266667,0.266667}%
\pgfsetstrokecolor{currentstroke}%
\pgfsetdash{}{0pt}%
\pgfpathmoveto{\pgfqpoint{0.973011in}{0.796245in}}%
\pgfpathlineto{\pgfqpoint{1.223914in}{0.796245in}}%
\pgfpathlineto{\pgfqpoint{1.223914in}{1.126553in}}%
\pgfpathlineto{\pgfqpoint{0.973011in}{1.126553in}}%
\pgfpathlineto{\pgfqpoint{0.973011in}{0.796245in}}%
\pgfpathclose%
\pgfusepath{stroke,fill}%
\end{pgfscope}%
\begin{pgfscope}%
\pgfpathrectangle{\pgfqpoint{0.781250in}{0.638889in}}{\pgfqpoint{4.218750in}{2.172222in}}%
\pgfusepath{clip}%
\pgfsetbuttcap%
\pgfsetmiterjoin%
\definecolor{currentfill}{rgb}{0.235294,0.490196,0.764706}%
\pgfsetfillcolor{currentfill}%
\pgfsetlinewidth{1.003750pt}%
\definecolor{currentstroke}{rgb}{0.266667,0.266667,0.266667}%
\pgfsetstrokecolor{currentstroke}%
\pgfsetdash{}{0pt}%
\pgfpathmoveto{\pgfqpoint{1.331444in}{0.751823in}}%
\pgfpathlineto{\pgfqpoint{1.582347in}{0.751823in}}%
\pgfpathlineto{\pgfqpoint{1.582347in}{1.003735in}}%
\pgfpathlineto{\pgfqpoint{1.331444in}{1.003735in}}%
\pgfpathlineto{\pgfqpoint{1.331444in}{0.751823in}}%
\pgfpathclose%
\pgfusepath{stroke,fill}%
\end{pgfscope}%
\begin{pgfscope}%
\pgfpathrectangle{\pgfqpoint{0.781250in}{0.638889in}}{\pgfqpoint{4.218750in}{2.172222in}}%
\pgfusepath{clip}%
\pgfsetbuttcap%
\pgfsetmiterjoin%
\definecolor{currentfill}{rgb}{0.235294,0.490196,0.764706}%
\pgfsetfillcolor{currentfill}%
\pgfsetlinewidth{1.003750pt}%
\definecolor{currentstroke}{rgb}{0.266667,0.266667,0.266667}%
\pgfsetstrokecolor{currentstroke}%
\pgfsetdash{}{0pt}%
\pgfpathmoveto{\pgfqpoint{1.689876in}{0.746436in}}%
\pgfpathlineto{\pgfqpoint{1.940779in}{0.746436in}}%
\pgfpathlineto{\pgfqpoint{1.940779in}{0.961790in}}%
\pgfpathlineto{\pgfqpoint{1.689876in}{0.961790in}}%
\pgfpathlineto{\pgfqpoint{1.689876in}{0.746436in}}%
\pgfpathclose%
\pgfusepath{stroke,fill}%
\end{pgfscope}%
\begin{pgfscope}%
\pgfpathrectangle{\pgfqpoint{0.781250in}{0.638889in}}{\pgfqpoint{4.218750in}{2.172222in}}%
\pgfusepath{clip}%
\pgfsetbuttcap%
\pgfsetmiterjoin%
\definecolor{currentfill}{rgb}{0.235294,0.490196,0.764706}%
\pgfsetfillcolor{currentfill}%
\pgfsetlinewidth{1.003750pt}%
\definecolor{currentstroke}{rgb}{0.266667,0.266667,0.266667}%
\pgfsetstrokecolor{currentstroke}%
\pgfsetdash{}{0pt}%
\pgfpathmoveto{\pgfqpoint{2.048309in}{0.770938in}}%
\pgfpathlineto{\pgfqpoint{2.299211in}{0.770938in}}%
\pgfpathlineto{\pgfqpoint{2.299211in}{0.984489in}}%
\pgfpathlineto{\pgfqpoint{2.048309in}{0.984489in}}%
\pgfpathlineto{\pgfqpoint{2.048309in}{0.770938in}}%
\pgfpathclose%
\pgfusepath{stroke,fill}%
\end{pgfscope}%
\begin{pgfscope}%
\pgfpathrectangle{\pgfqpoint{0.781250in}{0.638889in}}{\pgfqpoint{4.218750in}{2.172222in}}%
\pgfusepath{clip}%
\pgfsetbuttcap%
\pgfsetmiterjoin%
\definecolor{currentfill}{rgb}{0.235294,0.490196,0.764706}%
\pgfsetfillcolor{currentfill}%
\pgfsetlinewidth{1.003750pt}%
\definecolor{currentstroke}{rgb}{0.266667,0.266667,0.266667}%
\pgfsetstrokecolor{currentstroke}%
\pgfsetdash{}{0pt}%
\pgfpathmoveto{\pgfqpoint{2.406741in}{0.831847in}}%
\pgfpathlineto{\pgfqpoint{2.657644in}{0.831847in}}%
\pgfpathlineto{\pgfqpoint{2.657644in}{1.060843in}}%
\pgfpathlineto{\pgfqpoint{2.406741in}{1.060843in}}%
\pgfpathlineto{\pgfqpoint{2.406741in}{0.831847in}}%
\pgfpathclose%
\pgfusepath{stroke,fill}%
\end{pgfscope}%
\begin{pgfscope}%
\pgfpathrectangle{\pgfqpoint{0.781250in}{0.638889in}}{\pgfqpoint{4.218750in}{2.172222in}}%
\pgfusepath{clip}%
\pgfsetbuttcap%
\pgfsetmiterjoin%
\definecolor{currentfill}{rgb}{0.235294,0.490196,0.764706}%
\pgfsetfillcolor{currentfill}%
\pgfsetlinewidth{1.003750pt}%
\definecolor{currentstroke}{rgb}{0.266667,0.266667,0.266667}%
\pgfsetstrokecolor{currentstroke}%
\pgfsetdash{}{0pt}%
\pgfpathmoveto{\pgfqpoint{2.765174in}{0.909048in}}%
\pgfpathlineto{\pgfqpoint{3.016076in}{0.909048in}}%
\pgfpathlineto{\pgfqpoint{3.016076in}{1.142823in}}%
\pgfpathlineto{\pgfqpoint{2.765174in}{1.142823in}}%
\pgfpathlineto{\pgfqpoint{2.765174in}{0.909048in}}%
\pgfpathclose%
\pgfusepath{stroke,fill}%
\end{pgfscope}%
\begin{pgfscope}%
\pgfpathrectangle{\pgfqpoint{0.781250in}{0.638889in}}{\pgfqpoint{4.218750in}{2.172222in}}%
\pgfusepath{clip}%
\pgfsetbuttcap%
\pgfsetmiterjoin%
\definecolor{currentfill}{rgb}{0.235294,0.490196,0.764706}%
\pgfsetfillcolor{currentfill}%
\pgfsetlinewidth{1.003750pt}%
\definecolor{currentstroke}{rgb}{0.266667,0.266667,0.266667}%
\pgfsetstrokecolor{currentstroke}%
\pgfsetdash{}{0pt}%
\pgfpathmoveto{\pgfqpoint{3.123606in}{0.892474in}}%
\pgfpathlineto{\pgfqpoint{3.374509in}{0.892474in}}%
\pgfpathlineto{\pgfqpoint{3.374509in}{1.112260in}}%
\pgfpathlineto{\pgfqpoint{3.123606in}{1.112260in}}%
\pgfpathlineto{\pgfqpoint{3.123606in}{0.892474in}}%
\pgfpathclose%
\pgfusepath{stroke,fill}%
\end{pgfscope}%
\begin{pgfscope}%
\pgfpathrectangle{\pgfqpoint{0.781250in}{0.638889in}}{\pgfqpoint{4.218750in}{2.172222in}}%
\pgfusepath{clip}%
\pgfsetbuttcap%
\pgfsetmiterjoin%
\definecolor{currentfill}{rgb}{0.235294,0.490196,0.764706}%
\pgfsetfillcolor{currentfill}%
\pgfsetlinewidth{1.003750pt}%
\definecolor{currentstroke}{rgb}{0.266667,0.266667,0.266667}%
\pgfsetstrokecolor{currentstroke}%
\pgfsetdash{}{0pt}%
\pgfpathmoveto{\pgfqpoint{3.482039in}{0.860499in}}%
\pgfpathlineto{\pgfqpoint{3.732941in}{0.860499in}}%
\pgfpathlineto{\pgfqpoint{3.732941in}{1.098423in}}%
\pgfpathlineto{\pgfqpoint{3.482039in}{1.098423in}}%
\pgfpathlineto{\pgfqpoint{3.482039in}{0.860499in}}%
\pgfpathclose%
\pgfusepath{stroke,fill}%
\end{pgfscope}%
\begin{pgfscope}%
\pgfpathrectangle{\pgfqpoint{0.781250in}{0.638889in}}{\pgfqpoint{4.218750in}{2.172222in}}%
\pgfusepath{clip}%
\pgfsetbuttcap%
\pgfsetmiterjoin%
\definecolor{currentfill}{rgb}{0.235294,0.490196,0.764706}%
\pgfsetfillcolor{currentfill}%
\pgfsetlinewidth{1.003750pt}%
\definecolor{currentstroke}{rgb}{0.266667,0.266667,0.266667}%
\pgfsetstrokecolor{currentstroke}%
\pgfsetdash{}{0pt}%
\pgfpathmoveto{\pgfqpoint{3.840471in}{0.931031in}}%
\pgfpathlineto{\pgfqpoint{4.091374in}{0.931031in}}%
\pgfpathlineto{\pgfqpoint{4.091374in}{1.210944in}}%
\pgfpathlineto{\pgfqpoint{3.840471in}{1.210944in}}%
\pgfpathlineto{\pgfqpoint{3.840471in}{0.931031in}}%
\pgfpathclose%
\pgfusepath{stroke,fill}%
\end{pgfscope}%
\begin{pgfscope}%
\pgfpathrectangle{\pgfqpoint{0.781250in}{0.638889in}}{\pgfqpoint{4.218750in}{2.172222in}}%
\pgfusepath{clip}%
\pgfsetbuttcap%
\pgfsetmiterjoin%
\definecolor{currentfill}{rgb}{0.235294,0.490196,0.764706}%
\pgfsetfillcolor{currentfill}%
\pgfsetlinewidth{1.003750pt}%
\definecolor{currentstroke}{rgb}{0.266667,0.266667,0.266667}%
\pgfsetstrokecolor{currentstroke}%
\pgfsetdash{}{0pt}%
\pgfpathmoveto{\pgfqpoint{4.198903in}{1.020874in}}%
\pgfpathlineto{\pgfqpoint{4.449806in}{1.020874in}}%
\pgfpathlineto{\pgfqpoint{4.449806in}{1.282909in}}%
\pgfpathlineto{\pgfqpoint{4.198903in}{1.282909in}}%
\pgfpathlineto{\pgfqpoint{4.198903in}{1.020874in}}%
\pgfpathclose%
\pgfusepath{stroke,fill}%
\end{pgfscope}%
\begin{pgfscope}%
\pgfpathrectangle{\pgfqpoint{0.781250in}{0.638889in}}{\pgfqpoint{4.218750in}{2.172222in}}%
\pgfusepath{clip}%
\pgfsetbuttcap%
\pgfsetmiterjoin%
\definecolor{currentfill}{rgb}{0.235294,0.490196,0.764706}%
\pgfsetfillcolor{currentfill}%
\pgfsetlinewidth{1.003750pt}%
\definecolor{currentstroke}{rgb}{0.266667,0.266667,0.266667}%
\pgfsetstrokecolor{currentstroke}%
\pgfsetdash{}{0pt}%
\pgfpathmoveto{\pgfqpoint{4.557336in}{1.086823in}}%
\pgfpathlineto{\pgfqpoint{4.808239in}{1.086823in}}%
\pgfpathlineto{\pgfqpoint{4.808239in}{1.372622in}}%
\pgfpathlineto{\pgfqpoint{4.557336in}{1.372622in}}%
\pgfpathlineto{\pgfqpoint{4.557336in}{1.086823in}}%
\pgfpathclose%
\pgfusepath{stroke,fill}%
\end{pgfscope}%
\begin{pgfscope}%
\pgfpathrectangle{\pgfqpoint{0.781250in}{0.638889in}}{\pgfqpoint{4.218750in}{2.172222in}}%
\pgfusepath{clip}%
\pgfsetbuttcap%
\pgfsetmiterjoin%
\definecolor{currentfill}{rgb}{0.725490,0.486275,0.164706}%
\pgfsetfillcolor{currentfill}%
\pgfsetlinewidth{1.003750pt}%
\definecolor{currentstroke}{rgb}{0.266667,0.266667,0.266667}%
\pgfsetstrokecolor{currentstroke}%
\pgfsetdash{}{0pt}%
\pgfpathmoveto{\pgfqpoint{0.973011in}{1.126553in}}%
\pgfpathlineto{\pgfqpoint{1.223914in}{1.126553in}}%
\pgfpathlineto{\pgfqpoint{1.223914in}{1.472914in}}%
\pgfpathlineto{\pgfqpoint{0.973011in}{1.472914in}}%
\pgfpathlineto{\pgfqpoint{0.973011in}{1.126553in}}%
\pgfpathclose%
\pgfusepath{stroke,fill}%
\end{pgfscope}%
\begin{pgfscope}%
\pgfpathrectangle{\pgfqpoint{0.781250in}{0.638889in}}{\pgfqpoint{4.218750in}{2.172222in}}%
\pgfusepath{clip}%
\pgfsetbuttcap%
\pgfsetmiterjoin%
\definecolor{currentfill}{rgb}{0.725490,0.486275,0.164706}%
\pgfsetfillcolor{currentfill}%
\pgfsetlinewidth{1.003750pt}%
\definecolor{currentstroke}{rgb}{0.266667,0.266667,0.266667}%
\pgfsetstrokecolor{currentstroke}%
\pgfsetdash{}{0pt}%
\pgfpathmoveto{\pgfqpoint{1.331444in}{1.003735in}}%
\pgfpathlineto{\pgfqpoint{1.582347in}{1.003735in}}%
\pgfpathlineto{\pgfqpoint{1.582347in}{1.294118in}}%
\pgfpathlineto{\pgfqpoint{1.331444in}{1.294118in}}%
\pgfpathlineto{\pgfqpoint{1.331444in}{1.003735in}}%
\pgfpathclose%
\pgfusepath{stroke,fill}%
\end{pgfscope}%
\begin{pgfscope}%
\pgfpathrectangle{\pgfqpoint{0.781250in}{0.638889in}}{\pgfqpoint{4.218750in}{2.172222in}}%
\pgfusepath{clip}%
\pgfsetbuttcap%
\pgfsetmiterjoin%
\definecolor{currentfill}{rgb}{0.725490,0.486275,0.164706}%
\pgfsetfillcolor{currentfill}%
\pgfsetlinewidth{1.003750pt}%
\definecolor{currentstroke}{rgb}{0.266667,0.266667,0.266667}%
\pgfsetstrokecolor{currentstroke}%
\pgfsetdash{}{0pt}%
\pgfpathmoveto{\pgfqpoint{1.689876in}{0.961790in}}%
\pgfpathlineto{\pgfqpoint{1.940779in}{0.961790in}}%
\pgfpathlineto{\pgfqpoint{1.940779in}{1.210270in}}%
\pgfpathlineto{\pgfqpoint{1.689876in}{1.210270in}}%
\pgfpathlineto{\pgfqpoint{1.689876in}{0.961790in}}%
\pgfpathclose%
\pgfusepath{stroke,fill}%
\end{pgfscope}%
\begin{pgfscope}%
\pgfpathrectangle{\pgfqpoint{0.781250in}{0.638889in}}{\pgfqpoint{4.218750in}{2.172222in}}%
\pgfusepath{clip}%
\pgfsetbuttcap%
\pgfsetmiterjoin%
\definecolor{currentfill}{rgb}{0.725490,0.486275,0.164706}%
\pgfsetfillcolor{currentfill}%
\pgfsetlinewidth{1.003750pt}%
\definecolor{currentstroke}{rgb}{0.266667,0.266667,0.266667}%
\pgfsetstrokecolor{currentstroke}%
\pgfsetdash{}{0pt}%
\pgfpathmoveto{\pgfqpoint{2.048309in}{0.984489in}}%
\pgfpathlineto{\pgfqpoint{2.299211in}{0.984489in}}%
\pgfpathlineto{\pgfqpoint{2.299211in}{1.209097in}}%
\pgfpathlineto{\pgfqpoint{2.048309in}{1.209097in}}%
\pgfpathlineto{\pgfqpoint{2.048309in}{0.984489in}}%
\pgfpathclose%
\pgfusepath{stroke,fill}%
\end{pgfscope}%
\begin{pgfscope}%
\pgfpathrectangle{\pgfqpoint{0.781250in}{0.638889in}}{\pgfqpoint{4.218750in}{2.172222in}}%
\pgfusepath{clip}%
\pgfsetbuttcap%
\pgfsetmiterjoin%
\definecolor{currentfill}{rgb}{0.725490,0.486275,0.164706}%
\pgfsetfillcolor{currentfill}%
\pgfsetlinewidth{1.003750pt}%
\definecolor{currentstroke}{rgb}{0.266667,0.266667,0.266667}%
\pgfsetstrokecolor{currentstroke}%
\pgfsetdash{}{0pt}%
\pgfpathmoveto{\pgfqpoint{2.406741in}{1.060843in}}%
\pgfpathlineto{\pgfqpoint{2.657644in}{1.060843in}}%
\pgfpathlineto{\pgfqpoint{2.657644in}{1.308129in}}%
\pgfpathlineto{\pgfqpoint{2.406741in}{1.308129in}}%
\pgfpathlineto{\pgfqpoint{2.406741in}{1.060843in}}%
\pgfpathclose%
\pgfusepath{stroke,fill}%
\end{pgfscope}%
\begin{pgfscope}%
\pgfpathrectangle{\pgfqpoint{0.781250in}{0.638889in}}{\pgfqpoint{4.218750in}{2.172222in}}%
\pgfusepath{clip}%
\pgfsetbuttcap%
\pgfsetmiterjoin%
\definecolor{currentfill}{rgb}{0.725490,0.486275,0.164706}%
\pgfsetfillcolor{currentfill}%
\pgfsetlinewidth{1.003750pt}%
\definecolor{currentstroke}{rgb}{0.266667,0.266667,0.266667}%
\pgfsetstrokecolor{currentstroke}%
\pgfsetdash{}{0pt}%
\pgfpathmoveto{\pgfqpoint{2.765174in}{1.142823in}}%
\pgfpathlineto{\pgfqpoint{3.016076in}{1.142823in}}%
\pgfpathlineto{\pgfqpoint{3.016076in}{1.407334in}}%
\pgfpathlineto{\pgfqpoint{2.765174in}{1.407334in}}%
\pgfpathlineto{\pgfqpoint{2.765174in}{1.142823in}}%
\pgfpathclose%
\pgfusepath{stroke,fill}%
\end{pgfscope}%
\begin{pgfscope}%
\pgfpathrectangle{\pgfqpoint{0.781250in}{0.638889in}}{\pgfqpoint{4.218750in}{2.172222in}}%
\pgfusepath{clip}%
\pgfsetbuttcap%
\pgfsetmiterjoin%
\definecolor{currentfill}{rgb}{0.725490,0.486275,0.164706}%
\pgfsetfillcolor{currentfill}%
\pgfsetlinewidth{1.003750pt}%
\definecolor{currentstroke}{rgb}{0.266667,0.266667,0.266667}%
\pgfsetstrokecolor{currentstroke}%
\pgfsetdash{}{0pt}%
\pgfpathmoveto{\pgfqpoint{3.123606in}{1.112260in}}%
\pgfpathlineto{\pgfqpoint{3.374509in}{1.112260in}}%
\pgfpathlineto{\pgfqpoint{3.374509in}{1.328157in}}%
\pgfpathlineto{\pgfqpoint{3.123606in}{1.328157in}}%
\pgfpathlineto{\pgfqpoint{3.123606in}{1.112260in}}%
\pgfpathclose%
\pgfusepath{stroke,fill}%
\end{pgfscope}%
\begin{pgfscope}%
\pgfpathrectangle{\pgfqpoint{0.781250in}{0.638889in}}{\pgfqpoint{4.218750in}{2.172222in}}%
\pgfusepath{clip}%
\pgfsetbuttcap%
\pgfsetmiterjoin%
\definecolor{currentfill}{rgb}{0.725490,0.486275,0.164706}%
\pgfsetfillcolor{currentfill}%
\pgfsetlinewidth{1.003750pt}%
\definecolor{currentstroke}{rgb}{0.266667,0.266667,0.266667}%
\pgfsetstrokecolor{currentstroke}%
\pgfsetdash{}{0pt}%
\pgfpathmoveto{\pgfqpoint{3.482039in}{1.098423in}}%
\pgfpathlineto{\pgfqpoint{3.732941in}{1.098423in}}%
\pgfpathlineto{\pgfqpoint{3.732941in}{1.302155in}}%
\pgfpathlineto{\pgfqpoint{3.482039in}{1.302155in}}%
\pgfpathlineto{\pgfqpoint{3.482039in}{1.098423in}}%
\pgfpathclose%
\pgfusepath{stroke,fill}%
\end{pgfscope}%
\begin{pgfscope}%
\pgfpathrectangle{\pgfqpoint{0.781250in}{0.638889in}}{\pgfqpoint{4.218750in}{2.172222in}}%
\pgfusepath{clip}%
\pgfsetbuttcap%
\pgfsetmiterjoin%
\definecolor{currentfill}{rgb}{0.725490,0.486275,0.164706}%
\pgfsetfillcolor{currentfill}%
\pgfsetlinewidth{1.003750pt}%
\definecolor{currentstroke}{rgb}{0.266667,0.266667,0.266667}%
\pgfsetstrokecolor{currentstroke}%
\pgfsetdash{}{0pt}%
\pgfpathmoveto{\pgfqpoint{3.840471in}{1.210944in}}%
\pgfpathlineto{\pgfqpoint{4.091374in}{1.210944in}}%
\pgfpathlineto{\pgfqpoint{4.091374in}{1.449280in}}%
\pgfpathlineto{\pgfqpoint{3.840471in}{1.449280in}}%
\pgfpathlineto{\pgfqpoint{3.840471in}{1.210944in}}%
\pgfpathclose%
\pgfusepath{stroke,fill}%
\end{pgfscope}%
\begin{pgfscope}%
\pgfpathrectangle{\pgfqpoint{0.781250in}{0.638889in}}{\pgfqpoint{4.218750in}{2.172222in}}%
\pgfusepath{clip}%
\pgfsetbuttcap%
\pgfsetmiterjoin%
\definecolor{currentfill}{rgb}{0.725490,0.486275,0.164706}%
\pgfsetfillcolor{currentfill}%
\pgfsetlinewidth{1.003750pt}%
\definecolor{currentstroke}{rgb}{0.266667,0.266667,0.266667}%
\pgfsetstrokecolor{currentstroke}%
\pgfsetdash{}{0pt}%
\pgfpathmoveto{\pgfqpoint{4.198903in}{1.282909in}}%
\pgfpathlineto{\pgfqpoint{4.449806in}{1.282909in}}%
\pgfpathlineto{\pgfqpoint{4.449806in}{1.514055in}}%
\pgfpathlineto{\pgfqpoint{4.198903in}{1.514055in}}%
\pgfpathlineto{\pgfqpoint{4.198903in}{1.282909in}}%
\pgfpathclose%
\pgfusepath{stroke,fill}%
\end{pgfscope}%
\begin{pgfscope}%
\pgfpathrectangle{\pgfqpoint{0.781250in}{0.638889in}}{\pgfqpoint{4.218750in}{2.172222in}}%
\pgfusepath{clip}%
\pgfsetbuttcap%
\pgfsetmiterjoin%
\definecolor{currentfill}{rgb}{0.725490,0.486275,0.164706}%
\pgfsetfillcolor{currentfill}%
\pgfsetlinewidth{1.003750pt}%
\definecolor{currentstroke}{rgb}{0.266667,0.266667,0.266667}%
\pgfsetstrokecolor{currentstroke}%
\pgfsetdash{}{0pt}%
\pgfpathmoveto{\pgfqpoint{4.557336in}{1.372622in}}%
\pgfpathlineto{\pgfqpoint{4.808239in}{1.372622in}}%
\pgfpathlineto{\pgfqpoint{4.808239in}{1.638893in}}%
\pgfpathlineto{\pgfqpoint{4.557336in}{1.638893in}}%
\pgfpathlineto{\pgfqpoint{4.557336in}{1.372622in}}%
\pgfpathclose%
\pgfusepath{stroke,fill}%
\end{pgfscope}%
\begin{pgfscope}%
\pgfpathrectangle{\pgfqpoint{0.781250in}{0.638889in}}{\pgfqpoint{4.218750in}{2.172222in}}%
\pgfusepath{clip}%
\pgfsetbuttcap%
\pgfsetmiterjoin%
\definecolor{currentfill}{rgb}{0.733333,0.321569,0.733333}%
\pgfsetfillcolor{currentfill}%
\pgfsetlinewidth{1.003750pt}%
\definecolor{currentstroke}{rgb}{0.266667,0.266667,0.266667}%
\pgfsetstrokecolor{currentstroke}%
\pgfsetdash{}{0pt}%
\pgfpathmoveto{\pgfqpoint{0.973011in}{1.472914in}}%
\pgfpathlineto{\pgfqpoint{1.223914in}{1.472914in}}%
\pgfpathlineto{\pgfqpoint{1.223914in}{1.651535in}}%
\pgfpathlineto{\pgfqpoint{0.973011in}{1.651535in}}%
\pgfpathlineto{\pgfqpoint{0.973011in}{1.472914in}}%
\pgfpathclose%
\pgfusepath{stroke,fill}%
\end{pgfscope}%
\begin{pgfscope}%
\pgfpathrectangle{\pgfqpoint{0.781250in}{0.638889in}}{\pgfqpoint{4.218750in}{2.172222in}}%
\pgfusepath{clip}%
\pgfsetbuttcap%
\pgfsetmiterjoin%
\definecolor{currentfill}{rgb}{0.733333,0.321569,0.733333}%
\pgfsetfillcolor{currentfill}%
\pgfsetlinewidth{1.003750pt}%
\definecolor{currentstroke}{rgb}{0.266667,0.266667,0.266667}%
\pgfsetstrokecolor{currentstroke}%
\pgfsetdash{}{0pt}%
\pgfpathmoveto{\pgfqpoint{1.331444in}{1.294118in}}%
\pgfpathlineto{\pgfqpoint{1.582347in}{1.294118in}}%
\pgfpathlineto{\pgfqpoint{1.582347in}{1.432575in}}%
\pgfpathlineto{\pgfqpoint{1.331444in}{1.432575in}}%
\pgfpathlineto{\pgfqpoint{1.331444in}{1.294118in}}%
\pgfpathclose%
\pgfusepath{stroke,fill}%
\end{pgfscope}%
\begin{pgfscope}%
\pgfpathrectangle{\pgfqpoint{0.781250in}{0.638889in}}{\pgfqpoint{4.218750in}{2.172222in}}%
\pgfusepath{clip}%
\pgfsetbuttcap%
\pgfsetmiterjoin%
\definecolor{currentfill}{rgb}{0.733333,0.321569,0.733333}%
\pgfsetfillcolor{currentfill}%
\pgfsetlinewidth{1.003750pt}%
\definecolor{currentstroke}{rgb}{0.266667,0.266667,0.266667}%
\pgfsetstrokecolor{currentstroke}%
\pgfsetdash{}{0pt}%
\pgfpathmoveto{\pgfqpoint{1.689876in}{1.210270in}}%
\pgfpathlineto{\pgfqpoint{1.940779in}{1.210270in}}%
\pgfpathlineto{\pgfqpoint{1.940779in}{1.324486in}}%
\pgfpathlineto{\pgfqpoint{1.689876in}{1.324486in}}%
\pgfpathlineto{\pgfqpoint{1.689876in}{1.210270in}}%
\pgfpathclose%
\pgfusepath{stroke,fill}%
\end{pgfscope}%
\begin{pgfscope}%
\pgfpathrectangle{\pgfqpoint{0.781250in}{0.638889in}}{\pgfqpoint{4.218750in}{2.172222in}}%
\pgfusepath{clip}%
\pgfsetbuttcap%
\pgfsetmiterjoin%
\definecolor{currentfill}{rgb}{0.733333,0.321569,0.733333}%
\pgfsetfillcolor{currentfill}%
\pgfsetlinewidth{1.003750pt}%
\definecolor{currentstroke}{rgb}{0.266667,0.266667,0.266667}%
\pgfsetstrokecolor{currentstroke}%
\pgfsetdash{}{0pt}%
\pgfpathmoveto{\pgfqpoint{2.048309in}{1.209097in}}%
\pgfpathlineto{\pgfqpoint{2.299211in}{1.209097in}}%
\pgfpathlineto{\pgfqpoint{2.299211in}{1.324942in}}%
\pgfpathlineto{\pgfqpoint{2.048309in}{1.324942in}}%
\pgfpathlineto{\pgfqpoint{2.048309in}{1.209097in}}%
\pgfpathclose%
\pgfusepath{stroke,fill}%
\end{pgfscope}%
\begin{pgfscope}%
\pgfpathrectangle{\pgfqpoint{0.781250in}{0.638889in}}{\pgfqpoint{4.218750in}{2.172222in}}%
\pgfusepath{clip}%
\pgfsetbuttcap%
\pgfsetmiterjoin%
\definecolor{currentfill}{rgb}{0.733333,0.321569,0.733333}%
\pgfsetfillcolor{currentfill}%
\pgfsetlinewidth{1.003750pt}%
\definecolor{currentstroke}{rgb}{0.266667,0.266667,0.266667}%
\pgfsetstrokecolor{currentstroke}%
\pgfsetdash{}{0pt}%
\pgfpathmoveto{\pgfqpoint{2.406741in}{1.308129in}}%
\pgfpathlineto{\pgfqpoint{2.657644in}{1.308129in}}%
\pgfpathlineto{\pgfqpoint{2.657644in}{1.437333in}}%
\pgfpathlineto{\pgfqpoint{2.406741in}{1.437333in}}%
\pgfpathlineto{\pgfqpoint{2.406741in}{1.308129in}}%
\pgfpathclose%
\pgfusepath{stroke,fill}%
\end{pgfscope}%
\begin{pgfscope}%
\pgfpathrectangle{\pgfqpoint{0.781250in}{0.638889in}}{\pgfqpoint{4.218750in}{2.172222in}}%
\pgfusepath{clip}%
\pgfsetbuttcap%
\pgfsetmiterjoin%
\definecolor{currentfill}{rgb}{0.733333,0.321569,0.733333}%
\pgfsetfillcolor{currentfill}%
\pgfsetlinewidth{1.003750pt}%
\definecolor{currentstroke}{rgb}{0.266667,0.266667,0.266667}%
\pgfsetstrokecolor{currentstroke}%
\pgfsetdash{}{0pt}%
\pgfpathmoveto{\pgfqpoint{2.765174in}{1.407334in}}%
\pgfpathlineto{\pgfqpoint{3.016076in}{1.407334in}}%
\pgfpathlineto{\pgfqpoint{3.016076in}{1.565190in}}%
\pgfpathlineto{\pgfqpoint{2.765174in}{1.565190in}}%
\pgfpathlineto{\pgfqpoint{2.765174in}{1.407334in}}%
\pgfpathclose%
\pgfusepath{stroke,fill}%
\end{pgfscope}%
\begin{pgfscope}%
\pgfpathrectangle{\pgfqpoint{0.781250in}{0.638889in}}{\pgfqpoint{4.218750in}{2.172222in}}%
\pgfusepath{clip}%
\pgfsetbuttcap%
\pgfsetmiterjoin%
\definecolor{currentfill}{rgb}{0.733333,0.321569,0.733333}%
\pgfsetfillcolor{currentfill}%
\pgfsetlinewidth{1.003750pt}%
\definecolor{currentstroke}{rgb}{0.266667,0.266667,0.266667}%
\pgfsetstrokecolor{currentstroke}%
\pgfsetdash{}{0pt}%
\pgfpathmoveto{\pgfqpoint{3.123606in}{1.328157in}}%
\pgfpathlineto{\pgfqpoint{3.374509in}{1.328157in}}%
\pgfpathlineto{\pgfqpoint{3.374509in}{1.477345in}}%
\pgfpathlineto{\pgfqpoint{3.123606in}{1.477345in}}%
\pgfpathlineto{\pgfqpoint{3.123606in}{1.328157in}}%
\pgfpathclose%
\pgfusepath{stroke,fill}%
\end{pgfscope}%
\begin{pgfscope}%
\pgfpathrectangle{\pgfqpoint{0.781250in}{0.638889in}}{\pgfqpoint{4.218750in}{2.172222in}}%
\pgfusepath{clip}%
\pgfsetbuttcap%
\pgfsetmiterjoin%
\definecolor{currentfill}{rgb}{0.733333,0.321569,0.733333}%
\pgfsetfillcolor{currentfill}%
\pgfsetlinewidth{1.003750pt}%
\definecolor{currentstroke}{rgb}{0.266667,0.266667,0.266667}%
\pgfsetstrokecolor{currentstroke}%
\pgfsetdash{}{0pt}%
\pgfpathmoveto{\pgfqpoint{3.482039in}{1.302155in}}%
\pgfpathlineto{\pgfqpoint{3.732941in}{1.302155in}}%
\pgfpathlineto{\pgfqpoint{3.732941in}{1.468634in}}%
\pgfpathlineto{\pgfqpoint{3.482039in}{1.468634in}}%
\pgfpathlineto{\pgfqpoint{3.482039in}{1.302155in}}%
\pgfpathclose%
\pgfusepath{stroke,fill}%
\end{pgfscope}%
\begin{pgfscope}%
\pgfpathrectangle{\pgfqpoint{0.781250in}{0.638889in}}{\pgfqpoint{4.218750in}{2.172222in}}%
\pgfusepath{clip}%
\pgfsetbuttcap%
\pgfsetmiterjoin%
\definecolor{currentfill}{rgb}{0.733333,0.321569,0.733333}%
\pgfsetfillcolor{currentfill}%
\pgfsetlinewidth{1.003750pt}%
\definecolor{currentstroke}{rgb}{0.266667,0.266667,0.266667}%
\pgfsetstrokecolor{currentstroke}%
\pgfsetdash{}{0pt}%
\pgfpathmoveto{\pgfqpoint{3.840471in}{1.449280in}}%
\pgfpathlineto{\pgfqpoint{4.091374in}{1.449280in}}%
\pgfpathlineto{\pgfqpoint{4.091374in}{1.643585in}}%
\pgfpathlineto{\pgfqpoint{3.840471in}{1.643585in}}%
\pgfpathlineto{\pgfqpoint{3.840471in}{1.449280in}}%
\pgfpathclose%
\pgfusepath{stroke,fill}%
\end{pgfscope}%
\begin{pgfscope}%
\pgfpathrectangle{\pgfqpoint{0.781250in}{0.638889in}}{\pgfqpoint{4.218750in}{2.172222in}}%
\pgfusepath{clip}%
\pgfsetbuttcap%
\pgfsetmiterjoin%
\definecolor{currentfill}{rgb}{0.733333,0.321569,0.733333}%
\pgfsetfillcolor{currentfill}%
\pgfsetlinewidth{1.003750pt}%
\definecolor{currentstroke}{rgb}{0.266667,0.266667,0.266667}%
\pgfsetstrokecolor{currentstroke}%
\pgfsetdash{}{0pt}%
\pgfpathmoveto{\pgfqpoint{4.198903in}{1.514055in}}%
\pgfpathlineto{\pgfqpoint{4.449806in}{1.514055in}}%
\pgfpathlineto{\pgfqpoint{4.449806in}{1.701627in}}%
\pgfpathlineto{\pgfqpoint{4.198903in}{1.701627in}}%
\pgfpathlineto{\pgfqpoint{4.198903in}{1.514055in}}%
\pgfpathclose%
\pgfusepath{stroke,fill}%
\end{pgfscope}%
\begin{pgfscope}%
\pgfpathrectangle{\pgfqpoint{0.781250in}{0.638889in}}{\pgfqpoint{4.218750in}{2.172222in}}%
\pgfusepath{clip}%
\pgfsetbuttcap%
\pgfsetmiterjoin%
\definecolor{currentfill}{rgb}{0.733333,0.321569,0.733333}%
\pgfsetfillcolor{currentfill}%
\pgfsetlinewidth{1.003750pt}%
\definecolor{currentstroke}{rgb}{0.266667,0.266667,0.266667}%
\pgfsetstrokecolor{currentstroke}%
\pgfsetdash{}{0pt}%
\pgfpathmoveto{\pgfqpoint{4.557336in}{1.638893in}}%
\pgfpathlineto{\pgfqpoint{4.808239in}{1.638893in}}%
\pgfpathlineto{\pgfqpoint{4.808239in}{1.844994in}}%
\pgfpathlineto{\pgfqpoint{4.557336in}{1.844994in}}%
\pgfpathlineto{\pgfqpoint{4.557336in}{1.638893in}}%
\pgfpathclose%
\pgfusepath{stroke,fill}%
\end{pgfscope}%
\begin{pgfscope}%
\pgfpathrectangle{\pgfqpoint{0.781250in}{0.638889in}}{\pgfqpoint{4.218750in}{2.172222in}}%
\pgfusepath{clip}%
\pgfsetbuttcap%
\pgfsetmiterjoin%
\definecolor{currentfill}{rgb}{0.549020,0.247059,0.121569}%
\pgfsetfillcolor{currentfill}%
\pgfsetlinewidth{1.003750pt}%
\definecolor{currentstroke}{rgb}{0.266667,0.266667,0.266667}%
\pgfsetstrokecolor{currentstroke}%
\pgfsetdash{}{0pt}%
\pgfpathmoveto{\pgfqpoint{0.973011in}{1.651535in}}%
\pgfpathlineto{\pgfqpoint{1.223914in}{1.651535in}}%
\pgfpathlineto{\pgfqpoint{1.223914in}{1.904252in}}%
\pgfpathlineto{\pgfqpoint{0.973011in}{1.904252in}}%
\pgfpathlineto{\pgfqpoint{0.973011in}{1.651535in}}%
\pgfpathclose%
\pgfusepath{stroke,fill}%
\end{pgfscope}%
\begin{pgfscope}%
\pgfpathrectangle{\pgfqpoint{0.781250in}{0.638889in}}{\pgfqpoint{4.218750in}{2.172222in}}%
\pgfusepath{clip}%
\pgfsetbuttcap%
\pgfsetmiterjoin%
\definecolor{currentfill}{rgb}{0.549020,0.247059,0.121569}%
\pgfsetfillcolor{currentfill}%
\pgfsetlinewidth{1.003750pt}%
\definecolor{currentstroke}{rgb}{0.266667,0.266667,0.266667}%
\pgfsetstrokecolor{currentstroke}%
\pgfsetdash{}{0pt}%
\pgfpathmoveto{\pgfqpoint{1.331444in}{1.432575in}}%
\pgfpathlineto{\pgfqpoint{1.582347in}{1.432575in}}%
\pgfpathlineto{\pgfqpoint{1.582347in}{1.623970in}}%
\pgfpathlineto{\pgfqpoint{1.331444in}{1.623970in}}%
\pgfpathlineto{\pgfqpoint{1.331444in}{1.432575in}}%
\pgfpathclose%
\pgfusepath{stroke,fill}%
\end{pgfscope}%
\begin{pgfscope}%
\pgfpathrectangle{\pgfqpoint{0.781250in}{0.638889in}}{\pgfqpoint{4.218750in}{2.172222in}}%
\pgfusepath{clip}%
\pgfsetbuttcap%
\pgfsetmiterjoin%
\definecolor{currentfill}{rgb}{0.549020,0.247059,0.121569}%
\pgfsetfillcolor{currentfill}%
\pgfsetlinewidth{1.003750pt}%
\definecolor{currentstroke}{rgb}{0.266667,0.266667,0.266667}%
\pgfsetstrokecolor{currentstroke}%
\pgfsetdash{}{0pt}%
\pgfpathmoveto{\pgfqpoint{1.689876in}{1.324486in}}%
\pgfpathlineto{\pgfqpoint{1.940779in}{1.324486in}}%
\pgfpathlineto{\pgfqpoint{1.940779in}{1.476433in}}%
\pgfpathlineto{\pgfqpoint{1.689876in}{1.476433in}}%
\pgfpathlineto{\pgfqpoint{1.689876in}{1.324486in}}%
\pgfpathclose%
\pgfusepath{stroke,fill}%
\end{pgfscope}%
\begin{pgfscope}%
\pgfpathrectangle{\pgfqpoint{0.781250in}{0.638889in}}{\pgfqpoint{4.218750in}{2.172222in}}%
\pgfusepath{clip}%
\pgfsetbuttcap%
\pgfsetmiterjoin%
\definecolor{currentfill}{rgb}{0.549020,0.247059,0.121569}%
\pgfsetfillcolor{currentfill}%
\pgfsetlinewidth{1.003750pt}%
\definecolor{currentstroke}{rgb}{0.266667,0.266667,0.266667}%
\pgfsetstrokecolor{currentstroke}%
\pgfsetdash{}{0pt}%
\pgfpathmoveto{\pgfqpoint{2.048309in}{1.324942in}}%
\pgfpathlineto{\pgfqpoint{2.299211in}{1.324942in}}%
\pgfpathlineto{\pgfqpoint{2.299211in}{1.468482in}}%
\pgfpathlineto{\pgfqpoint{2.048309in}{1.468482in}}%
\pgfpathlineto{\pgfqpoint{2.048309in}{1.324942in}}%
\pgfpathclose%
\pgfusepath{stroke,fill}%
\end{pgfscope}%
\begin{pgfscope}%
\pgfpathrectangle{\pgfqpoint{0.781250in}{0.638889in}}{\pgfqpoint{4.218750in}{2.172222in}}%
\pgfusepath{clip}%
\pgfsetbuttcap%
\pgfsetmiterjoin%
\definecolor{currentfill}{rgb}{0.549020,0.247059,0.121569}%
\pgfsetfillcolor{currentfill}%
\pgfsetlinewidth{1.003750pt}%
\definecolor{currentstroke}{rgb}{0.266667,0.266667,0.266667}%
\pgfsetstrokecolor{currentstroke}%
\pgfsetdash{}{0pt}%
\pgfpathmoveto{\pgfqpoint{2.406741in}{1.437333in}}%
\pgfpathlineto{\pgfqpoint{2.657644in}{1.437333in}}%
\pgfpathlineto{\pgfqpoint{2.657644in}{1.578744in}}%
\pgfpathlineto{\pgfqpoint{2.406741in}{1.578744in}}%
\pgfpathlineto{\pgfqpoint{2.406741in}{1.437333in}}%
\pgfpathclose%
\pgfusepath{stroke,fill}%
\end{pgfscope}%
\begin{pgfscope}%
\pgfpathrectangle{\pgfqpoint{0.781250in}{0.638889in}}{\pgfqpoint{4.218750in}{2.172222in}}%
\pgfusepath{clip}%
\pgfsetbuttcap%
\pgfsetmiterjoin%
\definecolor{currentfill}{rgb}{0.549020,0.247059,0.121569}%
\pgfsetfillcolor{currentfill}%
\pgfsetlinewidth{1.003750pt}%
\definecolor{currentstroke}{rgb}{0.266667,0.266667,0.266667}%
\pgfsetstrokecolor{currentstroke}%
\pgfsetdash{}{0pt}%
\pgfpathmoveto{\pgfqpoint{2.765174in}{1.565190in}}%
\pgfpathlineto{\pgfqpoint{3.016076in}{1.565190in}}%
\pgfpathlineto{\pgfqpoint{3.016076in}{1.725348in}}%
\pgfpathlineto{\pgfqpoint{2.765174in}{1.725348in}}%
\pgfpathlineto{\pgfqpoint{2.765174in}{1.565190in}}%
\pgfpathclose%
\pgfusepath{stroke,fill}%
\end{pgfscope}%
\begin{pgfscope}%
\pgfpathrectangle{\pgfqpoint{0.781250in}{0.638889in}}{\pgfqpoint{4.218750in}{2.172222in}}%
\pgfusepath{clip}%
\pgfsetbuttcap%
\pgfsetmiterjoin%
\definecolor{currentfill}{rgb}{0.549020,0.247059,0.121569}%
\pgfsetfillcolor{currentfill}%
\pgfsetlinewidth{1.003750pt}%
\definecolor{currentstroke}{rgb}{0.266667,0.266667,0.266667}%
\pgfsetstrokecolor{currentstroke}%
\pgfsetdash{}{0pt}%
\pgfpathmoveto{\pgfqpoint{3.123606in}{1.477345in}}%
\pgfpathlineto{\pgfqpoint{3.374509in}{1.477345in}}%
\pgfpathlineto{\pgfqpoint{3.374509in}{1.633354in}}%
\pgfpathlineto{\pgfqpoint{3.123606in}{1.633354in}}%
\pgfpathlineto{\pgfqpoint{3.123606in}{1.477345in}}%
\pgfpathclose%
\pgfusepath{stroke,fill}%
\end{pgfscope}%
\begin{pgfscope}%
\pgfpathrectangle{\pgfqpoint{0.781250in}{0.638889in}}{\pgfqpoint{4.218750in}{2.172222in}}%
\pgfusepath{clip}%
\pgfsetbuttcap%
\pgfsetmiterjoin%
\definecolor{currentfill}{rgb}{0.549020,0.247059,0.121569}%
\pgfsetfillcolor{currentfill}%
\pgfsetlinewidth{1.003750pt}%
\definecolor{currentstroke}{rgb}{0.266667,0.266667,0.266667}%
\pgfsetstrokecolor{currentstroke}%
\pgfsetdash{}{0pt}%
\pgfpathmoveto{\pgfqpoint{3.482039in}{1.468634in}}%
\pgfpathlineto{\pgfqpoint{3.732941in}{1.468634in}}%
\pgfpathlineto{\pgfqpoint{3.732941in}{1.602139in}}%
\pgfpathlineto{\pgfqpoint{3.482039in}{1.602139in}}%
\pgfpathlineto{\pgfqpoint{3.482039in}{1.468634in}}%
\pgfpathclose%
\pgfusepath{stroke,fill}%
\end{pgfscope}%
\begin{pgfscope}%
\pgfpathrectangle{\pgfqpoint{0.781250in}{0.638889in}}{\pgfqpoint{4.218750in}{2.172222in}}%
\pgfusepath{clip}%
\pgfsetbuttcap%
\pgfsetmiterjoin%
\definecolor{currentfill}{rgb}{0.549020,0.247059,0.121569}%
\pgfsetfillcolor{currentfill}%
\pgfsetlinewidth{1.003750pt}%
\definecolor{currentstroke}{rgb}{0.266667,0.266667,0.266667}%
\pgfsetstrokecolor{currentstroke}%
\pgfsetdash{}{0pt}%
\pgfpathmoveto{\pgfqpoint{3.840471in}{1.643585in}}%
\pgfpathlineto{\pgfqpoint{4.091374in}{1.643585in}}%
\pgfpathlineto{\pgfqpoint{4.091374in}{1.805198in}}%
\pgfpathlineto{\pgfqpoint{3.840471in}{1.805198in}}%
\pgfpathlineto{\pgfqpoint{3.840471in}{1.643585in}}%
\pgfpathclose%
\pgfusepath{stroke,fill}%
\end{pgfscope}%
\begin{pgfscope}%
\pgfpathrectangle{\pgfqpoint{0.781250in}{0.638889in}}{\pgfqpoint{4.218750in}{2.172222in}}%
\pgfusepath{clip}%
\pgfsetbuttcap%
\pgfsetmiterjoin%
\definecolor{currentfill}{rgb}{0.549020,0.247059,0.121569}%
\pgfsetfillcolor{currentfill}%
\pgfsetlinewidth{1.003750pt}%
\definecolor{currentstroke}{rgb}{0.266667,0.266667,0.266667}%
\pgfsetstrokecolor{currentstroke}%
\pgfsetdash{}{0pt}%
\pgfpathmoveto{\pgfqpoint{4.198903in}{1.701627in}}%
\pgfpathlineto{\pgfqpoint{4.449806in}{1.701627in}}%
\pgfpathlineto{\pgfqpoint{4.449806in}{1.857028in}}%
\pgfpathlineto{\pgfqpoint{4.198903in}{1.857028in}}%
\pgfpathlineto{\pgfqpoint{4.198903in}{1.701627in}}%
\pgfpathclose%
\pgfusepath{stroke,fill}%
\end{pgfscope}%
\begin{pgfscope}%
\pgfpathrectangle{\pgfqpoint{0.781250in}{0.638889in}}{\pgfqpoint{4.218750in}{2.172222in}}%
\pgfusepath{clip}%
\pgfsetbuttcap%
\pgfsetmiterjoin%
\definecolor{currentfill}{rgb}{0.549020,0.247059,0.121569}%
\pgfsetfillcolor{currentfill}%
\pgfsetlinewidth{1.003750pt}%
\definecolor{currentstroke}{rgb}{0.266667,0.266667,0.266667}%
\pgfsetstrokecolor{currentstroke}%
\pgfsetdash{}{0pt}%
\pgfpathmoveto{\pgfqpoint{4.557336in}{1.844994in}}%
\pgfpathlineto{\pgfqpoint{4.808239in}{1.844994in}}%
\pgfpathlineto{\pgfqpoint{4.808239in}{2.012428in}}%
\pgfpathlineto{\pgfqpoint{4.557336in}{2.012428in}}%
\pgfpathlineto{\pgfqpoint{4.557336in}{1.844994in}}%
\pgfpathclose%
\pgfusepath{stroke,fill}%
\end{pgfscope}%
\begin{pgfscope}%
\pgfpathrectangle{\pgfqpoint{0.781250in}{0.638889in}}{\pgfqpoint{4.218750in}{2.172222in}}%
\pgfusepath{clip}%
\pgfsetbuttcap%
\pgfsetmiterjoin%
\definecolor{currentfill}{rgb}{0.701961,0.760784,0.360784}%
\pgfsetfillcolor{currentfill}%
\pgfsetlinewidth{1.003750pt}%
\definecolor{currentstroke}{rgb}{0.266667,0.266667,0.266667}%
\pgfsetstrokecolor{currentstroke}%
\pgfsetdash{}{0pt}%
\pgfpathmoveto{\pgfqpoint{0.973011in}{1.904252in}}%
\pgfpathlineto{\pgfqpoint{1.223914in}{1.904252in}}%
\pgfpathlineto{\pgfqpoint{1.223914in}{2.077095in}}%
\pgfpathlineto{\pgfqpoint{0.973011in}{2.077095in}}%
\pgfpathlineto{\pgfqpoint{0.973011in}{1.904252in}}%
\pgfpathclose%
\pgfusepath{stroke,fill}%
\end{pgfscope}%
\begin{pgfscope}%
\pgfpathrectangle{\pgfqpoint{0.781250in}{0.638889in}}{\pgfqpoint{4.218750in}{2.172222in}}%
\pgfusepath{clip}%
\pgfsetbuttcap%
\pgfsetmiterjoin%
\definecolor{currentfill}{rgb}{0.701961,0.760784,0.360784}%
\pgfsetfillcolor{currentfill}%
\pgfsetlinewidth{1.003750pt}%
\definecolor{currentstroke}{rgb}{0.266667,0.266667,0.266667}%
\pgfsetstrokecolor{currentstroke}%
\pgfsetdash{}{0pt}%
\pgfpathmoveto{\pgfqpoint{1.331444in}{1.623970in}}%
\pgfpathlineto{\pgfqpoint{1.582347in}{1.623970in}}%
\pgfpathlineto{\pgfqpoint{1.582347in}{1.761819in}}%
\pgfpathlineto{\pgfqpoint{1.331444in}{1.761819in}}%
\pgfpathlineto{\pgfqpoint{1.331444in}{1.623970in}}%
\pgfpathclose%
\pgfusepath{stroke,fill}%
\end{pgfscope}%
\begin{pgfscope}%
\pgfpathrectangle{\pgfqpoint{0.781250in}{0.638889in}}{\pgfqpoint{4.218750in}{2.172222in}}%
\pgfusepath{clip}%
\pgfsetbuttcap%
\pgfsetmiterjoin%
\definecolor{currentfill}{rgb}{0.701961,0.760784,0.360784}%
\pgfsetfillcolor{currentfill}%
\pgfsetlinewidth{1.003750pt}%
\definecolor{currentstroke}{rgb}{0.266667,0.266667,0.266667}%
\pgfsetstrokecolor{currentstroke}%
\pgfsetdash{}{0pt}%
\pgfpathmoveto{\pgfqpoint{1.689876in}{1.476433in}}%
\pgfpathlineto{\pgfqpoint{1.940779in}{1.476433in}}%
\pgfpathlineto{\pgfqpoint{1.940779in}{1.596709in}}%
\pgfpathlineto{\pgfqpoint{1.689876in}{1.596709in}}%
\pgfpathlineto{\pgfqpoint{1.689876in}{1.476433in}}%
\pgfpathclose%
\pgfusepath{stroke,fill}%
\end{pgfscope}%
\begin{pgfscope}%
\pgfpathrectangle{\pgfqpoint{0.781250in}{0.638889in}}{\pgfqpoint{4.218750in}{2.172222in}}%
\pgfusepath{clip}%
\pgfsetbuttcap%
\pgfsetmiterjoin%
\definecolor{currentfill}{rgb}{0.701961,0.760784,0.360784}%
\pgfsetfillcolor{currentfill}%
\pgfsetlinewidth{1.003750pt}%
\definecolor{currentstroke}{rgb}{0.266667,0.266667,0.266667}%
\pgfsetstrokecolor{currentstroke}%
\pgfsetdash{}{0pt}%
\pgfpathmoveto{\pgfqpoint{2.048309in}{1.468482in}}%
\pgfpathlineto{\pgfqpoint{2.299211in}{1.468482in}}%
\pgfpathlineto{\pgfqpoint{2.299211in}{1.585174in}}%
\pgfpathlineto{\pgfqpoint{2.048309in}{1.585174in}}%
\pgfpathlineto{\pgfqpoint{2.048309in}{1.468482in}}%
\pgfpathclose%
\pgfusepath{stroke,fill}%
\end{pgfscope}%
\begin{pgfscope}%
\pgfpathrectangle{\pgfqpoint{0.781250in}{0.638889in}}{\pgfqpoint{4.218750in}{2.172222in}}%
\pgfusepath{clip}%
\pgfsetbuttcap%
\pgfsetmiterjoin%
\definecolor{currentfill}{rgb}{0.701961,0.760784,0.360784}%
\pgfsetfillcolor{currentfill}%
\pgfsetlinewidth{1.003750pt}%
\definecolor{currentstroke}{rgb}{0.266667,0.266667,0.266667}%
\pgfsetstrokecolor{currentstroke}%
\pgfsetdash{}{0pt}%
\pgfpathmoveto{\pgfqpoint{2.406741in}{1.578744in}}%
\pgfpathlineto{\pgfqpoint{2.657644in}{1.578744in}}%
\pgfpathlineto{\pgfqpoint{2.657644in}{1.694545in}}%
\pgfpathlineto{\pgfqpoint{2.406741in}{1.694545in}}%
\pgfpathlineto{\pgfqpoint{2.406741in}{1.578744in}}%
\pgfpathclose%
\pgfusepath{stroke,fill}%
\end{pgfscope}%
\begin{pgfscope}%
\pgfpathrectangle{\pgfqpoint{0.781250in}{0.638889in}}{\pgfqpoint{4.218750in}{2.172222in}}%
\pgfusepath{clip}%
\pgfsetbuttcap%
\pgfsetmiterjoin%
\definecolor{currentfill}{rgb}{0.701961,0.760784,0.360784}%
\pgfsetfillcolor{currentfill}%
\pgfsetlinewidth{1.003750pt}%
\definecolor{currentstroke}{rgb}{0.266667,0.266667,0.266667}%
\pgfsetstrokecolor{currentstroke}%
\pgfsetdash{}{0pt}%
\pgfpathmoveto{\pgfqpoint{2.765174in}{1.725348in}}%
\pgfpathlineto{\pgfqpoint{3.016076in}{1.725348in}}%
\pgfpathlineto{\pgfqpoint{3.016076in}{1.862958in}}%
\pgfpathlineto{\pgfqpoint{2.765174in}{1.862958in}}%
\pgfpathlineto{\pgfqpoint{2.765174in}{1.725348in}}%
\pgfpathclose%
\pgfusepath{stroke,fill}%
\end{pgfscope}%
\begin{pgfscope}%
\pgfpathrectangle{\pgfqpoint{0.781250in}{0.638889in}}{\pgfqpoint{4.218750in}{2.172222in}}%
\pgfusepath{clip}%
\pgfsetbuttcap%
\pgfsetmiterjoin%
\definecolor{currentfill}{rgb}{0.701961,0.760784,0.360784}%
\pgfsetfillcolor{currentfill}%
\pgfsetlinewidth{1.003750pt}%
\definecolor{currentstroke}{rgb}{0.266667,0.266667,0.266667}%
\pgfsetstrokecolor{currentstroke}%
\pgfsetdash{}{0pt}%
\pgfpathmoveto{\pgfqpoint{3.123606in}{1.633354in}}%
\pgfpathlineto{\pgfqpoint{3.374509in}{1.633354in}}%
\pgfpathlineto{\pgfqpoint{3.374509in}{1.767923in}}%
\pgfpathlineto{\pgfqpoint{3.123606in}{1.767923in}}%
\pgfpathlineto{\pgfqpoint{3.123606in}{1.633354in}}%
\pgfpathclose%
\pgfusepath{stroke,fill}%
\end{pgfscope}%
\begin{pgfscope}%
\pgfpathrectangle{\pgfqpoint{0.781250in}{0.638889in}}{\pgfqpoint{4.218750in}{2.172222in}}%
\pgfusepath{clip}%
\pgfsetbuttcap%
\pgfsetmiterjoin%
\definecolor{currentfill}{rgb}{0.701961,0.760784,0.360784}%
\pgfsetfillcolor{currentfill}%
\pgfsetlinewidth{1.003750pt}%
\definecolor{currentstroke}{rgb}{0.266667,0.266667,0.266667}%
\pgfsetstrokecolor{currentstroke}%
\pgfsetdash{}{0pt}%
\pgfpathmoveto{\pgfqpoint{3.482039in}{1.602139in}}%
\pgfpathlineto{\pgfqpoint{3.732941in}{1.602139in}}%
\pgfpathlineto{\pgfqpoint{3.732941in}{1.736274in}}%
\pgfpathlineto{\pgfqpoint{3.482039in}{1.736274in}}%
\pgfpathlineto{\pgfqpoint{3.482039in}{1.602139in}}%
\pgfpathclose%
\pgfusepath{stroke,fill}%
\end{pgfscope}%
\begin{pgfscope}%
\pgfpathrectangle{\pgfqpoint{0.781250in}{0.638889in}}{\pgfqpoint{4.218750in}{2.172222in}}%
\pgfusepath{clip}%
\pgfsetbuttcap%
\pgfsetmiterjoin%
\definecolor{currentfill}{rgb}{0.701961,0.760784,0.360784}%
\pgfsetfillcolor{currentfill}%
\pgfsetlinewidth{1.003750pt}%
\definecolor{currentstroke}{rgb}{0.266667,0.266667,0.266667}%
\pgfsetstrokecolor{currentstroke}%
\pgfsetdash{}{0pt}%
\pgfpathmoveto{\pgfqpoint{3.840471in}{1.805198in}}%
\pgfpathlineto{\pgfqpoint{4.091374in}{1.805198in}}%
\pgfpathlineto{\pgfqpoint{4.091374in}{1.942657in}}%
\pgfpathlineto{\pgfqpoint{3.840471in}{1.942657in}}%
\pgfpathlineto{\pgfqpoint{3.840471in}{1.805198in}}%
\pgfpathclose%
\pgfusepath{stroke,fill}%
\end{pgfscope}%
\begin{pgfscope}%
\pgfpathrectangle{\pgfqpoint{0.781250in}{0.638889in}}{\pgfqpoint{4.218750in}{2.172222in}}%
\pgfusepath{clip}%
\pgfsetbuttcap%
\pgfsetmiterjoin%
\definecolor{currentfill}{rgb}{0.701961,0.760784,0.360784}%
\pgfsetfillcolor{currentfill}%
\pgfsetlinewidth{1.003750pt}%
\definecolor{currentstroke}{rgb}{0.266667,0.266667,0.266667}%
\pgfsetstrokecolor{currentstroke}%
\pgfsetdash{}{0pt}%
\pgfpathmoveto{\pgfqpoint{4.198903in}{1.857028in}}%
\pgfpathlineto{\pgfqpoint{4.449806in}{1.857028in}}%
\pgfpathlineto{\pgfqpoint{4.449806in}{1.992574in}}%
\pgfpathlineto{\pgfqpoint{4.198903in}{1.992574in}}%
\pgfpathlineto{\pgfqpoint{4.198903in}{1.857028in}}%
\pgfpathclose%
\pgfusepath{stroke,fill}%
\end{pgfscope}%
\begin{pgfscope}%
\pgfpathrectangle{\pgfqpoint{0.781250in}{0.638889in}}{\pgfqpoint{4.218750in}{2.172222in}}%
\pgfusepath{clip}%
\pgfsetbuttcap%
\pgfsetmiterjoin%
\definecolor{currentfill}{rgb}{0.701961,0.760784,0.360784}%
\pgfsetfillcolor{currentfill}%
\pgfsetlinewidth{1.003750pt}%
\definecolor{currentstroke}{rgb}{0.266667,0.266667,0.266667}%
\pgfsetstrokecolor{currentstroke}%
\pgfsetdash{}{0pt}%
\pgfpathmoveto{\pgfqpoint{4.557336in}{2.012428in}}%
\pgfpathlineto{\pgfqpoint{4.808239in}{2.012428in}}%
\pgfpathlineto{\pgfqpoint{4.808239in}{2.148649in}}%
\pgfpathlineto{\pgfqpoint{4.557336in}{2.148649in}}%
\pgfpathlineto{\pgfqpoint{4.557336in}{2.012428in}}%
\pgfpathclose%
\pgfusepath{stroke,fill}%
\end{pgfscope}%
\begin{pgfscope}%
\pgfpathrectangle{\pgfqpoint{0.781250in}{0.638889in}}{\pgfqpoint{4.218750in}{2.172222in}}%
\pgfusepath{clip}%
\pgfsetbuttcap%
\pgfsetmiterjoin%
\definecolor{currentfill}{rgb}{0.447059,0.447059,0.447059}%
\pgfsetfillcolor{currentfill}%
\pgfsetlinewidth{1.003750pt}%
\definecolor{currentstroke}{rgb}{0.266667,0.266667,0.266667}%
\pgfsetstrokecolor{currentstroke}%
\pgfsetdash{}{0pt}%
\pgfpathmoveto{\pgfqpoint{0.973011in}{2.077095in}}%
\pgfpathlineto{\pgfqpoint{1.223914in}{2.077095in}}%
\pgfpathlineto{\pgfqpoint{1.223914in}{2.208298in}}%
\pgfpathlineto{\pgfqpoint{0.973011in}{2.208298in}}%
\pgfpathlineto{\pgfqpoint{0.973011in}{2.077095in}}%
\pgfpathclose%
\pgfusepath{stroke,fill}%
\end{pgfscope}%
\begin{pgfscope}%
\pgfpathrectangle{\pgfqpoint{0.781250in}{0.638889in}}{\pgfqpoint{4.218750in}{2.172222in}}%
\pgfusepath{clip}%
\pgfsetbuttcap%
\pgfsetmiterjoin%
\definecolor{currentfill}{rgb}{0.447059,0.447059,0.447059}%
\pgfsetfillcolor{currentfill}%
\pgfsetlinewidth{1.003750pt}%
\definecolor{currentstroke}{rgb}{0.266667,0.266667,0.266667}%
\pgfsetstrokecolor{currentstroke}%
\pgfsetdash{}{0pt}%
\pgfpathmoveto{\pgfqpoint{1.331444in}{1.761819in}}%
\pgfpathlineto{\pgfqpoint{1.582347in}{1.761819in}}%
\pgfpathlineto{\pgfqpoint{1.582347in}{1.872755in}}%
\pgfpathlineto{\pgfqpoint{1.331444in}{1.872755in}}%
\pgfpathlineto{\pgfqpoint{1.331444in}{1.761819in}}%
\pgfpathclose%
\pgfusepath{stroke,fill}%
\end{pgfscope}%
\begin{pgfscope}%
\pgfpathrectangle{\pgfqpoint{0.781250in}{0.638889in}}{\pgfqpoint{4.218750in}{2.172222in}}%
\pgfusepath{clip}%
\pgfsetbuttcap%
\pgfsetmiterjoin%
\definecolor{currentfill}{rgb}{0.447059,0.447059,0.447059}%
\pgfsetfillcolor{currentfill}%
\pgfsetlinewidth{1.003750pt}%
\definecolor{currentstroke}{rgb}{0.266667,0.266667,0.266667}%
\pgfsetstrokecolor{currentstroke}%
\pgfsetdash{}{0pt}%
\pgfpathmoveto{\pgfqpoint{1.689876in}{1.596709in}}%
\pgfpathlineto{\pgfqpoint{1.940779in}{1.596709in}}%
\pgfpathlineto{\pgfqpoint{1.940779in}{1.717962in}}%
\pgfpathlineto{\pgfqpoint{1.689876in}{1.717962in}}%
\pgfpathlineto{\pgfqpoint{1.689876in}{1.596709in}}%
\pgfpathclose%
\pgfusepath{stroke,fill}%
\end{pgfscope}%
\begin{pgfscope}%
\pgfpathrectangle{\pgfqpoint{0.781250in}{0.638889in}}{\pgfqpoint{4.218750in}{2.172222in}}%
\pgfusepath{clip}%
\pgfsetbuttcap%
\pgfsetmiterjoin%
\definecolor{currentfill}{rgb}{0.447059,0.447059,0.447059}%
\pgfsetfillcolor{currentfill}%
\pgfsetlinewidth{1.003750pt}%
\definecolor{currentstroke}{rgb}{0.266667,0.266667,0.266667}%
\pgfsetstrokecolor{currentstroke}%
\pgfsetdash{}{0pt}%
\pgfpathmoveto{\pgfqpoint{2.048309in}{1.585174in}}%
\pgfpathlineto{\pgfqpoint{2.299211in}{1.585174in}}%
\pgfpathlineto{\pgfqpoint{2.299211in}{1.663200in}}%
\pgfpathlineto{\pgfqpoint{2.048309in}{1.663200in}}%
\pgfpathlineto{\pgfqpoint{2.048309in}{1.585174in}}%
\pgfpathclose%
\pgfusepath{stroke,fill}%
\end{pgfscope}%
\begin{pgfscope}%
\pgfpathrectangle{\pgfqpoint{0.781250in}{0.638889in}}{\pgfqpoint{4.218750in}{2.172222in}}%
\pgfusepath{clip}%
\pgfsetbuttcap%
\pgfsetmiterjoin%
\definecolor{currentfill}{rgb}{0.447059,0.447059,0.447059}%
\pgfsetfillcolor{currentfill}%
\pgfsetlinewidth{1.003750pt}%
\definecolor{currentstroke}{rgb}{0.266667,0.266667,0.266667}%
\pgfsetstrokecolor{currentstroke}%
\pgfsetdash{}{0pt}%
\pgfpathmoveto{\pgfqpoint{2.406741in}{1.694545in}}%
\pgfpathlineto{\pgfqpoint{2.657644in}{1.694545in}}%
\pgfpathlineto{\pgfqpoint{2.657644in}{1.774223in}}%
\pgfpathlineto{\pgfqpoint{2.406741in}{1.774223in}}%
\pgfpathlineto{\pgfqpoint{2.406741in}{1.694545in}}%
\pgfpathclose%
\pgfusepath{stroke,fill}%
\end{pgfscope}%
\begin{pgfscope}%
\pgfpathrectangle{\pgfqpoint{0.781250in}{0.638889in}}{\pgfqpoint{4.218750in}{2.172222in}}%
\pgfusepath{clip}%
\pgfsetbuttcap%
\pgfsetmiterjoin%
\definecolor{currentfill}{rgb}{0.447059,0.447059,0.447059}%
\pgfsetfillcolor{currentfill}%
\pgfsetlinewidth{1.003750pt}%
\definecolor{currentstroke}{rgb}{0.266667,0.266667,0.266667}%
\pgfsetstrokecolor{currentstroke}%
\pgfsetdash{}{0pt}%
\pgfpathmoveto{\pgfqpoint{2.765174in}{1.862958in}}%
\pgfpathlineto{\pgfqpoint{3.016076in}{1.862958in}}%
\pgfpathlineto{\pgfqpoint{3.016076in}{1.950607in}}%
\pgfpathlineto{\pgfqpoint{2.765174in}{1.950607in}}%
\pgfpathlineto{\pgfqpoint{2.765174in}{1.862958in}}%
\pgfpathclose%
\pgfusepath{stroke,fill}%
\end{pgfscope}%
\begin{pgfscope}%
\pgfpathrectangle{\pgfqpoint{0.781250in}{0.638889in}}{\pgfqpoint{4.218750in}{2.172222in}}%
\pgfusepath{clip}%
\pgfsetbuttcap%
\pgfsetmiterjoin%
\definecolor{currentfill}{rgb}{0.447059,0.447059,0.447059}%
\pgfsetfillcolor{currentfill}%
\pgfsetlinewidth{1.003750pt}%
\definecolor{currentstroke}{rgb}{0.266667,0.266667,0.266667}%
\pgfsetstrokecolor{currentstroke}%
\pgfsetdash{}{0pt}%
\pgfpathmoveto{\pgfqpoint{3.123606in}{1.767923in}}%
\pgfpathlineto{\pgfqpoint{3.374509in}{1.767923in}}%
\pgfpathlineto{\pgfqpoint{3.374509in}{1.848817in}}%
\pgfpathlineto{\pgfqpoint{3.123606in}{1.848817in}}%
\pgfpathlineto{\pgfqpoint{3.123606in}{1.767923in}}%
\pgfpathclose%
\pgfusepath{stroke,fill}%
\end{pgfscope}%
\begin{pgfscope}%
\pgfpathrectangle{\pgfqpoint{0.781250in}{0.638889in}}{\pgfqpoint{4.218750in}{2.172222in}}%
\pgfusepath{clip}%
\pgfsetbuttcap%
\pgfsetmiterjoin%
\definecolor{currentfill}{rgb}{0.447059,0.447059,0.447059}%
\pgfsetfillcolor{currentfill}%
\pgfsetlinewidth{1.003750pt}%
\definecolor{currentstroke}{rgb}{0.266667,0.266667,0.266667}%
\pgfsetstrokecolor{currentstroke}%
\pgfsetdash{}{0pt}%
\pgfpathmoveto{\pgfqpoint{3.482039in}{1.736274in}}%
\pgfpathlineto{\pgfqpoint{3.732941in}{1.736274in}}%
\pgfpathlineto{\pgfqpoint{3.732941in}{1.839541in}}%
\pgfpathlineto{\pgfqpoint{3.482039in}{1.839541in}}%
\pgfpathlineto{\pgfqpoint{3.482039in}{1.736274in}}%
\pgfpathclose%
\pgfusepath{stroke,fill}%
\end{pgfscope}%
\begin{pgfscope}%
\pgfpathrectangle{\pgfqpoint{0.781250in}{0.638889in}}{\pgfqpoint{4.218750in}{2.172222in}}%
\pgfusepath{clip}%
\pgfsetbuttcap%
\pgfsetmiterjoin%
\definecolor{currentfill}{rgb}{0.447059,0.447059,0.447059}%
\pgfsetfillcolor{currentfill}%
\pgfsetlinewidth{1.003750pt}%
\definecolor{currentstroke}{rgb}{0.266667,0.266667,0.266667}%
\pgfsetstrokecolor{currentstroke}%
\pgfsetdash{}{0pt}%
\pgfpathmoveto{\pgfqpoint{3.840471in}{1.942657in}}%
\pgfpathlineto{\pgfqpoint{4.091374in}{1.942657in}}%
\pgfpathlineto{\pgfqpoint{4.091374in}{2.049964in}}%
\pgfpathlineto{\pgfqpoint{3.840471in}{2.049964in}}%
\pgfpathlineto{\pgfqpoint{3.840471in}{1.942657in}}%
\pgfpathclose%
\pgfusepath{stroke,fill}%
\end{pgfscope}%
\begin{pgfscope}%
\pgfpathrectangle{\pgfqpoint{0.781250in}{0.638889in}}{\pgfqpoint{4.218750in}{2.172222in}}%
\pgfusepath{clip}%
\pgfsetbuttcap%
\pgfsetmiterjoin%
\definecolor{currentfill}{rgb}{0.447059,0.447059,0.447059}%
\pgfsetfillcolor{currentfill}%
\pgfsetlinewidth{1.003750pt}%
\definecolor{currentstroke}{rgb}{0.266667,0.266667,0.266667}%
\pgfsetstrokecolor{currentstroke}%
\pgfsetdash{}{0pt}%
\pgfpathmoveto{\pgfqpoint{4.198903in}{1.992574in}}%
\pgfpathlineto{\pgfqpoint{4.449806in}{1.992574in}}%
\pgfpathlineto{\pgfqpoint{4.449806in}{2.107072in}}%
\pgfpathlineto{\pgfqpoint{4.198903in}{2.107072in}}%
\pgfpathlineto{\pgfqpoint{4.198903in}{1.992574in}}%
\pgfpathclose%
\pgfusepath{stroke,fill}%
\end{pgfscope}%
\begin{pgfscope}%
\pgfpathrectangle{\pgfqpoint{0.781250in}{0.638889in}}{\pgfqpoint{4.218750in}{2.172222in}}%
\pgfusepath{clip}%
\pgfsetbuttcap%
\pgfsetmiterjoin%
\definecolor{currentfill}{rgb}{0.447059,0.447059,0.447059}%
\pgfsetfillcolor{currentfill}%
\pgfsetlinewidth{1.003750pt}%
\definecolor{currentstroke}{rgb}{0.266667,0.266667,0.266667}%
\pgfsetstrokecolor{currentstroke}%
\pgfsetdash{}{0pt}%
\pgfpathmoveto{\pgfqpoint{4.557336in}{2.148648in}}%
\pgfpathlineto{\pgfqpoint{4.808239in}{2.148648in}}%
\pgfpathlineto{\pgfqpoint{4.808239in}{2.248158in}}%
\pgfpathlineto{\pgfqpoint{4.557336in}{2.248158in}}%
\pgfpathlineto{\pgfqpoint{4.557336in}{2.148648in}}%
\pgfpathclose%
\pgfusepath{stroke,fill}%
\end{pgfscope}%
\begin{pgfscope}%
\pgfpathrectangle{\pgfqpoint{0.781250in}{0.638889in}}{\pgfqpoint{4.218750in}{2.172222in}}%
\pgfusepath{clip}%
\pgfsetbuttcap%
\pgfsetmiterjoin%
\definecolor{currentfill}{rgb}{0.447059,0.447059,0.447059}%
\pgfsetfillcolor{currentfill}%
\pgfsetlinewidth{1.003750pt}%
\definecolor{currentstroke}{rgb}{0.266667,0.266667,0.266667}%
\pgfsetstrokecolor{currentstroke}%
\pgfsetdash{}{0pt}%
\pgfpathmoveto{\pgfqpoint{0.973011in}{2.208298in}}%
\pgfpathlineto{\pgfqpoint{1.223914in}{2.208298in}}%
\pgfpathlineto{\pgfqpoint{1.223914in}{2.396847in}}%
\pgfpathlineto{\pgfqpoint{0.973011in}{2.396847in}}%
\pgfpathlineto{\pgfqpoint{0.973011in}{2.208298in}}%
\pgfpathclose%
\pgfusepath{stroke,fill}%
\end{pgfscope}%
\begin{pgfscope}%
\pgfpathrectangle{\pgfqpoint{0.781250in}{0.638889in}}{\pgfqpoint{4.218750in}{2.172222in}}%
\pgfusepath{clip}%
\pgfsetbuttcap%
\pgfsetmiterjoin%
\definecolor{currentfill}{rgb}{0.447059,0.447059,0.447059}%
\pgfsetfillcolor{currentfill}%
\pgfsetlinewidth{1.003750pt}%
\definecolor{currentstroke}{rgb}{0.266667,0.266667,0.266667}%
\pgfsetstrokecolor{currentstroke}%
\pgfsetdash{}{0pt}%
\pgfpathmoveto{\pgfqpoint{1.331444in}{1.872755in}}%
\pgfpathlineto{\pgfqpoint{1.582347in}{1.872755in}}%
\pgfpathlineto{\pgfqpoint{1.582347in}{1.999200in}}%
\pgfpathlineto{\pgfqpoint{1.331444in}{1.999200in}}%
\pgfpathlineto{\pgfqpoint{1.331444in}{1.872755in}}%
\pgfpathclose%
\pgfusepath{stroke,fill}%
\end{pgfscope}%
\begin{pgfscope}%
\pgfpathrectangle{\pgfqpoint{0.781250in}{0.638889in}}{\pgfqpoint{4.218750in}{2.172222in}}%
\pgfusepath{clip}%
\pgfsetbuttcap%
\pgfsetmiterjoin%
\definecolor{currentfill}{rgb}{0.447059,0.447059,0.447059}%
\pgfsetfillcolor{currentfill}%
\pgfsetlinewidth{1.003750pt}%
\definecolor{currentstroke}{rgb}{0.266667,0.266667,0.266667}%
\pgfsetstrokecolor{currentstroke}%
\pgfsetdash{}{0pt}%
\pgfpathmoveto{\pgfqpoint{1.689876in}{1.717962in}}%
\pgfpathlineto{\pgfqpoint{1.940779in}{1.717962in}}%
\pgfpathlineto{\pgfqpoint{1.940779in}{1.838586in}}%
\pgfpathlineto{\pgfqpoint{1.689876in}{1.838586in}}%
\pgfpathlineto{\pgfqpoint{1.689876in}{1.717962in}}%
\pgfpathclose%
\pgfusepath{stroke,fill}%
\end{pgfscope}%
\begin{pgfscope}%
\pgfpathrectangle{\pgfqpoint{0.781250in}{0.638889in}}{\pgfqpoint{4.218750in}{2.172222in}}%
\pgfusepath{clip}%
\pgfsetbuttcap%
\pgfsetmiterjoin%
\definecolor{currentfill}{rgb}{0.447059,0.447059,0.447059}%
\pgfsetfillcolor{currentfill}%
\pgfsetlinewidth{1.003750pt}%
\definecolor{currentstroke}{rgb}{0.266667,0.266667,0.266667}%
\pgfsetstrokecolor{currentstroke}%
\pgfsetdash{}{0pt}%
\pgfpathmoveto{\pgfqpoint{2.048309in}{1.663200in}}%
\pgfpathlineto{\pgfqpoint{2.299211in}{1.663200in}}%
\pgfpathlineto{\pgfqpoint{2.299211in}{1.784758in}}%
\pgfpathlineto{\pgfqpoint{2.048309in}{1.784758in}}%
\pgfpathlineto{\pgfqpoint{2.048309in}{1.663200in}}%
\pgfpathclose%
\pgfusepath{stroke,fill}%
\end{pgfscope}%
\begin{pgfscope}%
\pgfpathrectangle{\pgfqpoint{0.781250in}{0.638889in}}{\pgfqpoint{4.218750in}{2.172222in}}%
\pgfusepath{clip}%
\pgfsetbuttcap%
\pgfsetmiterjoin%
\definecolor{currentfill}{rgb}{0.447059,0.447059,0.447059}%
\pgfsetfillcolor{currentfill}%
\pgfsetlinewidth{1.003750pt}%
\definecolor{currentstroke}{rgb}{0.266667,0.266667,0.266667}%
\pgfsetstrokecolor{currentstroke}%
\pgfsetdash{}{0pt}%
\pgfpathmoveto{\pgfqpoint{2.406741in}{1.774223in}}%
\pgfpathlineto{\pgfqpoint{2.657644in}{1.774223in}}%
\pgfpathlineto{\pgfqpoint{2.657644in}{1.886570in}}%
\pgfpathlineto{\pgfqpoint{2.406741in}{1.886570in}}%
\pgfpathlineto{\pgfqpoint{2.406741in}{1.774223in}}%
\pgfpathclose%
\pgfusepath{stroke,fill}%
\end{pgfscope}%
\begin{pgfscope}%
\pgfpathrectangle{\pgfqpoint{0.781250in}{0.638889in}}{\pgfqpoint{4.218750in}{2.172222in}}%
\pgfusepath{clip}%
\pgfsetbuttcap%
\pgfsetmiterjoin%
\definecolor{currentfill}{rgb}{0.447059,0.447059,0.447059}%
\pgfsetfillcolor{currentfill}%
\pgfsetlinewidth{1.003750pt}%
\definecolor{currentstroke}{rgb}{0.266667,0.266667,0.266667}%
\pgfsetstrokecolor{currentstroke}%
\pgfsetdash{}{0pt}%
\pgfpathmoveto{\pgfqpoint{2.765174in}{1.950607in}}%
\pgfpathlineto{\pgfqpoint{3.016076in}{1.950607in}}%
\pgfpathlineto{\pgfqpoint{3.016076in}{2.079311in}}%
\pgfpathlineto{\pgfqpoint{2.765174in}{2.079311in}}%
\pgfpathlineto{\pgfqpoint{2.765174in}{1.950607in}}%
\pgfpathclose%
\pgfusepath{stroke,fill}%
\end{pgfscope}%
\begin{pgfscope}%
\pgfpathrectangle{\pgfqpoint{0.781250in}{0.638889in}}{\pgfqpoint{4.218750in}{2.172222in}}%
\pgfusepath{clip}%
\pgfsetbuttcap%
\pgfsetmiterjoin%
\definecolor{currentfill}{rgb}{0.447059,0.447059,0.447059}%
\pgfsetfillcolor{currentfill}%
\pgfsetlinewidth{1.003750pt}%
\definecolor{currentstroke}{rgb}{0.266667,0.266667,0.266667}%
\pgfsetstrokecolor{currentstroke}%
\pgfsetdash{}{0pt}%
\pgfpathmoveto{\pgfqpoint{3.123606in}{1.848817in}}%
\pgfpathlineto{\pgfqpoint{3.374509in}{1.848817in}}%
\pgfpathlineto{\pgfqpoint{3.374509in}{1.977868in}}%
\pgfpathlineto{\pgfqpoint{3.123606in}{1.977868in}}%
\pgfpathlineto{\pgfqpoint{3.123606in}{1.848817in}}%
\pgfpathclose%
\pgfusepath{stroke,fill}%
\end{pgfscope}%
\begin{pgfscope}%
\pgfpathrectangle{\pgfqpoint{0.781250in}{0.638889in}}{\pgfqpoint{4.218750in}{2.172222in}}%
\pgfusepath{clip}%
\pgfsetbuttcap%
\pgfsetmiterjoin%
\definecolor{currentfill}{rgb}{0.447059,0.447059,0.447059}%
\pgfsetfillcolor{currentfill}%
\pgfsetlinewidth{1.003750pt}%
\definecolor{currentstroke}{rgb}{0.266667,0.266667,0.266667}%
\pgfsetstrokecolor{currentstroke}%
\pgfsetdash{}{0pt}%
\pgfpathmoveto{\pgfqpoint{3.482039in}{1.839541in}}%
\pgfpathlineto{\pgfqpoint{3.732941in}{1.839541in}}%
\pgfpathlineto{\pgfqpoint{3.732941in}{1.954126in}}%
\pgfpathlineto{\pgfqpoint{3.482039in}{1.954126in}}%
\pgfpathlineto{\pgfqpoint{3.482039in}{1.839541in}}%
\pgfpathclose%
\pgfusepath{stroke,fill}%
\end{pgfscope}%
\begin{pgfscope}%
\pgfpathrectangle{\pgfqpoint{0.781250in}{0.638889in}}{\pgfqpoint{4.218750in}{2.172222in}}%
\pgfusepath{clip}%
\pgfsetbuttcap%
\pgfsetmiterjoin%
\definecolor{currentfill}{rgb}{0.447059,0.447059,0.447059}%
\pgfsetfillcolor{currentfill}%
\pgfsetlinewidth{1.003750pt}%
\definecolor{currentstroke}{rgb}{0.266667,0.266667,0.266667}%
\pgfsetstrokecolor{currentstroke}%
\pgfsetdash{}{0pt}%
\pgfpathmoveto{\pgfqpoint{3.840471in}{2.049964in}}%
\pgfpathlineto{\pgfqpoint{4.091374in}{2.049964in}}%
\pgfpathlineto{\pgfqpoint{4.091374in}{2.174650in}}%
\pgfpathlineto{\pgfqpoint{3.840471in}{2.174650in}}%
\pgfpathlineto{\pgfqpoint{3.840471in}{2.049964in}}%
\pgfpathclose%
\pgfusepath{stroke,fill}%
\end{pgfscope}%
\begin{pgfscope}%
\pgfpathrectangle{\pgfqpoint{0.781250in}{0.638889in}}{\pgfqpoint{4.218750in}{2.172222in}}%
\pgfusepath{clip}%
\pgfsetbuttcap%
\pgfsetmiterjoin%
\definecolor{currentfill}{rgb}{0.447059,0.447059,0.447059}%
\pgfsetfillcolor{currentfill}%
\pgfsetlinewidth{1.003750pt}%
\definecolor{currentstroke}{rgb}{0.266667,0.266667,0.266667}%
\pgfsetstrokecolor{currentstroke}%
\pgfsetdash{}{0pt}%
\pgfpathmoveto{\pgfqpoint{4.198903in}{2.107072in}}%
\pgfpathlineto{\pgfqpoint{4.449806in}{2.107072in}}%
\pgfpathlineto{\pgfqpoint{4.449806in}{2.214966in}}%
\pgfpathlineto{\pgfqpoint{4.198903in}{2.214966in}}%
\pgfpathlineto{\pgfqpoint{4.198903in}{2.107072in}}%
\pgfpathclose%
\pgfusepath{stroke,fill}%
\end{pgfscope}%
\begin{pgfscope}%
\pgfpathrectangle{\pgfqpoint{0.781250in}{0.638889in}}{\pgfqpoint{4.218750in}{2.172222in}}%
\pgfusepath{clip}%
\pgfsetbuttcap%
\pgfsetmiterjoin%
\definecolor{currentfill}{rgb}{0.447059,0.447059,0.447059}%
\pgfsetfillcolor{currentfill}%
\pgfsetlinewidth{1.003750pt}%
\definecolor{currentstroke}{rgb}{0.266667,0.266667,0.266667}%
\pgfsetstrokecolor{currentstroke}%
\pgfsetdash{}{0pt}%
\pgfpathmoveto{\pgfqpoint{4.557336in}{2.248158in}}%
\pgfpathlineto{\pgfqpoint{4.808239in}{2.248158in}}%
\pgfpathlineto{\pgfqpoint{4.808239in}{2.361657in}}%
\pgfpathlineto{\pgfqpoint{4.557336in}{2.361657in}}%
\pgfpathlineto{\pgfqpoint{4.557336in}{2.248158in}}%
\pgfpathclose%
\pgfusepath{stroke,fill}%
\end{pgfscope}%
\begin{pgfscope}%
\pgfpathrectangle{\pgfqpoint{0.781250in}{0.638889in}}{\pgfqpoint{4.218750in}{2.172222in}}%
\pgfusepath{clip}%
\pgfsetbuttcap%
\pgfsetmiterjoin%
\definecolor{currentfill}{rgb}{0.447059,0.447059,0.447059}%
\pgfsetfillcolor{currentfill}%
\pgfsetlinewidth{1.003750pt}%
\definecolor{currentstroke}{rgb}{0.266667,0.266667,0.266667}%
\pgfsetstrokecolor{currentstroke}%
\pgfsetdash{}{0pt}%
\pgfpathmoveto{\pgfqpoint{0.973011in}{2.396847in}}%
\pgfpathlineto{\pgfqpoint{1.223914in}{2.396847in}}%
\pgfpathlineto{\pgfqpoint{1.223914in}{2.539105in}}%
\pgfpathlineto{\pgfqpoint{0.973011in}{2.539105in}}%
\pgfpathlineto{\pgfqpoint{0.973011in}{2.396847in}}%
\pgfpathclose%
\pgfusepath{stroke,fill}%
\end{pgfscope}%
\begin{pgfscope}%
\pgfpathrectangle{\pgfqpoint{0.781250in}{0.638889in}}{\pgfqpoint{4.218750in}{2.172222in}}%
\pgfusepath{clip}%
\pgfsetbuttcap%
\pgfsetmiterjoin%
\definecolor{currentfill}{rgb}{0.447059,0.447059,0.447059}%
\pgfsetfillcolor{currentfill}%
\pgfsetlinewidth{1.003750pt}%
\definecolor{currentstroke}{rgb}{0.266667,0.266667,0.266667}%
\pgfsetstrokecolor{currentstroke}%
\pgfsetdash{}{0pt}%
\pgfpathmoveto{\pgfqpoint{1.331444in}{1.999200in}}%
\pgfpathlineto{\pgfqpoint{1.582347in}{1.999200in}}%
\pgfpathlineto{\pgfqpoint{1.582347in}{2.112047in}}%
\pgfpathlineto{\pgfqpoint{1.331444in}{2.112047in}}%
\pgfpathlineto{\pgfqpoint{1.331444in}{1.999200in}}%
\pgfpathclose%
\pgfusepath{stroke,fill}%
\end{pgfscope}%
\begin{pgfscope}%
\pgfpathrectangle{\pgfqpoint{0.781250in}{0.638889in}}{\pgfqpoint{4.218750in}{2.172222in}}%
\pgfusepath{clip}%
\pgfsetbuttcap%
\pgfsetmiterjoin%
\definecolor{currentfill}{rgb}{0.447059,0.447059,0.447059}%
\pgfsetfillcolor{currentfill}%
\pgfsetlinewidth{1.003750pt}%
\definecolor{currentstroke}{rgb}{0.266667,0.266667,0.266667}%
\pgfsetstrokecolor{currentstroke}%
\pgfsetdash{}{0pt}%
\pgfpathmoveto{\pgfqpoint{1.689876in}{1.838586in}}%
\pgfpathlineto{\pgfqpoint{1.940779in}{1.838586in}}%
\pgfpathlineto{\pgfqpoint{1.940779in}{1.937661in}}%
\pgfpathlineto{\pgfqpoint{1.689876in}{1.937661in}}%
\pgfpathlineto{\pgfqpoint{1.689876in}{1.838586in}}%
\pgfpathclose%
\pgfusepath{stroke,fill}%
\end{pgfscope}%
\begin{pgfscope}%
\pgfpathrectangle{\pgfqpoint{0.781250in}{0.638889in}}{\pgfqpoint{4.218750in}{2.172222in}}%
\pgfusepath{clip}%
\pgfsetbuttcap%
\pgfsetmiterjoin%
\definecolor{currentfill}{rgb}{0.447059,0.447059,0.447059}%
\pgfsetfillcolor{currentfill}%
\pgfsetlinewidth{1.003750pt}%
\definecolor{currentstroke}{rgb}{0.266667,0.266667,0.266667}%
\pgfsetstrokecolor{currentstroke}%
\pgfsetdash{}{0pt}%
\pgfpathmoveto{\pgfqpoint{2.048309in}{1.784758in}}%
\pgfpathlineto{\pgfqpoint{2.299211in}{1.784758in}}%
\pgfpathlineto{\pgfqpoint{2.299211in}{1.874710in}}%
\pgfpathlineto{\pgfqpoint{2.048309in}{1.874710in}}%
\pgfpathlineto{\pgfqpoint{2.048309in}{1.784758in}}%
\pgfpathclose%
\pgfusepath{stroke,fill}%
\end{pgfscope}%
\begin{pgfscope}%
\pgfpathrectangle{\pgfqpoint{0.781250in}{0.638889in}}{\pgfqpoint{4.218750in}{2.172222in}}%
\pgfusepath{clip}%
\pgfsetbuttcap%
\pgfsetmiterjoin%
\definecolor{currentfill}{rgb}{0.447059,0.447059,0.447059}%
\pgfsetfillcolor{currentfill}%
\pgfsetlinewidth{1.003750pt}%
\definecolor{currentstroke}{rgb}{0.266667,0.266667,0.266667}%
\pgfsetstrokecolor{currentstroke}%
\pgfsetdash{}{0pt}%
\pgfpathmoveto{\pgfqpoint{2.406741in}{1.886570in}}%
\pgfpathlineto{\pgfqpoint{2.657644in}{1.886570in}}%
\pgfpathlineto{\pgfqpoint{2.657644in}{1.979432in}}%
\pgfpathlineto{\pgfqpoint{2.406741in}{1.979432in}}%
\pgfpathlineto{\pgfqpoint{2.406741in}{1.886570in}}%
\pgfpathclose%
\pgfusepath{stroke,fill}%
\end{pgfscope}%
\begin{pgfscope}%
\pgfpathrectangle{\pgfqpoint{0.781250in}{0.638889in}}{\pgfqpoint{4.218750in}{2.172222in}}%
\pgfusepath{clip}%
\pgfsetbuttcap%
\pgfsetmiterjoin%
\definecolor{currentfill}{rgb}{0.447059,0.447059,0.447059}%
\pgfsetfillcolor{currentfill}%
\pgfsetlinewidth{1.003750pt}%
\definecolor{currentstroke}{rgb}{0.266667,0.266667,0.266667}%
\pgfsetstrokecolor{currentstroke}%
\pgfsetdash{}{0pt}%
\pgfpathmoveto{\pgfqpoint{2.765174in}{2.079311in}}%
\pgfpathlineto{\pgfqpoint{3.016076in}{2.079311in}}%
\pgfpathlineto{\pgfqpoint{3.016076in}{2.169437in}}%
\pgfpathlineto{\pgfqpoint{2.765174in}{2.169437in}}%
\pgfpathlineto{\pgfqpoint{2.765174in}{2.079311in}}%
\pgfpathclose%
\pgfusepath{stroke,fill}%
\end{pgfscope}%
\begin{pgfscope}%
\pgfpathrectangle{\pgfqpoint{0.781250in}{0.638889in}}{\pgfqpoint{4.218750in}{2.172222in}}%
\pgfusepath{clip}%
\pgfsetbuttcap%
\pgfsetmiterjoin%
\definecolor{currentfill}{rgb}{0.447059,0.447059,0.447059}%
\pgfsetfillcolor{currentfill}%
\pgfsetlinewidth{1.003750pt}%
\definecolor{currentstroke}{rgb}{0.266667,0.266667,0.266667}%
\pgfsetstrokecolor{currentstroke}%
\pgfsetdash{}{0pt}%
\pgfpathmoveto{\pgfqpoint{3.123606in}{1.977868in}}%
\pgfpathlineto{\pgfqpoint{3.374509in}{1.977868in}}%
\pgfpathlineto{\pgfqpoint{3.374509in}{2.062607in}}%
\pgfpathlineto{\pgfqpoint{3.123606in}{2.062607in}}%
\pgfpathlineto{\pgfqpoint{3.123606in}{1.977868in}}%
\pgfpathclose%
\pgfusepath{stroke,fill}%
\end{pgfscope}%
\begin{pgfscope}%
\pgfpathrectangle{\pgfqpoint{0.781250in}{0.638889in}}{\pgfqpoint{4.218750in}{2.172222in}}%
\pgfusepath{clip}%
\pgfsetbuttcap%
\pgfsetmiterjoin%
\definecolor{currentfill}{rgb}{0.447059,0.447059,0.447059}%
\pgfsetfillcolor{currentfill}%
\pgfsetlinewidth{1.003750pt}%
\definecolor{currentstroke}{rgb}{0.266667,0.266667,0.266667}%
\pgfsetstrokecolor{currentstroke}%
\pgfsetdash{}{0pt}%
\pgfpathmoveto{\pgfqpoint{3.482039in}{1.954126in}}%
\pgfpathlineto{\pgfqpoint{3.732941in}{1.954126in}}%
\pgfpathlineto{\pgfqpoint{3.732941in}{2.031370in}}%
\pgfpathlineto{\pgfqpoint{3.482039in}{2.031370in}}%
\pgfpathlineto{\pgfqpoint{3.482039in}{1.954126in}}%
\pgfpathclose%
\pgfusepath{stroke,fill}%
\end{pgfscope}%
\begin{pgfscope}%
\pgfpathrectangle{\pgfqpoint{0.781250in}{0.638889in}}{\pgfqpoint{4.218750in}{2.172222in}}%
\pgfusepath{clip}%
\pgfsetbuttcap%
\pgfsetmiterjoin%
\definecolor{currentfill}{rgb}{0.447059,0.447059,0.447059}%
\pgfsetfillcolor{currentfill}%
\pgfsetlinewidth{1.003750pt}%
\definecolor{currentstroke}{rgb}{0.266667,0.266667,0.266667}%
\pgfsetstrokecolor{currentstroke}%
\pgfsetdash{}{0pt}%
\pgfpathmoveto{\pgfqpoint{3.840471in}{2.174650in}}%
\pgfpathlineto{\pgfqpoint{4.091374in}{2.174650in}}%
\pgfpathlineto{\pgfqpoint{4.091374in}{2.255956in}}%
\pgfpathlineto{\pgfqpoint{3.840471in}{2.255956in}}%
\pgfpathlineto{\pgfqpoint{3.840471in}{2.174650in}}%
\pgfpathclose%
\pgfusepath{stroke,fill}%
\end{pgfscope}%
\begin{pgfscope}%
\pgfpathrectangle{\pgfqpoint{0.781250in}{0.638889in}}{\pgfqpoint{4.218750in}{2.172222in}}%
\pgfusepath{clip}%
\pgfsetbuttcap%
\pgfsetmiterjoin%
\definecolor{currentfill}{rgb}{0.447059,0.447059,0.447059}%
\pgfsetfillcolor{currentfill}%
\pgfsetlinewidth{1.003750pt}%
\definecolor{currentstroke}{rgb}{0.266667,0.266667,0.266667}%
\pgfsetstrokecolor{currentstroke}%
\pgfsetdash{}{0pt}%
\pgfpathmoveto{\pgfqpoint{4.198903in}{2.214966in}}%
\pgfpathlineto{\pgfqpoint{4.449806in}{2.214966in}}%
\pgfpathlineto{\pgfqpoint{4.449806in}{2.295404in}}%
\pgfpathlineto{\pgfqpoint{4.198903in}{2.295404in}}%
\pgfpathlineto{\pgfqpoint{4.198903in}{2.214966in}}%
\pgfpathclose%
\pgfusepath{stroke,fill}%
\end{pgfscope}%
\begin{pgfscope}%
\pgfpathrectangle{\pgfqpoint{0.781250in}{0.638889in}}{\pgfqpoint{4.218750in}{2.172222in}}%
\pgfusepath{clip}%
\pgfsetbuttcap%
\pgfsetmiterjoin%
\definecolor{currentfill}{rgb}{0.447059,0.447059,0.447059}%
\pgfsetfillcolor{currentfill}%
\pgfsetlinewidth{1.003750pt}%
\definecolor{currentstroke}{rgb}{0.266667,0.266667,0.266667}%
\pgfsetstrokecolor{currentstroke}%
\pgfsetdash{}{0pt}%
\pgfpathmoveto{\pgfqpoint{4.557336in}{2.361657in}}%
\pgfpathlineto{\pgfqpoint{4.808239in}{2.361657in}}%
\pgfpathlineto{\pgfqpoint{4.808239in}{2.444092in}}%
\pgfpathlineto{\pgfqpoint{4.557336in}{2.444092in}}%
\pgfpathlineto{\pgfqpoint{4.557336in}{2.361657in}}%
\pgfpathclose%
\pgfusepath{stroke,fill}%
\end{pgfscope}%
\begin{pgfscope}%
\pgfpathrectangle{\pgfqpoint{0.781250in}{0.638889in}}{\pgfqpoint{4.218750in}{2.172222in}}%
\pgfusepath{clip}%
\pgfsetbuttcap%
\pgfsetmiterjoin%
\definecolor{currentfill}{rgb}{0.447059,0.447059,0.447059}%
\pgfsetfillcolor{currentfill}%
\pgfsetlinewidth{1.003750pt}%
\definecolor{currentstroke}{rgb}{0.266667,0.266667,0.266667}%
\pgfsetstrokecolor{currentstroke}%
\pgfsetdash{}{0pt}%
\pgfpathmoveto{\pgfqpoint{0.973011in}{2.539105in}}%
\pgfpathlineto{\pgfqpoint{1.223914in}{2.539105in}}%
\pgfpathlineto{\pgfqpoint{1.223914in}{2.551856in}}%
\pgfpathlineto{\pgfqpoint{0.973011in}{2.551856in}}%
\pgfpathlineto{\pgfqpoint{0.973011in}{2.539105in}}%
\pgfpathclose%
\pgfusepath{stroke,fill}%
\end{pgfscope}%
\begin{pgfscope}%
\pgfpathrectangle{\pgfqpoint{0.781250in}{0.638889in}}{\pgfqpoint{4.218750in}{2.172222in}}%
\pgfusepath{clip}%
\pgfsetbuttcap%
\pgfsetmiterjoin%
\definecolor{currentfill}{rgb}{0.447059,0.447059,0.447059}%
\pgfsetfillcolor{currentfill}%
\pgfsetlinewidth{1.003750pt}%
\definecolor{currentstroke}{rgb}{0.266667,0.266667,0.266667}%
\pgfsetstrokecolor{currentstroke}%
\pgfsetdash{}{0pt}%
\pgfpathmoveto{\pgfqpoint{1.331444in}{2.112047in}}%
\pgfpathlineto{\pgfqpoint{1.582347in}{2.112047in}}%
\pgfpathlineto{\pgfqpoint{1.582347in}{2.123690in}}%
\pgfpathlineto{\pgfqpoint{1.331444in}{2.123690in}}%
\pgfpathlineto{\pgfqpoint{1.331444in}{2.112047in}}%
\pgfpathclose%
\pgfusepath{stroke,fill}%
\end{pgfscope}%
\begin{pgfscope}%
\pgfpathrectangle{\pgfqpoint{0.781250in}{0.638889in}}{\pgfqpoint{4.218750in}{2.172222in}}%
\pgfusepath{clip}%
\pgfsetbuttcap%
\pgfsetmiterjoin%
\definecolor{currentfill}{rgb}{0.447059,0.447059,0.447059}%
\pgfsetfillcolor{currentfill}%
\pgfsetlinewidth{1.003750pt}%
\definecolor{currentstroke}{rgb}{0.266667,0.266667,0.266667}%
\pgfsetstrokecolor{currentstroke}%
\pgfsetdash{}{0pt}%
\pgfpathmoveto{\pgfqpoint{1.689876in}{1.937661in}}%
\pgfpathlineto{\pgfqpoint{1.940779in}{1.937661in}}%
\pgfpathlineto{\pgfqpoint{1.940779in}{1.943982in}}%
\pgfpathlineto{\pgfqpoint{1.689876in}{1.943982in}}%
\pgfpathlineto{\pgfqpoint{1.689876in}{1.937661in}}%
\pgfpathclose%
\pgfusepath{stroke,fill}%
\end{pgfscope}%
\begin{pgfscope}%
\pgfpathrectangle{\pgfqpoint{0.781250in}{0.638889in}}{\pgfqpoint{4.218750in}{2.172222in}}%
\pgfusepath{clip}%
\pgfsetbuttcap%
\pgfsetmiterjoin%
\definecolor{currentfill}{rgb}{0.447059,0.447059,0.447059}%
\pgfsetfillcolor{currentfill}%
\pgfsetlinewidth{1.003750pt}%
\definecolor{currentstroke}{rgb}{0.266667,0.266667,0.266667}%
\pgfsetstrokecolor{currentstroke}%
\pgfsetdash{}{0pt}%
\pgfpathmoveto{\pgfqpoint{2.048309in}{1.874710in}}%
\pgfpathlineto{\pgfqpoint{2.299211in}{1.874710in}}%
\pgfpathlineto{\pgfqpoint{2.299211in}{1.880879in}}%
\pgfpathlineto{\pgfqpoint{2.048309in}{1.880879in}}%
\pgfpathlineto{\pgfqpoint{2.048309in}{1.874710in}}%
\pgfpathclose%
\pgfusepath{stroke,fill}%
\end{pgfscope}%
\begin{pgfscope}%
\pgfpathrectangle{\pgfqpoint{0.781250in}{0.638889in}}{\pgfqpoint{4.218750in}{2.172222in}}%
\pgfusepath{clip}%
\pgfsetbuttcap%
\pgfsetmiterjoin%
\definecolor{currentfill}{rgb}{0.447059,0.447059,0.447059}%
\pgfsetfillcolor{currentfill}%
\pgfsetlinewidth{1.003750pt}%
\definecolor{currentstroke}{rgb}{0.266667,0.266667,0.266667}%
\pgfsetstrokecolor{currentstroke}%
\pgfsetdash{}{0pt}%
\pgfpathmoveto{\pgfqpoint{2.406741in}{1.979432in}}%
\pgfpathlineto{\pgfqpoint{2.657644in}{1.979432in}}%
\pgfpathlineto{\pgfqpoint{2.657644in}{1.985058in}}%
\pgfpathlineto{\pgfqpoint{2.406741in}{1.985058in}}%
\pgfpathlineto{\pgfqpoint{2.406741in}{1.979432in}}%
\pgfpathclose%
\pgfusepath{stroke,fill}%
\end{pgfscope}%
\begin{pgfscope}%
\pgfpathrectangle{\pgfqpoint{0.781250in}{0.638889in}}{\pgfqpoint{4.218750in}{2.172222in}}%
\pgfusepath{clip}%
\pgfsetbuttcap%
\pgfsetmiterjoin%
\definecolor{currentfill}{rgb}{0.447059,0.447059,0.447059}%
\pgfsetfillcolor{currentfill}%
\pgfsetlinewidth{1.003750pt}%
\definecolor{currentstroke}{rgb}{0.266667,0.266667,0.266667}%
\pgfsetstrokecolor{currentstroke}%
\pgfsetdash{}{0pt}%
\pgfpathmoveto{\pgfqpoint{2.765174in}{2.169437in}}%
\pgfpathlineto{\pgfqpoint{3.016076in}{2.169437in}}%
\pgfpathlineto{\pgfqpoint{3.016076in}{2.175562in}}%
\pgfpathlineto{\pgfqpoint{2.765174in}{2.175562in}}%
\pgfpathlineto{\pgfqpoint{2.765174in}{2.169437in}}%
\pgfpathclose%
\pgfusepath{stroke,fill}%
\end{pgfscope}%
\begin{pgfscope}%
\pgfpathrectangle{\pgfqpoint{0.781250in}{0.638889in}}{\pgfqpoint{4.218750in}{2.172222in}}%
\pgfusepath{clip}%
\pgfsetbuttcap%
\pgfsetmiterjoin%
\definecolor{currentfill}{rgb}{0.447059,0.447059,0.447059}%
\pgfsetfillcolor{currentfill}%
\pgfsetlinewidth{1.003750pt}%
\definecolor{currentstroke}{rgb}{0.266667,0.266667,0.266667}%
\pgfsetstrokecolor{currentstroke}%
\pgfsetdash{}{0pt}%
\pgfpathmoveto{\pgfqpoint{3.123606in}{2.062607in}}%
\pgfpathlineto{\pgfqpoint{3.374509in}{2.062607in}}%
\pgfpathlineto{\pgfqpoint{3.374509in}{2.070861in}}%
\pgfpathlineto{\pgfqpoint{3.123606in}{2.070861in}}%
\pgfpathlineto{\pgfqpoint{3.123606in}{2.062607in}}%
\pgfpathclose%
\pgfusepath{stroke,fill}%
\end{pgfscope}%
\begin{pgfscope}%
\pgfpathrectangle{\pgfqpoint{0.781250in}{0.638889in}}{\pgfqpoint{4.218750in}{2.172222in}}%
\pgfusepath{clip}%
\pgfsetbuttcap%
\pgfsetmiterjoin%
\definecolor{currentfill}{rgb}{0.447059,0.447059,0.447059}%
\pgfsetfillcolor{currentfill}%
\pgfsetlinewidth{1.003750pt}%
\definecolor{currentstroke}{rgb}{0.266667,0.266667,0.266667}%
\pgfsetstrokecolor{currentstroke}%
\pgfsetdash{}{0pt}%
\pgfpathmoveto{\pgfqpoint{3.482039in}{2.031370in}}%
\pgfpathlineto{\pgfqpoint{3.732941in}{2.031370in}}%
\pgfpathlineto{\pgfqpoint{3.732941in}{2.039342in}}%
\pgfpathlineto{\pgfqpoint{3.482039in}{2.039342in}}%
\pgfpathlineto{\pgfqpoint{3.482039in}{2.031370in}}%
\pgfpathclose%
\pgfusepath{stroke,fill}%
\end{pgfscope}%
\begin{pgfscope}%
\pgfpathrectangle{\pgfqpoint{0.781250in}{0.638889in}}{\pgfqpoint{4.218750in}{2.172222in}}%
\pgfusepath{clip}%
\pgfsetbuttcap%
\pgfsetmiterjoin%
\definecolor{currentfill}{rgb}{0.447059,0.447059,0.447059}%
\pgfsetfillcolor{currentfill}%
\pgfsetlinewidth{1.003750pt}%
\definecolor{currentstroke}{rgb}{0.266667,0.266667,0.266667}%
\pgfsetstrokecolor{currentstroke}%
\pgfsetdash{}{0pt}%
\pgfpathmoveto{\pgfqpoint{3.840471in}{2.255956in}}%
\pgfpathlineto{\pgfqpoint{4.091374in}{2.255956in}}%
\pgfpathlineto{\pgfqpoint{4.091374in}{2.264645in}}%
\pgfpathlineto{\pgfqpoint{3.840471in}{2.264645in}}%
\pgfpathlineto{\pgfqpoint{3.840471in}{2.255956in}}%
\pgfpathclose%
\pgfusepath{stroke,fill}%
\end{pgfscope}%
\begin{pgfscope}%
\pgfpathrectangle{\pgfqpoint{0.781250in}{0.638889in}}{\pgfqpoint{4.218750in}{2.172222in}}%
\pgfusepath{clip}%
\pgfsetbuttcap%
\pgfsetmiterjoin%
\definecolor{currentfill}{rgb}{0.447059,0.447059,0.447059}%
\pgfsetfillcolor{currentfill}%
\pgfsetlinewidth{1.003750pt}%
\definecolor{currentstroke}{rgb}{0.266667,0.266667,0.266667}%
\pgfsetstrokecolor{currentstroke}%
\pgfsetdash{}{0pt}%
\pgfpathmoveto{\pgfqpoint{4.198903in}{2.295404in}}%
\pgfpathlineto{\pgfqpoint{4.449806in}{2.295404in}}%
\pgfpathlineto{\pgfqpoint{4.449806in}{2.304071in}}%
\pgfpathlineto{\pgfqpoint{4.198903in}{2.304071in}}%
\pgfpathlineto{\pgfqpoint{4.198903in}{2.295404in}}%
\pgfpathclose%
\pgfusepath{stroke,fill}%
\end{pgfscope}%
\begin{pgfscope}%
\pgfpathrectangle{\pgfqpoint{0.781250in}{0.638889in}}{\pgfqpoint{4.218750in}{2.172222in}}%
\pgfusepath{clip}%
\pgfsetbuttcap%
\pgfsetmiterjoin%
\definecolor{currentfill}{rgb}{0.447059,0.447059,0.447059}%
\pgfsetfillcolor{currentfill}%
\pgfsetlinewidth{1.003750pt}%
\definecolor{currentstroke}{rgb}{0.266667,0.266667,0.266667}%
\pgfsetstrokecolor{currentstroke}%
\pgfsetdash{}{0pt}%
\pgfpathmoveto{\pgfqpoint{4.557336in}{2.444092in}}%
\pgfpathlineto{\pgfqpoint{4.808239in}{2.444092in}}%
\pgfpathlineto{\pgfqpoint{4.808239in}{2.453172in}}%
\pgfpathlineto{\pgfqpoint{4.557336in}{2.453172in}}%
\pgfpathlineto{\pgfqpoint{4.557336in}{2.444092in}}%
\pgfpathclose%
\pgfusepath{stroke,fill}%
\end{pgfscope}%
\begin{pgfscope}%
\pgfpathrectangle{\pgfqpoint{0.781250in}{0.638889in}}{\pgfqpoint{4.218750in}{2.172222in}}%
\pgfusepath{clip}%
\pgfsetbuttcap%
\pgfsetmiterjoin%
\definecolor{currentfill}{rgb}{0.447059,0.447059,0.447059}%
\pgfsetfillcolor{currentfill}%
\pgfsetlinewidth{1.003750pt}%
\definecolor{currentstroke}{rgb}{0.266667,0.266667,0.266667}%
\pgfsetstrokecolor{currentstroke}%
\pgfsetdash{}{0pt}%
\pgfpathmoveto{\pgfqpoint{0.973011in}{2.551856in}}%
\pgfpathlineto{\pgfqpoint{1.223914in}{2.551856in}}%
\pgfpathlineto{\pgfqpoint{1.223914in}{2.557982in}}%
\pgfpathlineto{\pgfqpoint{0.973011in}{2.557982in}}%
\pgfpathlineto{\pgfqpoint{0.973011in}{2.551856in}}%
\pgfpathclose%
\pgfusepath{stroke,fill}%
\end{pgfscope}%
\begin{pgfscope}%
\pgfpathrectangle{\pgfqpoint{0.781250in}{0.638889in}}{\pgfqpoint{4.218750in}{2.172222in}}%
\pgfusepath{clip}%
\pgfsetbuttcap%
\pgfsetmiterjoin%
\definecolor{currentfill}{rgb}{0.447059,0.447059,0.447059}%
\pgfsetfillcolor{currentfill}%
\pgfsetlinewidth{1.003750pt}%
\definecolor{currentstroke}{rgb}{0.266667,0.266667,0.266667}%
\pgfsetstrokecolor{currentstroke}%
\pgfsetdash{}{0pt}%
\pgfpathmoveto{\pgfqpoint{1.331444in}{2.123690in}}%
\pgfpathlineto{\pgfqpoint{1.582347in}{2.123690in}}%
\pgfpathlineto{\pgfqpoint{1.582347in}{2.128078in}}%
\pgfpathlineto{\pgfqpoint{1.331444in}{2.128078in}}%
\pgfpathlineto{\pgfqpoint{1.331444in}{2.123690in}}%
\pgfpathclose%
\pgfusepath{stroke,fill}%
\end{pgfscope}%
\begin{pgfscope}%
\pgfpathrectangle{\pgfqpoint{0.781250in}{0.638889in}}{\pgfqpoint{4.218750in}{2.172222in}}%
\pgfusepath{clip}%
\pgfsetbuttcap%
\pgfsetmiterjoin%
\definecolor{currentfill}{rgb}{0.447059,0.447059,0.447059}%
\pgfsetfillcolor{currentfill}%
\pgfsetlinewidth{1.003750pt}%
\definecolor{currentstroke}{rgb}{0.266667,0.266667,0.266667}%
\pgfsetstrokecolor{currentstroke}%
\pgfsetdash{}{0pt}%
\pgfpathmoveto{\pgfqpoint{1.689876in}{1.943982in}}%
\pgfpathlineto{\pgfqpoint{1.940779in}{1.943982in}}%
\pgfpathlineto{\pgfqpoint{1.940779in}{1.950020in}}%
\pgfpathlineto{\pgfqpoint{1.689876in}{1.950020in}}%
\pgfpathlineto{\pgfqpoint{1.689876in}{1.943982in}}%
\pgfpathclose%
\pgfusepath{stroke,fill}%
\end{pgfscope}%
\begin{pgfscope}%
\pgfpathrectangle{\pgfqpoint{0.781250in}{0.638889in}}{\pgfqpoint{4.218750in}{2.172222in}}%
\pgfusepath{clip}%
\pgfsetbuttcap%
\pgfsetmiterjoin%
\definecolor{currentfill}{rgb}{0.447059,0.447059,0.447059}%
\pgfsetfillcolor{currentfill}%
\pgfsetlinewidth{1.003750pt}%
\definecolor{currentstroke}{rgb}{0.266667,0.266667,0.266667}%
\pgfsetstrokecolor{currentstroke}%
\pgfsetdash{}{0pt}%
\pgfpathmoveto{\pgfqpoint{2.048309in}{1.880879in}}%
\pgfpathlineto{\pgfqpoint{2.299211in}{1.880879in}}%
\pgfpathlineto{\pgfqpoint{2.299211in}{1.884941in}}%
\pgfpathlineto{\pgfqpoint{2.048309in}{1.884941in}}%
\pgfpathlineto{\pgfqpoint{2.048309in}{1.880879in}}%
\pgfpathclose%
\pgfusepath{stroke,fill}%
\end{pgfscope}%
\begin{pgfscope}%
\pgfpathrectangle{\pgfqpoint{0.781250in}{0.638889in}}{\pgfqpoint{4.218750in}{2.172222in}}%
\pgfusepath{clip}%
\pgfsetbuttcap%
\pgfsetmiterjoin%
\definecolor{currentfill}{rgb}{0.447059,0.447059,0.447059}%
\pgfsetfillcolor{currentfill}%
\pgfsetlinewidth{1.003750pt}%
\definecolor{currentstroke}{rgb}{0.266667,0.266667,0.266667}%
\pgfsetstrokecolor{currentstroke}%
\pgfsetdash{}{0pt}%
\pgfpathmoveto{\pgfqpoint{2.406741in}{1.985058in}}%
\pgfpathlineto{\pgfqpoint{2.657644in}{1.985058in}}%
\pgfpathlineto{\pgfqpoint{2.657644in}{1.988708in}}%
\pgfpathlineto{\pgfqpoint{2.406741in}{1.988708in}}%
\pgfpathlineto{\pgfqpoint{2.406741in}{1.985058in}}%
\pgfpathclose%
\pgfusepath{stroke,fill}%
\end{pgfscope}%
\begin{pgfscope}%
\pgfpathrectangle{\pgfqpoint{0.781250in}{0.638889in}}{\pgfqpoint{4.218750in}{2.172222in}}%
\pgfusepath{clip}%
\pgfsetbuttcap%
\pgfsetmiterjoin%
\definecolor{currentfill}{rgb}{0.447059,0.447059,0.447059}%
\pgfsetfillcolor{currentfill}%
\pgfsetlinewidth{1.003750pt}%
\definecolor{currentstroke}{rgb}{0.266667,0.266667,0.266667}%
\pgfsetstrokecolor{currentstroke}%
\pgfsetdash{}{0pt}%
\pgfpathmoveto{\pgfqpoint{2.765174in}{2.175562in}}%
\pgfpathlineto{\pgfqpoint{3.016076in}{2.175562in}}%
\pgfpathlineto{\pgfqpoint{3.016076in}{2.179603in}}%
\pgfpathlineto{\pgfqpoint{2.765174in}{2.179603in}}%
\pgfpathlineto{\pgfqpoint{2.765174in}{2.175562in}}%
\pgfpathclose%
\pgfusepath{stroke,fill}%
\end{pgfscope}%
\begin{pgfscope}%
\pgfpathrectangle{\pgfqpoint{0.781250in}{0.638889in}}{\pgfqpoint{4.218750in}{2.172222in}}%
\pgfusepath{clip}%
\pgfsetbuttcap%
\pgfsetmiterjoin%
\definecolor{currentfill}{rgb}{0.447059,0.447059,0.447059}%
\pgfsetfillcolor{currentfill}%
\pgfsetlinewidth{1.003750pt}%
\definecolor{currentstroke}{rgb}{0.266667,0.266667,0.266667}%
\pgfsetstrokecolor{currentstroke}%
\pgfsetdash{}{0pt}%
\pgfpathmoveto{\pgfqpoint{3.123606in}{2.070861in}}%
\pgfpathlineto{\pgfqpoint{3.374509in}{2.070861in}}%
\pgfpathlineto{\pgfqpoint{3.374509in}{2.074424in}}%
\pgfpathlineto{\pgfqpoint{3.123606in}{2.074424in}}%
\pgfpathlineto{\pgfqpoint{3.123606in}{2.070861in}}%
\pgfpathclose%
\pgfusepath{stroke,fill}%
\end{pgfscope}%
\begin{pgfscope}%
\pgfpathrectangle{\pgfqpoint{0.781250in}{0.638889in}}{\pgfqpoint{4.218750in}{2.172222in}}%
\pgfusepath{clip}%
\pgfsetbuttcap%
\pgfsetmiterjoin%
\definecolor{currentfill}{rgb}{0.447059,0.447059,0.447059}%
\pgfsetfillcolor{currentfill}%
\pgfsetlinewidth{1.003750pt}%
\definecolor{currentstroke}{rgb}{0.266667,0.266667,0.266667}%
\pgfsetstrokecolor{currentstroke}%
\pgfsetdash{}{0pt}%
\pgfpathmoveto{\pgfqpoint{3.482039in}{2.039342in}}%
\pgfpathlineto{\pgfqpoint{3.732941in}{2.039342in}}%
\pgfpathlineto{\pgfqpoint{3.732941in}{2.042362in}}%
\pgfpathlineto{\pgfqpoint{3.482039in}{2.042362in}}%
\pgfpathlineto{\pgfqpoint{3.482039in}{2.039342in}}%
\pgfpathclose%
\pgfusepath{stroke,fill}%
\end{pgfscope}%
\begin{pgfscope}%
\pgfpathrectangle{\pgfqpoint{0.781250in}{0.638889in}}{\pgfqpoint{4.218750in}{2.172222in}}%
\pgfusepath{clip}%
\pgfsetbuttcap%
\pgfsetmiterjoin%
\definecolor{currentfill}{rgb}{0.447059,0.447059,0.447059}%
\pgfsetfillcolor{currentfill}%
\pgfsetlinewidth{1.003750pt}%
\definecolor{currentstroke}{rgb}{0.266667,0.266667,0.266667}%
\pgfsetstrokecolor{currentstroke}%
\pgfsetdash{}{0pt}%
\pgfpathmoveto{\pgfqpoint{3.840471in}{2.264645in}}%
\pgfpathlineto{\pgfqpoint{4.091374in}{2.264645in}}%
\pgfpathlineto{\pgfqpoint{4.091374in}{2.268599in}}%
\pgfpathlineto{\pgfqpoint{3.840471in}{2.268599in}}%
\pgfpathlineto{\pgfqpoint{3.840471in}{2.264645in}}%
\pgfpathclose%
\pgfusepath{stroke,fill}%
\end{pgfscope}%
\begin{pgfscope}%
\pgfpathrectangle{\pgfqpoint{0.781250in}{0.638889in}}{\pgfqpoint{4.218750in}{2.172222in}}%
\pgfusepath{clip}%
\pgfsetbuttcap%
\pgfsetmiterjoin%
\definecolor{currentfill}{rgb}{0.447059,0.447059,0.447059}%
\pgfsetfillcolor{currentfill}%
\pgfsetlinewidth{1.003750pt}%
\definecolor{currentstroke}{rgb}{0.266667,0.266667,0.266667}%
\pgfsetstrokecolor{currentstroke}%
\pgfsetdash{}{0pt}%
\pgfpathmoveto{\pgfqpoint{4.198903in}{2.304071in}}%
\pgfpathlineto{\pgfqpoint{4.449806in}{2.304071in}}%
\pgfpathlineto{\pgfqpoint{4.449806in}{2.310305in}}%
\pgfpathlineto{\pgfqpoint{4.198903in}{2.310305in}}%
\pgfpathlineto{\pgfqpoint{4.198903in}{2.304071in}}%
\pgfpathclose%
\pgfusepath{stroke,fill}%
\end{pgfscope}%
\begin{pgfscope}%
\pgfpathrectangle{\pgfqpoint{0.781250in}{0.638889in}}{\pgfqpoint{4.218750in}{2.172222in}}%
\pgfusepath{clip}%
\pgfsetbuttcap%
\pgfsetmiterjoin%
\definecolor{currentfill}{rgb}{0.447059,0.447059,0.447059}%
\pgfsetfillcolor{currentfill}%
\pgfsetlinewidth{1.003750pt}%
\definecolor{currentstroke}{rgb}{0.266667,0.266667,0.266667}%
\pgfsetstrokecolor{currentstroke}%
\pgfsetdash{}{0pt}%
\pgfpathmoveto{\pgfqpoint{4.557336in}{2.453172in}}%
\pgfpathlineto{\pgfqpoint{4.808239in}{2.453172in}}%
\pgfpathlineto{\pgfqpoint{4.808239in}{2.461057in}}%
\pgfpathlineto{\pgfqpoint{4.557336in}{2.461057in}}%
\pgfpathlineto{\pgfqpoint{4.557336in}{2.453172in}}%
\pgfpathclose%
\pgfusepath{stroke,fill}%
\end{pgfscope}%
\begin{pgfscope}%
\pgfpathrectangle{\pgfqpoint{0.781250in}{0.638889in}}{\pgfqpoint{4.218750in}{2.172222in}}%
\pgfusepath{clip}%
\pgfsetbuttcap%
\pgfsetmiterjoin%
\definecolor{currentfill}{rgb}{0.447059,0.447059,0.447059}%
\pgfsetfillcolor{currentfill}%
\pgfsetlinewidth{1.003750pt}%
\definecolor{currentstroke}{rgb}{0.266667,0.266667,0.266667}%
\pgfsetstrokecolor{currentstroke}%
\pgfsetdash{}{0pt}%
\pgfpathmoveto{\pgfqpoint{0.973011in}{2.557982in}}%
\pgfpathlineto{\pgfqpoint{1.223914in}{2.557982in}}%
\pgfpathlineto{\pgfqpoint{1.223914in}{2.563630in}}%
\pgfpathlineto{\pgfqpoint{0.973011in}{2.563630in}}%
\pgfpathlineto{\pgfqpoint{0.973011in}{2.557982in}}%
\pgfpathclose%
\pgfusepath{stroke,fill}%
\end{pgfscope}%
\begin{pgfscope}%
\pgfpathrectangle{\pgfqpoint{0.781250in}{0.638889in}}{\pgfqpoint{4.218750in}{2.172222in}}%
\pgfusepath{clip}%
\pgfsetbuttcap%
\pgfsetmiterjoin%
\definecolor{currentfill}{rgb}{0.447059,0.447059,0.447059}%
\pgfsetfillcolor{currentfill}%
\pgfsetlinewidth{1.003750pt}%
\definecolor{currentstroke}{rgb}{0.266667,0.266667,0.266667}%
\pgfsetstrokecolor{currentstroke}%
\pgfsetdash{}{0pt}%
\pgfpathmoveto{\pgfqpoint{1.331444in}{2.128078in}}%
\pgfpathlineto{\pgfqpoint{1.582347in}{2.128078in}}%
\pgfpathlineto{\pgfqpoint{1.582347in}{2.133008in}}%
\pgfpathlineto{\pgfqpoint{1.331444in}{2.133008in}}%
\pgfpathlineto{\pgfqpoint{1.331444in}{2.128078in}}%
\pgfpathclose%
\pgfusepath{stroke,fill}%
\end{pgfscope}%
\begin{pgfscope}%
\pgfpathrectangle{\pgfqpoint{0.781250in}{0.638889in}}{\pgfqpoint{4.218750in}{2.172222in}}%
\pgfusepath{clip}%
\pgfsetbuttcap%
\pgfsetmiterjoin%
\definecolor{currentfill}{rgb}{0.447059,0.447059,0.447059}%
\pgfsetfillcolor{currentfill}%
\pgfsetlinewidth{1.003750pt}%
\definecolor{currentstroke}{rgb}{0.266667,0.266667,0.266667}%
\pgfsetstrokecolor{currentstroke}%
\pgfsetdash{}{0pt}%
\pgfpathmoveto{\pgfqpoint{1.689876in}{1.950020in}}%
\pgfpathlineto{\pgfqpoint{1.940779in}{1.950020in}}%
\pgfpathlineto{\pgfqpoint{1.940779in}{1.955169in}}%
\pgfpathlineto{\pgfqpoint{1.689876in}{1.955169in}}%
\pgfpathlineto{\pgfqpoint{1.689876in}{1.950020in}}%
\pgfpathclose%
\pgfusepath{stroke,fill}%
\end{pgfscope}%
\begin{pgfscope}%
\pgfpathrectangle{\pgfqpoint{0.781250in}{0.638889in}}{\pgfqpoint{4.218750in}{2.172222in}}%
\pgfusepath{clip}%
\pgfsetbuttcap%
\pgfsetmiterjoin%
\definecolor{currentfill}{rgb}{0.447059,0.447059,0.447059}%
\pgfsetfillcolor{currentfill}%
\pgfsetlinewidth{1.003750pt}%
\definecolor{currentstroke}{rgb}{0.266667,0.266667,0.266667}%
\pgfsetstrokecolor{currentstroke}%
\pgfsetdash{}{0pt}%
\pgfpathmoveto{\pgfqpoint{2.048309in}{1.884941in}}%
\pgfpathlineto{\pgfqpoint{2.299211in}{1.884941in}}%
\pgfpathlineto{\pgfqpoint{2.299211in}{1.889633in}}%
\pgfpathlineto{\pgfqpoint{2.048309in}{1.889633in}}%
\pgfpathlineto{\pgfqpoint{2.048309in}{1.884941in}}%
\pgfpathclose%
\pgfusepath{stroke,fill}%
\end{pgfscope}%
\begin{pgfscope}%
\pgfpathrectangle{\pgfqpoint{0.781250in}{0.638889in}}{\pgfqpoint{4.218750in}{2.172222in}}%
\pgfusepath{clip}%
\pgfsetbuttcap%
\pgfsetmiterjoin%
\definecolor{currentfill}{rgb}{0.447059,0.447059,0.447059}%
\pgfsetfillcolor{currentfill}%
\pgfsetlinewidth{1.003750pt}%
\definecolor{currentstroke}{rgb}{0.266667,0.266667,0.266667}%
\pgfsetstrokecolor{currentstroke}%
\pgfsetdash{}{0pt}%
\pgfpathmoveto{\pgfqpoint{2.406741in}{1.988708in}}%
\pgfpathlineto{\pgfqpoint{2.657644in}{1.988708in}}%
\pgfpathlineto{\pgfqpoint{2.657644in}{1.993791in}}%
\pgfpathlineto{\pgfqpoint{2.406741in}{1.993791in}}%
\pgfpathlineto{\pgfqpoint{2.406741in}{1.988708in}}%
\pgfpathclose%
\pgfusepath{stroke,fill}%
\end{pgfscope}%
\begin{pgfscope}%
\pgfpathrectangle{\pgfqpoint{0.781250in}{0.638889in}}{\pgfqpoint{4.218750in}{2.172222in}}%
\pgfusepath{clip}%
\pgfsetbuttcap%
\pgfsetmiterjoin%
\definecolor{currentfill}{rgb}{0.447059,0.447059,0.447059}%
\pgfsetfillcolor{currentfill}%
\pgfsetlinewidth{1.003750pt}%
\definecolor{currentstroke}{rgb}{0.266667,0.266667,0.266667}%
\pgfsetstrokecolor{currentstroke}%
\pgfsetdash{}{0pt}%
\pgfpathmoveto{\pgfqpoint{2.765174in}{2.179603in}}%
\pgfpathlineto{\pgfqpoint{3.016076in}{2.179603in}}%
\pgfpathlineto{\pgfqpoint{3.016076in}{2.183882in}}%
\pgfpathlineto{\pgfqpoint{2.765174in}{2.183882in}}%
\pgfpathlineto{\pgfqpoint{2.765174in}{2.179603in}}%
\pgfpathclose%
\pgfusepath{stroke,fill}%
\end{pgfscope}%
\begin{pgfscope}%
\pgfpathrectangle{\pgfqpoint{0.781250in}{0.638889in}}{\pgfqpoint{4.218750in}{2.172222in}}%
\pgfusepath{clip}%
\pgfsetbuttcap%
\pgfsetmiterjoin%
\definecolor{currentfill}{rgb}{0.447059,0.447059,0.447059}%
\pgfsetfillcolor{currentfill}%
\pgfsetlinewidth{1.003750pt}%
\definecolor{currentstroke}{rgb}{0.266667,0.266667,0.266667}%
\pgfsetstrokecolor{currentstroke}%
\pgfsetdash{}{0pt}%
\pgfpathmoveto{\pgfqpoint{3.123606in}{2.074424in}}%
\pgfpathlineto{\pgfqpoint{3.374509in}{2.074424in}}%
\pgfpathlineto{\pgfqpoint{3.374509in}{2.078355in}}%
\pgfpathlineto{\pgfqpoint{3.123606in}{2.078355in}}%
\pgfpathlineto{\pgfqpoint{3.123606in}{2.074424in}}%
\pgfpathclose%
\pgfusepath{stroke,fill}%
\end{pgfscope}%
\begin{pgfscope}%
\pgfpathrectangle{\pgfqpoint{0.781250in}{0.638889in}}{\pgfqpoint{4.218750in}{2.172222in}}%
\pgfusepath{clip}%
\pgfsetbuttcap%
\pgfsetmiterjoin%
\definecolor{currentfill}{rgb}{0.447059,0.447059,0.447059}%
\pgfsetfillcolor{currentfill}%
\pgfsetlinewidth{1.003750pt}%
\definecolor{currentstroke}{rgb}{0.266667,0.266667,0.266667}%
\pgfsetstrokecolor{currentstroke}%
\pgfsetdash{}{0pt}%
\pgfpathmoveto{\pgfqpoint{3.482039in}{2.042362in}}%
\pgfpathlineto{\pgfqpoint{3.732941in}{2.042362in}}%
\pgfpathlineto{\pgfqpoint{3.732941in}{2.046098in}}%
\pgfpathlineto{\pgfqpoint{3.482039in}{2.046098in}}%
\pgfpathlineto{\pgfqpoint{3.482039in}{2.042362in}}%
\pgfpathclose%
\pgfusepath{stroke,fill}%
\end{pgfscope}%
\begin{pgfscope}%
\pgfpathrectangle{\pgfqpoint{0.781250in}{0.638889in}}{\pgfqpoint{4.218750in}{2.172222in}}%
\pgfusepath{clip}%
\pgfsetbuttcap%
\pgfsetmiterjoin%
\definecolor{currentfill}{rgb}{0.447059,0.447059,0.447059}%
\pgfsetfillcolor{currentfill}%
\pgfsetlinewidth{1.003750pt}%
\definecolor{currentstroke}{rgb}{0.266667,0.266667,0.266667}%
\pgfsetstrokecolor{currentstroke}%
\pgfsetdash{}{0pt}%
\pgfpathmoveto{\pgfqpoint{3.840471in}{2.268599in}}%
\pgfpathlineto{\pgfqpoint{4.091374in}{2.268599in}}%
\pgfpathlineto{\pgfqpoint{4.091374in}{2.272617in}}%
\pgfpathlineto{\pgfqpoint{3.840471in}{2.272617in}}%
\pgfpathlineto{\pgfqpoint{3.840471in}{2.268599in}}%
\pgfpathclose%
\pgfusepath{stroke,fill}%
\end{pgfscope}%
\begin{pgfscope}%
\pgfpathrectangle{\pgfqpoint{0.781250in}{0.638889in}}{\pgfqpoint{4.218750in}{2.172222in}}%
\pgfusepath{clip}%
\pgfsetbuttcap%
\pgfsetmiterjoin%
\definecolor{currentfill}{rgb}{0.447059,0.447059,0.447059}%
\pgfsetfillcolor{currentfill}%
\pgfsetlinewidth{1.003750pt}%
\definecolor{currentstroke}{rgb}{0.266667,0.266667,0.266667}%
\pgfsetstrokecolor{currentstroke}%
\pgfsetdash{}{0pt}%
\pgfpathmoveto{\pgfqpoint{4.198903in}{2.310305in}}%
\pgfpathlineto{\pgfqpoint{4.449806in}{2.310305in}}%
\pgfpathlineto{\pgfqpoint{4.449806in}{2.313998in}}%
\pgfpathlineto{\pgfqpoint{4.198903in}{2.313998in}}%
\pgfpathlineto{\pgfqpoint{4.198903in}{2.310305in}}%
\pgfpathclose%
\pgfusepath{stroke,fill}%
\end{pgfscope}%
\begin{pgfscope}%
\pgfpathrectangle{\pgfqpoint{0.781250in}{0.638889in}}{\pgfqpoint{4.218750in}{2.172222in}}%
\pgfusepath{clip}%
\pgfsetbuttcap%
\pgfsetmiterjoin%
\definecolor{currentfill}{rgb}{0.447059,0.447059,0.447059}%
\pgfsetfillcolor{currentfill}%
\pgfsetlinewidth{1.003750pt}%
\definecolor{currentstroke}{rgb}{0.266667,0.266667,0.266667}%
\pgfsetstrokecolor{currentstroke}%
\pgfsetdash{}{0pt}%
\pgfpathmoveto{\pgfqpoint{4.557336in}{2.461057in}}%
\pgfpathlineto{\pgfqpoint{4.808239in}{2.461057in}}%
\pgfpathlineto{\pgfqpoint{4.808239in}{2.464620in}}%
\pgfpathlineto{\pgfqpoint{4.557336in}{2.464620in}}%
\pgfpathlineto{\pgfqpoint{4.557336in}{2.461057in}}%
\pgfpathclose%
\pgfusepath{stroke,fill}%
\end{pgfscope}%
\begin{pgfscope}%
\pgfpathrectangle{\pgfqpoint{0.781250in}{0.638889in}}{\pgfqpoint{4.218750in}{2.172222in}}%
\pgfusepath{clip}%
\pgfsetbuttcap%
\pgfsetmiterjoin%
\definecolor{currentfill}{rgb}{0.447059,0.447059,0.447059}%
\pgfsetfillcolor{currentfill}%
\pgfsetlinewidth{1.003750pt}%
\definecolor{currentstroke}{rgb}{0.266667,0.266667,0.266667}%
\pgfsetstrokecolor{currentstroke}%
\pgfsetdash{}{0pt}%
\pgfpathmoveto{\pgfqpoint{0.973011in}{2.563630in}}%
\pgfpathlineto{\pgfqpoint{1.223914in}{2.563630in}}%
\pgfpathlineto{\pgfqpoint{1.223914in}{2.563630in}}%
\pgfpathlineto{\pgfqpoint{0.973011in}{2.563630in}}%
\pgfpathlineto{\pgfqpoint{0.973011in}{2.563630in}}%
\pgfpathclose%
\pgfusepath{stroke,fill}%
\end{pgfscope}%
\begin{pgfscope}%
\pgfpathrectangle{\pgfqpoint{0.781250in}{0.638889in}}{\pgfqpoint{4.218750in}{2.172222in}}%
\pgfusepath{clip}%
\pgfsetbuttcap%
\pgfsetmiterjoin%
\definecolor{currentfill}{rgb}{0.447059,0.447059,0.447059}%
\pgfsetfillcolor{currentfill}%
\pgfsetlinewidth{1.003750pt}%
\definecolor{currentstroke}{rgb}{0.266667,0.266667,0.266667}%
\pgfsetstrokecolor{currentstroke}%
\pgfsetdash{}{0pt}%
\pgfpathmoveto{\pgfqpoint{1.331444in}{2.133009in}}%
\pgfpathlineto{\pgfqpoint{1.582347in}{2.133009in}}%
\pgfpathlineto{\pgfqpoint{1.582347in}{2.133009in}}%
\pgfpathlineto{\pgfqpoint{1.331444in}{2.133009in}}%
\pgfpathlineto{\pgfqpoint{1.331444in}{2.133009in}}%
\pgfpathclose%
\pgfusepath{stroke,fill}%
\end{pgfscope}%
\begin{pgfscope}%
\pgfpathrectangle{\pgfqpoint{0.781250in}{0.638889in}}{\pgfqpoint{4.218750in}{2.172222in}}%
\pgfusepath{clip}%
\pgfsetbuttcap%
\pgfsetmiterjoin%
\definecolor{currentfill}{rgb}{0.447059,0.447059,0.447059}%
\pgfsetfillcolor{currentfill}%
\pgfsetlinewidth{1.003750pt}%
\definecolor{currentstroke}{rgb}{0.266667,0.266667,0.266667}%
\pgfsetstrokecolor{currentstroke}%
\pgfsetdash{}{0pt}%
\pgfpathmoveto{\pgfqpoint{1.689876in}{1.955169in}}%
\pgfpathlineto{\pgfqpoint{1.940779in}{1.955169in}}%
\pgfpathlineto{\pgfqpoint{1.940779in}{1.958731in}}%
\pgfpathlineto{\pgfqpoint{1.689876in}{1.958731in}}%
\pgfpathlineto{\pgfqpoint{1.689876in}{1.955169in}}%
\pgfpathclose%
\pgfusepath{stroke,fill}%
\end{pgfscope}%
\begin{pgfscope}%
\pgfpathrectangle{\pgfqpoint{0.781250in}{0.638889in}}{\pgfqpoint{4.218750in}{2.172222in}}%
\pgfusepath{clip}%
\pgfsetbuttcap%
\pgfsetmiterjoin%
\definecolor{currentfill}{rgb}{0.447059,0.447059,0.447059}%
\pgfsetfillcolor{currentfill}%
\pgfsetlinewidth{1.003750pt}%
\definecolor{currentstroke}{rgb}{0.266667,0.266667,0.266667}%
\pgfsetstrokecolor{currentstroke}%
\pgfsetdash{}{0pt}%
\pgfpathmoveto{\pgfqpoint{2.048309in}{1.889633in}}%
\pgfpathlineto{\pgfqpoint{2.299211in}{1.889633in}}%
\pgfpathlineto{\pgfqpoint{2.299211in}{1.893000in}}%
\pgfpathlineto{\pgfqpoint{2.048309in}{1.893000in}}%
\pgfpathlineto{\pgfqpoint{2.048309in}{1.889633in}}%
\pgfpathclose%
\pgfusepath{stroke,fill}%
\end{pgfscope}%
\begin{pgfscope}%
\pgfpathrectangle{\pgfqpoint{0.781250in}{0.638889in}}{\pgfqpoint{4.218750in}{2.172222in}}%
\pgfusepath{clip}%
\pgfsetbuttcap%
\pgfsetmiterjoin%
\definecolor{currentfill}{rgb}{0.447059,0.447059,0.447059}%
\pgfsetfillcolor{currentfill}%
\pgfsetlinewidth{1.003750pt}%
\definecolor{currentstroke}{rgb}{0.266667,0.266667,0.266667}%
\pgfsetstrokecolor{currentstroke}%
\pgfsetdash{}{0pt}%
\pgfpathmoveto{\pgfqpoint{2.406741in}{1.993791in}}%
\pgfpathlineto{\pgfqpoint{2.657644in}{1.993791in}}%
\pgfpathlineto{\pgfqpoint{2.657644in}{1.996376in}}%
\pgfpathlineto{\pgfqpoint{2.406741in}{1.996376in}}%
\pgfpathlineto{\pgfqpoint{2.406741in}{1.993791in}}%
\pgfpathclose%
\pgfusepath{stroke,fill}%
\end{pgfscope}%
\begin{pgfscope}%
\pgfpathrectangle{\pgfqpoint{0.781250in}{0.638889in}}{\pgfqpoint{4.218750in}{2.172222in}}%
\pgfusepath{clip}%
\pgfsetbuttcap%
\pgfsetmiterjoin%
\definecolor{currentfill}{rgb}{0.447059,0.447059,0.447059}%
\pgfsetfillcolor{currentfill}%
\pgfsetlinewidth{1.003750pt}%
\definecolor{currentstroke}{rgb}{0.266667,0.266667,0.266667}%
\pgfsetstrokecolor{currentstroke}%
\pgfsetdash{}{0pt}%
\pgfpathmoveto{\pgfqpoint{2.765174in}{2.183882in}}%
\pgfpathlineto{\pgfqpoint{3.016076in}{2.183882in}}%
\pgfpathlineto{\pgfqpoint{3.016076in}{2.189182in}}%
\pgfpathlineto{\pgfqpoint{2.765174in}{2.189182in}}%
\pgfpathlineto{\pgfqpoint{2.765174in}{2.183882in}}%
\pgfpathclose%
\pgfusepath{stroke,fill}%
\end{pgfscope}%
\begin{pgfscope}%
\pgfpathrectangle{\pgfqpoint{0.781250in}{0.638889in}}{\pgfqpoint{4.218750in}{2.172222in}}%
\pgfusepath{clip}%
\pgfsetbuttcap%
\pgfsetmiterjoin%
\definecolor{currentfill}{rgb}{0.447059,0.447059,0.447059}%
\pgfsetfillcolor{currentfill}%
\pgfsetlinewidth{1.003750pt}%
\definecolor{currentstroke}{rgb}{0.266667,0.266667,0.266667}%
\pgfsetstrokecolor{currentstroke}%
\pgfsetdash{}{0pt}%
\pgfpathmoveto{\pgfqpoint{3.123606in}{2.078355in}}%
\pgfpathlineto{\pgfqpoint{3.374509in}{2.078355in}}%
\pgfpathlineto{\pgfqpoint{3.374509in}{2.082874in}}%
\pgfpathlineto{\pgfqpoint{3.123606in}{2.082874in}}%
\pgfpathlineto{\pgfqpoint{3.123606in}{2.078355in}}%
\pgfpathclose%
\pgfusepath{stroke,fill}%
\end{pgfscope}%
\begin{pgfscope}%
\pgfpathrectangle{\pgfqpoint{0.781250in}{0.638889in}}{\pgfqpoint{4.218750in}{2.172222in}}%
\pgfusepath{clip}%
\pgfsetbuttcap%
\pgfsetmiterjoin%
\definecolor{currentfill}{rgb}{0.447059,0.447059,0.447059}%
\pgfsetfillcolor{currentfill}%
\pgfsetlinewidth{1.003750pt}%
\definecolor{currentstroke}{rgb}{0.266667,0.266667,0.266667}%
\pgfsetstrokecolor{currentstroke}%
\pgfsetdash{}{0pt}%
\pgfpathmoveto{\pgfqpoint{3.482039in}{2.046098in}}%
\pgfpathlineto{\pgfqpoint{3.732941in}{2.046098in}}%
\pgfpathlineto{\pgfqpoint{3.732941in}{2.050247in}}%
\pgfpathlineto{\pgfqpoint{3.482039in}{2.050247in}}%
\pgfpathlineto{\pgfqpoint{3.482039in}{2.046098in}}%
\pgfpathclose%
\pgfusepath{stroke,fill}%
\end{pgfscope}%
\begin{pgfscope}%
\pgfpathrectangle{\pgfqpoint{0.781250in}{0.638889in}}{\pgfqpoint{4.218750in}{2.172222in}}%
\pgfusepath{clip}%
\pgfsetbuttcap%
\pgfsetmiterjoin%
\definecolor{currentfill}{rgb}{0.447059,0.447059,0.447059}%
\pgfsetfillcolor{currentfill}%
\pgfsetlinewidth{1.003750pt}%
\definecolor{currentstroke}{rgb}{0.266667,0.266667,0.266667}%
\pgfsetstrokecolor{currentstroke}%
\pgfsetdash{}{0pt}%
\pgfpathmoveto{\pgfqpoint{3.840471in}{2.272617in}}%
\pgfpathlineto{\pgfqpoint{4.091374in}{2.272617in}}%
\pgfpathlineto{\pgfqpoint{4.091374in}{2.276158in}}%
\pgfpathlineto{\pgfqpoint{3.840471in}{2.276158in}}%
\pgfpathlineto{\pgfqpoint{3.840471in}{2.272617in}}%
\pgfpathclose%
\pgfusepath{stroke,fill}%
\end{pgfscope}%
\begin{pgfscope}%
\pgfpathrectangle{\pgfqpoint{0.781250in}{0.638889in}}{\pgfqpoint{4.218750in}{2.172222in}}%
\pgfusepath{clip}%
\pgfsetbuttcap%
\pgfsetmiterjoin%
\definecolor{currentfill}{rgb}{0.447059,0.447059,0.447059}%
\pgfsetfillcolor{currentfill}%
\pgfsetlinewidth{1.003750pt}%
\definecolor{currentstroke}{rgb}{0.266667,0.266667,0.266667}%
\pgfsetstrokecolor{currentstroke}%
\pgfsetdash{}{0pt}%
\pgfpathmoveto{\pgfqpoint{4.198903in}{2.313998in}}%
\pgfpathlineto{\pgfqpoint{4.449806in}{2.313998in}}%
\pgfpathlineto{\pgfqpoint{4.449806in}{2.317626in}}%
\pgfpathlineto{\pgfqpoint{4.198903in}{2.317626in}}%
\pgfpathlineto{\pgfqpoint{4.198903in}{2.313998in}}%
\pgfpathclose%
\pgfusepath{stroke,fill}%
\end{pgfscope}%
\begin{pgfscope}%
\pgfpathrectangle{\pgfqpoint{0.781250in}{0.638889in}}{\pgfqpoint{4.218750in}{2.172222in}}%
\pgfusepath{clip}%
\pgfsetbuttcap%
\pgfsetmiterjoin%
\definecolor{currentfill}{rgb}{0.447059,0.447059,0.447059}%
\pgfsetfillcolor{currentfill}%
\pgfsetlinewidth{1.003750pt}%
\definecolor{currentstroke}{rgb}{0.266667,0.266667,0.266667}%
\pgfsetstrokecolor{currentstroke}%
\pgfsetdash{}{0pt}%
\pgfpathmoveto{\pgfqpoint{4.557336in}{2.464620in}}%
\pgfpathlineto{\pgfqpoint{4.808239in}{2.464620in}}%
\pgfpathlineto{\pgfqpoint{4.808239in}{2.469073in}}%
\pgfpathlineto{\pgfqpoint{4.557336in}{2.469073in}}%
\pgfpathlineto{\pgfqpoint{4.557336in}{2.464620in}}%
\pgfpathclose%
\pgfusepath{stroke,fill}%
\end{pgfscope}%
\begin{pgfscope}%
\pgfpathrectangle{\pgfqpoint{0.781250in}{0.638889in}}{\pgfqpoint{4.218750in}{2.172222in}}%
\pgfusepath{clip}%
\pgfsetbuttcap%
\pgfsetmiterjoin%
\definecolor{currentfill}{rgb}{0.447059,0.447059,0.447059}%
\pgfsetfillcolor{currentfill}%
\pgfsetlinewidth{1.003750pt}%
\definecolor{currentstroke}{rgb}{0.266667,0.266667,0.266667}%
\pgfsetstrokecolor{currentstroke}%
\pgfsetdash{}{0pt}%
\pgfpathmoveto{\pgfqpoint{0.973011in}{2.563630in}}%
\pgfpathlineto{\pgfqpoint{1.223914in}{2.563630in}}%
\pgfpathlineto{\pgfqpoint{1.223914in}{2.569669in}}%
\pgfpathlineto{\pgfqpoint{0.973011in}{2.569669in}}%
\pgfpathlineto{\pgfqpoint{0.973011in}{2.563630in}}%
\pgfpathclose%
\pgfusepath{stroke,fill}%
\end{pgfscope}%
\begin{pgfscope}%
\pgfpathrectangle{\pgfqpoint{0.781250in}{0.638889in}}{\pgfqpoint{4.218750in}{2.172222in}}%
\pgfusepath{clip}%
\pgfsetbuttcap%
\pgfsetmiterjoin%
\definecolor{currentfill}{rgb}{0.447059,0.447059,0.447059}%
\pgfsetfillcolor{currentfill}%
\pgfsetlinewidth{1.003750pt}%
\definecolor{currentstroke}{rgb}{0.266667,0.266667,0.266667}%
\pgfsetstrokecolor{currentstroke}%
\pgfsetdash{}{0pt}%
\pgfpathmoveto{\pgfqpoint{1.331444in}{2.133009in}}%
\pgfpathlineto{\pgfqpoint{1.582347in}{2.133009in}}%
\pgfpathlineto{\pgfqpoint{1.582347in}{2.137874in}}%
\pgfpathlineto{\pgfqpoint{1.331444in}{2.137874in}}%
\pgfpathlineto{\pgfqpoint{1.331444in}{2.133009in}}%
\pgfpathclose%
\pgfusepath{stroke,fill}%
\end{pgfscope}%
\begin{pgfscope}%
\pgfpathrectangle{\pgfqpoint{0.781250in}{0.638889in}}{\pgfqpoint{4.218750in}{2.172222in}}%
\pgfusepath{clip}%
\pgfsetbuttcap%
\pgfsetmiterjoin%
\definecolor{currentfill}{rgb}{0.447059,0.447059,0.447059}%
\pgfsetfillcolor{currentfill}%
\pgfsetlinewidth{1.003750pt}%
\definecolor{currentstroke}{rgb}{0.266667,0.266667,0.266667}%
\pgfsetstrokecolor{currentstroke}%
\pgfsetdash{}{0pt}%
\pgfpathmoveto{\pgfqpoint{1.689876in}{1.958731in}}%
\pgfpathlineto{\pgfqpoint{1.940779in}{1.958731in}}%
\pgfpathlineto{\pgfqpoint{1.940779in}{1.961968in}}%
\pgfpathlineto{\pgfqpoint{1.689876in}{1.961968in}}%
\pgfpathlineto{\pgfqpoint{1.689876in}{1.958731in}}%
\pgfpathclose%
\pgfusepath{stroke,fill}%
\end{pgfscope}%
\begin{pgfscope}%
\pgfpathrectangle{\pgfqpoint{0.781250in}{0.638889in}}{\pgfqpoint{4.218750in}{2.172222in}}%
\pgfusepath{clip}%
\pgfsetbuttcap%
\pgfsetmiterjoin%
\definecolor{currentfill}{rgb}{0.447059,0.447059,0.447059}%
\pgfsetfillcolor{currentfill}%
\pgfsetlinewidth{1.003750pt}%
\definecolor{currentstroke}{rgb}{0.266667,0.266667,0.266667}%
\pgfsetstrokecolor{currentstroke}%
\pgfsetdash{}{0pt}%
\pgfpathmoveto{\pgfqpoint{2.048309in}{1.893000in}}%
\pgfpathlineto{\pgfqpoint{2.299211in}{1.893000in}}%
\pgfpathlineto{\pgfqpoint{2.299211in}{1.896236in}}%
\pgfpathlineto{\pgfqpoint{2.048309in}{1.896236in}}%
\pgfpathlineto{\pgfqpoint{2.048309in}{1.893000in}}%
\pgfpathclose%
\pgfusepath{stroke,fill}%
\end{pgfscope}%
\begin{pgfscope}%
\pgfpathrectangle{\pgfqpoint{0.781250in}{0.638889in}}{\pgfqpoint{4.218750in}{2.172222in}}%
\pgfusepath{clip}%
\pgfsetbuttcap%
\pgfsetmiterjoin%
\definecolor{currentfill}{rgb}{0.447059,0.447059,0.447059}%
\pgfsetfillcolor{currentfill}%
\pgfsetlinewidth{1.003750pt}%
\definecolor{currentstroke}{rgb}{0.266667,0.266667,0.266667}%
\pgfsetstrokecolor{currentstroke}%
\pgfsetdash{}{0pt}%
\pgfpathmoveto{\pgfqpoint{2.406741in}{1.996376in}}%
\pgfpathlineto{\pgfqpoint{2.657644in}{1.996376in}}%
\pgfpathlineto{\pgfqpoint{2.657644in}{1.999287in}}%
\pgfpathlineto{\pgfqpoint{2.406741in}{1.999287in}}%
\pgfpathlineto{\pgfqpoint{2.406741in}{1.996376in}}%
\pgfpathclose%
\pgfusepath{stroke,fill}%
\end{pgfscope}%
\begin{pgfscope}%
\pgfpathrectangle{\pgfqpoint{0.781250in}{0.638889in}}{\pgfqpoint{4.218750in}{2.172222in}}%
\pgfusepath{clip}%
\pgfsetbuttcap%
\pgfsetmiterjoin%
\definecolor{currentfill}{rgb}{0.447059,0.447059,0.447059}%
\pgfsetfillcolor{currentfill}%
\pgfsetlinewidth{1.003750pt}%
\definecolor{currentstroke}{rgb}{0.266667,0.266667,0.266667}%
\pgfsetstrokecolor{currentstroke}%
\pgfsetdash{}{0pt}%
\pgfpathmoveto{\pgfqpoint{2.765174in}{2.189182in}}%
\pgfpathlineto{\pgfqpoint{3.016076in}{2.189182in}}%
\pgfpathlineto{\pgfqpoint{3.016076in}{2.191897in}}%
\pgfpathlineto{\pgfqpoint{2.765174in}{2.191897in}}%
\pgfpathlineto{\pgfqpoint{2.765174in}{2.189182in}}%
\pgfpathclose%
\pgfusepath{stroke,fill}%
\end{pgfscope}%
\begin{pgfscope}%
\pgfpathrectangle{\pgfqpoint{0.781250in}{0.638889in}}{\pgfqpoint{4.218750in}{2.172222in}}%
\pgfusepath{clip}%
\pgfsetbuttcap%
\pgfsetmiterjoin%
\definecolor{currentfill}{rgb}{0.447059,0.447059,0.447059}%
\pgfsetfillcolor{currentfill}%
\pgfsetlinewidth{1.003750pt}%
\definecolor{currentstroke}{rgb}{0.266667,0.266667,0.266667}%
\pgfsetstrokecolor{currentstroke}%
\pgfsetdash{}{0pt}%
\pgfpathmoveto{\pgfqpoint{3.123606in}{2.082874in}}%
\pgfpathlineto{\pgfqpoint{3.374509in}{2.082874in}}%
\pgfpathlineto{\pgfqpoint{3.374509in}{2.085285in}}%
\pgfpathlineto{\pgfqpoint{3.123606in}{2.085285in}}%
\pgfpathlineto{\pgfqpoint{3.123606in}{2.082874in}}%
\pgfpathclose%
\pgfusepath{stroke,fill}%
\end{pgfscope}%
\begin{pgfscope}%
\pgfpathrectangle{\pgfqpoint{0.781250in}{0.638889in}}{\pgfqpoint{4.218750in}{2.172222in}}%
\pgfusepath{clip}%
\pgfsetbuttcap%
\pgfsetmiterjoin%
\definecolor{currentfill}{rgb}{0.447059,0.447059,0.447059}%
\pgfsetfillcolor{currentfill}%
\pgfsetlinewidth{1.003750pt}%
\definecolor{currentstroke}{rgb}{0.266667,0.266667,0.266667}%
\pgfsetstrokecolor{currentstroke}%
\pgfsetdash{}{0pt}%
\pgfpathmoveto{\pgfqpoint{3.482039in}{2.050247in}}%
\pgfpathlineto{\pgfqpoint{3.732941in}{2.050247in}}%
\pgfpathlineto{\pgfqpoint{3.732941in}{2.052767in}}%
\pgfpathlineto{\pgfqpoint{3.482039in}{2.052767in}}%
\pgfpathlineto{\pgfqpoint{3.482039in}{2.050247in}}%
\pgfpathclose%
\pgfusepath{stroke,fill}%
\end{pgfscope}%
\begin{pgfscope}%
\pgfpathrectangle{\pgfqpoint{0.781250in}{0.638889in}}{\pgfqpoint{4.218750in}{2.172222in}}%
\pgfusepath{clip}%
\pgfsetbuttcap%
\pgfsetmiterjoin%
\definecolor{currentfill}{rgb}{0.447059,0.447059,0.447059}%
\pgfsetfillcolor{currentfill}%
\pgfsetlinewidth{1.003750pt}%
\definecolor{currentstroke}{rgb}{0.266667,0.266667,0.266667}%
\pgfsetstrokecolor{currentstroke}%
\pgfsetdash{}{0pt}%
\pgfpathmoveto{\pgfqpoint{3.840471in}{2.276158in}}%
\pgfpathlineto{\pgfqpoint{4.091374in}{2.276158in}}%
\pgfpathlineto{\pgfqpoint{4.091374in}{2.278743in}}%
\pgfpathlineto{\pgfqpoint{3.840471in}{2.278743in}}%
\pgfpathlineto{\pgfqpoint{3.840471in}{2.276158in}}%
\pgfpathclose%
\pgfusepath{stroke,fill}%
\end{pgfscope}%
\begin{pgfscope}%
\pgfpathrectangle{\pgfqpoint{0.781250in}{0.638889in}}{\pgfqpoint{4.218750in}{2.172222in}}%
\pgfusepath{clip}%
\pgfsetbuttcap%
\pgfsetmiterjoin%
\definecolor{currentfill}{rgb}{0.447059,0.447059,0.447059}%
\pgfsetfillcolor{currentfill}%
\pgfsetlinewidth{1.003750pt}%
\definecolor{currentstroke}{rgb}{0.266667,0.266667,0.266667}%
\pgfsetstrokecolor{currentstroke}%
\pgfsetdash{}{0pt}%
\pgfpathmoveto{\pgfqpoint{4.198903in}{2.317626in}}%
\pgfpathlineto{\pgfqpoint{4.449806in}{2.317626in}}%
\pgfpathlineto{\pgfqpoint{4.449806in}{2.320276in}}%
\pgfpathlineto{\pgfqpoint{4.198903in}{2.320276in}}%
\pgfpathlineto{\pgfqpoint{4.198903in}{2.317626in}}%
\pgfpathclose%
\pgfusepath{stroke,fill}%
\end{pgfscope}%
\begin{pgfscope}%
\pgfpathrectangle{\pgfqpoint{0.781250in}{0.638889in}}{\pgfqpoint{4.218750in}{2.172222in}}%
\pgfusepath{clip}%
\pgfsetbuttcap%
\pgfsetmiterjoin%
\definecolor{currentfill}{rgb}{0.447059,0.447059,0.447059}%
\pgfsetfillcolor{currentfill}%
\pgfsetlinewidth{1.003750pt}%
\definecolor{currentstroke}{rgb}{0.266667,0.266667,0.266667}%
\pgfsetstrokecolor{currentstroke}%
\pgfsetdash{}{0pt}%
\pgfpathmoveto{\pgfqpoint{4.557336in}{2.469073in}}%
\pgfpathlineto{\pgfqpoint{4.808239in}{2.469073in}}%
\pgfpathlineto{\pgfqpoint{4.808239in}{2.477197in}}%
\pgfpathlineto{\pgfqpoint{4.557336in}{2.477197in}}%
\pgfpathlineto{\pgfqpoint{4.557336in}{2.469073in}}%
\pgfpathclose%
\pgfusepath{stroke,fill}%
\end{pgfscope}%
\begin{pgfscope}%
\pgfpathrectangle{\pgfqpoint{0.781250in}{0.638889in}}{\pgfqpoint{4.218750in}{2.172222in}}%
\pgfusepath{clip}%
\pgfsetbuttcap%
\pgfsetmiterjoin%
\definecolor{currentfill}{rgb}{0.447059,0.447059,0.447059}%
\pgfsetfillcolor{currentfill}%
\pgfsetlinewidth{1.003750pt}%
\definecolor{currentstroke}{rgb}{0.266667,0.266667,0.266667}%
\pgfsetstrokecolor{currentstroke}%
\pgfsetdash{}{0pt}%
\pgfpathmoveto{\pgfqpoint{0.973011in}{2.569669in}}%
\pgfpathlineto{\pgfqpoint{1.223914in}{2.569669in}}%
\pgfpathlineto{\pgfqpoint{1.223914in}{2.573752in}}%
\pgfpathlineto{\pgfqpoint{0.973011in}{2.573752in}}%
\pgfpathlineto{\pgfqpoint{0.973011in}{2.569669in}}%
\pgfpathclose%
\pgfusepath{stroke,fill}%
\end{pgfscope}%
\begin{pgfscope}%
\pgfpathrectangle{\pgfqpoint{0.781250in}{0.638889in}}{\pgfqpoint{4.218750in}{2.172222in}}%
\pgfusepath{clip}%
\pgfsetbuttcap%
\pgfsetmiterjoin%
\definecolor{currentfill}{rgb}{0.447059,0.447059,0.447059}%
\pgfsetfillcolor{currentfill}%
\pgfsetlinewidth{1.003750pt}%
\definecolor{currentstroke}{rgb}{0.266667,0.266667,0.266667}%
\pgfsetstrokecolor{currentstroke}%
\pgfsetdash{}{0pt}%
\pgfpathmoveto{\pgfqpoint{1.331444in}{2.137874in}}%
\pgfpathlineto{\pgfqpoint{1.582347in}{2.137874in}}%
\pgfpathlineto{\pgfqpoint{1.582347in}{2.140937in}}%
\pgfpathlineto{\pgfqpoint{1.331444in}{2.140937in}}%
\pgfpathlineto{\pgfqpoint{1.331444in}{2.137874in}}%
\pgfpathclose%
\pgfusepath{stroke,fill}%
\end{pgfscope}%
\begin{pgfscope}%
\pgfpathrectangle{\pgfqpoint{0.781250in}{0.638889in}}{\pgfqpoint{4.218750in}{2.172222in}}%
\pgfusepath{clip}%
\pgfsetbuttcap%
\pgfsetmiterjoin%
\definecolor{currentfill}{rgb}{0.447059,0.447059,0.447059}%
\pgfsetfillcolor{currentfill}%
\pgfsetlinewidth{1.003750pt}%
\definecolor{currentstroke}{rgb}{0.266667,0.266667,0.266667}%
\pgfsetstrokecolor{currentstroke}%
\pgfsetdash{}{0pt}%
\pgfpathmoveto{\pgfqpoint{1.689876in}{1.961968in}}%
\pgfpathlineto{\pgfqpoint{1.940779in}{1.961968in}}%
\pgfpathlineto{\pgfqpoint{1.940779in}{1.964726in}}%
\pgfpathlineto{\pgfqpoint{1.689876in}{1.964726in}}%
\pgfpathlineto{\pgfqpoint{1.689876in}{1.961968in}}%
\pgfpathclose%
\pgfusepath{stroke,fill}%
\end{pgfscope}%
\begin{pgfscope}%
\pgfpathrectangle{\pgfqpoint{0.781250in}{0.638889in}}{\pgfqpoint{4.218750in}{2.172222in}}%
\pgfusepath{clip}%
\pgfsetbuttcap%
\pgfsetmiterjoin%
\definecolor{currentfill}{rgb}{0.447059,0.447059,0.447059}%
\pgfsetfillcolor{currentfill}%
\pgfsetlinewidth{1.003750pt}%
\definecolor{currentstroke}{rgb}{0.266667,0.266667,0.266667}%
\pgfsetstrokecolor{currentstroke}%
\pgfsetdash{}{0pt}%
\pgfpathmoveto{\pgfqpoint{2.048309in}{1.896236in}}%
\pgfpathlineto{\pgfqpoint{2.299211in}{1.896236in}}%
\pgfpathlineto{\pgfqpoint{2.299211in}{1.898148in}}%
\pgfpathlineto{\pgfqpoint{2.048309in}{1.898148in}}%
\pgfpathlineto{\pgfqpoint{2.048309in}{1.896236in}}%
\pgfpathclose%
\pgfusepath{stroke,fill}%
\end{pgfscope}%
\begin{pgfscope}%
\pgfpathrectangle{\pgfqpoint{0.781250in}{0.638889in}}{\pgfqpoint{4.218750in}{2.172222in}}%
\pgfusepath{clip}%
\pgfsetbuttcap%
\pgfsetmiterjoin%
\definecolor{currentfill}{rgb}{0.447059,0.447059,0.447059}%
\pgfsetfillcolor{currentfill}%
\pgfsetlinewidth{1.003750pt}%
\definecolor{currentstroke}{rgb}{0.266667,0.266667,0.266667}%
\pgfsetstrokecolor{currentstroke}%
\pgfsetdash{}{0pt}%
\pgfpathmoveto{\pgfqpoint{2.406741in}{1.999287in}}%
\pgfpathlineto{\pgfqpoint{2.657644in}{1.999287in}}%
\pgfpathlineto{\pgfqpoint{2.657644in}{2.000916in}}%
\pgfpathlineto{\pgfqpoint{2.406741in}{2.000916in}}%
\pgfpathlineto{\pgfqpoint{2.406741in}{1.999287in}}%
\pgfpathclose%
\pgfusepath{stroke,fill}%
\end{pgfscope}%
\begin{pgfscope}%
\pgfpathrectangle{\pgfqpoint{0.781250in}{0.638889in}}{\pgfqpoint{4.218750in}{2.172222in}}%
\pgfusepath{clip}%
\pgfsetbuttcap%
\pgfsetmiterjoin%
\definecolor{currentfill}{rgb}{0.447059,0.447059,0.447059}%
\pgfsetfillcolor{currentfill}%
\pgfsetlinewidth{1.003750pt}%
\definecolor{currentstroke}{rgb}{0.266667,0.266667,0.266667}%
\pgfsetstrokecolor{currentstroke}%
\pgfsetdash{}{0pt}%
\pgfpathmoveto{\pgfqpoint{2.765174in}{2.191897in}}%
\pgfpathlineto{\pgfqpoint{3.016076in}{2.191897in}}%
\pgfpathlineto{\pgfqpoint{3.016076in}{2.193657in}}%
\pgfpathlineto{\pgfqpoint{2.765174in}{2.193657in}}%
\pgfpathlineto{\pgfqpoint{2.765174in}{2.191897in}}%
\pgfpathclose%
\pgfusepath{stroke,fill}%
\end{pgfscope}%
\begin{pgfscope}%
\pgfpathrectangle{\pgfqpoint{0.781250in}{0.638889in}}{\pgfqpoint{4.218750in}{2.172222in}}%
\pgfusepath{clip}%
\pgfsetbuttcap%
\pgfsetmiterjoin%
\definecolor{currentfill}{rgb}{0.447059,0.447059,0.447059}%
\pgfsetfillcolor{currentfill}%
\pgfsetlinewidth{1.003750pt}%
\definecolor{currentstroke}{rgb}{0.266667,0.266667,0.266667}%
\pgfsetstrokecolor{currentstroke}%
\pgfsetdash{}{0pt}%
\pgfpathmoveto{\pgfqpoint{3.123606in}{2.085285in}}%
\pgfpathlineto{\pgfqpoint{3.374509in}{2.085285in}}%
\pgfpathlineto{\pgfqpoint{3.374509in}{2.086892in}}%
\pgfpathlineto{\pgfqpoint{3.123606in}{2.086892in}}%
\pgfpathlineto{\pgfqpoint{3.123606in}{2.085285in}}%
\pgfpathclose%
\pgfusepath{stroke,fill}%
\end{pgfscope}%
\begin{pgfscope}%
\pgfpathrectangle{\pgfqpoint{0.781250in}{0.638889in}}{\pgfqpoint{4.218750in}{2.172222in}}%
\pgfusepath{clip}%
\pgfsetbuttcap%
\pgfsetmiterjoin%
\definecolor{currentfill}{rgb}{0.447059,0.447059,0.447059}%
\pgfsetfillcolor{currentfill}%
\pgfsetlinewidth{1.003750pt}%
\definecolor{currentstroke}{rgb}{0.266667,0.266667,0.266667}%
\pgfsetstrokecolor{currentstroke}%
\pgfsetdash{}{0pt}%
\pgfpathmoveto{\pgfqpoint{3.482039in}{2.052767in}}%
\pgfpathlineto{\pgfqpoint{3.732941in}{2.052767in}}%
\pgfpathlineto{\pgfqpoint{3.732941in}{2.055243in}}%
\pgfpathlineto{\pgfqpoint{3.482039in}{2.055243in}}%
\pgfpathlineto{\pgfqpoint{3.482039in}{2.052767in}}%
\pgfpathclose%
\pgfusepath{stroke,fill}%
\end{pgfscope}%
\begin{pgfscope}%
\pgfpathrectangle{\pgfqpoint{0.781250in}{0.638889in}}{\pgfqpoint{4.218750in}{2.172222in}}%
\pgfusepath{clip}%
\pgfsetbuttcap%
\pgfsetmiterjoin%
\definecolor{currentfill}{rgb}{0.447059,0.447059,0.447059}%
\pgfsetfillcolor{currentfill}%
\pgfsetlinewidth{1.003750pt}%
\definecolor{currentstroke}{rgb}{0.266667,0.266667,0.266667}%
\pgfsetstrokecolor{currentstroke}%
\pgfsetdash{}{0pt}%
\pgfpathmoveto{\pgfqpoint{3.840471in}{2.278743in}}%
\pgfpathlineto{\pgfqpoint{4.091374in}{2.278743in}}%
\pgfpathlineto{\pgfqpoint{4.091374in}{2.280807in}}%
\pgfpathlineto{\pgfqpoint{3.840471in}{2.280807in}}%
\pgfpathlineto{\pgfqpoint{3.840471in}{2.278743in}}%
\pgfpathclose%
\pgfusepath{stroke,fill}%
\end{pgfscope}%
\begin{pgfscope}%
\pgfpathrectangle{\pgfqpoint{0.781250in}{0.638889in}}{\pgfqpoint{4.218750in}{2.172222in}}%
\pgfusepath{clip}%
\pgfsetbuttcap%
\pgfsetmiterjoin%
\definecolor{currentfill}{rgb}{0.447059,0.447059,0.447059}%
\pgfsetfillcolor{currentfill}%
\pgfsetlinewidth{1.003750pt}%
\definecolor{currentstroke}{rgb}{0.266667,0.266667,0.266667}%
\pgfsetstrokecolor{currentstroke}%
\pgfsetdash{}{0pt}%
\pgfpathmoveto{\pgfqpoint{4.198903in}{2.320276in}}%
\pgfpathlineto{\pgfqpoint{4.449806in}{2.320276in}}%
\pgfpathlineto{\pgfqpoint{4.449806in}{2.322166in}}%
\pgfpathlineto{\pgfqpoint{4.198903in}{2.322166in}}%
\pgfpathlineto{\pgfqpoint{4.198903in}{2.320276in}}%
\pgfpathclose%
\pgfusepath{stroke,fill}%
\end{pgfscope}%
\begin{pgfscope}%
\pgfpathrectangle{\pgfqpoint{0.781250in}{0.638889in}}{\pgfqpoint{4.218750in}{2.172222in}}%
\pgfusepath{clip}%
\pgfsetbuttcap%
\pgfsetmiterjoin%
\definecolor{currentfill}{rgb}{0.447059,0.447059,0.447059}%
\pgfsetfillcolor{currentfill}%
\pgfsetlinewidth{1.003750pt}%
\definecolor{currentstroke}{rgb}{0.266667,0.266667,0.266667}%
\pgfsetstrokecolor{currentstroke}%
\pgfsetdash{}{0pt}%
\pgfpathmoveto{\pgfqpoint{4.557336in}{2.477197in}}%
\pgfpathlineto{\pgfqpoint{4.808239in}{2.477197in}}%
\pgfpathlineto{\pgfqpoint{4.808239in}{2.478522in}}%
\pgfpathlineto{\pgfqpoint{4.557336in}{2.478522in}}%
\pgfpathlineto{\pgfqpoint{4.557336in}{2.477197in}}%
\pgfpathclose%
\pgfusepath{stroke,fill}%
\end{pgfscope}%
\begin{pgfscope}%
\pgfpathrectangle{\pgfqpoint{0.781250in}{0.638889in}}{\pgfqpoint{4.218750in}{2.172222in}}%
\pgfusepath{clip}%
\pgfsetbuttcap%
\pgfsetmiterjoin%
\definecolor{currentfill}{rgb}{0.447059,0.447059,0.447059}%
\pgfsetfillcolor{currentfill}%
\pgfsetlinewidth{1.003750pt}%
\definecolor{currentstroke}{rgb}{0.266667,0.266667,0.266667}%
\pgfsetstrokecolor{currentstroke}%
\pgfsetdash{}{0pt}%
\pgfpathmoveto{\pgfqpoint{0.973011in}{2.573752in}}%
\pgfpathlineto{\pgfqpoint{1.223914in}{2.573752in}}%
\pgfpathlineto{\pgfqpoint{1.223914in}{2.575099in}}%
\pgfpathlineto{\pgfqpoint{0.973011in}{2.575099in}}%
\pgfpathlineto{\pgfqpoint{0.973011in}{2.573752in}}%
\pgfpathclose%
\pgfusepath{stroke,fill}%
\end{pgfscope}%
\begin{pgfscope}%
\pgfpathrectangle{\pgfqpoint{0.781250in}{0.638889in}}{\pgfqpoint{4.218750in}{2.172222in}}%
\pgfusepath{clip}%
\pgfsetbuttcap%
\pgfsetmiterjoin%
\definecolor{currentfill}{rgb}{0.447059,0.447059,0.447059}%
\pgfsetfillcolor{currentfill}%
\pgfsetlinewidth{1.003750pt}%
\definecolor{currentstroke}{rgb}{0.266667,0.266667,0.266667}%
\pgfsetstrokecolor{currentstroke}%
\pgfsetdash{}{0pt}%
\pgfpathmoveto{\pgfqpoint{1.331444in}{2.140937in}}%
\pgfpathlineto{\pgfqpoint{1.582347in}{2.140937in}}%
\pgfpathlineto{\pgfqpoint{1.582347in}{2.142349in}}%
\pgfpathlineto{\pgfqpoint{1.331444in}{2.142349in}}%
\pgfpathlineto{\pgfqpoint{1.331444in}{2.140937in}}%
\pgfpathclose%
\pgfusepath{stroke,fill}%
\end{pgfscope}%
\begin{pgfscope}%
\pgfpathrectangle{\pgfqpoint{0.781250in}{0.638889in}}{\pgfqpoint{4.218750in}{2.172222in}}%
\pgfusepath{clip}%
\pgfsetbuttcap%
\pgfsetmiterjoin%
\definecolor{currentfill}{rgb}{0.447059,0.447059,0.447059}%
\pgfsetfillcolor{currentfill}%
\pgfsetlinewidth{1.003750pt}%
\definecolor{currentstroke}{rgb}{0.266667,0.266667,0.266667}%
\pgfsetstrokecolor{currentstroke}%
\pgfsetdash{}{0pt}%
\pgfpathmoveto{\pgfqpoint{1.689876in}{1.964726in}}%
\pgfpathlineto{\pgfqpoint{1.940779in}{1.964726in}}%
\pgfpathlineto{\pgfqpoint{1.940779in}{1.966030in}}%
\pgfpathlineto{\pgfqpoint{1.689876in}{1.966030in}}%
\pgfpathlineto{\pgfqpoint{1.689876in}{1.964726in}}%
\pgfpathclose%
\pgfusepath{stroke,fill}%
\end{pgfscope}%
\begin{pgfscope}%
\pgfpathrectangle{\pgfqpoint{0.781250in}{0.638889in}}{\pgfqpoint{4.218750in}{2.172222in}}%
\pgfusepath{clip}%
\pgfsetbuttcap%
\pgfsetmiterjoin%
\definecolor{currentfill}{rgb}{0.447059,0.447059,0.447059}%
\pgfsetfillcolor{currentfill}%
\pgfsetlinewidth{1.003750pt}%
\definecolor{currentstroke}{rgb}{0.266667,0.266667,0.266667}%
\pgfsetstrokecolor{currentstroke}%
\pgfsetdash{}{0pt}%
\pgfpathmoveto{\pgfqpoint{2.048309in}{1.898148in}}%
\pgfpathlineto{\pgfqpoint{2.299211in}{1.898148in}}%
\pgfpathlineto{\pgfqpoint{2.299211in}{1.899495in}}%
\pgfpathlineto{\pgfqpoint{2.048309in}{1.899495in}}%
\pgfpathlineto{\pgfqpoint{2.048309in}{1.898148in}}%
\pgfpathclose%
\pgfusepath{stroke,fill}%
\end{pgfscope}%
\begin{pgfscope}%
\pgfpathrectangle{\pgfqpoint{0.781250in}{0.638889in}}{\pgfqpoint{4.218750in}{2.172222in}}%
\pgfusepath{clip}%
\pgfsetbuttcap%
\pgfsetmiterjoin%
\definecolor{currentfill}{rgb}{0.447059,0.447059,0.447059}%
\pgfsetfillcolor{currentfill}%
\pgfsetlinewidth{1.003750pt}%
\definecolor{currentstroke}{rgb}{0.266667,0.266667,0.266667}%
\pgfsetstrokecolor{currentstroke}%
\pgfsetdash{}{0pt}%
\pgfpathmoveto{\pgfqpoint{2.406741in}{2.000916in}}%
\pgfpathlineto{\pgfqpoint{2.657644in}{2.000916in}}%
\pgfpathlineto{\pgfqpoint{2.657644in}{2.003479in}}%
\pgfpathlineto{\pgfqpoint{2.406741in}{2.003479in}}%
\pgfpathlineto{\pgfqpoint{2.406741in}{2.000916in}}%
\pgfpathclose%
\pgfusepath{stroke,fill}%
\end{pgfscope}%
\begin{pgfscope}%
\pgfpathrectangle{\pgfqpoint{0.781250in}{0.638889in}}{\pgfqpoint{4.218750in}{2.172222in}}%
\pgfusepath{clip}%
\pgfsetbuttcap%
\pgfsetmiterjoin%
\definecolor{currentfill}{rgb}{0.447059,0.447059,0.447059}%
\pgfsetfillcolor{currentfill}%
\pgfsetlinewidth{1.003750pt}%
\definecolor{currentstroke}{rgb}{0.266667,0.266667,0.266667}%
\pgfsetstrokecolor{currentstroke}%
\pgfsetdash{}{0pt}%
\pgfpathmoveto{\pgfqpoint{2.765174in}{2.193657in}}%
\pgfpathlineto{\pgfqpoint{3.016076in}{2.193657in}}%
\pgfpathlineto{\pgfqpoint{3.016076in}{2.195786in}}%
\pgfpathlineto{\pgfqpoint{2.765174in}{2.195786in}}%
\pgfpathlineto{\pgfqpoint{2.765174in}{2.193657in}}%
\pgfpathclose%
\pgfusepath{stroke,fill}%
\end{pgfscope}%
\begin{pgfscope}%
\pgfpathrectangle{\pgfqpoint{0.781250in}{0.638889in}}{\pgfqpoint{4.218750in}{2.172222in}}%
\pgfusepath{clip}%
\pgfsetbuttcap%
\pgfsetmiterjoin%
\definecolor{currentfill}{rgb}{0.447059,0.447059,0.447059}%
\pgfsetfillcolor{currentfill}%
\pgfsetlinewidth{1.003750pt}%
\definecolor{currentstroke}{rgb}{0.266667,0.266667,0.266667}%
\pgfsetstrokecolor{currentstroke}%
\pgfsetdash{}{0pt}%
\pgfpathmoveto{\pgfqpoint{3.123606in}{2.086892in}}%
\pgfpathlineto{\pgfqpoint{3.374509in}{2.086892in}}%
\pgfpathlineto{\pgfqpoint{3.374509in}{2.088456in}}%
\pgfpathlineto{\pgfqpoint{3.123606in}{2.088456in}}%
\pgfpathlineto{\pgfqpoint{3.123606in}{2.086892in}}%
\pgfpathclose%
\pgfusepath{stroke,fill}%
\end{pgfscope}%
\begin{pgfscope}%
\pgfpathrectangle{\pgfqpoint{0.781250in}{0.638889in}}{\pgfqpoint{4.218750in}{2.172222in}}%
\pgfusepath{clip}%
\pgfsetbuttcap%
\pgfsetmiterjoin%
\definecolor{currentfill}{rgb}{0.447059,0.447059,0.447059}%
\pgfsetfillcolor{currentfill}%
\pgfsetlinewidth{1.003750pt}%
\definecolor{currentstroke}{rgb}{0.266667,0.266667,0.266667}%
\pgfsetstrokecolor{currentstroke}%
\pgfsetdash{}{0pt}%
\pgfpathmoveto{\pgfqpoint{3.482039in}{2.055243in}}%
\pgfpathlineto{\pgfqpoint{3.732941in}{2.055243in}}%
\pgfpathlineto{\pgfqpoint{3.732941in}{2.056525in}}%
\pgfpathlineto{\pgfqpoint{3.482039in}{2.056525in}}%
\pgfpathlineto{\pgfqpoint{3.482039in}{2.055243in}}%
\pgfpathclose%
\pgfusepath{stroke,fill}%
\end{pgfscope}%
\begin{pgfscope}%
\pgfpathrectangle{\pgfqpoint{0.781250in}{0.638889in}}{\pgfqpoint{4.218750in}{2.172222in}}%
\pgfusepath{clip}%
\pgfsetbuttcap%
\pgfsetmiterjoin%
\definecolor{currentfill}{rgb}{0.447059,0.447059,0.447059}%
\pgfsetfillcolor{currentfill}%
\pgfsetlinewidth{1.003750pt}%
\definecolor{currentstroke}{rgb}{0.266667,0.266667,0.266667}%
\pgfsetstrokecolor{currentstroke}%
\pgfsetdash{}{0pt}%
\pgfpathmoveto{\pgfqpoint{3.840471in}{2.280807in}}%
\pgfpathlineto{\pgfqpoint{4.091374in}{2.280807in}}%
\pgfpathlineto{\pgfqpoint{4.091374in}{2.281936in}}%
\pgfpathlineto{\pgfqpoint{3.840471in}{2.281936in}}%
\pgfpathlineto{\pgfqpoint{3.840471in}{2.280807in}}%
\pgfpathclose%
\pgfusepath{stroke,fill}%
\end{pgfscope}%
\begin{pgfscope}%
\pgfpathrectangle{\pgfqpoint{0.781250in}{0.638889in}}{\pgfqpoint{4.218750in}{2.172222in}}%
\pgfusepath{clip}%
\pgfsetbuttcap%
\pgfsetmiterjoin%
\definecolor{currentfill}{rgb}{0.447059,0.447059,0.447059}%
\pgfsetfillcolor{currentfill}%
\pgfsetlinewidth{1.003750pt}%
\definecolor{currentstroke}{rgb}{0.266667,0.266667,0.266667}%
\pgfsetstrokecolor{currentstroke}%
\pgfsetdash{}{0pt}%
\pgfpathmoveto{\pgfqpoint{4.198903in}{2.322166in}}%
\pgfpathlineto{\pgfqpoint{4.449806in}{2.322166in}}%
\pgfpathlineto{\pgfqpoint{4.449806in}{2.323143in}}%
\pgfpathlineto{\pgfqpoint{4.198903in}{2.323143in}}%
\pgfpathlineto{\pgfqpoint{4.198903in}{2.322166in}}%
\pgfpathclose%
\pgfusepath{stroke,fill}%
\end{pgfscope}%
\begin{pgfscope}%
\pgfpathrectangle{\pgfqpoint{0.781250in}{0.638889in}}{\pgfqpoint{4.218750in}{2.172222in}}%
\pgfusepath{clip}%
\pgfsetbuttcap%
\pgfsetmiterjoin%
\definecolor{currentfill}{rgb}{0.447059,0.447059,0.447059}%
\pgfsetfillcolor{currentfill}%
\pgfsetlinewidth{1.003750pt}%
\definecolor{currentstroke}{rgb}{0.266667,0.266667,0.266667}%
\pgfsetstrokecolor{currentstroke}%
\pgfsetdash{}{0pt}%
\pgfpathmoveto{\pgfqpoint{4.557336in}{2.478522in}}%
\pgfpathlineto{\pgfqpoint{4.808239in}{2.478522in}}%
\pgfpathlineto{\pgfqpoint{4.808239in}{2.479478in}}%
\pgfpathlineto{\pgfqpoint{4.557336in}{2.479478in}}%
\pgfpathlineto{\pgfqpoint{4.557336in}{2.478522in}}%
\pgfpathclose%
\pgfusepath{stroke,fill}%
\end{pgfscope}%
\begin{pgfscope}%
\pgfpathrectangle{\pgfqpoint{0.781250in}{0.638889in}}{\pgfqpoint{4.218750in}{2.172222in}}%
\pgfusepath{clip}%
\pgfsetbuttcap%
\pgfsetmiterjoin%
\definecolor{currentfill}{rgb}{0.447059,0.447059,0.447059}%
\pgfsetfillcolor{currentfill}%
\pgfsetlinewidth{1.003750pt}%
\definecolor{currentstroke}{rgb}{0.266667,0.266667,0.266667}%
\pgfsetstrokecolor{currentstroke}%
\pgfsetdash{}{0pt}%
\pgfpathmoveto{\pgfqpoint{0.973011in}{2.575099in}}%
\pgfpathlineto{\pgfqpoint{1.223914in}{2.575099in}}%
\pgfpathlineto{\pgfqpoint{1.223914in}{2.575773in}}%
\pgfpathlineto{\pgfqpoint{0.973011in}{2.575773in}}%
\pgfpathlineto{\pgfqpoint{0.973011in}{2.575099in}}%
\pgfpathclose%
\pgfusepath{stroke,fill}%
\end{pgfscope}%
\begin{pgfscope}%
\pgfpathrectangle{\pgfqpoint{0.781250in}{0.638889in}}{\pgfqpoint{4.218750in}{2.172222in}}%
\pgfusepath{clip}%
\pgfsetbuttcap%
\pgfsetmiterjoin%
\definecolor{currentfill}{rgb}{0.447059,0.447059,0.447059}%
\pgfsetfillcolor{currentfill}%
\pgfsetlinewidth{1.003750pt}%
\definecolor{currentstroke}{rgb}{0.266667,0.266667,0.266667}%
\pgfsetstrokecolor{currentstroke}%
\pgfsetdash{}{0pt}%
\pgfpathmoveto{\pgfqpoint{1.331444in}{2.142349in}}%
\pgfpathlineto{\pgfqpoint{1.582347in}{2.142349in}}%
\pgfpathlineto{\pgfqpoint{1.582347in}{2.142653in}}%
\pgfpathlineto{\pgfqpoint{1.331444in}{2.142653in}}%
\pgfpathlineto{\pgfqpoint{1.331444in}{2.142349in}}%
\pgfpathclose%
\pgfusepath{stroke,fill}%
\end{pgfscope}%
\begin{pgfscope}%
\pgfpathrectangle{\pgfqpoint{0.781250in}{0.638889in}}{\pgfqpoint{4.218750in}{2.172222in}}%
\pgfusepath{clip}%
\pgfsetbuttcap%
\pgfsetmiterjoin%
\definecolor{currentfill}{rgb}{0.447059,0.447059,0.447059}%
\pgfsetfillcolor{currentfill}%
\pgfsetlinewidth{1.003750pt}%
\definecolor{currentstroke}{rgb}{0.266667,0.266667,0.266667}%
\pgfsetstrokecolor{currentstroke}%
\pgfsetdash{}{0pt}%
\pgfpathmoveto{\pgfqpoint{1.689876in}{1.966030in}}%
\pgfpathlineto{\pgfqpoint{1.940779in}{1.966030in}}%
\pgfpathlineto{\pgfqpoint{1.940779in}{1.966073in}}%
\pgfpathlineto{\pgfqpoint{1.689876in}{1.966073in}}%
\pgfpathlineto{\pgfqpoint{1.689876in}{1.966030in}}%
\pgfpathclose%
\pgfusepath{stroke,fill}%
\end{pgfscope}%
\begin{pgfscope}%
\pgfpathrectangle{\pgfqpoint{0.781250in}{0.638889in}}{\pgfqpoint{4.218750in}{2.172222in}}%
\pgfusepath{clip}%
\pgfsetbuttcap%
\pgfsetmiterjoin%
\definecolor{currentfill}{rgb}{0.447059,0.447059,0.447059}%
\pgfsetfillcolor{currentfill}%
\pgfsetlinewidth{1.003750pt}%
\definecolor{currentstroke}{rgb}{0.266667,0.266667,0.266667}%
\pgfsetstrokecolor{currentstroke}%
\pgfsetdash{}{0pt}%
\pgfpathmoveto{\pgfqpoint{2.048309in}{1.899495in}}%
\pgfpathlineto{\pgfqpoint{2.299211in}{1.899495in}}%
\pgfpathlineto{\pgfqpoint{2.299211in}{1.899647in}}%
\pgfpathlineto{\pgfqpoint{2.048309in}{1.899647in}}%
\pgfpathlineto{\pgfqpoint{2.048309in}{1.899495in}}%
\pgfpathclose%
\pgfusepath{stroke,fill}%
\end{pgfscope}%
\begin{pgfscope}%
\pgfpathrectangle{\pgfqpoint{0.781250in}{0.638889in}}{\pgfqpoint{4.218750in}{2.172222in}}%
\pgfusepath{clip}%
\pgfsetbuttcap%
\pgfsetmiterjoin%
\definecolor{currentfill}{rgb}{0.447059,0.447059,0.447059}%
\pgfsetfillcolor{currentfill}%
\pgfsetlinewidth{1.003750pt}%
\definecolor{currentstroke}{rgb}{0.266667,0.266667,0.266667}%
\pgfsetstrokecolor{currentstroke}%
\pgfsetdash{}{0pt}%
\pgfpathmoveto{\pgfqpoint{2.406741in}{2.003479in}}%
\pgfpathlineto{\pgfqpoint{2.657644in}{2.003479in}}%
\pgfpathlineto{\pgfqpoint{2.657644in}{2.003609in}}%
\pgfpathlineto{\pgfqpoint{2.406741in}{2.003609in}}%
\pgfpathlineto{\pgfqpoint{2.406741in}{2.003479in}}%
\pgfpathclose%
\pgfusepath{stroke,fill}%
\end{pgfscope}%
\begin{pgfscope}%
\pgfpathrectangle{\pgfqpoint{0.781250in}{0.638889in}}{\pgfqpoint{4.218750in}{2.172222in}}%
\pgfusepath{clip}%
\pgfsetbuttcap%
\pgfsetmiterjoin%
\definecolor{currentfill}{rgb}{0.447059,0.447059,0.447059}%
\pgfsetfillcolor{currentfill}%
\pgfsetlinewidth{1.003750pt}%
\definecolor{currentstroke}{rgb}{0.266667,0.266667,0.266667}%
\pgfsetstrokecolor{currentstroke}%
\pgfsetdash{}{0pt}%
\pgfpathmoveto{\pgfqpoint{2.765174in}{2.195786in}}%
\pgfpathlineto{\pgfqpoint{3.016076in}{2.195786in}}%
\pgfpathlineto{\pgfqpoint{3.016076in}{2.195873in}}%
\pgfpathlineto{\pgfqpoint{2.765174in}{2.195873in}}%
\pgfpathlineto{\pgfqpoint{2.765174in}{2.195786in}}%
\pgfpathclose%
\pgfusepath{stroke,fill}%
\end{pgfscope}%
\begin{pgfscope}%
\pgfpathrectangle{\pgfqpoint{0.781250in}{0.638889in}}{\pgfqpoint{4.218750in}{2.172222in}}%
\pgfusepath{clip}%
\pgfsetbuttcap%
\pgfsetmiterjoin%
\definecolor{currentfill}{rgb}{0.447059,0.447059,0.447059}%
\pgfsetfillcolor{currentfill}%
\pgfsetlinewidth{1.003750pt}%
\definecolor{currentstroke}{rgb}{0.266667,0.266667,0.266667}%
\pgfsetstrokecolor{currentstroke}%
\pgfsetdash{}{0pt}%
\pgfpathmoveto{\pgfqpoint{3.123606in}{2.088456in}}%
\pgfpathlineto{\pgfqpoint{3.374509in}{2.088456in}}%
\pgfpathlineto{\pgfqpoint{3.374509in}{2.088543in}}%
\pgfpathlineto{\pgfqpoint{3.123606in}{2.088543in}}%
\pgfpathlineto{\pgfqpoint{3.123606in}{2.088456in}}%
\pgfpathclose%
\pgfusepath{stroke,fill}%
\end{pgfscope}%
\begin{pgfscope}%
\pgfpathrectangle{\pgfqpoint{0.781250in}{0.638889in}}{\pgfqpoint{4.218750in}{2.172222in}}%
\pgfusepath{clip}%
\pgfsetbuttcap%
\pgfsetmiterjoin%
\definecolor{currentfill}{rgb}{0.447059,0.447059,0.447059}%
\pgfsetfillcolor{currentfill}%
\pgfsetlinewidth{1.003750pt}%
\definecolor{currentstroke}{rgb}{0.266667,0.266667,0.266667}%
\pgfsetstrokecolor{currentstroke}%
\pgfsetdash{}{0pt}%
\pgfpathmoveto{\pgfqpoint{3.482039in}{2.056525in}}%
\pgfpathlineto{\pgfqpoint{3.732941in}{2.056525in}}%
\pgfpathlineto{\pgfqpoint{3.732941in}{2.056742in}}%
\pgfpathlineto{\pgfqpoint{3.482039in}{2.056742in}}%
\pgfpathlineto{\pgfqpoint{3.482039in}{2.056525in}}%
\pgfpathclose%
\pgfusepath{stroke,fill}%
\end{pgfscope}%
\begin{pgfscope}%
\pgfpathrectangle{\pgfqpoint{0.781250in}{0.638889in}}{\pgfqpoint{4.218750in}{2.172222in}}%
\pgfusepath{clip}%
\pgfsetbuttcap%
\pgfsetmiterjoin%
\definecolor{currentfill}{rgb}{0.447059,0.447059,0.447059}%
\pgfsetfillcolor{currentfill}%
\pgfsetlinewidth{1.003750pt}%
\definecolor{currentstroke}{rgb}{0.266667,0.266667,0.266667}%
\pgfsetstrokecolor{currentstroke}%
\pgfsetdash{}{0pt}%
\pgfpathmoveto{\pgfqpoint{3.840471in}{2.281936in}}%
\pgfpathlineto{\pgfqpoint{4.091374in}{2.281936in}}%
\pgfpathlineto{\pgfqpoint{4.091374in}{2.282088in}}%
\pgfpathlineto{\pgfqpoint{3.840471in}{2.282088in}}%
\pgfpathlineto{\pgfqpoint{3.840471in}{2.281936in}}%
\pgfpathclose%
\pgfusepath{stroke,fill}%
\end{pgfscope}%
\begin{pgfscope}%
\pgfpathrectangle{\pgfqpoint{0.781250in}{0.638889in}}{\pgfqpoint{4.218750in}{2.172222in}}%
\pgfusepath{clip}%
\pgfsetbuttcap%
\pgfsetmiterjoin%
\definecolor{currentfill}{rgb}{0.447059,0.447059,0.447059}%
\pgfsetfillcolor{currentfill}%
\pgfsetlinewidth{1.003750pt}%
\definecolor{currentstroke}{rgb}{0.266667,0.266667,0.266667}%
\pgfsetstrokecolor{currentstroke}%
\pgfsetdash{}{0pt}%
\pgfpathmoveto{\pgfqpoint{4.198903in}{2.323143in}}%
\pgfpathlineto{\pgfqpoint{4.449806in}{2.323143in}}%
\pgfpathlineto{\pgfqpoint{4.449806in}{2.323273in}}%
\pgfpathlineto{\pgfqpoint{4.198903in}{2.323273in}}%
\pgfpathlineto{\pgfqpoint{4.198903in}{2.323143in}}%
\pgfpathclose%
\pgfusepath{stroke,fill}%
\end{pgfscope}%
\begin{pgfscope}%
\pgfpathrectangle{\pgfqpoint{0.781250in}{0.638889in}}{\pgfqpoint{4.218750in}{2.172222in}}%
\pgfusepath{clip}%
\pgfsetbuttcap%
\pgfsetmiterjoin%
\definecolor{currentfill}{rgb}{0.447059,0.447059,0.447059}%
\pgfsetfillcolor{currentfill}%
\pgfsetlinewidth{1.003750pt}%
\definecolor{currentstroke}{rgb}{0.266667,0.266667,0.266667}%
\pgfsetstrokecolor{currentstroke}%
\pgfsetdash{}{0pt}%
\pgfpathmoveto{\pgfqpoint{4.557336in}{2.479478in}}%
\pgfpathlineto{\pgfqpoint{4.808239in}{2.479478in}}%
\pgfpathlineto{\pgfqpoint{4.808239in}{2.479782in}}%
\pgfpathlineto{\pgfqpoint{4.557336in}{2.479782in}}%
\pgfpathlineto{\pgfqpoint{4.557336in}{2.479478in}}%
\pgfpathclose%
\pgfusepath{stroke,fill}%
\end{pgfscope}%
\begin{pgfscope}%
\definecolor{textcolor}{rgb}{0.000000,0.000000,0.000000}%
\pgfsetstrokecolor{textcolor}%
\pgfsetfillcolor{textcolor}%
\pgftext[x=1.098463in,y=2.603550in,,bottom]{\color{textcolor}{\ifdefined\pdftexversion\else\setmainfont{NanumMyeongjo}\rmfamily\fi\fontsize{5.000000}{6.000000}\selectfont\catcode`\^=\active\def^{\ifmmode\sp\else\^{}\fi}\catcode`\%=\active\def%{\%}89,166}}%
\end{pgfscope}%
\begin{pgfscope}%
\definecolor{textcolor}{rgb}{0.000000,0.000000,0.000000}%
\pgfsetstrokecolor{textcolor}%
\pgfsetfillcolor{textcolor}%
\pgftext[x=1.456895in,y=2.170431in,,bottom]{\color{textcolor}{\ifdefined\pdftexversion\else\setmainfont{NanumMyeongjo}\rmfamily\fi\fontsize{5.000000}{6.000000}\selectfont\catcode`\^=\active\def^{\ifmmode\sp\else\^{}\fi}\catcode`\%=\active\def%{\%}69,227}}%
\end{pgfscope}%
\begin{pgfscope}%
\definecolor{textcolor}{rgb}{0.000000,0.000000,0.000000}%
\pgfsetstrokecolor{textcolor}%
\pgfsetfillcolor{textcolor}%
\pgftext[x=1.815328in,y=1.993851in,,bottom]{\color{textcolor}{\ifdefined\pdftexversion\else\setmainfont{NanumMyeongjo}\rmfamily\fi\fontsize{5.000000}{6.000000}\selectfont\catcode`\^=\active\def^{\ifmmode\sp\else\^{}\fi}\catcode`\%=\active\def%{\%}61,098}}%
\end{pgfscope}%
\begin{pgfscope}%
\definecolor{textcolor}{rgb}{0.000000,0.000000,0.000000}%
\pgfsetstrokecolor{textcolor}%
\pgfsetfillcolor{textcolor}%
\pgftext[x=2.173760in,y=1.927424in,,bottom]{\color{textcolor}{\ifdefined\pdftexversion\else\setmainfont{NanumMyeongjo}\rmfamily\fi\fontsize{5.000000}{6.000000}\selectfont\catcode`\^=\active\def^{\ifmmode\sp\else\^{}\fi}\catcode`\%=\active\def%{\%}58,040}}%
\end{pgfscope}%
\begin{pgfscope}%
\definecolor{textcolor}{rgb}{0.000000,0.000000,0.000000}%
\pgfsetstrokecolor{textcolor}%
\pgfsetfillcolor{textcolor}%
\pgftext[x=2.532193in,y=2.031387in,,bottom]{\color{textcolor}{\ifdefined\pdftexversion\else\setmainfont{NanumMyeongjo}\rmfamily\fi\fontsize{5.000000}{6.000000}\selectfont\catcode`\^=\active\def^{\ifmmode\sp\else\^{}\fi}\catcode`\%=\active\def%{\%}62,826}}%
\end{pgfscope}%
\begin{pgfscope}%
\definecolor{textcolor}{rgb}{0.000000,0.000000,0.000000}%
\pgfsetstrokecolor{textcolor}%
\pgfsetfillcolor{textcolor}%
\pgftext[x=2.890625in,y=2.223650in,,bottom]{\color{textcolor}{\ifdefined\pdftexversion\else\setmainfont{NanumMyeongjo}\rmfamily\fi\fontsize{5.000000}{6.000000}\selectfont\catcode`\^=\active\def^{\ifmmode\sp\else\^{}\fi}\catcode`\%=\active\def%{\%}71,677}}%
\end{pgfscope}%
\begin{pgfscope}%
\definecolor{textcolor}{rgb}{0.000000,0.000000,0.000000}%
\pgfsetstrokecolor{textcolor}%
\pgfsetfillcolor{textcolor}%
\pgftext[x=3.249057in,y=2.116321in,,bottom]{\color{textcolor}{\ifdefined\pdftexversion\else\setmainfont{NanumMyeongjo}\rmfamily\fi\fontsize{5.000000}{6.000000}\selectfont\catcode`\^=\active\def^{\ifmmode\sp\else\^{}\fi}\catcode`\%=\active\def%{\%}66,736}}%
\end{pgfscope}%
\begin{pgfscope}%
\definecolor{textcolor}{rgb}{0.000000,0.000000,0.000000}%
\pgfsetstrokecolor{textcolor}%
\pgfsetfillcolor{textcolor}%
\pgftext[x=3.607490in,y=2.084520in,,bottom]{\color{textcolor}{\ifdefined\pdftexversion\else\setmainfont{NanumMyeongjo}\rmfamily\fi\fontsize{5.000000}{6.000000}\selectfont\catcode`\^=\active\def^{\ifmmode\sp\else\^{}\fi}\catcode`\%=\active\def%{\%}65,272}}%
\end{pgfscope}%
\begin{pgfscope}%
\definecolor{textcolor}{rgb}{0.000000,0.000000,0.000000}%
\pgfsetstrokecolor{textcolor}%
\pgfsetfillcolor{textcolor}%
\pgftext[x=3.965922in,y=2.309866in,,bottom]{\color{textcolor}{\ifdefined\pdftexversion\else\setmainfont{NanumMyeongjo}\rmfamily\fi\fontsize{5.000000}{6.000000}\selectfont\catcode`\^=\active\def^{\ifmmode\sp\else\^{}\fi}\catcode`\%=\active\def%{\%}75,646}}%
\end{pgfscope}%
\begin{pgfscope}%
\definecolor{textcolor}{rgb}{0.000000,0.000000,0.000000}%
\pgfsetstrokecolor{textcolor}%
\pgfsetfillcolor{textcolor}%
\pgftext[x=4.324355in,y=2.351051in,,bottom]{\color{textcolor}{\ifdefined\pdftexversion\else\setmainfont{NanumMyeongjo}\rmfamily\fi\fontsize{5.000000}{6.000000}\selectfont\catcode`\^=\active\def^{\ifmmode\sp\else\^{}\fi}\catcode`\%=\active\def%{\%}77,542}}%
\end{pgfscope}%
\begin{pgfscope}%
\definecolor{textcolor}{rgb}{0.000000,0.000000,0.000000}%
\pgfsetstrokecolor{textcolor}%
\pgfsetfillcolor{textcolor}%
\pgftext[x=4.682787in,y=2.507560in,,bottom]{\color{textcolor}{\ifdefined\pdftexversion\else\setmainfont{NanumMyeongjo}\rmfamily\fi\fontsize{5.000000}{6.000000}\selectfont\catcode`\^=\active\def^{\ifmmode\sp\else\^{}\fi}\catcode`\%=\active\def%{\%}84,747}}%
\end{pgfscope}%
\begin{pgfscope}%
\definecolor{textcolor}{rgb}{1.000000,1.000000,1.000000}%
\pgfsetstrokecolor{textcolor}%
\pgfsetfillcolor{textcolor}%
\pgftext[x=1.098463in,y=0.731078in,,]{\color{textcolor}{\ifdefined\pdftexversion\else\setmainfont{NanumMyeongjo}\rmfamily\fi\fontsize{5.000000}{6.000000}\selectfont\catcode`\^=\active\def^{\ifmmode\sp\else\^{}\fi}\catcode`\%=\active\def%{\%}7,244}}%
\end{pgfscope}%
\begin{pgfscope}%
\definecolor{textcolor}{rgb}{1.000000,1.000000,1.000000}%
\pgfsetstrokecolor{textcolor}%
\pgfsetfillcolor{textcolor}%
\pgftext[x=1.456895in,y=0.686656in,,]{\color{textcolor}{\ifdefined\pdftexversion\else\setmainfont{NanumMyeongjo}\rmfamily\fi\fontsize{5.000000}{6.000000}\selectfont\catcode`\^=\active\def^{\ifmmode\sp\else\^{}\fi}\catcode`\%=\active\def%{\%}5,199}}%
\end{pgfscope}%
\begin{pgfscope}%
\definecolor{textcolor}{rgb}{1.000000,1.000000,1.000000}%
\pgfsetstrokecolor{textcolor}%
\pgfsetfillcolor{textcolor}%
\pgftext[x=1.815328in,y=0.681269in,,]{\color{textcolor}{\ifdefined\pdftexversion\else\setmainfont{NanumMyeongjo}\rmfamily\fi\fontsize{5.000000}{6.000000}\selectfont\catcode`\^=\active\def^{\ifmmode\sp\else\^{}\fi}\catcode`\%=\active\def%{\%}4,951}}%
\end{pgfscope}%
\begin{pgfscope}%
\definecolor{textcolor}{rgb}{1.000000,1.000000,1.000000}%
\pgfsetstrokecolor{textcolor}%
\pgfsetfillcolor{textcolor}%
\pgftext[x=2.173760in,y=0.705772in,,]{\color{textcolor}{\ifdefined\pdftexversion\else\setmainfont{NanumMyeongjo}\rmfamily\fi\fontsize{5.000000}{6.000000}\selectfont\catcode`\^=\active\def^{\ifmmode\sp\else\^{}\fi}\catcode`\%=\active\def%{\%}6,079}}%
\end{pgfscope}%
\begin{pgfscope}%
\definecolor{textcolor}{rgb}{1.000000,1.000000,1.000000}%
\pgfsetstrokecolor{textcolor}%
\pgfsetfillcolor{textcolor}%
\pgftext[x=2.532193in,y=0.766681in,,]{\color{textcolor}{\ifdefined\pdftexversion\else\setmainfont{NanumMyeongjo}\rmfamily\fi\fontsize{5.000000}{6.000000}\selectfont\catcode`\^=\active\def^{\ifmmode\sp\else\^{}\fi}\catcode`\%=\active\def%{\%}8,883}}%
\end{pgfscope}%
\begin{pgfscope}%
\definecolor{textcolor}{rgb}{1.000000,1.000000,1.000000}%
\pgfsetstrokecolor{textcolor}%
\pgfsetfillcolor{textcolor}%
\pgftext[x=2.890625in,y=0.843882in,,]{\color{textcolor}{\ifdefined\pdftexversion\else\setmainfont{NanumMyeongjo}\rmfamily\fi\fontsize{5.000000}{6.000000}\selectfont\catcode`\^=\active\def^{\ifmmode\sp\else\^{}\fi}\catcode`\%=\active\def%{\%}12,437}}%
\end{pgfscope}%
\begin{pgfscope}%
\definecolor{textcolor}{rgb}{1.000000,1.000000,1.000000}%
\pgfsetstrokecolor{textcolor}%
\pgfsetfillcolor{textcolor}%
\pgftext[x=3.249057in,y=0.827307in,,]{\color{textcolor}{\ifdefined\pdftexversion\else\setmainfont{NanumMyeongjo}\rmfamily\fi\fontsize{5.000000}{6.000000}\selectfont\catcode`\^=\active\def^{\ifmmode\sp\else\^{}\fi}\catcode`\%=\active\def%{\%}11,674}}%
\end{pgfscope}%
\begin{pgfscope}%
\definecolor{textcolor}{rgb}{1.000000,1.000000,1.000000}%
\pgfsetstrokecolor{textcolor}%
\pgfsetfillcolor{textcolor}%
\pgftext[x=3.607490in,y=0.795332in,,]{\color{textcolor}{\ifdefined\pdftexversion\else\setmainfont{NanumMyeongjo}\rmfamily\fi\fontsize{5.000000}{6.000000}\selectfont\catcode`\^=\active\def^{\ifmmode\sp\else\^{}\fi}\catcode`\%=\active\def%{\%}10,202}}%
\end{pgfscope}%
\begin{pgfscope}%
\definecolor{textcolor}{rgb}{1.000000,1.000000,1.000000}%
\pgfsetstrokecolor{textcolor}%
\pgfsetfillcolor{textcolor}%
\pgftext[x=3.965922in,y=0.865864in,,]{\color{textcolor}{\ifdefined\pdftexversion\else\setmainfont{NanumMyeongjo}\rmfamily\fi\fontsize{5.000000}{6.000000}\selectfont\catcode`\^=\active\def^{\ifmmode\sp\else\^{}\fi}\catcode`\%=\active\def%{\%}13,449}}%
\end{pgfscope}%
\begin{pgfscope}%
\definecolor{textcolor}{rgb}{1.000000,1.000000,1.000000}%
\pgfsetstrokecolor{textcolor}%
\pgfsetfillcolor{textcolor}%
\pgftext[x=4.324355in,y=0.955708in,,]{\color{textcolor}{\ifdefined\pdftexversion\else\setmainfont{NanumMyeongjo}\rmfamily\fi\fontsize{5.000000}{6.000000}\selectfont\catcode`\^=\active\def^{\ifmmode\sp\else\^{}\fi}\catcode`\%=\active\def%{\%}17,585}}%
\end{pgfscope}%
\begin{pgfscope}%
\definecolor{textcolor}{rgb}{1.000000,1.000000,1.000000}%
\pgfsetstrokecolor{textcolor}%
\pgfsetfillcolor{textcolor}%
\pgftext[x=4.682787in,y=1.021656in,,]{\color{textcolor}{\ifdefined\pdftexversion\else\setmainfont{NanumMyeongjo}\rmfamily\fi\fontsize{5.000000}{6.000000}\selectfont\catcode`\^=\active\def^{\ifmmode\sp\else\^{}\fi}\catcode`\%=\active\def%{\%}20,621}}%
\end{pgfscope}%
\begin{pgfscope}%
\definecolor{textcolor}{rgb}{1.000000,1.000000,1.000000}%
\pgfsetstrokecolor{textcolor}%
\pgfsetfillcolor{textcolor}%
\pgftext[x=1.098463in,y=1.061386in,,]{\color{textcolor}{\ifdefined\pdftexversion\else\setmainfont{NanumMyeongjo}\rmfamily\fi\fontsize{5.000000}{6.000000}\selectfont\catcode`\^=\active\def^{\ifmmode\sp\else\^{}\fi}\catcode`\%=\active\def%{\%}15,206}}%
\end{pgfscope}%
\begin{pgfscope}%
\definecolor{textcolor}{rgb}{1.000000,1.000000,1.000000}%
\pgfsetstrokecolor{textcolor}%
\pgfsetfillcolor{textcolor}%
\pgftext[x=1.456895in,y=0.938569in,,]{\color{textcolor}{\ifdefined\pdftexversion\else\setmainfont{NanumMyeongjo}\rmfamily\fi\fontsize{5.000000}{6.000000}\selectfont\catcode`\^=\active\def^{\ifmmode\sp\else\^{}\fi}\catcode`\%=\active\def%{\%}11,597}}%
\end{pgfscope}%
\begin{pgfscope}%
\definecolor{textcolor}{rgb}{1.000000,1.000000,1.000000}%
\pgfsetstrokecolor{textcolor}%
\pgfsetfillcolor{textcolor}%
\pgftext[x=1.815328in,y=0.896623in,,]{\color{textcolor}{\ifdefined\pdftexversion\else\setmainfont{NanumMyeongjo}\rmfamily\fi\fontsize{5.000000}{6.000000}\selectfont\catcode`\^=\active\def^{\ifmmode\sp\else\^{}\fi}\catcode`\%=\active\def%{\%}9,914}}%
\end{pgfscope}%
\begin{pgfscope}%
\definecolor{textcolor}{rgb}{1.000000,1.000000,1.000000}%
\pgfsetstrokecolor{textcolor}%
\pgfsetfillcolor{textcolor}%
\pgftext[x=2.173760in,y=0.919323in,,]{\color{textcolor}{\ifdefined\pdftexversion\else\setmainfont{NanumMyeongjo}\rmfamily\fi\fontsize{5.000000}{6.000000}\selectfont\catcode`\^=\active\def^{\ifmmode\sp\else\^{}\fi}\catcode`\%=\active\def%{\%}9,831}}%
\end{pgfscope}%
\begin{pgfscope}%
\definecolor{textcolor}{rgb}{1.000000,1.000000,1.000000}%
\pgfsetstrokecolor{textcolor}%
\pgfsetfillcolor{textcolor}%
\pgftext[x=2.532193in,y=0.995676in,,]{\color{textcolor}{\ifdefined\pdftexversion\else\setmainfont{NanumMyeongjo}\rmfamily\fi\fontsize{5.000000}{6.000000}\selectfont\catcode`\^=\active\def^{\ifmmode\sp\else\^{}\fi}\catcode`\%=\active\def%{\%}10,542}}%
\end{pgfscope}%
\begin{pgfscope}%
\definecolor{textcolor}{rgb}{1.000000,1.000000,1.000000}%
\pgfsetstrokecolor{textcolor}%
\pgfsetfillcolor{textcolor}%
\pgftext[x=2.890625in,y=1.077656in,,]{\color{textcolor}{\ifdefined\pdftexversion\else\setmainfont{NanumMyeongjo}\rmfamily\fi\fontsize{5.000000}{6.000000}\selectfont\catcode`\^=\active\def^{\ifmmode\sp\else\^{}\fi}\catcode`\%=\active\def%{\%}10,762}}%
\end{pgfscope}%
\begin{pgfscope}%
\definecolor{textcolor}{rgb}{1.000000,1.000000,1.000000}%
\pgfsetstrokecolor{textcolor}%
\pgfsetfillcolor{textcolor}%
\pgftext[x=3.249057in,y=1.047093in,,]{\color{textcolor}{\ifdefined\pdftexversion\else\setmainfont{NanumMyeongjo}\rmfamily\fi\fontsize{5.000000}{6.000000}\selectfont\catcode`\^=\active\def^{\ifmmode\sp\else\^{}\fi}\catcode`\%=\active\def%{\%}10,118}}%
\end{pgfscope}%
\begin{pgfscope}%
\definecolor{textcolor}{rgb}{1.000000,1.000000,1.000000}%
\pgfsetstrokecolor{textcolor}%
\pgfsetfillcolor{textcolor}%
\pgftext[x=3.607490in,y=1.033256in,,]{\color{textcolor}{\ifdefined\pdftexversion\else\setmainfont{NanumMyeongjo}\rmfamily\fi\fontsize{5.000000}{6.000000}\selectfont\catcode`\^=\active\def^{\ifmmode\sp\else\^{}\fi}\catcode`\%=\active\def%{\%}10,953}}%
\end{pgfscope}%
\begin{pgfscope}%
\definecolor{textcolor}{rgb}{1.000000,1.000000,1.000000}%
\pgfsetstrokecolor{textcolor}%
\pgfsetfillcolor{textcolor}%
\pgftext[x=3.965922in,y=1.145777in,,]{\color{textcolor}{\ifdefined\pdftexversion\else\setmainfont{NanumMyeongjo}\rmfamily\fi\fontsize{5.000000}{6.000000}\selectfont\catcode`\^=\active\def^{\ifmmode\sp\else\^{}\fi}\catcode`\%=\active\def%{\%}12,886}}%
\end{pgfscope}%
\begin{pgfscope}%
\definecolor{textcolor}{rgb}{1.000000,1.000000,1.000000}%
\pgfsetstrokecolor{textcolor}%
\pgfsetfillcolor{textcolor}%
\pgftext[x=4.324355in,y=1.217743in,,]{\color{textcolor}{\ifdefined\pdftexversion\else\setmainfont{NanumMyeongjo}\rmfamily\fi\fontsize{5.000000}{6.000000}\selectfont\catcode`\^=\active\def^{\ifmmode\sp\else\^{}\fi}\catcode`\%=\active\def%{\%}12,063}}%
\end{pgfscope}%
\begin{pgfscope}%
\definecolor{textcolor}{rgb}{1.000000,1.000000,1.000000}%
\pgfsetstrokecolor{textcolor}%
\pgfsetfillcolor{textcolor}%
\pgftext[x=4.682787in,y=1.307455in,,]{\color{textcolor}{\ifdefined\pdftexversion\else\setmainfont{NanumMyeongjo}\rmfamily\fi\fontsize{5.000000}{6.000000}\selectfont\catcode`\^=\active\def^{\ifmmode\sp\else\^{}\fi}\catcode`\%=\active\def%{\%}13,157}}%
\end{pgfscope}%
\begin{pgfscope}%
\definecolor{textcolor}{rgb}{1.000000,1.000000,1.000000}%
\pgfsetstrokecolor{textcolor}%
\pgfsetfillcolor{textcolor}%
\pgftext[x=1.098463in,y=1.407747in,,]{\color{textcolor}{\ifdefined\pdftexversion\else\setmainfont{NanumMyeongjo}\rmfamily\fi\fontsize{5.000000}{6.000000}\selectfont\catcode`\^=\active\def^{\ifmmode\sp\else\^{}\fi}\catcode`\%=\active\def%{\%}15,945}}%
\end{pgfscope}%
\begin{pgfscope}%
\definecolor{textcolor}{rgb}{1.000000,1.000000,1.000000}%
\pgfsetstrokecolor{textcolor}%
\pgfsetfillcolor{textcolor}%
\pgftext[x=1.456895in,y=1.228951in,,]{\color{textcolor}{\ifdefined\pdftexversion\else\setmainfont{NanumMyeongjo}\rmfamily\fi\fontsize{5.000000}{6.000000}\selectfont\catcode`\^=\active\def^{\ifmmode\sp\else\^{}\fi}\catcode`\%=\active\def%{\%}13,368}}%
\end{pgfscope}%
\begin{pgfscope}%
\definecolor{textcolor}{rgb}{1.000000,1.000000,1.000000}%
\pgfsetstrokecolor{textcolor}%
\pgfsetfillcolor{textcolor}%
\pgftext[x=1.815328in,y=1.145104in,,]{\color{textcolor}{\ifdefined\pdftexversion\else\setmainfont{NanumMyeongjo}\rmfamily\fi\fontsize{5.000000}{6.000000}\selectfont\catcode`\^=\active\def^{\ifmmode\sp\else\^{}\fi}\catcode`\%=\active\def%{\%}11,439}}%
\end{pgfscope}%
\begin{pgfscope}%
\definecolor{textcolor}{rgb}{1.000000,1.000000,1.000000}%
\pgfsetstrokecolor{textcolor}%
\pgfsetfillcolor{textcolor}%
\pgftext[x=2.173760in,y=1.143931in,,]{\color{textcolor}{\ifdefined\pdftexversion\else\setmainfont{NanumMyeongjo}\rmfamily\fi\fontsize{5.000000}{6.000000}\selectfont\catcode`\^=\active\def^{\ifmmode\sp\else\^{}\fi}\catcode`\%=\active\def%{\%}10,340}}%
\end{pgfscope}%
\begin{pgfscope}%
\definecolor{textcolor}{rgb}{1.000000,1.000000,1.000000}%
\pgfsetstrokecolor{textcolor}%
\pgfsetfillcolor{textcolor}%
\pgftext[x=2.532193in,y=1.242962in,,]{\color{textcolor}{\ifdefined\pdftexversion\else\setmainfont{NanumMyeongjo}\rmfamily\fi\fontsize{5.000000}{6.000000}\selectfont\catcode`\^=\active\def^{\ifmmode\sp\else\^{}\fi}\catcode`\%=\active\def%{\%}11,384}}%
\end{pgfscope}%
\begin{pgfscope}%
\definecolor{textcolor}{rgb}{1.000000,1.000000,1.000000}%
\pgfsetstrokecolor{textcolor}%
\pgfsetfillcolor{textcolor}%
\pgftext[x=2.890625in,y=1.342168in,,]{\color{textcolor}{\ifdefined\pdftexversion\else\setmainfont{NanumMyeongjo}\rmfamily\fi\fontsize{5.000000}{6.000000}\selectfont\catcode`\^=\active\def^{\ifmmode\sp\else\^{}\fi}\catcode`\%=\active\def%{\%}12,177}}%
\end{pgfscope}%
\begin{pgfscope}%
\definecolor{textcolor}{rgb}{1.000000,1.000000,1.000000}%
\pgfsetstrokecolor{textcolor}%
\pgfsetfillcolor{textcolor}%
\pgftext[x=3.249057in,y=1.262990in,,]{\color{textcolor}{\ifdefined\pdftexversion\else\setmainfont{NanumMyeongjo}\rmfamily\fi\fontsize{5.000000}{6.000000}\selectfont\catcode`\^=\active\def^{\ifmmode\sp\else\^{}\fi}\catcode`\%=\active\def%{\%}9,939}}%
\end{pgfscope}%
\begin{pgfscope}%
\definecolor{textcolor}{rgb}{1.000000,1.000000,1.000000}%
\pgfsetstrokecolor{textcolor}%
\pgfsetfillcolor{textcolor}%
\pgftext[x=3.607490in,y=1.236989in,,]{\color{textcolor}{\ifdefined\pdftexversion\else\setmainfont{NanumMyeongjo}\rmfamily\fi\fontsize{5.000000}{6.000000}\selectfont\catcode`\^=\active\def^{\ifmmode\sp\else\^{}\fi}\catcode`\%=\active\def%{\%}9,379}}%
\end{pgfscope}%
\begin{pgfscope}%
\definecolor{textcolor}{rgb}{1.000000,1.000000,1.000000}%
\pgfsetstrokecolor{textcolor}%
\pgfsetfillcolor{textcolor}%
\pgftext[x=3.965922in,y=1.384113in,,]{\color{textcolor}{\ifdefined\pdftexversion\else\setmainfont{NanumMyeongjo}\rmfamily\fi\fontsize{5.000000}{6.000000}\selectfont\catcode`\^=\active\def^{\ifmmode\sp\else\^{}\fi}\catcode`\%=\active\def%{\%}10,972}}%
\end{pgfscope}%
\begin{pgfscope}%
\definecolor{textcolor}{rgb}{1.000000,1.000000,1.000000}%
\pgfsetstrokecolor{textcolor}%
\pgfsetfillcolor{textcolor}%
\pgftext[x=4.324355in,y=1.448889in,,]{\color{textcolor}{\ifdefined\pdftexversion\else\setmainfont{NanumMyeongjo}\rmfamily\fi\fontsize{5.000000}{6.000000}\selectfont\catcode`\^=\active\def^{\ifmmode\sp\else\^{}\fi}\catcode`\%=\active\def%{\%}10,641}}%
\end{pgfscope}%
\begin{pgfscope}%
\definecolor{textcolor}{rgb}{1.000000,1.000000,1.000000}%
\pgfsetstrokecolor{textcolor}%
\pgfsetfillcolor{textcolor}%
\pgftext[x=4.682787in,y=1.573726in,,]{\color{textcolor}{\ifdefined\pdftexversion\else\setmainfont{NanumMyeongjo}\rmfamily\fi\fontsize{5.000000}{6.000000}\selectfont\catcode`\^=\active\def^{\ifmmode\sp\else\^{}\fi}\catcode`\%=\active\def%{\%}12,258}}%
\end{pgfscope}%
\begin{pgfscope}%
\pgfsetbuttcap%
\pgfsetmiterjoin%
\definecolor{currentfill}{rgb}{0.337255,0.713725,0.627451}%
\pgfsetfillcolor{currentfill}%
\pgfsetlinewidth{1.003750pt}%
\definecolor{currentstroke}{rgb}{0.266667,0.266667,0.266667}%
\pgfsetstrokecolor{currentstroke}%
\pgfsetdash{}{0pt}%
\pgfpathmoveto{\pgfqpoint{5.028125in}{2.598758in}}%
\pgfpathlineto{\pgfqpoint{5.278125in}{2.598758in}}%
\pgfpathlineto{\pgfqpoint{5.278125in}{2.686258in}}%
\pgfpathlineto{\pgfqpoint{5.028125in}{2.686258in}}%
\pgfpathlineto{\pgfqpoint{5.028125in}{2.598758in}}%
\pgfpathclose%
\pgfusepath{stroke,fill}%
\end{pgfscope}%
\begin{pgfscope}%
\definecolor{textcolor}{rgb}{0.000000,0.000000,0.000000}%
\pgfsetstrokecolor{textcolor}%
\pgfsetfillcolor{textcolor}%
\pgftext[x=5.378125in,y=2.598758in,left,base]{\color{textcolor}{\ifdefined\pdftexversion\else\setmainfont{NanumMyeongjo}\rmfamily\fi\fontsize{9.000000}{10.800000}\selectfont\catcode`\^=\active\def^{\ifmmode\sp\else\^{}\fi}\catcode`\%=\active\def%{\%}전라북도}}%
\end{pgfscope}%
\begin{pgfscope}%
\pgfsetbuttcap%
\pgfsetmiterjoin%
\definecolor{currentfill}{rgb}{0.235294,0.490196,0.764706}%
\pgfsetfillcolor{currentfill}%
\pgfsetlinewidth{1.003750pt}%
\definecolor{currentstroke}{rgb}{0.266667,0.266667,0.266667}%
\pgfsetstrokecolor{currentstroke}%
\pgfsetdash{}{0pt}%
\pgfpathmoveto{\pgfqpoint{5.028125in}{2.407474in}}%
\pgfpathlineto{\pgfqpoint{5.278125in}{2.407474in}}%
\pgfpathlineto{\pgfqpoint{5.278125in}{2.494974in}}%
\pgfpathlineto{\pgfqpoint{5.028125in}{2.494974in}}%
\pgfpathlineto{\pgfqpoint{5.028125in}{2.407474in}}%
\pgfpathclose%
\pgfusepath{stroke,fill}%
\end{pgfscope}%
\begin{pgfscope}%
\definecolor{textcolor}{rgb}{0.000000,0.000000,0.000000}%
\pgfsetstrokecolor{textcolor}%
\pgfsetfillcolor{textcolor}%
\pgftext[x=5.378125in,y=2.407474in,left,base]{\color{textcolor}{\ifdefined\pdftexversion\else\setmainfont{NanumMyeongjo}\rmfamily\fi\fontsize{9.000000}{10.800000}\selectfont\catcode`\^=\active\def^{\ifmmode\sp\else\^{}\fi}\catcode`\%=\active\def%{\%}경상북도}}%
\end{pgfscope}%
\begin{pgfscope}%
\pgfsetbuttcap%
\pgfsetmiterjoin%
\definecolor{currentfill}{rgb}{0.725490,0.486275,0.164706}%
\pgfsetfillcolor{currentfill}%
\pgfsetlinewidth{1.003750pt}%
\definecolor{currentstroke}{rgb}{0.266667,0.266667,0.266667}%
\pgfsetstrokecolor{currentstroke}%
\pgfsetdash{}{0pt}%
\pgfpathmoveto{\pgfqpoint{5.028125in}{2.216190in}}%
\pgfpathlineto{\pgfqpoint{5.278125in}{2.216190in}}%
\pgfpathlineto{\pgfqpoint{5.278125in}{2.303690in}}%
\pgfpathlineto{\pgfqpoint{5.028125in}{2.303690in}}%
\pgfpathlineto{\pgfqpoint{5.028125in}{2.216190in}}%
\pgfpathclose%
\pgfusepath{stroke,fill}%
\end{pgfscope}%
\begin{pgfscope}%
\definecolor{textcolor}{rgb}{0.000000,0.000000,0.000000}%
\pgfsetstrokecolor{textcolor}%
\pgfsetfillcolor{textcolor}%
\pgftext[x=5.378125in,y=2.216190in,left,base]{\color{textcolor}{\ifdefined\pdftexversion\else\setmainfont{NanumMyeongjo}\rmfamily\fi\fontsize{9.000000}{10.800000}\selectfont\catcode`\^=\active\def^{\ifmmode\sp\else\^{}\fi}\catcode`\%=\active\def%{\%}전라남도}}%
\end{pgfscope}%
\begin{pgfscope}%
\pgfsetbuttcap%
\pgfsetmiterjoin%
\definecolor{currentfill}{rgb}{0.733333,0.321569,0.733333}%
\pgfsetfillcolor{currentfill}%
\pgfsetlinewidth{1.003750pt}%
\definecolor{currentstroke}{rgb}{0.266667,0.266667,0.266667}%
\pgfsetstrokecolor{currentstroke}%
\pgfsetdash{}{0pt}%
\pgfpathmoveto{\pgfqpoint{5.028125in}{2.024906in}}%
\pgfpathlineto{\pgfqpoint{5.278125in}{2.024906in}}%
\pgfpathlineto{\pgfqpoint{5.278125in}{2.112406in}}%
\pgfpathlineto{\pgfqpoint{5.028125in}{2.112406in}}%
\pgfpathlineto{\pgfqpoint{5.028125in}{2.024906in}}%
\pgfpathclose%
\pgfusepath{stroke,fill}%
\end{pgfscope}%
\begin{pgfscope}%
\definecolor{textcolor}{rgb}{0.000000,0.000000,0.000000}%
\pgfsetstrokecolor{textcolor}%
\pgfsetfillcolor{textcolor}%
\pgftext[x=5.378125in,y=2.024906in,left,base]{\color{textcolor}{\ifdefined\pdftexversion\else\setmainfont{NanumMyeongjo}\rmfamily\fi\fontsize{9.000000}{10.800000}\selectfont\catcode`\^=\active\def^{\ifmmode\sp\else\^{}\fi}\catcode`\%=\active\def%{\%}충청남도}}%
\end{pgfscope}%
\begin{pgfscope}%
\pgfsetbuttcap%
\pgfsetmiterjoin%
\definecolor{currentfill}{rgb}{0.549020,0.247059,0.121569}%
\pgfsetfillcolor{currentfill}%
\pgfsetlinewidth{1.003750pt}%
\definecolor{currentstroke}{rgb}{0.266667,0.266667,0.266667}%
\pgfsetstrokecolor{currentstroke}%
\pgfsetdash{}{0pt}%
\pgfpathmoveto{\pgfqpoint{5.028125in}{1.833622in}}%
\pgfpathlineto{\pgfqpoint{5.278125in}{1.833622in}}%
\pgfpathlineto{\pgfqpoint{5.278125in}{1.921122in}}%
\pgfpathlineto{\pgfqpoint{5.028125in}{1.921122in}}%
\pgfpathlineto{\pgfqpoint{5.028125in}{1.833622in}}%
\pgfpathclose%
\pgfusepath{stroke,fill}%
\end{pgfscope}%
\begin{pgfscope}%
\definecolor{textcolor}{rgb}{0.000000,0.000000,0.000000}%
\pgfsetstrokecolor{textcolor}%
\pgfsetfillcolor{textcolor}%
\pgftext[x=5.378125in,y=1.833622in,left,base]{\color{textcolor}{\ifdefined\pdftexversion\else\setmainfont{NanumMyeongjo}\rmfamily\fi\fontsize{9.000000}{10.800000}\selectfont\catcode`\^=\active\def^{\ifmmode\sp\else\^{}\fi}\catcode`\%=\active\def%{\%}충청북도}}%
\end{pgfscope}%
\begin{pgfscope}%
\pgfsetbuttcap%
\pgfsetmiterjoin%
\definecolor{currentfill}{rgb}{0.701961,0.760784,0.360784}%
\pgfsetfillcolor{currentfill}%
\pgfsetlinewidth{1.003750pt}%
\definecolor{currentstroke}{rgb}{0.266667,0.266667,0.266667}%
\pgfsetstrokecolor{currentstroke}%
\pgfsetdash{}{0pt}%
\pgfpathmoveto{\pgfqpoint{5.028125in}{1.642338in}}%
\pgfpathlineto{\pgfqpoint{5.278125in}{1.642338in}}%
\pgfpathlineto{\pgfqpoint{5.278125in}{1.729837in}}%
\pgfpathlineto{\pgfqpoint{5.028125in}{1.729837in}}%
\pgfpathlineto{\pgfqpoint{5.028125in}{1.642338in}}%
\pgfpathclose%
\pgfusepath{stroke,fill}%
\end{pgfscope}%
\begin{pgfscope}%
\definecolor{textcolor}{rgb}{0.000000,0.000000,0.000000}%
\pgfsetstrokecolor{textcolor}%
\pgfsetfillcolor{textcolor}%
\pgftext[x=5.378125in,y=1.642337in,left,base]{\color{textcolor}{\ifdefined\pdftexversion\else\setmainfont{NanumMyeongjo}\rmfamily\fi\fontsize{9.000000}{10.800000}\selectfont\catcode`\^=\active\def^{\ifmmode\sp\else\^{}\fi}\catcode`\%=\active\def%{\%}경기도}}%
\end{pgfscope}%
\begin{pgfscope}%
\pgfsetbuttcap%
\pgfsetmiterjoin%
\definecolor{currentfill}{rgb}{0.447059,0.447059,0.447059}%
\pgfsetfillcolor{currentfill}%
\pgfsetlinewidth{1.003750pt}%
\definecolor{currentstroke}{rgb}{0.266667,0.266667,0.266667}%
\pgfsetstrokecolor{currentstroke}%
\pgfsetdash{}{0pt}%
\pgfpathmoveto{\pgfqpoint{5.028125in}{1.451053in}}%
\pgfpathlineto{\pgfqpoint{5.278125in}{1.451053in}}%
\pgfpathlineto{\pgfqpoint{5.278125in}{1.538553in}}%
\pgfpathlineto{\pgfqpoint{5.028125in}{1.538553in}}%
\pgfpathlineto{\pgfqpoint{5.028125in}{1.451053in}}%
\pgfpathclose%
\pgfusepath{stroke,fill}%
\end{pgfscope}%
\begin{pgfscope}%
\definecolor{textcolor}{rgb}{0.000000,0.000000,0.000000}%
\pgfsetstrokecolor{textcolor}%
\pgfsetfillcolor{textcolor}%
\pgftext[x=5.378125in,y=1.451053in,left,base]{\color{textcolor}{\ifdefined\pdftexversion\else\setmainfont{NanumMyeongjo}\rmfamily\fi\fontsize{9.000000}{10.800000}\selectfont\catcode`\^=\active\def^{\ifmmode\sp\else\^{}\fi}\catcode`\%=\active\def%{\%}기타}}%
\end{pgfscope}%
\begin{pgfscope}%
\definecolor{textcolor}{rgb}{0.333333,0.333333,0.333333}%
\pgfsetstrokecolor{textcolor}%
\pgfsetfillcolor{textcolor}%
\pgftext[x=2.062500in,y=0.159722in,,top]{\color{textcolor}{\ifdefined\pdftexversion\else\setmainfont{NanumMyeongjo}\rmfamily\fi\fontsize{9.000000}{10.800000}\selectfont\catcode`\^=\active\def^{\ifmmode\sp\else\^{}\fi}\catcode`\%=\active\def%{\%}출처: 국가농식품통계서비스(KASS) 자료 기반 저자 작성}}%
\end{pgfscope}%
\begin{pgfscope}%
\definecolor{textcolor}{rgb}{0.333333,0.333333,0.333333}%
\pgfsetstrokecolor{textcolor}%
\pgfsetfillcolor{textcolor}%
\pgftext[x=4.687500in,y=3.034722in,,top]{\color{textcolor}{\ifdefined\pdftexversion\else\setmainfont{NanumMyeongjo}\rmfamily\fi\fontsize{9.000000}{10.800000}\selectfont\catcode`\^=\active\def^{\ifmmode\sp\else\^{}\fi}\catcode`\%=\active\def%{\%}(단위: ha)}}%
\end{pgfscope}%
\end{pgfpicture}%
\makeatother%
\endgroup%
}
\end{center}
}


\slide
{\maintitle}
{\chapterthree}
{주요 지역 콩 생산 논/밭 면적}{
\begin{center}
    \hspace*{-30pt}{%% Creator: Matplotlib, PGF backend
%%
%% To include the figure in your LaTeX document, write
%%   \input{<filename>.pgf}
%%
%% Make sure the required packages are loaded in your preamble
%%   \usepackage{pgf}
%%
%% Also ensure that all the required font packages are loaded; for instance,
%% the lmodern package is sometimes necessary when using math font.
%%   \usepackage{lmodern}
%%
%% Figures using additional raster images can only be included by \input if
%% they are in the same directory as the main LaTeX file. For loading figures
%% from other directories you can use the `import` package
%%   \usepackage{import}
%%
%% and then include the figures with
%%   \import{<path to file>}{<filename>.pgf}
%%
%% Matplotlib used the following preamble
%%   \def\mathdefault#1{#1}
%%   \everymath=\expandafter{\the\everymath\displaystyle}
%%   \IfFileExists{scrextend.sty}{
%%     \usepackage[fontsize=5.000000pt]{scrextend}
%%   }{
%%     \renewcommand{\normalsize}{\fontsize{5.000000}{6.000000}\selectfont}
%%     \normalsize
%%   }
%%   
%%   \ifdefined\pdftexversion\else  % non-pdftex case.
%%     \usepackage{fontspec}
%%     \setmainfont{DejaVuSerif.ttf}[Path=\detokenize{/home/user/.cache/pypoetry/virtualenvs/graph-KASAOWVd-py3.12/lib/python3.12/site-packages/matplotlib/mpl-data/fonts/ttf/}]
%%     \setsansfont{DejaVuSans.ttf}[Path=\detokenize{/home/user/.cache/pypoetry/virtualenvs/graph-KASAOWVd-py3.12/lib/python3.12/site-packages/matplotlib/mpl-data/fonts/ttf/}]
%%     \setmonofont{DejaVuSansMono.ttf}[Path=\detokenize{/home/user/.cache/pypoetry/virtualenvs/graph-KASAOWVd-py3.12/lib/python3.12/site-packages/matplotlib/mpl-data/fonts/ttf/}]
%%   \fi
%%   \makeatletter\@ifpackageloaded{underscore}{}{\usepackage[strings]{underscore}}\makeatother
%%
\begingroup%
\makeatletter%
\begin{pgfpicture}%
\pgfpathrectangle{\pgfpointorigin}{\pgfqpoint{5.555556in}{2.083333in}}%
\pgfusepath{use as bounding box, clip}%
\begin{pgfscope}%
\pgfsetbuttcap%
\pgfsetmiterjoin%
\definecolor{currentfill}{rgb}{1.000000,1.000000,1.000000}%
\pgfsetfillcolor{currentfill}%
\pgfsetlinewidth{0.000000pt}%
\definecolor{currentstroke}{rgb}{1.000000,1.000000,1.000000}%
\pgfsetstrokecolor{currentstroke}%
\pgfsetdash{}{0pt}%
\pgfpathmoveto{\pgfqpoint{0.000000in}{0.000000in}}%
\pgfpathlineto{\pgfqpoint{5.555556in}{0.000000in}}%
\pgfpathlineto{\pgfqpoint{5.555556in}{2.083333in}}%
\pgfpathlineto{\pgfqpoint{0.000000in}{2.083333in}}%
\pgfpathlineto{\pgfqpoint{0.000000in}{0.000000in}}%
\pgfpathclose%
\pgfusepath{fill}%
\end{pgfscope}%
\begin{pgfscope}%
\pgfsetbuttcap%
\pgfsetmiterjoin%
\definecolor{currentfill}{rgb}{1.000000,1.000000,1.000000}%
\pgfsetfillcolor{currentfill}%
\pgfsetlinewidth{0.000000pt}%
\definecolor{currentstroke}{rgb}{0.000000,0.000000,0.000000}%
\pgfsetstrokecolor{currentstroke}%
\pgfsetstrokeopacity{0.000000}%
\pgfsetdash{}{0pt}%
\pgfpathmoveto{\pgfqpoint{0.694444in}{0.416667in}}%
\pgfpathlineto{\pgfqpoint{4.444444in}{0.416667in}}%
\pgfpathlineto{\pgfqpoint{4.444444in}{1.833333in}}%
\pgfpathlineto{\pgfqpoint{0.694444in}{1.833333in}}%
\pgfpathlineto{\pgfqpoint{0.694444in}{0.416667in}}%
\pgfpathclose%
\pgfusepath{fill}%
\end{pgfscope}%
\begin{pgfscope}%
\pgfsetbuttcap%
\pgfsetroundjoin%
\definecolor{currentfill}{rgb}{0.000000,0.000000,0.000000}%
\pgfsetfillcolor{currentfill}%
\pgfsetlinewidth{0.803000pt}%
\definecolor{currentstroke}{rgb}{0.000000,0.000000,0.000000}%
\pgfsetstrokecolor{currentstroke}%
\pgfsetdash{}{0pt}%
\pgfsys@defobject{currentmarker}{\pgfqpoint{0.000000in}{-0.048611in}}{\pgfqpoint{0.000000in}{0.000000in}}{%
\pgfpathmoveto{\pgfqpoint{0.000000in}{0.000000in}}%
\pgfpathlineto{\pgfqpoint{0.000000in}{-0.048611in}}%
\pgfusepath{stroke,fill}%
}%
\begin{pgfscope}%
\pgfsys@transformshift{1.187381in}{0.416667in}%
\pgfsys@useobject{currentmarker}{}%
\end{pgfscope}%
\end{pgfscope}%
\begin{pgfscope}%
\definecolor{textcolor}{rgb}{0.000000,0.000000,0.000000}%
\pgfsetstrokecolor{textcolor}%
\pgfsetfillcolor{textcolor}%
\pgftext[x=1.187381in,y=0.319444in,,top]{\color{textcolor}{\ifdefined\pdftexversion\else\setmainfont{NanumMyeongjo}\rmfamily\fi\fontsize{5.000000}{6.000000}\selectfont\catcode`\^=\active\def^{\ifmmode\sp\else\^{}\fi}\catcode`\%=\active\def%{\%}2020}}%
\end{pgfscope}%
\begin{pgfscope}%
\pgfsetbuttcap%
\pgfsetroundjoin%
\definecolor{currentfill}{rgb}{0.000000,0.000000,0.000000}%
\pgfsetfillcolor{currentfill}%
\pgfsetlinewidth{0.803000pt}%
\definecolor{currentstroke}{rgb}{0.000000,0.000000,0.000000}%
\pgfsetstrokecolor{currentstroke}%
\pgfsetdash{}{0pt}%
\pgfsys@defobject{currentmarker}{\pgfqpoint{0.000000in}{-0.048611in}}{\pgfqpoint{0.000000in}{0.000000in}}{%
\pgfpathmoveto{\pgfqpoint{0.000000in}{0.000000in}}%
\pgfpathlineto{\pgfqpoint{0.000000in}{-0.048611in}}%
\pgfusepath{stroke,fill}%
}%
\begin{pgfscope}%
\pgfsys@transformshift{2.108756in}{0.416667in}%
\pgfsys@useobject{currentmarker}{}%
\end{pgfscope}%
\end{pgfscope}%
\begin{pgfscope}%
\definecolor{textcolor}{rgb}{0.000000,0.000000,0.000000}%
\pgfsetstrokecolor{textcolor}%
\pgfsetfillcolor{textcolor}%
\pgftext[x=2.108756in,y=0.319444in,,top]{\color{textcolor}{\ifdefined\pdftexversion\else\setmainfont{NanumMyeongjo}\rmfamily\fi\fontsize{5.000000}{6.000000}\selectfont\catcode`\^=\active\def^{\ifmmode\sp\else\^{}\fi}\catcode`\%=\active\def%{\%}2021}}%
\end{pgfscope}%
\begin{pgfscope}%
\pgfsetbuttcap%
\pgfsetroundjoin%
\definecolor{currentfill}{rgb}{0.000000,0.000000,0.000000}%
\pgfsetfillcolor{currentfill}%
\pgfsetlinewidth{0.803000pt}%
\definecolor{currentstroke}{rgb}{0.000000,0.000000,0.000000}%
\pgfsetstrokecolor{currentstroke}%
\pgfsetdash{}{0pt}%
\pgfsys@defobject{currentmarker}{\pgfqpoint{0.000000in}{-0.048611in}}{\pgfqpoint{0.000000in}{0.000000in}}{%
\pgfpathmoveto{\pgfqpoint{0.000000in}{0.000000in}}%
\pgfpathlineto{\pgfqpoint{0.000000in}{-0.048611in}}%
\pgfusepath{stroke,fill}%
}%
\begin{pgfscope}%
\pgfsys@transformshift{3.030132in}{0.416667in}%
\pgfsys@useobject{currentmarker}{}%
\end{pgfscope}%
\end{pgfscope}%
\begin{pgfscope}%
\definecolor{textcolor}{rgb}{0.000000,0.000000,0.000000}%
\pgfsetstrokecolor{textcolor}%
\pgfsetfillcolor{textcolor}%
\pgftext[x=3.030132in,y=0.319444in,,top]{\color{textcolor}{\ifdefined\pdftexversion\else\setmainfont{NanumMyeongjo}\rmfamily\fi\fontsize{5.000000}{6.000000}\selectfont\catcode`\^=\active\def^{\ifmmode\sp\else\^{}\fi}\catcode`\%=\active\def%{\%}2022}}%
\end{pgfscope}%
\begin{pgfscope}%
\pgfsetbuttcap%
\pgfsetroundjoin%
\definecolor{currentfill}{rgb}{0.000000,0.000000,0.000000}%
\pgfsetfillcolor{currentfill}%
\pgfsetlinewidth{0.803000pt}%
\definecolor{currentstroke}{rgb}{0.000000,0.000000,0.000000}%
\pgfsetstrokecolor{currentstroke}%
\pgfsetdash{}{0pt}%
\pgfsys@defobject{currentmarker}{\pgfqpoint{0.000000in}{-0.048611in}}{\pgfqpoint{0.000000in}{0.000000in}}{%
\pgfpathmoveto{\pgfqpoint{0.000000in}{0.000000in}}%
\pgfpathlineto{\pgfqpoint{0.000000in}{-0.048611in}}%
\pgfusepath{stroke,fill}%
}%
\begin{pgfscope}%
\pgfsys@transformshift{3.951508in}{0.416667in}%
\pgfsys@useobject{currentmarker}{}%
\end{pgfscope}%
\end{pgfscope}%
\begin{pgfscope}%
\definecolor{textcolor}{rgb}{0.000000,0.000000,0.000000}%
\pgfsetstrokecolor{textcolor}%
\pgfsetfillcolor{textcolor}%
\pgftext[x=3.951508in,y=0.319444in,,top]{\color{textcolor}{\ifdefined\pdftexversion\else\setmainfont{NanumMyeongjo}\rmfamily\fi\fontsize{5.000000}{6.000000}\selectfont\catcode`\^=\active\def^{\ifmmode\sp\else\^{}\fi}\catcode`\%=\active\def%{\%}2023}}%
\end{pgfscope}%
\begin{pgfscope}%
\pgfpathrectangle{\pgfqpoint{0.694444in}{0.416667in}}{\pgfqpoint{3.750000in}{1.416667in}}%
\pgfusepath{clip}%
\pgfsetbuttcap%
\pgfsetroundjoin%
\pgfsetlinewidth{0.602250pt}%
\definecolor{currentstroke}{rgb}{0.690196,0.690196,0.690196}%
\pgfsetstrokecolor{currentstroke}%
\pgfsetstrokeopacity{0.400000}%
\pgfsetdash{{2.220000pt}{0.960000pt}}{0.000000pt}%
\pgfpathmoveto{\pgfqpoint{0.694444in}{0.416667in}}%
\pgfpathlineto{\pgfqpoint{4.444444in}{0.416667in}}%
\pgfusepath{stroke}%
\end{pgfscope}%
\begin{pgfscope}%
\pgfsetbuttcap%
\pgfsetroundjoin%
\definecolor{currentfill}{rgb}{0.000000,0.000000,0.000000}%
\pgfsetfillcolor{currentfill}%
\pgfsetlinewidth{0.803000pt}%
\definecolor{currentstroke}{rgb}{0.000000,0.000000,0.000000}%
\pgfsetstrokecolor{currentstroke}%
\pgfsetdash{}{0pt}%
\pgfsys@defobject{currentmarker}{\pgfqpoint{-0.048611in}{0.000000in}}{\pgfqpoint{-0.000000in}{0.000000in}}{%
\pgfpathmoveto{\pgfqpoint{-0.000000in}{0.000000in}}%
\pgfpathlineto{\pgfqpoint{-0.048611in}{0.000000in}}%
\pgfusepath{stroke,fill}%
}%
\begin{pgfscope}%
\pgfsys@transformshift{0.694444in}{0.416667in}%
\pgfsys@useobject{currentmarker}{}%
\end{pgfscope}%
\end{pgfscope}%
\begin{pgfscope}%
\definecolor{textcolor}{rgb}{0.000000,0.000000,0.000000}%
\pgfsetstrokecolor{textcolor}%
\pgfsetfillcolor{textcolor}%
\pgftext[x=0.559923in, y=0.388930in, left, base]{\color{textcolor}{\ifdefined\pdftexversion\else\setmainfont{NanumMyeongjo}\rmfamily\fi\fontsize{5.000000}{6.000000}\selectfont\catcode`\^=\active\def^{\ifmmode\sp\else\^{}\fi}\catcode`\%=\active\def%{\%}0}}%
\end{pgfscope}%
\begin{pgfscope}%
\pgfpathrectangle{\pgfqpoint{0.694444in}{0.416667in}}{\pgfqpoint{3.750000in}{1.416667in}}%
\pgfusepath{clip}%
\pgfsetbuttcap%
\pgfsetroundjoin%
\pgfsetlinewidth{0.602250pt}%
\definecolor{currentstroke}{rgb}{0.690196,0.690196,0.690196}%
\pgfsetstrokecolor{currentstroke}%
\pgfsetstrokeopacity{0.400000}%
\pgfsetdash{{2.220000pt}{0.960000pt}}{0.000000pt}%
\pgfpathmoveto{\pgfqpoint{0.694444in}{0.570108in}}%
\pgfpathlineto{\pgfqpoint{4.444444in}{0.570108in}}%
\pgfusepath{stroke}%
\end{pgfscope}%
\begin{pgfscope}%
\pgfsetbuttcap%
\pgfsetroundjoin%
\definecolor{currentfill}{rgb}{0.000000,0.000000,0.000000}%
\pgfsetfillcolor{currentfill}%
\pgfsetlinewidth{0.803000pt}%
\definecolor{currentstroke}{rgb}{0.000000,0.000000,0.000000}%
\pgfsetstrokecolor{currentstroke}%
\pgfsetdash{}{0pt}%
\pgfsys@defobject{currentmarker}{\pgfqpoint{-0.048611in}{0.000000in}}{\pgfqpoint{-0.000000in}{0.000000in}}{%
\pgfpathmoveto{\pgfqpoint{-0.000000in}{0.000000in}}%
\pgfpathlineto{\pgfqpoint{-0.048611in}{0.000000in}}%
\pgfusepath{stroke,fill}%
}%
\begin{pgfscope}%
\pgfsys@transformshift{0.694444in}{0.570108in}%
\pgfsys@useobject{currentmarker}{}%
\end{pgfscope}%
\end{pgfscope}%
\begin{pgfscope}%
\definecolor{textcolor}{rgb}{0.000000,0.000000,0.000000}%
\pgfsetstrokecolor{textcolor}%
\pgfsetfillcolor{textcolor}%
\pgftext[x=0.429036in, y=0.542371in, left, base]{\color{textcolor}{\ifdefined\pdftexversion\else\setmainfont{NanumMyeongjo}\rmfamily\fi\fontsize{5.000000}{6.000000}\selectfont\catcode`\^=\active\def^{\ifmmode\sp\else\^{}\fi}\catcode`\%=\active\def%{\%}2,000}}%
\end{pgfscope}%
\begin{pgfscope}%
\pgfpathrectangle{\pgfqpoint{0.694444in}{0.416667in}}{\pgfqpoint{3.750000in}{1.416667in}}%
\pgfusepath{clip}%
\pgfsetbuttcap%
\pgfsetroundjoin%
\pgfsetlinewidth{0.602250pt}%
\definecolor{currentstroke}{rgb}{0.690196,0.690196,0.690196}%
\pgfsetstrokecolor{currentstroke}%
\pgfsetstrokeopacity{0.400000}%
\pgfsetdash{{2.220000pt}{0.960000pt}}{0.000000pt}%
\pgfpathmoveto{\pgfqpoint{0.694444in}{0.723549in}}%
\pgfpathlineto{\pgfqpoint{4.444444in}{0.723549in}}%
\pgfusepath{stroke}%
\end{pgfscope}%
\begin{pgfscope}%
\pgfsetbuttcap%
\pgfsetroundjoin%
\definecolor{currentfill}{rgb}{0.000000,0.000000,0.000000}%
\pgfsetfillcolor{currentfill}%
\pgfsetlinewidth{0.803000pt}%
\definecolor{currentstroke}{rgb}{0.000000,0.000000,0.000000}%
\pgfsetstrokecolor{currentstroke}%
\pgfsetdash{}{0pt}%
\pgfsys@defobject{currentmarker}{\pgfqpoint{-0.048611in}{0.000000in}}{\pgfqpoint{-0.000000in}{0.000000in}}{%
\pgfpathmoveto{\pgfqpoint{-0.000000in}{0.000000in}}%
\pgfpathlineto{\pgfqpoint{-0.048611in}{0.000000in}}%
\pgfusepath{stroke,fill}%
}%
\begin{pgfscope}%
\pgfsys@transformshift{0.694444in}{0.723549in}%
\pgfsys@useobject{currentmarker}{}%
\end{pgfscope}%
\end{pgfscope}%
\begin{pgfscope}%
\definecolor{textcolor}{rgb}{0.000000,0.000000,0.000000}%
\pgfsetstrokecolor{textcolor}%
\pgfsetfillcolor{textcolor}%
\pgftext[x=0.429036in, y=0.695812in, left, base]{\color{textcolor}{\ifdefined\pdftexversion\else\setmainfont{NanumMyeongjo}\rmfamily\fi\fontsize{5.000000}{6.000000}\selectfont\catcode`\^=\active\def^{\ifmmode\sp\else\^{}\fi}\catcode`\%=\active\def%{\%}4,000}}%
\end{pgfscope}%
\begin{pgfscope}%
\pgfpathrectangle{\pgfqpoint{0.694444in}{0.416667in}}{\pgfqpoint{3.750000in}{1.416667in}}%
\pgfusepath{clip}%
\pgfsetbuttcap%
\pgfsetroundjoin%
\pgfsetlinewidth{0.602250pt}%
\definecolor{currentstroke}{rgb}{0.690196,0.690196,0.690196}%
\pgfsetstrokecolor{currentstroke}%
\pgfsetstrokeopacity{0.400000}%
\pgfsetdash{{2.220000pt}{0.960000pt}}{0.000000pt}%
\pgfpathmoveto{\pgfqpoint{0.694444in}{0.876990in}}%
\pgfpathlineto{\pgfqpoint{4.444444in}{0.876990in}}%
\pgfusepath{stroke}%
\end{pgfscope}%
\begin{pgfscope}%
\pgfsetbuttcap%
\pgfsetroundjoin%
\definecolor{currentfill}{rgb}{0.000000,0.000000,0.000000}%
\pgfsetfillcolor{currentfill}%
\pgfsetlinewidth{0.803000pt}%
\definecolor{currentstroke}{rgb}{0.000000,0.000000,0.000000}%
\pgfsetstrokecolor{currentstroke}%
\pgfsetdash{}{0pt}%
\pgfsys@defobject{currentmarker}{\pgfqpoint{-0.048611in}{0.000000in}}{\pgfqpoint{-0.000000in}{0.000000in}}{%
\pgfpathmoveto{\pgfqpoint{-0.000000in}{0.000000in}}%
\pgfpathlineto{\pgfqpoint{-0.048611in}{0.000000in}}%
\pgfusepath{stroke,fill}%
}%
\begin{pgfscope}%
\pgfsys@transformshift{0.694444in}{0.876990in}%
\pgfsys@useobject{currentmarker}{}%
\end{pgfscope}%
\end{pgfscope}%
\begin{pgfscope}%
\definecolor{textcolor}{rgb}{0.000000,0.000000,0.000000}%
\pgfsetstrokecolor{textcolor}%
\pgfsetfillcolor{textcolor}%
\pgftext[x=0.429036in, y=0.849252in, left, base]{\color{textcolor}{\ifdefined\pdftexversion\else\setmainfont{NanumMyeongjo}\rmfamily\fi\fontsize{5.000000}{6.000000}\selectfont\catcode`\^=\active\def^{\ifmmode\sp\else\^{}\fi}\catcode`\%=\active\def%{\%}6,000}}%
\end{pgfscope}%
\begin{pgfscope}%
\pgfpathrectangle{\pgfqpoint{0.694444in}{0.416667in}}{\pgfqpoint{3.750000in}{1.416667in}}%
\pgfusepath{clip}%
\pgfsetbuttcap%
\pgfsetroundjoin%
\pgfsetlinewidth{0.602250pt}%
\definecolor{currentstroke}{rgb}{0.690196,0.690196,0.690196}%
\pgfsetstrokecolor{currentstroke}%
\pgfsetstrokeopacity{0.400000}%
\pgfsetdash{{2.220000pt}{0.960000pt}}{0.000000pt}%
\pgfpathmoveto{\pgfqpoint{0.694444in}{1.030431in}}%
\pgfpathlineto{\pgfqpoint{4.444444in}{1.030431in}}%
\pgfusepath{stroke}%
\end{pgfscope}%
\begin{pgfscope}%
\pgfsetbuttcap%
\pgfsetroundjoin%
\definecolor{currentfill}{rgb}{0.000000,0.000000,0.000000}%
\pgfsetfillcolor{currentfill}%
\pgfsetlinewidth{0.803000pt}%
\definecolor{currentstroke}{rgb}{0.000000,0.000000,0.000000}%
\pgfsetstrokecolor{currentstroke}%
\pgfsetdash{}{0pt}%
\pgfsys@defobject{currentmarker}{\pgfqpoint{-0.048611in}{0.000000in}}{\pgfqpoint{-0.000000in}{0.000000in}}{%
\pgfpathmoveto{\pgfqpoint{-0.000000in}{0.000000in}}%
\pgfpathlineto{\pgfqpoint{-0.048611in}{0.000000in}}%
\pgfusepath{stroke,fill}%
}%
\begin{pgfscope}%
\pgfsys@transformshift{0.694444in}{1.030431in}%
\pgfsys@useobject{currentmarker}{}%
\end{pgfscope}%
\end{pgfscope}%
\begin{pgfscope}%
\definecolor{textcolor}{rgb}{0.000000,0.000000,0.000000}%
\pgfsetstrokecolor{textcolor}%
\pgfsetfillcolor{textcolor}%
\pgftext[x=0.429036in, y=1.002693in, left, base]{\color{textcolor}{\ifdefined\pdftexversion\else\setmainfont{NanumMyeongjo}\rmfamily\fi\fontsize{5.000000}{6.000000}\selectfont\catcode`\^=\active\def^{\ifmmode\sp\else\^{}\fi}\catcode`\%=\active\def%{\%}8,000}}%
\end{pgfscope}%
\begin{pgfscope}%
\pgfpathrectangle{\pgfqpoint{0.694444in}{0.416667in}}{\pgfqpoint{3.750000in}{1.416667in}}%
\pgfusepath{clip}%
\pgfsetbuttcap%
\pgfsetroundjoin%
\pgfsetlinewidth{0.602250pt}%
\definecolor{currentstroke}{rgb}{0.690196,0.690196,0.690196}%
\pgfsetstrokecolor{currentstroke}%
\pgfsetstrokeopacity{0.400000}%
\pgfsetdash{{2.220000pt}{0.960000pt}}{0.000000pt}%
\pgfpathmoveto{\pgfqpoint{0.694444in}{1.183871in}}%
\pgfpathlineto{\pgfqpoint{4.444444in}{1.183871in}}%
\pgfusepath{stroke}%
\end{pgfscope}%
\begin{pgfscope}%
\pgfsetbuttcap%
\pgfsetroundjoin%
\definecolor{currentfill}{rgb}{0.000000,0.000000,0.000000}%
\pgfsetfillcolor{currentfill}%
\pgfsetlinewidth{0.803000pt}%
\definecolor{currentstroke}{rgb}{0.000000,0.000000,0.000000}%
\pgfsetstrokecolor{currentstroke}%
\pgfsetdash{}{0pt}%
\pgfsys@defobject{currentmarker}{\pgfqpoint{-0.048611in}{0.000000in}}{\pgfqpoint{-0.000000in}{0.000000in}}{%
\pgfpathmoveto{\pgfqpoint{-0.000000in}{0.000000in}}%
\pgfpathlineto{\pgfqpoint{-0.048611in}{0.000000in}}%
\pgfusepath{stroke,fill}%
}%
\begin{pgfscope}%
\pgfsys@transformshift{0.694444in}{1.183871in}%
\pgfsys@useobject{currentmarker}{}%
\end{pgfscope}%
\end{pgfscope}%
\begin{pgfscope}%
\definecolor{textcolor}{rgb}{0.000000,0.000000,0.000000}%
\pgfsetstrokecolor{textcolor}%
\pgfsetfillcolor{textcolor}%
\pgftext[x=0.391737in, y=1.156134in, left, base]{\color{textcolor}{\ifdefined\pdftexversion\else\setmainfont{NanumMyeongjo}\rmfamily\fi\fontsize{5.000000}{6.000000}\selectfont\catcode`\^=\active\def^{\ifmmode\sp\else\^{}\fi}\catcode`\%=\active\def%{\%}10,000}}%
\end{pgfscope}%
\begin{pgfscope}%
\pgfpathrectangle{\pgfqpoint{0.694444in}{0.416667in}}{\pgfqpoint{3.750000in}{1.416667in}}%
\pgfusepath{clip}%
\pgfsetbuttcap%
\pgfsetroundjoin%
\pgfsetlinewidth{0.602250pt}%
\definecolor{currentstroke}{rgb}{0.690196,0.690196,0.690196}%
\pgfsetstrokecolor{currentstroke}%
\pgfsetstrokeopacity{0.400000}%
\pgfsetdash{{2.220000pt}{0.960000pt}}{0.000000pt}%
\pgfpathmoveto{\pgfqpoint{0.694444in}{1.337312in}}%
\pgfpathlineto{\pgfqpoint{4.444444in}{1.337312in}}%
\pgfusepath{stroke}%
\end{pgfscope}%
\begin{pgfscope}%
\pgfsetbuttcap%
\pgfsetroundjoin%
\definecolor{currentfill}{rgb}{0.000000,0.000000,0.000000}%
\pgfsetfillcolor{currentfill}%
\pgfsetlinewidth{0.803000pt}%
\definecolor{currentstroke}{rgb}{0.000000,0.000000,0.000000}%
\pgfsetstrokecolor{currentstroke}%
\pgfsetdash{}{0pt}%
\pgfsys@defobject{currentmarker}{\pgfqpoint{-0.048611in}{0.000000in}}{\pgfqpoint{-0.000000in}{0.000000in}}{%
\pgfpathmoveto{\pgfqpoint{-0.000000in}{0.000000in}}%
\pgfpathlineto{\pgfqpoint{-0.048611in}{0.000000in}}%
\pgfusepath{stroke,fill}%
}%
\begin{pgfscope}%
\pgfsys@transformshift{0.694444in}{1.337312in}%
\pgfsys@useobject{currentmarker}{}%
\end{pgfscope}%
\end{pgfscope}%
\begin{pgfscope}%
\definecolor{textcolor}{rgb}{0.000000,0.000000,0.000000}%
\pgfsetstrokecolor{textcolor}%
\pgfsetfillcolor{textcolor}%
\pgftext[x=0.391737in, y=1.309575in, left, base]{\color{textcolor}{\ifdefined\pdftexversion\else\setmainfont{NanumMyeongjo}\rmfamily\fi\fontsize{5.000000}{6.000000}\selectfont\catcode`\^=\active\def^{\ifmmode\sp\else\^{}\fi}\catcode`\%=\active\def%{\%}12,000}}%
\end{pgfscope}%
\begin{pgfscope}%
\pgfpathrectangle{\pgfqpoint{0.694444in}{0.416667in}}{\pgfqpoint{3.750000in}{1.416667in}}%
\pgfusepath{clip}%
\pgfsetbuttcap%
\pgfsetroundjoin%
\pgfsetlinewidth{0.602250pt}%
\definecolor{currentstroke}{rgb}{0.690196,0.690196,0.690196}%
\pgfsetstrokecolor{currentstroke}%
\pgfsetstrokeopacity{0.400000}%
\pgfsetdash{{2.220000pt}{0.960000pt}}{0.000000pt}%
\pgfpathmoveto{\pgfqpoint{0.694444in}{1.490753in}}%
\pgfpathlineto{\pgfqpoint{4.444444in}{1.490753in}}%
\pgfusepath{stroke}%
\end{pgfscope}%
\begin{pgfscope}%
\pgfsetbuttcap%
\pgfsetroundjoin%
\definecolor{currentfill}{rgb}{0.000000,0.000000,0.000000}%
\pgfsetfillcolor{currentfill}%
\pgfsetlinewidth{0.803000pt}%
\definecolor{currentstroke}{rgb}{0.000000,0.000000,0.000000}%
\pgfsetstrokecolor{currentstroke}%
\pgfsetdash{}{0pt}%
\pgfsys@defobject{currentmarker}{\pgfqpoint{-0.048611in}{0.000000in}}{\pgfqpoint{-0.000000in}{0.000000in}}{%
\pgfpathmoveto{\pgfqpoint{-0.000000in}{0.000000in}}%
\pgfpathlineto{\pgfqpoint{-0.048611in}{0.000000in}}%
\pgfusepath{stroke,fill}%
}%
\begin{pgfscope}%
\pgfsys@transformshift{0.694444in}{1.490753in}%
\pgfsys@useobject{currentmarker}{}%
\end{pgfscope}%
\end{pgfscope}%
\begin{pgfscope}%
\definecolor{textcolor}{rgb}{0.000000,0.000000,0.000000}%
\pgfsetstrokecolor{textcolor}%
\pgfsetfillcolor{textcolor}%
\pgftext[x=0.391737in, y=1.463016in, left, base]{\color{textcolor}{\ifdefined\pdftexversion\else\setmainfont{NanumMyeongjo}\rmfamily\fi\fontsize{5.000000}{6.000000}\selectfont\catcode`\^=\active\def^{\ifmmode\sp\else\^{}\fi}\catcode`\%=\active\def%{\%}14,000}}%
\end{pgfscope}%
\begin{pgfscope}%
\pgfpathrectangle{\pgfqpoint{0.694444in}{0.416667in}}{\pgfqpoint{3.750000in}{1.416667in}}%
\pgfusepath{clip}%
\pgfsetbuttcap%
\pgfsetroundjoin%
\pgfsetlinewidth{0.602250pt}%
\definecolor{currentstroke}{rgb}{0.690196,0.690196,0.690196}%
\pgfsetstrokecolor{currentstroke}%
\pgfsetstrokeopacity{0.400000}%
\pgfsetdash{{2.220000pt}{0.960000pt}}{0.000000pt}%
\pgfpathmoveto{\pgfqpoint{0.694444in}{1.644194in}}%
\pgfpathlineto{\pgfqpoint{4.444444in}{1.644194in}}%
\pgfusepath{stroke}%
\end{pgfscope}%
\begin{pgfscope}%
\pgfsetbuttcap%
\pgfsetroundjoin%
\definecolor{currentfill}{rgb}{0.000000,0.000000,0.000000}%
\pgfsetfillcolor{currentfill}%
\pgfsetlinewidth{0.803000pt}%
\definecolor{currentstroke}{rgb}{0.000000,0.000000,0.000000}%
\pgfsetstrokecolor{currentstroke}%
\pgfsetdash{}{0pt}%
\pgfsys@defobject{currentmarker}{\pgfqpoint{-0.048611in}{0.000000in}}{\pgfqpoint{-0.000000in}{0.000000in}}{%
\pgfpathmoveto{\pgfqpoint{-0.000000in}{0.000000in}}%
\pgfpathlineto{\pgfqpoint{-0.048611in}{0.000000in}}%
\pgfusepath{stroke,fill}%
}%
\begin{pgfscope}%
\pgfsys@transformshift{0.694444in}{1.644194in}%
\pgfsys@useobject{currentmarker}{}%
\end{pgfscope}%
\end{pgfscope}%
\begin{pgfscope}%
\definecolor{textcolor}{rgb}{0.000000,0.000000,0.000000}%
\pgfsetstrokecolor{textcolor}%
\pgfsetfillcolor{textcolor}%
\pgftext[x=0.391737in, y=1.616457in, left, base]{\color{textcolor}{\ifdefined\pdftexversion\else\setmainfont{NanumMyeongjo}\rmfamily\fi\fontsize{5.000000}{6.000000}\selectfont\catcode`\^=\active\def^{\ifmmode\sp\else\^{}\fi}\catcode`\%=\active\def%{\%}16,000}}%
\end{pgfscope}%
\begin{pgfscope}%
\pgfpathrectangle{\pgfqpoint{0.694444in}{0.416667in}}{\pgfqpoint{3.750000in}{1.416667in}}%
\pgfusepath{clip}%
\pgfsetbuttcap%
\pgfsetroundjoin%
\pgfsetlinewidth{0.602250pt}%
\definecolor{currentstroke}{rgb}{0.690196,0.690196,0.690196}%
\pgfsetstrokecolor{currentstroke}%
\pgfsetstrokeopacity{0.400000}%
\pgfsetdash{{2.220000pt}{0.960000pt}}{0.000000pt}%
\pgfpathmoveto{\pgfqpoint{0.694444in}{1.797635in}}%
\pgfpathlineto{\pgfqpoint{4.444444in}{1.797635in}}%
\pgfusepath{stroke}%
\end{pgfscope}%
\begin{pgfscope}%
\pgfsetbuttcap%
\pgfsetroundjoin%
\definecolor{currentfill}{rgb}{0.000000,0.000000,0.000000}%
\pgfsetfillcolor{currentfill}%
\pgfsetlinewidth{0.803000pt}%
\definecolor{currentstroke}{rgb}{0.000000,0.000000,0.000000}%
\pgfsetstrokecolor{currentstroke}%
\pgfsetdash{}{0pt}%
\pgfsys@defobject{currentmarker}{\pgfqpoint{-0.048611in}{0.000000in}}{\pgfqpoint{-0.000000in}{0.000000in}}{%
\pgfpathmoveto{\pgfqpoint{-0.000000in}{0.000000in}}%
\pgfpathlineto{\pgfqpoint{-0.048611in}{0.000000in}}%
\pgfusepath{stroke,fill}%
}%
\begin{pgfscope}%
\pgfsys@transformshift{0.694444in}{1.797635in}%
\pgfsys@useobject{currentmarker}{}%
\end{pgfscope}%
\end{pgfscope}%
\begin{pgfscope}%
\definecolor{textcolor}{rgb}{0.000000,0.000000,0.000000}%
\pgfsetstrokecolor{textcolor}%
\pgfsetfillcolor{textcolor}%
\pgftext[x=0.391737in, y=1.769898in, left, base]{\color{textcolor}{\ifdefined\pdftexversion\else\setmainfont{NanumMyeongjo}\rmfamily\fi\fontsize{5.000000}{6.000000}\selectfont\catcode`\^=\active\def^{\ifmmode\sp\else\^{}\fi}\catcode`\%=\active\def%{\%}18,000}}%
\end{pgfscope}%
\begin{pgfscope}%
\pgfsetrectcap%
\pgfsetmiterjoin%
\pgfsetlinewidth{0.803000pt}%
\definecolor{currentstroke}{rgb}{0.000000,0.000000,0.000000}%
\pgfsetstrokecolor{currentstroke}%
\pgfsetdash{}{0pt}%
\pgfpathmoveto{\pgfqpoint{0.694444in}{0.416667in}}%
\pgfpathlineto{\pgfqpoint{0.694444in}{1.833333in}}%
\pgfusepath{stroke}%
\end{pgfscope}%
\begin{pgfscope}%
\pgfsetrectcap%
\pgfsetmiterjoin%
\pgfsetlinewidth{0.803000pt}%
\definecolor{currentstroke}{rgb}{0.000000,0.000000,0.000000}%
\pgfsetstrokecolor{currentstroke}%
\pgfsetdash{}{0pt}%
\pgfpathmoveto{\pgfqpoint{0.694444in}{0.416667in}}%
\pgfpathlineto{\pgfqpoint{4.444444in}{0.416667in}}%
\pgfusepath{stroke}%
\end{pgfscope}%
\begin{pgfscope}%
\pgfpathrectangle{\pgfqpoint{0.694444in}{0.416667in}}{\pgfqpoint{3.750000in}{1.416667in}}%
\pgfusepath{clip}%
\pgfsetbuttcap%
\pgfsetmiterjoin%
\definecolor{currentfill}{rgb}{0.725490,0.486275,0.164706}%
\pgfsetfillcolor{currentfill}%
\pgfsetlinewidth{1.003750pt}%
\definecolor{currentstroke}{rgb}{0.266667,0.266667,0.266667}%
\pgfsetstrokecolor{currentstroke}%
\pgfsetdash{}{0pt}%
\pgfpathmoveto{\pgfqpoint{0.864899in}{0.416667in}}%
\pgfpathlineto{\pgfqpoint{1.079887in}{0.416667in}}%
\pgfpathlineto{\pgfqpoint{1.079887in}{1.117432in}}%
\pgfpathlineto{\pgfqpoint{0.864899in}{1.117432in}}%
\pgfpathlineto{\pgfqpoint{0.864899in}{0.416667in}}%
\pgfpathclose%
\pgfusepath{stroke,fill}%
\end{pgfscope}%
\begin{pgfscope}%
\pgfpathrectangle{\pgfqpoint{0.694444in}{0.416667in}}{\pgfqpoint{3.750000in}{1.416667in}}%
\pgfusepath{clip}%
\pgfsetbuttcap%
\pgfsetmiterjoin%
\definecolor{currentfill}{rgb}{0.725490,0.486275,0.164706}%
\pgfsetfillcolor{currentfill}%
\pgfsetlinewidth{1.003750pt}%
\definecolor{currentstroke}{rgb}{0.266667,0.266667,0.266667}%
\pgfsetstrokecolor{currentstroke}%
\pgfsetdash{}{0pt}%
\pgfpathmoveto{\pgfqpoint{1.786275in}{0.416667in}}%
\pgfpathlineto{\pgfqpoint{2.001263in}{0.416667in}}%
\pgfpathlineto{\pgfqpoint{2.001263in}{1.042706in}}%
\pgfpathlineto{\pgfqpoint{1.786275in}{1.042706in}}%
\pgfpathlineto{\pgfqpoint{1.786275in}{0.416667in}}%
\pgfpathclose%
\pgfusepath{stroke,fill}%
\end{pgfscope}%
\begin{pgfscope}%
\pgfpathrectangle{\pgfqpoint{0.694444in}{0.416667in}}{\pgfqpoint{3.750000in}{1.416667in}}%
\pgfusepath{clip}%
\pgfsetbuttcap%
\pgfsetmiterjoin%
\definecolor{currentfill}{rgb}{0.725490,0.486275,0.164706}%
\pgfsetfillcolor{currentfill}%
\pgfsetlinewidth{1.003750pt}%
\definecolor{currentstroke}{rgb}{0.266667,0.266667,0.266667}%
\pgfsetstrokecolor{currentstroke}%
\pgfsetdash{}{0pt}%
\pgfpathmoveto{\pgfqpoint{2.707651in}{0.416667in}}%
\pgfpathlineto{\pgfqpoint{2.922639in}{0.416667in}}%
\pgfpathlineto{\pgfqpoint{2.922639in}{1.136075in}}%
\pgfpathlineto{\pgfqpoint{2.707651in}{1.136075in}}%
\pgfpathlineto{\pgfqpoint{2.707651in}{0.416667in}}%
\pgfpathclose%
\pgfusepath{stroke,fill}%
\end{pgfscope}%
\begin{pgfscope}%
\pgfpathrectangle{\pgfqpoint{0.694444in}{0.416667in}}{\pgfqpoint{3.750000in}{1.416667in}}%
\pgfusepath{clip}%
\pgfsetbuttcap%
\pgfsetmiterjoin%
\definecolor{currentfill}{rgb}{0.725490,0.486275,0.164706}%
\pgfsetfillcolor{currentfill}%
\pgfsetlinewidth{1.003750pt}%
\definecolor{currentstroke}{rgb}{0.266667,0.266667,0.266667}%
\pgfsetstrokecolor{currentstroke}%
\pgfsetdash{}{0pt}%
\pgfpathmoveto{\pgfqpoint{3.629027in}{0.416667in}}%
\pgfpathlineto{\pgfqpoint{3.844014in}{0.416667in}}%
\pgfpathlineto{\pgfqpoint{3.844014in}{1.077767in}}%
\pgfpathlineto{\pgfqpoint{3.629027in}{1.077767in}}%
\pgfpathlineto{\pgfqpoint{3.629027in}{0.416667in}}%
\pgfpathclose%
\pgfusepath{stroke,fill}%
\end{pgfscope}%
\begin{pgfscope}%
\pgfpathrectangle{\pgfqpoint{0.694444in}{0.416667in}}{\pgfqpoint{3.750000in}{1.416667in}}%
\pgfusepath{clip}%
\pgfsetbuttcap%
\pgfsetmiterjoin%
\definecolor{currentfill}{rgb}{0.725490,0.486275,0.164706}%
\pgfsetfillcolor{currentfill}%
\pgfsetfillopacity{0.700000}%
\pgfsetlinewidth{1.003750pt}%
\definecolor{currentstroke}{rgb}{0.266667,0.266667,0.266667}%
\pgfsetstrokecolor{currentstroke}%
\pgfsetstrokeopacity{0.700000}%
\pgfsetdash{}{0pt}%
\pgfpathmoveto{\pgfqpoint{0.864899in}{1.117432in}}%
\pgfpathlineto{\pgfqpoint{1.079887in}{1.117432in}}%
\pgfpathlineto{\pgfqpoint{1.079887in}{1.179192in}}%
\pgfpathlineto{\pgfqpoint{0.864899in}{1.179192in}}%
\pgfpathlineto{\pgfqpoint{0.864899in}{1.117432in}}%
\pgfpathclose%
\pgfusepath{stroke,fill}%
\end{pgfscope}%
\begin{pgfscope}%
\pgfpathrectangle{\pgfqpoint{0.694444in}{0.416667in}}{\pgfqpoint{3.750000in}{1.416667in}}%
\pgfusepath{clip}%
\pgfsetbuttcap%
\pgfsetmiterjoin%
\definecolor{currentfill}{rgb}{0.725490,0.486275,0.164706}%
\pgfsetfillcolor{currentfill}%
\pgfsetfillopacity{0.700000}%
\pgfsetlinewidth{1.003750pt}%
\definecolor{currentstroke}{rgb}{0.266667,0.266667,0.266667}%
\pgfsetstrokecolor{currentstroke}%
\pgfsetstrokeopacity{0.700000}%
\pgfsetdash{}{0pt}%
\pgfpathmoveto{\pgfqpoint{1.786275in}{1.042706in}}%
\pgfpathlineto{\pgfqpoint{2.001263in}{1.042706in}}%
\pgfpathlineto{\pgfqpoint{2.001263in}{1.136305in}}%
\pgfpathlineto{\pgfqpoint{1.786275in}{1.136305in}}%
\pgfpathlineto{\pgfqpoint{1.786275in}{1.042706in}}%
\pgfpathclose%
\pgfusepath{stroke,fill}%
\end{pgfscope}%
\begin{pgfscope}%
\pgfpathrectangle{\pgfqpoint{0.694444in}{0.416667in}}{\pgfqpoint{3.750000in}{1.416667in}}%
\pgfusepath{clip}%
\pgfsetbuttcap%
\pgfsetmiterjoin%
\definecolor{currentfill}{rgb}{0.725490,0.486275,0.164706}%
\pgfsetfillcolor{currentfill}%
\pgfsetfillopacity{0.700000}%
\pgfsetlinewidth{1.003750pt}%
\definecolor{currentstroke}{rgb}{0.266667,0.266667,0.266667}%
\pgfsetstrokecolor{currentstroke}%
\pgfsetstrokeopacity{0.700000}%
\pgfsetdash{}{0pt}%
\pgfpathmoveto{\pgfqpoint{2.707651in}{1.136075in}}%
\pgfpathlineto{\pgfqpoint{2.922639in}{1.136075in}}%
\pgfpathlineto{\pgfqpoint{2.922639in}{1.258520in}}%
\pgfpathlineto{\pgfqpoint{2.707651in}{1.258520in}}%
\pgfpathlineto{\pgfqpoint{2.707651in}{1.136075in}}%
\pgfpathclose%
\pgfusepath{stroke,fill}%
\end{pgfscope}%
\begin{pgfscope}%
\pgfpathrectangle{\pgfqpoint{0.694444in}{0.416667in}}{\pgfqpoint{3.750000in}{1.416667in}}%
\pgfusepath{clip}%
\pgfsetbuttcap%
\pgfsetmiterjoin%
\definecolor{currentfill}{rgb}{0.725490,0.486275,0.164706}%
\pgfsetfillcolor{currentfill}%
\pgfsetfillopacity{0.700000}%
\pgfsetlinewidth{1.003750pt}%
\definecolor{currentstroke}{rgb}{0.266667,0.266667,0.266667}%
\pgfsetstrokecolor{currentstroke}%
\pgfsetstrokeopacity{0.700000}%
\pgfsetdash{}{0pt}%
\pgfpathmoveto{\pgfqpoint{3.629027in}{1.077767in}}%
\pgfpathlineto{\pgfqpoint{3.844014in}{1.077767in}}%
\pgfpathlineto{\pgfqpoint{3.844014in}{1.233049in}}%
\pgfpathlineto{\pgfqpoint{3.629027in}{1.233049in}}%
\pgfpathlineto{\pgfqpoint{3.629027in}{1.077767in}}%
\pgfpathclose%
\pgfusepath{stroke,fill}%
\end{pgfscope}%
\begin{pgfscope}%
\pgfpathrectangle{\pgfqpoint{0.694444in}{0.416667in}}{\pgfqpoint{3.750000in}{1.416667in}}%
\pgfusepath{clip}%
\pgfsetbuttcap%
\pgfsetmiterjoin%
\definecolor{currentfill}{rgb}{0.235294,0.490196,0.764706}%
\pgfsetfillcolor{currentfill}%
\pgfsetlinewidth{1.003750pt}%
\definecolor{currentstroke}{rgb}{0.266667,0.266667,0.266667}%
\pgfsetstrokecolor{currentstroke}%
\pgfsetdash{}{0pt}%
\pgfpathmoveto{\pgfqpoint{1.079887in}{0.416667in}}%
\pgfpathlineto{\pgfqpoint{1.294874in}{0.416667in}}%
\pgfpathlineto{\pgfqpoint{1.294874in}{1.087971in}}%
\pgfpathlineto{\pgfqpoint{1.079887in}{1.087971in}}%
\pgfpathlineto{\pgfqpoint{1.079887in}{0.416667in}}%
\pgfpathclose%
\pgfusepath{stroke,fill}%
\end{pgfscope}%
\begin{pgfscope}%
\pgfpathrectangle{\pgfqpoint{0.694444in}{0.416667in}}{\pgfqpoint{3.750000in}{1.416667in}}%
\pgfusepath{clip}%
\pgfsetbuttcap%
\pgfsetmiterjoin%
\definecolor{currentfill}{rgb}{0.235294,0.490196,0.764706}%
\pgfsetfillcolor{currentfill}%
\pgfsetlinewidth{1.003750pt}%
\definecolor{currentstroke}{rgb}{0.266667,0.266667,0.266667}%
\pgfsetstrokecolor{currentstroke}%
\pgfsetdash{}{0pt}%
\pgfpathmoveto{\pgfqpoint{2.001263in}{0.416667in}}%
\pgfpathlineto{\pgfqpoint{2.216250in}{0.416667in}}%
\pgfpathlineto{\pgfqpoint{2.216250in}{1.131318in}}%
\pgfpathlineto{\pgfqpoint{2.001263in}{1.131318in}}%
\pgfpathlineto{\pgfqpoint{2.001263in}{0.416667in}}%
\pgfpathclose%
\pgfusepath{stroke,fill}%
\end{pgfscope}%
\begin{pgfscope}%
\pgfpathrectangle{\pgfqpoint{0.694444in}{0.416667in}}{\pgfqpoint{3.750000in}{1.416667in}}%
\pgfusepath{clip}%
\pgfsetbuttcap%
\pgfsetmiterjoin%
\definecolor{currentfill}{rgb}{0.235294,0.490196,0.764706}%
\pgfsetfillcolor{currentfill}%
\pgfsetlinewidth{1.003750pt}%
\definecolor{currentstroke}{rgb}{0.266667,0.266667,0.266667}%
\pgfsetstrokecolor{currentstroke}%
\pgfsetdash{}{0pt}%
\pgfpathmoveto{\pgfqpoint{2.922639in}{0.416667in}}%
\pgfpathlineto{\pgfqpoint{3.137626in}{0.416667in}}%
\pgfpathlineto{\pgfqpoint{3.137626in}{1.272714in}}%
\pgfpathlineto{\pgfqpoint{2.922639in}{1.272714in}}%
\pgfpathlineto{\pgfqpoint{2.922639in}{0.416667in}}%
\pgfpathclose%
\pgfusepath{stroke,fill}%
\end{pgfscope}%
\begin{pgfscope}%
\pgfpathrectangle{\pgfqpoint{0.694444in}{0.416667in}}{\pgfqpoint{3.750000in}{1.416667in}}%
\pgfusepath{clip}%
\pgfsetbuttcap%
\pgfsetmiterjoin%
\definecolor{currentfill}{rgb}{0.235294,0.490196,0.764706}%
\pgfsetfillcolor{currentfill}%
\pgfsetlinewidth{1.003750pt}%
\definecolor{currentstroke}{rgb}{0.266667,0.266667,0.266667}%
\pgfsetstrokecolor{currentstroke}%
\pgfsetdash{}{0pt}%
\pgfpathmoveto{\pgfqpoint{3.844014in}{0.416667in}}%
\pgfpathlineto{\pgfqpoint{4.059002in}{0.416667in}}%
\pgfpathlineto{\pgfqpoint{4.059002in}{1.190776in}}%
\pgfpathlineto{\pgfqpoint{3.844014in}{1.190776in}}%
\pgfpathlineto{\pgfqpoint{3.844014in}{0.416667in}}%
\pgfpathclose%
\pgfusepath{stroke,fill}%
\end{pgfscope}%
\begin{pgfscope}%
\pgfpathrectangle{\pgfqpoint{0.694444in}{0.416667in}}{\pgfqpoint{3.750000in}{1.416667in}}%
\pgfusepath{clip}%
\pgfsetbuttcap%
\pgfsetmiterjoin%
\definecolor{currentfill}{rgb}{0.235294,0.490196,0.764706}%
\pgfsetfillcolor{currentfill}%
\pgfsetfillopacity{0.700000}%
\pgfsetlinewidth{1.003750pt}%
\definecolor{currentstroke}{rgb}{0.266667,0.266667,0.266667}%
\pgfsetstrokecolor{currentstroke}%
\pgfsetstrokeopacity{0.700000}%
\pgfsetdash{}{0pt}%
\pgfpathmoveto{\pgfqpoint{1.079887in}{1.087971in}}%
\pgfpathlineto{\pgfqpoint{1.294874in}{1.087971in}}%
\pgfpathlineto{\pgfqpoint{1.294874in}{1.192848in}}%
\pgfpathlineto{\pgfqpoint{1.079887in}{1.192848in}}%
\pgfpathlineto{\pgfqpoint{1.079887in}{1.087971in}}%
\pgfpathclose%
\pgfusepath{stroke,fill}%
\end{pgfscope}%
\begin{pgfscope}%
\pgfpathrectangle{\pgfqpoint{0.694444in}{0.416667in}}{\pgfqpoint{3.750000in}{1.416667in}}%
\pgfusepath{clip}%
\pgfsetbuttcap%
\pgfsetmiterjoin%
\definecolor{currentfill}{rgb}{0.235294,0.490196,0.764706}%
\pgfsetfillcolor{currentfill}%
\pgfsetfillopacity{0.700000}%
\pgfsetlinewidth{1.003750pt}%
\definecolor{currentstroke}{rgb}{0.266667,0.266667,0.266667}%
\pgfsetstrokecolor{currentstroke}%
\pgfsetstrokeopacity{0.700000}%
\pgfsetdash{}{0pt}%
\pgfpathmoveto{\pgfqpoint{2.001263in}{1.131318in}}%
\pgfpathlineto{\pgfqpoint{2.216250in}{1.131318in}}%
\pgfpathlineto{\pgfqpoint{2.216250in}{1.256986in}}%
\pgfpathlineto{\pgfqpoint{2.001263in}{1.256986in}}%
\pgfpathlineto{\pgfqpoint{2.001263in}{1.131318in}}%
\pgfpathclose%
\pgfusepath{stroke,fill}%
\end{pgfscope}%
\begin{pgfscope}%
\pgfpathrectangle{\pgfqpoint{0.694444in}{0.416667in}}{\pgfqpoint{3.750000in}{1.416667in}}%
\pgfusepath{clip}%
\pgfsetbuttcap%
\pgfsetmiterjoin%
\definecolor{currentfill}{rgb}{0.235294,0.490196,0.764706}%
\pgfsetfillcolor{currentfill}%
\pgfsetfillopacity{0.700000}%
\pgfsetlinewidth{1.003750pt}%
\definecolor{currentstroke}{rgb}{0.266667,0.266667,0.266667}%
\pgfsetstrokecolor{currentstroke}%
\pgfsetstrokeopacity{0.700000}%
\pgfsetdash{}{0pt}%
\pgfpathmoveto{\pgfqpoint{2.922639in}{1.272714in}}%
\pgfpathlineto{\pgfqpoint{3.137626in}{1.272714in}}%
\pgfpathlineto{\pgfqpoint{3.137626in}{1.405363in}}%
\pgfpathlineto{\pgfqpoint{2.922639in}{1.405363in}}%
\pgfpathlineto{\pgfqpoint{2.922639in}{1.272714in}}%
\pgfpathclose%
\pgfusepath{stroke,fill}%
\end{pgfscope}%
\begin{pgfscope}%
\pgfpathrectangle{\pgfqpoint{0.694444in}{0.416667in}}{\pgfqpoint{3.750000in}{1.416667in}}%
\pgfusepath{clip}%
\pgfsetbuttcap%
\pgfsetmiterjoin%
\definecolor{currentfill}{rgb}{0.235294,0.490196,0.764706}%
\pgfsetfillcolor{currentfill}%
\pgfsetfillopacity{0.700000}%
\pgfsetlinewidth{1.003750pt}%
\definecolor{currentstroke}{rgb}{0.266667,0.266667,0.266667}%
\pgfsetstrokecolor{currentstroke}%
\pgfsetstrokeopacity{0.700000}%
\pgfsetdash{}{0pt}%
\pgfpathmoveto{\pgfqpoint{3.844014in}{1.190776in}}%
\pgfpathlineto{\pgfqpoint{4.059002in}{1.190776in}}%
\pgfpathlineto{\pgfqpoint{4.059002in}{1.342146in}}%
\pgfpathlineto{\pgfqpoint{3.844014in}{1.342146in}}%
\pgfpathlineto{\pgfqpoint{3.844014in}{1.190776in}}%
\pgfpathclose%
\pgfusepath{stroke,fill}%
\end{pgfscope}%
\begin{pgfscope}%
\pgfpathrectangle{\pgfqpoint{0.694444in}{0.416667in}}{\pgfqpoint{3.750000in}{1.416667in}}%
\pgfusepath{clip}%
\pgfsetbuttcap%
\pgfsetmiterjoin%
\definecolor{currentfill}{rgb}{0.337255,0.713725,0.627451}%
\pgfsetfillcolor{currentfill}%
\pgfsetlinewidth{1.003750pt}%
\definecolor{currentstroke}{rgb}{0.266667,0.266667,0.266667}%
\pgfsetstrokecolor{currentstroke}%
\pgfsetdash{}{0pt}%
\pgfpathmoveto{\pgfqpoint{1.294874in}{0.416667in}}%
\pgfpathlineto{\pgfqpoint{1.509862in}{0.416667in}}%
\pgfpathlineto{\pgfqpoint{1.509862in}{0.846225in}}%
\pgfpathlineto{\pgfqpoint{1.294874in}{0.846225in}}%
\pgfpathlineto{\pgfqpoint{1.294874in}{0.416667in}}%
\pgfpathclose%
\pgfusepath{stroke,fill}%
\end{pgfscope}%
\begin{pgfscope}%
\pgfpathrectangle{\pgfqpoint{0.694444in}{0.416667in}}{\pgfqpoint{3.750000in}{1.416667in}}%
\pgfusepath{clip}%
\pgfsetbuttcap%
\pgfsetmiterjoin%
\definecolor{currentfill}{rgb}{0.337255,0.713725,0.627451}%
\pgfsetfillcolor{currentfill}%
\pgfsetlinewidth{1.003750pt}%
\definecolor{currentstroke}{rgb}{0.266667,0.266667,0.266667}%
\pgfsetstrokecolor{currentstroke}%
\pgfsetdash{}{0pt}%
\pgfpathmoveto{\pgfqpoint{2.216250in}{0.416667in}}%
\pgfpathlineto{\pgfqpoint{2.431238in}{0.416667in}}%
\pgfpathlineto{\pgfqpoint{2.431238in}{0.816150in}}%
\pgfpathlineto{\pgfqpoint{2.216250in}{0.816150in}}%
\pgfpathlineto{\pgfqpoint{2.216250in}{0.416667in}}%
\pgfpathclose%
\pgfusepath{stroke,fill}%
\end{pgfscope}%
\begin{pgfscope}%
\pgfpathrectangle{\pgfqpoint{0.694444in}{0.416667in}}{\pgfqpoint{3.750000in}{1.416667in}}%
\pgfusepath{clip}%
\pgfsetbuttcap%
\pgfsetmiterjoin%
\definecolor{currentfill}{rgb}{0.337255,0.713725,0.627451}%
\pgfsetfillcolor{currentfill}%
\pgfsetlinewidth{1.003750pt}%
\definecolor{currentstroke}{rgb}{0.266667,0.266667,0.266667}%
\pgfsetstrokecolor{currentstroke}%
\pgfsetdash{}{0pt}%
\pgfpathmoveto{\pgfqpoint{3.137626in}{0.416667in}}%
\pgfpathlineto{\pgfqpoint{3.352614in}{0.416667in}}%
\pgfpathlineto{\pgfqpoint{3.352614in}{0.961382in}}%
\pgfpathlineto{\pgfqpoint{3.137626in}{0.961382in}}%
\pgfpathlineto{\pgfqpoint{3.137626in}{0.416667in}}%
\pgfpathclose%
\pgfusepath{stroke,fill}%
\end{pgfscope}%
\begin{pgfscope}%
\pgfpathrectangle{\pgfqpoint{0.694444in}{0.416667in}}{\pgfqpoint{3.750000in}{1.416667in}}%
\pgfusepath{clip}%
\pgfsetbuttcap%
\pgfsetmiterjoin%
\definecolor{currentfill}{rgb}{0.337255,0.713725,0.627451}%
\pgfsetfillcolor{currentfill}%
\pgfsetlinewidth{1.003750pt}%
\definecolor{currentstroke}{rgb}{0.266667,0.266667,0.266667}%
\pgfsetstrokecolor{currentstroke}%
\pgfsetdash{}{0pt}%
\pgfpathmoveto{\pgfqpoint{4.059002in}{0.416667in}}%
\pgfpathlineto{\pgfqpoint{4.273990in}{0.416667in}}%
\pgfpathlineto{\pgfqpoint{4.273990in}{0.938980in}}%
\pgfpathlineto{\pgfqpoint{4.059002in}{0.938980in}}%
\pgfpathlineto{\pgfqpoint{4.059002in}{0.416667in}}%
\pgfpathclose%
\pgfusepath{stroke,fill}%
\end{pgfscope}%
\begin{pgfscope}%
\pgfpathrectangle{\pgfqpoint{0.694444in}{0.416667in}}{\pgfqpoint{3.750000in}{1.416667in}}%
\pgfusepath{clip}%
\pgfsetbuttcap%
\pgfsetmiterjoin%
\definecolor{currentfill}{rgb}{0.337255,0.713725,0.627451}%
\pgfsetfillcolor{currentfill}%
\pgfsetfillopacity{0.700000}%
\pgfsetlinewidth{1.003750pt}%
\definecolor{currentstroke}{rgb}{0.266667,0.266667,0.266667}%
\pgfsetstrokecolor{currentstroke}%
\pgfsetstrokeopacity{0.700000}%
\pgfsetdash{}{0pt}%
\pgfpathmoveto{\pgfqpoint{1.294874in}{0.846225in}}%
\pgfpathlineto{\pgfqpoint{1.509862in}{0.846225in}}%
\pgfpathlineto{\pgfqpoint{1.509862in}{1.312225in}}%
\pgfpathlineto{\pgfqpoint{1.294874in}{1.312225in}}%
\pgfpathlineto{\pgfqpoint{1.294874in}{0.846225in}}%
\pgfpathclose%
\pgfusepath{stroke,fill}%
\end{pgfscope}%
\begin{pgfscope}%
\pgfpathrectangle{\pgfqpoint{0.694444in}{0.416667in}}{\pgfqpoint{3.750000in}{1.416667in}}%
\pgfusepath{clip}%
\pgfsetbuttcap%
\pgfsetmiterjoin%
\definecolor{currentfill}{rgb}{0.337255,0.713725,0.627451}%
\pgfsetfillcolor{currentfill}%
\pgfsetfillopacity{0.700000}%
\pgfsetlinewidth{1.003750pt}%
\definecolor{currentstroke}{rgb}{0.266667,0.266667,0.266667}%
\pgfsetstrokecolor{currentstroke}%
\pgfsetstrokeopacity{0.700000}%
\pgfsetdash{}{0pt}%
\pgfpathmoveto{\pgfqpoint{2.216250in}{0.816150in}}%
\pgfpathlineto{\pgfqpoint{2.431238in}{0.816150in}}%
\pgfpathlineto{\pgfqpoint{2.431238in}{1.199369in}}%
\pgfpathlineto{\pgfqpoint{2.216250in}{1.199369in}}%
\pgfpathlineto{\pgfqpoint{2.216250in}{0.816150in}}%
\pgfpathclose%
\pgfusepath{stroke,fill}%
\end{pgfscope}%
\begin{pgfscope}%
\pgfpathrectangle{\pgfqpoint{0.694444in}{0.416667in}}{\pgfqpoint{3.750000in}{1.416667in}}%
\pgfusepath{clip}%
\pgfsetbuttcap%
\pgfsetmiterjoin%
\definecolor{currentfill}{rgb}{0.337255,0.713725,0.627451}%
\pgfsetfillcolor{currentfill}%
\pgfsetfillopacity{0.700000}%
\pgfsetlinewidth{1.003750pt}%
\definecolor{currentstroke}{rgb}{0.266667,0.266667,0.266667}%
\pgfsetstrokecolor{currentstroke}%
\pgfsetstrokeopacity{0.700000}%
\pgfsetdash{}{0pt}%
\pgfpathmoveto{\pgfqpoint{3.137626in}{0.961382in}}%
\pgfpathlineto{\pgfqpoint{3.352614in}{0.961382in}}%
\pgfpathlineto{\pgfqpoint{3.352614in}{1.448404in}}%
\pgfpathlineto{\pgfqpoint{3.137626in}{1.448404in}}%
\pgfpathlineto{\pgfqpoint{3.137626in}{0.961382in}}%
\pgfpathclose%
\pgfusepath{stroke,fill}%
\end{pgfscope}%
\begin{pgfscope}%
\pgfpathrectangle{\pgfqpoint{0.694444in}{0.416667in}}{\pgfqpoint{3.750000in}{1.416667in}}%
\pgfusepath{clip}%
\pgfsetbuttcap%
\pgfsetmiterjoin%
\definecolor{currentfill}{rgb}{0.337255,0.713725,0.627451}%
\pgfsetfillcolor{currentfill}%
\pgfsetfillopacity{0.700000}%
\pgfsetlinewidth{1.003750pt}%
\definecolor{currentstroke}{rgb}{0.266667,0.266667,0.266667}%
\pgfsetstrokecolor{currentstroke}%
\pgfsetstrokeopacity{0.700000}%
\pgfsetdash{}{0pt}%
\pgfpathmoveto{\pgfqpoint{4.059002in}{0.938980in}}%
\pgfpathlineto{\pgfqpoint{4.273990in}{0.938980in}}%
\pgfpathlineto{\pgfqpoint{4.273990in}{1.765873in}}%
\pgfpathlineto{\pgfqpoint{4.059002in}{1.765873in}}%
\pgfpathlineto{\pgfqpoint{4.059002in}{0.938980in}}%
\pgfpathclose%
\pgfusepath{stroke,fill}%
\end{pgfscope}%
\begin{pgfscope}%
\definecolor{textcolor}{rgb}{1.000000,1.000000,1.000000}%
\pgfsetstrokecolor{textcolor}%
\pgfsetfillcolor{textcolor}%
\pgftext[x=0.972393in,y=1.079071in,,]{\color{textcolor}{\ifdefined\pdftexversion\else\setmainfont{NanumMyeongjo}\rmfamily\fi\fontsize{5.000000}{6.000000}\selectfont\catcode`\^=\active\def^{\ifmmode\sp\else\^{}\fi}\catcode`\%=\active\def%{\%}9,134}}%
\end{pgfscope}%
\begin{pgfscope}%
\definecolor{textcolor}{rgb}{1.000000,1.000000,1.000000}%
\pgfsetstrokecolor{textcolor}%
\pgfsetfillcolor{textcolor}%
\pgftext[x=1.893769in,y=1.004346in,,]{\color{textcolor}{\ifdefined\pdftexversion\else\setmainfont{NanumMyeongjo}\rmfamily\fi\fontsize{5.000000}{6.000000}\selectfont\catcode`\^=\active\def^{\ifmmode\sp\else\^{}\fi}\catcode`\%=\active\def%{\%}8,160}}%
\end{pgfscope}%
\begin{pgfscope}%
\definecolor{textcolor}{rgb}{1.000000,1.000000,1.000000}%
\pgfsetstrokecolor{textcolor}%
\pgfsetfillcolor{textcolor}%
\pgftext[x=2.815145in,y=1.097714in,,]{\color{textcolor}{\ifdefined\pdftexversion\else\setmainfont{NanumMyeongjo}\rmfamily\fi\fontsize{5.000000}{6.000000}\selectfont\catcode`\^=\active\def^{\ifmmode\sp\else\^{}\fi}\catcode`\%=\active\def%{\%}9,377}}%
\end{pgfscope}%
\begin{pgfscope}%
\definecolor{textcolor}{rgb}{1.000000,1.000000,1.000000}%
\pgfsetstrokecolor{textcolor}%
\pgfsetfillcolor{textcolor}%
\pgftext[x=3.736521in,y=1.039407in,,]{\color{textcolor}{\ifdefined\pdftexversion\else\setmainfont{NanumMyeongjo}\rmfamily\fi\fontsize{5.000000}{6.000000}\selectfont\catcode`\^=\active\def^{\ifmmode\sp\else\^{}\fi}\catcode`\%=\active\def%{\%}8,617}}%
\end{pgfscope}%
\begin{pgfscope}%
\definecolor{textcolor}{rgb}{1.000000,1.000000,1.000000}%
\pgfsetstrokecolor{textcolor}%
\pgfsetfillcolor{textcolor}%
\pgftext[x=0.972393in,y=1.140831in,,]{\color{textcolor}{\ifdefined\pdftexversion\else\setmainfont{NanumMyeongjo}\rmfamily\fi\fontsize{5.000000}{6.000000}\selectfont\catcode`\^=\active\def^{\ifmmode\sp\else\^{}\fi}\catcode`\%=\active\def%{\%}805}}%
\end{pgfscope}%
\begin{pgfscope}%
\definecolor{textcolor}{rgb}{1.000000,1.000000,1.000000}%
\pgfsetstrokecolor{textcolor}%
\pgfsetfillcolor{textcolor}%
\pgftext[x=1.893769in,y=1.097945in,,]{\color{textcolor}{\ifdefined\pdftexversion\else\setmainfont{NanumMyeongjo}\rmfamily\fi\fontsize{5.000000}{6.000000}\selectfont\catcode`\^=\active\def^{\ifmmode\sp\else\^{}\fi}\catcode`\%=\active\def%{\%}1,220}}%
\end{pgfscope}%
\begin{pgfscope}%
\definecolor{textcolor}{rgb}{1.000000,1.000000,1.000000}%
\pgfsetstrokecolor{textcolor}%
\pgfsetfillcolor{textcolor}%
\pgftext[x=2.815145in,y=1.220160in,,]{\color{textcolor}{\ifdefined\pdftexversion\else\setmainfont{NanumMyeongjo}\rmfamily\fi\fontsize{5.000000}{6.000000}\selectfont\catcode`\^=\active\def^{\ifmmode\sp\else\^{}\fi}\catcode`\%=\active\def%{\%}1,596}}%
\end{pgfscope}%
\begin{pgfscope}%
\definecolor{textcolor}{rgb}{1.000000,1.000000,1.000000}%
\pgfsetstrokecolor{textcolor}%
\pgfsetfillcolor{textcolor}%
\pgftext[x=3.736521in,y=1.194689in,,]{\color{textcolor}{\ifdefined\pdftexversion\else\setmainfont{NanumMyeongjo}\rmfamily\fi\fontsize{5.000000}{6.000000}\selectfont\catcode`\^=\active\def^{\ifmmode\sp\else\^{}\fi}\catcode`\%=\active\def%{\%}2,024}}%
\end{pgfscope}%
\begin{pgfscope}%
\definecolor{textcolor}{rgb}{1.000000,1.000000,1.000000}%
\pgfsetstrokecolor{textcolor}%
\pgfsetfillcolor{textcolor}%
\pgftext[x=1.187381in,y=1.049611in,,]{\color{textcolor}{\ifdefined\pdftexversion\else\setmainfont{NanumMyeongjo}\rmfamily\fi\fontsize{5.000000}{6.000000}\selectfont\catcode`\^=\active\def^{\ifmmode\sp\else\^{}\fi}\catcode`\%=\active\def%{\%}8,750}}%
\end{pgfscope}%
\begin{pgfscope}%
\definecolor{textcolor}{rgb}{1.000000,1.000000,1.000000}%
\pgfsetstrokecolor{textcolor}%
\pgfsetfillcolor{textcolor}%
\pgftext[x=2.108756in,y=1.092958in,,]{\color{textcolor}{\ifdefined\pdftexversion\else\setmainfont{NanumMyeongjo}\rmfamily\fi\fontsize{5.000000}{6.000000}\selectfont\catcode`\^=\active\def^{\ifmmode\sp\else\^{}\fi}\catcode`\%=\active\def%{\%}9,315}}%
\end{pgfscope}%
\begin{pgfscope}%
\definecolor{textcolor}{rgb}{1.000000,1.000000,1.000000}%
\pgfsetstrokecolor{textcolor}%
\pgfsetfillcolor{textcolor}%
\pgftext[x=3.030132in,y=1.234354in,,]{\color{textcolor}{\ifdefined\pdftexversion\else\setmainfont{NanumMyeongjo}\rmfamily\fi\fontsize{5.000000}{6.000000}\selectfont\catcode`\^=\active\def^{\ifmmode\sp\else\^{}\fi}\catcode`\%=\active\def%{\%}11,158}}%
\end{pgfscope}%
\begin{pgfscope}%
\definecolor{textcolor}{rgb}{1.000000,1.000000,1.000000}%
\pgfsetstrokecolor{textcolor}%
\pgfsetfillcolor{textcolor}%
\pgftext[x=3.951508in,y=1.152416in,,]{\color{textcolor}{\ifdefined\pdftexversion\else\setmainfont{NanumMyeongjo}\rmfamily\fi\fontsize{5.000000}{6.000000}\selectfont\catcode`\^=\active\def^{\ifmmode\sp\else\^{}\fi}\catcode`\%=\active\def%{\%}10,090}}%
\end{pgfscope}%
\begin{pgfscope}%
\definecolor{textcolor}{rgb}{1.000000,1.000000,1.000000}%
\pgfsetstrokecolor{textcolor}%
\pgfsetfillcolor{textcolor}%
\pgftext[x=1.187381in,y=1.154488in,,]{\color{textcolor}{\ifdefined\pdftexversion\else\setmainfont{NanumMyeongjo}\rmfamily\fi\fontsize{5.000000}{6.000000}\selectfont\catcode`\^=\active\def^{\ifmmode\sp\else\^{}\fi}\catcode`\%=\active\def%{\%}1,367}}%
\end{pgfscope}%
\begin{pgfscope}%
\definecolor{textcolor}{rgb}{1.000000,1.000000,1.000000}%
\pgfsetstrokecolor{textcolor}%
\pgfsetfillcolor{textcolor}%
\pgftext[x=2.108756in,y=1.218626in,,]{\color{textcolor}{\ifdefined\pdftexversion\else\setmainfont{NanumMyeongjo}\rmfamily\fi\fontsize{5.000000}{6.000000}\selectfont\catcode`\^=\active\def^{\ifmmode\sp\else\^{}\fi}\catcode`\%=\active\def%{\%}1,638}}%
\end{pgfscope}%
\begin{pgfscope}%
\definecolor{textcolor}{rgb}{1.000000,1.000000,1.000000}%
\pgfsetstrokecolor{textcolor}%
\pgfsetfillcolor{textcolor}%
\pgftext[x=3.030132in,y=1.367003in,,]{\color{textcolor}{\ifdefined\pdftexversion\else\setmainfont{NanumMyeongjo}\rmfamily\fi\fontsize{5.000000}{6.000000}\selectfont\catcode`\^=\active\def^{\ifmmode\sp\else\^{}\fi}\catcode`\%=\active\def%{\%}1,729}}%
\end{pgfscope}%
\begin{pgfscope}%
\definecolor{textcolor}{rgb}{1.000000,1.000000,1.000000}%
\pgfsetstrokecolor{textcolor}%
\pgfsetfillcolor{textcolor}%
\pgftext[x=3.951508in,y=1.303786in,,]{\color{textcolor}{\ifdefined\pdftexversion\else\setmainfont{NanumMyeongjo}\rmfamily\fi\fontsize{5.000000}{6.000000}\selectfont\catcode`\^=\active\def^{\ifmmode\sp\else\^{}\fi}\catcode`\%=\active\def%{\%}1,973}}%
\end{pgfscope}%
\begin{pgfscope}%
\definecolor{textcolor}{rgb}{1.000000,1.000000,1.000000}%
\pgfsetstrokecolor{textcolor}%
\pgfsetfillcolor{textcolor}%
\pgftext[x=1.402368in,y=0.807864in,,]{\color{textcolor}{\ifdefined\pdftexversion\else\setmainfont{NanumMyeongjo}\rmfamily\fi\fontsize{5.000000}{6.000000}\selectfont\catcode`\^=\active\def^{\ifmmode\sp\else\^{}\fi}\catcode`\%=\active\def%{\%}5,599}}%
\end{pgfscope}%
\begin{pgfscope}%
\definecolor{textcolor}{rgb}{1.000000,1.000000,1.000000}%
\pgfsetstrokecolor{textcolor}%
\pgfsetfillcolor{textcolor}%
\pgftext[x=2.323744in,y=0.777790in,,]{\color{textcolor}{\ifdefined\pdftexversion\else\setmainfont{NanumMyeongjo}\rmfamily\fi\fontsize{5.000000}{6.000000}\selectfont\catcode`\^=\active\def^{\ifmmode\sp\else\^{}\fi}\catcode`\%=\active\def%{\%}5,207}}%
\end{pgfscope}%
\begin{pgfscope}%
\definecolor{textcolor}{rgb}{1.000000,1.000000,1.000000}%
\pgfsetstrokecolor{textcolor}%
\pgfsetfillcolor{textcolor}%
\pgftext[x=3.245120in,y=0.923022in,,]{\color{textcolor}{\ifdefined\pdftexversion\else\setmainfont{NanumMyeongjo}\rmfamily\fi\fontsize{5.000000}{6.000000}\selectfont\catcode`\^=\active\def^{\ifmmode\sp\else\^{}\fi}\catcode`\%=\active\def%{\%}7,100}}%
\end{pgfscope}%
\begin{pgfscope}%
\definecolor{textcolor}{rgb}{1.000000,1.000000,1.000000}%
\pgfsetstrokecolor{textcolor}%
\pgfsetfillcolor{textcolor}%
\pgftext[x=4.166496in,y=0.900619in,,]{\color{textcolor}{\ifdefined\pdftexversion\else\setmainfont{NanumMyeongjo}\rmfamily\fi\fontsize{5.000000}{6.000000}\selectfont\catcode`\^=\active\def^{\ifmmode\sp\else\^{}\fi}\catcode`\%=\active\def%{\%}6,808}}%
\end{pgfscope}%
\begin{pgfscope}%
\definecolor{textcolor}{rgb}{1.000000,1.000000,1.000000}%
\pgfsetstrokecolor{textcolor}%
\pgfsetfillcolor{textcolor}%
\pgftext[x=1.402368in,y=1.273865in,,]{\color{textcolor}{\ifdefined\pdftexversion\else\setmainfont{NanumMyeongjo}\rmfamily\fi\fontsize{5.000000}{6.000000}\selectfont\catcode`\^=\active\def^{\ifmmode\sp\else\^{}\fi}\catcode`\%=\active\def%{\%}6,074}}%
\end{pgfscope}%
\begin{pgfscope}%
\definecolor{textcolor}{rgb}{1.000000,1.000000,1.000000}%
\pgfsetstrokecolor{textcolor}%
\pgfsetfillcolor{textcolor}%
\pgftext[x=2.323744in,y=1.161009in,,]{\color{textcolor}{\ifdefined\pdftexversion\else\setmainfont{NanumMyeongjo}\rmfamily\fi\fontsize{5.000000}{6.000000}\selectfont\catcode`\^=\active\def^{\ifmmode\sp\else\^{}\fi}\catcode`\%=\active\def%{\%}4,995}}%
\end{pgfscope}%
\begin{pgfscope}%
\definecolor{textcolor}{rgb}{1.000000,1.000000,1.000000}%
\pgfsetstrokecolor{textcolor}%
\pgfsetfillcolor{textcolor}%
\pgftext[x=3.245120in,y=1.410043in,,]{\color{textcolor}{\ifdefined\pdftexversion\else\setmainfont{NanumMyeongjo}\rmfamily\fi\fontsize{5.000000}{6.000000}\selectfont\catcode`\^=\active\def^{\ifmmode\sp\else\^{}\fi}\catcode`\%=\active\def%{\%}6,348}}%
\end{pgfscope}%
\begin{pgfscope}%
\definecolor{textcolor}{rgb}{1.000000,1.000000,1.000000}%
\pgfsetstrokecolor{textcolor}%
\pgfsetfillcolor{textcolor}%
\pgftext[x=4.166496in,y=1.727513in,,]{\color{textcolor}{\ifdefined\pdftexversion\else\setmainfont{NanumMyeongjo}\rmfamily\fi\fontsize{5.000000}{6.000000}\selectfont\catcode`\^=\active\def^{\ifmmode\sp\else\^{}\fi}\catcode`\%=\active\def%{\%}10,778}}%
\end{pgfscope}%
\begin{pgfscope}%
\pgfsetbuttcap%
\pgfsetmiterjoin%
\definecolor{currentfill}{rgb}{0.725490,0.486275,0.164706}%
\pgfsetfillcolor{currentfill}%
\pgfsetlinewidth{1.003750pt}%
\definecolor{currentstroke}{rgb}{0.266667,0.266667,0.266667}%
\pgfsetstrokecolor{currentstroke}%
\pgfsetdash{}{0pt}%
\pgfpathmoveto{\pgfqpoint{4.506944in}{1.715359in}}%
\pgfpathlineto{\pgfqpoint{4.645833in}{1.715359in}}%
\pgfpathlineto{\pgfqpoint{4.645833in}{1.763970in}}%
\pgfpathlineto{\pgfqpoint{4.506944in}{1.763970in}}%
\pgfpathlineto{\pgfqpoint{4.506944in}{1.715359in}}%
\pgfpathclose%
\pgfusepath{stroke,fill}%
\end{pgfscope}%
\begin{pgfscope}%
\definecolor{textcolor}{rgb}{0.000000,0.000000,0.000000}%
\pgfsetstrokecolor{textcolor}%
\pgfsetfillcolor{textcolor}%
\pgftext[x=4.701389in,y=1.715359in,left,base]{\color{textcolor}{\ifdefined\pdftexversion\else\setmainfont{NanumMyeongjo}\rmfamily\fi\fontsize{5.000000}{6.000000}\selectfont\catcode`\^=\active\def^{\ifmmode\sp\else\^{}\fi}\catcode`\%=\active\def%{\%}전남-밭}}%
\end{pgfscope}%
\begin{pgfscope}%
\pgfsetbuttcap%
\pgfsetmiterjoin%
\definecolor{currentfill}{rgb}{0.725490,0.486275,0.164706}%
\pgfsetfillcolor{currentfill}%
\pgfsetfillopacity{0.700000}%
\pgfsetlinewidth{1.003750pt}%
\definecolor{currentstroke}{rgb}{0.266667,0.266667,0.266667}%
\pgfsetstrokecolor{currentstroke}%
\pgfsetstrokeopacity{0.700000}%
\pgfsetdash{}{0pt}%
\pgfpathmoveto{\pgfqpoint{4.506944in}{1.609090in}}%
\pgfpathlineto{\pgfqpoint{4.645833in}{1.609090in}}%
\pgfpathlineto{\pgfqpoint{4.645833in}{1.657702in}}%
\pgfpathlineto{\pgfqpoint{4.506944in}{1.657702in}}%
\pgfpathlineto{\pgfqpoint{4.506944in}{1.609090in}}%
\pgfpathclose%
\pgfusepath{stroke,fill}%
\end{pgfscope}%
\begin{pgfscope}%
\definecolor{textcolor}{rgb}{0.000000,0.000000,0.000000}%
\pgfsetstrokecolor{textcolor}%
\pgfsetfillcolor{textcolor}%
\pgftext[x=4.701389in,y=1.609090in,left,base]{\color{textcolor}{\ifdefined\pdftexversion\else\setmainfont{NanumMyeongjo}\rmfamily\fi\fontsize{5.000000}{6.000000}\selectfont\catcode`\^=\active\def^{\ifmmode\sp\else\^{}\fi}\catcode`\%=\active\def%{\%}전남-논}}%
\end{pgfscope}%
\begin{pgfscope}%
\pgfsetbuttcap%
\pgfsetmiterjoin%
\definecolor{currentfill}{rgb}{0.235294,0.490196,0.764706}%
\pgfsetfillcolor{currentfill}%
\pgfsetlinewidth{1.003750pt}%
\definecolor{currentstroke}{rgb}{0.266667,0.266667,0.266667}%
\pgfsetstrokecolor{currentstroke}%
\pgfsetdash{}{0pt}%
\pgfpathmoveto{\pgfqpoint{4.506944in}{1.502822in}}%
\pgfpathlineto{\pgfqpoint{4.645833in}{1.502822in}}%
\pgfpathlineto{\pgfqpoint{4.645833in}{1.551433in}}%
\pgfpathlineto{\pgfqpoint{4.506944in}{1.551433in}}%
\pgfpathlineto{\pgfqpoint{4.506944in}{1.502822in}}%
\pgfpathclose%
\pgfusepath{stroke,fill}%
\end{pgfscope}%
\begin{pgfscope}%
\definecolor{textcolor}{rgb}{0.000000,0.000000,0.000000}%
\pgfsetstrokecolor{textcolor}%
\pgfsetfillcolor{textcolor}%
\pgftext[x=4.701389in,y=1.502822in,left,base]{\color{textcolor}{\ifdefined\pdftexversion\else\setmainfont{NanumMyeongjo}\rmfamily\fi\fontsize{5.000000}{6.000000}\selectfont\catcode`\^=\active\def^{\ifmmode\sp\else\^{}\fi}\catcode`\%=\active\def%{\%}경북-밭}}%
\end{pgfscope}%
\begin{pgfscope}%
\pgfsetbuttcap%
\pgfsetmiterjoin%
\definecolor{currentfill}{rgb}{0.235294,0.490196,0.764706}%
\pgfsetfillcolor{currentfill}%
\pgfsetfillopacity{0.700000}%
\pgfsetlinewidth{1.003750pt}%
\definecolor{currentstroke}{rgb}{0.266667,0.266667,0.266667}%
\pgfsetstrokecolor{currentstroke}%
\pgfsetstrokeopacity{0.700000}%
\pgfsetdash{}{0pt}%
\pgfpathmoveto{\pgfqpoint{4.506944in}{1.396553in}}%
\pgfpathlineto{\pgfqpoint{4.645833in}{1.396553in}}%
\pgfpathlineto{\pgfqpoint{4.645833in}{1.445164in}}%
\pgfpathlineto{\pgfqpoint{4.506944in}{1.445164in}}%
\pgfpathlineto{\pgfqpoint{4.506944in}{1.396553in}}%
\pgfpathclose%
\pgfusepath{stroke,fill}%
\end{pgfscope}%
\begin{pgfscope}%
\definecolor{textcolor}{rgb}{0.000000,0.000000,0.000000}%
\pgfsetstrokecolor{textcolor}%
\pgfsetfillcolor{textcolor}%
\pgftext[x=4.701389in,y=1.396553in,left,base]{\color{textcolor}{\ifdefined\pdftexversion\else\setmainfont{NanumMyeongjo}\rmfamily\fi\fontsize{5.000000}{6.000000}\selectfont\catcode`\^=\active\def^{\ifmmode\sp\else\^{}\fi}\catcode`\%=\active\def%{\%}경북-논}}%
\end{pgfscope}%
\begin{pgfscope}%
\pgfsetbuttcap%
\pgfsetmiterjoin%
\definecolor{currentfill}{rgb}{0.337255,0.713725,0.627451}%
\pgfsetfillcolor{currentfill}%
\pgfsetlinewidth{1.003750pt}%
\definecolor{currentstroke}{rgb}{0.266667,0.266667,0.266667}%
\pgfsetstrokecolor{currentstroke}%
\pgfsetdash{}{0pt}%
\pgfpathmoveto{\pgfqpoint{4.506944in}{1.290284in}}%
\pgfpathlineto{\pgfqpoint{4.645833in}{1.290284in}}%
\pgfpathlineto{\pgfqpoint{4.645833in}{1.338895in}}%
\pgfpathlineto{\pgfqpoint{4.506944in}{1.338895in}}%
\pgfpathlineto{\pgfqpoint{4.506944in}{1.290284in}}%
\pgfpathclose%
\pgfusepath{stroke,fill}%
\end{pgfscope}%
\begin{pgfscope}%
\definecolor{textcolor}{rgb}{0.000000,0.000000,0.000000}%
\pgfsetstrokecolor{textcolor}%
\pgfsetfillcolor{textcolor}%
\pgftext[x=4.701389in,y=1.290284in,left,base]{\color{textcolor}{\ifdefined\pdftexversion\else\setmainfont{NanumMyeongjo}\rmfamily\fi\fontsize{5.000000}{6.000000}\selectfont\catcode`\^=\active\def^{\ifmmode\sp\else\^{}\fi}\catcode`\%=\active\def%{\%}전북-밭}}%
\end{pgfscope}%
\begin{pgfscope}%
\pgfsetbuttcap%
\pgfsetmiterjoin%
\definecolor{currentfill}{rgb}{0.337255,0.713725,0.627451}%
\pgfsetfillcolor{currentfill}%
\pgfsetfillopacity{0.700000}%
\pgfsetlinewidth{1.003750pt}%
\definecolor{currentstroke}{rgb}{0.266667,0.266667,0.266667}%
\pgfsetstrokecolor{currentstroke}%
\pgfsetstrokeopacity{0.700000}%
\pgfsetdash{}{0pt}%
\pgfpathmoveto{\pgfqpoint{4.506944in}{1.184015in}}%
\pgfpathlineto{\pgfqpoint{4.645833in}{1.184015in}}%
\pgfpathlineto{\pgfqpoint{4.645833in}{1.232627in}}%
\pgfpathlineto{\pgfqpoint{4.506944in}{1.232627in}}%
\pgfpathlineto{\pgfqpoint{4.506944in}{1.184015in}}%
\pgfpathclose%
\pgfusepath{stroke,fill}%
\end{pgfscope}%
\begin{pgfscope}%
\definecolor{textcolor}{rgb}{0.000000,0.000000,0.000000}%
\pgfsetstrokecolor{textcolor}%
\pgfsetfillcolor{textcolor}%
\pgftext[x=4.701389in,y=1.184015in,left,base]{\color{textcolor}{\ifdefined\pdftexversion\else\setmainfont{NanumMyeongjo}\rmfamily\fi\fontsize{5.000000}{6.000000}\selectfont\catcode`\^=\active\def^{\ifmmode\sp\else\^{}\fi}\catcode`\%=\active\def%{\%}전북-논}}%
\end{pgfscope}%
\begin{pgfscope}%
\definecolor{textcolor}{rgb}{0.000000,0.000000,0.000000}%
\pgfsetstrokecolor{textcolor}%
\pgfsetfillcolor{textcolor}%
\pgftext[x=0.972393in,y=1.220858in,,bottom]{\color{textcolor}{\ifdefined\pdftexversion\else\setmainfont{NanumMyeongjo}\rmfamily\fi\fontsize{5.000000}{6.000000}\bfseries\selectfont\catcode`\^=\active\def^{\ifmmode\sp\else\^{}\fi}\catcode`\%=\active\def%{\%}9,939}}%
\end{pgfscope}%
\begin{pgfscope}%
\definecolor{textcolor}{rgb}{0.000000,0.000000,0.000000}%
\pgfsetstrokecolor{textcolor}%
\pgfsetfillcolor{textcolor}%
\pgftext[x=1.893769in,y=1.177971in,,bottom]{\color{textcolor}{\ifdefined\pdftexversion\else\setmainfont{NanumMyeongjo}\rmfamily\fi\fontsize{5.000000}{6.000000}\bfseries\selectfont\catcode`\^=\active\def^{\ifmmode\sp\else\^{}\fi}\catcode`\%=\active\def%{\%}9,380}}%
\end{pgfscope}%
\begin{pgfscope}%
\definecolor{textcolor}{rgb}{0.000000,0.000000,0.000000}%
\pgfsetstrokecolor{textcolor}%
\pgfsetfillcolor{textcolor}%
\pgftext[x=2.815145in,y=1.300187in,,bottom]{\color{textcolor}{\ifdefined\pdftexversion\else\setmainfont{NanumMyeongjo}\rmfamily\fi\fontsize{5.000000}{6.000000}\bfseries\selectfont\catcode`\^=\active\def^{\ifmmode\sp\else\^{}\fi}\catcode`\%=\active\def%{\%}10,973}}%
\end{pgfscope}%
\begin{pgfscope}%
\definecolor{textcolor}{rgb}{0.000000,0.000000,0.000000}%
\pgfsetstrokecolor{textcolor}%
\pgfsetfillcolor{textcolor}%
\pgftext[x=3.736521in,y=1.274716in,,bottom]{\color{textcolor}{\ifdefined\pdftexversion\else\setmainfont{NanumMyeongjo}\rmfamily\fi\fontsize{5.000000}{6.000000}\bfseries\selectfont\catcode`\^=\active\def^{\ifmmode\sp\else\^{}\fi}\catcode`\%=\active\def%{\%}10,641}}%
\end{pgfscope}%
\begin{pgfscope}%
\definecolor{textcolor}{rgb}{0.000000,0.000000,0.000000}%
\pgfsetstrokecolor{textcolor}%
\pgfsetfillcolor{textcolor}%
\pgftext[x=1.187381in,y=1.234514in,,bottom]{\color{textcolor}{\ifdefined\pdftexversion\else\setmainfont{NanumMyeongjo}\rmfamily\fi\fontsize{5.000000}{6.000000}\bfseries\selectfont\catcode`\^=\active\def^{\ifmmode\sp\else\^{}\fi}\catcode`\%=\active\def%{\%}10,117}}%
\end{pgfscope}%
\begin{pgfscope}%
\definecolor{textcolor}{rgb}{0.000000,0.000000,0.000000}%
\pgfsetstrokecolor{textcolor}%
\pgfsetfillcolor{textcolor}%
\pgftext[x=2.108756in,y=1.298653in,,bottom]{\color{textcolor}{\ifdefined\pdftexversion\else\setmainfont{NanumMyeongjo}\rmfamily\fi\fontsize{5.000000}{6.000000}\bfseries\selectfont\catcode`\^=\active\def^{\ifmmode\sp\else\^{}\fi}\catcode`\%=\active\def%{\%}10,953}}%
\end{pgfscope}%
\begin{pgfscope}%
\definecolor{textcolor}{rgb}{0.000000,0.000000,0.000000}%
\pgfsetstrokecolor{textcolor}%
\pgfsetfillcolor{textcolor}%
\pgftext[x=3.030132in,y=1.447030in,,bottom]{\color{textcolor}{\ifdefined\pdftexversion\else\setmainfont{NanumMyeongjo}\rmfamily\fi\fontsize{5.000000}{6.000000}\bfseries\selectfont\catcode`\^=\active\def^{\ifmmode\sp\else\^{}\fi}\catcode`\%=\active\def%{\%}12,887}}%
\end{pgfscope}%
\begin{pgfscope}%
\definecolor{textcolor}{rgb}{0.000000,0.000000,0.000000}%
\pgfsetstrokecolor{textcolor}%
\pgfsetfillcolor{textcolor}%
\pgftext[x=3.951508in,y=1.383812in,,bottom]{\color{textcolor}{\ifdefined\pdftexversion\else\setmainfont{NanumMyeongjo}\rmfamily\fi\fontsize{5.000000}{6.000000}\bfseries\selectfont\catcode`\^=\active\def^{\ifmmode\sp\else\^{}\fi}\catcode`\%=\active\def%{\%}12,063}}%
\end{pgfscope}%
\begin{pgfscope}%
\definecolor{textcolor}{rgb}{0.000000,0.000000,0.000000}%
\pgfsetstrokecolor{textcolor}%
\pgfsetfillcolor{textcolor}%
\pgftext[x=1.402368in,y=1.353891in,,bottom]{\color{textcolor}{\ifdefined\pdftexversion\else\setmainfont{NanumMyeongjo}\rmfamily\fi\fontsize{5.000000}{6.000000}\bfseries\selectfont\catcode`\^=\active\def^{\ifmmode\sp\else\^{}\fi}\catcode`\%=\active\def%{\%}11,673}}%
\end{pgfscope}%
\begin{pgfscope}%
\definecolor{textcolor}{rgb}{0.000000,0.000000,0.000000}%
\pgfsetstrokecolor{textcolor}%
\pgfsetfillcolor{textcolor}%
\pgftext[x=2.323744in,y=1.241036in,,bottom]{\color{textcolor}{\ifdefined\pdftexversion\else\setmainfont{NanumMyeongjo}\rmfamily\fi\fontsize{5.000000}{6.000000}\bfseries\selectfont\catcode`\^=\active\def^{\ifmmode\sp\else\^{}\fi}\catcode`\%=\active\def%{\%}10,202}}%
\end{pgfscope}%
\begin{pgfscope}%
\definecolor{textcolor}{rgb}{0.000000,0.000000,0.000000}%
\pgfsetstrokecolor{textcolor}%
\pgfsetfillcolor{textcolor}%
\pgftext[x=3.245120in,y=1.490070in,,bottom]{\color{textcolor}{\ifdefined\pdftexversion\else\setmainfont{NanumMyeongjo}\rmfamily\fi\fontsize{5.000000}{6.000000}\bfseries\selectfont\catcode`\^=\active\def^{\ifmmode\sp\else\^{}\fi}\catcode`\%=\active\def%{\%}13,448}}%
\end{pgfscope}%
\begin{pgfscope}%
\definecolor{textcolor}{rgb}{0.000000,0.000000,0.000000}%
\pgfsetstrokecolor{textcolor}%
\pgfsetfillcolor{textcolor}%
\pgftext[x=4.166496in,y=1.807540in,,bottom]{\color{textcolor}{\ifdefined\pdftexversion\else\setmainfont{NanumMyeongjo}\rmfamily\fi\fontsize{5.000000}{6.000000}\bfseries\selectfont\catcode`\^=\active\def^{\ifmmode\sp\else\^{}\fi}\catcode`\%=\active\def%{\%}17,586}}%
\end{pgfscope}%
\begin{pgfscope}%
\definecolor{textcolor}{rgb}{0.333333,0.333333,0.333333}%
\pgfsetstrokecolor{textcolor}%
\pgfsetfillcolor{textcolor}%
\pgftext[x=1.277778in,y=0.208333in,,top]{\color{textcolor}{\ifdefined\pdftexversion\else\setmainfont{NanumMyeongjo}\rmfamily\fi\fontsize{5.000000}{6.000000}\selectfont\catcode`\^=\active\def^{\ifmmode\sp\else\^{}\fi}\catcode`\%=\active\def%{\%}출처: 국가농식품통계서비스(KASS) 자료 기반 저자 작성}}%
\end{pgfscope}%
\begin{pgfscope}%
\definecolor{textcolor}{rgb}{0.333333,0.333333,0.333333}%
\pgfsetstrokecolor{textcolor}%
\pgfsetfillcolor{textcolor}%
\pgftext[x=4.166667in,y=1.979167in,,top]{\color{textcolor}{\ifdefined\pdftexversion\else\setmainfont{NanumMyeongjo}\rmfamily\fi\fontsize{5.000000}{6.000000}\selectfont\catcode`\^=\active\def^{\ifmmode\sp\else\^{}\fi}\catcode`\%=\active\def%{\%}(단위: ha)}}%
\end{pgfscope}%
\end{pgfpicture}%
\makeatother%
\endgroup%
}
\end{center}
}


\slide
{\maintitle}
{\chapterthree}
{논콩 생산지역별 면적}{
\begin{center}
    \hspace*{-25pt}{%% Creator: Matplotlib, PGF backend
%%
%% To include the figure in your LaTeX document, write
%%   \input{<filename>.pgf}
%%
%% Make sure the required packages are loaded in your preamble
%%   \usepackage{pgf}
%%
%% Also ensure that all the required font packages are loaded; for instance,
%% the lmodern package is sometimes necessary when using math font.
%%   \usepackage{lmodern}
%%
%% Figures using additional raster images can only be included by \input if
%% they are in the same directory as the main LaTeX file. For loading figures
%% from other directories you can use the `import` package
%%   \usepackage{import}
%%
%% and then include the figures with
%%   \import{<path to file>}{<filename>.pgf}
%%
%% Matplotlib used the following preamble
%%   \def\mathdefault#1{#1}
%%   \everymath=\expandafter{\the\everymath\displaystyle}
%%   \IfFileExists{scrextend.sty}{
%%     \usepackage[fontsize=9.000000pt]{scrextend}
%%   }{
%%     \renewcommand{\normalsize}{\fontsize{9.000000}{10.800000}\selectfont}
%%     \normalsize
%%   }
%%   
%%   \ifdefined\pdftexversion\else  % non-pdftex case.
%%     \usepackage{fontspec}
%%     \setmainfont{DejaVuSerif.ttf}[Path=\detokenize{/home/user/.cache/pypoetry/virtualenvs/graph-KASAOWVd-py3.12/lib/python3.12/site-packages/matplotlib/mpl-data/fonts/ttf/}]
%%     \setsansfont{DejaVuSans.ttf}[Path=\detokenize{/home/user/.cache/pypoetry/virtualenvs/graph-KASAOWVd-py3.12/lib/python3.12/site-packages/matplotlib/mpl-data/fonts/ttf/}]
%%     \setmonofont{DejaVuSansMono.ttf}[Path=\detokenize{/home/user/.cache/pypoetry/virtualenvs/graph-KASAOWVd-py3.12/lib/python3.12/site-packages/matplotlib/mpl-data/fonts/ttf/}]
%%   \fi
%%   \makeatletter\@ifpackageloaded{underscore}{}{\usepackage[strings]{underscore}}\makeatother
%%
\begingroup%
\makeatletter%
\begin{pgfpicture}%
\pgfpathrectangle{\pgfpointorigin}{\pgfqpoint{6.250000in}{3.194444in}}%
\pgfusepath{use as bounding box, clip}%
\begin{pgfscope}%
\pgfsetbuttcap%
\pgfsetmiterjoin%
\definecolor{currentfill}{rgb}{1.000000,1.000000,1.000000}%
\pgfsetfillcolor{currentfill}%
\pgfsetlinewidth{0.000000pt}%
\definecolor{currentstroke}{rgb}{1.000000,1.000000,1.000000}%
\pgfsetstrokecolor{currentstroke}%
\pgfsetdash{}{0pt}%
\pgfpathmoveto{\pgfqpoint{0.000000in}{0.000000in}}%
\pgfpathlineto{\pgfqpoint{6.250000in}{0.000000in}}%
\pgfpathlineto{\pgfqpoint{6.250000in}{3.194444in}}%
\pgfpathlineto{\pgfqpoint{0.000000in}{3.194444in}}%
\pgfpathlineto{\pgfqpoint{0.000000in}{0.000000in}}%
\pgfpathclose%
\pgfusepath{fill}%
\end{pgfscope}%
\begin{pgfscope}%
\pgfsetbuttcap%
\pgfsetmiterjoin%
\definecolor{currentfill}{rgb}{1.000000,1.000000,1.000000}%
\pgfsetfillcolor{currentfill}%
\pgfsetlinewidth{0.000000pt}%
\definecolor{currentstroke}{rgb}{0.000000,0.000000,0.000000}%
\pgfsetstrokecolor{currentstroke}%
\pgfsetstrokeopacity{0.000000}%
\pgfsetdash{}{0pt}%
\pgfpathmoveto{\pgfqpoint{0.781250in}{0.638889in}}%
\pgfpathlineto{\pgfqpoint{5.000000in}{0.638889in}}%
\pgfpathlineto{\pgfqpoint{5.000000in}{2.811111in}}%
\pgfpathlineto{\pgfqpoint{0.781250in}{2.811111in}}%
\pgfpathlineto{\pgfqpoint{0.781250in}{0.638889in}}%
\pgfpathclose%
\pgfusepath{fill}%
\end{pgfscope}%
\begin{pgfscope}%
\pgfsetbuttcap%
\pgfsetroundjoin%
\definecolor{currentfill}{rgb}{0.000000,0.000000,0.000000}%
\pgfsetfillcolor{currentfill}%
\pgfsetlinewidth{0.752812pt}%
\definecolor{currentstroke}{rgb}{0.000000,0.000000,0.000000}%
\pgfsetstrokecolor{currentstroke}%
\pgfsetdash{}{0pt}%
\pgfsys@defobject{currentmarker}{\pgfqpoint{0.000000in}{-0.013889in}}{\pgfqpoint{0.000000in}{0.000000in}}{%
\pgfpathmoveto{\pgfqpoint{0.000000in}{0.000000in}}%
\pgfpathlineto{\pgfqpoint{0.000000in}{-0.013889in}}%
\pgfusepath{stroke,fill}%
}%
\begin{pgfscope}%
\pgfsys@transformshift{1.335803in}{0.638889in}%
\pgfsys@useobject{currentmarker}{}%
\end{pgfscope}%
\end{pgfscope}%
\begin{pgfscope}%
\definecolor{textcolor}{rgb}{0.000000,0.000000,0.000000}%
\pgfsetstrokecolor{textcolor}%
\pgfsetfillcolor{textcolor}%
\pgftext[x=1.335803in,y=0.576389in,,top]{\color{textcolor}{\ifdefined\pdftexversion\else\setmainfont{NanumMyeongjo}\rmfamily\fi\fontsize{9.000000}{10.800000}\selectfont\catcode`\^=\active\def^{\ifmmode\sp\else\^{}\fi}\catcode`\%=\active\def%{\%}2020}}%
\end{pgfscope}%
\begin{pgfscope}%
\pgfsetbuttcap%
\pgfsetroundjoin%
\definecolor{currentfill}{rgb}{0.000000,0.000000,0.000000}%
\pgfsetfillcolor{currentfill}%
\pgfsetlinewidth{0.752812pt}%
\definecolor{currentstroke}{rgb}{0.000000,0.000000,0.000000}%
\pgfsetstrokecolor{currentstroke}%
\pgfsetdash{}{0pt}%
\pgfsys@defobject{currentmarker}{\pgfqpoint{0.000000in}{-0.013889in}}{\pgfqpoint{0.000000in}{0.000000in}}{%
\pgfpathmoveto{\pgfqpoint{0.000000in}{0.000000in}}%
\pgfpathlineto{\pgfqpoint{0.000000in}{-0.013889in}}%
\pgfusepath{stroke,fill}%
}%
\begin{pgfscope}%
\pgfsys@transformshift{2.372351in}{0.638889in}%
\pgfsys@useobject{currentmarker}{}%
\end{pgfscope}%
\end{pgfscope}%
\begin{pgfscope}%
\definecolor{textcolor}{rgb}{0.000000,0.000000,0.000000}%
\pgfsetstrokecolor{textcolor}%
\pgfsetfillcolor{textcolor}%
\pgftext[x=2.372351in,y=0.576389in,,top]{\color{textcolor}{\ifdefined\pdftexversion\else\setmainfont{NanumMyeongjo}\rmfamily\fi\fontsize{9.000000}{10.800000}\selectfont\catcode`\^=\active\def^{\ifmmode\sp\else\^{}\fi}\catcode`\%=\active\def%{\%}2021}}%
\end{pgfscope}%
\begin{pgfscope}%
\pgfsetbuttcap%
\pgfsetroundjoin%
\definecolor{currentfill}{rgb}{0.000000,0.000000,0.000000}%
\pgfsetfillcolor{currentfill}%
\pgfsetlinewidth{0.752812pt}%
\definecolor{currentstroke}{rgb}{0.000000,0.000000,0.000000}%
\pgfsetstrokecolor{currentstroke}%
\pgfsetdash{}{0pt}%
\pgfsys@defobject{currentmarker}{\pgfqpoint{0.000000in}{-0.013889in}}{\pgfqpoint{0.000000in}{0.000000in}}{%
\pgfpathmoveto{\pgfqpoint{0.000000in}{0.000000in}}%
\pgfpathlineto{\pgfqpoint{0.000000in}{-0.013889in}}%
\pgfusepath{stroke,fill}%
}%
\begin{pgfscope}%
\pgfsys@transformshift{3.408899in}{0.638889in}%
\pgfsys@useobject{currentmarker}{}%
\end{pgfscope}%
\end{pgfscope}%
\begin{pgfscope}%
\definecolor{textcolor}{rgb}{0.000000,0.000000,0.000000}%
\pgfsetstrokecolor{textcolor}%
\pgfsetfillcolor{textcolor}%
\pgftext[x=3.408899in,y=0.576389in,,top]{\color{textcolor}{\ifdefined\pdftexversion\else\setmainfont{NanumMyeongjo}\rmfamily\fi\fontsize{9.000000}{10.800000}\selectfont\catcode`\^=\active\def^{\ifmmode\sp\else\^{}\fi}\catcode`\%=\active\def%{\%}2022}}%
\end{pgfscope}%
\begin{pgfscope}%
\pgfsetbuttcap%
\pgfsetroundjoin%
\definecolor{currentfill}{rgb}{0.000000,0.000000,0.000000}%
\pgfsetfillcolor{currentfill}%
\pgfsetlinewidth{0.752812pt}%
\definecolor{currentstroke}{rgb}{0.000000,0.000000,0.000000}%
\pgfsetstrokecolor{currentstroke}%
\pgfsetdash{}{0pt}%
\pgfsys@defobject{currentmarker}{\pgfqpoint{0.000000in}{-0.013889in}}{\pgfqpoint{0.000000in}{0.000000in}}{%
\pgfpathmoveto{\pgfqpoint{0.000000in}{0.000000in}}%
\pgfpathlineto{\pgfqpoint{0.000000in}{-0.013889in}}%
\pgfusepath{stroke,fill}%
}%
\begin{pgfscope}%
\pgfsys@transformshift{4.445447in}{0.638889in}%
\pgfsys@useobject{currentmarker}{}%
\end{pgfscope}%
\end{pgfscope}%
\begin{pgfscope}%
\definecolor{textcolor}{rgb}{0.000000,0.000000,0.000000}%
\pgfsetstrokecolor{textcolor}%
\pgfsetfillcolor{textcolor}%
\pgftext[x=4.445447in,y=0.576389in,,top]{\color{textcolor}{\ifdefined\pdftexversion\else\setmainfont{NanumMyeongjo}\rmfamily\fi\fontsize{9.000000}{10.800000}\selectfont\catcode`\^=\active\def^{\ifmmode\sp\else\^{}\fi}\catcode`\%=\active\def%{\%}2023}}%
\end{pgfscope}%
\begin{pgfscope}%
\pgfpathrectangle{\pgfqpoint{0.781250in}{0.638889in}}{\pgfqpoint{4.218750in}{2.172222in}}%
\pgfusepath{clip}%
\pgfsetbuttcap%
\pgfsetroundjoin%
\pgfsetlinewidth{0.602250pt}%
\definecolor{currentstroke}{rgb}{0.690196,0.690196,0.690196}%
\pgfsetstrokecolor{currentstroke}%
\pgfsetstrokeopacity{0.400000}%
\pgfsetdash{{2.220000pt}{0.960000pt}}{0.000000pt}%
\pgfpathmoveto{\pgfqpoint{0.781250in}{0.638889in}}%
\pgfpathlineto{\pgfqpoint{5.000000in}{0.638889in}}%
\pgfusepath{stroke}%
\end{pgfscope}%
\begin{pgfscope}%
\pgfsetbuttcap%
\pgfsetroundjoin%
\definecolor{currentfill}{rgb}{0.000000,0.000000,0.000000}%
\pgfsetfillcolor{currentfill}%
\pgfsetlinewidth{0.752812pt}%
\definecolor{currentstroke}{rgb}{0.000000,0.000000,0.000000}%
\pgfsetstrokecolor{currentstroke}%
\pgfsetdash{}{0pt}%
\pgfsys@defobject{currentmarker}{\pgfqpoint{-0.013889in}{0.000000in}}{\pgfqpoint{-0.000000in}{0.000000in}}{%
\pgfpathmoveto{\pgfqpoint{-0.000000in}{0.000000in}}%
\pgfpathlineto{\pgfqpoint{-0.013889in}{0.000000in}}%
\pgfusepath{stroke,fill}%
}%
\begin{pgfscope}%
\pgfsys@transformshift{0.781250in}{0.638889in}%
\pgfsys@useobject{currentmarker}{}%
\end{pgfscope}%
\end{pgfscope}%
\begin{pgfscope}%
\definecolor{textcolor}{rgb}{0.000000,0.000000,0.000000}%
\pgfsetstrokecolor{textcolor}%
\pgfsetfillcolor{textcolor}%
\pgftext[x=0.651611in, y=0.588962in, left, base]{\color{textcolor}{\ifdefined\pdftexversion\else\setmainfont{NanumMyeongjo}\rmfamily\fi\fontsize{9.000000}{10.800000}\selectfont\catcode`\^=\active\def^{\ifmmode\sp\else\^{}\fi}\catcode`\%=\active\def%{\%}0}}%
\end{pgfscope}%
\begin{pgfscope}%
\pgfpathrectangle{\pgfqpoint{0.781250in}{0.638889in}}{\pgfqpoint{4.218750in}{2.172222in}}%
\pgfusepath{clip}%
\pgfsetbuttcap%
\pgfsetroundjoin%
\pgfsetlinewidth{0.602250pt}%
\definecolor{currentstroke}{rgb}{0.690196,0.690196,0.690196}%
\pgfsetstrokecolor{currentstroke}%
\pgfsetstrokeopacity{0.400000}%
\pgfsetdash{{2.220000pt}{0.960000pt}}{0.000000pt}%
\pgfpathmoveto{\pgfqpoint{0.781250in}{0.856111in}}%
\pgfpathlineto{\pgfqpoint{5.000000in}{0.856111in}}%
\pgfusepath{stroke}%
\end{pgfscope}%
\begin{pgfscope}%
\pgfsetbuttcap%
\pgfsetroundjoin%
\definecolor{currentfill}{rgb}{0.000000,0.000000,0.000000}%
\pgfsetfillcolor{currentfill}%
\pgfsetlinewidth{0.752812pt}%
\definecolor{currentstroke}{rgb}{0.000000,0.000000,0.000000}%
\pgfsetstrokecolor{currentstroke}%
\pgfsetdash{}{0pt}%
\pgfsys@defobject{currentmarker}{\pgfqpoint{-0.013889in}{0.000000in}}{\pgfqpoint{-0.000000in}{0.000000in}}{%
\pgfpathmoveto{\pgfqpoint{-0.000000in}{0.000000in}}%
\pgfpathlineto{\pgfqpoint{-0.013889in}{0.000000in}}%
\pgfusepath{stroke,fill}%
}%
\begin{pgfscope}%
\pgfsys@transformshift{0.781250in}{0.856111in}%
\pgfsys@useobject{currentmarker}{}%
\end{pgfscope}%
\end{pgfscope}%
\begin{pgfscope}%
\definecolor{textcolor}{rgb}{0.000000,0.000000,0.000000}%
\pgfsetstrokecolor{textcolor}%
\pgfsetfillcolor{textcolor}%
\pgftext[x=0.416016in, y=0.806184in, left, base]{\color{textcolor}{\ifdefined\pdftexversion\else\setmainfont{NanumMyeongjo}\rmfamily\fi\fontsize{9.000000}{10.800000}\selectfont\catcode`\^=\active\def^{\ifmmode\sp\else\^{}\fi}\catcode`\%=\active\def%{\%}2,000}}%
\end{pgfscope}%
\begin{pgfscope}%
\pgfpathrectangle{\pgfqpoint{0.781250in}{0.638889in}}{\pgfqpoint{4.218750in}{2.172222in}}%
\pgfusepath{clip}%
\pgfsetbuttcap%
\pgfsetroundjoin%
\pgfsetlinewidth{0.602250pt}%
\definecolor{currentstroke}{rgb}{0.690196,0.690196,0.690196}%
\pgfsetstrokecolor{currentstroke}%
\pgfsetstrokeopacity{0.400000}%
\pgfsetdash{{2.220000pt}{0.960000pt}}{0.000000pt}%
\pgfpathmoveto{\pgfqpoint{0.781250in}{1.073333in}}%
\pgfpathlineto{\pgfqpoint{5.000000in}{1.073333in}}%
\pgfusepath{stroke}%
\end{pgfscope}%
\begin{pgfscope}%
\pgfsetbuttcap%
\pgfsetroundjoin%
\definecolor{currentfill}{rgb}{0.000000,0.000000,0.000000}%
\pgfsetfillcolor{currentfill}%
\pgfsetlinewidth{0.752812pt}%
\definecolor{currentstroke}{rgb}{0.000000,0.000000,0.000000}%
\pgfsetstrokecolor{currentstroke}%
\pgfsetdash{}{0pt}%
\pgfsys@defobject{currentmarker}{\pgfqpoint{-0.013889in}{0.000000in}}{\pgfqpoint{-0.000000in}{0.000000in}}{%
\pgfpathmoveto{\pgfqpoint{-0.000000in}{0.000000in}}%
\pgfpathlineto{\pgfqpoint{-0.013889in}{0.000000in}}%
\pgfusepath{stroke,fill}%
}%
\begin{pgfscope}%
\pgfsys@transformshift{0.781250in}{1.073333in}%
\pgfsys@useobject{currentmarker}{}%
\end{pgfscope}%
\end{pgfscope}%
\begin{pgfscope}%
\definecolor{textcolor}{rgb}{0.000000,0.000000,0.000000}%
\pgfsetstrokecolor{textcolor}%
\pgfsetfillcolor{textcolor}%
\pgftext[x=0.416016in, y=1.023407in, left, base]{\color{textcolor}{\ifdefined\pdftexversion\else\setmainfont{NanumMyeongjo}\rmfamily\fi\fontsize{9.000000}{10.800000}\selectfont\catcode`\^=\active\def^{\ifmmode\sp\else\^{}\fi}\catcode`\%=\active\def%{\%}4,000}}%
\end{pgfscope}%
\begin{pgfscope}%
\pgfpathrectangle{\pgfqpoint{0.781250in}{0.638889in}}{\pgfqpoint{4.218750in}{2.172222in}}%
\pgfusepath{clip}%
\pgfsetbuttcap%
\pgfsetroundjoin%
\pgfsetlinewidth{0.602250pt}%
\definecolor{currentstroke}{rgb}{0.690196,0.690196,0.690196}%
\pgfsetstrokecolor{currentstroke}%
\pgfsetstrokeopacity{0.400000}%
\pgfsetdash{{2.220000pt}{0.960000pt}}{0.000000pt}%
\pgfpathmoveto{\pgfqpoint{0.781250in}{1.290556in}}%
\pgfpathlineto{\pgfqpoint{5.000000in}{1.290556in}}%
\pgfusepath{stroke}%
\end{pgfscope}%
\begin{pgfscope}%
\pgfsetbuttcap%
\pgfsetroundjoin%
\definecolor{currentfill}{rgb}{0.000000,0.000000,0.000000}%
\pgfsetfillcolor{currentfill}%
\pgfsetlinewidth{0.752812pt}%
\definecolor{currentstroke}{rgb}{0.000000,0.000000,0.000000}%
\pgfsetstrokecolor{currentstroke}%
\pgfsetdash{}{0pt}%
\pgfsys@defobject{currentmarker}{\pgfqpoint{-0.013889in}{0.000000in}}{\pgfqpoint{-0.000000in}{0.000000in}}{%
\pgfpathmoveto{\pgfqpoint{-0.000000in}{0.000000in}}%
\pgfpathlineto{\pgfqpoint{-0.013889in}{0.000000in}}%
\pgfusepath{stroke,fill}%
}%
\begin{pgfscope}%
\pgfsys@transformshift{0.781250in}{1.290556in}%
\pgfsys@useobject{currentmarker}{}%
\end{pgfscope}%
\end{pgfscope}%
\begin{pgfscope}%
\definecolor{textcolor}{rgb}{0.000000,0.000000,0.000000}%
\pgfsetstrokecolor{textcolor}%
\pgfsetfillcolor{textcolor}%
\pgftext[x=0.416016in, y=1.240629in, left, base]{\color{textcolor}{\ifdefined\pdftexversion\else\setmainfont{NanumMyeongjo}\rmfamily\fi\fontsize{9.000000}{10.800000}\selectfont\catcode`\^=\active\def^{\ifmmode\sp\else\^{}\fi}\catcode`\%=\active\def%{\%}6,000}}%
\end{pgfscope}%
\begin{pgfscope}%
\pgfpathrectangle{\pgfqpoint{0.781250in}{0.638889in}}{\pgfqpoint{4.218750in}{2.172222in}}%
\pgfusepath{clip}%
\pgfsetbuttcap%
\pgfsetroundjoin%
\pgfsetlinewidth{0.602250pt}%
\definecolor{currentstroke}{rgb}{0.690196,0.690196,0.690196}%
\pgfsetstrokecolor{currentstroke}%
\pgfsetstrokeopacity{0.400000}%
\pgfsetdash{{2.220000pt}{0.960000pt}}{0.000000pt}%
\pgfpathmoveto{\pgfqpoint{0.781250in}{1.507778in}}%
\pgfpathlineto{\pgfqpoint{5.000000in}{1.507778in}}%
\pgfusepath{stroke}%
\end{pgfscope}%
\begin{pgfscope}%
\pgfsetbuttcap%
\pgfsetroundjoin%
\definecolor{currentfill}{rgb}{0.000000,0.000000,0.000000}%
\pgfsetfillcolor{currentfill}%
\pgfsetlinewidth{0.752812pt}%
\definecolor{currentstroke}{rgb}{0.000000,0.000000,0.000000}%
\pgfsetstrokecolor{currentstroke}%
\pgfsetdash{}{0pt}%
\pgfsys@defobject{currentmarker}{\pgfqpoint{-0.013889in}{0.000000in}}{\pgfqpoint{-0.000000in}{0.000000in}}{%
\pgfpathmoveto{\pgfqpoint{-0.000000in}{0.000000in}}%
\pgfpathlineto{\pgfqpoint{-0.013889in}{0.000000in}}%
\pgfusepath{stroke,fill}%
}%
\begin{pgfscope}%
\pgfsys@transformshift{0.781250in}{1.507778in}%
\pgfsys@useobject{currentmarker}{}%
\end{pgfscope}%
\end{pgfscope}%
\begin{pgfscope}%
\definecolor{textcolor}{rgb}{0.000000,0.000000,0.000000}%
\pgfsetstrokecolor{textcolor}%
\pgfsetfillcolor{textcolor}%
\pgftext[x=0.416016in, y=1.457851in, left, base]{\color{textcolor}{\ifdefined\pdftexversion\else\setmainfont{NanumMyeongjo}\rmfamily\fi\fontsize{9.000000}{10.800000}\selectfont\catcode`\^=\active\def^{\ifmmode\sp\else\^{}\fi}\catcode`\%=\active\def%{\%}8,000}}%
\end{pgfscope}%
\begin{pgfscope}%
\pgfpathrectangle{\pgfqpoint{0.781250in}{0.638889in}}{\pgfqpoint{4.218750in}{2.172222in}}%
\pgfusepath{clip}%
\pgfsetbuttcap%
\pgfsetroundjoin%
\pgfsetlinewidth{0.602250pt}%
\definecolor{currentstroke}{rgb}{0.690196,0.690196,0.690196}%
\pgfsetstrokecolor{currentstroke}%
\pgfsetstrokeopacity{0.400000}%
\pgfsetdash{{2.220000pt}{0.960000pt}}{0.000000pt}%
\pgfpathmoveto{\pgfqpoint{0.781250in}{1.725000in}}%
\pgfpathlineto{\pgfqpoint{5.000000in}{1.725000in}}%
\pgfusepath{stroke}%
\end{pgfscope}%
\begin{pgfscope}%
\pgfsetbuttcap%
\pgfsetroundjoin%
\definecolor{currentfill}{rgb}{0.000000,0.000000,0.000000}%
\pgfsetfillcolor{currentfill}%
\pgfsetlinewidth{0.752812pt}%
\definecolor{currentstroke}{rgb}{0.000000,0.000000,0.000000}%
\pgfsetstrokecolor{currentstroke}%
\pgfsetdash{}{0pt}%
\pgfsys@defobject{currentmarker}{\pgfqpoint{-0.013889in}{0.000000in}}{\pgfqpoint{-0.000000in}{0.000000in}}{%
\pgfpathmoveto{\pgfqpoint{-0.000000in}{0.000000in}}%
\pgfpathlineto{\pgfqpoint{-0.013889in}{0.000000in}}%
\pgfusepath{stroke,fill}%
}%
\begin{pgfscope}%
\pgfsys@transformshift{0.781250in}{1.725000in}%
\pgfsys@useobject{currentmarker}{}%
\end{pgfscope}%
\end{pgfscope}%
\begin{pgfscope}%
\definecolor{textcolor}{rgb}{0.000000,0.000000,0.000000}%
\pgfsetstrokecolor{textcolor}%
\pgfsetfillcolor{textcolor}%
\pgftext[x=0.348877in, y=1.675073in, left, base]{\color{textcolor}{\ifdefined\pdftexversion\else\setmainfont{NanumMyeongjo}\rmfamily\fi\fontsize{9.000000}{10.800000}\selectfont\catcode`\^=\active\def^{\ifmmode\sp\else\^{}\fi}\catcode`\%=\active\def%{\%}10,000}}%
\end{pgfscope}%
\begin{pgfscope}%
\pgfpathrectangle{\pgfqpoint{0.781250in}{0.638889in}}{\pgfqpoint{4.218750in}{2.172222in}}%
\pgfusepath{clip}%
\pgfsetbuttcap%
\pgfsetroundjoin%
\pgfsetlinewidth{0.602250pt}%
\definecolor{currentstroke}{rgb}{0.690196,0.690196,0.690196}%
\pgfsetstrokecolor{currentstroke}%
\pgfsetstrokeopacity{0.400000}%
\pgfsetdash{{2.220000pt}{0.960000pt}}{0.000000pt}%
\pgfpathmoveto{\pgfqpoint{0.781250in}{1.942222in}}%
\pgfpathlineto{\pgfqpoint{5.000000in}{1.942222in}}%
\pgfusepath{stroke}%
\end{pgfscope}%
\begin{pgfscope}%
\pgfsetbuttcap%
\pgfsetroundjoin%
\definecolor{currentfill}{rgb}{0.000000,0.000000,0.000000}%
\pgfsetfillcolor{currentfill}%
\pgfsetlinewidth{0.752812pt}%
\definecolor{currentstroke}{rgb}{0.000000,0.000000,0.000000}%
\pgfsetstrokecolor{currentstroke}%
\pgfsetdash{}{0pt}%
\pgfsys@defobject{currentmarker}{\pgfqpoint{-0.013889in}{0.000000in}}{\pgfqpoint{-0.000000in}{0.000000in}}{%
\pgfpathmoveto{\pgfqpoint{-0.000000in}{0.000000in}}%
\pgfpathlineto{\pgfqpoint{-0.013889in}{0.000000in}}%
\pgfusepath{stroke,fill}%
}%
\begin{pgfscope}%
\pgfsys@transformshift{0.781250in}{1.942222in}%
\pgfsys@useobject{currentmarker}{}%
\end{pgfscope}%
\end{pgfscope}%
\begin{pgfscope}%
\definecolor{textcolor}{rgb}{0.000000,0.000000,0.000000}%
\pgfsetstrokecolor{textcolor}%
\pgfsetfillcolor{textcolor}%
\pgftext[x=0.348877in, y=1.892295in, left, base]{\color{textcolor}{\ifdefined\pdftexversion\else\setmainfont{NanumMyeongjo}\rmfamily\fi\fontsize{9.000000}{10.800000}\selectfont\catcode`\^=\active\def^{\ifmmode\sp\else\^{}\fi}\catcode`\%=\active\def%{\%}12,000}}%
\end{pgfscope}%
\begin{pgfscope}%
\pgfpathrectangle{\pgfqpoint{0.781250in}{0.638889in}}{\pgfqpoint{4.218750in}{2.172222in}}%
\pgfusepath{clip}%
\pgfsetbuttcap%
\pgfsetroundjoin%
\pgfsetlinewidth{0.602250pt}%
\definecolor{currentstroke}{rgb}{0.690196,0.690196,0.690196}%
\pgfsetstrokecolor{currentstroke}%
\pgfsetstrokeopacity{0.400000}%
\pgfsetdash{{2.220000pt}{0.960000pt}}{0.000000pt}%
\pgfpathmoveto{\pgfqpoint{0.781250in}{2.159444in}}%
\pgfpathlineto{\pgfqpoint{5.000000in}{2.159444in}}%
\pgfusepath{stroke}%
\end{pgfscope}%
\begin{pgfscope}%
\pgfsetbuttcap%
\pgfsetroundjoin%
\definecolor{currentfill}{rgb}{0.000000,0.000000,0.000000}%
\pgfsetfillcolor{currentfill}%
\pgfsetlinewidth{0.752812pt}%
\definecolor{currentstroke}{rgb}{0.000000,0.000000,0.000000}%
\pgfsetstrokecolor{currentstroke}%
\pgfsetdash{}{0pt}%
\pgfsys@defobject{currentmarker}{\pgfqpoint{-0.013889in}{0.000000in}}{\pgfqpoint{-0.000000in}{0.000000in}}{%
\pgfpathmoveto{\pgfqpoint{-0.000000in}{0.000000in}}%
\pgfpathlineto{\pgfqpoint{-0.013889in}{0.000000in}}%
\pgfusepath{stroke,fill}%
}%
\begin{pgfscope}%
\pgfsys@transformshift{0.781250in}{2.159444in}%
\pgfsys@useobject{currentmarker}{}%
\end{pgfscope}%
\end{pgfscope}%
\begin{pgfscope}%
\definecolor{textcolor}{rgb}{0.000000,0.000000,0.000000}%
\pgfsetstrokecolor{textcolor}%
\pgfsetfillcolor{textcolor}%
\pgftext[x=0.348877in, y=2.109518in, left, base]{\color{textcolor}{\ifdefined\pdftexversion\else\setmainfont{NanumMyeongjo}\rmfamily\fi\fontsize{9.000000}{10.800000}\selectfont\catcode`\^=\active\def^{\ifmmode\sp\else\^{}\fi}\catcode`\%=\active\def%{\%}14,000}}%
\end{pgfscope}%
\begin{pgfscope}%
\pgfpathrectangle{\pgfqpoint{0.781250in}{0.638889in}}{\pgfqpoint{4.218750in}{2.172222in}}%
\pgfusepath{clip}%
\pgfsetbuttcap%
\pgfsetroundjoin%
\pgfsetlinewidth{0.602250pt}%
\definecolor{currentstroke}{rgb}{0.690196,0.690196,0.690196}%
\pgfsetstrokecolor{currentstroke}%
\pgfsetstrokeopacity{0.400000}%
\pgfsetdash{{2.220000pt}{0.960000pt}}{0.000000pt}%
\pgfpathmoveto{\pgfqpoint{0.781250in}{2.376667in}}%
\pgfpathlineto{\pgfqpoint{5.000000in}{2.376667in}}%
\pgfusepath{stroke}%
\end{pgfscope}%
\begin{pgfscope}%
\pgfsetbuttcap%
\pgfsetroundjoin%
\definecolor{currentfill}{rgb}{0.000000,0.000000,0.000000}%
\pgfsetfillcolor{currentfill}%
\pgfsetlinewidth{0.752812pt}%
\definecolor{currentstroke}{rgb}{0.000000,0.000000,0.000000}%
\pgfsetstrokecolor{currentstroke}%
\pgfsetdash{}{0pt}%
\pgfsys@defobject{currentmarker}{\pgfqpoint{-0.013889in}{0.000000in}}{\pgfqpoint{-0.000000in}{0.000000in}}{%
\pgfpathmoveto{\pgfqpoint{-0.000000in}{0.000000in}}%
\pgfpathlineto{\pgfqpoint{-0.013889in}{0.000000in}}%
\pgfusepath{stroke,fill}%
}%
\begin{pgfscope}%
\pgfsys@transformshift{0.781250in}{2.376667in}%
\pgfsys@useobject{currentmarker}{}%
\end{pgfscope}%
\end{pgfscope}%
\begin{pgfscope}%
\definecolor{textcolor}{rgb}{0.000000,0.000000,0.000000}%
\pgfsetstrokecolor{textcolor}%
\pgfsetfillcolor{textcolor}%
\pgftext[x=0.348877in, y=2.326740in, left, base]{\color{textcolor}{\ifdefined\pdftexversion\else\setmainfont{NanumMyeongjo}\rmfamily\fi\fontsize{9.000000}{10.800000}\selectfont\catcode`\^=\active\def^{\ifmmode\sp\else\^{}\fi}\catcode`\%=\active\def%{\%}16,000}}%
\end{pgfscope}%
\begin{pgfscope}%
\pgfpathrectangle{\pgfqpoint{0.781250in}{0.638889in}}{\pgfqpoint{4.218750in}{2.172222in}}%
\pgfusepath{clip}%
\pgfsetbuttcap%
\pgfsetroundjoin%
\pgfsetlinewidth{0.602250pt}%
\definecolor{currentstroke}{rgb}{0.690196,0.690196,0.690196}%
\pgfsetstrokecolor{currentstroke}%
\pgfsetstrokeopacity{0.400000}%
\pgfsetdash{{2.220000pt}{0.960000pt}}{0.000000pt}%
\pgfpathmoveto{\pgfqpoint{0.781250in}{2.593889in}}%
\pgfpathlineto{\pgfqpoint{5.000000in}{2.593889in}}%
\pgfusepath{stroke}%
\end{pgfscope}%
\begin{pgfscope}%
\pgfsetbuttcap%
\pgfsetroundjoin%
\definecolor{currentfill}{rgb}{0.000000,0.000000,0.000000}%
\pgfsetfillcolor{currentfill}%
\pgfsetlinewidth{0.752812pt}%
\definecolor{currentstroke}{rgb}{0.000000,0.000000,0.000000}%
\pgfsetstrokecolor{currentstroke}%
\pgfsetdash{}{0pt}%
\pgfsys@defobject{currentmarker}{\pgfqpoint{-0.013889in}{0.000000in}}{\pgfqpoint{-0.000000in}{0.000000in}}{%
\pgfpathmoveto{\pgfqpoint{-0.000000in}{0.000000in}}%
\pgfpathlineto{\pgfqpoint{-0.013889in}{0.000000in}}%
\pgfusepath{stroke,fill}%
}%
\begin{pgfscope}%
\pgfsys@transformshift{0.781250in}{2.593889in}%
\pgfsys@useobject{currentmarker}{}%
\end{pgfscope}%
\end{pgfscope}%
\begin{pgfscope}%
\definecolor{textcolor}{rgb}{0.000000,0.000000,0.000000}%
\pgfsetstrokecolor{textcolor}%
\pgfsetfillcolor{textcolor}%
\pgftext[x=0.348877in, y=2.543962in, left, base]{\color{textcolor}{\ifdefined\pdftexversion\else\setmainfont{NanumMyeongjo}\rmfamily\fi\fontsize{9.000000}{10.800000}\selectfont\catcode`\^=\active\def^{\ifmmode\sp\else\^{}\fi}\catcode`\%=\active\def%{\%}18,000}}%
\end{pgfscope}%
\begin{pgfscope}%
\pgfpathrectangle{\pgfqpoint{0.781250in}{0.638889in}}{\pgfqpoint{4.218750in}{2.172222in}}%
\pgfusepath{clip}%
\pgfsetbuttcap%
\pgfsetroundjoin%
\pgfsetlinewidth{0.602250pt}%
\definecolor{currentstroke}{rgb}{0.690196,0.690196,0.690196}%
\pgfsetstrokecolor{currentstroke}%
\pgfsetstrokeopacity{0.400000}%
\pgfsetdash{{2.220000pt}{0.960000pt}}{0.000000pt}%
\pgfpathmoveto{\pgfqpoint{0.781250in}{2.811111in}}%
\pgfpathlineto{\pgfqpoint{5.000000in}{2.811111in}}%
\pgfusepath{stroke}%
\end{pgfscope}%
\begin{pgfscope}%
\pgfsetbuttcap%
\pgfsetroundjoin%
\definecolor{currentfill}{rgb}{0.000000,0.000000,0.000000}%
\pgfsetfillcolor{currentfill}%
\pgfsetlinewidth{0.752812pt}%
\definecolor{currentstroke}{rgb}{0.000000,0.000000,0.000000}%
\pgfsetstrokecolor{currentstroke}%
\pgfsetdash{}{0pt}%
\pgfsys@defobject{currentmarker}{\pgfqpoint{-0.013889in}{0.000000in}}{\pgfqpoint{-0.000000in}{0.000000in}}{%
\pgfpathmoveto{\pgfqpoint{-0.000000in}{0.000000in}}%
\pgfpathlineto{\pgfqpoint{-0.013889in}{0.000000in}}%
\pgfusepath{stroke,fill}%
}%
\begin{pgfscope}%
\pgfsys@transformshift{0.781250in}{2.811111in}%
\pgfsys@useobject{currentmarker}{}%
\end{pgfscope}%
\end{pgfscope}%
\begin{pgfscope}%
\definecolor{textcolor}{rgb}{0.000000,0.000000,0.000000}%
\pgfsetstrokecolor{textcolor}%
\pgfsetfillcolor{textcolor}%
\pgftext[x=0.348877in, y=2.761184in, left, base]{\color{textcolor}{\ifdefined\pdftexversion\else\setmainfont{NanumMyeongjo}\rmfamily\fi\fontsize{9.000000}{10.800000}\selectfont\catcode`\^=\active\def^{\ifmmode\sp\else\^{}\fi}\catcode`\%=\active\def%{\%}20,000}}%
\end{pgfscope}%
\begin{pgfscope}%
\pgfsetrectcap%
\pgfsetmiterjoin%
\pgfsetlinewidth{0.752812pt}%
\definecolor{currentstroke}{rgb}{0.000000,0.000000,0.000000}%
\pgfsetstrokecolor{currentstroke}%
\pgfsetdash{}{0pt}%
\pgfpathmoveto{\pgfqpoint{0.781250in}{0.638889in}}%
\pgfpathlineto{\pgfqpoint{0.781250in}{2.811111in}}%
\pgfusepath{stroke}%
\end{pgfscope}%
\begin{pgfscope}%
\pgfsetrectcap%
\pgfsetmiterjoin%
\pgfsetlinewidth{0.752812pt}%
\definecolor{currentstroke}{rgb}{0.000000,0.000000,0.000000}%
\pgfsetstrokecolor{currentstroke}%
\pgfsetdash{}{0pt}%
\pgfpathmoveto{\pgfqpoint{0.781250in}{0.638889in}}%
\pgfpathlineto{\pgfqpoint{5.000000in}{0.638889in}}%
\pgfusepath{stroke}%
\end{pgfscope}%
\begin{pgfscope}%
\pgfpathrectangle{\pgfqpoint{0.781250in}{0.638889in}}{\pgfqpoint{4.218750in}{2.172222in}}%
\pgfusepath{clip}%
\pgfsetbuttcap%
\pgfsetmiterjoin%
\definecolor{currentfill}{rgb}{0.337255,0.713725,0.627451}%
\pgfsetfillcolor{currentfill}%
\pgfsetlinewidth{1.003750pt}%
\definecolor{currentstroke}{rgb}{0.266667,0.266667,0.266667}%
\pgfsetstrokecolor{currentstroke}%
\pgfsetdash{}{0pt}%
\pgfpathmoveto{\pgfqpoint{0.973011in}{0.638889in}}%
\pgfpathlineto{\pgfqpoint{1.698595in}{0.638889in}}%
\pgfpathlineto{\pgfqpoint{1.698595in}{1.298593in}}%
\pgfpathlineto{\pgfqpoint{0.973011in}{1.298593in}}%
\pgfpathlineto{\pgfqpoint{0.973011in}{0.638889in}}%
\pgfpathclose%
\pgfusepath{stroke,fill}%
\end{pgfscope}%
\begin{pgfscope}%
\pgfpathrectangle{\pgfqpoint{0.781250in}{0.638889in}}{\pgfqpoint{4.218750in}{2.172222in}}%
\pgfusepath{clip}%
\pgfsetbuttcap%
\pgfsetmiterjoin%
\definecolor{currentfill}{rgb}{0.337255,0.713725,0.627451}%
\pgfsetfillcolor{currentfill}%
\pgfsetlinewidth{1.003750pt}%
\definecolor{currentstroke}{rgb}{0.266667,0.266667,0.266667}%
\pgfsetstrokecolor{currentstroke}%
\pgfsetdash{}{0pt}%
\pgfpathmoveto{\pgfqpoint{2.009559in}{0.638889in}}%
\pgfpathlineto{\pgfqpoint{2.735143in}{0.638889in}}%
\pgfpathlineto{\pgfqpoint{2.735143in}{1.181401in}}%
\pgfpathlineto{\pgfqpoint{2.009559in}{1.181401in}}%
\pgfpathlineto{\pgfqpoint{2.009559in}{0.638889in}}%
\pgfpathclose%
\pgfusepath{stroke,fill}%
\end{pgfscope}%
\begin{pgfscope}%
\pgfpathrectangle{\pgfqpoint{0.781250in}{0.638889in}}{\pgfqpoint{4.218750in}{2.172222in}}%
\pgfusepath{clip}%
\pgfsetbuttcap%
\pgfsetmiterjoin%
\definecolor{currentfill}{rgb}{0.337255,0.713725,0.627451}%
\pgfsetfillcolor{currentfill}%
\pgfsetlinewidth{1.003750pt}%
\definecolor{currentstroke}{rgb}{0.266667,0.266667,0.266667}%
\pgfsetstrokecolor{currentstroke}%
\pgfsetdash{}{0pt}%
\pgfpathmoveto{\pgfqpoint{3.046107in}{0.638889in}}%
\pgfpathlineto{\pgfqpoint{3.771691in}{0.638889in}}%
\pgfpathlineto{\pgfqpoint{3.771691in}{1.328352in}}%
\pgfpathlineto{\pgfqpoint{3.046107in}{1.328352in}}%
\pgfpathlineto{\pgfqpoint{3.046107in}{0.638889in}}%
\pgfpathclose%
\pgfusepath{stroke,fill}%
\end{pgfscope}%
\begin{pgfscope}%
\pgfpathrectangle{\pgfqpoint{0.781250in}{0.638889in}}{\pgfqpoint{4.218750in}{2.172222in}}%
\pgfusepath{clip}%
\pgfsetbuttcap%
\pgfsetmiterjoin%
\definecolor{currentfill}{rgb}{0.337255,0.713725,0.627451}%
\pgfsetfillcolor{currentfill}%
\pgfsetlinewidth{1.003750pt}%
\definecolor{currentstroke}{rgb}{0.266667,0.266667,0.266667}%
\pgfsetstrokecolor{currentstroke}%
\pgfsetdash{}{0pt}%
\pgfpathmoveto{\pgfqpoint{4.082655in}{0.638889in}}%
\pgfpathlineto{\pgfqpoint{4.808239in}{0.638889in}}%
\pgfpathlineto{\pgfqpoint{4.808239in}{1.809499in}}%
\pgfpathlineto{\pgfqpoint{4.082655in}{1.809499in}}%
\pgfpathlineto{\pgfqpoint{4.082655in}{0.638889in}}%
\pgfpathclose%
\pgfusepath{stroke,fill}%
\end{pgfscope}%
\begin{pgfscope}%
\pgfpathrectangle{\pgfqpoint{0.781250in}{0.638889in}}{\pgfqpoint{4.218750in}{2.172222in}}%
\pgfusepath{clip}%
\pgfsetbuttcap%
\pgfsetmiterjoin%
\definecolor{currentfill}{rgb}{0.725490,0.486275,0.164706}%
\pgfsetfillcolor{currentfill}%
\pgfsetlinewidth{1.003750pt}%
\definecolor{currentstroke}{rgb}{0.266667,0.266667,0.266667}%
\pgfsetstrokecolor{currentstroke}%
\pgfsetdash{}{0pt}%
\pgfpathmoveto{\pgfqpoint{0.973011in}{1.298593in}}%
\pgfpathlineto{\pgfqpoint{1.698595in}{1.298593in}}%
\pgfpathlineto{\pgfqpoint{1.698595in}{1.386025in}}%
\pgfpathlineto{\pgfqpoint{0.973011in}{1.386025in}}%
\pgfpathlineto{\pgfqpoint{0.973011in}{1.298593in}}%
\pgfpathclose%
\pgfusepath{stroke,fill}%
\end{pgfscope}%
\begin{pgfscope}%
\pgfpathrectangle{\pgfqpoint{0.781250in}{0.638889in}}{\pgfqpoint{4.218750in}{2.172222in}}%
\pgfusepath{clip}%
\pgfsetbuttcap%
\pgfsetmiterjoin%
\definecolor{currentfill}{rgb}{0.725490,0.486275,0.164706}%
\pgfsetfillcolor{currentfill}%
\pgfsetlinewidth{1.003750pt}%
\definecolor{currentstroke}{rgb}{0.266667,0.266667,0.266667}%
\pgfsetstrokecolor{currentstroke}%
\pgfsetdash{}{0pt}%
\pgfpathmoveto{\pgfqpoint{2.009559in}{1.181401in}}%
\pgfpathlineto{\pgfqpoint{2.735143in}{1.181401in}}%
\pgfpathlineto{\pgfqpoint{2.735143in}{1.313907in}}%
\pgfpathlineto{\pgfqpoint{2.009559in}{1.313907in}}%
\pgfpathlineto{\pgfqpoint{2.009559in}{1.181401in}}%
\pgfpathclose%
\pgfusepath{stroke,fill}%
\end{pgfscope}%
\begin{pgfscope}%
\pgfpathrectangle{\pgfqpoint{0.781250in}{0.638889in}}{\pgfqpoint{4.218750in}{2.172222in}}%
\pgfusepath{clip}%
\pgfsetbuttcap%
\pgfsetmiterjoin%
\definecolor{currentfill}{rgb}{0.725490,0.486275,0.164706}%
\pgfsetfillcolor{currentfill}%
\pgfsetlinewidth{1.003750pt}%
\definecolor{currentstroke}{rgb}{0.266667,0.266667,0.266667}%
\pgfsetstrokecolor{currentstroke}%
\pgfsetdash{}{0pt}%
\pgfpathmoveto{\pgfqpoint{3.046107in}{1.328352in}}%
\pgfpathlineto{\pgfqpoint{3.771691in}{1.328352in}}%
\pgfpathlineto{\pgfqpoint{3.771691in}{1.501696in}}%
\pgfpathlineto{\pgfqpoint{3.046107in}{1.501696in}}%
\pgfpathlineto{\pgfqpoint{3.046107in}{1.328352in}}%
\pgfpathclose%
\pgfusepath{stroke,fill}%
\end{pgfscope}%
\begin{pgfscope}%
\pgfpathrectangle{\pgfqpoint{0.781250in}{0.638889in}}{\pgfqpoint{4.218750in}{2.172222in}}%
\pgfusepath{clip}%
\pgfsetbuttcap%
\pgfsetmiterjoin%
\definecolor{currentfill}{rgb}{0.725490,0.486275,0.164706}%
\pgfsetfillcolor{currentfill}%
\pgfsetlinewidth{1.003750pt}%
\definecolor{currentstroke}{rgb}{0.266667,0.266667,0.266667}%
\pgfsetstrokecolor{currentstroke}%
\pgfsetdash{}{0pt}%
\pgfpathmoveto{\pgfqpoint{4.082655in}{1.809499in}}%
\pgfpathlineto{\pgfqpoint{4.808239in}{1.809499in}}%
\pgfpathlineto{\pgfqpoint{4.808239in}{2.029328in}}%
\pgfpathlineto{\pgfqpoint{4.082655in}{2.029328in}}%
\pgfpathlineto{\pgfqpoint{4.082655in}{1.809499in}}%
\pgfpathclose%
\pgfusepath{stroke,fill}%
\end{pgfscope}%
\begin{pgfscope}%
\pgfpathrectangle{\pgfqpoint{0.781250in}{0.638889in}}{\pgfqpoint{4.218750in}{2.172222in}}%
\pgfusepath{clip}%
\pgfsetbuttcap%
\pgfsetmiterjoin%
\definecolor{currentfill}{rgb}{0.235294,0.490196,0.764706}%
\pgfsetfillcolor{currentfill}%
\pgfsetlinewidth{1.003750pt}%
\definecolor{currentstroke}{rgb}{0.266667,0.266667,0.266667}%
\pgfsetstrokecolor{currentstroke}%
\pgfsetdash{}{0pt}%
\pgfpathmoveto{\pgfqpoint{0.973011in}{1.386025in}}%
\pgfpathlineto{\pgfqpoint{1.698595in}{1.386025in}}%
\pgfpathlineto{\pgfqpoint{1.698595in}{1.534496in}}%
\pgfpathlineto{\pgfqpoint{0.973011in}{1.534496in}}%
\pgfpathlineto{\pgfqpoint{0.973011in}{1.386025in}}%
\pgfpathclose%
\pgfusepath{stroke,fill}%
\end{pgfscope}%
\begin{pgfscope}%
\pgfpathrectangle{\pgfqpoint{0.781250in}{0.638889in}}{\pgfqpoint{4.218750in}{2.172222in}}%
\pgfusepath{clip}%
\pgfsetbuttcap%
\pgfsetmiterjoin%
\definecolor{currentfill}{rgb}{0.235294,0.490196,0.764706}%
\pgfsetfillcolor{currentfill}%
\pgfsetlinewidth{1.003750pt}%
\definecolor{currentstroke}{rgb}{0.266667,0.266667,0.266667}%
\pgfsetstrokecolor{currentstroke}%
\pgfsetdash{}{0pt}%
\pgfpathmoveto{\pgfqpoint{2.009559in}{1.313907in}}%
\pgfpathlineto{\pgfqpoint{2.735143in}{1.313907in}}%
\pgfpathlineto{\pgfqpoint{2.735143in}{1.491812in}}%
\pgfpathlineto{\pgfqpoint{2.009559in}{1.491812in}}%
\pgfpathlineto{\pgfqpoint{2.009559in}{1.313907in}}%
\pgfpathclose%
\pgfusepath{stroke,fill}%
\end{pgfscope}%
\begin{pgfscope}%
\pgfpathrectangle{\pgfqpoint{0.781250in}{0.638889in}}{\pgfqpoint{4.218750in}{2.172222in}}%
\pgfusepath{clip}%
\pgfsetbuttcap%
\pgfsetmiterjoin%
\definecolor{currentfill}{rgb}{0.235294,0.490196,0.764706}%
\pgfsetfillcolor{currentfill}%
\pgfsetlinewidth{1.003750pt}%
\definecolor{currentstroke}{rgb}{0.266667,0.266667,0.266667}%
\pgfsetstrokecolor{currentstroke}%
\pgfsetdash{}{0pt}%
\pgfpathmoveto{\pgfqpoint{3.046107in}{1.501696in}}%
\pgfpathlineto{\pgfqpoint{3.771691in}{1.501696in}}%
\pgfpathlineto{\pgfqpoint{3.771691in}{1.689484in}}%
\pgfpathlineto{\pgfqpoint{3.046107in}{1.689484in}}%
\pgfpathlineto{\pgfqpoint{3.046107in}{1.501696in}}%
\pgfpathclose%
\pgfusepath{stroke,fill}%
\end{pgfscope}%
\begin{pgfscope}%
\pgfpathrectangle{\pgfqpoint{0.781250in}{0.638889in}}{\pgfqpoint{4.218750in}{2.172222in}}%
\pgfusepath{clip}%
\pgfsetbuttcap%
\pgfsetmiterjoin%
\definecolor{currentfill}{rgb}{0.235294,0.490196,0.764706}%
\pgfsetfillcolor{currentfill}%
\pgfsetlinewidth{1.003750pt}%
\definecolor{currentstroke}{rgb}{0.266667,0.266667,0.266667}%
\pgfsetstrokecolor{currentstroke}%
\pgfsetdash{}{0pt}%
\pgfpathmoveto{\pgfqpoint{4.082655in}{2.029328in}}%
\pgfpathlineto{\pgfqpoint{4.808239in}{2.029328in}}%
\pgfpathlineto{\pgfqpoint{4.808239in}{2.243618in}}%
\pgfpathlineto{\pgfqpoint{4.082655in}{2.243618in}}%
\pgfpathlineto{\pgfqpoint{4.082655in}{2.029328in}}%
\pgfpathclose%
\pgfusepath{stroke,fill}%
\end{pgfscope}%
\begin{pgfscope}%
\pgfpathrectangle{\pgfqpoint{0.781250in}{0.638889in}}{\pgfqpoint{4.218750in}{2.172222in}}%
\pgfusepath{clip}%
\pgfsetbuttcap%
\pgfsetmiterjoin%
\definecolor{currentfill}{rgb}{0.549020,0.247059,0.121569}%
\pgfsetfillcolor{currentfill}%
\pgfsetlinewidth{1.003750pt}%
\definecolor{currentstroke}{rgb}{0.266667,0.266667,0.266667}%
\pgfsetstrokecolor{currentstroke}%
\pgfsetdash{}{0pt}%
\pgfpathmoveto{\pgfqpoint{0.973011in}{1.534496in}}%
\pgfpathlineto{\pgfqpoint{1.698595in}{1.534496in}}%
\pgfpathlineto{\pgfqpoint{1.698595in}{1.594993in}}%
\pgfpathlineto{\pgfqpoint{0.973011in}{1.594993in}}%
\pgfpathlineto{\pgfqpoint{0.973011in}{1.534496in}}%
\pgfpathclose%
\pgfusepath{stroke,fill}%
\end{pgfscope}%
\begin{pgfscope}%
\pgfpathrectangle{\pgfqpoint{0.781250in}{0.638889in}}{\pgfqpoint{4.218750in}{2.172222in}}%
\pgfusepath{clip}%
\pgfsetbuttcap%
\pgfsetmiterjoin%
\definecolor{currentfill}{rgb}{0.549020,0.247059,0.121569}%
\pgfsetfillcolor{currentfill}%
\pgfsetlinewidth{1.003750pt}%
\definecolor{currentstroke}{rgb}{0.266667,0.266667,0.266667}%
\pgfsetstrokecolor{currentstroke}%
\pgfsetdash{}{0pt}%
\pgfpathmoveto{\pgfqpoint{2.009559in}{1.491812in}}%
\pgfpathlineto{\pgfqpoint{2.735143in}{1.491812in}}%
\pgfpathlineto{\pgfqpoint{2.735143in}{1.634961in}}%
\pgfpathlineto{\pgfqpoint{2.009559in}{1.634961in}}%
\pgfpathlineto{\pgfqpoint{2.009559in}{1.491812in}}%
\pgfpathclose%
\pgfusepath{stroke,fill}%
\end{pgfscope}%
\begin{pgfscope}%
\pgfpathrectangle{\pgfqpoint{0.781250in}{0.638889in}}{\pgfqpoint{4.218750in}{2.172222in}}%
\pgfusepath{clip}%
\pgfsetbuttcap%
\pgfsetmiterjoin%
\definecolor{currentfill}{rgb}{0.549020,0.247059,0.121569}%
\pgfsetfillcolor{currentfill}%
\pgfsetlinewidth{1.003750pt}%
\definecolor{currentstroke}{rgb}{0.266667,0.266667,0.266667}%
\pgfsetstrokecolor{currentstroke}%
\pgfsetdash{}{0pt}%
\pgfpathmoveto{\pgfqpoint{3.046107in}{1.689484in}}%
\pgfpathlineto{\pgfqpoint{3.771691in}{1.689484in}}%
\pgfpathlineto{\pgfqpoint{3.771691in}{1.868258in}}%
\pgfpathlineto{\pgfqpoint{3.046107in}{1.868258in}}%
\pgfpathlineto{\pgfqpoint{3.046107in}{1.689484in}}%
\pgfpathclose%
\pgfusepath{stroke,fill}%
\end{pgfscope}%
\begin{pgfscope}%
\pgfpathrectangle{\pgfqpoint{0.781250in}{0.638889in}}{\pgfqpoint{4.218750in}{2.172222in}}%
\pgfusepath{clip}%
\pgfsetbuttcap%
\pgfsetmiterjoin%
\definecolor{currentfill}{rgb}{0.549020,0.247059,0.121569}%
\pgfsetfillcolor{currentfill}%
\pgfsetlinewidth{1.003750pt}%
\definecolor{currentstroke}{rgb}{0.266667,0.266667,0.266667}%
\pgfsetstrokecolor{currentstroke}%
\pgfsetdash{}{0pt}%
\pgfpathmoveto{\pgfqpoint{4.082655in}{2.243618in}}%
\pgfpathlineto{\pgfqpoint{4.808239in}{2.243618in}}%
\pgfpathlineto{\pgfqpoint{4.808239in}{2.425433in}}%
\pgfpathlineto{\pgfqpoint{4.082655in}{2.425433in}}%
\pgfpathlineto{\pgfqpoint{4.082655in}{2.243618in}}%
\pgfpathclose%
\pgfusepath{stroke,fill}%
\end{pgfscope}%
\begin{pgfscope}%
\pgfpathrectangle{\pgfqpoint{0.781250in}{0.638889in}}{\pgfqpoint{4.218750in}{2.172222in}}%
\pgfusepath{clip}%
\pgfsetbuttcap%
\pgfsetmiterjoin%
\definecolor{currentfill}{rgb}{0.447059,0.447059,0.447059}%
\pgfsetfillcolor{currentfill}%
\pgfsetlinewidth{1.003750pt}%
\definecolor{currentstroke}{rgb}{0.266667,0.266667,0.266667}%
\pgfsetstrokecolor{currentstroke}%
\pgfsetdash{}{0pt}%
\pgfpathmoveto{\pgfqpoint{0.973011in}{1.594993in}}%
\pgfpathlineto{\pgfqpoint{1.698595in}{1.594993in}}%
\pgfpathlineto{\pgfqpoint{1.698595in}{1.685140in}}%
\pgfpathlineto{\pgfqpoint{0.973011in}{1.685140in}}%
\pgfpathlineto{\pgfqpoint{0.973011in}{1.594993in}}%
\pgfpathclose%
\pgfusepath{stroke,fill}%
\end{pgfscope}%
\begin{pgfscope}%
\pgfpathrectangle{\pgfqpoint{0.781250in}{0.638889in}}{\pgfqpoint{4.218750in}{2.172222in}}%
\pgfusepath{clip}%
\pgfsetbuttcap%
\pgfsetmiterjoin%
\definecolor{currentfill}{rgb}{0.447059,0.447059,0.447059}%
\pgfsetfillcolor{currentfill}%
\pgfsetlinewidth{1.003750pt}%
\definecolor{currentstroke}{rgb}{0.266667,0.266667,0.266667}%
\pgfsetstrokecolor{currentstroke}%
\pgfsetdash{}{0pt}%
\pgfpathmoveto{\pgfqpoint{2.009559in}{1.634961in}}%
\pgfpathlineto{\pgfqpoint{2.735143in}{1.634961in}}%
\pgfpathlineto{\pgfqpoint{2.735143in}{1.704690in}}%
\pgfpathlineto{\pgfqpoint{2.009559in}{1.704690in}}%
\pgfpathlineto{\pgfqpoint{2.009559in}{1.634961in}}%
\pgfpathclose%
\pgfusepath{stroke,fill}%
\end{pgfscope}%
\begin{pgfscope}%
\pgfpathrectangle{\pgfqpoint{0.781250in}{0.638889in}}{\pgfqpoint{4.218750in}{2.172222in}}%
\pgfusepath{clip}%
\pgfsetbuttcap%
\pgfsetmiterjoin%
\definecolor{currentfill}{rgb}{0.447059,0.447059,0.447059}%
\pgfsetfillcolor{currentfill}%
\pgfsetlinewidth{1.003750pt}%
\definecolor{currentstroke}{rgb}{0.266667,0.266667,0.266667}%
\pgfsetstrokecolor{currentstroke}%
\pgfsetdash{}{0pt}%
\pgfpathmoveto{\pgfqpoint{3.046107in}{1.868258in}}%
\pgfpathlineto{\pgfqpoint{3.771691in}{1.868258in}}%
\pgfpathlineto{\pgfqpoint{3.771691in}{1.934728in}}%
\pgfpathlineto{\pgfqpoint{3.046107in}{1.934728in}}%
\pgfpathlineto{\pgfqpoint{3.046107in}{1.868258in}}%
\pgfpathclose%
\pgfusepath{stroke,fill}%
\end{pgfscope}%
\begin{pgfscope}%
\pgfpathrectangle{\pgfqpoint{0.781250in}{0.638889in}}{\pgfqpoint{4.218750in}{2.172222in}}%
\pgfusepath{clip}%
\pgfsetbuttcap%
\pgfsetmiterjoin%
\definecolor{currentfill}{rgb}{0.447059,0.447059,0.447059}%
\pgfsetfillcolor{currentfill}%
\pgfsetlinewidth{1.003750pt}%
\definecolor{currentstroke}{rgb}{0.266667,0.266667,0.266667}%
\pgfsetstrokecolor{currentstroke}%
\pgfsetdash{}{0pt}%
\pgfpathmoveto{\pgfqpoint{4.082655in}{2.425433in}}%
\pgfpathlineto{\pgfqpoint{4.808239in}{2.425433in}}%
\pgfpathlineto{\pgfqpoint{4.808239in}{2.518187in}}%
\pgfpathlineto{\pgfqpoint{4.082655in}{2.518187in}}%
\pgfpathlineto{\pgfqpoint{4.082655in}{2.425433in}}%
\pgfpathclose%
\pgfusepath{stroke,fill}%
\end{pgfscope}%
\begin{pgfscope}%
\pgfpathrectangle{\pgfqpoint{0.781250in}{0.638889in}}{\pgfqpoint{4.218750in}{2.172222in}}%
\pgfusepath{clip}%
\pgfsetbuttcap%
\pgfsetmiterjoin%
\definecolor{currentfill}{rgb}{0.447059,0.447059,0.447059}%
\pgfsetfillcolor{currentfill}%
\pgfsetlinewidth{1.003750pt}%
\definecolor{currentstroke}{rgb}{0.266667,0.266667,0.266667}%
\pgfsetstrokecolor{currentstroke}%
\pgfsetdash{}{0pt}%
\pgfpathmoveto{\pgfqpoint{0.973011in}{1.685140in}}%
\pgfpathlineto{\pgfqpoint{1.698595in}{1.685140in}}%
\pgfpathlineto{\pgfqpoint{1.698595in}{1.737056in}}%
\pgfpathlineto{\pgfqpoint{0.973011in}{1.737056in}}%
\pgfpathlineto{\pgfqpoint{0.973011in}{1.685140in}}%
\pgfpathclose%
\pgfusepath{stroke,fill}%
\end{pgfscope}%
\begin{pgfscope}%
\pgfpathrectangle{\pgfqpoint{0.781250in}{0.638889in}}{\pgfqpoint{4.218750in}{2.172222in}}%
\pgfusepath{clip}%
\pgfsetbuttcap%
\pgfsetmiterjoin%
\definecolor{currentfill}{rgb}{0.447059,0.447059,0.447059}%
\pgfsetfillcolor{currentfill}%
\pgfsetlinewidth{1.003750pt}%
\definecolor{currentstroke}{rgb}{0.266667,0.266667,0.266667}%
\pgfsetstrokecolor{currentstroke}%
\pgfsetdash{}{0pt}%
\pgfpathmoveto{\pgfqpoint{2.009559in}{1.704690in}}%
\pgfpathlineto{\pgfqpoint{2.735143in}{1.704690in}}%
\pgfpathlineto{\pgfqpoint{2.735143in}{1.750306in}}%
\pgfpathlineto{\pgfqpoint{2.009559in}{1.750306in}}%
\pgfpathlineto{\pgfqpoint{2.009559in}{1.704690in}}%
\pgfpathclose%
\pgfusepath{stroke,fill}%
\end{pgfscope}%
\begin{pgfscope}%
\pgfpathrectangle{\pgfqpoint{0.781250in}{0.638889in}}{\pgfqpoint{4.218750in}{2.172222in}}%
\pgfusepath{clip}%
\pgfsetbuttcap%
\pgfsetmiterjoin%
\definecolor{currentfill}{rgb}{0.447059,0.447059,0.447059}%
\pgfsetfillcolor{currentfill}%
\pgfsetlinewidth{1.003750pt}%
\definecolor{currentstroke}{rgb}{0.266667,0.266667,0.266667}%
\pgfsetstrokecolor{currentstroke}%
\pgfsetdash{}{0pt}%
\pgfpathmoveto{\pgfqpoint{3.046107in}{1.934728in}}%
\pgfpathlineto{\pgfqpoint{3.771691in}{1.934728in}}%
\pgfpathlineto{\pgfqpoint{3.771691in}{1.984798in}}%
\pgfpathlineto{\pgfqpoint{3.046107in}{1.984798in}}%
\pgfpathlineto{\pgfqpoint{3.046107in}{1.934728in}}%
\pgfpathclose%
\pgfusepath{stroke,fill}%
\end{pgfscope}%
\begin{pgfscope}%
\pgfpathrectangle{\pgfqpoint{0.781250in}{0.638889in}}{\pgfqpoint{4.218750in}{2.172222in}}%
\pgfusepath{clip}%
\pgfsetbuttcap%
\pgfsetmiterjoin%
\definecolor{currentfill}{rgb}{0.447059,0.447059,0.447059}%
\pgfsetfillcolor{currentfill}%
\pgfsetlinewidth{1.003750pt}%
\definecolor{currentstroke}{rgb}{0.266667,0.266667,0.266667}%
\pgfsetstrokecolor{currentstroke}%
\pgfsetdash{}{0pt}%
\pgfpathmoveto{\pgfqpoint{4.082655in}{2.518187in}}%
\pgfpathlineto{\pgfqpoint{4.808239in}{2.518187in}}%
\pgfpathlineto{\pgfqpoint{4.808239in}{2.586612in}}%
\pgfpathlineto{\pgfqpoint{4.082655in}{2.586612in}}%
\pgfpathlineto{\pgfqpoint{4.082655in}{2.518187in}}%
\pgfpathclose%
\pgfusepath{stroke,fill}%
\end{pgfscope}%
\begin{pgfscope}%
\pgfpathrectangle{\pgfqpoint{0.781250in}{0.638889in}}{\pgfqpoint{4.218750in}{2.172222in}}%
\pgfusepath{clip}%
\pgfsetbuttcap%
\pgfsetmiterjoin%
\definecolor{currentfill}{rgb}{0.447059,0.447059,0.447059}%
\pgfsetfillcolor{currentfill}%
\pgfsetlinewidth{1.003750pt}%
\definecolor{currentstroke}{rgb}{0.266667,0.266667,0.266667}%
\pgfsetstrokecolor{currentstroke}%
\pgfsetdash{}{0pt}%
\pgfpathmoveto{\pgfqpoint{0.973011in}{1.737056in}}%
\pgfpathlineto{\pgfqpoint{1.698595in}{1.737056in}}%
\pgfpathlineto{\pgfqpoint{1.698595in}{1.773223in}}%
\pgfpathlineto{\pgfqpoint{0.973011in}{1.773223in}}%
\pgfpathlineto{\pgfqpoint{0.973011in}{1.737056in}}%
\pgfpathclose%
\pgfusepath{stroke,fill}%
\end{pgfscope}%
\begin{pgfscope}%
\pgfpathrectangle{\pgfqpoint{0.781250in}{0.638889in}}{\pgfqpoint{4.218750in}{2.172222in}}%
\pgfusepath{clip}%
\pgfsetbuttcap%
\pgfsetmiterjoin%
\definecolor{currentfill}{rgb}{0.447059,0.447059,0.447059}%
\pgfsetfillcolor{currentfill}%
\pgfsetlinewidth{1.003750pt}%
\definecolor{currentstroke}{rgb}{0.266667,0.266667,0.266667}%
\pgfsetstrokecolor{currentstroke}%
\pgfsetdash{}{0pt}%
\pgfpathmoveto{\pgfqpoint{2.009559in}{1.750306in}}%
\pgfpathlineto{\pgfqpoint{2.735143in}{1.750306in}}%
\pgfpathlineto{\pgfqpoint{2.735143in}{1.835349in}}%
\pgfpathlineto{\pgfqpoint{2.009559in}{1.835349in}}%
\pgfpathlineto{\pgfqpoint{2.009559in}{1.750306in}}%
\pgfpathclose%
\pgfusepath{stroke,fill}%
\end{pgfscope}%
\begin{pgfscope}%
\pgfpathrectangle{\pgfqpoint{0.781250in}{0.638889in}}{\pgfqpoint{4.218750in}{2.172222in}}%
\pgfusepath{clip}%
\pgfsetbuttcap%
\pgfsetmiterjoin%
\definecolor{currentfill}{rgb}{0.447059,0.447059,0.447059}%
\pgfsetfillcolor{currentfill}%
\pgfsetlinewidth{1.003750pt}%
\definecolor{currentstroke}{rgb}{0.266667,0.266667,0.266667}%
\pgfsetstrokecolor{currentstroke}%
\pgfsetdash{}{0pt}%
\pgfpathmoveto{\pgfqpoint{3.046107in}{1.984798in}}%
\pgfpathlineto{\pgfqpoint{3.771691in}{1.984798in}}%
\pgfpathlineto{\pgfqpoint{3.771691in}{2.042579in}}%
\pgfpathlineto{\pgfqpoint{3.046107in}{2.042579in}}%
\pgfpathlineto{\pgfqpoint{3.046107in}{1.984798in}}%
\pgfpathclose%
\pgfusepath{stroke,fill}%
\end{pgfscope}%
\begin{pgfscope}%
\pgfpathrectangle{\pgfqpoint{0.781250in}{0.638889in}}{\pgfqpoint{4.218750in}{2.172222in}}%
\pgfusepath{clip}%
\pgfsetbuttcap%
\pgfsetmiterjoin%
\definecolor{currentfill}{rgb}{0.447059,0.447059,0.447059}%
\pgfsetfillcolor{currentfill}%
\pgfsetlinewidth{1.003750pt}%
\definecolor{currentstroke}{rgb}{0.266667,0.266667,0.266667}%
\pgfsetstrokecolor{currentstroke}%
\pgfsetdash{}{0pt}%
\pgfpathmoveto{\pgfqpoint{4.082655in}{2.586612in}}%
\pgfpathlineto{\pgfqpoint{4.808239in}{2.586612in}}%
\pgfpathlineto{\pgfqpoint{4.808239in}{2.629513in}}%
\pgfpathlineto{\pgfqpoint{4.082655in}{2.629513in}}%
\pgfpathlineto{\pgfqpoint{4.082655in}{2.586612in}}%
\pgfpathclose%
\pgfusepath{stroke,fill}%
\end{pgfscope}%
\begin{pgfscope}%
\pgfpathrectangle{\pgfqpoint{0.781250in}{0.638889in}}{\pgfqpoint{4.218750in}{2.172222in}}%
\pgfusepath{clip}%
\pgfsetbuttcap%
\pgfsetmiterjoin%
\definecolor{currentfill}{rgb}{0.447059,0.447059,0.447059}%
\pgfsetfillcolor{currentfill}%
\pgfsetlinewidth{1.003750pt}%
\definecolor{currentstroke}{rgb}{0.266667,0.266667,0.266667}%
\pgfsetstrokecolor{currentstroke}%
\pgfsetdash{}{0pt}%
\pgfpathmoveto{\pgfqpoint{0.973011in}{1.773223in}}%
\pgfpathlineto{\pgfqpoint{1.698595in}{1.773223in}}%
\pgfpathlineto{\pgfqpoint{1.698595in}{1.787343in}}%
\pgfpathlineto{\pgfqpoint{0.973011in}{1.787343in}}%
\pgfpathlineto{\pgfqpoint{0.973011in}{1.773223in}}%
\pgfpathclose%
\pgfusepath{stroke,fill}%
\end{pgfscope}%
\begin{pgfscope}%
\pgfpathrectangle{\pgfqpoint{0.781250in}{0.638889in}}{\pgfqpoint{4.218750in}{2.172222in}}%
\pgfusepath{clip}%
\pgfsetbuttcap%
\pgfsetmiterjoin%
\definecolor{currentfill}{rgb}{0.447059,0.447059,0.447059}%
\pgfsetfillcolor{currentfill}%
\pgfsetlinewidth{1.003750pt}%
\definecolor{currentstroke}{rgb}{0.266667,0.266667,0.266667}%
\pgfsetstrokecolor{currentstroke}%
\pgfsetdash{}{0pt}%
\pgfpathmoveto{\pgfqpoint{2.009559in}{1.835349in}}%
\pgfpathlineto{\pgfqpoint{2.735143in}{1.835349in}}%
\pgfpathlineto{\pgfqpoint{2.735143in}{1.868910in}}%
\pgfpathlineto{\pgfqpoint{2.009559in}{1.868910in}}%
\pgfpathlineto{\pgfqpoint{2.009559in}{1.835349in}}%
\pgfpathclose%
\pgfusepath{stroke,fill}%
\end{pgfscope}%
\begin{pgfscope}%
\pgfpathrectangle{\pgfqpoint{0.781250in}{0.638889in}}{\pgfqpoint{4.218750in}{2.172222in}}%
\pgfusepath{clip}%
\pgfsetbuttcap%
\pgfsetmiterjoin%
\definecolor{currentfill}{rgb}{0.447059,0.447059,0.447059}%
\pgfsetfillcolor{currentfill}%
\pgfsetlinewidth{1.003750pt}%
\definecolor{currentstroke}{rgb}{0.266667,0.266667,0.266667}%
\pgfsetstrokecolor{currentstroke}%
\pgfsetdash{}{0pt}%
\pgfpathmoveto{\pgfqpoint{3.046107in}{2.042579in}}%
\pgfpathlineto{\pgfqpoint{3.771691in}{2.042579in}}%
\pgfpathlineto{\pgfqpoint{3.771691in}{2.065061in}}%
\pgfpathlineto{\pgfqpoint{3.046107in}{2.065061in}}%
\pgfpathlineto{\pgfqpoint{3.046107in}{2.042579in}}%
\pgfpathclose%
\pgfusepath{stroke,fill}%
\end{pgfscope}%
\begin{pgfscope}%
\pgfpathrectangle{\pgfqpoint{0.781250in}{0.638889in}}{\pgfqpoint{4.218750in}{2.172222in}}%
\pgfusepath{clip}%
\pgfsetbuttcap%
\pgfsetmiterjoin%
\definecolor{currentfill}{rgb}{0.447059,0.447059,0.447059}%
\pgfsetfillcolor{currentfill}%
\pgfsetlinewidth{1.003750pt}%
\definecolor{currentstroke}{rgb}{0.266667,0.266667,0.266667}%
\pgfsetstrokecolor{currentstroke}%
\pgfsetdash{}{0pt}%
\pgfpathmoveto{\pgfqpoint{4.082655in}{2.629513in}}%
\pgfpathlineto{\pgfqpoint{4.808239in}{2.629513in}}%
\pgfpathlineto{\pgfqpoint{4.808239in}{2.646457in}}%
\pgfpathlineto{\pgfqpoint{4.082655in}{2.646457in}}%
\pgfpathlineto{\pgfqpoint{4.082655in}{2.629513in}}%
\pgfpathclose%
\pgfusepath{stroke,fill}%
\end{pgfscope}%
\begin{pgfscope}%
\pgfpathrectangle{\pgfqpoint{0.781250in}{0.638889in}}{\pgfqpoint{4.218750in}{2.172222in}}%
\pgfusepath{clip}%
\pgfsetbuttcap%
\pgfsetmiterjoin%
\definecolor{currentfill}{rgb}{0.447059,0.447059,0.447059}%
\pgfsetfillcolor{currentfill}%
\pgfsetlinewidth{1.003750pt}%
\definecolor{currentstroke}{rgb}{0.266667,0.266667,0.266667}%
\pgfsetstrokecolor{currentstroke}%
\pgfsetdash{}{0pt}%
\pgfpathmoveto{\pgfqpoint{0.973011in}{1.787343in}}%
\pgfpathlineto{\pgfqpoint{1.698595in}{1.787343in}}%
\pgfpathlineto{\pgfqpoint{1.698595in}{1.790275in}}%
\pgfpathlineto{\pgfqpoint{0.973011in}{1.790275in}}%
\pgfpathlineto{\pgfqpoint{0.973011in}{1.787343in}}%
\pgfpathclose%
\pgfusepath{stroke,fill}%
\end{pgfscope}%
\begin{pgfscope}%
\pgfpathrectangle{\pgfqpoint{0.781250in}{0.638889in}}{\pgfqpoint{4.218750in}{2.172222in}}%
\pgfusepath{clip}%
\pgfsetbuttcap%
\pgfsetmiterjoin%
\definecolor{currentfill}{rgb}{0.447059,0.447059,0.447059}%
\pgfsetfillcolor{currentfill}%
\pgfsetlinewidth{1.003750pt}%
\definecolor{currentstroke}{rgb}{0.266667,0.266667,0.266667}%
\pgfsetstrokecolor{currentstroke}%
\pgfsetdash{}{0pt}%
\pgfpathmoveto{\pgfqpoint{2.009559in}{1.868910in}}%
\pgfpathlineto{\pgfqpoint{2.735143in}{1.868910in}}%
\pgfpathlineto{\pgfqpoint{2.735143in}{1.870756in}}%
\pgfpathlineto{\pgfqpoint{2.009559in}{1.870756in}}%
\pgfpathlineto{\pgfqpoint{2.009559in}{1.868910in}}%
\pgfpathclose%
\pgfusepath{stroke,fill}%
\end{pgfscope}%
\begin{pgfscope}%
\pgfpathrectangle{\pgfqpoint{0.781250in}{0.638889in}}{\pgfqpoint{4.218750in}{2.172222in}}%
\pgfusepath{clip}%
\pgfsetbuttcap%
\pgfsetmiterjoin%
\definecolor{currentfill}{rgb}{0.447059,0.447059,0.447059}%
\pgfsetfillcolor{currentfill}%
\pgfsetlinewidth{1.003750pt}%
\definecolor{currentstroke}{rgb}{0.266667,0.266667,0.266667}%
\pgfsetstrokecolor{currentstroke}%
\pgfsetdash{}{0pt}%
\pgfpathmoveto{\pgfqpoint{3.046107in}{2.065061in}}%
\pgfpathlineto{\pgfqpoint{3.771691in}{2.065061in}}%
\pgfpathlineto{\pgfqpoint{3.771691in}{2.065604in}}%
\pgfpathlineto{\pgfqpoint{3.046107in}{2.065604in}}%
\pgfpathlineto{\pgfqpoint{3.046107in}{2.065061in}}%
\pgfpathclose%
\pgfusepath{stroke,fill}%
\end{pgfscope}%
\begin{pgfscope}%
\pgfpathrectangle{\pgfqpoint{0.781250in}{0.638889in}}{\pgfqpoint{4.218750in}{2.172222in}}%
\pgfusepath{clip}%
\pgfsetbuttcap%
\pgfsetmiterjoin%
\definecolor{currentfill}{rgb}{0.447059,0.447059,0.447059}%
\pgfsetfillcolor{currentfill}%
\pgfsetlinewidth{1.003750pt}%
\definecolor{currentstroke}{rgb}{0.266667,0.266667,0.266667}%
\pgfsetstrokecolor{currentstroke}%
\pgfsetdash{}{0pt}%
\pgfpathmoveto{\pgfqpoint{4.082655in}{2.646457in}}%
\pgfpathlineto{\pgfqpoint{4.808239in}{2.646457in}}%
\pgfpathlineto{\pgfqpoint{4.808239in}{2.658838in}}%
\pgfpathlineto{\pgfqpoint{4.082655in}{2.658838in}}%
\pgfpathlineto{\pgfqpoint{4.082655in}{2.646457in}}%
\pgfpathclose%
\pgfusepath{stroke,fill}%
\end{pgfscope}%
\begin{pgfscope}%
\pgfpathrectangle{\pgfqpoint{0.781250in}{0.638889in}}{\pgfqpoint{4.218750in}{2.172222in}}%
\pgfusepath{clip}%
\pgfsetbuttcap%
\pgfsetmiterjoin%
\definecolor{currentfill}{rgb}{0.447059,0.447059,0.447059}%
\pgfsetfillcolor{currentfill}%
\pgfsetlinewidth{1.003750pt}%
\definecolor{currentstroke}{rgb}{0.266667,0.266667,0.266667}%
\pgfsetstrokecolor{currentstroke}%
\pgfsetdash{}{0pt}%
\pgfpathmoveto{\pgfqpoint{0.973011in}{1.790275in}}%
\pgfpathlineto{\pgfqpoint{1.698595in}{1.790275in}}%
\pgfpathlineto{\pgfqpoint{1.698595in}{1.792773in}}%
\pgfpathlineto{\pgfqpoint{0.973011in}{1.792773in}}%
\pgfpathlineto{\pgfqpoint{0.973011in}{1.790275in}}%
\pgfpathclose%
\pgfusepath{stroke,fill}%
\end{pgfscope}%
\begin{pgfscope}%
\pgfpathrectangle{\pgfqpoint{0.781250in}{0.638889in}}{\pgfqpoint{4.218750in}{2.172222in}}%
\pgfusepath{clip}%
\pgfsetbuttcap%
\pgfsetmiterjoin%
\definecolor{currentfill}{rgb}{0.447059,0.447059,0.447059}%
\pgfsetfillcolor{currentfill}%
\pgfsetlinewidth{1.003750pt}%
\definecolor{currentstroke}{rgb}{0.266667,0.266667,0.266667}%
\pgfsetstrokecolor{currentstroke}%
\pgfsetdash{}{0pt}%
\pgfpathmoveto{\pgfqpoint{2.009559in}{1.870756in}}%
\pgfpathlineto{\pgfqpoint{2.735143in}{1.870756in}}%
\pgfpathlineto{\pgfqpoint{2.735143in}{1.871516in}}%
\pgfpathlineto{\pgfqpoint{2.009559in}{1.871516in}}%
\pgfpathlineto{\pgfqpoint{2.009559in}{1.870756in}}%
\pgfpathclose%
\pgfusepath{stroke,fill}%
\end{pgfscope}%
\begin{pgfscope}%
\pgfpathrectangle{\pgfqpoint{0.781250in}{0.638889in}}{\pgfqpoint{4.218750in}{2.172222in}}%
\pgfusepath{clip}%
\pgfsetbuttcap%
\pgfsetmiterjoin%
\definecolor{currentfill}{rgb}{0.447059,0.447059,0.447059}%
\pgfsetfillcolor{currentfill}%
\pgfsetlinewidth{1.003750pt}%
\definecolor{currentstroke}{rgb}{0.266667,0.266667,0.266667}%
\pgfsetstrokecolor{currentstroke}%
\pgfsetdash{}{0pt}%
\pgfpathmoveto{\pgfqpoint{3.046107in}{2.065604in}}%
\pgfpathlineto{\pgfqpoint{3.771691in}{2.065604in}}%
\pgfpathlineto{\pgfqpoint{3.771691in}{2.066039in}}%
\pgfpathlineto{\pgfqpoint{3.046107in}{2.066039in}}%
\pgfpathlineto{\pgfqpoint{3.046107in}{2.065604in}}%
\pgfpathclose%
\pgfusepath{stroke,fill}%
\end{pgfscope}%
\begin{pgfscope}%
\pgfpathrectangle{\pgfqpoint{0.781250in}{0.638889in}}{\pgfqpoint{4.218750in}{2.172222in}}%
\pgfusepath{clip}%
\pgfsetbuttcap%
\pgfsetmiterjoin%
\definecolor{currentfill}{rgb}{0.447059,0.447059,0.447059}%
\pgfsetfillcolor{currentfill}%
\pgfsetlinewidth{1.003750pt}%
\definecolor{currentstroke}{rgb}{0.266667,0.266667,0.266667}%
\pgfsetstrokecolor{currentstroke}%
\pgfsetdash{}{0pt}%
\pgfpathmoveto{\pgfqpoint{4.082655in}{2.658838in}}%
\pgfpathlineto{\pgfqpoint{4.808239in}{2.658838in}}%
\pgfpathlineto{\pgfqpoint{4.808239in}{2.661011in}}%
\pgfpathlineto{\pgfqpoint{4.082655in}{2.661011in}}%
\pgfpathlineto{\pgfqpoint{4.082655in}{2.658838in}}%
\pgfpathclose%
\pgfusepath{stroke,fill}%
\end{pgfscope}%
\begin{pgfscope}%
\pgfpathrectangle{\pgfqpoint{0.781250in}{0.638889in}}{\pgfqpoint{4.218750in}{2.172222in}}%
\pgfusepath{clip}%
\pgfsetbuttcap%
\pgfsetmiterjoin%
\definecolor{currentfill}{rgb}{0.447059,0.447059,0.447059}%
\pgfsetfillcolor{currentfill}%
\pgfsetlinewidth{1.003750pt}%
\definecolor{currentstroke}{rgb}{0.266667,0.266667,0.266667}%
\pgfsetstrokecolor{currentstroke}%
\pgfsetdash{}{0pt}%
\pgfpathmoveto{\pgfqpoint{0.973011in}{1.792773in}}%
\pgfpathlineto{\pgfqpoint{1.698595in}{1.792773in}}%
\pgfpathlineto{\pgfqpoint{1.698595in}{1.794294in}}%
\pgfpathlineto{\pgfqpoint{0.973011in}{1.794294in}}%
\pgfpathlineto{\pgfqpoint{0.973011in}{1.792773in}}%
\pgfpathclose%
\pgfusepath{stroke,fill}%
\end{pgfscope}%
\begin{pgfscope}%
\pgfpathrectangle{\pgfqpoint{0.781250in}{0.638889in}}{\pgfqpoint{4.218750in}{2.172222in}}%
\pgfusepath{clip}%
\pgfsetbuttcap%
\pgfsetmiterjoin%
\definecolor{currentfill}{rgb}{0.447059,0.447059,0.447059}%
\pgfsetfillcolor{currentfill}%
\pgfsetlinewidth{1.003750pt}%
\definecolor{currentstroke}{rgb}{0.266667,0.266667,0.266667}%
\pgfsetstrokecolor{currentstroke}%
\pgfsetdash{}{0pt}%
\pgfpathmoveto{\pgfqpoint{2.009559in}{1.871516in}}%
\pgfpathlineto{\pgfqpoint{2.735143in}{1.871516in}}%
\pgfpathlineto{\pgfqpoint{2.735143in}{1.872385in}}%
\pgfpathlineto{\pgfqpoint{2.009559in}{1.872385in}}%
\pgfpathlineto{\pgfqpoint{2.009559in}{1.871516in}}%
\pgfpathclose%
\pgfusepath{stroke,fill}%
\end{pgfscope}%
\begin{pgfscope}%
\pgfpathrectangle{\pgfqpoint{0.781250in}{0.638889in}}{\pgfqpoint{4.218750in}{2.172222in}}%
\pgfusepath{clip}%
\pgfsetbuttcap%
\pgfsetmiterjoin%
\definecolor{currentfill}{rgb}{0.447059,0.447059,0.447059}%
\pgfsetfillcolor{currentfill}%
\pgfsetlinewidth{1.003750pt}%
\definecolor{currentstroke}{rgb}{0.266667,0.266667,0.266667}%
\pgfsetstrokecolor{currentstroke}%
\pgfsetdash{}{0pt}%
\pgfpathmoveto{\pgfqpoint{3.046107in}{2.066039in}}%
\pgfpathlineto{\pgfqpoint{3.771691in}{2.066039in}}%
\pgfpathlineto{\pgfqpoint{3.771691in}{2.067016in}}%
\pgfpathlineto{\pgfqpoint{3.046107in}{2.067016in}}%
\pgfpathlineto{\pgfqpoint{3.046107in}{2.066039in}}%
\pgfpathclose%
\pgfusepath{stroke,fill}%
\end{pgfscope}%
\begin{pgfscope}%
\pgfpathrectangle{\pgfqpoint{0.781250in}{0.638889in}}{\pgfqpoint{4.218750in}{2.172222in}}%
\pgfusepath{clip}%
\pgfsetbuttcap%
\pgfsetmiterjoin%
\definecolor{currentfill}{rgb}{0.447059,0.447059,0.447059}%
\pgfsetfillcolor{currentfill}%
\pgfsetlinewidth{1.003750pt}%
\definecolor{currentstroke}{rgb}{0.266667,0.266667,0.266667}%
\pgfsetstrokecolor{currentstroke}%
\pgfsetdash{}{0pt}%
\pgfpathmoveto{\pgfqpoint{4.082655in}{2.661011in}}%
\pgfpathlineto{\pgfqpoint{4.808239in}{2.661011in}}%
\pgfpathlineto{\pgfqpoint{4.808239in}{2.662640in}}%
\pgfpathlineto{\pgfqpoint{4.082655in}{2.662640in}}%
\pgfpathlineto{\pgfqpoint{4.082655in}{2.661011in}}%
\pgfpathclose%
\pgfusepath{stroke,fill}%
\end{pgfscope}%
\begin{pgfscope}%
\pgfpathrectangle{\pgfqpoint{0.781250in}{0.638889in}}{\pgfqpoint{4.218750in}{2.172222in}}%
\pgfusepath{clip}%
\pgfsetbuttcap%
\pgfsetmiterjoin%
\definecolor{currentfill}{rgb}{0.447059,0.447059,0.447059}%
\pgfsetfillcolor{currentfill}%
\pgfsetlinewidth{1.003750pt}%
\definecolor{currentstroke}{rgb}{0.266667,0.266667,0.266667}%
\pgfsetstrokecolor{currentstroke}%
\pgfsetdash{}{0pt}%
\pgfpathmoveto{\pgfqpoint{0.973011in}{1.794294in}}%
\pgfpathlineto{\pgfqpoint{1.698595in}{1.794294in}}%
\pgfpathlineto{\pgfqpoint{1.698595in}{1.794728in}}%
\pgfpathlineto{\pgfqpoint{0.973011in}{1.794728in}}%
\pgfpathlineto{\pgfqpoint{0.973011in}{1.794294in}}%
\pgfpathclose%
\pgfusepath{stroke,fill}%
\end{pgfscope}%
\begin{pgfscope}%
\pgfpathrectangle{\pgfqpoint{0.781250in}{0.638889in}}{\pgfqpoint{4.218750in}{2.172222in}}%
\pgfusepath{clip}%
\pgfsetbuttcap%
\pgfsetmiterjoin%
\definecolor{currentfill}{rgb}{0.447059,0.447059,0.447059}%
\pgfsetfillcolor{currentfill}%
\pgfsetlinewidth{1.003750pt}%
\definecolor{currentstroke}{rgb}{0.266667,0.266667,0.266667}%
\pgfsetstrokecolor{currentstroke}%
\pgfsetdash{}{0pt}%
\pgfpathmoveto{\pgfqpoint{2.009559in}{1.872385in}}%
\pgfpathlineto{\pgfqpoint{2.735143in}{1.872385in}}%
\pgfpathlineto{\pgfqpoint{2.735143in}{1.872928in}}%
\pgfpathlineto{\pgfqpoint{2.009559in}{1.872928in}}%
\pgfpathlineto{\pgfqpoint{2.009559in}{1.872385in}}%
\pgfpathclose%
\pgfusepath{stroke,fill}%
\end{pgfscope}%
\begin{pgfscope}%
\pgfpathrectangle{\pgfqpoint{0.781250in}{0.638889in}}{\pgfqpoint{4.218750in}{2.172222in}}%
\pgfusepath{clip}%
\pgfsetbuttcap%
\pgfsetmiterjoin%
\definecolor{currentfill}{rgb}{0.447059,0.447059,0.447059}%
\pgfsetfillcolor{currentfill}%
\pgfsetlinewidth{1.003750pt}%
\definecolor{currentstroke}{rgb}{0.266667,0.266667,0.266667}%
\pgfsetstrokecolor{currentstroke}%
\pgfsetdash{}{0pt}%
\pgfpathmoveto{\pgfqpoint{3.046107in}{2.067016in}}%
\pgfpathlineto{\pgfqpoint{3.771691in}{2.067016in}}%
\pgfpathlineto{\pgfqpoint{3.771691in}{2.067125in}}%
\pgfpathlineto{\pgfqpoint{3.046107in}{2.067125in}}%
\pgfpathlineto{\pgfqpoint{3.046107in}{2.067016in}}%
\pgfpathclose%
\pgfusepath{stroke,fill}%
\end{pgfscope}%
\begin{pgfscope}%
\pgfpathrectangle{\pgfqpoint{0.781250in}{0.638889in}}{\pgfqpoint{4.218750in}{2.172222in}}%
\pgfusepath{clip}%
\pgfsetbuttcap%
\pgfsetmiterjoin%
\definecolor{currentfill}{rgb}{0.447059,0.447059,0.447059}%
\pgfsetfillcolor{currentfill}%
\pgfsetlinewidth{1.003750pt}%
\definecolor{currentstroke}{rgb}{0.266667,0.266667,0.266667}%
\pgfsetstrokecolor{currentstroke}%
\pgfsetdash{}{0pt}%
\pgfpathmoveto{\pgfqpoint{4.082655in}{2.662640in}}%
\pgfpathlineto{\pgfqpoint{4.808239in}{2.662640in}}%
\pgfpathlineto{\pgfqpoint{4.808239in}{2.663617in}}%
\pgfpathlineto{\pgfqpoint{4.082655in}{2.663617in}}%
\pgfpathlineto{\pgfqpoint{4.082655in}{2.662640in}}%
\pgfpathclose%
\pgfusepath{stroke,fill}%
\end{pgfscope}%
\begin{pgfscope}%
\pgfpathrectangle{\pgfqpoint{0.781250in}{0.638889in}}{\pgfqpoint{4.218750in}{2.172222in}}%
\pgfusepath{clip}%
\pgfsetbuttcap%
\pgfsetmiterjoin%
\definecolor{currentfill}{rgb}{0.447059,0.447059,0.447059}%
\pgfsetfillcolor{currentfill}%
\pgfsetlinewidth{1.003750pt}%
\definecolor{currentstroke}{rgb}{0.266667,0.266667,0.266667}%
\pgfsetstrokecolor{currentstroke}%
\pgfsetdash{}{0pt}%
\pgfpathmoveto{\pgfqpoint{0.973011in}{1.794728in}}%
\pgfpathlineto{\pgfqpoint{1.698595in}{1.794728in}}%
\pgfpathlineto{\pgfqpoint{1.698595in}{1.802548in}}%
\pgfpathlineto{\pgfqpoint{0.973011in}{1.802548in}}%
\pgfpathlineto{\pgfqpoint{0.973011in}{1.794728in}}%
\pgfpathclose%
\pgfusepath{stroke,fill}%
\end{pgfscope}%
\begin{pgfscope}%
\pgfpathrectangle{\pgfqpoint{0.781250in}{0.638889in}}{\pgfqpoint{4.218750in}{2.172222in}}%
\pgfusepath{clip}%
\pgfsetbuttcap%
\pgfsetmiterjoin%
\definecolor{currentfill}{rgb}{0.447059,0.447059,0.447059}%
\pgfsetfillcolor{currentfill}%
\pgfsetlinewidth{1.003750pt}%
\definecolor{currentstroke}{rgb}{0.266667,0.266667,0.266667}%
\pgfsetstrokecolor{currentstroke}%
\pgfsetdash{}{0pt}%
\pgfpathmoveto{\pgfqpoint{2.009559in}{1.872928in}}%
\pgfpathlineto{\pgfqpoint{2.735143in}{1.872928in}}%
\pgfpathlineto{\pgfqpoint{2.735143in}{1.876838in}}%
\pgfpathlineto{\pgfqpoint{2.009559in}{1.876838in}}%
\pgfpathlineto{\pgfqpoint{2.009559in}{1.872928in}}%
\pgfpathclose%
\pgfusepath{stroke,fill}%
\end{pgfscope}%
\begin{pgfscope}%
\pgfpathrectangle{\pgfqpoint{0.781250in}{0.638889in}}{\pgfqpoint{4.218750in}{2.172222in}}%
\pgfusepath{clip}%
\pgfsetbuttcap%
\pgfsetmiterjoin%
\definecolor{currentfill}{rgb}{0.447059,0.447059,0.447059}%
\pgfsetfillcolor{currentfill}%
\pgfsetlinewidth{1.003750pt}%
\definecolor{currentstroke}{rgb}{0.266667,0.266667,0.266667}%
\pgfsetstrokecolor{currentstroke}%
\pgfsetdash{}{0pt}%
\pgfpathmoveto{\pgfqpoint{3.046107in}{2.067125in}}%
\pgfpathlineto{\pgfqpoint{3.771691in}{2.067125in}}%
\pgfpathlineto{\pgfqpoint{3.771691in}{2.067342in}}%
\pgfpathlineto{\pgfqpoint{3.046107in}{2.067342in}}%
\pgfpathlineto{\pgfqpoint{3.046107in}{2.067125in}}%
\pgfpathclose%
\pgfusepath{stroke,fill}%
\end{pgfscope}%
\begin{pgfscope}%
\pgfpathrectangle{\pgfqpoint{0.781250in}{0.638889in}}{\pgfqpoint{4.218750in}{2.172222in}}%
\pgfusepath{clip}%
\pgfsetbuttcap%
\pgfsetmiterjoin%
\definecolor{currentfill}{rgb}{0.447059,0.447059,0.447059}%
\pgfsetfillcolor{currentfill}%
\pgfsetlinewidth{1.003750pt}%
\definecolor{currentstroke}{rgb}{0.266667,0.266667,0.266667}%
\pgfsetstrokecolor{currentstroke}%
\pgfsetdash{}{0pt}%
\pgfpathmoveto{\pgfqpoint{4.082655in}{2.663617in}}%
\pgfpathlineto{\pgfqpoint{4.808239in}{2.663617in}}%
\pgfpathlineto{\pgfqpoint{4.808239in}{2.664378in}}%
\pgfpathlineto{\pgfqpoint{4.082655in}{2.664378in}}%
\pgfpathlineto{\pgfqpoint{4.082655in}{2.663617in}}%
\pgfpathclose%
\pgfusepath{stroke,fill}%
\end{pgfscope}%
\begin{pgfscope}%
\pgfpathrectangle{\pgfqpoint{0.781250in}{0.638889in}}{\pgfqpoint{4.218750in}{2.172222in}}%
\pgfusepath{clip}%
\pgfsetbuttcap%
\pgfsetmiterjoin%
\definecolor{currentfill}{rgb}{0.447059,0.447059,0.447059}%
\pgfsetfillcolor{currentfill}%
\pgfsetlinewidth{1.003750pt}%
\definecolor{currentstroke}{rgb}{0.266667,0.266667,0.266667}%
\pgfsetstrokecolor{currentstroke}%
\pgfsetdash{}{0pt}%
\pgfpathmoveto{\pgfqpoint{0.973011in}{1.802548in}}%
\pgfpathlineto{\pgfqpoint{1.698595in}{1.802548in}}%
\pgfpathlineto{\pgfqpoint{1.698595in}{1.803634in}}%
\pgfpathlineto{\pgfqpoint{0.973011in}{1.803634in}}%
\pgfpathlineto{\pgfqpoint{0.973011in}{1.802548in}}%
\pgfpathclose%
\pgfusepath{stroke,fill}%
\end{pgfscope}%
\begin{pgfscope}%
\pgfpathrectangle{\pgfqpoint{0.781250in}{0.638889in}}{\pgfqpoint{4.218750in}{2.172222in}}%
\pgfusepath{clip}%
\pgfsetbuttcap%
\pgfsetmiterjoin%
\definecolor{currentfill}{rgb}{0.447059,0.447059,0.447059}%
\pgfsetfillcolor{currentfill}%
\pgfsetlinewidth{1.003750pt}%
\definecolor{currentstroke}{rgb}{0.266667,0.266667,0.266667}%
\pgfsetstrokecolor{currentstroke}%
\pgfsetdash{}{0pt}%
\pgfpathmoveto{\pgfqpoint{2.009559in}{1.876838in}}%
\pgfpathlineto{\pgfqpoint{2.735143in}{1.876838in}}%
\pgfpathlineto{\pgfqpoint{2.735143in}{1.877816in}}%
\pgfpathlineto{\pgfqpoint{2.009559in}{1.877816in}}%
\pgfpathlineto{\pgfqpoint{2.009559in}{1.876838in}}%
\pgfpathclose%
\pgfusepath{stroke,fill}%
\end{pgfscope}%
\begin{pgfscope}%
\pgfpathrectangle{\pgfqpoint{0.781250in}{0.638889in}}{\pgfqpoint{4.218750in}{2.172222in}}%
\pgfusepath{clip}%
\pgfsetbuttcap%
\pgfsetmiterjoin%
\definecolor{currentfill}{rgb}{0.447059,0.447059,0.447059}%
\pgfsetfillcolor{currentfill}%
\pgfsetlinewidth{1.003750pt}%
\definecolor{currentstroke}{rgb}{0.266667,0.266667,0.266667}%
\pgfsetstrokecolor{currentstroke}%
\pgfsetdash{}{0pt}%
\pgfpathmoveto{\pgfqpoint{3.046107in}{2.067342in}}%
\pgfpathlineto{\pgfqpoint{3.771691in}{2.067342in}}%
\pgfpathlineto{\pgfqpoint{3.771691in}{2.068103in}}%
\pgfpathlineto{\pgfqpoint{3.046107in}{2.068103in}}%
\pgfpathlineto{\pgfqpoint{3.046107in}{2.067342in}}%
\pgfpathclose%
\pgfusepath{stroke,fill}%
\end{pgfscope}%
\begin{pgfscope}%
\pgfpathrectangle{\pgfqpoint{0.781250in}{0.638889in}}{\pgfqpoint{4.218750in}{2.172222in}}%
\pgfusepath{clip}%
\pgfsetbuttcap%
\pgfsetmiterjoin%
\definecolor{currentfill}{rgb}{0.447059,0.447059,0.447059}%
\pgfsetfillcolor{currentfill}%
\pgfsetlinewidth{1.003750pt}%
\definecolor{currentstroke}{rgb}{0.266667,0.266667,0.266667}%
\pgfsetstrokecolor{currentstroke}%
\pgfsetdash{}{0pt}%
\pgfpathmoveto{\pgfqpoint{4.082655in}{2.664378in}}%
\pgfpathlineto{\pgfqpoint{4.808239in}{2.664378in}}%
\pgfpathlineto{\pgfqpoint{4.808239in}{2.664812in}}%
\pgfpathlineto{\pgfqpoint{4.082655in}{2.664812in}}%
\pgfpathlineto{\pgfqpoint{4.082655in}{2.664378in}}%
\pgfpathclose%
\pgfusepath{stroke,fill}%
\end{pgfscope}%
\begin{pgfscope}%
\pgfpathrectangle{\pgfqpoint{0.781250in}{0.638889in}}{\pgfqpoint{4.218750in}{2.172222in}}%
\pgfusepath{clip}%
\pgfsetbuttcap%
\pgfsetmiterjoin%
\definecolor{currentfill}{rgb}{0.447059,0.447059,0.447059}%
\pgfsetfillcolor{currentfill}%
\pgfsetlinewidth{1.003750pt}%
\definecolor{currentstroke}{rgb}{0.266667,0.266667,0.266667}%
\pgfsetstrokecolor{currentstroke}%
\pgfsetdash{}{0pt}%
\pgfpathmoveto{\pgfqpoint{0.973011in}{1.803634in}}%
\pgfpathlineto{\pgfqpoint{1.698595in}{1.803634in}}%
\pgfpathlineto{\pgfqpoint{1.698595in}{1.803743in}}%
\pgfpathlineto{\pgfqpoint{0.973011in}{1.803743in}}%
\pgfpathlineto{\pgfqpoint{0.973011in}{1.803634in}}%
\pgfpathclose%
\pgfusepath{stroke,fill}%
\end{pgfscope}%
\begin{pgfscope}%
\pgfpathrectangle{\pgfqpoint{0.781250in}{0.638889in}}{\pgfqpoint{4.218750in}{2.172222in}}%
\pgfusepath{clip}%
\pgfsetbuttcap%
\pgfsetmiterjoin%
\definecolor{currentfill}{rgb}{0.447059,0.447059,0.447059}%
\pgfsetfillcolor{currentfill}%
\pgfsetlinewidth{1.003750pt}%
\definecolor{currentstroke}{rgb}{0.266667,0.266667,0.266667}%
\pgfsetstrokecolor{currentstroke}%
\pgfsetdash{}{0pt}%
\pgfpathmoveto{\pgfqpoint{2.009559in}{1.877816in}}%
\pgfpathlineto{\pgfqpoint{2.735143in}{1.877816in}}%
\pgfpathlineto{\pgfqpoint{2.735143in}{1.878793in}}%
\pgfpathlineto{\pgfqpoint{2.009559in}{1.878793in}}%
\pgfpathlineto{\pgfqpoint{2.009559in}{1.877816in}}%
\pgfpathclose%
\pgfusepath{stroke,fill}%
\end{pgfscope}%
\begin{pgfscope}%
\pgfpathrectangle{\pgfqpoint{0.781250in}{0.638889in}}{\pgfqpoint{4.218750in}{2.172222in}}%
\pgfusepath{clip}%
\pgfsetbuttcap%
\pgfsetmiterjoin%
\definecolor{currentfill}{rgb}{0.447059,0.447059,0.447059}%
\pgfsetfillcolor{currentfill}%
\pgfsetlinewidth{1.003750pt}%
\definecolor{currentstroke}{rgb}{0.266667,0.266667,0.266667}%
\pgfsetstrokecolor{currentstroke}%
\pgfsetdash{}{0pt}%
\pgfpathmoveto{\pgfqpoint{3.046107in}{2.068103in}}%
\pgfpathlineto{\pgfqpoint{3.771691in}{2.068103in}}%
\pgfpathlineto{\pgfqpoint{3.771691in}{2.068428in}}%
\pgfpathlineto{\pgfqpoint{3.046107in}{2.068428in}}%
\pgfpathlineto{\pgfqpoint{3.046107in}{2.068103in}}%
\pgfpathclose%
\pgfusepath{stroke,fill}%
\end{pgfscope}%
\begin{pgfscope}%
\pgfpathrectangle{\pgfqpoint{0.781250in}{0.638889in}}{\pgfqpoint{4.218750in}{2.172222in}}%
\pgfusepath{clip}%
\pgfsetbuttcap%
\pgfsetmiterjoin%
\definecolor{currentfill}{rgb}{0.447059,0.447059,0.447059}%
\pgfsetfillcolor{currentfill}%
\pgfsetlinewidth{1.003750pt}%
\definecolor{currentstroke}{rgb}{0.266667,0.266667,0.266667}%
\pgfsetstrokecolor{currentstroke}%
\pgfsetdash{}{0pt}%
\pgfpathmoveto{\pgfqpoint{4.082655in}{2.664812in}}%
\pgfpathlineto{\pgfqpoint{4.808239in}{2.664812in}}%
\pgfpathlineto{\pgfqpoint{4.808239in}{2.665246in}}%
\pgfpathlineto{\pgfqpoint{4.082655in}{2.665246in}}%
\pgfpathlineto{\pgfqpoint{4.082655in}{2.664812in}}%
\pgfpathclose%
\pgfusepath{stroke,fill}%
\end{pgfscope}%
\begin{pgfscope}%
\pgfpathrectangle{\pgfqpoint{0.781250in}{0.638889in}}{\pgfqpoint{4.218750in}{2.172222in}}%
\pgfusepath{clip}%
\pgfsetbuttcap%
\pgfsetmiterjoin%
\definecolor{currentfill}{rgb}{0.447059,0.447059,0.447059}%
\pgfsetfillcolor{currentfill}%
\pgfsetlinewidth{1.003750pt}%
\definecolor{currentstroke}{rgb}{0.266667,0.266667,0.266667}%
\pgfsetstrokecolor{currentstroke}%
\pgfsetdash{}{0pt}%
\pgfpathmoveto{\pgfqpoint{0.973011in}{1.803743in}}%
\pgfpathlineto{\pgfqpoint{1.698595in}{1.803743in}}%
\pgfpathlineto{\pgfqpoint{1.698595in}{1.803743in}}%
\pgfpathlineto{\pgfqpoint{0.973011in}{1.803743in}}%
\pgfpathlineto{\pgfqpoint{0.973011in}{1.803743in}}%
\pgfpathclose%
\pgfusepath{stroke,fill}%
\end{pgfscope}%
\begin{pgfscope}%
\pgfpathrectangle{\pgfqpoint{0.781250in}{0.638889in}}{\pgfqpoint{4.218750in}{2.172222in}}%
\pgfusepath{clip}%
\pgfsetbuttcap%
\pgfsetmiterjoin%
\definecolor{currentfill}{rgb}{0.447059,0.447059,0.447059}%
\pgfsetfillcolor{currentfill}%
\pgfsetlinewidth{1.003750pt}%
\definecolor{currentstroke}{rgb}{0.266667,0.266667,0.266667}%
\pgfsetstrokecolor{currentstroke}%
\pgfsetdash{}{0pt}%
\pgfpathmoveto{\pgfqpoint{2.009559in}{1.878793in}}%
\pgfpathlineto{\pgfqpoint{2.735143in}{1.878793in}}%
\pgfpathlineto{\pgfqpoint{2.735143in}{1.878902in}}%
\pgfpathlineto{\pgfqpoint{2.009559in}{1.878902in}}%
\pgfpathlineto{\pgfqpoint{2.009559in}{1.878793in}}%
\pgfpathclose%
\pgfusepath{stroke,fill}%
\end{pgfscope}%
\begin{pgfscope}%
\pgfpathrectangle{\pgfqpoint{0.781250in}{0.638889in}}{\pgfqpoint{4.218750in}{2.172222in}}%
\pgfusepath{clip}%
\pgfsetbuttcap%
\pgfsetmiterjoin%
\definecolor{currentfill}{rgb}{0.447059,0.447059,0.447059}%
\pgfsetfillcolor{currentfill}%
\pgfsetlinewidth{1.003750pt}%
\definecolor{currentstroke}{rgb}{0.266667,0.266667,0.266667}%
\pgfsetstrokecolor{currentstroke}%
\pgfsetdash{}{0pt}%
\pgfpathmoveto{\pgfqpoint{3.046107in}{2.068428in}}%
\pgfpathlineto{\pgfqpoint{3.771691in}{2.068428in}}%
\pgfpathlineto{\pgfqpoint{3.771691in}{2.068537in}}%
\pgfpathlineto{\pgfqpoint{3.046107in}{2.068537in}}%
\pgfpathlineto{\pgfqpoint{3.046107in}{2.068428in}}%
\pgfpathclose%
\pgfusepath{stroke,fill}%
\end{pgfscope}%
\begin{pgfscope}%
\pgfpathrectangle{\pgfqpoint{0.781250in}{0.638889in}}{\pgfqpoint{4.218750in}{2.172222in}}%
\pgfusepath{clip}%
\pgfsetbuttcap%
\pgfsetmiterjoin%
\definecolor{currentfill}{rgb}{0.447059,0.447059,0.447059}%
\pgfsetfillcolor{currentfill}%
\pgfsetlinewidth{1.003750pt}%
\definecolor{currentstroke}{rgb}{0.266667,0.266667,0.266667}%
\pgfsetstrokecolor{currentstroke}%
\pgfsetdash{}{0pt}%
\pgfpathmoveto{\pgfqpoint{4.082655in}{2.665246in}}%
\pgfpathlineto{\pgfqpoint{4.808239in}{2.665246in}}%
\pgfpathlineto{\pgfqpoint{4.808239in}{2.665246in}}%
\pgfpathlineto{\pgfqpoint{4.082655in}{2.665246in}}%
\pgfpathlineto{\pgfqpoint{4.082655in}{2.665246in}}%
\pgfpathclose%
\pgfusepath{stroke,fill}%
\end{pgfscope}%
\begin{pgfscope}%
\pgfpathrectangle{\pgfqpoint{0.781250in}{0.638889in}}{\pgfqpoint{4.218750in}{2.172222in}}%
\pgfusepath{clip}%
\pgfsetbuttcap%
\pgfsetmiterjoin%
\definecolor{currentfill}{rgb}{0.447059,0.447059,0.447059}%
\pgfsetfillcolor{currentfill}%
\pgfsetlinewidth{1.003750pt}%
\definecolor{currentstroke}{rgb}{0.266667,0.266667,0.266667}%
\pgfsetstrokecolor{currentstroke}%
\pgfsetdash{}{0pt}%
\pgfpathmoveto{\pgfqpoint{0.973011in}{1.803743in}}%
\pgfpathlineto{\pgfqpoint{1.698595in}{1.803743in}}%
\pgfpathlineto{\pgfqpoint{1.698595in}{1.803743in}}%
\pgfpathlineto{\pgfqpoint{0.973011in}{1.803743in}}%
\pgfpathlineto{\pgfqpoint{0.973011in}{1.803743in}}%
\pgfpathclose%
\pgfusepath{stroke,fill}%
\end{pgfscope}%
\begin{pgfscope}%
\pgfpathrectangle{\pgfqpoint{0.781250in}{0.638889in}}{\pgfqpoint{4.218750in}{2.172222in}}%
\pgfusepath{clip}%
\pgfsetbuttcap%
\pgfsetmiterjoin%
\definecolor{currentfill}{rgb}{0.447059,0.447059,0.447059}%
\pgfsetfillcolor{currentfill}%
\pgfsetlinewidth{1.003750pt}%
\definecolor{currentstroke}{rgb}{0.266667,0.266667,0.266667}%
\pgfsetstrokecolor{currentstroke}%
\pgfsetdash{}{0pt}%
\pgfpathmoveto{\pgfqpoint{2.009559in}{1.878902in}}%
\pgfpathlineto{\pgfqpoint{2.735143in}{1.878902in}}%
\pgfpathlineto{\pgfqpoint{2.735143in}{1.878902in}}%
\pgfpathlineto{\pgfqpoint{2.009559in}{1.878902in}}%
\pgfpathlineto{\pgfqpoint{2.009559in}{1.878902in}}%
\pgfpathclose%
\pgfusepath{stroke,fill}%
\end{pgfscope}%
\begin{pgfscope}%
\pgfpathrectangle{\pgfqpoint{0.781250in}{0.638889in}}{\pgfqpoint{4.218750in}{2.172222in}}%
\pgfusepath{clip}%
\pgfsetbuttcap%
\pgfsetmiterjoin%
\definecolor{currentfill}{rgb}{0.447059,0.447059,0.447059}%
\pgfsetfillcolor{currentfill}%
\pgfsetlinewidth{1.003750pt}%
\definecolor{currentstroke}{rgb}{0.266667,0.266667,0.266667}%
\pgfsetstrokecolor{currentstroke}%
\pgfsetdash{}{0pt}%
\pgfpathmoveto{\pgfqpoint{3.046107in}{2.068537in}}%
\pgfpathlineto{\pgfqpoint{3.771691in}{2.068537in}}%
\pgfpathlineto{\pgfqpoint{3.771691in}{2.068537in}}%
\pgfpathlineto{\pgfqpoint{3.046107in}{2.068537in}}%
\pgfpathlineto{\pgfqpoint{3.046107in}{2.068537in}}%
\pgfpathclose%
\pgfusepath{stroke,fill}%
\end{pgfscope}%
\begin{pgfscope}%
\pgfpathrectangle{\pgfqpoint{0.781250in}{0.638889in}}{\pgfqpoint{4.218750in}{2.172222in}}%
\pgfusepath{clip}%
\pgfsetbuttcap%
\pgfsetmiterjoin%
\definecolor{currentfill}{rgb}{0.447059,0.447059,0.447059}%
\pgfsetfillcolor{currentfill}%
\pgfsetlinewidth{1.003750pt}%
\definecolor{currentstroke}{rgb}{0.266667,0.266667,0.266667}%
\pgfsetstrokecolor{currentstroke}%
\pgfsetdash{}{0pt}%
\pgfpathmoveto{\pgfqpoint{4.082655in}{2.665246in}}%
\pgfpathlineto{\pgfqpoint{4.808239in}{2.665246in}}%
\pgfpathlineto{\pgfqpoint{4.808239in}{2.665246in}}%
\pgfpathlineto{\pgfqpoint{4.082655in}{2.665246in}}%
\pgfpathlineto{\pgfqpoint{4.082655in}{2.665246in}}%
\pgfpathclose%
\pgfusepath{stroke,fill}%
\end{pgfscope}%
\begin{pgfscope}%
\definecolor{textcolor}{rgb}{0.000000,0.000000,0.000000}%
\pgfsetstrokecolor{textcolor}%
\pgfsetfillcolor{textcolor}%
\pgftext[x=1.335803in,y=1.831521in,,bottom]{\color{textcolor}{\ifdefined\pdftexversion\else\setmainfont{NanumMyeongjo}\rmfamily\fi\fontsize{7.000000}{8.400000}\selectfont\catcode`\^=\active\def^{\ifmmode\sp\else\^{}\fi}\catcode`\%=\active\def%{\%}10,725}}%
\end{pgfscope}%
\begin{pgfscope}%
\definecolor{textcolor}{rgb}{0.000000,0.000000,0.000000}%
\pgfsetstrokecolor{textcolor}%
\pgfsetfillcolor{textcolor}%
\pgftext[x=2.372351in,y=1.906680in,,bottom]{\color{textcolor}{\ifdefined\pdftexversion\else\setmainfont{NanumMyeongjo}\rmfamily\fi\fontsize{7.000000}{8.400000}\selectfont\catcode`\^=\active\def^{\ifmmode\sp\else\^{}\fi}\catcode`\%=\active\def%{\%}11,417}}%
\end{pgfscope}%
\begin{pgfscope}%
\definecolor{textcolor}{rgb}{0.000000,0.000000,0.000000}%
\pgfsetstrokecolor{textcolor}%
\pgfsetfillcolor{textcolor}%
\pgftext[x=3.408899in,y=2.096315in,,bottom]{\color{textcolor}{\ifdefined\pdftexversion\else\setmainfont{NanumMyeongjo}\rmfamily\fi\fontsize{7.000000}{8.400000}\selectfont\catcode`\^=\active\def^{\ifmmode\sp\else\^{}\fi}\catcode`\%=\active\def%{\%}13,163}}%
\end{pgfscope}%
\begin{pgfscope}%
\definecolor{textcolor}{rgb}{0.000000,0.000000,0.000000}%
\pgfsetstrokecolor{textcolor}%
\pgfsetfillcolor{textcolor}%
\pgftext[x=4.445447in,y=2.693024in,,bottom]{\color{textcolor}{\ifdefined\pdftexversion\else\setmainfont{NanumMyeongjo}\rmfamily\fi\fontsize{7.000000}{8.400000}\selectfont\catcode`\^=\active\def^{\ifmmode\sp\else\^{}\fi}\catcode`\%=\active\def%{\%}18,657}}%
\end{pgfscope}%
\begin{pgfscope}%
\definecolor{textcolor}{rgb}{1.000000,1.000000,1.000000}%
\pgfsetstrokecolor{textcolor}%
\pgfsetfillcolor{textcolor}%
\pgftext[x=1.335803in,y=1.189982in,,]{\color{textcolor}{\ifdefined\pdftexversion\else\setmainfont{NanumMyeongjo}\rmfamily\fi\fontsize{7.000000}{8.400000}\selectfont\catcode`\^=\active\def^{\ifmmode\sp\else\^{}\fi}\catcode`\%=\active\def%{\%}6,074}}%
\end{pgfscope}%
\begin{pgfscope}%
\definecolor{textcolor}{rgb}{1.000000,1.000000,1.000000}%
\pgfsetstrokecolor{textcolor}%
\pgfsetfillcolor{textcolor}%
\pgftext[x=2.372351in,y=1.072790in,,]{\color{textcolor}{\ifdefined\pdftexversion\else\setmainfont{NanumMyeongjo}\rmfamily\fi\fontsize{7.000000}{8.400000}\selectfont\catcode`\^=\active\def^{\ifmmode\sp\else\^{}\fi}\catcode`\%=\active\def%{\%}4,995}}%
\end{pgfscope}%
\begin{pgfscope}%
\definecolor{textcolor}{rgb}{1.000000,1.000000,1.000000}%
\pgfsetstrokecolor{textcolor}%
\pgfsetfillcolor{textcolor}%
\pgftext[x=3.408899in,y=1.219741in,,]{\color{textcolor}{\ifdefined\pdftexversion\else\setmainfont{NanumMyeongjo}\rmfamily\fi\fontsize{7.000000}{8.400000}\selectfont\catcode`\^=\active\def^{\ifmmode\sp\else\^{}\fi}\catcode`\%=\active\def%{\%}6,348}}%
\end{pgfscope}%
\begin{pgfscope}%
\definecolor{textcolor}{rgb}{1.000000,1.000000,1.000000}%
\pgfsetstrokecolor{textcolor}%
\pgfsetfillcolor{textcolor}%
\pgftext[x=4.445447in,y=1.700888in,,]{\color{textcolor}{\ifdefined\pdftexversion\else\setmainfont{NanumMyeongjo}\rmfamily\fi\fontsize{7.000000}{8.400000}\selectfont\catcode`\^=\active\def^{\ifmmode\sp\else\^{}\fi}\catcode`\%=\active\def%{\%}10,778}}%
\end{pgfscope}%
\begin{pgfscope}%
\pgfsetbuttcap%
\pgfsetmiterjoin%
\definecolor{currentfill}{rgb}{0.337255,0.713725,0.627451}%
\pgfsetfillcolor{currentfill}%
\pgfsetlinewidth{1.003750pt}%
\definecolor{currentstroke}{rgb}{0.266667,0.266667,0.266667}%
\pgfsetstrokecolor{currentstroke}%
\pgfsetdash{}{0pt}%
\pgfpathmoveto{\pgfqpoint{5.112500in}{2.598758in}}%
\pgfpathlineto{\pgfqpoint{5.362500in}{2.598758in}}%
\pgfpathlineto{\pgfqpoint{5.362500in}{2.686258in}}%
\pgfpathlineto{\pgfqpoint{5.112500in}{2.686258in}}%
\pgfpathlineto{\pgfqpoint{5.112500in}{2.598758in}}%
\pgfpathclose%
\pgfusepath{stroke,fill}%
\end{pgfscope}%
\begin{pgfscope}%
\definecolor{textcolor}{rgb}{0.000000,0.000000,0.000000}%
\pgfsetstrokecolor{textcolor}%
\pgfsetfillcolor{textcolor}%
\pgftext[x=5.462500in,y=2.598758in,left,base]{\color{textcolor}{\ifdefined\pdftexversion\else\setmainfont{NanumMyeongjo}\rmfamily\fi\fontsize{9.000000}{10.800000}\selectfont\catcode`\^=\active\def^{\ifmmode\sp\else\^{}\fi}\catcode`\%=\active\def%{\%}전북}}%
\end{pgfscope}%
\begin{pgfscope}%
\pgfsetbuttcap%
\pgfsetmiterjoin%
\definecolor{currentfill}{rgb}{0.725490,0.486275,0.164706}%
\pgfsetfillcolor{currentfill}%
\pgfsetlinewidth{1.003750pt}%
\definecolor{currentstroke}{rgb}{0.266667,0.266667,0.266667}%
\pgfsetstrokecolor{currentstroke}%
\pgfsetdash{}{0pt}%
\pgfpathmoveto{\pgfqpoint{5.112500in}{2.407474in}}%
\pgfpathlineto{\pgfqpoint{5.362500in}{2.407474in}}%
\pgfpathlineto{\pgfqpoint{5.362500in}{2.494974in}}%
\pgfpathlineto{\pgfqpoint{5.112500in}{2.494974in}}%
\pgfpathlineto{\pgfqpoint{5.112500in}{2.407474in}}%
\pgfpathclose%
\pgfusepath{stroke,fill}%
\end{pgfscope}%
\begin{pgfscope}%
\definecolor{textcolor}{rgb}{0.000000,0.000000,0.000000}%
\pgfsetstrokecolor{textcolor}%
\pgfsetfillcolor{textcolor}%
\pgftext[x=5.462500in,y=2.407474in,left,base]{\color{textcolor}{\ifdefined\pdftexversion\else\setmainfont{NanumMyeongjo}\rmfamily\fi\fontsize{9.000000}{10.800000}\selectfont\catcode`\^=\active\def^{\ifmmode\sp\else\^{}\fi}\catcode`\%=\active\def%{\%}전남}}%
\end{pgfscope}%
\begin{pgfscope}%
\pgfsetbuttcap%
\pgfsetmiterjoin%
\definecolor{currentfill}{rgb}{0.235294,0.490196,0.764706}%
\pgfsetfillcolor{currentfill}%
\pgfsetlinewidth{1.003750pt}%
\definecolor{currentstroke}{rgb}{0.266667,0.266667,0.266667}%
\pgfsetstrokecolor{currentstroke}%
\pgfsetdash{}{0pt}%
\pgfpathmoveto{\pgfqpoint{5.112500in}{2.216190in}}%
\pgfpathlineto{\pgfqpoint{5.362500in}{2.216190in}}%
\pgfpathlineto{\pgfqpoint{5.362500in}{2.303690in}}%
\pgfpathlineto{\pgfqpoint{5.112500in}{2.303690in}}%
\pgfpathlineto{\pgfqpoint{5.112500in}{2.216190in}}%
\pgfpathclose%
\pgfusepath{stroke,fill}%
\end{pgfscope}%
\begin{pgfscope}%
\definecolor{textcolor}{rgb}{0.000000,0.000000,0.000000}%
\pgfsetstrokecolor{textcolor}%
\pgfsetfillcolor{textcolor}%
\pgftext[x=5.462500in,y=2.216190in,left,base]{\color{textcolor}{\ifdefined\pdftexversion\else\setmainfont{NanumMyeongjo}\rmfamily\fi\fontsize{9.000000}{10.800000}\selectfont\catcode`\^=\active\def^{\ifmmode\sp\else\^{}\fi}\catcode`\%=\active\def%{\%}경북}}%
\end{pgfscope}%
\begin{pgfscope}%
\pgfsetbuttcap%
\pgfsetmiterjoin%
\definecolor{currentfill}{rgb}{0.549020,0.247059,0.121569}%
\pgfsetfillcolor{currentfill}%
\pgfsetlinewidth{1.003750pt}%
\definecolor{currentstroke}{rgb}{0.266667,0.266667,0.266667}%
\pgfsetstrokecolor{currentstroke}%
\pgfsetdash{}{0pt}%
\pgfpathmoveto{\pgfqpoint{5.112500in}{2.024906in}}%
\pgfpathlineto{\pgfqpoint{5.362500in}{2.024906in}}%
\pgfpathlineto{\pgfqpoint{5.362500in}{2.112406in}}%
\pgfpathlineto{\pgfqpoint{5.112500in}{2.112406in}}%
\pgfpathlineto{\pgfqpoint{5.112500in}{2.024906in}}%
\pgfpathclose%
\pgfusepath{stroke,fill}%
\end{pgfscope}%
\begin{pgfscope}%
\definecolor{textcolor}{rgb}{0.000000,0.000000,0.000000}%
\pgfsetstrokecolor{textcolor}%
\pgfsetfillcolor{textcolor}%
\pgftext[x=5.462500in,y=2.024906in,left,base]{\color{textcolor}{\ifdefined\pdftexversion\else\setmainfont{NanumMyeongjo}\rmfamily\fi\fontsize{9.000000}{10.800000}\selectfont\catcode`\^=\active\def^{\ifmmode\sp\else\^{}\fi}\catcode`\%=\active\def%{\%}충남}}%
\end{pgfscope}%
\begin{pgfscope}%
\pgfsetbuttcap%
\pgfsetmiterjoin%
\definecolor{currentfill}{rgb}{0.447059,0.447059,0.447059}%
\pgfsetfillcolor{currentfill}%
\pgfsetlinewidth{1.003750pt}%
\definecolor{currentstroke}{rgb}{0.266667,0.266667,0.266667}%
\pgfsetstrokecolor{currentstroke}%
\pgfsetdash{}{0pt}%
\pgfpathmoveto{\pgfqpoint{5.112500in}{1.833622in}}%
\pgfpathlineto{\pgfqpoint{5.362500in}{1.833622in}}%
\pgfpathlineto{\pgfqpoint{5.362500in}{1.921122in}}%
\pgfpathlineto{\pgfqpoint{5.112500in}{1.921122in}}%
\pgfpathlineto{\pgfqpoint{5.112500in}{1.833622in}}%
\pgfpathclose%
\pgfusepath{stroke,fill}%
\end{pgfscope}%
\begin{pgfscope}%
\definecolor{textcolor}{rgb}{0.000000,0.000000,0.000000}%
\pgfsetstrokecolor{textcolor}%
\pgfsetfillcolor{textcolor}%
\pgftext[x=5.462500in,y=1.833622in,left,base]{\color{textcolor}{\ifdefined\pdftexversion\else\setmainfont{NanumMyeongjo}\rmfamily\fi\fontsize{9.000000}{10.800000}\selectfont\catcode`\^=\active\def^{\ifmmode\sp\else\^{}\fi}\catcode`\%=\active\def%{\%}기타}}%
\end{pgfscope}%
\begin{pgfscope}%
\definecolor{textcolor}{rgb}{0.333333,0.333333,0.333333}%
\pgfsetstrokecolor{textcolor}%
\pgfsetfillcolor{textcolor}%
\pgftext[x=1.875000in,y=0.319444in,,top]{\color{textcolor}{\ifdefined\pdftexversion\else\setmainfont{NanumMyeongjo}\rmfamily\fi\fontsize{9.000000}{10.800000}\selectfont\catcode`\^=\active\def^{\ifmmode\sp\else\^{}\fi}\catcode`\%=\active\def%{\%}출처: 국가농식품통계서비스(KASS) 자료 기반 저자 작성}}%
\end{pgfscope}%
\begin{pgfscope}%
\definecolor{textcolor}{rgb}{0.333333,0.333333,0.333333}%
\pgfsetstrokecolor{textcolor}%
\pgfsetfillcolor{textcolor}%
\pgftext[x=4.562500in,y=2.970833in,,top]{\color{textcolor}{\ifdefined\pdftexversion\else\setmainfont{NanumMyeongjo}\rmfamily\fi\fontsize{9.000000}{10.800000}\selectfont\catcode`\^=\active\def^{\ifmmode\sp\else\^{}\fi}\catcode`\%=\active\def%{\%}(단위: ha)}}%
\end{pgfscope}%
\end{pgfpicture}%
\makeatother%
\endgroup%
}
\end{center}
}


\slide
{\maintitle}
{\chapterthree}
{전라북도 논콩 선제적 지원}{
\begin{minipage}[c]{0.50\textwidth}
    \centering
    \includegraphics[width=\textwidth]{asset/jtv보도.png}
\end{minipage}
\hfill
\begin{minipage}[c]{0.50\textwidth}
    \centering
    \includegraphics[width=0.5\textwidth]{asset/전북_날짜.png} \\
    \includegraphics[width=\textwidth]{asset/전북_텍스트.png}
\end{minipage}
\par
\vspace{10pt}
\small ※ 김진형. “전라북도, 콩 관련 산업에 470억 원 투입”. JTV. 2022.01.31.
}

\slide
{\maintitle}
{\chapterthree}
{타지역 논콩 지원 확대}{

\includegraphics[width=0.8\textwidth]{asset/김제.png}
\par
\vspace{5pt}
\includegraphics[width=0.7\textwidth]{asset/영주.png}
\par
\vspace{5pt}
\includegraphics[width=\textwidth]{asset/충주.png} 

\par
\vspace{10pt}
\small ※ 출처: 한국농어민신문
}



\slide
{\maintitle}
{\chapterthree}
{적지선정의 중요성}{
\centering {\includegraphics[width=0.5\textwidth]{asset/논콩재배관리.png}
}
\begin{itemize}
    \item 일반적으로 논은 낮은 지대의 평야지에 위치함
    \item 대부분 지하수위가 높고 물빠짐이 좋지 않음
    \item 따라서 부적지에 콩을 재배할 경우 습해와 병해로 \\목표수량을 기대하기 어려움
\end{itemize}
\vspace{10pt}
\small ※ 유용환. 2005. 논 콩재배기술 및 재해관리 방법. 물만 먹고 자라요.
}

\slide
{\maintitle}
{\chapterthree}
{배수관리의 중요성}{
\vspace{10pt}
\centering {\includegraphics[width=0.7\textwidth]{asset/문윤만.png}
}
\vspace{10pt}
\begin{itemize}
    \item 밭작물인 콩을 논에 재배하기 위해서 \\가장 먼저 고려해야 할 사항이 배수 관리임
    \item 늦여름 시기는 콩이 영양생장기에서 \\생식생장기로 전환되는 중요한 시점
    \item 우리나라는 대부분의 강수량이 늦여름 시기에 몰려있음
    \item 배수시설이 완비되지 못한 논에 콩을 재배하면 \\수량감소와 생육장해를 초래
\end{itemize}
\vspace{10pt}
\small ※ 문윤만. 2018. 논 콩 재배 확대를 위한 새로운 배수개선 기술. 콩산업정보지, Vol.1 No.- [2018]
}


\slide
{\maintitle}
{\chapterthree}
{부안군 25년도 강수량}{
\begin{center}
    \hspace*{-40pt}{%% Creator: Matplotlib, PGF backend
%%
%% To include the figure in your LaTeX document, write
%%   \input{<filename>.pgf}
%%
%% Make sure the required packages are loaded in your preamble
%%   \usepackage{pgf}
%%
%% Also ensure that all the required font packages are loaded; for instance,
%% the lmodern package is sometimes necessary when using math font.
%%   \usepackage{lmodern}
%%
%% Figures using additional raster images can only be included by \input if
%% they are in the same directory as the main LaTeX file. For loading figures
%% from other directories you can use the `import` package
%%   \usepackage{import}
%%
%% and then include the figures with
%%   \import{<path to file>}{<filename>.pgf}
%%
%% Matplotlib used the following preamble
%%   \def\mathdefault#1{#1}
%%   \everymath=\expandafter{\the\everymath\displaystyle}
%%   \IfFileExists{scrextend.sty}{
%%     \usepackage[fontsize=9.000000pt]{scrextend}
%%   }{
%%     \renewcommand{\normalsize}{\fontsize{9.000000}{10.800000}\selectfont}
%%     \normalsize
%%   }
%%   
%%   \ifdefined\pdftexversion\else  % non-pdftex case.
%%     \usepackage{fontspec}
%%     \setmainfont{DejaVuSerif.ttf}[Path=\detokenize{/home/user/.cache/pypoetry/virtualenvs/graph-KASAOWVd-py3.12/lib/python3.12/site-packages/matplotlib/mpl-data/fonts/ttf/}]
%%     \setsansfont{DejaVuSans.ttf}[Path=\detokenize{/home/user/.cache/pypoetry/virtualenvs/graph-KASAOWVd-py3.12/lib/python3.12/site-packages/matplotlib/mpl-data/fonts/ttf/}]
%%     \setmonofont{DejaVuSansMono.ttf}[Path=\detokenize{/home/user/.cache/pypoetry/virtualenvs/graph-KASAOWVd-py3.12/lib/python3.12/site-packages/matplotlib/mpl-data/fonts/ttf/}]
%%   \fi
%%   \makeatletter\@ifpackageloaded{underscore}{}{\usepackage[strings]{underscore}}\makeatother
%%
\begingroup%
\makeatletter%
\begin{pgfpicture}%
\pgfpathrectangle{\pgfpointorigin}{\pgfqpoint{6.250000in}{3.194444in}}%
\pgfusepath{use as bounding box, clip}%
\begin{pgfscope}%
\pgfsetbuttcap%
\pgfsetmiterjoin%
\definecolor{currentfill}{rgb}{1.000000,1.000000,1.000000}%
\pgfsetfillcolor{currentfill}%
\pgfsetlinewidth{0.000000pt}%
\definecolor{currentstroke}{rgb}{1.000000,1.000000,1.000000}%
\pgfsetstrokecolor{currentstroke}%
\pgfsetdash{}{0pt}%
\pgfpathmoveto{\pgfqpoint{0.000000in}{0.000000in}}%
\pgfpathlineto{\pgfqpoint{6.250000in}{0.000000in}}%
\pgfpathlineto{\pgfqpoint{6.250000in}{3.194444in}}%
\pgfpathlineto{\pgfqpoint{0.000000in}{3.194444in}}%
\pgfpathlineto{\pgfqpoint{0.000000in}{0.000000in}}%
\pgfpathclose%
\pgfusepath{fill}%
\end{pgfscope}%
\begin{pgfscope}%
\pgfsetbuttcap%
\pgfsetmiterjoin%
\definecolor{currentfill}{rgb}{1.000000,1.000000,1.000000}%
\pgfsetfillcolor{currentfill}%
\pgfsetlinewidth{0.000000pt}%
\definecolor{currentstroke}{rgb}{0.000000,0.000000,0.000000}%
\pgfsetstrokecolor{currentstroke}%
\pgfsetstrokeopacity{0.000000}%
\pgfsetdash{}{0pt}%
\pgfpathmoveto{\pgfqpoint{0.781250in}{0.638889in}}%
\pgfpathlineto{\pgfqpoint{5.000000in}{0.638889in}}%
\pgfpathlineto{\pgfqpoint{5.000000in}{2.811111in}}%
\pgfpathlineto{\pgfqpoint{0.781250in}{2.811111in}}%
\pgfpathlineto{\pgfqpoint{0.781250in}{0.638889in}}%
\pgfpathclose%
\pgfusepath{fill}%
\end{pgfscope}%
\begin{pgfscope}%
\pgfsetbuttcap%
\pgfsetroundjoin%
\definecolor{currentfill}{rgb}{0.000000,0.000000,0.000000}%
\pgfsetfillcolor{currentfill}%
\pgfsetlinewidth{0.752812pt}%
\definecolor{currentstroke}{rgb}{0.000000,0.000000,0.000000}%
\pgfsetstrokecolor{currentstroke}%
\pgfsetdash{}{0pt}%
\pgfsys@defobject{currentmarker}{\pgfqpoint{0.000000in}{-0.013889in}}{\pgfqpoint{0.000000in}{0.000000in}}{%
\pgfpathmoveto{\pgfqpoint{0.000000in}{0.000000in}}%
\pgfpathlineto{\pgfqpoint{0.000000in}{-0.013889in}}%
\pgfusepath{stroke,fill}%
}%
\begin{pgfscope}%
\pgfsys@transformshift{1.087740in}{0.638889in}%
\pgfsys@useobject{currentmarker}{}%
\end{pgfscope}%
\end{pgfscope}%
\begin{pgfscope}%
\definecolor{textcolor}{rgb}{0.000000,0.000000,0.000000}%
\pgfsetstrokecolor{textcolor}%
\pgfsetfillcolor{textcolor}%
\pgftext[x=1.087740in,y=0.576389in,,top]{\color{textcolor}{\ifdefined\pdftexversion\else\setmainfont{NanumMyeongjo}\rmfamily\fi\fontsize{9.000000}{10.800000}\selectfont\catcode`\^=\active\def^{\ifmmode\sp\else\^{}\fi}\catcode`\%=\active\def%{\%}1월}}%
\end{pgfscope}%
\begin{pgfscope}%
\pgfsetbuttcap%
\pgfsetroundjoin%
\definecolor{currentfill}{rgb}{0.000000,0.000000,0.000000}%
\pgfsetfillcolor{currentfill}%
\pgfsetlinewidth{0.752812pt}%
\definecolor{currentstroke}{rgb}{0.000000,0.000000,0.000000}%
\pgfsetstrokecolor{currentstroke}%
\pgfsetdash{}{0pt}%
\pgfsys@defobject{currentmarker}{\pgfqpoint{0.000000in}{-0.013889in}}{\pgfqpoint{0.000000in}{0.000000in}}{%
\pgfpathmoveto{\pgfqpoint{0.000000in}{0.000000in}}%
\pgfpathlineto{\pgfqpoint{0.000000in}{-0.013889in}}%
\pgfusepath{stroke,fill}%
}%
\begin{pgfscope}%
\pgfsys@transformshift{1.415538in}{0.638889in}%
\pgfsys@useobject{currentmarker}{}%
\end{pgfscope}%
\end{pgfscope}%
\begin{pgfscope}%
\definecolor{textcolor}{rgb}{0.000000,0.000000,0.000000}%
\pgfsetstrokecolor{textcolor}%
\pgfsetfillcolor{textcolor}%
\pgftext[x=1.415538in,y=0.576389in,,top]{\color{textcolor}{\ifdefined\pdftexversion\else\setmainfont{NanumMyeongjo}\rmfamily\fi\fontsize{9.000000}{10.800000}\selectfont\catcode`\^=\active\def^{\ifmmode\sp\else\^{}\fi}\catcode`\%=\active\def%{\%}2월}}%
\end{pgfscope}%
\begin{pgfscope}%
\pgfsetbuttcap%
\pgfsetroundjoin%
\definecolor{currentfill}{rgb}{0.000000,0.000000,0.000000}%
\pgfsetfillcolor{currentfill}%
\pgfsetlinewidth{0.752812pt}%
\definecolor{currentstroke}{rgb}{0.000000,0.000000,0.000000}%
\pgfsetstrokecolor{currentstroke}%
\pgfsetdash{}{0pt}%
\pgfsys@defobject{currentmarker}{\pgfqpoint{0.000000in}{-0.013889in}}{\pgfqpoint{0.000000in}{0.000000in}}{%
\pgfpathmoveto{\pgfqpoint{0.000000in}{0.000000in}}%
\pgfpathlineto{\pgfqpoint{0.000000in}{-0.013889in}}%
\pgfusepath{stroke,fill}%
}%
\begin{pgfscope}%
\pgfsys@transformshift{1.743335in}{0.638889in}%
\pgfsys@useobject{currentmarker}{}%
\end{pgfscope}%
\end{pgfscope}%
\begin{pgfscope}%
\definecolor{textcolor}{rgb}{0.000000,0.000000,0.000000}%
\pgfsetstrokecolor{textcolor}%
\pgfsetfillcolor{textcolor}%
\pgftext[x=1.743335in,y=0.576389in,,top]{\color{textcolor}{\ifdefined\pdftexversion\else\setmainfont{NanumMyeongjo}\rmfamily\fi\fontsize{9.000000}{10.800000}\selectfont\catcode`\^=\active\def^{\ifmmode\sp\else\^{}\fi}\catcode`\%=\active\def%{\%}3월}}%
\end{pgfscope}%
\begin{pgfscope}%
\pgfsetbuttcap%
\pgfsetroundjoin%
\definecolor{currentfill}{rgb}{0.000000,0.000000,0.000000}%
\pgfsetfillcolor{currentfill}%
\pgfsetlinewidth{0.752812pt}%
\definecolor{currentstroke}{rgb}{0.000000,0.000000,0.000000}%
\pgfsetstrokecolor{currentstroke}%
\pgfsetdash{}{0pt}%
\pgfsys@defobject{currentmarker}{\pgfqpoint{0.000000in}{-0.013889in}}{\pgfqpoint{0.000000in}{0.000000in}}{%
\pgfpathmoveto{\pgfqpoint{0.000000in}{0.000000in}}%
\pgfpathlineto{\pgfqpoint{0.000000in}{-0.013889in}}%
\pgfusepath{stroke,fill}%
}%
\begin{pgfscope}%
\pgfsys@transformshift{2.071132in}{0.638889in}%
\pgfsys@useobject{currentmarker}{}%
\end{pgfscope}%
\end{pgfscope}%
\begin{pgfscope}%
\definecolor{textcolor}{rgb}{0.000000,0.000000,0.000000}%
\pgfsetstrokecolor{textcolor}%
\pgfsetfillcolor{textcolor}%
\pgftext[x=2.071132in,y=0.576389in,,top]{\color{textcolor}{\ifdefined\pdftexversion\else\setmainfont{NanumMyeongjo}\rmfamily\fi\fontsize{9.000000}{10.800000}\selectfont\catcode`\^=\active\def^{\ifmmode\sp\else\^{}\fi}\catcode`\%=\active\def%{\%}4월}}%
\end{pgfscope}%
\begin{pgfscope}%
\pgfsetbuttcap%
\pgfsetroundjoin%
\definecolor{currentfill}{rgb}{0.000000,0.000000,0.000000}%
\pgfsetfillcolor{currentfill}%
\pgfsetlinewidth{0.752812pt}%
\definecolor{currentstroke}{rgb}{0.000000,0.000000,0.000000}%
\pgfsetstrokecolor{currentstroke}%
\pgfsetdash{}{0pt}%
\pgfsys@defobject{currentmarker}{\pgfqpoint{0.000000in}{-0.013889in}}{\pgfqpoint{0.000000in}{0.000000in}}{%
\pgfpathmoveto{\pgfqpoint{0.000000in}{0.000000in}}%
\pgfpathlineto{\pgfqpoint{0.000000in}{-0.013889in}}%
\pgfusepath{stroke,fill}%
}%
\begin{pgfscope}%
\pgfsys@transformshift{2.398929in}{0.638889in}%
\pgfsys@useobject{currentmarker}{}%
\end{pgfscope}%
\end{pgfscope}%
\begin{pgfscope}%
\definecolor{textcolor}{rgb}{0.000000,0.000000,0.000000}%
\pgfsetstrokecolor{textcolor}%
\pgfsetfillcolor{textcolor}%
\pgftext[x=2.398929in,y=0.576389in,,top]{\color{textcolor}{\ifdefined\pdftexversion\else\setmainfont{NanumMyeongjo}\rmfamily\fi\fontsize{9.000000}{10.800000}\selectfont\catcode`\^=\active\def^{\ifmmode\sp\else\^{}\fi}\catcode`\%=\active\def%{\%}5월}}%
\end{pgfscope}%
\begin{pgfscope}%
\pgfsetbuttcap%
\pgfsetroundjoin%
\definecolor{currentfill}{rgb}{0.000000,0.000000,0.000000}%
\pgfsetfillcolor{currentfill}%
\pgfsetlinewidth{0.752812pt}%
\definecolor{currentstroke}{rgb}{0.000000,0.000000,0.000000}%
\pgfsetstrokecolor{currentstroke}%
\pgfsetdash{}{0pt}%
\pgfsys@defobject{currentmarker}{\pgfqpoint{0.000000in}{-0.013889in}}{\pgfqpoint{0.000000in}{0.000000in}}{%
\pgfpathmoveto{\pgfqpoint{0.000000in}{0.000000in}}%
\pgfpathlineto{\pgfqpoint{0.000000in}{-0.013889in}}%
\pgfusepath{stroke,fill}%
}%
\begin{pgfscope}%
\pgfsys@transformshift{2.726726in}{0.638889in}%
\pgfsys@useobject{currentmarker}{}%
\end{pgfscope}%
\end{pgfscope}%
\begin{pgfscope}%
\definecolor{textcolor}{rgb}{0.000000,0.000000,0.000000}%
\pgfsetstrokecolor{textcolor}%
\pgfsetfillcolor{textcolor}%
\pgftext[x=2.726726in,y=0.576389in,,top]{\color{textcolor}{\ifdefined\pdftexversion\else\setmainfont{NanumMyeongjo}\rmfamily\fi\fontsize{9.000000}{10.800000}\selectfont\catcode`\^=\active\def^{\ifmmode\sp\else\^{}\fi}\catcode`\%=\active\def%{\%}6월}}%
\end{pgfscope}%
\begin{pgfscope}%
\pgfsetbuttcap%
\pgfsetroundjoin%
\definecolor{currentfill}{rgb}{0.000000,0.000000,0.000000}%
\pgfsetfillcolor{currentfill}%
\pgfsetlinewidth{0.752812pt}%
\definecolor{currentstroke}{rgb}{0.000000,0.000000,0.000000}%
\pgfsetstrokecolor{currentstroke}%
\pgfsetdash{}{0pt}%
\pgfsys@defobject{currentmarker}{\pgfqpoint{0.000000in}{-0.013889in}}{\pgfqpoint{0.000000in}{0.000000in}}{%
\pgfpathmoveto{\pgfqpoint{0.000000in}{0.000000in}}%
\pgfpathlineto{\pgfqpoint{0.000000in}{-0.013889in}}%
\pgfusepath{stroke,fill}%
}%
\begin{pgfscope}%
\pgfsys@transformshift{3.054524in}{0.638889in}%
\pgfsys@useobject{currentmarker}{}%
\end{pgfscope}%
\end{pgfscope}%
\begin{pgfscope}%
\definecolor{textcolor}{rgb}{0.000000,0.000000,0.000000}%
\pgfsetstrokecolor{textcolor}%
\pgfsetfillcolor{textcolor}%
\pgftext[x=3.054524in,y=0.576389in,,top]{\color{textcolor}{\ifdefined\pdftexversion\else\setmainfont{NanumMyeongjo}\rmfamily\fi\fontsize{9.000000}{10.800000}\selectfont\catcode`\^=\active\def^{\ifmmode\sp\else\^{}\fi}\catcode`\%=\active\def%{\%}7월}}%
\end{pgfscope}%
\begin{pgfscope}%
\pgfsetbuttcap%
\pgfsetroundjoin%
\definecolor{currentfill}{rgb}{0.000000,0.000000,0.000000}%
\pgfsetfillcolor{currentfill}%
\pgfsetlinewidth{0.752812pt}%
\definecolor{currentstroke}{rgb}{0.000000,0.000000,0.000000}%
\pgfsetstrokecolor{currentstroke}%
\pgfsetdash{}{0pt}%
\pgfsys@defobject{currentmarker}{\pgfqpoint{0.000000in}{-0.013889in}}{\pgfqpoint{0.000000in}{0.000000in}}{%
\pgfpathmoveto{\pgfqpoint{0.000000in}{0.000000in}}%
\pgfpathlineto{\pgfqpoint{0.000000in}{-0.013889in}}%
\pgfusepath{stroke,fill}%
}%
\begin{pgfscope}%
\pgfsys@transformshift{3.382321in}{0.638889in}%
\pgfsys@useobject{currentmarker}{}%
\end{pgfscope}%
\end{pgfscope}%
\begin{pgfscope}%
\definecolor{textcolor}{rgb}{0.000000,0.000000,0.000000}%
\pgfsetstrokecolor{textcolor}%
\pgfsetfillcolor{textcolor}%
\pgftext[x=3.382321in,y=0.576389in,,top]{\color{textcolor}{\ifdefined\pdftexversion\else\setmainfont{NanumMyeongjo}\rmfamily\fi\fontsize{9.000000}{10.800000}\selectfont\catcode`\^=\active\def^{\ifmmode\sp\else\^{}\fi}\catcode`\%=\active\def%{\%}8월}}%
\end{pgfscope}%
\begin{pgfscope}%
\pgfsetbuttcap%
\pgfsetroundjoin%
\definecolor{currentfill}{rgb}{0.000000,0.000000,0.000000}%
\pgfsetfillcolor{currentfill}%
\pgfsetlinewidth{0.752812pt}%
\definecolor{currentstroke}{rgb}{0.000000,0.000000,0.000000}%
\pgfsetstrokecolor{currentstroke}%
\pgfsetdash{}{0pt}%
\pgfsys@defobject{currentmarker}{\pgfqpoint{0.000000in}{-0.013889in}}{\pgfqpoint{0.000000in}{0.000000in}}{%
\pgfpathmoveto{\pgfqpoint{0.000000in}{0.000000in}}%
\pgfpathlineto{\pgfqpoint{0.000000in}{-0.013889in}}%
\pgfusepath{stroke,fill}%
}%
\begin{pgfscope}%
\pgfsys@transformshift{3.710118in}{0.638889in}%
\pgfsys@useobject{currentmarker}{}%
\end{pgfscope}%
\end{pgfscope}%
\begin{pgfscope}%
\definecolor{textcolor}{rgb}{0.000000,0.000000,0.000000}%
\pgfsetstrokecolor{textcolor}%
\pgfsetfillcolor{textcolor}%
\pgftext[x=3.710118in,y=0.576389in,,top]{\color{textcolor}{\ifdefined\pdftexversion\else\setmainfont{NanumMyeongjo}\rmfamily\fi\fontsize{9.000000}{10.800000}\selectfont\catcode`\^=\active\def^{\ifmmode\sp\else\^{}\fi}\catcode`\%=\active\def%{\%}9월}}%
\end{pgfscope}%
\begin{pgfscope}%
\pgfsetbuttcap%
\pgfsetroundjoin%
\definecolor{currentfill}{rgb}{0.000000,0.000000,0.000000}%
\pgfsetfillcolor{currentfill}%
\pgfsetlinewidth{0.752812pt}%
\definecolor{currentstroke}{rgb}{0.000000,0.000000,0.000000}%
\pgfsetstrokecolor{currentstroke}%
\pgfsetdash{}{0pt}%
\pgfsys@defobject{currentmarker}{\pgfqpoint{0.000000in}{-0.013889in}}{\pgfqpoint{0.000000in}{0.000000in}}{%
\pgfpathmoveto{\pgfqpoint{0.000000in}{0.000000in}}%
\pgfpathlineto{\pgfqpoint{0.000000in}{-0.013889in}}%
\pgfusepath{stroke,fill}%
}%
\begin{pgfscope}%
\pgfsys@transformshift{4.037915in}{0.638889in}%
\pgfsys@useobject{currentmarker}{}%
\end{pgfscope}%
\end{pgfscope}%
\begin{pgfscope}%
\definecolor{textcolor}{rgb}{0.000000,0.000000,0.000000}%
\pgfsetstrokecolor{textcolor}%
\pgfsetfillcolor{textcolor}%
\pgftext[x=4.037915in,y=0.576389in,,top]{\color{textcolor}{\ifdefined\pdftexversion\else\setmainfont{NanumMyeongjo}\rmfamily\fi\fontsize{9.000000}{10.800000}\selectfont\catcode`\^=\active\def^{\ifmmode\sp\else\^{}\fi}\catcode`\%=\active\def%{\%}10월}}%
\end{pgfscope}%
\begin{pgfscope}%
\pgfsetbuttcap%
\pgfsetroundjoin%
\definecolor{currentfill}{rgb}{0.000000,0.000000,0.000000}%
\pgfsetfillcolor{currentfill}%
\pgfsetlinewidth{0.752812pt}%
\definecolor{currentstroke}{rgb}{0.000000,0.000000,0.000000}%
\pgfsetstrokecolor{currentstroke}%
\pgfsetdash{}{0pt}%
\pgfsys@defobject{currentmarker}{\pgfqpoint{0.000000in}{-0.013889in}}{\pgfqpoint{0.000000in}{0.000000in}}{%
\pgfpathmoveto{\pgfqpoint{0.000000in}{0.000000in}}%
\pgfpathlineto{\pgfqpoint{0.000000in}{-0.013889in}}%
\pgfusepath{stroke,fill}%
}%
\begin{pgfscope}%
\pgfsys@transformshift{4.365712in}{0.638889in}%
\pgfsys@useobject{currentmarker}{}%
\end{pgfscope}%
\end{pgfscope}%
\begin{pgfscope}%
\definecolor{textcolor}{rgb}{0.000000,0.000000,0.000000}%
\pgfsetstrokecolor{textcolor}%
\pgfsetfillcolor{textcolor}%
\pgftext[x=4.365712in,y=0.576389in,,top]{\color{textcolor}{\ifdefined\pdftexversion\else\setmainfont{NanumMyeongjo}\rmfamily\fi\fontsize{9.000000}{10.800000}\selectfont\catcode`\^=\active\def^{\ifmmode\sp\else\^{}\fi}\catcode`\%=\active\def%{\%}11월}}%
\end{pgfscope}%
\begin{pgfscope}%
\pgfsetbuttcap%
\pgfsetroundjoin%
\definecolor{currentfill}{rgb}{0.000000,0.000000,0.000000}%
\pgfsetfillcolor{currentfill}%
\pgfsetlinewidth{0.752812pt}%
\definecolor{currentstroke}{rgb}{0.000000,0.000000,0.000000}%
\pgfsetstrokecolor{currentstroke}%
\pgfsetdash{}{0pt}%
\pgfsys@defobject{currentmarker}{\pgfqpoint{0.000000in}{-0.013889in}}{\pgfqpoint{0.000000in}{0.000000in}}{%
\pgfpathmoveto{\pgfqpoint{0.000000in}{0.000000in}}%
\pgfpathlineto{\pgfqpoint{0.000000in}{-0.013889in}}%
\pgfusepath{stroke,fill}%
}%
\begin{pgfscope}%
\pgfsys@transformshift{4.693510in}{0.638889in}%
\pgfsys@useobject{currentmarker}{}%
\end{pgfscope}%
\end{pgfscope}%
\begin{pgfscope}%
\definecolor{textcolor}{rgb}{0.000000,0.000000,0.000000}%
\pgfsetstrokecolor{textcolor}%
\pgfsetfillcolor{textcolor}%
\pgftext[x=4.693510in,y=0.576389in,,top]{\color{textcolor}{\ifdefined\pdftexversion\else\setmainfont{NanumMyeongjo}\rmfamily\fi\fontsize{9.000000}{10.800000}\selectfont\catcode`\^=\active\def^{\ifmmode\sp\else\^{}\fi}\catcode`\%=\active\def%{\%}12월}}%
\end{pgfscope}%
\begin{pgfscope}%
\pgfpathrectangle{\pgfqpoint{0.781250in}{0.638889in}}{\pgfqpoint{4.218750in}{2.172222in}}%
\pgfusepath{clip}%
\pgfsetbuttcap%
\pgfsetroundjoin%
\pgfsetlinewidth{0.602250pt}%
\definecolor{currentstroke}{rgb}{0.690196,0.690196,0.690196}%
\pgfsetstrokecolor{currentstroke}%
\pgfsetstrokeopacity{0.450000}%
\pgfsetdash{{2.220000pt}{0.960000pt}}{0.000000pt}%
\pgfpathmoveto{\pgfqpoint{0.781250in}{0.638889in}}%
\pgfpathlineto{\pgfqpoint{5.000000in}{0.638889in}}%
\pgfusepath{stroke}%
\end{pgfscope}%
\begin{pgfscope}%
\pgfsetbuttcap%
\pgfsetroundjoin%
\definecolor{currentfill}{rgb}{0.000000,0.000000,0.000000}%
\pgfsetfillcolor{currentfill}%
\pgfsetlinewidth{0.752812pt}%
\definecolor{currentstroke}{rgb}{0.000000,0.000000,0.000000}%
\pgfsetstrokecolor{currentstroke}%
\pgfsetdash{}{0pt}%
\pgfsys@defobject{currentmarker}{\pgfqpoint{-0.013889in}{0.000000in}}{\pgfqpoint{-0.000000in}{0.000000in}}{%
\pgfpathmoveto{\pgfqpoint{-0.000000in}{0.000000in}}%
\pgfpathlineto{\pgfqpoint{-0.013889in}{0.000000in}}%
\pgfusepath{stroke,fill}%
}%
\begin{pgfscope}%
\pgfsys@transformshift{0.781250in}{0.638889in}%
\pgfsys@useobject{currentmarker}{}%
\end{pgfscope}%
\end{pgfscope}%
\begin{pgfscope}%
\definecolor{textcolor}{rgb}{0.000000,0.000000,0.000000}%
\pgfsetstrokecolor{textcolor}%
\pgfsetfillcolor{textcolor}%
\pgftext[x=0.651611in, y=0.588962in, left, base]{\color{textcolor}{\ifdefined\pdftexversion\else\setmainfont{NanumMyeongjo}\rmfamily\fi\fontsize{9.000000}{10.800000}\selectfont\catcode`\^=\active\def^{\ifmmode\sp\else\^{}\fi}\catcode`\%=\active\def%{\%}0}}%
\end{pgfscope}%
\begin{pgfscope}%
\pgfpathrectangle{\pgfqpoint{0.781250in}{0.638889in}}{\pgfqpoint{4.218750in}{2.172222in}}%
\pgfusepath{clip}%
\pgfsetbuttcap%
\pgfsetroundjoin%
\pgfsetlinewidth{0.602250pt}%
\definecolor{currentstroke}{rgb}{0.690196,0.690196,0.690196}%
\pgfsetstrokecolor{currentstroke}%
\pgfsetstrokeopacity{0.450000}%
\pgfsetdash{{2.220000pt}{0.960000pt}}{0.000000pt}%
\pgfpathmoveto{\pgfqpoint{0.781250in}{0.910417in}}%
\pgfpathlineto{\pgfqpoint{5.000000in}{0.910417in}}%
\pgfusepath{stroke}%
\end{pgfscope}%
\begin{pgfscope}%
\pgfsetbuttcap%
\pgfsetroundjoin%
\definecolor{currentfill}{rgb}{0.000000,0.000000,0.000000}%
\pgfsetfillcolor{currentfill}%
\pgfsetlinewidth{0.752812pt}%
\definecolor{currentstroke}{rgb}{0.000000,0.000000,0.000000}%
\pgfsetstrokecolor{currentstroke}%
\pgfsetdash{}{0pt}%
\pgfsys@defobject{currentmarker}{\pgfqpoint{-0.013889in}{0.000000in}}{\pgfqpoint{-0.000000in}{0.000000in}}{%
\pgfpathmoveto{\pgfqpoint{-0.000000in}{0.000000in}}%
\pgfpathlineto{\pgfqpoint{-0.013889in}{0.000000in}}%
\pgfusepath{stroke,fill}%
}%
\begin{pgfscope}%
\pgfsys@transformshift{0.781250in}{0.910417in}%
\pgfsys@useobject{currentmarker}{}%
\end{pgfscope}%
\end{pgfscope}%
\begin{pgfscope}%
\definecolor{textcolor}{rgb}{0.000000,0.000000,0.000000}%
\pgfsetstrokecolor{textcolor}%
\pgfsetfillcolor{textcolor}%
\pgftext[x=0.584473in, y=0.860490in, left, base]{\color{textcolor}{\ifdefined\pdftexversion\else\setmainfont{NanumMyeongjo}\rmfamily\fi\fontsize{9.000000}{10.800000}\selectfont\catcode`\^=\active\def^{\ifmmode\sp\else\^{}\fi}\catcode`\%=\active\def%{\%}50}}%
\end{pgfscope}%
\begin{pgfscope}%
\pgfpathrectangle{\pgfqpoint{0.781250in}{0.638889in}}{\pgfqpoint{4.218750in}{2.172222in}}%
\pgfusepath{clip}%
\pgfsetbuttcap%
\pgfsetroundjoin%
\pgfsetlinewidth{0.602250pt}%
\definecolor{currentstroke}{rgb}{0.690196,0.690196,0.690196}%
\pgfsetstrokecolor{currentstroke}%
\pgfsetstrokeopacity{0.450000}%
\pgfsetdash{{2.220000pt}{0.960000pt}}{0.000000pt}%
\pgfpathmoveto{\pgfqpoint{0.781250in}{1.181944in}}%
\pgfpathlineto{\pgfqpoint{5.000000in}{1.181944in}}%
\pgfusepath{stroke}%
\end{pgfscope}%
\begin{pgfscope}%
\pgfsetbuttcap%
\pgfsetroundjoin%
\definecolor{currentfill}{rgb}{0.000000,0.000000,0.000000}%
\pgfsetfillcolor{currentfill}%
\pgfsetlinewidth{0.752812pt}%
\definecolor{currentstroke}{rgb}{0.000000,0.000000,0.000000}%
\pgfsetstrokecolor{currentstroke}%
\pgfsetdash{}{0pt}%
\pgfsys@defobject{currentmarker}{\pgfqpoint{-0.013889in}{0.000000in}}{\pgfqpoint{-0.000000in}{0.000000in}}{%
\pgfpathmoveto{\pgfqpoint{-0.000000in}{0.000000in}}%
\pgfpathlineto{\pgfqpoint{-0.013889in}{0.000000in}}%
\pgfusepath{stroke,fill}%
}%
\begin{pgfscope}%
\pgfsys@transformshift{0.781250in}{1.181944in}%
\pgfsys@useobject{currentmarker}{}%
\end{pgfscope}%
\end{pgfscope}%
\begin{pgfscope}%
\definecolor{textcolor}{rgb}{0.000000,0.000000,0.000000}%
\pgfsetstrokecolor{textcolor}%
\pgfsetfillcolor{textcolor}%
\pgftext[x=0.517334in, y=1.132018in, left, base]{\color{textcolor}{\ifdefined\pdftexversion\else\setmainfont{NanumMyeongjo}\rmfamily\fi\fontsize{9.000000}{10.800000}\selectfont\catcode`\^=\active\def^{\ifmmode\sp\else\^{}\fi}\catcode`\%=\active\def%{\%}100}}%
\end{pgfscope}%
\begin{pgfscope}%
\pgfpathrectangle{\pgfqpoint{0.781250in}{0.638889in}}{\pgfqpoint{4.218750in}{2.172222in}}%
\pgfusepath{clip}%
\pgfsetbuttcap%
\pgfsetroundjoin%
\pgfsetlinewidth{0.602250pt}%
\definecolor{currentstroke}{rgb}{0.690196,0.690196,0.690196}%
\pgfsetstrokecolor{currentstroke}%
\pgfsetstrokeopacity{0.450000}%
\pgfsetdash{{2.220000pt}{0.960000pt}}{0.000000pt}%
\pgfpathmoveto{\pgfqpoint{0.781250in}{1.453472in}}%
\pgfpathlineto{\pgfqpoint{5.000000in}{1.453472in}}%
\pgfusepath{stroke}%
\end{pgfscope}%
\begin{pgfscope}%
\pgfsetbuttcap%
\pgfsetroundjoin%
\definecolor{currentfill}{rgb}{0.000000,0.000000,0.000000}%
\pgfsetfillcolor{currentfill}%
\pgfsetlinewidth{0.752812pt}%
\definecolor{currentstroke}{rgb}{0.000000,0.000000,0.000000}%
\pgfsetstrokecolor{currentstroke}%
\pgfsetdash{}{0pt}%
\pgfsys@defobject{currentmarker}{\pgfqpoint{-0.013889in}{0.000000in}}{\pgfqpoint{-0.000000in}{0.000000in}}{%
\pgfpathmoveto{\pgfqpoint{-0.000000in}{0.000000in}}%
\pgfpathlineto{\pgfqpoint{-0.013889in}{0.000000in}}%
\pgfusepath{stroke,fill}%
}%
\begin{pgfscope}%
\pgfsys@transformshift{0.781250in}{1.453472in}%
\pgfsys@useobject{currentmarker}{}%
\end{pgfscope}%
\end{pgfscope}%
\begin{pgfscope}%
\definecolor{textcolor}{rgb}{0.000000,0.000000,0.000000}%
\pgfsetstrokecolor{textcolor}%
\pgfsetfillcolor{textcolor}%
\pgftext[x=0.517334in, y=1.403545in, left, base]{\color{textcolor}{\ifdefined\pdftexversion\else\setmainfont{NanumMyeongjo}\rmfamily\fi\fontsize{9.000000}{10.800000}\selectfont\catcode`\^=\active\def^{\ifmmode\sp\else\^{}\fi}\catcode`\%=\active\def%{\%}150}}%
\end{pgfscope}%
\begin{pgfscope}%
\pgfpathrectangle{\pgfqpoint{0.781250in}{0.638889in}}{\pgfqpoint{4.218750in}{2.172222in}}%
\pgfusepath{clip}%
\pgfsetbuttcap%
\pgfsetroundjoin%
\pgfsetlinewidth{0.602250pt}%
\definecolor{currentstroke}{rgb}{0.690196,0.690196,0.690196}%
\pgfsetstrokecolor{currentstroke}%
\pgfsetstrokeopacity{0.450000}%
\pgfsetdash{{2.220000pt}{0.960000pt}}{0.000000pt}%
\pgfpathmoveto{\pgfqpoint{0.781250in}{1.725000in}}%
\pgfpathlineto{\pgfqpoint{5.000000in}{1.725000in}}%
\pgfusepath{stroke}%
\end{pgfscope}%
\begin{pgfscope}%
\pgfsetbuttcap%
\pgfsetroundjoin%
\definecolor{currentfill}{rgb}{0.000000,0.000000,0.000000}%
\pgfsetfillcolor{currentfill}%
\pgfsetlinewidth{0.752812pt}%
\definecolor{currentstroke}{rgb}{0.000000,0.000000,0.000000}%
\pgfsetstrokecolor{currentstroke}%
\pgfsetdash{}{0pt}%
\pgfsys@defobject{currentmarker}{\pgfqpoint{-0.013889in}{0.000000in}}{\pgfqpoint{-0.000000in}{0.000000in}}{%
\pgfpathmoveto{\pgfqpoint{-0.000000in}{0.000000in}}%
\pgfpathlineto{\pgfqpoint{-0.013889in}{0.000000in}}%
\pgfusepath{stroke,fill}%
}%
\begin{pgfscope}%
\pgfsys@transformshift{0.781250in}{1.725000in}%
\pgfsys@useobject{currentmarker}{}%
\end{pgfscope}%
\end{pgfscope}%
\begin{pgfscope}%
\definecolor{textcolor}{rgb}{0.000000,0.000000,0.000000}%
\pgfsetstrokecolor{textcolor}%
\pgfsetfillcolor{textcolor}%
\pgftext[x=0.517334in, y=1.675073in, left, base]{\color{textcolor}{\ifdefined\pdftexversion\else\setmainfont{NanumMyeongjo}\rmfamily\fi\fontsize{9.000000}{10.800000}\selectfont\catcode`\^=\active\def^{\ifmmode\sp\else\^{}\fi}\catcode`\%=\active\def%{\%}200}}%
\end{pgfscope}%
\begin{pgfscope}%
\pgfpathrectangle{\pgfqpoint{0.781250in}{0.638889in}}{\pgfqpoint{4.218750in}{2.172222in}}%
\pgfusepath{clip}%
\pgfsetbuttcap%
\pgfsetroundjoin%
\pgfsetlinewidth{0.602250pt}%
\definecolor{currentstroke}{rgb}{0.690196,0.690196,0.690196}%
\pgfsetstrokecolor{currentstroke}%
\pgfsetstrokeopacity{0.450000}%
\pgfsetdash{{2.220000pt}{0.960000pt}}{0.000000pt}%
\pgfpathmoveto{\pgfqpoint{0.781250in}{1.996528in}}%
\pgfpathlineto{\pgfqpoint{5.000000in}{1.996528in}}%
\pgfusepath{stroke}%
\end{pgfscope}%
\begin{pgfscope}%
\pgfsetbuttcap%
\pgfsetroundjoin%
\definecolor{currentfill}{rgb}{0.000000,0.000000,0.000000}%
\pgfsetfillcolor{currentfill}%
\pgfsetlinewidth{0.752812pt}%
\definecolor{currentstroke}{rgb}{0.000000,0.000000,0.000000}%
\pgfsetstrokecolor{currentstroke}%
\pgfsetdash{}{0pt}%
\pgfsys@defobject{currentmarker}{\pgfqpoint{-0.013889in}{0.000000in}}{\pgfqpoint{-0.000000in}{0.000000in}}{%
\pgfpathmoveto{\pgfqpoint{-0.000000in}{0.000000in}}%
\pgfpathlineto{\pgfqpoint{-0.013889in}{0.000000in}}%
\pgfusepath{stroke,fill}%
}%
\begin{pgfscope}%
\pgfsys@transformshift{0.781250in}{1.996528in}%
\pgfsys@useobject{currentmarker}{}%
\end{pgfscope}%
\end{pgfscope}%
\begin{pgfscope}%
\definecolor{textcolor}{rgb}{0.000000,0.000000,0.000000}%
\pgfsetstrokecolor{textcolor}%
\pgfsetfillcolor{textcolor}%
\pgftext[x=0.517334in, y=1.946601in, left, base]{\color{textcolor}{\ifdefined\pdftexversion\else\setmainfont{NanumMyeongjo}\rmfamily\fi\fontsize{9.000000}{10.800000}\selectfont\catcode`\^=\active\def^{\ifmmode\sp\else\^{}\fi}\catcode`\%=\active\def%{\%}250}}%
\end{pgfscope}%
\begin{pgfscope}%
\pgfpathrectangle{\pgfqpoint{0.781250in}{0.638889in}}{\pgfqpoint{4.218750in}{2.172222in}}%
\pgfusepath{clip}%
\pgfsetbuttcap%
\pgfsetroundjoin%
\pgfsetlinewidth{0.602250pt}%
\definecolor{currentstroke}{rgb}{0.690196,0.690196,0.690196}%
\pgfsetstrokecolor{currentstroke}%
\pgfsetstrokeopacity{0.450000}%
\pgfsetdash{{2.220000pt}{0.960000pt}}{0.000000pt}%
\pgfpathmoveto{\pgfqpoint{0.781250in}{2.268056in}}%
\pgfpathlineto{\pgfqpoint{5.000000in}{2.268056in}}%
\pgfusepath{stroke}%
\end{pgfscope}%
\begin{pgfscope}%
\pgfsetbuttcap%
\pgfsetroundjoin%
\definecolor{currentfill}{rgb}{0.000000,0.000000,0.000000}%
\pgfsetfillcolor{currentfill}%
\pgfsetlinewidth{0.752812pt}%
\definecolor{currentstroke}{rgb}{0.000000,0.000000,0.000000}%
\pgfsetstrokecolor{currentstroke}%
\pgfsetdash{}{0pt}%
\pgfsys@defobject{currentmarker}{\pgfqpoint{-0.013889in}{0.000000in}}{\pgfqpoint{-0.000000in}{0.000000in}}{%
\pgfpathmoveto{\pgfqpoint{-0.000000in}{0.000000in}}%
\pgfpathlineto{\pgfqpoint{-0.013889in}{0.000000in}}%
\pgfusepath{stroke,fill}%
}%
\begin{pgfscope}%
\pgfsys@transformshift{0.781250in}{2.268056in}%
\pgfsys@useobject{currentmarker}{}%
\end{pgfscope}%
\end{pgfscope}%
\begin{pgfscope}%
\definecolor{textcolor}{rgb}{0.000000,0.000000,0.000000}%
\pgfsetstrokecolor{textcolor}%
\pgfsetfillcolor{textcolor}%
\pgftext[x=0.517334in, y=2.218129in, left, base]{\color{textcolor}{\ifdefined\pdftexversion\else\setmainfont{NanumMyeongjo}\rmfamily\fi\fontsize{9.000000}{10.800000}\selectfont\catcode`\^=\active\def^{\ifmmode\sp\else\^{}\fi}\catcode`\%=\active\def%{\%}300}}%
\end{pgfscope}%
\begin{pgfscope}%
\pgfpathrectangle{\pgfqpoint{0.781250in}{0.638889in}}{\pgfqpoint{4.218750in}{2.172222in}}%
\pgfusepath{clip}%
\pgfsetbuttcap%
\pgfsetroundjoin%
\pgfsetlinewidth{0.602250pt}%
\definecolor{currentstroke}{rgb}{0.690196,0.690196,0.690196}%
\pgfsetstrokecolor{currentstroke}%
\pgfsetstrokeopacity{0.450000}%
\pgfsetdash{{2.220000pt}{0.960000pt}}{0.000000pt}%
\pgfpathmoveto{\pgfqpoint{0.781250in}{2.539583in}}%
\pgfpathlineto{\pgfqpoint{5.000000in}{2.539583in}}%
\pgfusepath{stroke}%
\end{pgfscope}%
\begin{pgfscope}%
\pgfsetbuttcap%
\pgfsetroundjoin%
\definecolor{currentfill}{rgb}{0.000000,0.000000,0.000000}%
\pgfsetfillcolor{currentfill}%
\pgfsetlinewidth{0.752812pt}%
\definecolor{currentstroke}{rgb}{0.000000,0.000000,0.000000}%
\pgfsetstrokecolor{currentstroke}%
\pgfsetdash{}{0pt}%
\pgfsys@defobject{currentmarker}{\pgfqpoint{-0.013889in}{0.000000in}}{\pgfqpoint{-0.000000in}{0.000000in}}{%
\pgfpathmoveto{\pgfqpoint{-0.000000in}{0.000000in}}%
\pgfpathlineto{\pgfqpoint{-0.013889in}{0.000000in}}%
\pgfusepath{stroke,fill}%
}%
\begin{pgfscope}%
\pgfsys@transformshift{0.781250in}{2.539583in}%
\pgfsys@useobject{currentmarker}{}%
\end{pgfscope}%
\end{pgfscope}%
\begin{pgfscope}%
\definecolor{textcolor}{rgb}{0.000000,0.000000,0.000000}%
\pgfsetstrokecolor{textcolor}%
\pgfsetfillcolor{textcolor}%
\pgftext[x=0.517334in, y=2.489657in, left, base]{\color{textcolor}{\ifdefined\pdftexversion\else\setmainfont{NanumMyeongjo}\rmfamily\fi\fontsize{9.000000}{10.800000}\selectfont\catcode`\^=\active\def^{\ifmmode\sp\else\^{}\fi}\catcode`\%=\active\def%{\%}350}}%
\end{pgfscope}%
\begin{pgfscope}%
\pgfpathrectangle{\pgfqpoint{0.781250in}{0.638889in}}{\pgfqpoint{4.218750in}{2.172222in}}%
\pgfusepath{clip}%
\pgfsetbuttcap%
\pgfsetroundjoin%
\pgfsetlinewidth{0.602250pt}%
\definecolor{currentstroke}{rgb}{0.690196,0.690196,0.690196}%
\pgfsetstrokecolor{currentstroke}%
\pgfsetstrokeopacity{0.450000}%
\pgfsetdash{{2.220000pt}{0.960000pt}}{0.000000pt}%
\pgfpathmoveto{\pgfqpoint{0.781250in}{2.811111in}}%
\pgfpathlineto{\pgfqpoint{5.000000in}{2.811111in}}%
\pgfusepath{stroke}%
\end{pgfscope}%
\begin{pgfscope}%
\pgfsetbuttcap%
\pgfsetroundjoin%
\definecolor{currentfill}{rgb}{0.000000,0.000000,0.000000}%
\pgfsetfillcolor{currentfill}%
\pgfsetlinewidth{0.752812pt}%
\definecolor{currentstroke}{rgb}{0.000000,0.000000,0.000000}%
\pgfsetstrokecolor{currentstroke}%
\pgfsetdash{}{0pt}%
\pgfsys@defobject{currentmarker}{\pgfqpoint{-0.013889in}{0.000000in}}{\pgfqpoint{-0.000000in}{0.000000in}}{%
\pgfpathmoveto{\pgfqpoint{-0.000000in}{0.000000in}}%
\pgfpathlineto{\pgfqpoint{-0.013889in}{0.000000in}}%
\pgfusepath{stroke,fill}%
}%
\begin{pgfscope}%
\pgfsys@transformshift{0.781250in}{2.811111in}%
\pgfsys@useobject{currentmarker}{}%
\end{pgfscope}%
\end{pgfscope}%
\begin{pgfscope}%
\definecolor{textcolor}{rgb}{0.000000,0.000000,0.000000}%
\pgfsetstrokecolor{textcolor}%
\pgfsetfillcolor{textcolor}%
\pgftext[x=0.517334in, y=2.761184in, left, base]{\color{textcolor}{\ifdefined\pdftexversion\else\setmainfont{NanumMyeongjo}\rmfamily\fi\fontsize{9.000000}{10.800000}\selectfont\catcode`\^=\active\def^{\ifmmode\sp\else\^{}\fi}\catcode`\%=\active\def%{\%}400}}%
\end{pgfscope}%
\begin{pgfscope}%
\pgfsetrectcap%
\pgfsetmiterjoin%
\pgfsetlinewidth{0.752812pt}%
\definecolor{currentstroke}{rgb}{0.000000,0.000000,0.000000}%
\pgfsetstrokecolor{currentstroke}%
\pgfsetdash{}{0pt}%
\pgfpathmoveto{\pgfqpoint{0.781250in}{0.638889in}}%
\pgfpathlineto{\pgfqpoint{0.781250in}{2.811111in}}%
\pgfusepath{stroke}%
\end{pgfscope}%
\begin{pgfscope}%
\pgfsetrectcap%
\pgfsetmiterjoin%
\pgfsetlinewidth{0.752812pt}%
\definecolor{currentstroke}{rgb}{0.000000,0.000000,0.000000}%
\pgfsetstrokecolor{currentstroke}%
\pgfsetdash{}{0pt}%
\pgfpathmoveto{\pgfqpoint{0.781250in}{0.638889in}}%
\pgfpathlineto{\pgfqpoint{5.000000in}{0.638889in}}%
\pgfusepath{stroke}%
\end{pgfscope}%
\begin{pgfscope}%
\pgfpathrectangle{\pgfqpoint{0.781250in}{0.638889in}}{\pgfqpoint{4.218750in}{2.172222in}}%
\pgfusepath{clip}%
\pgfsetbuttcap%
\pgfsetmiterjoin%
\definecolor{currentfill}{rgb}{0.227451,0.192157,0.427451}%
\pgfsetfillcolor{currentfill}%
\pgfsetlinewidth{1.003750pt}%
\definecolor{currentstroke}{rgb}{0.266667,0.266667,0.266667}%
\pgfsetstrokecolor{currentstroke}%
\pgfsetdash{}{0pt}%
\pgfpathmoveto{\pgfqpoint{0.973011in}{0.638889in}}%
\pgfpathlineto{\pgfqpoint{1.202469in}{0.638889in}}%
\pgfpathlineto{\pgfqpoint{1.202469in}{0.882178in}}%
\pgfpathlineto{\pgfqpoint{0.973011in}{0.882178in}}%
\pgfpathlineto{\pgfqpoint{0.973011in}{0.638889in}}%
\pgfpathclose%
\pgfusepath{stroke,fill}%
\end{pgfscope}%
\begin{pgfscope}%
\pgfpathrectangle{\pgfqpoint{0.781250in}{0.638889in}}{\pgfqpoint{4.218750in}{2.172222in}}%
\pgfusepath{clip}%
\pgfsetbuttcap%
\pgfsetmiterjoin%
\definecolor{currentfill}{rgb}{0.227451,0.192157,0.427451}%
\pgfsetfillcolor{currentfill}%
\pgfsetlinewidth{1.003750pt}%
\definecolor{currentstroke}{rgb}{0.266667,0.266667,0.266667}%
\pgfsetstrokecolor{currentstroke}%
\pgfsetdash{}{0pt}%
\pgfpathmoveto{\pgfqpoint{1.300809in}{0.638889in}}%
\pgfpathlineto{\pgfqpoint{1.530267in}{0.638889in}}%
\pgfpathlineto{\pgfqpoint{1.530267in}{0.832217in}}%
\pgfpathlineto{\pgfqpoint{1.300809in}{0.832217in}}%
\pgfpathlineto{\pgfqpoint{1.300809in}{0.638889in}}%
\pgfpathclose%
\pgfusepath{stroke,fill}%
\end{pgfscope}%
\begin{pgfscope}%
\pgfpathrectangle{\pgfqpoint{0.781250in}{0.638889in}}{\pgfqpoint{4.218750in}{2.172222in}}%
\pgfusepath{clip}%
\pgfsetbuttcap%
\pgfsetmiterjoin%
\definecolor{currentfill}{rgb}{0.227451,0.192157,0.427451}%
\pgfsetfillcolor{currentfill}%
\pgfsetlinewidth{1.003750pt}%
\definecolor{currentstroke}{rgb}{0.266667,0.266667,0.266667}%
\pgfsetstrokecolor{currentstroke}%
\pgfsetdash{}{0pt}%
\pgfpathmoveto{\pgfqpoint{1.628606in}{0.638889in}}%
\pgfpathlineto{\pgfqpoint{1.858064in}{0.638889in}}%
\pgfpathlineto{\pgfqpoint{1.858064in}{0.872403in}}%
\pgfpathlineto{\pgfqpoint{1.628606in}{0.872403in}}%
\pgfpathlineto{\pgfqpoint{1.628606in}{0.638889in}}%
\pgfpathclose%
\pgfusepath{stroke,fill}%
\end{pgfscope}%
\begin{pgfscope}%
\pgfpathrectangle{\pgfqpoint{0.781250in}{0.638889in}}{\pgfqpoint{4.218750in}{2.172222in}}%
\pgfusepath{clip}%
\pgfsetbuttcap%
\pgfsetmiterjoin%
\definecolor{currentfill}{rgb}{0.227451,0.192157,0.427451}%
\pgfsetfillcolor{currentfill}%
\pgfsetlinewidth{1.003750pt}%
\definecolor{currentstroke}{rgb}{0.266667,0.266667,0.266667}%
\pgfsetstrokecolor{currentstroke}%
\pgfsetdash{}{0pt}%
\pgfpathmoveto{\pgfqpoint{1.956403in}{0.638889in}}%
\pgfpathlineto{\pgfqpoint{2.185861in}{0.638889in}}%
\pgfpathlineto{\pgfqpoint{2.185861in}{0.978842in}}%
\pgfpathlineto{\pgfqpoint{1.956403in}{0.978842in}}%
\pgfpathlineto{\pgfqpoint{1.956403in}{0.638889in}}%
\pgfpathclose%
\pgfusepath{stroke,fill}%
\end{pgfscope}%
\begin{pgfscope}%
\pgfpathrectangle{\pgfqpoint{0.781250in}{0.638889in}}{\pgfqpoint{4.218750in}{2.172222in}}%
\pgfusepath{clip}%
\pgfsetbuttcap%
\pgfsetmiterjoin%
\definecolor{currentfill}{rgb}{0.227451,0.192157,0.427451}%
\pgfsetfillcolor{currentfill}%
\pgfsetlinewidth{1.003750pt}%
\definecolor{currentstroke}{rgb}{0.266667,0.266667,0.266667}%
\pgfsetstrokecolor{currentstroke}%
\pgfsetdash{}{0pt}%
\pgfpathmoveto{\pgfqpoint{2.284200in}{0.638889in}}%
\pgfpathlineto{\pgfqpoint{2.513658in}{0.638889in}}%
\pgfpathlineto{\pgfqpoint{2.513658in}{1.089082in}}%
\pgfpathlineto{\pgfqpoint{2.284200in}{1.089082in}}%
\pgfpathlineto{\pgfqpoint{2.284200in}{0.638889in}}%
\pgfpathclose%
\pgfusepath{stroke,fill}%
\end{pgfscope}%
\begin{pgfscope}%
\pgfpathrectangle{\pgfqpoint{0.781250in}{0.638889in}}{\pgfqpoint{4.218750in}{2.172222in}}%
\pgfusepath{clip}%
\pgfsetbuttcap%
\pgfsetmiterjoin%
\definecolor{currentfill}{rgb}{0.227451,0.192157,0.427451}%
\pgfsetfillcolor{currentfill}%
\pgfsetlinewidth{1.003750pt}%
\definecolor{currentstroke}{rgb}{0.266667,0.266667,0.266667}%
\pgfsetstrokecolor{currentstroke}%
\pgfsetdash{}{0pt}%
\pgfpathmoveto{\pgfqpoint{2.611997in}{0.638889in}}%
\pgfpathlineto{\pgfqpoint{2.841455in}{0.638889in}}%
\pgfpathlineto{\pgfqpoint{2.841455in}{2.363633in}}%
\pgfpathlineto{\pgfqpoint{2.611997in}{2.363633in}}%
\pgfpathlineto{\pgfqpoint{2.611997in}{0.638889in}}%
\pgfpathclose%
\pgfusepath{stroke,fill}%
\end{pgfscope}%
\begin{pgfscope}%
\pgfpathrectangle{\pgfqpoint{0.781250in}{0.638889in}}{\pgfqpoint{4.218750in}{2.172222in}}%
\pgfusepath{clip}%
\pgfsetbuttcap%
\pgfsetmiterjoin%
\definecolor{currentfill}{rgb}{0.227451,0.192157,0.427451}%
\pgfsetfillcolor{currentfill}%
\pgfsetlinewidth{1.003750pt}%
\definecolor{currentstroke}{rgb}{0.266667,0.266667,0.266667}%
\pgfsetstrokecolor{currentstroke}%
\pgfsetdash{}{0pt}%
\pgfpathmoveto{\pgfqpoint{2.939795in}{0.638889in}}%
\pgfpathlineto{\pgfqpoint{3.169253in}{0.638889in}}%
\pgfpathlineto{\pgfqpoint{3.169253in}{1.773875in}}%
\pgfpathlineto{\pgfqpoint{2.939795in}{1.773875in}}%
\pgfpathlineto{\pgfqpoint{2.939795in}{0.638889in}}%
\pgfpathclose%
\pgfusepath{stroke,fill}%
\end{pgfscope}%
\begin{pgfscope}%
\pgfpathrectangle{\pgfqpoint{0.781250in}{0.638889in}}{\pgfqpoint{4.218750in}{2.172222in}}%
\pgfusepath{clip}%
\pgfsetbuttcap%
\pgfsetmiterjoin%
\definecolor{currentfill}{rgb}{0.227451,0.192157,0.427451}%
\pgfsetfillcolor{currentfill}%
\pgfsetlinewidth{1.003750pt}%
\definecolor{currentstroke}{rgb}{0.266667,0.266667,0.266667}%
\pgfsetstrokecolor{currentstroke}%
\pgfsetdash{}{0pt}%
\pgfpathmoveto{\pgfqpoint{3.267592in}{0.638889in}}%
\pgfpathlineto{\pgfqpoint{3.497050in}{0.638889in}}%
\pgfpathlineto{\pgfqpoint{3.497050in}{1.632681in}}%
\pgfpathlineto{\pgfqpoint{3.267592in}{1.632681in}}%
\pgfpathlineto{\pgfqpoint{3.267592in}{0.638889in}}%
\pgfpathclose%
\pgfusepath{stroke,fill}%
\end{pgfscope}%
\begin{pgfscope}%
\pgfpathrectangle{\pgfqpoint{0.781250in}{0.638889in}}{\pgfqpoint{4.218750in}{2.172222in}}%
\pgfusepath{clip}%
\pgfsetbuttcap%
\pgfsetmiterjoin%
\definecolor{currentfill}{rgb}{0.227451,0.192157,0.427451}%
\pgfsetfillcolor{currentfill}%
\pgfsetlinewidth{1.003750pt}%
\definecolor{currentstroke}{rgb}{0.266667,0.266667,0.266667}%
\pgfsetstrokecolor{currentstroke}%
\pgfsetdash{}{0pt}%
\pgfpathmoveto{\pgfqpoint{3.595389in}{0.638889in}}%
\pgfpathlineto{\pgfqpoint{3.824847in}{0.638889in}}%
\pgfpathlineto{\pgfqpoint{3.824847in}{2.529265in}}%
\pgfpathlineto{\pgfqpoint{3.595389in}{2.529265in}}%
\pgfpathlineto{\pgfqpoint{3.595389in}{0.638889in}}%
\pgfpathclose%
\pgfusepath{stroke,fill}%
\end{pgfscope}%
\begin{pgfscope}%
\pgfpathrectangle{\pgfqpoint{0.781250in}{0.638889in}}{\pgfqpoint{4.218750in}{2.172222in}}%
\pgfusepath{clip}%
\pgfsetbuttcap%
\pgfsetmiterjoin%
\definecolor{currentfill}{rgb}{0.227451,0.192157,0.427451}%
\pgfsetfillcolor{currentfill}%
\pgfsetlinewidth{1.003750pt}%
\definecolor{currentstroke}{rgb}{0.266667,0.266667,0.266667}%
\pgfsetstrokecolor{currentstroke}%
\pgfsetdash{}{0pt}%
\pgfpathmoveto{\pgfqpoint{3.923186in}{0.638889in}}%
\pgfpathlineto{\pgfqpoint{4.152644in}{0.638889in}}%
\pgfpathlineto{\pgfqpoint{4.152644in}{1.131440in}}%
\pgfpathlineto{\pgfqpoint{3.923186in}{1.131440in}}%
\pgfpathlineto{\pgfqpoint{3.923186in}{0.638889in}}%
\pgfpathclose%
\pgfusepath{stroke,fill}%
\end{pgfscope}%
\begin{pgfscope}%
\pgfpathrectangle{\pgfqpoint{0.781250in}{0.638889in}}{\pgfqpoint{4.218750in}{2.172222in}}%
\pgfusepath{clip}%
\pgfsetbuttcap%
\pgfsetmiterjoin%
\definecolor{currentfill}{rgb}{0.227451,0.192157,0.427451}%
\pgfsetfillcolor{currentfill}%
\pgfsetlinewidth{1.003750pt}%
\definecolor{currentstroke}{rgb}{0.266667,0.266667,0.266667}%
\pgfsetstrokecolor{currentstroke}%
\pgfsetdash{}{0pt}%
\pgfpathmoveto{\pgfqpoint{4.250983in}{0.638889in}}%
\pgfpathlineto{\pgfqpoint{4.480441in}{0.638889in}}%
\pgfpathlineto{\pgfqpoint{4.480441in}{0.755646in}}%
\pgfpathlineto{\pgfqpoint{4.250983in}{0.755646in}}%
\pgfpathlineto{\pgfqpoint{4.250983in}{0.638889in}}%
\pgfpathclose%
\pgfusepath{stroke,fill}%
\end{pgfscope}%
\begin{pgfscope}%
\pgfpathrectangle{\pgfqpoint{0.781250in}{0.638889in}}{\pgfqpoint{4.218750in}{2.172222in}}%
\pgfusepath{clip}%
\pgfsetbuttcap%
\pgfsetmiterjoin%
\definecolor{currentfill}{rgb}{0.227451,0.192157,0.427451}%
\pgfsetfillcolor{currentfill}%
\pgfsetlinewidth{1.003750pt}%
\definecolor{currentstroke}{rgb}{0.266667,0.266667,0.266667}%
\pgfsetstrokecolor{currentstroke}%
\pgfsetdash{}{0pt}%
\pgfpathmoveto{\pgfqpoint{4.578781in}{0.638889in}}%
\pgfpathlineto{\pgfqpoint{4.808239in}{0.638889in}}%
\pgfpathlineto{\pgfqpoint{4.808239in}{0.650293in}}%
\pgfpathlineto{\pgfqpoint{4.578781in}{0.650293in}}%
\pgfpathlineto{\pgfqpoint{4.578781in}{0.638889in}}%
\pgfpathclose%
\pgfusepath{stroke,fill}%
\end{pgfscope}%
\begin{pgfscope}%
\definecolor{textcolor}{rgb}{0.000000,0.000000,0.000000}%
\pgfsetstrokecolor{textcolor}%
\pgfsetfillcolor{textcolor}%
\pgftext[x=1.087740in,y=0.909956in,,bottom]{\color{textcolor}{\ifdefined\pdftexversion\else\setmainfont{NanumMyeongjo}\rmfamily\fi\fontsize{7.000000}{8.400000}\bfseries\selectfont\catcode`\^=\active\def^{\ifmmode\sp\else\^{}\fi}\catcode`\%=\active\def%{\%}45}}%
\end{pgfscope}%
\begin{pgfscope}%
\definecolor{textcolor}{rgb}{0.000000,0.000000,0.000000}%
\pgfsetstrokecolor{textcolor}%
\pgfsetfillcolor{textcolor}%
\pgftext[x=1.415538in,y=0.859994in,,bottom]{\color{textcolor}{\ifdefined\pdftexversion\else\setmainfont{NanumMyeongjo}\rmfamily\fi\fontsize{7.000000}{8.400000}\bfseries\selectfont\catcode`\^=\active\def^{\ifmmode\sp\else\^{}\fi}\catcode`\%=\active\def%{\%}36}}%
\end{pgfscope}%
\begin{pgfscope}%
\definecolor{textcolor}{rgb}{0.000000,0.000000,0.000000}%
\pgfsetstrokecolor{textcolor}%
\pgfsetfillcolor{textcolor}%
\pgftext[x=1.743335in,y=0.900181in,,bottom]{\color{textcolor}{\ifdefined\pdftexversion\else\setmainfont{NanumMyeongjo}\rmfamily\fi\fontsize{7.000000}{8.400000}\bfseries\selectfont\catcode`\^=\active\def^{\ifmmode\sp\else\^{}\fi}\catcode`\%=\active\def%{\%}43}}%
\end{pgfscope}%
\begin{pgfscope}%
\definecolor{textcolor}{rgb}{0.000000,0.000000,0.000000}%
\pgfsetstrokecolor{textcolor}%
\pgfsetfillcolor{textcolor}%
\pgftext[x=2.071132in,y=1.006619in,,bottom]{\color{textcolor}{\ifdefined\pdftexversion\else\setmainfont{NanumMyeongjo}\rmfamily\fi\fontsize{7.000000}{8.400000}\bfseries\selectfont\catcode`\^=\active\def^{\ifmmode\sp\else\^{}\fi}\catcode`\%=\active\def%{\%}63}}%
\end{pgfscope}%
\begin{pgfscope}%
\definecolor{textcolor}{rgb}{0.000000,0.000000,0.000000}%
\pgfsetstrokecolor{textcolor}%
\pgfsetfillcolor{textcolor}%
\pgftext[x=2.398929in,y=1.116860in,,bottom]{\color{textcolor}{\ifdefined\pdftexversion\else\setmainfont{NanumMyeongjo}\rmfamily\fi\fontsize{7.000000}{8.400000}\bfseries\selectfont\catcode`\^=\active\def^{\ifmmode\sp\else\^{}\fi}\catcode`\%=\active\def%{\%}83}}%
\end{pgfscope}%
\begin{pgfscope}%
\definecolor{textcolor}{rgb}{0.000000,0.000000,0.000000}%
\pgfsetstrokecolor{textcolor}%
\pgfsetfillcolor{textcolor}%
\pgftext[x=2.726726in,y=2.391411in,,bottom]{\color{textcolor}{\ifdefined\pdftexversion\else\setmainfont{NanumMyeongjo}\rmfamily\fi\fontsize{7.000000}{8.400000}\bfseries\selectfont\catcode`\^=\active\def^{\ifmmode\sp\else\^{}\fi}\catcode`\%=\active\def%{\%}318}}%
\end{pgfscope}%
\begin{pgfscope}%
\definecolor{textcolor}{rgb}{0.000000,0.000000,0.000000}%
\pgfsetstrokecolor{textcolor}%
\pgfsetfillcolor{textcolor}%
\pgftext[x=3.054524in,y=1.801653in,,bottom]{\color{textcolor}{\ifdefined\pdftexversion\else\setmainfont{NanumMyeongjo}\rmfamily\fi\fontsize{7.000000}{8.400000}\bfseries\selectfont\catcode`\^=\active\def^{\ifmmode\sp\else\^{}\fi}\catcode`\%=\active\def%{\%}209}}%
\end{pgfscope}%
\begin{pgfscope}%
\definecolor{textcolor}{rgb}{0.000000,0.000000,0.000000}%
\pgfsetstrokecolor{textcolor}%
\pgfsetfillcolor{textcolor}%
\pgftext[x=3.382321in,y=1.660458in,,bottom]{\color{textcolor}{\ifdefined\pdftexversion\else\setmainfont{NanumMyeongjo}\rmfamily\fi\fontsize{7.000000}{8.400000}\bfseries\selectfont\catcode`\^=\active\def^{\ifmmode\sp\else\^{}\fi}\catcode`\%=\active\def%{\%}183}}%
\end{pgfscope}%
\begin{pgfscope}%
\definecolor{textcolor}{rgb}{0.000000,0.000000,0.000000}%
\pgfsetstrokecolor{textcolor}%
\pgfsetfillcolor{textcolor}%
\pgftext[x=3.710118in,y=2.557043in,,bottom]{\color{textcolor}{\ifdefined\pdftexversion\else\setmainfont{NanumMyeongjo}\rmfamily\fi\fontsize{7.000000}{8.400000}\bfseries\selectfont\catcode`\^=\active\def^{\ifmmode\sp\else\^{}\fi}\catcode`\%=\active\def%{\%}348}}%
\end{pgfscope}%
\begin{pgfscope}%
\definecolor{textcolor}{rgb}{0.000000,0.000000,0.000000}%
\pgfsetstrokecolor{textcolor}%
\pgfsetfillcolor{textcolor}%
\pgftext[x=4.037915in,y=1.159218in,,bottom]{\color{textcolor}{\ifdefined\pdftexversion\else\setmainfont{NanumMyeongjo}\rmfamily\fi\fontsize{7.000000}{8.400000}\bfseries\selectfont\catcode`\^=\active\def^{\ifmmode\sp\else\^{}\fi}\catcode`\%=\active\def%{\%}91}}%
\end{pgfscope}%
\begin{pgfscope}%
\definecolor{textcolor}{rgb}{0.000000,0.000000,0.000000}%
\pgfsetstrokecolor{textcolor}%
\pgfsetfillcolor{textcolor}%
\pgftext[x=4.365712in,y=0.783424in,,bottom]{\color{textcolor}{\ifdefined\pdftexversion\else\setmainfont{NanumMyeongjo}\rmfamily\fi\fontsize{7.000000}{8.400000}\bfseries\selectfont\catcode`\^=\active\def^{\ifmmode\sp\else\^{}\fi}\catcode`\%=\active\def%{\%}22}}%
\end{pgfscope}%
\begin{pgfscope}%
\definecolor{textcolor}{rgb}{0.000000,0.000000,0.000000}%
\pgfsetstrokecolor{textcolor}%
\pgfsetfillcolor{textcolor}%
\pgftext[x=4.693510in,y=0.678071in,,bottom]{\color{textcolor}{\ifdefined\pdftexversion\else\setmainfont{NanumMyeongjo}\rmfamily\fi\fontsize{7.000000}{8.400000}\bfseries\selectfont\catcode`\^=\active\def^{\ifmmode\sp\else\^{}\fi}\catcode`\%=\active\def%{\%}2}}%
\end{pgfscope}%
\begin{pgfscope}%
\pgfpathrectangle{\pgfqpoint{0.781250in}{0.638889in}}{\pgfqpoint{4.218750in}{2.172222in}}%
\pgfusepath{clip}%
\pgfsetrectcap%
\pgfsetroundjoin%
\pgfsetlinewidth{0.903375pt}%
\definecolor{currentstroke}{rgb}{0.164706,0.615686,0.560784}%
\pgfsetstrokecolor{currentstroke}%
\pgfsetdash{}{0pt}%
\pgfpathmoveto{\pgfqpoint{1.087740in}{0.811038in}}%
\pgfpathlineto{\pgfqpoint{1.415538in}{0.836561in}}%
\pgfpathlineto{\pgfqpoint{1.743335in}{0.895754in}}%
\pgfpathlineto{\pgfqpoint{2.071132in}{1.073333in}}%
\pgfpathlineto{\pgfqpoint{2.398929in}{1.099943in}}%
\pgfpathlineto{\pgfqpoint{2.726726in}{1.354636in}}%
\pgfpathlineto{\pgfqpoint{3.054524in}{2.135007in}}%
\pgfpathlineto{\pgfqpoint{3.382321in}{2.052462in}}%
\pgfpathlineto{\pgfqpoint{3.710118in}{1.395908in}}%
\pgfpathlineto{\pgfqpoint{4.037915in}{0.943000in}}%
\pgfpathlineto{\pgfqpoint{4.365712in}{0.922907in}}%
\pgfpathlineto{\pgfqpoint{4.693510in}{0.869144in}}%
\pgfusepath{stroke}%
\end{pgfscope}%
\begin{pgfscope}%
\pgfpathrectangle{\pgfqpoint{0.781250in}{0.638889in}}{\pgfqpoint{4.218750in}{2.172222in}}%
\pgfusepath{clip}%
\pgfsetbuttcap%
\pgfsetroundjoin%
\definecolor{currentfill}{rgb}{0.164706,0.615686,0.560784}%
\pgfsetfillcolor{currentfill}%
\pgfsetlinewidth{1.003750pt}%
\definecolor{currentstroke}{rgb}{0.164706,0.615686,0.560784}%
\pgfsetstrokecolor{currentstroke}%
\pgfsetdash{}{0pt}%
\pgfsys@defobject{currentmarker}{\pgfqpoint{-0.018750in}{-0.018750in}}{\pgfqpoint{0.018750in}{0.018750in}}{%
\pgfpathmoveto{\pgfqpoint{0.000000in}{-0.018750in}}%
\pgfpathcurveto{\pgfqpoint{0.004973in}{-0.018750in}}{\pgfqpoint{0.009742in}{-0.016774in}}{\pgfqpoint{0.013258in}{-0.013258in}}%
\pgfpathcurveto{\pgfqpoint{0.016774in}{-0.009742in}}{\pgfqpoint{0.018750in}{-0.004973in}}{\pgfqpoint{0.018750in}{0.000000in}}%
\pgfpathcurveto{\pgfqpoint{0.018750in}{0.004973in}}{\pgfqpoint{0.016774in}{0.009742in}}{\pgfqpoint{0.013258in}{0.013258in}}%
\pgfpathcurveto{\pgfqpoint{0.009742in}{0.016774in}}{\pgfqpoint{0.004973in}{0.018750in}}{\pgfqpoint{0.000000in}{0.018750in}}%
\pgfpathcurveto{\pgfqpoint{-0.004973in}{0.018750in}}{\pgfqpoint{-0.009742in}{0.016774in}}{\pgfqpoint{-0.013258in}{0.013258in}}%
\pgfpathcurveto{\pgfqpoint{-0.016774in}{0.009742in}}{\pgfqpoint{-0.018750in}{0.004973in}}{\pgfqpoint{-0.018750in}{0.000000in}}%
\pgfpathcurveto{\pgfqpoint{-0.018750in}{-0.004973in}}{\pgfqpoint{-0.016774in}{-0.009742in}}{\pgfqpoint{-0.013258in}{-0.013258in}}%
\pgfpathcurveto{\pgfqpoint{-0.009742in}{-0.016774in}}{\pgfqpoint{-0.004973in}{-0.018750in}}{\pgfqpoint{0.000000in}{-0.018750in}}%
\pgfpathlineto{\pgfqpoint{0.000000in}{-0.018750in}}%
\pgfpathclose%
\pgfusepath{stroke,fill}%
}%
\begin{pgfscope}%
\pgfsys@transformshift{1.087740in}{0.811038in}%
\pgfsys@useobject{currentmarker}{}%
\end{pgfscope}%
\begin{pgfscope}%
\pgfsys@transformshift{1.415538in}{0.836561in}%
\pgfsys@useobject{currentmarker}{}%
\end{pgfscope}%
\begin{pgfscope}%
\pgfsys@transformshift{1.743335in}{0.895754in}%
\pgfsys@useobject{currentmarker}{}%
\end{pgfscope}%
\begin{pgfscope}%
\pgfsys@transformshift{2.071132in}{1.073333in}%
\pgfsys@useobject{currentmarker}{}%
\end{pgfscope}%
\begin{pgfscope}%
\pgfsys@transformshift{2.398929in}{1.099943in}%
\pgfsys@useobject{currentmarker}{}%
\end{pgfscope}%
\begin{pgfscope}%
\pgfsys@transformshift{2.726726in}{1.354636in}%
\pgfsys@useobject{currentmarker}{}%
\end{pgfscope}%
\begin{pgfscope}%
\pgfsys@transformshift{3.054524in}{2.135007in}%
\pgfsys@useobject{currentmarker}{}%
\end{pgfscope}%
\begin{pgfscope}%
\pgfsys@transformshift{3.382321in}{2.052462in}%
\pgfsys@useobject{currentmarker}{}%
\end{pgfscope}%
\begin{pgfscope}%
\pgfsys@transformshift{3.710118in}{1.395908in}%
\pgfsys@useobject{currentmarker}{}%
\end{pgfscope}%
\begin{pgfscope}%
\pgfsys@transformshift{4.037915in}{0.943000in}%
\pgfsys@useobject{currentmarker}{}%
\end{pgfscope}%
\begin{pgfscope}%
\pgfsys@transformshift{4.365712in}{0.922907in}%
\pgfsys@useobject{currentmarker}{}%
\end{pgfscope}%
\begin{pgfscope}%
\pgfsys@transformshift{4.693510in}{0.869144in}%
\pgfsys@useobject{currentmarker}{}%
\end{pgfscope}%
\end{pgfscope}%
\begin{pgfscope}%
\pgfsetrectcap%
\pgfsetroundjoin%
\pgfsetlinewidth{0.903375pt}%
\definecolor{currentstroke}{rgb}{0.164706,0.615686,0.560784}%
\pgfsetstrokecolor{currentstroke}%
\pgfsetdash{}{0pt}%
\pgfpathmoveto{\pgfqpoint{5.112500in}{2.762986in}}%
\pgfpathlineto{\pgfqpoint{5.237500in}{2.762986in}}%
\pgfpathlineto{\pgfqpoint{5.362500in}{2.762986in}}%
\pgfusepath{stroke}%
\end{pgfscope}%
\begin{pgfscope}%
\pgfsetbuttcap%
\pgfsetroundjoin%
\definecolor{currentfill}{rgb}{0.164706,0.615686,0.560784}%
\pgfsetfillcolor{currentfill}%
\pgfsetlinewidth{1.003750pt}%
\definecolor{currentstroke}{rgb}{0.164706,0.615686,0.560784}%
\pgfsetstrokecolor{currentstroke}%
\pgfsetdash{}{0pt}%
\pgfsys@defobject{currentmarker}{\pgfqpoint{-0.018750in}{-0.018750in}}{\pgfqpoint{0.018750in}{0.018750in}}{%
\pgfpathmoveto{\pgfqpoint{0.000000in}{-0.018750in}}%
\pgfpathcurveto{\pgfqpoint{0.004973in}{-0.018750in}}{\pgfqpoint{0.009742in}{-0.016774in}}{\pgfqpoint{0.013258in}{-0.013258in}}%
\pgfpathcurveto{\pgfqpoint{0.016774in}{-0.009742in}}{\pgfqpoint{0.018750in}{-0.004973in}}{\pgfqpoint{0.018750in}{0.000000in}}%
\pgfpathcurveto{\pgfqpoint{0.018750in}{0.004973in}}{\pgfqpoint{0.016774in}{0.009742in}}{\pgfqpoint{0.013258in}{0.013258in}}%
\pgfpathcurveto{\pgfqpoint{0.009742in}{0.016774in}}{\pgfqpoint{0.004973in}{0.018750in}}{\pgfqpoint{0.000000in}{0.018750in}}%
\pgfpathcurveto{\pgfqpoint{-0.004973in}{0.018750in}}{\pgfqpoint{-0.009742in}{0.016774in}}{\pgfqpoint{-0.013258in}{0.013258in}}%
\pgfpathcurveto{\pgfqpoint{-0.016774in}{0.009742in}}{\pgfqpoint{-0.018750in}{0.004973in}}{\pgfqpoint{-0.018750in}{0.000000in}}%
\pgfpathcurveto{\pgfqpoint{-0.018750in}{-0.004973in}}{\pgfqpoint{-0.016774in}{-0.009742in}}{\pgfqpoint{-0.013258in}{-0.013258in}}%
\pgfpathcurveto{\pgfqpoint{-0.009742in}{-0.016774in}}{\pgfqpoint{-0.004973in}{-0.018750in}}{\pgfqpoint{0.000000in}{-0.018750in}}%
\pgfpathlineto{\pgfqpoint{0.000000in}{-0.018750in}}%
\pgfpathclose%
\pgfusepath{stroke,fill}%
}%
\begin{pgfscope}%
\pgfsys@transformshift{5.237500in}{2.762986in}%
\pgfsys@useobject{currentmarker}{}%
\end{pgfscope}%
\end{pgfscope}%
\begin{pgfscope}%
\definecolor{textcolor}{rgb}{0.000000,0.000000,0.000000}%
\pgfsetstrokecolor{textcolor}%
\pgfsetfillcolor{textcolor}%
\pgftext[x=5.462500in,y=2.719236in,left,base]{\color{textcolor}{\ifdefined\pdftexversion\else\setmainfont{NanumMyeongjo}\rmfamily\fi\fontsize{9.000000}{10.800000}\selectfont\catcode`\^=\active\def^{\ifmmode\sp\else\^{}\fi}\catcode`\%=\active\def%{\%}평년}}%
\end{pgfscope}%
\begin{pgfscope}%
\pgfsetbuttcap%
\pgfsetmiterjoin%
\definecolor{currentfill}{rgb}{0.227451,0.192157,0.427451}%
\pgfsetfillcolor{currentfill}%
\pgfsetlinewidth{1.003750pt}%
\definecolor{currentstroke}{rgb}{0.266667,0.266667,0.266667}%
\pgfsetstrokecolor{currentstroke}%
\pgfsetdash{}{0pt}%
\pgfpathmoveto{\pgfqpoint{5.112500in}{2.527952in}}%
\pgfpathlineto{\pgfqpoint{5.362500in}{2.527952in}}%
\pgfpathlineto{\pgfqpoint{5.362500in}{2.615452in}}%
\pgfpathlineto{\pgfqpoint{5.112500in}{2.615452in}}%
\pgfpathlineto{\pgfqpoint{5.112500in}{2.527952in}}%
\pgfpathclose%
\pgfusepath{stroke,fill}%
\end{pgfscope}%
\begin{pgfscope}%
\definecolor{textcolor}{rgb}{0.000000,0.000000,0.000000}%
\pgfsetstrokecolor{textcolor}%
\pgfsetfillcolor{textcolor}%
\pgftext[x=5.462500in,y=2.527952in,left,base]{\color{textcolor}{\ifdefined\pdftexversion\else\setmainfont{NanumMyeongjo}\rmfamily\fi\fontsize{9.000000}{10.800000}\selectfont\catcode`\^=\active\def^{\ifmmode\sp\else\^{}\fi}\catcode`\%=\active\def%{\%}25년도}}%
\end{pgfscope}%
\begin{pgfscope}%
\definecolor{textcolor}{rgb}{0.333333,0.333333,0.333333}%
\pgfsetstrokecolor{textcolor}%
\pgfsetfillcolor{textcolor}%
\pgftext[x=1.875000in,y=0.319444in,,top]{\color{textcolor}{\ifdefined\pdftexversion\else\setmainfont{NanumMyeongjo}\rmfamily\fi\fontsize{9.000000}{10.800000}\selectfont\catcode`\^=\active\def^{\ifmmode\sp\else\^{}\fi}\catcode`\%=\active\def%{\%}출처: 기상자료개방포털 자료 기반 저자 작성}}%
\end{pgfscope}%
\begin{pgfscope}%
\definecolor{textcolor}{rgb}{0.333333,0.333333,0.333333}%
\pgfsetstrokecolor{textcolor}%
\pgfsetfillcolor{textcolor}%
\pgftext[x=4.687500in,y=2.970833in,,top]{\color{textcolor}{\ifdefined\pdftexversion\else\setmainfont{NanumMyeongjo}\rmfamily\fi\fontsize{9.000000}{10.800000}\selectfont\catcode`\^=\active\def^{\ifmmode\sp\else\^{}\fi}\catcode`\%=\active\def%{\%}(단위: mm)}}%
\end{pgfscope}%
\end{pgfpicture}%
\makeatother%
\endgroup%
}
\end{center}
}


\slide
{\maintitle}
{\chapterthree}
{여름 장마}{

\begin{center}
\includegraphics[width=0.7\textwidth]{asset/여름장마.png}
\end{center}
\vspace{-10pt}
\small ※ 전업농신문

\begin{center}
\includegraphics[width=0.7\textwidth]{asset/여름장마_차관.png}
\end{center}
\vspace{-20pt}
\small ※ 농림축산식품부 보도자료
}


\slide
{\maintitle}
{\chapterthree}
{가을 장마}{
\vspace{10pt}

\begin{center}
\includegraphics[width=0.7\textwidth]{asset/가을장마.png}
\end{center}
\vspace{-10pt}
\small ※ 한국농정

\begin{center}
\includegraphics[width=0.7\textwidth]{asset/가을장마_김제.png}
\end{center}
\vspace{-20pt}
\small ※ 전북일보
}



\slide
{\maintitle}
{\chapterthree}
{부안군 과거 강수량}{
\begin{center}
    \hspace*{-40pt}{%% Creator: Matplotlib, PGF backend
%%
%% To include the figure in your LaTeX document, write
%%   \input{<filename>.pgf}
%%
%% Make sure the required packages are loaded in your preamble
%%   \usepackage{pgf}
%%
%% Also ensure that all the required font packages are loaded; for instance,
%% the lmodern package is sometimes necessary when using math font.
%%   \usepackage{lmodern}
%%
%% Figures using additional raster images can only be included by \input if
%% they are in the same directory as the main LaTeX file. For loading figures
%% from other directories you can use the `import` package
%%   \usepackage{import}
%%
%% and then include the figures with
%%   \import{<path to file>}{<filename>.pgf}
%%
%% Matplotlib used the following preamble
%%   \def\mathdefault#1{#1}
%%   \everymath=\expandafter{\the\everymath\displaystyle}
%%   \IfFileExists{scrextend.sty}{
%%     \usepackage[fontsize=9.000000pt]{scrextend}
%%   }{
%%     \renewcommand{\normalsize}{\fontsize{9.000000}{10.800000}\selectfont}
%%     \normalsize
%%   }
%%   
%%   \ifdefined\pdftexversion\else  % non-pdftex case.
%%     \usepackage{fontspec}
%%     \setmainfont{DejaVuSerif.ttf}[Path=\detokenize{/home/user/.cache/pypoetry/virtualenvs/graph-KASAOWVd-py3.12/lib/python3.12/site-packages/matplotlib/mpl-data/fonts/ttf/}]
%%     \setsansfont{DejaVuSans.ttf}[Path=\detokenize{/home/user/.cache/pypoetry/virtualenvs/graph-KASAOWVd-py3.12/lib/python3.12/site-packages/matplotlib/mpl-data/fonts/ttf/}]
%%     \setmonofont{DejaVuSansMono.ttf}[Path=\detokenize{/home/user/.cache/pypoetry/virtualenvs/graph-KASAOWVd-py3.12/lib/python3.12/site-packages/matplotlib/mpl-data/fonts/ttf/}]
%%   \fi
%%   \makeatletter\@ifpackageloaded{underscore}{}{\usepackage[strings]{underscore}}\makeatother
%%
\begingroup%
\makeatletter%
\begin{pgfpicture}%
\pgfpathrectangle{\pgfpointorigin}{\pgfqpoint{6.250000in}{3.194444in}}%
\pgfusepath{use as bounding box, clip}%
\begin{pgfscope}%
\pgfsetbuttcap%
\pgfsetmiterjoin%
\definecolor{currentfill}{rgb}{1.000000,1.000000,1.000000}%
\pgfsetfillcolor{currentfill}%
\pgfsetlinewidth{0.000000pt}%
\definecolor{currentstroke}{rgb}{1.000000,1.000000,1.000000}%
\pgfsetstrokecolor{currentstroke}%
\pgfsetdash{}{0pt}%
\pgfpathmoveto{\pgfqpoint{0.000000in}{0.000000in}}%
\pgfpathlineto{\pgfqpoint{6.250000in}{0.000000in}}%
\pgfpathlineto{\pgfqpoint{6.250000in}{3.194444in}}%
\pgfpathlineto{\pgfqpoint{0.000000in}{3.194444in}}%
\pgfpathlineto{\pgfqpoint{0.000000in}{0.000000in}}%
\pgfpathclose%
\pgfusepath{fill}%
\end{pgfscope}%
\begin{pgfscope}%
\pgfsetbuttcap%
\pgfsetmiterjoin%
\definecolor{currentfill}{rgb}{1.000000,1.000000,1.000000}%
\pgfsetfillcolor{currentfill}%
\pgfsetlinewidth{0.000000pt}%
\definecolor{currentstroke}{rgb}{0.000000,0.000000,0.000000}%
\pgfsetstrokecolor{currentstroke}%
\pgfsetstrokeopacity{0.000000}%
\pgfsetdash{}{0pt}%
\pgfpathmoveto{\pgfqpoint{0.781250in}{0.638889in}}%
\pgfpathlineto{\pgfqpoint{5.625000in}{0.638889in}}%
\pgfpathlineto{\pgfqpoint{5.625000in}{2.811111in}}%
\pgfpathlineto{\pgfqpoint{0.781250in}{2.811111in}}%
\pgfpathlineto{\pgfqpoint{0.781250in}{0.638889in}}%
\pgfpathclose%
\pgfusepath{fill}%
\end{pgfscope}%
\begin{pgfscope}%
\pgfsetbuttcap%
\pgfsetroundjoin%
\definecolor{currentfill}{rgb}{0.000000,0.000000,0.000000}%
\pgfsetfillcolor{currentfill}%
\pgfsetlinewidth{0.752812pt}%
\definecolor{currentstroke}{rgb}{0.000000,0.000000,0.000000}%
\pgfsetstrokecolor{currentstroke}%
\pgfsetdash{}{0pt}%
\pgfsys@defobject{currentmarker}{\pgfqpoint{0.000000in}{-0.013889in}}{\pgfqpoint{0.000000in}{0.000000in}}{%
\pgfpathmoveto{\pgfqpoint{0.000000in}{0.000000in}}%
\pgfpathlineto{\pgfqpoint{0.000000in}{-0.013889in}}%
\pgfusepath{stroke,fill}%
}%
\begin{pgfscope}%
\pgfsys@transformshift{1.014296in}{0.638889in}%
\pgfsys@useobject{currentmarker}{}%
\end{pgfscope}%
\end{pgfscope}%
\begin{pgfscope}%
\definecolor{textcolor}{rgb}{0.000000,0.000000,0.000000}%
\pgfsetstrokecolor{textcolor}%
\pgfsetfillcolor{textcolor}%
\pgftext[x=1.014296in,y=0.576389in,,top]{\color{textcolor}{\ifdefined\pdftexversion\else\setmainfont{NanumMyeongjo}\rmfamily\fi\fontsize{9.000000}{10.800000}\selectfont\catcode`\^=\active\def^{\ifmmode\sp\else\^{}\fi}\catcode`\%=\active\def%{\%}2016}}%
\end{pgfscope}%
\begin{pgfscope}%
\pgfsetbuttcap%
\pgfsetroundjoin%
\definecolor{currentfill}{rgb}{0.000000,0.000000,0.000000}%
\pgfsetfillcolor{currentfill}%
\pgfsetlinewidth{0.752812pt}%
\definecolor{currentstroke}{rgb}{0.000000,0.000000,0.000000}%
\pgfsetstrokecolor{currentstroke}%
\pgfsetdash{}{0pt}%
\pgfsys@defobject{currentmarker}{\pgfqpoint{0.000000in}{-0.013889in}}{\pgfqpoint{0.000000in}{0.000000in}}{%
\pgfpathmoveto{\pgfqpoint{0.000000in}{0.000000in}}%
\pgfpathlineto{\pgfqpoint{0.000000in}{-0.013889in}}%
\pgfusepath{stroke,fill}%
}%
\begin{pgfscope}%
\pgfsys@transformshift{1.455740in}{0.638889in}%
\pgfsys@useobject{currentmarker}{}%
\end{pgfscope}%
\end{pgfscope}%
\begin{pgfscope}%
\definecolor{textcolor}{rgb}{0.000000,0.000000,0.000000}%
\pgfsetstrokecolor{textcolor}%
\pgfsetfillcolor{textcolor}%
\pgftext[x=1.455740in,y=0.576389in,,top]{\color{textcolor}{\ifdefined\pdftexversion\else\setmainfont{NanumMyeongjo}\rmfamily\fi\fontsize{9.000000}{10.800000}\selectfont\catcode`\^=\active\def^{\ifmmode\sp\else\^{}\fi}\catcode`\%=\active\def%{\%}2017}}%
\end{pgfscope}%
\begin{pgfscope}%
\pgfsetbuttcap%
\pgfsetroundjoin%
\definecolor{currentfill}{rgb}{0.000000,0.000000,0.000000}%
\pgfsetfillcolor{currentfill}%
\pgfsetlinewidth{0.752812pt}%
\definecolor{currentstroke}{rgb}{0.000000,0.000000,0.000000}%
\pgfsetstrokecolor{currentstroke}%
\pgfsetdash{}{0pt}%
\pgfsys@defobject{currentmarker}{\pgfqpoint{0.000000in}{-0.013889in}}{\pgfqpoint{0.000000in}{0.000000in}}{%
\pgfpathmoveto{\pgfqpoint{0.000000in}{0.000000in}}%
\pgfpathlineto{\pgfqpoint{0.000000in}{-0.013889in}}%
\pgfusepath{stroke,fill}%
}%
\begin{pgfscope}%
\pgfsys@transformshift{1.897185in}{0.638889in}%
\pgfsys@useobject{currentmarker}{}%
\end{pgfscope}%
\end{pgfscope}%
\begin{pgfscope}%
\definecolor{textcolor}{rgb}{0.000000,0.000000,0.000000}%
\pgfsetstrokecolor{textcolor}%
\pgfsetfillcolor{textcolor}%
\pgftext[x=1.897185in,y=0.576389in,,top]{\color{textcolor}{\ifdefined\pdftexversion\else\setmainfont{NanumMyeongjo}\rmfamily\fi\fontsize{9.000000}{10.800000}\selectfont\catcode`\^=\active\def^{\ifmmode\sp\else\^{}\fi}\catcode`\%=\active\def%{\%}2018}}%
\end{pgfscope}%
\begin{pgfscope}%
\pgfsetbuttcap%
\pgfsetroundjoin%
\definecolor{currentfill}{rgb}{0.000000,0.000000,0.000000}%
\pgfsetfillcolor{currentfill}%
\pgfsetlinewidth{0.752812pt}%
\definecolor{currentstroke}{rgb}{0.000000,0.000000,0.000000}%
\pgfsetstrokecolor{currentstroke}%
\pgfsetdash{}{0pt}%
\pgfsys@defobject{currentmarker}{\pgfqpoint{0.000000in}{-0.013889in}}{\pgfqpoint{0.000000in}{0.000000in}}{%
\pgfpathmoveto{\pgfqpoint{0.000000in}{0.000000in}}%
\pgfpathlineto{\pgfqpoint{0.000000in}{-0.013889in}}%
\pgfusepath{stroke,fill}%
}%
\begin{pgfscope}%
\pgfsys@transformshift{2.338629in}{0.638889in}%
\pgfsys@useobject{currentmarker}{}%
\end{pgfscope}%
\end{pgfscope}%
\begin{pgfscope}%
\definecolor{textcolor}{rgb}{0.000000,0.000000,0.000000}%
\pgfsetstrokecolor{textcolor}%
\pgfsetfillcolor{textcolor}%
\pgftext[x=2.338629in,y=0.576389in,,top]{\color{textcolor}{\ifdefined\pdftexversion\else\setmainfont{NanumMyeongjo}\rmfamily\fi\fontsize{9.000000}{10.800000}\selectfont\catcode`\^=\active\def^{\ifmmode\sp\else\^{}\fi}\catcode`\%=\active\def%{\%}2019}}%
\end{pgfscope}%
\begin{pgfscope}%
\pgfsetbuttcap%
\pgfsetroundjoin%
\definecolor{currentfill}{rgb}{0.000000,0.000000,0.000000}%
\pgfsetfillcolor{currentfill}%
\pgfsetlinewidth{0.752812pt}%
\definecolor{currentstroke}{rgb}{0.000000,0.000000,0.000000}%
\pgfsetstrokecolor{currentstroke}%
\pgfsetdash{}{0pt}%
\pgfsys@defobject{currentmarker}{\pgfqpoint{0.000000in}{-0.013889in}}{\pgfqpoint{0.000000in}{0.000000in}}{%
\pgfpathmoveto{\pgfqpoint{0.000000in}{0.000000in}}%
\pgfpathlineto{\pgfqpoint{0.000000in}{-0.013889in}}%
\pgfusepath{stroke,fill}%
}%
\begin{pgfscope}%
\pgfsys@transformshift{2.780074in}{0.638889in}%
\pgfsys@useobject{currentmarker}{}%
\end{pgfscope}%
\end{pgfscope}%
\begin{pgfscope}%
\definecolor{textcolor}{rgb}{0.000000,0.000000,0.000000}%
\pgfsetstrokecolor{textcolor}%
\pgfsetfillcolor{textcolor}%
\pgftext[x=2.780074in,y=0.576389in,,top]{\color{textcolor}{\ifdefined\pdftexversion\else\setmainfont{NanumMyeongjo}\rmfamily\fi\fontsize{9.000000}{10.800000}\selectfont\catcode`\^=\active\def^{\ifmmode\sp\else\^{}\fi}\catcode`\%=\active\def%{\%}2020}}%
\end{pgfscope}%
\begin{pgfscope}%
\pgfsetbuttcap%
\pgfsetroundjoin%
\definecolor{currentfill}{rgb}{0.000000,0.000000,0.000000}%
\pgfsetfillcolor{currentfill}%
\pgfsetlinewidth{0.752812pt}%
\definecolor{currentstroke}{rgb}{0.000000,0.000000,0.000000}%
\pgfsetstrokecolor{currentstroke}%
\pgfsetdash{}{0pt}%
\pgfsys@defobject{currentmarker}{\pgfqpoint{0.000000in}{-0.013889in}}{\pgfqpoint{0.000000in}{0.000000in}}{%
\pgfpathmoveto{\pgfqpoint{0.000000in}{0.000000in}}%
\pgfpathlineto{\pgfqpoint{0.000000in}{-0.013889in}}%
\pgfusepath{stroke,fill}%
}%
\begin{pgfscope}%
\pgfsys@transformshift{3.221519in}{0.638889in}%
\pgfsys@useobject{currentmarker}{}%
\end{pgfscope}%
\end{pgfscope}%
\begin{pgfscope}%
\definecolor{textcolor}{rgb}{0.000000,0.000000,0.000000}%
\pgfsetstrokecolor{textcolor}%
\pgfsetfillcolor{textcolor}%
\pgftext[x=3.221519in,y=0.576389in,,top]{\color{textcolor}{\ifdefined\pdftexversion\else\setmainfont{NanumMyeongjo}\rmfamily\fi\fontsize{9.000000}{10.800000}\selectfont\catcode`\^=\active\def^{\ifmmode\sp\else\^{}\fi}\catcode`\%=\active\def%{\%}2021}}%
\end{pgfscope}%
\begin{pgfscope}%
\pgfsetbuttcap%
\pgfsetroundjoin%
\definecolor{currentfill}{rgb}{0.000000,0.000000,0.000000}%
\pgfsetfillcolor{currentfill}%
\pgfsetlinewidth{0.752812pt}%
\definecolor{currentstroke}{rgb}{0.000000,0.000000,0.000000}%
\pgfsetstrokecolor{currentstroke}%
\pgfsetdash{}{0pt}%
\pgfsys@defobject{currentmarker}{\pgfqpoint{0.000000in}{-0.013889in}}{\pgfqpoint{0.000000in}{0.000000in}}{%
\pgfpathmoveto{\pgfqpoint{0.000000in}{0.000000in}}%
\pgfpathlineto{\pgfqpoint{0.000000in}{-0.013889in}}%
\pgfusepath{stroke,fill}%
}%
\begin{pgfscope}%
\pgfsys@transformshift{3.662963in}{0.638889in}%
\pgfsys@useobject{currentmarker}{}%
\end{pgfscope}%
\end{pgfscope}%
\begin{pgfscope}%
\definecolor{textcolor}{rgb}{0.000000,0.000000,0.000000}%
\pgfsetstrokecolor{textcolor}%
\pgfsetfillcolor{textcolor}%
\pgftext[x=3.662963in,y=0.576389in,,top]{\color{textcolor}{\ifdefined\pdftexversion\else\setmainfont{NanumMyeongjo}\rmfamily\fi\fontsize{9.000000}{10.800000}\selectfont\catcode`\^=\active\def^{\ifmmode\sp\else\^{}\fi}\catcode`\%=\active\def%{\%}2022}}%
\end{pgfscope}%
\begin{pgfscope}%
\pgfsetbuttcap%
\pgfsetroundjoin%
\definecolor{currentfill}{rgb}{0.000000,0.000000,0.000000}%
\pgfsetfillcolor{currentfill}%
\pgfsetlinewidth{0.752812pt}%
\definecolor{currentstroke}{rgb}{0.000000,0.000000,0.000000}%
\pgfsetstrokecolor{currentstroke}%
\pgfsetdash{}{0pt}%
\pgfsys@defobject{currentmarker}{\pgfqpoint{0.000000in}{-0.013889in}}{\pgfqpoint{0.000000in}{0.000000in}}{%
\pgfpathmoveto{\pgfqpoint{0.000000in}{0.000000in}}%
\pgfpathlineto{\pgfqpoint{0.000000in}{-0.013889in}}%
\pgfusepath{stroke,fill}%
}%
\begin{pgfscope}%
\pgfsys@transformshift{4.104408in}{0.638889in}%
\pgfsys@useobject{currentmarker}{}%
\end{pgfscope}%
\end{pgfscope}%
\begin{pgfscope}%
\definecolor{textcolor}{rgb}{0.000000,0.000000,0.000000}%
\pgfsetstrokecolor{textcolor}%
\pgfsetfillcolor{textcolor}%
\pgftext[x=4.104408in,y=0.576389in,,top]{\color{textcolor}{\ifdefined\pdftexversion\else\setmainfont{NanumMyeongjo}\rmfamily\fi\fontsize{9.000000}{10.800000}\selectfont\catcode`\^=\active\def^{\ifmmode\sp\else\^{}\fi}\catcode`\%=\active\def%{\%}2023}}%
\end{pgfscope}%
\begin{pgfscope}%
\pgfsetbuttcap%
\pgfsetroundjoin%
\definecolor{currentfill}{rgb}{0.000000,0.000000,0.000000}%
\pgfsetfillcolor{currentfill}%
\pgfsetlinewidth{0.752812pt}%
\definecolor{currentstroke}{rgb}{0.000000,0.000000,0.000000}%
\pgfsetstrokecolor{currentstroke}%
\pgfsetdash{}{0pt}%
\pgfsys@defobject{currentmarker}{\pgfqpoint{0.000000in}{-0.013889in}}{\pgfqpoint{0.000000in}{0.000000in}}{%
\pgfpathmoveto{\pgfqpoint{0.000000in}{0.000000in}}%
\pgfpathlineto{\pgfqpoint{0.000000in}{-0.013889in}}%
\pgfusepath{stroke,fill}%
}%
\begin{pgfscope}%
\pgfsys@transformshift{4.545852in}{0.638889in}%
\pgfsys@useobject{currentmarker}{}%
\end{pgfscope}%
\end{pgfscope}%
\begin{pgfscope}%
\definecolor{textcolor}{rgb}{0.000000,0.000000,0.000000}%
\pgfsetstrokecolor{textcolor}%
\pgfsetfillcolor{textcolor}%
\pgftext[x=4.545852in,y=0.576389in,,top]{\color{textcolor}{\ifdefined\pdftexversion\else\setmainfont{NanumMyeongjo}\rmfamily\fi\fontsize{9.000000}{10.800000}\selectfont\catcode`\^=\active\def^{\ifmmode\sp\else\^{}\fi}\catcode`\%=\active\def%{\%}2024}}%
\end{pgfscope}%
\begin{pgfscope}%
\pgfsetbuttcap%
\pgfsetroundjoin%
\definecolor{currentfill}{rgb}{0.000000,0.000000,0.000000}%
\pgfsetfillcolor{currentfill}%
\pgfsetlinewidth{0.752812pt}%
\definecolor{currentstroke}{rgb}{0.000000,0.000000,0.000000}%
\pgfsetstrokecolor{currentstroke}%
\pgfsetdash{}{0pt}%
\pgfsys@defobject{currentmarker}{\pgfqpoint{0.000000in}{-0.013889in}}{\pgfqpoint{0.000000in}{0.000000in}}{%
\pgfpathmoveto{\pgfqpoint{0.000000in}{0.000000in}}%
\pgfpathlineto{\pgfqpoint{0.000000in}{-0.013889in}}%
\pgfusepath{stroke,fill}%
}%
\begin{pgfscope}%
\pgfsys@transformshift{4.987297in}{0.638889in}%
\pgfsys@useobject{currentmarker}{}%
\end{pgfscope}%
\end{pgfscope}%
\begin{pgfscope}%
\definecolor{textcolor}{rgb}{0.000000,0.000000,0.000000}%
\pgfsetstrokecolor{textcolor}%
\pgfsetfillcolor{textcolor}%
\pgftext[x=4.987297in,y=0.576389in,,top]{\color{textcolor}{\ifdefined\pdftexversion\else\setmainfont{NanumMyeongjo}\rmfamily\fi\fontsize{9.000000}{10.800000}\selectfont\catcode`\^=\active\def^{\ifmmode\sp\else\^{}\fi}\catcode`\%=\active\def%{\%}2025}}%
\end{pgfscope}%
\begin{pgfscope}%
\pgfpathrectangle{\pgfqpoint{0.781250in}{0.638889in}}{\pgfqpoint{4.843750in}{2.172222in}}%
\pgfusepath{clip}%
\pgfsetbuttcap%
\pgfsetroundjoin%
\pgfsetlinewidth{0.602250pt}%
\definecolor{currentstroke}{rgb}{0.690196,0.690196,0.690196}%
\pgfsetstrokecolor{currentstroke}%
\pgfsetstrokeopacity{0.450000}%
\pgfsetdash{{2.220000pt}{0.960000pt}}{0.000000pt}%
\pgfpathmoveto{\pgfqpoint{0.781250in}{0.638889in}}%
\pgfpathlineto{\pgfqpoint{5.625000in}{0.638889in}}%
\pgfusepath{stroke}%
\end{pgfscope}%
\begin{pgfscope}%
\pgfsetbuttcap%
\pgfsetroundjoin%
\definecolor{currentfill}{rgb}{0.000000,0.000000,0.000000}%
\pgfsetfillcolor{currentfill}%
\pgfsetlinewidth{0.752812pt}%
\definecolor{currentstroke}{rgb}{0.000000,0.000000,0.000000}%
\pgfsetstrokecolor{currentstroke}%
\pgfsetdash{}{0pt}%
\pgfsys@defobject{currentmarker}{\pgfqpoint{-0.013889in}{0.000000in}}{\pgfqpoint{-0.000000in}{0.000000in}}{%
\pgfpathmoveto{\pgfqpoint{-0.000000in}{0.000000in}}%
\pgfpathlineto{\pgfqpoint{-0.013889in}{0.000000in}}%
\pgfusepath{stroke,fill}%
}%
\begin{pgfscope}%
\pgfsys@transformshift{0.781250in}{0.638889in}%
\pgfsys@useobject{currentmarker}{}%
\end{pgfscope}%
\end{pgfscope}%
\begin{pgfscope}%
\definecolor{textcolor}{rgb}{0.000000,0.000000,0.000000}%
\pgfsetstrokecolor{textcolor}%
\pgfsetfillcolor{textcolor}%
\pgftext[x=0.651611in, y=0.588962in, left, base]{\color{textcolor}{\ifdefined\pdftexversion\else\setmainfont{NanumMyeongjo}\rmfamily\fi\fontsize{9.000000}{10.800000}\selectfont\catcode`\^=\active\def^{\ifmmode\sp\else\^{}\fi}\catcode`\%=\active\def%{\%}0}}%
\end{pgfscope}%
\begin{pgfscope}%
\pgfpathrectangle{\pgfqpoint{0.781250in}{0.638889in}}{\pgfqpoint{4.843750in}{2.172222in}}%
\pgfusepath{clip}%
\pgfsetbuttcap%
\pgfsetroundjoin%
\pgfsetlinewidth{0.602250pt}%
\definecolor{currentstroke}{rgb}{0.690196,0.690196,0.690196}%
\pgfsetstrokecolor{currentstroke}%
\pgfsetstrokeopacity{0.450000}%
\pgfsetdash{{2.220000pt}{0.960000pt}}{0.000000pt}%
\pgfpathmoveto{\pgfqpoint{0.781250in}{0.949206in}}%
\pgfpathlineto{\pgfqpoint{5.625000in}{0.949206in}}%
\pgfusepath{stroke}%
\end{pgfscope}%
\begin{pgfscope}%
\pgfsetbuttcap%
\pgfsetroundjoin%
\definecolor{currentfill}{rgb}{0.000000,0.000000,0.000000}%
\pgfsetfillcolor{currentfill}%
\pgfsetlinewidth{0.752812pt}%
\definecolor{currentstroke}{rgb}{0.000000,0.000000,0.000000}%
\pgfsetstrokecolor{currentstroke}%
\pgfsetdash{}{0pt}%
\pgfsys@defobject{currentmarker}{\pgfqpoint{-0.013889in}{0.000000in}}{\pgfqpoint{-0.000000in}{0.000000in}}{%
\pgfpathmoveto{\pgfqpoint{-0.000000in}{0.000000in}}%
\pgfpathlineto{\pgfqpoint{-0.013889in}{0.000000in}}%
\pgfusepath{stroke,fill}%
}%
\begin{pgfscope}%
\pgfsys@transformshift{0.781250in}{0.949206in}%
\pgfsys@useobject{currentmarker}{}%
\end{pgfscope}%
\end{pgfscope}%
\begin{pgfscope}%
\definecolor{textcolor}{rgb}{0.000000,0.000000,0.000000}%
\pgfsetstrokecolor{textcolor}%
\pgfsetfillcolor{textcolor}%
\pgftext[x=0.517334in, y=0.899280in, left, base]{\color{textcolor}{\ifdefined\pdftexversion\else\setmainfont{NanumMyeongjo}\rmfamily\fi\fontsize{9.000000}{10.800000}\selectfont\catcode`\^=\active\def^{\ifmmode\sp\else\^{}\fi}\catcode`\%=\active\def%{\%}100}}%
\end{pgfscope}%
\begin{pgfscope}%
\pgfpathrectangle{\pgfqpoint{0.781250in}{0.638889in}}{\pgfqpoint{4.843750in}{2.172222in}}%
\pgfusepath{clip}%
\pgfsetbuttcap%
\pgfsetroundjoin%
\pgfsetlinewidth{0.602250pt}%
\definecolor{currentstroke}{rgb}{0.690196,0.690196,0.690196}%
\pgfsetstrokecolor{currentstroke}%
\pgfsetstrokeopacity{0.450000}%
\pgfsetdash{{2.220000pt}{0.960000pt}}{0.000000pt}%
\pgfpathmoveto{\pgfqpoint{0.781250in}{1.259524in}}%
\pgfpathlineto{\pgfqpoint{5.625000in}{1.259524in}}%
\pgfusepath{stroke}%
\end{pgfscope}%
\begin{pgfscope}%
\pgfsetbuttcap%
\pgfsetroundjoin%
\definecolor{currentfill}{rgb}{0.000000,0.000000,0.000000}%
\pgfsetfillcolor{currentfill}%
\pgfsetlinewidth{0.752812pt}%
\definecolor{currentstroke}{rgb}{0.000000,0.000000,0.000000}%
\pgfsetstrokecolor{currentstroke}%
\pgfsetdash{}{0pt}%
\pgfsys@defobject{currentmarker}{\pgfqpoint{-0.013889in}{0.000000in}}{\pgfqpoint{-0.000000in}{0.000000in}}{%
\pgfpathmoveto{\pgfqpoint{-0.000000in}{0.000000in}}%
\pgfpathlineto{\pgfqpoint{-0.013889in}{0.000000in}}%
\pgfusepath{stroke,fill}%
}%
\begin{pgfscope}%
\pgfsys@transformshift{0.781250in}{1.259524in}%
\pgfsys@useobject{currentmarker}{}%
\end{pgfscope}%
\end{pgfscope}%
\begin{pgfscope}%
\definecolor{textcolor}{rgb}{0.000000,0.000000,0.000000}%
\pgfsetstrokecolor{textcolor}%
\pgfsetfillcolor{textcolor}%
\pgftext[x=0.517334in, y=1.209597in, left, base]{\color{textcolor}{\ifdefined\pdftexversion\else\setmainfont{NanumMyeongjo}\rmfamily\fi\fontsize{9.000000}{10.800000}\selectfont\catcode`\^=\active\def^{\ifmmode\sp\else\^{}\fi}\catcode`\%=\active\def%{\%}200}}%
\end{pgfscope}%
\begin{pgfscope}%
\pgfpathrectangle{\pgfqpoint{0.781250in}{0.638889in}}{\pgfqpoint{4.843750in}{2.172222in}}%
\pgfusepath{clip}%
\pgfsetbuttcap%
\pgfsetroundjoin%
\pgfsetlinewidth{0.602250pt}%
\definecolor{currentstroke}{rgb}{0.690196,0.690196,0.690196}%
\pgfsetstrokecolor{currentstroke}%
\pgfsetstrokeopacity{0.450000}%
\pgfsetdash{{2.220000pt}{0.960000pt}}{0.000000pt}%
\pgfpathmoveto{\pgfqpoint{0.781250in}{1.569841in}}%
\pgfpathlineto{\pgfqpoint{5.625000in}{1.569841in}}%
\pgfusepath{stroke}%
\end{pgfscope}%
\begin{pgfscope}%
\pgfsetbuttcap%
\pgfsetroundjoin%
\definecolor{currentfill}{rgb}{0.000000,0.000000,0.000000}%
\pgfsetfillcolor{currentfill}%
\pgfsetlinewidth{0.752812pt}%
\definecolor{currentstroke}{rgb}{0.000000,0.000000,0.000000}%
\pgfsetstrokecolor{currentstroke}%
\pgfsetdash{}{0pt}%
\pgfsys@defobject{currentmarker}{\pgfqpoint{-0.013889in}{0.000000in}}{\pgfqpoint{-0.000000in}{0.000000in}}{%
\pgfpathmoveto{\pgfqpoint{-0.000000in}{0.000000in}}%
\pgfpathlineto{\pgfqpoint{-0.013889in}{0.000000in}}%
\pgfusepath{stroke,fill}%
}%
\begin{pgfscope}%
\pgfsys@transformshift{0.781250in}{1.569841in}%
\pgfsys@useobject{currentmarker}{}%
\end{pgfscope}%
\end{pgfscope}%
\begin{pgfscope}%
\definecolor{textcolor}{rgb}{0.000000,0.000000,0.000000}%
\pgfsetstrokecolor{textcolor}%
\pgfsetfillcolor{textcolor}%
\pgftext[x=0.517334in, y=1.519915in, left, base]{\color{textcolor}{\ifdefined\pdftexversion\else\setmainfont{NanumMyeongjo}\rmfamily\fi\fontsize{9.000000}{10.800000}\selectfont\catcode`\^=\active\def^{\ifmmode\sp\else\^{}\fi}\catcode`\%=\active\def%{\%}300}}%
\end{pgfscope}%
\begin{pgfscope}%
\pgfpathrectangle{\pgfqpoint{0.781250in}{0.638889in}}{\pgfqpoint{4.843750in}{2.172222in}}%
\pgfusepath{clip}%
\pgfsetbuttcap%
\pgfsetroundjoin%
\pgfsetlinewidth{0.602250pt}%
\definecolor{currentstroke}{rgb}{0.690196,0.690196,0.690196}%
\pgfsetstrokecolor{currentstroke}%
\pgfsetstrokeopacity{0.450000}%
\pgfsetdash{{2.220000pt}{0.960000pt}}{0.000000pt}%
\pgfpathmoveto{\pgfqpoint{0.781250in}{1.880159in}}%
\pgfpathlineto{\pgfqpoint{5.625000in}{1.880159in}}%
\pgfusepath{stroke}%
\end{pgfscope}%
\begin{pgfscope}%
\pgfsetbuttcap%
\pgfsetroundjoin%
\definecolor{currentfill}{rgb}{0.000000,0.000000,0.000000}%
\pgfsetfillcolor{currentfill}%
\pgfsetlinewidth{0.752812pt}%
\definecolor{currentstroke}{rgb}{0.000000,0.000000,0.000000}%
\pgfsetstrokecolor{currentstroke}%
\pgfsetdash{}{0pt}%
\pgfsys@defobject{currentmarker}{\pgfqpoint{-0.013889in}{0.000000in}}{\pgfqpoint{-0.000000in}{0.000000in}}{%
\pgfpathmoveto{\pgfqpoint{-0.000000in}{0.000000in}}%
\pgfpathlineto{\pgfqpoint{-0.013889in}{0.000000in}}%
\pgfusepath{stroke,fill}%
}%
\begin{pgfscope}%
\pgfsys@transformshift{0.781250in}{1.880159in}%
\pgfsys@useobject{currentmarker}{}%
\end{pgfscope}%
\end{pgfscope}%
\begin{pgfscope}%
\definecolor{textcolor}{rgb}{0.000000,0.000000,0.000000}%
\pgfsetstrokecolor{textcolor}%
\pgfsetfillcolor{textcolor}%
\pgftext[x=0.517334in, y=1.830232in, left, base]{\color{textcolor}{\ifdefined\pdftexversion\else\setmainfont{NanumMyeongjo}\rmfamily\fi\fontsize{9.000000}{10.800000}\selectfont\catcode`\^=\active\def^{\ifmmode\sp\else\^{}\fi}\catcode`\%=\active\def%{\%}400}}%
\end{pgfscope}%
\begin{pgfscope}%
\pgfpathrectangle{\pgfqpoint{0.781250in}{0.638889in}}{\pgfqpoint{4.843750in}{2.172222in}}%
\pgfusepath{clip}%
\pgfsetbuttcap%
\pgfsetroundjoin%
\pgfsetlinewidth{0.602250pt}%
\definecolor{currentstroke}{rgb}{0.690196,0.690196,0.690196}%
\pgfsetstrokecolor{currentstroke}%
\pgfsetstrokeopacity{0.450000}%
\pgfsetdash{{2.220000pt}{0.960000pt}}{0.000000pt}%
\pgfpathmoveto{\pgfqpoint{0.781250in}{2.190476in}}%
\pgfpathlineto{\pgfqpoint{5.625000in}{2.190476in}}%
\pgfusepath{stroke}%
\end{pgfscope}%
\begin{pgfscope}%
\pgfsetbuttcap%
\pgfsetroundjoin%
\definecolor{currentfill}{rgb}{0.000000,0.000000,0.000000}%
\pgfsetfillcolor{currentfill}%
\pgfsetlinewidth{0.752812pt}%
\definecolor{currentstroke}{rgb}{0.000000,0.000000,0.000000}%
\pgfsetstrokecolor{currentstroke}%
\pgfsetdash{}{0pt}%
\pgfsys@defobject{currentmarker}{\pgfqpoint{-0.013889in}{0.000000in}}{\pgfqpoint{-0.000000in}{0.000000in}}{%
\pgfpathmoveto{\pgfqpoint{-0.000000in}{0.000000in}}%
\pgfpathlineto{\pgfqpoint{-0.013889in}{0.000000in}}%
\pgfusepath{stroke,fill}%
}%
\begin{pgfscope}%
\pgfsys@transformshift{0.781250in}{2.190476in}%
\pgfsys@useobject{currentmarker}{}%
\end{pgfscope}%
\end{pgfscope}%
\begin{pgfscope}%
\definecolor{textcolor}{rgb}{0.000000,0.000000,0.000000}%
\pgfsetstrokecolor{textcolor}%
\pgfsetfillcolor{textcolor}%
\pgftext[x=0.517334in, y=2.140549in, left, base]{\color{textcolor}{\ifdefined\pdftexversion\else\setmainfont{NanumMyeongjo}\rmfamily\fi\fontsize{9.000000}{10.800000}\selectfont\catcode`\^=\active\def^{\ifmmode\sp\else\^{}\fi}\catcode`\%=\active\def%{\%}500}}%
\end{pgfscope}%
\begin{pgfscope}%
\pgfpathrectangle{\pgfqpoint{0.781250in}{0.638889in}}{\pgfqpoint{4.843750in}{2.172222in}}%
\pgfusepath{clip}%
\pgfsetbuttcap%
\pgfsetroundjoin%
\pgfsetlinewidth{0.602250pt}%
\definecolor{currentstroke}{rgb}{0.690196,0.690196,0.690196}%
\pgfsetstrokecolor{currentstroke}%
\pgfsetstrokeopacity{0.450000}%
\pgfsetdash{{2.220000pt}{0.960000pt}}{0.000000pt}%
\pgfpathmoveto{\pgfqpoint{0.781250in}{2.500794in}}%
\pgfpathlineto{\pgfqpoint{5.625000in}{2.500794in}}%
\pgfusepath{stroke}%
\end{pgfscope}%
\begin{pgfscope}%
\pgfsetbuttcap%
\pgfsetroundjoin%
\definecolor{currentfill}{rgb}{0.000000,0.000000,0.000000}%
\pgfsetfillcolor{currentfill}%
\pgfsetlinewidth{0.752812pt}%
\definecolor{currentstroke}{rgb}{0.000000,0.000000,0.000000}%
\pgfsetstrokecolor{currentstroke}%
\pgfsetdash{}{0pt}%
\pgfsys@defobject{currentmarker}{\pgfqpoint{-0.013889in}{0.000000in}}{\pgfqpoint{-0.000000in}{0.000000in}}{%
\pgfpathmoveto{\pgfqpoint{-0.000000in}{0.000000in}}%
\pgfpathlineto{\pgfqpoint{-0.013889in}{0.000000in}}%
\pgfusepath{stroke,fill}%
}%
\begin{pgfscope}%
\pgfsys@transformshift{0.781250in}{2.500794in}%
\pgfsys@useobject{currentmarker}{}%
\end{pgfscope}%
\end{pgfscope}%
\begin{pgfscope}%
\definecolor{textcolor}{rgb}{0.000000,0.000000,0.000000}%
\pgfsetstrokecolor{textcolor}%
\pgfsetfillcolor{textcolor}%
\pgftext[x=0.517334in, y=2.450867in, left, base]{\color{textcolor}{\ifdefined\pdftexversion\else\setmainfont{NanumMyeongjo}\rmfamily\fi\fontsize{9.000000}{10.800000}\selectfont\catcode`\^=\active\def^{\ifmmode\sp\else\^{}\fi}\catcode`\%=\active\def%{\%}600}}%
\end{pgfscope}%
\begin{pgfscope}%
\pgfpathrectangle{\pgfqpoint{0.781250in}{0.638889in}}{\pgfqpoint{4.843750in}{2.172222in}}%
\pgfusepath{clip}%
\pgfsetbuttcap%
\pgfsetroundjoin%
\pgfsetlinewidth{0.602250pt}%
\definecolor{currentstroke}{rgb}{0.690196,0.690196,0.690196}%
\pgfsetstrokecolor{currentstroke}%
\pgfsetstrokeopacity{0.450000}%
\pgfsetdash{{2.220000pt}{0.960000pt}}{0.000000pt}%
\pgfpathmoveto{\pgfqpoint{0.781250in}{2.811111in}}%
\pgfpathlineto{\pgfqpoint{5.625000in}{2.811111in}}%
\pgfusepath{stroke}%
\end{pgfscope}%
\begin{pgfscope}%
\pgfsetbuttcap%
\pgfsetroundjoin%
\definecolor{currentfill}{rgb}{0.000000,0.000000,0.000000}%
\pgfsetfillcolor{currentfill}%
\pgfsetlinewidth{0.752812pt}%
\definecolor{currentstroke}{rgb}{0.000000,0.000000,0.000000}%
\pgfsetstrokecolor{currentstroke}%
\pgfsetdash{}{0pt}%
\pgfsys@defobject{currentmarker}{\pgfqpoint{-0.013889in}{0.000000in}}{\pgfqpoint{-0.000000in}{0.000000in}}{%
\pgfpathmoveto{\pgfqpoint{-0.000000in}{0.000000in}}%
\pgfpathlineto{\pgfqpoint{-0.013889in}{0.000000in}}%
\pgfusepath{stroke,fill}%
}%
\begin{pgfscope}%
\pgfsys@transformshift{0.781250in}{2.811111in}%
\pgfsys@useobject{currentmarker}{}%
\end{pgfscope}%
\end{pgfscope}%
\begin{pgfscope}%
\definecolor{textcolor}{rgb}{0.000000,0.000000,0.000000}%
\pgfsetstrokecolor{textcolor}%
\pgfsetfillcolor{textcolor}%
\pgftext[x=0.517334in, y=2.761184in, left, base]{\color{textcolor}{\ifdefined\pdftexversion\else\setmainfont{NanumMyeongjo}\rmfamily\fi\fontsize{9.000000}{10.800000}\selectfont\catcode`\^=\active\def^{\ifmmode\sp\else\^{}\fi}\catcode`\%=\active\def%{\%}700}}%
\end{pgfscope}%
\begin{pgfscope}%
\pgfsetrectcap%
\pgfsetmiterjoin%
\pgfsetlinewidth{0.752812pt}%
\definecolor{currentstroke}{rgb}{0.000000,0.000000,0.000000}%
\pgfsetstrokecolor{currentstroke}%
\pgfsetdash{}{0pt}%
\pgfpathmoveto{\pgfqpoint{0.781250in}{0.638889in}}%
\pgfpathlineto{\pgfqpoint{0.781250in}{2.811111in}}%
\pgfusepath{stroke}%
\end{pgfscope}%
\begin{pgfscope}%
\pgfsetrectcap%
\pgfsetmiterjoin%
\pgfsetlinewidth{0.752812pt}%
\definecolor{currentstroke}{rgb}{0.000000,0.000000,0.000000}%
\pgfsetstrokecolor{currentstroke}%
\pgfsetdash{}{0pt}%
\pgfpathmoveto{\pgfqpoint{0.781250in}{0.638889in}}%
\pgfpathlineto{\pgfqpoint{5.625000in}{0.638889in}}%
\pgfusepath{stroke}%
\end{pgfscope}%
\begin{pgfscope}%
\pgfpathrectangle{\pgfqpoint{0.781250in}{0.638889in}}{\pgfqpoint{4.843750in}{2.172222in}}%
\pgfusepath{clip}%
\pgfsetbuttcap%
\pgfsetmiterjoin%
\definecolor{currentfill}{rgb}{0.227451,0.192157,0.427451}%
\pgfsetfillcolor{currentfill}%
\pgfsetlinewidth{0.100375pt}%
\definecolor{currentstroke}{rgb}{0.266667,0.266667,0.266667}%
\pgfsetstrokecolor{currentstroke}%
\pgfsetdash{}{0pt}%
\pgfpathmoveto{\pgfqpoint{1.001420in}{0.638889in}}%
\pgfpathlineto{\pgfqpoint{1.027171in}{0.638889in}}%
\pgfpathlineto{\pgfqpoint{1.027171in}{0.797461in}}%
\pgfpathlineto{\pgfqpoint{1.001420in}{0.797461in}}%
\pgfpathlineto{\pgfqpoint{1.001420in}{0.638889in}}%
\pgfpathclose%
\pgfusepath{stroke,fill}%
\end{pgfscope}%
\begin{pgfscope}%
\pgfpathrectangle{\pgfqpoint{0.781250in}{0.638889in}}{\pgfqpoint{4.843750in}{2.172222in}}%
\pgfusepath{clip}%
\pgfsetbuttcap%
\pgfsetmiterjoin%
\definecolor{currentfill}{rgb}{0.227451,0.192157,0.427451}%
\pgfsetfillcolor{currentfill}%
\pgfsetlinewidth{0.100375pt}%
\definecolor{currentstroke}{rgb}{0.266667,0.266667,0.266667}%
\pgfsetstrokecolor{currentstroke}%
\pgfsetdash{}{0pt}%
\pgfpathmoveto{\pgfqpoint{1.038207in}{0.638889in}}%
\pgfpathlineto{\pgfqpoint{1.063958in}{0.638889in}}%
\pgfpathlineto{\pgfqpoint{1.063958in}{0.764878in}}%
\pgfpathlineto{\pgfqpoint{1.038207in}{0.764878in}}%
\pgfpathlineto{\pgfqpoint{1.038207in}{0.638889in}}%
\pgfpathclose%
\pgfusepath{stroke,fill}%
\end{pgfscope}%
\begin{pgfscope}%
\pgfpathrectangle{\pgfqpoint{0.781250in}{0.638889in}}{\pgfqpoint{4.843750in}{2.172222in}}%
\pgfusepath{clip}%
\pgfsetbuttcap%
\pgfsetmiterjoin%
\definecolor{currentfill}{rgb}{0.227451,0.192157,0.427451}%
\pgfsetfillcolor{currentfill}%
\pgfsetlinewidth{0.100375pt}%
\definecolor{currentstroke}{rgb}{0.266667,0.266667,0.266667}%
\pgfsetstrokecolor{currentstroke}%
\pgfsetdash{}{0pt}%
\pgfpathmoveto{\pgfqpoint{1.074995in}{0.638889in}}%
\pgfpathlineto{\pgfqpoint{1.100745in}{0.638889in}}%
\pgfpathlineto{\pgfqpoint{1.100745in}{0.800254in}}%
\pgfpathlineto{\pgfqpoint{1.074995in}{0.800254in}}%
\pgfpathlineto{\pgfqpoint{1.074995in}{0.638889in}}%
\pgfpathclose%
\pgfusepath{stroke,fill}%
\end{pgfscope}%
\begin{pgfscope}%
\pgfpathrectangle{\pgfqpoint{0.781250in}{0.638889in}}{\pgfqpoint{4.843750in}{2.172222in}}%
\pgfusepath{clip}%
\pgfsetbuttcap%
\pgfsetmiterjoin%
\definecolor{currentfill}{rgb}{0.227451,0.192157,0.427451}%
\pgfsetfillcolor{currentfill}%
\pgfsetlinewidth{0.100375pt}%
\definecolor{currentstroke}{rgb}{0.266667,0.266667,0.266667}%
\pgfsetstrokecolor{currentstroke}%
\pgfsetdash{}{0pt}%
\pgfpathmoveto{\pgfqpoint{1.111782in}{0.638889in}}%
\pgfpathlineto{\pgfqpoint{1.137533in}{0.638889in}}%
\pgfpathlineto{\pgfqpoint{1.137533in}{1.195909in}}%
\pgfpathlineto{\pgfqpoint{1.111782in}{1.195909in}}%
\pgfpathlineto{\pgfqpoint{1.111782in}{0.638889in}}%
\pgfpathclose%
\pgfusepath{stroke,fill}%
\end{pgfscope}%
\begin{pgfscope}%
\pgfpathrectangle{\pgfqpoint{0.781250in}{0.638889in}}{\pgfqpoint{4.843750in}{2.172222in}}%
\pgfusepath{clip}%
\pgfsetbuttcap%
\pgfsetmiterjoin%
\definecolor{currentfill}{rgb}{0.227451,0.192157,0.427451}%
\pgfsetfillcolor{currentfill}%
\pgfsetlinewidth{0.100375pt}%
\definecolor{currentstroke}{rgb}{0.266667,0.266667,0.266667}%
\pgfsetstrokecolor{currentstroke}%
\pgfsetdash{}{0pt}%
\pgfpathmoveto{\pgfqpoint{1.148569in}{0.638889in}}%
\pgfpathlineto{\pgfqpoint{1.174320in}{0.638889in}}%
\pgfpathlineto{\pgfqpoint{1.174320in}{0.948896in}}%
\pgfpathlineto{\pgfqpoint{1.148569in}{0.948896in}}%
\pgfpathlineto{\pgfqpoint{1.148569in}{0.638889in}}%
\pgfpathclose%
\pgfusepath{stroke,fill}%
\end{pgfscope}%
\begin{pgfscope}%
\pgfpathrectangle{\pgfqpoint{0.781250in}{0.638889in}}{\pgfqpoint{4.843750in}{2.172222in}}%
\pgfusepath{clip}%
\pgfsetbuttcap%
\pgfsetmiterjoin%
\definecolor{currentfill}{rgb}{0.164706,0.615686,0.560784}%
\pgfsetfillcolor{currentfill}%
\pgfsetlinewidth{0.100375pt}%
\definecolor{currentstroke}{rgb}{0.266667,0.266667,0.266667}%
\pgfsetstrokecolor{currentstroke}%
\pgfsetdash{}{0pt}%
\pgfpathmoveto{\pgfqpoint{1.185356in}{0.638889in}}%
\pgfpathlineto{\pgfqpoint{1.211107in}{0.638889in}}%
\pgfpathlineto{\pgfqpoint{1.211107in}{0.769222in}}%
\pgfpathlineto{\pgfqpoint{1.185356in}{0.769222in}}%
\pgfpathlineto{\pgfqpoint{1.185356in}{0.638889in}}%
\pgfpathclose%
\pgfusepath{stroke,fill}%
\end{pgfscope}%
\begin{pgfscope}%
\pgfpathrectangle{\pgfqpoint{0.781250in}{0.638889in}}{\pgfqpoint{4.843750in}{2.172222in}}%
\pgfusepath{clip}%
\pgfsetbuttcap%
\pgfsetmiterjoin%
\definecolor{currentfill}{rgb}{0.164706,0.615686,0.560784}%
\pgfsetfillcolor{currentfill}%
\pgfsetlinewidth{0.100375pt}%
\definecolor{currentstroke}{rgb}{0.266667,0.266667,0.266667}%
\pgfsetstrokecolor{currentstroke}%
\pgfsetdash{}{0pt}%
\pgfpathmoveto{\pgfqpoint{1.222143in}{0.638889in}}%
\pgfpathlineto{\pgfqpoint{1.247894in}{0.638889in}}%
\pgfpathlineto{\pgfqpoint{1.247894in}{1.323760in}}%
\pgfpathlineto{\pgfqpoint{1.222143in}{1.323760in}}%
\pgfpathlineto{\pgfqpoint{1.222143in}{0.638889in}}%
\pgfpathclose%
\pgfusepath{stroke,fill}%
\end{pgfscope}%
\begin{pgfscope}%
\pgfpathrectangle{\pgfqpoint{0.781250in}{0.638889in}}{\pgfqpoint{4.843750in}{2.172222in}}%
\pgfusepath{clip}%
\pgfsetbuttcap%
\pgfsetmiterjoin%
\definecolor{currentfill}{rgb}{0.164706,0.615686,0.560784}%
\pgfsetfillcolor{currentfill}%
\pgfsetlinewidth{0.100375pt}%
\definecolor{currentstroke}{rgb}{0.266667,0.266667,0.266667}%
\pgfsetstrokecolor{currentstroke}%
\pgfsetdash{}{0pt}%
\pgfpathmoveto{\pgfqpoint{1.258930in}{0.638889in}}%
\pgfpathlineto{\pgfqpoint{1.284681in}{0.638889in}}%
\pgfpathlineto{\pgfqpoint{1.284681in}{0.732605in}}%
\pgfpathlineto{\pgfqpoint{1.258930in}{0.732605in}}%
\pgfpathlineto{\pgfqpoint{1.258930in}{0.638889in}}%
\pgfpathclose%
\pgfusepath{stroke,fill}%
\end{pgfscope}%
\begin{pgfscope}%
\pgfpathrectangle{\pgfqpoint{0.781250in}{0.638889in}}{\pgfqpoint{4.843750in}{2.172222in}}%
\pgfusepath{clip}%
\pgfsetbuttcap%
\pgfsetmiterjoin%
\definecolor{currentfill}{rgb}{0.164706,0.615686,0.560784}%
\pgfsetfillcolor{currentfill}%
\pgfsetlinewidth{0.100375pt}%
\definecolor{currentstroke}{rgb}{0.266667,0.266667,0.266667}%
\pgfsetstrokecolor{currentstroke}%
\pgfsetdash{}{0pt}%
\pgfpathmoveto{\pgfqpoint{1.295717in}{0.638889in}}%
\pgfpathlineto{\pgfqpoint{1.321468in}{0.638889in}}%
\pgfpathlineto{\pgfqpoint{1.321468in}{1.071161in}}%
\pgfpathlineto{\pgfqpoint{1.295717in}{1.071161in}}%
\pgfpathlineto{\pgfqpoint{1.295717in}{0.638889in}}%
\pgfpathclose%
\pgfusepath{stroke,fill}%
\end{pgfscope}%
\begin{pgfscope}%
\pgfpathrectangle{\pgfqpoint{0.781250in}{0.638889in}}{\pgfqpoint{4.843750in}{2.172222in}}%
\pgfusepath{clip}%
\pgfsetbuttcap%
\pgfsetmiterjoin%
\definecolor{currentfill}{rgb}{0.227451,0.192157,0.427451}%
\pgfsetfillcolor{currentfill}%
\pgfsetlinewidth{0.100375pt}%
\definecolor{currentstroke}{rgb}{0.266667,0.266667,0.266667}%
\pgfsetstrokecolor{currentstroke}%
\pgfsetdash{}{0pt}%
\pgfpathmoveto{\pgfqpoint{1.332504in}{0.638889in}}%
\pgfpathlineto{\pgfqpoint{1.358255in}{0.638889in}}%
\pgfpathlineto{\pgfqpoint{1.358255in}{1.038267in}}%
\pgfpathlineto{\pgfqpoint{1.332504in}{1.038267in}}%
\pgfpathlineto{\pgfqpoint{1.332504in}{0.638889in}}%
\pgfpathclose%
\pgfusepath{stroke,fill}%
\end{pgfscope}%
\begin{pgfscope}%
\pgfpathrectangle{\pgfqpoint{0.781250in}{0.638889in}}{\pgfqpoint{4.843750in}{2.172222in}}%
\pgfusepath{clip}%
\pgfsetbuttcap%
\pgfsetmiterjoin%
\definecolor{currentfill}{rgb}{0.227451,0.192157,0.427451}%
\pgfsetfillcolor{currentfill}%
\pgfsetlinewidth{0.100375pt}%
\definecolor{currentstroke}{rgb}{0.266667,0.266667,0.266667}%
\pgfsetstrokecolor{currentstroke}%
\pgfsetdash{}{0pt}%
\pgfpathmoveto{\pgfqpoint{1.369291in}{0.638889in}}%
\pgfpathlineto{\pgfqpoint{1.395042in}{0.638889in}}%
\pgfpathlineto{\pgfqpoint{1.395042in}{0.746259in}}%
\pgfpathlineto{\pgfqpoint{1.369291in}{0.746259in}}%
\pgfpathlineto{\pgfqpoint{1.369291in}{0.638889in}}%
\pgfpathclose%
\pgfusepath{stroke,fill}%
\end{pgfscope}%
\begin{pgfscope}%
\pgfpathrectangle{\pgfqpoint{0.781250in}{0.638889in}}{\pgfqpoint{4.843750in}{2.172222in}}%
\pgfusepath{clip}%
\pgfsetbuttcap%
\pgfsetmiterjoin%
\definecolor{currentfill}{rgb}{0.227451,0.192157,0.427451}%
\pgfsetfillcolor{currentfill}%
\pgfsetlinewidth{0.100375pt}%
\definecolor{currentstroke}{rgb}{0.266667,0.266667,0.266667}%
\pgfsetstrokecolor{currentstroke}%
\pgfsetdash{}{0pt}%
\pgfpathmoveto{\pgfqpoint{1.406078in}{0.638889in}}%
\pgfpathlineto{\pgfqpoint{1.431829in}{0.638889in}}%
\pgfpathlineto{\pgfqpoint{1.431829in}{0.815460in}}%
\pgfpathlineto{\pgfqpoint{1.406078in}{0.815460in}}%
\pgfpathlineto{\pgfqpoint{1.406078in}{0.638889in}}%
\pgfpathclose%
\pgfusepath{stroke,fill}%
\end{pgfscope}%
\begin{pgfscope}%
\pgfpathrectangle{\pgfqpoint{0.781250in}{0.638889in}}{\pgfqpoint{4.843750in}{2.172222in}}%
\pgfusepath{clip}%
\pgfsetbuttcap%
\pgfsetmiterjoin%
\definecolor{currentfill}{rgb}{0.227451,0.192157,0.427451}%
\pgfsetfillcolor{currentfill}%
\pgfsetlinewidth{0.100375pt}%
\definecolor{currentstroke}{rgb}{0.266667,0.266667,0.266667}%
\pgfsetstrokecolor{currentstroke}%
\pgfsetdash{}{0pt}%
\pgfpathmoveto{\pgfqpoint{1.442865in}{0.638889in}}%
\pgfpathlineto{\pgfqpoint{1.468616in}{0.638889in}}%
\pgfpathlineto{\pgfqpoint{1.468616in}{0.696918in}}%
\pgfpathlineto{\pgfqpoint{1.442865in}{0.696918in}}%
\pgfpathlineto{\pgfqpoint{1.442865in}{0.638889in}}%
\pgfpathclose%
\pgfusepath{stroke,fill}%
\end{pgfscope}%
\begin{pgfscope}%
\pgfpathrectangle{\pgfqpoint{0.781250in}{0.638889in}}{\pgfqpoint{4.843750in}{2.172222in}}%
\pgfusepath{clip}%
\pgfsetbuttcap%
\pgfsetmiterjoin%
\definecolor{currentfill}{rgb}{0.227451,0.192157,0.427451}%
\pgfsetfillcolor{currentfill}%
\pgfsetlinewidth{0.100375pt}%
\definecolor{currentstroke}{rgb}{0.266667,0.266667,0.266667}%
\pgfsetstrokecolor{currentstroke}%
\pgfsetdash{}{0pt}%
\pgfpathmoveto{\pgfqpoint{1.479652in}{0.638889in}}%
\pgfpathlineto{\pgfqpoint{1.505403in}{0.638889in}}%
\pgfpathlineto{\pgfqpoint{1.505403in}{0.756189in}}%
\pgfpathlineto{\pgfqpoint{1.479652in}{0.756189in}}%
\pgfpathlineto{\pgfqpoint{1.479652in}{0.638889in}}%
\pgfpathclose%
\pgfusepath{stroke,fill}%
\end{pgfscope}%
\begin{pgfscope}%
\pgfpathrectangle{\pgfqpoint{0.781250in}{0.638889in}}{\pgfqpoint{4.843750in}{2.172222in}}%
\pgfusepath{clip}%
\pgfsetbuttcap%
\pgfsetmiterjoin%
\definecolor{currentfill}{rgb}{0.227451,0.192157,0.427451}%
\pgfsetfillcolor{currentfill}%
\pgfsetlinewidth{0.100375pt}%
\definecolor{currentstroke}{rgb}{0.266667,0.266667,0.266667}%
\pgfsetstrokecolor{currentstroke}%
\pgfsetdash{}{0pt}%
\pgfpathmoveto{\pgfqpoint{1.516439in}{0.638889in}}%
\pgfpathlineto{\pgfqpoint{1.542190in}{0.638889in}}%
\pgfpathlineto{\pgfqpoint{1.542190in}{0.706848in}}%
\pgfpathlineto{\pgfqpoint{1.516439in}{0.706848in}}%
\pgfpathlineto{\pgfqpoint{1.516439in}{0.638889in}}%
\pgfpathclose%
\pgfusepath{stroke,fill}%
\end{pgfscope}%
\begin{pgfscope}%
\pgfpathrectangle{\pgfqpoint{0.781250in}{0.638889in}}{\pgfqpoint{4.843750in}{2.172222in}}%
\pgfusepath{clip}%
\pgfsetbuttcap%
\pgfsetmiterjoin%
\definecolor{currentfill}{rgb}{0.227451,0.192157,0.427451}%
\pgfsetfillcolor{currentfill}%
\pgfsetlinewidth{0.100375pt}%
\definecolor{currentstroke}{rgb}{0.266667,0.266667,0.266667}%
\pgfsetstrokecolor{currentstroke}%
\pgfsetdash{}{0pt}%
\pgfpathmoveto{\pgfqpoint{1.553226in}{0.638889in}}%
\pgfpathlineto{\pgfqpoint{1.578977in}{0.638889in}}%
\pgfpathlineto{\pgfqpoint{1.578977in}{0.828493in}}%
\pgfpathlineto{\pgfqpoint{1.553226in}{0.828493in}}%
\pgfpathlineto{\pgfqpoint{1.553226in}{0.638889in}}%
\pgfpathclose%
\pgfusepath{stroke,fill}%
\end{pgfscope}%
\begin{pgfscope}%
\pgfpathrectangle{\pgfqpoint{0.781250in}{0.638889in}}{\pgfqpoint{4.843750in}{2.172222in}}%
\pgfusepath{clip}%
\pgfsetbuttcap%
\pgfsetmiterjoin%
\definecolor{currentfill}{rgb}{0.227451,0.192157,0.427451}%
\pgfsetfillcolor{currentfill}%
\pgfsetlinewidth{0.100375pt}%
\definecolor{currentstroke}{rgb}{0.266667,0.266667,0.266667}%
\pgfsetstrokecolor{currentstroke}%
\pgfsetdash{}{0pt}%
\pgfpathmoveto{\pgfqpoint{1.590013in}{0.638889in}}%
\pgfpathlineto{\pgfqpoint{1.615764in}{0.638889in}}%
\pgfpathlineto{\pgfqpoint{1.615764in}{0.757740in}}%
\pgfpathlineto{\pgfqpoint{1.590013in}{0.757740in}}%
\pgfpathlineto{\pgfqpoint{1.590013in}{0.638889in}}%
\pgfpathclose%
\pgfusepath{stroke,fill}%
\end{pgfscope}%
\begin{pgfscope}%
\pgfpathrectangle{\pgfqpoint{0.781250in}{0.638889in}}{\pgfqpoint{4.843750in}{2.172222in}}%
\pgfusepath{clip}%
\pgfsetbuttcap%
\pgfsetmiterjoin%
\definecolor{currentfill}{rgb}{0.164706,0.615686,0.560784}%
\pgfsetfillcolor{currentfill}%
\pgfsetlinewidth{0.100375pt}%
\definecolor{currentstroke}{rgb}{0.266667,0.266667,0.266667}%
\pgfsetstrokecolor{currentstroke}%
\pgfsetdash{}{0pt}%
\pgfpathmoveto{\pgfqpoint{1.626800in}{0.638889in}}%
\pgfpathlineto{\pgfqpoint{1.652551in}{0.638889in}}%
\pgfpathlineto{\pgfqpoint{1.652551in}{0.717089in}}%
\pgfpathlineto{\pgfqpoint{1.626800in}{0.717089in}}%
\pgfpathlineto{\pgfqpoint{1.626800in}{0.638889in}}%
\pgfpathclose%
\pgfusepath{stroke,fill}%
\end{pgfscope}%
\begin{pgfscope}%
\pgfpathrectangle{\pgfqpoint{0.781250in}{0.638889in}}{\pgfqpoint{4.843750in}{2.172222in}}%
\pgfusepath{clip}%
\pgfsetbuttcap%
\pgfsetmiterjoin%
\definecolor{currentfill}{rgb}{0.164706,0.615686,0.560784}%
\pgfsetfillcolor{currentfill}%
\pgfsetlinewidth{0.100375pt}%
\definecolor{currentstroke}{rgb}{0.266667,0.266667,0.266667}%
\pgfsetstrokecolor{currentstroke}%
\pgfsetdash{}{0pt}%
\pgfpathmoveto{\pgfqpoint{1.663587in}{0.638889in}}%
\pgfpathlineto{\pgfqpoint{1.689338in}{0.638889in}}%
\pgfpathlineto{\pgfqpoint{1.689338in}{1.795132in}}%
\pgfpathlineto{\pgfqpoint{1.663587in}{1.795132in}}%
\pgfpathlineto{\pgfqpoint{1.663587in}{0.638889in}}%
\pgfpathclose%
\pgfusepath{stroke,fill}%
\end{pgfscope}%
\begin{pgfscope}%
\pgfpathrectangle{\pgfqpoint{0.781250in}{0.638889in}}{\pgfqpoint{4.843750in}{2.172222in}}%
\pgfusepath{clip}%
\pgfsetbuttcap%
\pgfsetmiterjoin%
\definecolor{currentfill}{rgb}{0.164706,0.615686,0.560784}%
\pgfsetfillcolor{currentfill}%
\pgfsetlinewidth{0.100375pt}%
\definecolor{currentstroke}{rgb}{0.266667,0.266667,0.266667}%
\pgfsetstrokecolor{currentstroke}%
\pgfsetdash{}{0pt}%
\pgfpathmoveto{\pgfqpoint{1.700374in}{0.638889in}}%
\pgfpathlineto{\pgfqpoint{1.726125in}{0.638889in}}%
\pgfpathlineto{\pgfqpoint{1.726125in}{1.066506in}}%
\pgfpathlineto{\pgfqpoint{1.700374in}{1.066506in}}%
\pgfpathlineto{\pgfqpoint{1.700374in}{0.638889in}}%
\pgfpathclose%
\pgfusepath{stroke,fill}%
\end{pgfscope}%
\begin{pgfscope}%
\pgfpathrectangle{\pgfqpoint{0.781250in}{0.638889in}}{\pgfqpoint{4.843750in}{2.172222in}}%
\pgfusepath{clip}%
\pgfsetbuttcap%
\pgfsetmiterjoin%
\definecolor{currentfill}{rgb}{0.164706,0.615686,0.560784}%
\pgfsetfillcolor{currentfill}%
\pgfsetlinewidth{0.100375pt}%
\definecolor{currentstroke}{rgb}{0.266667,0.266667,0.266667}%
\pgfsetstrokecolor{currentstroke}%
\pgfsetdash{}{0pt}%
\pgfpathmoveto{\pgfqpoint{1.737161in}{0.638889in}}%
\pgfpathlineto{\pgfqpoint{1.762912in}{0.638889in}}%
\pgfpathlineto{\pgfqpoint{1.762912in}{0.964722in}}%
\pgfpathlineto{\pgfqpoint{1.737161in}{0.964722in}}%
\pgfpathlineto{\pgfqpoint{1.737161in}{0.638889in}}%
\pgfpathclose%
\pgfusepath{stroke,fill}%
\end{pgfscope}%
\begin{pgfscope}%
\pgfpathrectangle{\pgfqpoint{0.781250in}{0.638889in}}{\pgfqpoint{4.843750in}{2.172222in}}%
\pgfusepath{clip}%
\pgfsetbuttcap%
\pgfsetmiterjoin%
\definecolor{currentfill}{rgb}{0.227451,0.192157,0.427451}%
\pgfsetfillcolor{currentfill}%
\pgfsetlinewidth{0.100375pt}%
\definecolor{currentstroke}{rgb}{0.266667,0.266667,0.266667}%
\pgfsetstrokecolor{currentstroke}%
\pgfsetdash{}{0pt}%
\pgfpathmoveto{\pgfqpoint{1.773948in}{0.638889in}}%
\pgfpathlineto{\pgfqpoint{1.799699in}{0.638889in}}%
\pgfpathlineto{\pgfqpoint{1.799699in}{0.749052in}}%
\pgfpathlineto{\pgfqpoint{1.773948in}{0.749052in}}%
\pgfpathlineto{\pgfqpoint{1.773948in}{0.638889in}}%
\pgfpathclose%
\pgfusepath{stroke,fill}%
\end{pgfscope}%
\begin{pgfscope}%
\pgfpathrectangle{\pgfqpoint{0.781250in}{0.638889in}}{\pgfqpoint{4.843750in}{2.172222in}}%
\pgfusepath{clip}%
\pgfsetbuttcap%
\pgfsetmiterjoin%
\definecolor{currentfill}{rgb}{0.227451,0.192157,0.427451}%
\pgfsetfillcolor{currentfill}%
\pgfsetlinewidth{0.100375pt}%
\definecolor{currentstroke}{rgb}{0.266667,0.266667,0.266667}%
\pgfsetstrokecolor{currentstroke}%
\pgfsetdash{}{0pt}%
\pgfpathmoveto{\pgfqpoint{1.810735in}{0.638889in}}%
\pgfpathlineto{\pgfqpoint{1.836486in}{0.638889in}}%
\pgfpathlineto{\pgfqpoint{1.836486in}{0.671783in}}%
\pgfpathlineto{\pgfqpoint{1.810735in}{0.671783in}}%
\pgfpathlineto{\pgfqpoint{1.810735in}{0.638889in}}%
\pgfpathclose%
\pgfusepath{stroke,fill}%
\end{pgfscope}%
\begin{pgfscope}%
\pgfpathrectangle{\pgfqpoint{0.781250in}{0.638889in}}{\pgfqpoint{4.843750in}{2.172222in}}%
\pgfusepath{clip}%
\pgfsetbuttcap%
\pgfsetmiterjoin%
\definecolor{currentfill}{rgb}{0.227451,0.192157,0.427451}%
\pgfsetfillcolor{currentfill}%
\pgfsetlinewidth{0.100375pt}%
\definecolor{currentstroke}{rgb}{0.266667,0.266667,0.266667}%
\pgfsetstrokecolor{currentstroke}%
\pgfsetdash{}{0pt}%
\pgfpathmoveto{\pgfqpoint{1.847522in}{0.638889in}}%
\pgfpathlineto{\pgfqpoint{1.873273in}{0.638889in}}%
\pgfpathlineto{\pgfqpoint{1.873273in}{0.769222in}}%
\pgfpathlineto{\pgfqpoint{1.847522in}{0.769222in}}%
\pgfpathlineto{\pgfqpoint{1.847522in}{0.638889in}}%
\pgfpathclose%
\pgfusepath{stroke,fill}%
\end{pgfscope}%
\begin{pgfscope}%
\pgfpathrectangle{\pgfqpoint{0.781250in}{0.638889in}}{\pgfqpoint{4.843750in}{2.172222in}}%
\pgfusepath{clip}%
\pgfsetbuttcap%
\pgfsetmiterjoin%
\definecolor{currentfill}{rgb}{0.227451,0.192157,0.427451}%
\pgfsetfillcolor{currentfill}%
\pgfsetlinewidth{0.100375pt}%
\definecolor{currentstroke}{rgb}{0.266667,0.266667,0.266667}%
\pgfsetstrokecolor{currentstroke}%
\pgfsetdash{}{0pt}%
\pgfpathmoveto{\pgfqpoint{1.884309in}{0.638889in}}%
\pgfpathlineto{\pgfqpoint{1.910060in}{0.638889in}}%
\pgfpathlineto{\pgfqpoint{1.910060in}{0.746259in}}%
\pgfpathlineto{\pgfqpoint{1.884309in}{0.746259in}}%
\pgfpathlineto{\pgfqpoint{1.884309in}{0.638889in}}%
\pgfpathclose%
\pgfusepath{stroke,fill}%
\end{pgfscope}%
\begin{pgfscope}%
\pgfpathrectangle{\pgfqpoint{0.781250in}{0.638889in}}{\pgfqpoint{4.843750in}{2.172222in}}%
\pgfusepath{clip}%
\pgfsetbuttcap%
\pgfsetmiterjoin%
\definecolor{currentfill}{rgb}{0.227451,0.192157,0.427451}%
\pgfsetfillcolor{currentfill}%
\pgfsetlinewidth{0.100375pt}%
\definecolor{currentstroke}{rgb}{0.266667,0.266667,0.266667}%
\pgfsetstrokecolor{currentstroke}%
\pgfsetdash{}{0pt}%
\pgfpathmoveto{\pgfqpoint{1.921097in}{0.638889in}}%
\pgfpathlineto{\pgfqpoint{1.946847in}{0.638889in}}%
\pgfpathlineto{\pgfqpoint{1.946847in}{0.736639in}}%
\pgfpathlineto{\pgfqpoint{1.921097in}{0.736639in}}%
\pgfpathlineto{\pgfqpoint{1.921097in}{0.638889in}}%
\pgfpathclose%
\pgfusepath{stroke,fill}%
\end{pgfscope}%
\begin{pgfscope}%
\pgfpathrectangle{\pgfqpoint{0.781250in}{0.638889in}}{\pgfqpoint{4.843750in}{2.172222in}}%
\pgfusepath{clip}%
\pgfsetbuttcap%
\pgfsetmiterjoin%
\definecolor{currentfill}{rgb}{0.227451,0.192157,0.427451}%
\pgfsetfillcolor{currentfill}%
\pgfsetlinewidth{0.100375pt}%
\definecolor{currentstroke}{rgb}{0.266667,0.266667,0.266667}%
\pgfsetstrokecolor{currentstroke}%
\pgfsetdash{}{0pt}%
\pgfpathmoveto{\pgfqpoint{1.957884in}{0.638889in}}%
\pgfpathlineto{\pgfqpoint{1.983635in}{0.638889in}}%
\pgfpathlineto{\pgfqpoint{1.983635in}{0.944552in}}%
\pgfpathlineto{\pgfqpoint{1.957884in}{0.944552in}}%
\pgfpathlineto{\pgfqpoint{1.957884in}{0.638889in}}%
\pgfpathclose%
\pgfusepath{stroke,fill}%
\end{pgfscope}%
\begin{pgfscope}%
\pgfpathrectangle{\pgfqpoint{0.781250in}{0.638889in}}{\pgfqpoint{4.843750in}{2.172222in}}%
\pgfusepath{clip}%
\pgfsetbuttcap%
\pgfsetmiterjoin%
\definecolor{currentfill}{rgb}{0.227451,0.192157,0.427451}%
\pgfsetfillcolor{currentfill}%
\pgfsetlinewidth{0.100375pt}%
\definecolor{currentstroke}{rgb}{0.266667,0.266667,0.266667}%
\pgfsetstrokecolor{currentstroke}%
\pgfsetdash{}{0pt}%
\pgfpathmoveto{\pgfqpoint{1.994671in}{0.638889in}}%
\pgfpathlineto{\pgfqpoint{2.020422in}{0.638889in}}%
\pgfpathlineto{\pgfqpoint{2.020422in}{1.025544in}}%
\pgfpathlineto{\pgfqpoint{1.994671in}{1.025544in}}%
\pgfpathlineto{\pgfqpoint{1.994671in}{0.638889in}}%
\pgfpathclose%
\pgfusepath{stroke,fill}%
\end{pgfscope}%
\begin{pgfscope}%
\pgfpathrectangle{\pgfqpoint{0.781250in}{0.638889in}}{\pgfqpoint{4.843750in}{2.172222in}}%
\pgfusepath{clip}%
\pgfsetbuttcap%
\pgfsetmiterjoin%
\definecolor{currentfill}{rgb}{0.227451,0.192157,0.427451}%
\pgfsetfillcolor{currentfill}%
\pgfsetlinewidth{0.100375pt}%
\definecolor{currentstroke}{rgb}{0.266667,0.266667,0.266667}%
\pgfsetstrokecolor{currentstroke}%
\pgfsetdash{}{0pt}%
\pgfpathmoveto{\pgfqpoint{2.031458in}{0.638889in}}%
\pgfpathlineto{\pgfqpoint{2.057209in}{0.638889in}}%
\pgfpathlineto{\pgfqpoint{2.057209in}{0.987996in}}%
\pgfpathlineto{\pgfqpoint{2.031458in}{0.987996in}}%
\pgfpathlineto{\pgfqpoint{2.031458in}{0.638889in}}%
\pgfpathclose%
\pgfusepath{stroke,fill}%
\end{pgfscope}%
\begin{pgfscope}%
\pgfpathrectangle{\pgfqpoint{0.781250in}{0.638889in}}{\pgfqpoint{4.843750in}{2.172222in}}%
\pgfusepath{clip}%
\pgfsetbuttcap%
\pgfsetmiterjoin%
\definecolor{currentfill}{rgb}{0.164706,0.615686,0.560784}%
\pgfsetfillcolor{currentfill}%
\pgfsetlinewidth{0.100375pt}%
\definecolor{currentstroke}{rgb}{0.266667,0.266667,0.266667}%
\pgfsetstrokecolor{currentstroke}%
\pgfsetdash{}{0pt}%
\pgfpathmoveto{\pgfqpoint{2.068245in}{0.638889in}}%
\pgfpathlineto{\pgfqpoint{2.093996in}{0.638889in}}%
\pgfpathlineto{\pgfqpoint{2.093996in}{0.964722in}}%
\pgfpathlineto{\pgfqpoint{2.068245in}{0.964722in}}%
\pgfpathlineto{\pgfqpoint{2.068245in}{0.638889in}}%
\pgfpathclose%
\pgfusepath{stroke,fill}%
\end{pgfscope}%
\begin{pgfscope}%
\pgfpathrectangle{\pgfqpoint{0.781250in}{0.638889in}}{\pgfqpoint{4.843750in}{2.172222in}}%
\pgfusepath{clip}%
\pgfsetbuttcap%
\pgfsetmiterjoin%
\definecolor{currentfill}{rgb}{0.164706,0.615686,0.560784}%
\pgfsetfillcolor{currentfill}%
\pgfsetlinewidth{0.100375pt}%
\definecolor{currentstroke}{rgb}{0.266667,0.266667,0.266667}%
\pgfsetstrokecolor{currentstroke}%
\pgfsetdash{}{0pt}%
\pgfpathmoveto{\pgfqpoint{2.105032in}{0.638889in}}%
\pgfpathlineto{\pgfqpoint{2.130783in}{0.638889in}}%
\pgfpathlineto{\pgfqpoint{2.130783in}{1.440129in}}%
\pgfpathlineto{\pgfqpoint{2.105032in}{1.440129in}}%
\pgfpathlineto{\pgfqpoint{2.105032in}{0.638889in}}%
\pgfpathclose%
\pgfusepath{stroke,fill}%
\end{pgfscope}%
\begin{pgfscope}%
\pgfpathrectangle{\pgfqpoint{0.781250in}{0.638889in}}{\pgfqpoint{4.843750in}{2.172222in}}%
\pgfusepath{clip}%
\pgfsetbuttcap%
\pgfsetmiterjoin%
\definecolor{currentfill}{rgb}{0.164706,0.615686,0.560784}%
\pgfsetfillcolor{currentfill}%
\pgfsetlinewidth{0.100375pt}%
\definecolor{currentstroke}{rgb}{0.266667,0.266667,0.266667}%
\pgfsetstrokecolor{currentstroke}%
\pgfsetdash{}{0pt}%
\pgfpathmoveto{\pgfqpoint{2.141819in}{0.638889in}}%
\pgfpathlineto{\pgfqpoint{2.167570in}{0.638889in}}%
\pgfpathlineto{\pgfqpoint{2.167570in}{1.568290in}}%
\pgfpathlineto{\pgfqpoint{2.141819in}{1.568290in}}%
\pgfpathlineto{\pgfqpoint{2.141819in}{0.638889in}}%
\pgfpathclose%
\pgfusepath{stroke,fill}%
\end{pgfscope}%
\begin{pgfscope}%
\pgfpathrectangle{\pgfqpoint{0.781250in}{0.638889in}}{\pgfqpoint{4.843750in}{2.172222in}}%
\pgfusepath{clip}%
\pgfsetbuttcap%
\pgfsetmiterjoin%
\definecolor{currentfill}{rgb}{0.164706,0.615686,0.560784}%
\pgfsetfillcolor{currentfill}%
\pgfsetlinewidth{0.100375pt}%
\definecolor{currentstroke}{rgb}{0.266667,0.266667,0.266667}%
\pgfsetstrokecolor{currentstroke}%
\pgfsetdash{}{0pt}%
\pgfpathmoveto{\pgfqpoint{2.178606in}{0.638889in}}%
\pgfpathlineto{\pgfqpoint{2.204357in}{0.638889in}}%
\pgfpathlineto{\pgfqpoint{2.204357in}{0.846802in}}%
\pgfpathlineto{\pgfqpoint{2.178606in}{0.846802in}}%
\pgfpathlineto{\pgfqpoint{2.178606in}{0.638889in}}%
\pgfpathclose%
\pgfusepath{stroke,fill}%
\end{pgfscope}%
\begin{pgfscope}%
\pgfpathrectangle{\pgfqpoint{0.781250in}{0.638889in}}{\pgfqpoint{4.843750in}{2.172222in}}%
\pgfusepath{clip}%
\pgfsetbuttcap%
\pgfsetmiterjoin%
\definecolor{currentfill}{rgb}{0.227451,0.192157,0.427451}%
\pgfsetfillcolor{currentfill}%
\pgfsetlinewidth{0.100375pt}%
\definecolor{currentstroke}{rgb}{0.266667,0.266667,0.266667}%
\pgfsetstrokecolor{currentstroke}%
\pgfsetdash{}{0pt}%
\pgfpathmoveto{\pgfqpoint{2.215393in}{0.638889in}}%
\pgfpathlineto{\pgfqpoint{2.241144in}{0.638889in}}%
\pgfpathlineto{\pgfqpoint{2.241144in}{1.140052in}}%
\pgfpathlineto{\pgfqpoint{2.215393in}{1.140052in}}%
\pgfpathlineto{\pgfqpoint{2.215393in}{0.638889in}}%
\pgfpathclose%
\pgfusepath{stroke,fill}%
\end{pgfscope}%
\begin{pgfscope}%
\pgfpathrectangle{\pgfqpoint{0.781250in}{0.638889in}}{\pgfqpoint{4.843750in}{2.172222in}}%
\pgfusepath{clip}%
\pgfsetbuttcap%
\pgfsetmiterjoin%
\definecolor{currentfill}{rgb}{0.227451,0.192157,0.427451}%
\pgfsetfillcolor{currentfill}%
\pgfsetlinewidth{0.100375pt}%
\definecolor{currentstroke}{rgb}{0.266667,0.266667,0.266667}%
\pgfsetstrokecolor{currentstroke}%
\pgfsetdash{}{0pt}%
\pgfpathmoveto{\pgfqpoint{2.252180in}{0.638889in}}%
\pgfpathlineto{\pgfqpoint{2.277931in}{0.638889in}}%
\pgfpathlineto{\pgfqpoint{2.277931in}{0.762085in}}%
\pgfpathlineto{\pgfqpoint{2.252180in}{0.762085in}}%
\pgfpathlineto{\pgfqpoint{2.252180in}{0.638889in}}%
\pgfpathclose%
\pgfusepath{stroke,fill}%
\end{pgfscope}%
\begin{pgfscope}%
\pgfpathrectangle{\pgfqpoint{0.781250in}{0.638889in}}{\pgfqpoint{4.843750in}{2.172222in}}%
\pgfusepath{clip}%
\pgfsetbuttcap%
\pgfsetmiterjoin%
\definecolor{currentfill}{rgb}{0.227451,0.192157,0.427451}%
\pgfsetfillcolor{currentfill}%
\pgfsetlinewidth{0.100375pt}%
\definecolor{currentstroke}{rgb}{0.266667,0.266667,0.266667}%
\pgfsetstrokecolor{currentstroke}%
\pgfsetdash{}{0pt}%
\pgfpathmoveto{\pgfqpoint{2.288967in}{0.638889in}}%
\pgfpathlineto{\pgfqpoint{2.314718in}{0.638889in}}%
\pgfpathlineto{\pgfqpoint{2.314718in}{0.720813in}}%
\pgfpathlineto{\pgfqpoint{2.288967in}{0.720813in}}%
\pgfpathlineto{\pgfqpoint{2.288967in}{0.638889in}}%
\pgfpathclose%
\pgfusepath{stroke,fill}%
\end{pgfscope}%
\begin{pgfscope}%
\pgfpathrectangle{\pgfqpoint{0.781250in}{0.638889in}}{\pgfqpoint{4.843750in}{2.172222in}}%
\pgfusepath{clip}%
\pgfsetbuttcap%
\pgfsetmiterjoin%
\definecolor{currentfill}{rgb}{0.227451,0.192157,0.427451}%
\pgfsetfillcolor{currentfill}%
\pgfsetlinewidth{0.100375pt}%
\definecolor{currentstroke}{rgb}{0.266667,0.266667,0.266667}%
\pgfsetstrokecolor{currentstroke}%
\pgfsetdash{}{0pt}%
\pgfpathmoveto{\pgfqpoint{2.325754in}{0.638889in}}%
\pgfpathlineto{\pgfqpoint{2.351505in}{0.638889in}}%
\pgfpathlineto{\pgfqpoint{2.351505in}{0.666197in}}%
\pgfpathlineto{\pgfqpoint{2.325754in}{0.666197in}}%
\pgfpathlineto{\pgfqpoint{2.325754in}{0.638889in}}%
\pgfpathclose%
\pgfusepath{stroke,fill}%
\end{pgfscope}%
\begin{pgfscope}%
\pgfpathrectangle{\pgfqpoint{0.781250in}{0.638889in}}{\pgfqpoint{4.843750in}{2.172222in}}%
\pgfusepath{clip}%
\pgfsetbuttcap%
\pgfsetmiterjoin%
\definecolor{currentfill}{rgb}{0.227451,0.192157,0.427451}%
\pgfsetfillcolor{currentfill}%
\pgfsetlinewidth{0.100375pt}%
\definecolor{currentstroke}{rgb}{0.266667,0.266667,0.266667}%
\pgfsetstrokecolor{currentstroke}%
\pgfsetdash{}{0pt}%
\pgfpathmoveto{\pgfqpoint{2.362541in}{0.638889in}}%
\pgfpathlineto{\pgfqpoint{2.388292in}{0.638889in}}%
\pgfpathlineto{\pgfqpoint{2.388292in}{0.716779in}}%
\pgfpathlineto{\pgfqpoint{2.362541in}{0.716779in}}%
\pgfpathlineto{\pgfqpoint{2.362541in}{0.638889in}}%
\pgfpathclose%
\pgfusepath{stroke,fill}%
\end{pgfscope}%
\begin{pgfscope}%
\pgfpathrectangle{\pgfqpoint{0.781250in}{0.638889in}}{\pgfqpoint{4.843750in}{2.172222in}}%
\pgfusepath{clip}%
\pgfsetbuttcap%
\pgfsetmiterjoin%
\definecolor{currentfill}{rgb}{0.227451,0.192157,0.427451}%
\pgfsetfillcolor{currentfill}%
\pgfsetlinewidth{0.100375pt}%
\definecolor{currentstroke}{rgb}{0.266667,0.266667,0.266667}%
\pgfsetstrokecolor{currentstroke}%
\pgfsetdash{}{0pt}%
\pgfpathmoveto{\pgfqpoint{2.399328in}{0.638889in}}%
\pgfpathlineto{\pgfqpoint{2.425079in}{0.638889in}}%
\pgfpathlineto{\pgfqpoint{2.425079in}{0.733225in}}%
\pgfpathlineto{\pgfqpoint{2.399328in}{0.733225in}}%
\pgfpathlineto{\pgfqpoint{2.399328in}{0.638889in}}%
\pgfpathclose%
\pgfusepath{stroke,fill}%
\end{pgfscope}%
\begin{pgfscope}%
\pgfpathrectangle{\pgfqpoint{0.781250in}{0.638889in}}{\pgfqpoint{4.843750in}{2.172222in}}%
\pgfusepath{clip}%
\pgfsetbuttcap%
\pgfsetmiterjoin%
\definecolor{currentfill}{rgb}{0.227451,0.192157,0.427451}%
\pgfsetfillcolor{currentfill}%
\pgfsetlinewidth{0.100375pt}%
\definecolor{currentstroke}{rgb}{0.266667,0.266667,0.266667}%
\pgfsetstrokecolor{currentstroke}%
\pgfsetdash{}{0pt}%
\pgfpathmoveto{\pgfqpoint{2.436115in}{0.638889in}}%
\pgfpathlineto{\pgfqpoint{2.461866in}{0.638889in}}%
\pgfpathlineto{\pgfqpoint{2.461866in}{0.963171in}}%
\pgfpathlineto{\pgfqpoint{2.436115in}{0.963171in}}%
\pgfpathlineto{\pgfqpoint{2.436115in}{0.638889in}}%
\pgfpathclose%
\pgfusepath{stroke,fill}%
\end{pgfscope}%
\begin{pgfscope}%
\pgfpathrectangle{\pgfqpoint{0.781250in}{0.638889in}}{\pgfqpoint{4.843750in}{2.172222in}}%
\pgfusepath{clip}%
\pgfsetbuttcap%
\pgfsetmiterjoin%
\definecolor{currentfill}{rgb}{0.227451,0.192157,0.427451}%
\pgfsetfillcolor{currentfill}%
\pgfsetlinewidth{0.100375pt}%
\definecolor{currentstroke}{rgb}{0.266667,0.266667,0.266667}%
\pgfsetstrokecolor{currentstroke}%
\pgfsetdash{}{0pt}%
\pgfpathmoveto{\pgfqpoint{2.472902in}{0.638889in}}%
\pgfpathlineto{\pgfqpoint{2.498653in}{0.638889in}}%
\pgfpathlineto{\pgfqpoint{2.498653in}{0.827252in}}%
\pgfpathlineto{\pgfqpoint{2.472902in}{0.827252in}}%
\pgfpathlineto{\pgfqpoint{2.472902in}{0.638889in}}%
\pgfpathclose%
\pgfusepath{stroke,fill}%
\end{pgfscope}%
\begin{pgfscope}%
\pgfpathrectangle{\pgfqpoint{0.781250in}{0.638889in}}{\pgfqpoint{4.843750in}{2.172222in}}%
\pgfusepath{clip}%
\pgfsetbuttcap%
\pgfsetmiterjoin%
\definecolor{currentfill}{rgb}{0.164706,0.615686,0.560784}%
\pgfsetfillcolor{currentfill}%
\pgfsetlinewidth{0.100375pt}%
\definecolor{currentstroke}{rgb}{0.266667,0.266667,0.266667}%
\pgfsetstrokecolor{currentstroke}%
\pgfsetdash{}{0pt}%
\pgfpathmoveto{\pgfqpoint{2.509689in}{0.638889in}}%
\pgfpathlineto{\pgfqpoint{2.535440in}{0.638889in}}%
\pgfpathlineto{\pgfqpoint{2.535440in}{0.915071in}}%
\pgfpathlineto{\pgfqpoint{2.509689in}{0.915071in}}%
\pgfpathlineto{\pgfqpoint{2.509689in}{0.638889in}}%
\pgfpathclose%
\pgfusepath{stroke,fill}%
\end{pgfscope}%
\begin{pgfscope}%
\pgfpathrectangle{\pgfqpoint{0.781250in}{0.638889in}}{\pgfqpoint{4.843750in}{2.172222in}}%
\pgfusepath{clip}%
\pgfsetbuttcap%
\pgfsetmiterjoin%
\definecolor{currentfill}{rgb}{0.164706,0.615686,0.560784}%
\pgfsetfillcolor{currentfill}%
\pgfsetlinewidth{0.100375pt}%
\definecolor{currentstroke}{rgb}{0.266667,0.266667,0.266667}%
\pgfsetstrokecolor{currentstroke}%
\pgfsetdash{}{0pt}%
\pgfpathmoveto{\pgfqpoint{2.546476in}{0.638889in}}%
\pgfpathlineto{\pgfqpoint{2.572227in}{0.638889in}}%
\pgfpathlineto{\pgfqpoint{2.572227in}{1.338655in}}%
\pgfpathlineto{\pgfqpoint{2.546476in}{1.338655in}}%
\pgfpathlineto{\pgfqpoint{2.546476in}{0.638889in}}%
\pgfpathclose%
\pgfusepath{stroke,fill}%
\end{pgfscope}%
\begin{pgfscope}%
\pgfpathrectangle{\pgfqpoint{0.781250in}{0.638889in}}{\pgfqpoint{4.843750in}{2.172222in}}%
\pgfusepath{clip}%
\pgfsetbuttcap%
\pgfsetmiterjoin%
\definecolor{currentfill}{rgb}{0.164706,0.615686,0.560784}%
\pgfsetfillcolor{currentfill}%
\pgfsetlinewidth{0.100375pt}%
\definecolor{currentstroke}{rgb}{0.266667,0.266667,0.266667}%
\pgfsetstrokecolor{currentstroke}%
\pgfsetdash{}{0pt}%
\pgfpathmoveto{\pgfqpoint{2.583263in}{0.638889in}}%
\pgfpathlineto{\pgfqpoint{2.609014in}{0.638889in}}%
\pgfpathlineto{\pgfqpoint{2.609014in}{1.034233in}}%
\pgfpathlineto{\pgfqpoint{2.583263in}{1.034233in}}%
\pgfpathlineto{\pgfqpoint{2.583263in}{0.638889in}}%
\pgfpathclose%
\pgfusepath{stroke,fill}%
\end{pgfscope}%
\begin{pgfscope}%
\pgfpathrectangle{\pgfqpoint{0.781250in}{0.638889in}}{\pgfqpoint{4.843750in}{2.172222in}}%
\pgfusepath{clip}%
\pgfsetbuttcap%
\pgfsetmiterjoin%
\definecolor{currentfill}{rgb}{0.164706,0.615686,0.560784}%
\pgfsetfillcolor{currentfill}%
\pgfsetlinewidth{0.100375pt}%
\definecolor{currentstroke}{rgb}{0.266667,0.266667,0.266667}%
\pgfsetstrokecolor{currentstroke}%
\pgfsetdash{}{0pt}%
\pgfpathmoveto{\pgfqpoint{2.620050in}{0.638889in}}%
\pgfpathlineto{\pgfqpoint{2.645801in}{0.638889in}}%
\pgfpathlineto{\pgfqpoint{2.645801in}{1.392960in}}%
\pgfpathlineto{\pgfqpoint{2.620050in}{1.392960in}}%
\pgfpathlineto{\pgfqpoint{2.620050in}{0.638889in}}%
\pgfpathclose%
\pgfusepath{stroke,fill}%
\end{pgfscope}%
\begin{pgfscope}%
\pgfpathrectangle{\pgfqpoint{0.781250in}{0.638889in}}{\pgfqpoint{4.843750in}{2.172222in}}%
\pgfusepath{clip}%
\pgfsetbuttcap%
\pgfsetmiterjoin%
\definecolor{currentfill}{rgb}{0.227451,0.192157,0.427451}%
\pgfsetfillcolor{currentfill}%
\pgfsetlinewidth{0.100375pt}%
\definecolor{currentstroke}{rgb}{0.266667,0.266667,0.266667}%
\pgfsetstrokecolor{currentstroke}%
\pgfsetdash{}{0pt}%
\pgfpathmoveto{\pgfqpoint{2.656837in}{0.638889in}}%
\pgfpathlineto{\pgfqpoint{2.682588in}{0.638889in}}%
\pgfpathlineto{\pgfqpoint{2.682588in}{0.907624in}}%
\pgfpathlineto{\pgfqpoint{2.656837in}{0.907624in}}%
\pgfpathlineto{\pgfqpoint{2.656837in}{0.638889in}}%
\pgfpathclose%
\pgfusepath{stroke,fill}%
\end{pgfscope}%
\begin{pgfscope}%
\pgfpathrectangle{\pgfqpoint{0.781250in}{0.638889in}}{\pgfqpoint{4.843750in}{2.172222in}}%
\pgfusepath{clip}%
\pgfsetbuttcap%
\pgfsetmiterjoin%
\definecolor{currentfill}{rgb}{0.227451,0.192157,0.427451}%
\pgfsetfillcolor{currentfill}%
\pgfsetlinewidth{0.100375pt}%
\definecolor{currentstroke}{rgb}{0.266667,0.266667,0.266667}%
\pgfsetstrokecolor{currentstroke}%
\pgfsetdash{}{0pt}%
\pgfpathmoveto{\pgfqpoint{2.693624in}{0.638889in}}%
\pgfpathlineto{\pgfqpoint{2.719375in}{0.638889in}}%
\pgfpathlineto{\pgfqpoint{2.719375in}{0.811425in}}%
\pgfpathlineto{\pgfqpoint{2.693624in}{0.811425in}}%
\pgfpathlineto{\pgfqpoint{2.693624in}{0.638889in}}%
\pgfpathclose%
\pgfusepath{stroke,fill}%
\end{pgfscope}%
\begin{pgfscope}%
\pgfpathrectangle{\pgfqpoint{0.781250in}{0.638889in}}{\pgfqpoint{4.843750in}{2.172222in}}%
\pgfusepath{clip}%
\pgfsetbuttcap%
\pgfsetmiterjoin%
\definecolor{currentfill}{rgb}{0.227451,0.192157,0.427451}%
\pgfsetfillcolor{currentfill}%
\pgfsetlinewidth{0.100375pt}%
\definecolor{currentstroke}{rgb}{0.266667,0.266667,0.266667}%
\pgfsetstrokecolor{currentstroke}%
\pgfsetdash{}{0pt}%
\pgfpathmoveto{\pgfqpoint{2.730411in}{0.638889in}}%
\pgfpathlineto{\pgfqpoint{2.756162in}{0.638889in}}%
\pgfpathlineto{\pgfqpoint{2.756162in}{0.734777in}}%
\pgfpathlineto{\pgfqpoint{2.730411in}{0.734777in}}%
\pgfpathlineto{\pgfqpoint{2.730411in}{0.638889in}}%
\pgfpathclose%
\pgfusepath{stroke,fill}%
\end{pgfscope}%
\begin{pgfscope}%
\pgfpathrectangle{\pgfqpoint{0.781250in}{0.638889in}}{\pgfqpoint{4.843750in}{2.172222in}}%
\pgfusepath{clip}%
\pgfsetbuttcap%
\pgfsetmiterjoin%
\definecolor{currentfill}{rgb}{0.227451,0.192157,0.427451}%
\pgfsetfillcolor{currentfill}%
\pgfsetlinewidth{0.100375pt}%
\definecolor{currentstroke}{rgb}{0.266667,0.266667,0.266667}%
\pgfsetstrokecolor{currentstroke}%
\pgfsetdash{}{0pt}%
\pgfpathmoveto{\pgfqpoint{2.767199in}{0.638889in}}%
\pgfpathlineto{\pgfqpoint{2.792949in}{0.638889in}}%
\pgfpathlineto{\pgfqpoint{2.792949in}{0.884350in}}%
\pgfpathlineto{\pgfqpoint{2.767199in}{0.884350in}}%
\pgfpathlineto{\pgfqpoint{2.767199in}{0.638889in}}%
\pgfpathclose%
\pgfusepath{stroke,fill}%
\end{pgfscope}%
\begin{pgfscope}%
\pgfpathrectangle{\pgfqpoint{0.781250in}{0.638889in}}{\pgfqpoint{4.843750in}{2.172222in}}%
\pgfusepath{clip}%
\pgfsetbuttcap%
\pgfsetmiterjoin%
\definecolor{currentfill}{rgb}{0.227451,0.192157,0.427451}%
\pgfsetfillcolor{currentfill}%
\pgfsetlinewidth{0.100375pt}%
\definecolor{currentstroke}{rgb}{0.266667,0.266667,0.266667}%
\pgfsetstrokecolor{currentstroke}%
\pgfsetdash{}{0pt}%
\pgfpathmoveto{\pgfqpoint{2.803986in}{0.638889in}}%
\pgfpathlineto{\pgfqpoint{2.829737in}{0.638889in}}%
\pgfpathlineto{\pgfqpoint{2.829737in}{0.821976in}}%
\pgfpathlineto{\pgfqpoint{2.803986in}{0.821976in}}%
\pgfpathlineto{\pgfqpoint{2.803986in}{0.638889in}}%
\pgfpathclose%
\pgfusepath{stroke,fill}%
\end{pgfscope}%
\begin{pgfscope}%
\pgfpathrectangle{\pgfqpoint{0.781250in}{0.638889in}}{\pgfqpoint{4.843750in}{2.172222in}}%
\pgfusepath{clip}%
\pgfsetbuttcap%
\pgfsetmiterjoin%
\definecolor{currentfill}{rgb}{0.227451,0.192157,0.427451}%
\pgfsetfillcolor{currentfill}%
\pgfsetlinewidth{0.100375pt}%
\definecolor{currentstroke}{rgb}{0.266667,0.266667,0.266667}%
\pgfsetstrokecolor{currentstroke}%
\pgfsetdash{}{0pt}%
\pgfpathmoveto{\pgfqpoint{2.840773in}{0.638889in}}%
\pgfpathlineto{\pgfqpoint{2.866524in}{0.638889in}}%
\pgfpathlineto{\pgfqpoint{2.866524in}{0.707469in}}%
\pgfpathlineto{\pgfqpoint{2.840773in}{0.707469in}}%
\pgfpathlineto{\pgfqpoint{2.840773in}{0.638889in}}%
\pgfpathclose%
\pgfusepath{stroke,fill}%
\end{pgfscope}%
\begin{pgfscope}%
\pgfpathrectangle{\pgfqpoint{0.781250in}{0.638889in}}{\pgfqpoint{4.843750in}{2.172222in}}%
\pgfusepath{clip}%
\pgfsetbuttcap%
\pgfsetmiterjoin%
\definecolor{currentfill}{rgb}{0.227451,0.192157,0.427451}%
\pgfsetfillcolor{currentfill}%
\pgfsetlinewidth{0.100375pt}%
\definecolor{currentstroke}{rgb}{0.266667,0.266667,0.266667}%
\pgfsetstrokecolor{currentstroke}%
\pgfsetdash{}{0pt}%
\pgfpathmoveto{\pgfqpoint{2.877560in}{0.638889in}}%
\pgfpathlineto{\pgfqpoint{2.903311in}{0.638889in}}%
\pgfpathlineto{\pgfqpoint{2.903311in}{0.707779in}}%
\pgfpathlineto{\pgfqpoint{2.877560in}{0.707779in}}%
\pgfpathlineto{\pgfqpoint{2.877560in}{0.638889in}}%
\pgfpathclose%
\pgfusepath{stroke,fill}%
\end{pgfscope}%
\begin{pgfscope}%
\pgfpathrectangle{\pgfqpoint{0.781250in}{0.638889in}}{\pgfqpoint{4.843750in}{2.172222in}}%
\pgfusepath{clip}%
\pgfsetbuttcap%
\pgfsetmiterjoin%
\definecolor{currentfill}{rgb}{0.227451,0.192157,0.427451}%
\pgfsetfillcolor{currentfill}%
\pgfsetlinewidth{0.100375pt}%
\definecolor{currentstroke}{rgb}{0.266667,0.266667,0.266667}%
\pgfsetstrokecolor{currentstroke}%
\pgfsetdash{}{0pt}%
\pgfpathmoveto{\pgfqpoint{2.914347in}{0.638889in}}%
\pgfpathlineto{\pgfqpoint{2.940098in}{0.638889in}}%
\pgfpathlineto{\pgfqpoint{2.940098in}{0.986134in}}%
\pgfpathlineto{\pgfqpoint{2.914347in}{0.986134in}}%
\pgfpathlineto{\pgfqpoint{2.914347in}{0.638889in}}%
\pgfpathclose%
\pgfusepath{stroke,fill}%
\end{pgfscope}%
\begin{pgfscope}%
\pgfpathrectangle{\pgfqpoint{0.781250in}{0.638889in}}{\pgfqpoint{4.843750in}{2.172222in}}%
\pgfusepath{clip}%
\pgfsetbuttcap%
\pgfsetmiterjoin%
\definecolor{currentfill}{rgb}{0.164706,0.615686,0.560784}%
\pgfsetfillcolor{currentfill}%
\pgfsetlinewidth{0.100375pt}%
\definecolor{currentstroke}{rgb}{0.266667,0.266667,0.266667}%
\pgfsetstrokecolor{currentstroke}%
\pgfsetdash{}{0pt}%
\pgfpathmoveto{\pgfqpoint{2.951134in}{0.638889in}}%
\pgfpathlineto{\pgfqpoint{2.976885in}{0.638889in}}%
\pgfpathlineto{\pgfqpoint{2.976885in}{1.127949in}}%
\pgfpathlineto{\pgfqpoint{2.951134in}{1.127949in}}%
\pgfpathlineto{\pgfqpoint{2.951134in}{0.638889in}}%
\pgfpathclose%
\pgfusepath{stroke,fill}%
\end{pgfscope}%
\begin{pgfscope}%
\pgfpathrectangle{\pgfqpoint{0.781250in}{0.638889in}}{\pgfqpoint{4.843750in}{2.172222in}}%
\pgfusepath{clip}%
\pgfsetbuttcap%
\pgfsetmiterjoin%
\definecolor{currentfill}{rgb}{0.164706,0.615686,0.560784}%
\pgfsetfillcolor{currentfill}%
\pgfsetlinewidth{0.100375pt}%
\definecolor{currentstroke}{rgb}{0.266667,0.266667,0.266667}%
\pgfsetstrokecolor{currentstroke}%
\pgfsetdash{}{0pt}%
\pgfpathmoveto{\pgfqpoint{2.987921in}{0.638889in}}%
\pgfpathlineto{\pgfqpoint{3.013672in}{0.638889in}}%
\pgfpathlineto{\pgfqpoint{3.013672in}{2.220887in}}%
\pgfpathlineto{\pgfqpoint{2.987921in}{2.220887in}}%
\pgfpathlineto{\pgfqpoint{2.987921in}{0.638889in}}%
\pgfpathclose%
\pgfusepath{stroke,fill}%
\end{pgfscope}%
\begin{pgfscope}%
\pgfpathrectangle{\pgfqpoint{0.781250in}{0.638889in}}{\pgfqpoint{4.843750in}{2.172222in}}%
\pgfusepath{clip}%
\pgfsetbuttcap%
\pgfsetmiterjoin%
\definecolor{currentfill}{rgb}{0.164706,0.615686,0.560784}%
\pgfsetfillcolor{currentfill}%
\pgfsetlinewidth{0.100375pt}%
\definecolor{currentstroke}{rgb}{0.266667,0.266667,0.266667}%
\pgfsetstrokecolor{currentstroke}%
\pgfsetdash{}{0pt}%
\pgfpathmoveto{\pgfqpoint{3.024708in}{0.638889in}}%
\pgfpathlineto{\pgfqpoint{3.050459in}{0.638889in}}%
\pgfpathlineto{\pgfqpoint{3.050459in}{1.947187in}}%
\pgfpathlineto{\pgfqpoint{3.024708in}{1.947187in}}%
\pgfpathlineto{\pgfqpoint{3.024708in}{0.638889in}}%
\pgfpathclose%
\pgfusepath{stroke,fill}%
\end{pgfscope}%
\begin{pgfscope}%
\pgfpathrectangle{\pgfqpoint{0.781250in}{0.638889in}}{\pgfqpoint{4.843750in}{2.172222in}}%
\pgfusepath{clip}%
\pgfsetbuttcap%
\pgfsetmiterjoin%
\definecolor{currentfill}{rgb}{0.164706,0.615686,0.560784}%
\pgfsetfillcolor{currentfill}%
\pgfsetlinewidth{0.100375pt}%
\definecolor{currentstroke}{rgb}{0.266667,0.266667,0.266667}%
\pgfsetstrokecolor{currentstroke}%
\pgfsetdash{}{0pt}%
\pgfpathmoveto{\pgfqpoint{3.061495in}{0.638889in}}%
\pgfpathlineto{\pgfqpoint{3.087246in}{0.638889in}}%
\pgfpathlineto{\pgfqpoint{3.087246in}{1.334931in}}%
\pgfpathlineto{\pgfqpoint{3.061495in}{1.334931in}}%
\pgfpathlineto{\pgfqpoint{3.061495in}{0.638889in}}%
\pgfpathclose%
\pgfusepath{stroke,fill}%
\end{pgfscope}%
\begin{pgfscope}%
\pgfpathrectangle{\pgfqpoint{0.781250in}{0.638889in}}{\pgfqpoint{4.843750in}{2.172222in}}%
\pgfusepath{clip}%
\pgfsetbuttcap%
\pgfsetmiterjoin%
\definecolor{currentfill}{rgb}{0.227451,0.192157,0.427451}%
\pgfsetfillcolor{currentfill}%
\pgfsetlinewidth{0.100375pt}%
\definecolor{currentstroke}{rgb}{0.266667,0.266667,0.266667}%
\pgfsetstrokecolor{currentstroke}%
\pgfsetdash{}{0pt}%
\pgfpathmoveto{\pgfqpoint{3.098282in}{0.638889in}}%
\pgfpathlineto{\pgfqpoint{3.124033in}{0.638889in}}%
\pgfpathlineto{\pgfqpoint{3.124033in}{0.659990in}}%
\pgfpathlineto{\pgfqpoint{3.098282in}{0.659990in}}%
\pgfpathlineto{\pgfqpoint{3.098282in}{0.638889in}}%
\pgfpathclose%
\pgfusepath{stroke,fill}%
\end{pgfscope}%
\begin{pgfscope}%
\pgfpathrectangle{\pgfqpoint{0.781250in}{0.638889in}}{\pgfqpoint{4.843750in}{2.172222in}}%
\pgfusepath{clip}%
\pgfsetbuttcap%
\pgfsetmiterjoin%
\definecolor{currentfill}{rgb}{0.227451,0.192157,0.427451}%
\pgfsetfillcolor{currentfill}%
\pgfsetlinewidth{0.100375pt}%
\definecolor{currentstroke}{rgb}{0.266667,0.266667,0.266667}%
\pgfsetstrokecolor{currentstroke}%
\pgfsetdash{}{0pt}%
\pgfpathmoveto{\pgfqpoint{3.135069in}{0.638889in}}%
\pgfpathlineto{\pgfqpoint{3.160820in}{0.638889in}}%
\pgfpathlineto{\pgfqpoint{3.160820in}{0.792186in}}%
\pgfpathlineto{\pgfqpoint{3.135069in}{0.792186in}}%
\pgfpathlineto{\pgfqpoint{3.135069in}{0.638889in}}%
\pgfpathclose%
\pgfusepath{stroke,fill}%
\end{pgfscope}%
\begin{pgfscope}%
\pgfpathrectangle{\pgfqpoint{0.781250in}{0.638889in}}{\pgfqpoint{4.843750in}{2.172222in}}%
\pgfusepath{clip}%
\pgfsetbuttcap%
\pgfsetmiterjoin%
\definecolor{currentfill}{rgb}{0.227451,0.192157,0.427451}%
\pgfsetfillcolor{currentfill}%
\pgfsetlinewidth{0.100375pt}%
\definecolor{currentstroke}{rgb}{0.266667,0.266667,0.266667}%
\pgfsetstrokecolor{currentstroke}%
\pgfsetdash{}{0pt}%
\pgfpathmoveto{\pgfqpoint{3.171856in}{0.638889in}}%
\pgfpathlineto{\pgfqpoint{3.197607in}{0.638889in}}%
\pgfpathlineto{\pgfqpoint{3.197607in}{0.735087in}}%
\pgfpathlineto{\pgfqpoint{3.171856in}{0.735087in}}%
\pgfpathlineto{\pgfqpoint{3.171856in}{0.638889in}}%
\pgfpathclose%
\pgfusepath{stroke,fill}%
\end{pgfscope}%
\begin{pgfscope}%
\pgfpathrectangle{\pgfqpoint{0.781250in}{0.638889in}}{\pgfqpoint{4.843750in}{2.172222in}}%
\pgfusepath{clip}%
\pgfsetbuttcap%
\pgfsetmiterjoin%
\definecolor{currentfill}{rgb}{0.227451,0.192157,0.427451}%
\pgfsetfillcolor{currentfill}%
\pgfsetlinewidth{0.100375pt}%
\definecolor{currentstroke}{rgb}{0.266667,0.266667,0.266667}%
\pgfsetstrokecolor{currentstroke}%
\pgfsetdash{}{0pt}%
\pgfpathmoveto{\pgfqpoint{3.208643in}{0.638889in}}%
\pgfpathlineto{\pgfqpoint{3.234394in}{0.638889in}}%
\pgfpathlineto{\pgfqpoint{3.234394in}{0.740052in}}%
\pgfpathlineto{\pgfqpoint{3.208643in}{0.740052in}}%
\pgfpathlineto{\pgfqpoint{3.208643in}{0.638889in}}%
\pgfpathclose%
\pgfusepath{stroke,fill}%
\end{pgfscope}%
\begin{pgfscope}%
\pgfpathrectangle{\pgfqpoint{0.781250in}{0.638889in}}{\pgfqpoint{4.843750in}{2.172222in}}%
\pgfusepath{clip}%
\pgfsetbuttcap%
\pgfsetmiterjoin%
\definecolor{currentfill}{rgb}{0.227451,0.192157,0.427451}%
\pgfsetfillcolor{currentfill}%
\pgfsetlinewidth{0.100375pt}%
\definecolor{currentstroke}{rgb}{0.266667,0.266667,0.266667}%
\pgfsetstrokecolor{currentstroke}%
\pgfsetdash{}{0pt}%
\pgfpathmoveto{\pgfqpoint{3.245430in}{0.638889in}}%
\pgfpathlineto{\pgfqpoint{3.271181in}{0.638889in}}%
\pgfpathlineto{\pgfqpoint{3.271181in}{0.711193in}}%
\pgfpathlineto{\pgfqpoint{3.245430in}{0.711193in}}%
\pgfpathlineto{\pgfqpoint{3.245430in}{0.638889in}}%
\pgfpathclose%
\pgfusepath{stroke,fill}%
\end{pgfscope}%
\begin{pgfscope}%
\pgfpathrectangle{\pgfqpoint{0.781250in}{0.638889in}}{\pgfqpoint{4.843750in}{2.172222in}}%
\pgfusepath{clip}%
\pgfsetbuttcap%
\pgfsetmiterjoin%
\definecolor{currentfill}{rgb}{0.227451,0.192157,0.427451}%
\pgfsetfillcolor{currentfill}%
\pgfsetlinewidth{0.100375pt}%
\definecolor{currentstroke}{rgb}{0.266667,0.266667,0.266667}%
\pgfsetstrokecolor{currentstroke}%
\pgfsetdash{}{0pt}%
\pgfpathmoveto{\pgfqpoint{3.282217in}{0.638889in}}%
\pgfpathlineto{\pgfqpoint{3.307968in}{0.638889in}}%
\pgfpathlineto{\pgfqpoint{3.307968in}{0.950448in}}%
\pgfpathlineto{\pgfqpoint{3.282217in}{0.950448in}}%
\pgfpathlineto{\pgfqpoint{3.282217in}{0.638889in}}%
\pgfpathclose%
\pgfusepath{stroke,fill}%
\end{pgfscope}%
\begin{pgfscope}%
\pgfpathrectangle{\pgfqpoint{0.781250in}{0.638889in}}{\pgfqpoint{4.843750in}{2.172222in}}%
\pgfusepath{clip}%
\pgfsetbuttcap%
\pgfsetmiterjoin%
\definecolor{currentfill}{rgb}{0.227451,0.192157,0.427451}%
\pgfsetfillcolor{currentfill}%
\pgfsetlinewidth{0.100375pt}%
\definecolor{currentstroke}{rgb}{0.266667,0.266667,0.266667}%
\pgfsetstrokecolor{currentstroke}%
\pgfsetdash{}{0pt}%
\pgfpathmoveto{\pgfqpoint{3.319004in}{0.638889in}}%
\pgfpathlineto{\pgfqpoint{3.344755in}{0.638889in}}%
\pgfpathlineto{\pgfqpoint{3.344755in}{0.782876in}}%
\pgfpathlineto{\pgfqpoint{3.319004in}{0.782876in}}%
\pgfpathlineto{\pgfqpoint{3.319004in}{0.638889in}}%
\pgfpathclose%
\pgfusepath{stroke,fill}%
\end{pgfscope}%
\begin{pgfscope}%
\pgfpathrectangle{\pgfqpoint{0.781250in}{0.638889in}}{\pgfqpoint{4.843750in}{2.172222in}}%
\pgfusepath{clip}%
\pgfsetbuttcap%
\pgfsetmiterjoin%
\definecolor{currentfill}{rgb}{0.227451,0.192157,0.427451}%
\pgfsetfillcolor{currentfill}%
\pgfsetlinewidth{0.100375pt}%
\definecolor{currentstroke}{rgb}{0.266667,0.266667,0.266667}%
\pgfsetstrokecolor{currentstroke}%
\pgfsetdash{}{0pt}%
\pgfpathmoveto{\pgfqpoint{3.355791in}{0.638889in}}%
\pgfpathlineto{\pgfqpoint{3.381542in}{0.638889in}}%
\pgfpathlineto{\pgfqpoint{3.381542in}{0.892418in}}%
\pgfpathlineto{\pgfqpoint{3.355791in}{0.892418in}}%
\pgfpathlineto{\pgfqpoint{3.355791in}{0.638889in}}%
\pgfpathclose%
\pgfusepath{stroke,fill}%
\end{pgfscope}%
\begin{pgfscope}%
\pgfpathrectangle{\pgfqpoint{0.781250in}{0.638889in}}{\pgfqpoint{4.843750in}{2.172222in}}%
\pgfusepath{clip}%
\pgfsetbuttcap%
\pgfsetmiterjoin%
\definecolor{currentfill}{rgb}{0.164706,0.615686,0.560784}%
\pgfsetfillcolor{currentfill}%
\pgfsetlinewidth{0.100375pt}%
\definecolor{currentstroke}{rgb}{0.266667,0.266667,0.266667}%
\pgfsetstrokecolor{currentstroke}%
\pgfsetdash{}{0pt}%
\pgfpathmoveto{\pgfqpoint{3.392578in}{0.638889in}}%
\pgfpathlineto{\pgfqpoint{3.418329in}{0.638889in}}%
\pgfpathlineto{\pgfqpoint{3.418329in}{0.940207in}}%
\pgfpathlineto{\pgfqpoint{3.392578in}{0.940207in}}%
\pgfpathlineto{\pgfqpoint{3.392578in}{0.638889in}}%
\pgfpathclose%
\pgfusepath{stroke,fill}%
\end{pgfscope}%
\begin{pgfscope}%
\pgfpathrectangle{\pgfqpoint{0.781250in}{0.638889in}}{\pgfqpoint{4.843750in}{2.172222in}}%
\pgfusepath{clip}%
\pgfsetbuttcap%
\pgfsetmiterjoin%
\definecolor{currentfill}{rgb}{0.164706,0.615686,0.560784}%
\pgfsetfillcolor{currentfill}%
\pgfsetlinewidth{0.100375pt}%
\definecolor{currentstroke}{rgb}{0.266667,0.266667,0.266667}%
\pgfsetstrokecolor{currentstroke}%
\pgfsetdash{}{0pt}%
\pgfpathmoveto{\pgfqpoint{3.429365in}{0.638889in}}%
\pgfpathlineto{\pgfqpoint{3.455116in}{0.638889in}}%
\pgfpathlineto{\pgfqpoint{3.455116in}{1.335862in}}%
\pgfpathlineto{\pgfqpoint{3.429365in}{1.335862in}}%
\pgfpathlineto{\pgfqpoint{3.429365in}{0.638889in}}%
\pgfpathclose%
\pgfusepath{stroke,fill}%
\end{pgfscope}%
\begin{pgfscope}%
\pgfpathrectangle{\pgfqpoint{0.781250in}{0.638889in}}{\pgfqpoint{4.843750in}{2.172222in}}%
\pgfusepath{clip}%
\pgfsetbuttcap%
\pgfsetmiterjoin%
\definecolor{currentfill}{rgb}{0.164706,0.615686,0.560784}%
\pgfsetfillcolor{currentfill}%
\pgfsetlinewidth{0.100375pt}%
\definecolor{currentstroke}{rgb}{0.266667,0.266667,0.266667}%
\pgfsetstrokecolor{currentstroke}%
\pgfsetdash{}{0pt}%
\pgfpathmoveto{\pgfqpoint{3.466152in}{0.638889in}}%
\pgfpathlineto{\pgfqpoint{3.491903in}{0.638889in}}%
\pgfpathlineto{\pgfqpoint{3.491903in}{1.910260in}}%
\pgfpathlineto{\pgfqpoint{3.466152in}{1.910260in}}%
\pgfpathlineto{\pgfqpoint{3.466152in}{0.638889in}}%
\pgfpathclose%
\pgfusepath{stroke,fill}%
\end{pgfscope}%
\begin{pgfscope}%
\pgfpathrectangle{\pgfqpoint{0.781250in}{0.638889in}}{\pgfqpoint{4.843750in}{2.172222in}}%
\pgfusepath{clip}%
\pgfsetbuttcap%
\pgfsetmiterjoin%
\definecolor{currentfill}{rgb}{0.164706,0.615686,0.560784}%
\pgfsetfillcolor{currentfill}%
\pgfsetlinewidth{0.100375pt}%
\definecolor{currentstroke}{rgb}{0.266667,0.266667,0.266667}%
\pgfsetstrokecolor{currentstroke}%
\pgfsetdash{}{0pt}%
\pgfpathmoveto{\pgfqpoint{3.502939in}{0.638889in}}%
\pgfpathlineto{\pgfqpoint{3.528690in}{0.638889in}}%
\pgfpathlineto{\pgfqpoint{3.528690in}{1.155878in}}%
\pgfpathlineto{\pgfqpoint{3.502939in}{1.155878in}}%
\pgfpathlineto{\pgfqpoint{3.502939in}{0.638889in}}%
\pgfpathclose%
\pgfusepath{stroke,fill}%
\end{pgfscope}%
\begin{pgfscope}%
\pgfpathrectangle{\pgfqpoint{0.781250in}{0.638889in}}{\pgfqpoint{4.843750in}{2.172222in}}%
\pgfusepath{clip}%
\pgfsetbuttcap%
\pgfsetmiterjoin%
\definecolor{currentfill}{rgb}{0.227451,0.192157,0.427451}%
\pgfsetfillcolor{currentfill}%
\pgfsetlinewidth{0.100375pt}%
\definecolor{currentstroke}{rgb}{0.266667,0.266667,0.266667}%
\pgfsetstrokecolor{currentstroke}%
\pgfsetdash{}{0pt}%
\pgfpathmoveto{\pgfqpoint{3.539726in}{0.638889in}}%
\pgfpathlineto{\pgfqpoint{3.565477in}{0.638889in}}%
\pgfpathlineto{\pgfqpoint{3.565477in}{0.794979in}}%
\pgfpathlineto{\pgfqpoint{3.539726in}{0.794979in}}%
\pgfpathlineto{\pgfqpoint{3.539726in}{0.638889in}}%
\pgfpathclose%
\pgfusepath{stroke,fill}%
\end{pgfscope}%
\begin{pgfscope}%
\pgfpathrectangle{\pgfqpoint{0.781250in}{0.638889in}}{\pgfqpoint{4.843750in}{2.172222in}}%
\pgfusepath{clip}%
\pgfsetbuttcap%
\pgfsetmiterjoin%
\definecolor{currentfill}{rgb}{0.227451,0.192157,0.427451}%
\pgfsetfillcolor{currentfill}%
\pgfsetlinewidth{0.100375pt}%
\definecolor{currentstroke}{rgb}{0.266667,0.266667,0.266667}%
\pgfsetstrokecolor{currentstroke}%
\pgfsetdash{}{0pt}%
\pgfpathmoveto{\pgfqpoint{3.576513in}{0.638889in}}%
\pgfpathlineto{\pgfqpoint{3.602264in}{0.638889in}}%
\pgfpathlineto{\pgfqpoint{3.602264in}{0.950137in}}%
\pgfpathlineto{\pgfqpoint{3.576513in}{0.950137in}}%
\pgfpathlineto{\pgfqpoint{3.576513in}{0.638889in}}%
\pgfpathclose%
\pgfusepath{stroke,fill}%
\end{pgfscope}%
\begin{pgfscope}%
\pgfpathrectangle{\pgfqpoint{0.781250in}{0.638889in}}{\pgfqpoint{4.843750in}{2.172222in}}%
\pgfusepath{clip}%
\pgfsetbuttcap%
\pgfsetmiterjoin%
\definecolor{currentfill}{rgb}{0.227451,0.192157,0.427451}%
\pgfsetfillcolor{currentfill}%
\pgfsetlinewidth{0.100375pt}%
\definecolor{currentstroke}{rgb}{0.266667,0.266667,0.266667}%
\pgfsetstrokecolor{currentstroke}%
\pgfsetdash{}{0pt}%
\pgfpathmoveto{\pgfqpoint{3.613301in}{0.638889in}}%
\pgfpathlineto{\pgfqpoint{3.639051in}{0.638889in}}%
\pgfpathlineto{\pgfqpoint{3.639051in}{0.704056in}}%
\pgfpathlineto{\pgfqpoint{3.613301in}{0.704056in}}%
\pgfpathlineto{\pgfqpoint{3.613301in}{0.638889in}}%
\pgfpathclose%
\pgfusepath{stroke,fill}%
\end{pgfscope}%
\begin{pgfscope}%
\pgfpathrectangle{\pgfqpoint{0.781250in}{0.638889in}}{\pgfqpoint{4.843750in}{2.172222in}}%
\pgfusepath{clip}%
\pgfsetbuttcap%
\pgfsetmiterjoin%
\definecolor{currentfill}{rgb}{0.227451,0.192157,0.427451}%
\pgfsetfillcolor{currentfill}%
\pgfsetlinewidth{0.100375pt}%
\definecolor{currentstroke}{rgb}{0.266667,0.266667,0.266667}%
\pgfsetstrokecolor{currentstroke}%
\pgfsetdash{}{0pt}%
\pgfpathmoveto{\pgfqpoint{3.650088in}{0.638889in}}%
\pgfpathlineto{\pgfqpoint{3.675839in}{0.638889in}}%
\pgfpathlineto{\pgfqpoint{3.675839in}{0.655025in}}%
\pgfpathlineto{\pgfqpoint{3.650088in}{0.655025in}}%
\pgfpathlineto{\pgfqpoint{3.650088in}{0.638889in}}%
\pgfpathclose%
\pgfusepath{stroke,fill}%
\end{pgfscope}%
\begin{pgfscope}%
\pgfpathrectangle{\pgfqpoint{0.781250in}{0.638889in}}{\pgfqpoint{4.843750in}{2.172222in}}%
\pgfusepath{clip}%
\pgfsetbuttcap%
\pgfsetmiterjoin%
\definecolor{currentfill}{rgb}{0.227451,0.192157,0.427451}%
\pgfsetfillcolor{currentfill}%
\pgfsetlinewidth{0.100375pt}%
\definecolor{currentstroke}{rgb}{0.266667,0.266667,0.266667}%
\pgfsetstrokecolor{currentstroke}%
\pgfsetdash{}{0pt}%
\pgfpathmoveto{\pgfqpoint{3.686875in}{0.638889in}}%
\pgfpathlineto{\pgfqpoint{3.712626in}{0.638889in}}%
\pgfpathlineto{\pgfqpoint{3.712626in}{0.673024in}}%
\pgfpathlineto{\pgfqpoint{3.686875in}{0.673024in}}%
\pgfpathlineto{\pgfqpoint{3.686875in}{0.638889in}}%
\pgfpathclose%
\pgfusepath{stroke,fill}%
\end{pgfscope}%
\begin{pgfscope}%
\pgfpathrectangle{\pgfqpoint{0.781250in}{0.638889in}}{\pgfqpoint{4.843750in}{2.172222in}}%
\pgfusepath{clip}%
\pgfsetbuttcap%
\pgfsetmiterjoin%
\definecolor{currentfill}{rgb}{0.227451,0.192157,0.427451}%
\pgfsetfillcolor{currentfill}%
\pgfsetlinewidth{0.100375pt}%
\definecolor{currentstroke}{rgb}{0.266667,0.266667,0.266667}%
\pgfsetstrokecolor{currentstroke}%
\pgfsetdash{}{0pt}%
\pgfpathmoveto{\pgfqpoint{3.723662in}{0.638889in}}%
\pgfpathlineto{\pgfqpoint{3.749413in}{0.638889in}}%
\pgfpathlineto{\pgfqpoint{3.749413in}{0.855180in}}%
\pgfpathlineto{\pgfqpoint{3.723662in}{0.855180in}}%
\pgfpathlineto{\pgfqpoint{3.723662in}{0.638889in}}%
\pgfpathclose%
\pgfusepath{stroke,fill}%
\end{pgfscope}%
\begin{pgfscope}%
\pgfpathrectangle{\pgfqpoint{0.781250in}{0.638889in}}{\pgfqpoint{4.843750in}{2.172222in}}%
\pgfusepath{clip}%
\pgfsetbuttcap%
\pgfsetmiterjoin%
\definecolor{currentfill}{rgb}{0.227451,0.192157,0.427451}%
\pgfsetfillcolor{currentfill}%
\pgfsetlinewidth{0.100375pt}%
\definecolor{currentstroke}{rgb}{0.266667,0.266667,0.266667}%
\pgfsetstrokecolor{currentstroke}%
\pgfsetdash{}{0pt}%
\pgfpathmoveto{\pgfqpoint{3.760449in}{0.638889in}}%
\pgfpathlineto{\pgfqpoint{3.786200in}{0.638889in}}%
\pgfpathlineto{\pgfqpoint{3.786200in}{0.819804in}}%
\pgfpathlineto{\pgfqpoint{3.760449in}{0.819804in}}%
\pgfpathlineto{\pgfqpoint{3.760449in}{0.638889in}}%
\pgfpathclose%
\pgfusepath{stroke,fill}%
\end{pgfscope}%
\begin{pgfscope}%
\pgfpathrectangle{\pgfqpoint{0.781250in}{0.638889in}}{\pgfqpoint{4.843750in}{2.172222in}}%
\pgfusepath{clip}%
\pgfsetbuttcap%
\pgfsetmiterjoin%
\definecolor{currentfill}{rgb}{0.227451,0.192157,0.427451}%
\pgfsetfillcolor{currentfill}%
\pgfsetlinewidth{0.100375pt}%
\definecolor{currentstroke}{rgb}{0.266667,0.266667,0.266667}%
\pgfsetstrokecolor{currentstroke}%
\pgfsetdash{}{0pt}%
\pgfpathmoveto{\pgfqpoint{3.797236in}{0.638889in}}%
\pgfpathlineto{\pgfqpoint{3.822987in}{0.638889in}}%
\pgfpathlineto{\pgfqpoint{3.822987in}{0.661232in}}%
\pgfpathlineto{\pgfqpoint{3.797236in}{0.661232in}}%
\pgfpathlineto{\pgfqpoint{3.797236in}{0.638889in}}%
\pgfpathclose%
\pgfusepath{stroke,fill}%
\end{pgfscope}%
\begin{pgfscope}%
\pgfpathrectangle{\pgfqpoint{0.781250in}{0.638889in}}{\pgfqpoint{4.843750in}{2.172222in}}%
\pgfusepath{clip}%
\pgfsetbuttcap%
\pgfsetmiterjoin%
\definecolor{currentfill}{rgb}{0.164706,0.615686,0.560784}%
\pgfsetfillcolor{currentfill}%
\pgfsetlinewidth{0.100375pt}%
\definecolor{currentstroke}{rgb}{0.266667,0.266667,0.266667}%
\pgfsetstrokecolor{currentstroke}%
\pgfsetdash{}{0pt}%
\pgfpathmoveto{\pgfqpoint{3.834023in}{0.638889in}}%
\pgfpathlineto{\pgfqpoint{3.859774in}{0.638889in}}%
\pgfpathlineto{\pgfqpoint{3.859774in}{1.175117in}}%
\pgfpathlineto{\pgfqpoint{3.834023in}{1.175117in}}%
\pgfpathlineto{\pgfqpoint{3.834023in}{0.638889in}}%
\pgfpathclose%
\pgfusepath{stroke,fill}%
\end{pgfscope}%
\begin{pgfscope}%
\pgfpathrectangle{\pgfqpoint{0.781250in}{0.638889in}}{\pgfqpoint{4.843750in}{2.172222in}}%
\pgfusepath{clip}%
\pgfsetbuttcap%
\pgfsetmiterjoin%
\definecolor{currentfill}{rgb}{0.164706,0.615686,0.560784}%
\pgfsetfillcolor{currentfill}%
\pgfsetlinewidth{0.100375pt}%
\definecolor{currentstroke}{rgb}{0.266667,0.266667,0.266667}%
\pgfsetstrokecolor{currentstroke}%
\pgfsetdash{}{0pt}%
\pgfpathmoveto{\pgfqpoint{3.870810in}{0.638889in}}%
\pgfpathlineto{\pgfqpoint{3.896561in}{0.638889in}}%
\pgfpathlineto{\pgfqpoint{3.896561in}{1.030199in}}%
\pgfpathlineto{\pgfqpoint{3.870810in}{1.030199in}}%
\pgfpathlineto{\pgfqpoint{3.870810in}{0.638889in}}%
\pgfpathclose%
\pgfusepath{stroke,fill}%
\end{pgfscope}%
\begin{pgfscope}%
\pgfpathrectangle{\pgfqpoint{0.781250in}{0.638889in}}{\pgfqpoint{4.843750in}{2.172222in}}%
\pgfusepath{clip}%
\pgfsetbuttcap%
\pgfsetmiterjoin%
\definecolor{currentfill}{rgb}{0.164706,0.615686,0.560784}%
\pgfsetfillcolor{currentfill}%
\pgfsetlinewidth{0.100375pt}%
\definecolor{currentstroke}{rgb}{0.266667,0.266667,0.266667}%
\pgfsetstrokecolor{currentstroke}%
\pgfsetdash{}{0pt}%
\pgfpathmoveto{\pgfqpoint{3.907597in}{0.638889in}}%
\pgfpathlineto{\pgfqpoint{3.933348in}{0.638889in}}%
\pgfpathlineto{\pgfqpoint{3.933348in}{1.497537in}}%
\pgfpathlineto{\pgfqpoint{3.907597in}{1.497537in}}%
\pgfpathlineto{\pgfqpoint{3.907597in}{0.638889in}}%
\pgfpathclose%
\pgfusepath{stroke,fill}%
\end{pgfscope}%
\begin{pgfscope}%
\pgfpathrectangle{\pgfqpoint{0.781250in}{0.638889in}}{\pgfqpoint{4.843750in}{2.172222in}}%
\pgfusepath{clip}%
\pgfsetbuttcap%
\pgfsetmiterjoin%
\definecolor{currentfill}{rgb}{0.164706,0.615686,0.560784}%
\pgfsetfillcolor{currentfill}%
\pgfsetlinewidth{0.100375pt}%
\definecolor{currentstroke}{rgb}{0.266667,0.266667,0.266667}%
\pgfsetstrokecolor{currentstroke}%
\pgfsetdash{}{0pt}%
\pgfpathmoveto{\pgfqpoint{3.944384in}{0.638889in}}%
\pgfpathlineto{\pgfqpoint{3.970135in}{0.638889in}}%
\pgfpathlineto{\pgfqpoint{3.970135in}{0.905452in}}%
\pgfpathlineto{\pgfqpoint{3.944384in}{0.905452in}}%
\pgfpathlineto{\pgfqpoint{3.944384in}{0.638889in}}%
\pgfpathclose%
\pgfusepath{stroke,fill}%
\end{pgfscope}%
\begin{pgfscope}%
\pgfpathrectangle{\pgfqpoint{0.781250in}{0.638889in}}{\pgfqpoint{4.843750in}{2.172222in}}%
\pgfusepath{clip}%
\pgfsetbuttcap%
\pgfsetmiterjoin%
\definecolor{currentfill}{rgb}{0.227451,0.192157,0.427451}%
\pgfsetfillcolor{currentfill}%
\pgfsetlinewidth{0.100375pt}%
\definecolor{currentstroke}{rgb}{0.266667,0.266667,0.266667}%
\pgfsetstrokecolor{currentstroke}%
\pgfsetdash{}{0pt}%
\pgfpathmoveto{\pgfqpoint{3.981171in}{0.638889in}}%
\pgfpathlineto{\pgfqpoint{4.006922in}{0.638889in}}%
\pgfpathlineto{\pgfqpoint{4.006922in}{0.755568in}}%
\pgfpathlineto{\pgfqpoint{3.981171in}{0.755568in}}%
\pgfpathlineto{\pgfqpoint{3.981171in}{0.638889in}}%
\pgfpathclose%
\pgfusepath{stroke,fill}%
\end{pgfscope}%
\begin{pgfscope}%
\pgfpathrectangle{\pgfqpoint{0.781250in}{0.638889in}}{\pgfqpoint{4.843750in}{2.172222in}}%
\pgfusepath{clip}%
\pgfsetbuttcap%
\pgfsetmiterjoin%
\definecolor{currentfill}{rgb}{0.227451,0.192157,0.427451}%
\pgfsetfillcolor{currentfill}%
\pgfsetlinewidth{0.100375pt}%
\definecolor{currentstroke}{rgb}{0.266667,0.266667,0.266667}%
\pgfsetstrokecolor{currentstroke}%
\pgfsetdash{}{0pt}%
\pgfpathmoveto{\pgfqpoint{4.017958in}{0.638889in}}%
\pgfpathlineto{\pgfqpoint{4.043709in}{0.638889in}}%
\pgfpathlineto{\pgfqpoint{4.043709in}{0.836561in}}%
\pgfpathlineto{\pgfqpoint{4.017958in}{0.836561in}}%
\pgfpathlineto{\pgfqpoint{4.017958in}{0.638889in}}%
\pgfpathclose%
\pgfusepath{stroke,fill}%
\end{pgfscope}%
\begin{pgfscope}%
\pgfpathrectangle{\pgfqpoint{0.781250in}{0.638889in}}{\pgfqpoint{4.843750in}{2.172222in}}%
\pgfusepath{clip}%
\pgfsetbuttcap%
\pgfsetmiterjoin%
\definecolor{currentfill}{rgb}{0.227451,0.192157,0.427451}%
\pgfsetfillcolor{currentfill}%
\pgfsetlinewidth{0.100375pt}%
\definecolor{currentstroke}{rgb}{0.266667,0.266667,0.266667}%
\pgfsetstrokecolor{currentstroke}%
\pgfsetdash{}{0pt}%
\pgfpathmoveto{\pgfqpoint{4.054745in}{0.638889in}}%
\pgfpathlineto{\pgfqpoint{4.080496in}{0.638889in}}%
\pgfpathlineto{\pgfqpoint{4.080496in}{0.815770in}}%
\pgfpathlineto{\pgfqpoint{4.054745in}{0.815770in}}%
\pgfpathlineto{\pgfqpoint{4.054745in}{0.638889in}}%
\pgfpathclose%
\pgfusepath{stroke,fill}%
\end{pgfscope}%
\begin{pgfscope}%
\pgfpathrectangle{\pgfqpoint{0.781250in}{0.638889in}}{\pgfqpoint{4.843750in}{2.172222in}}%
\pgfusepath{clip}%
\pgfsetbuttcap%
\pgfsetmiterjoin%
\definecolor{currentfill}{rgb}{0.227451,0.192157,0.427451}%
\pgfsetfillcolor{currentfill}%
\pgfsetlinewidth{0.100375pt}%
\definecolor{currentstroke}{rgb}{0.266667,0.266667,0.266667}%
\pgfsetstrokecolor{currentstroke}%
\pgfsetdash{}{0pt}%
\pgfpathmoveto{\pgfqpoint{4.091532in}{0.638889in}}%
\pgfpathlineto{\pgfqpoint{4.117283in}{0.638889in}}%
\pgfpathlineto{\pgfqpoint{4.117283in}{0.748741in}}%
\pgfpathlineto{\pgfqpoint{4.091532in}{0.748741in}}%
\pgfpathlineto{\pgfqpoint{4.091532in}{0.638889in}}%
\pgfpathclose%
\pgfusepath{stroke,fill}%
\end{pgfscope}%
\begin{pgfscope}%
\pgfpathrectangle{\pgfqpoint{0.781250in}{0.638889in}}{\pgfqpoint{4.843750in}{2.172222in}}%
\pgfusepath{clip}%
\pgfsetbuttcap%
\pgfsetmiterjoin%
\definecolor{currentfill}{rgb}{0.227451,0.192157,0.427451}%
\pgfsetfillcolor{currentfill}%
\pgfsetlinewidth{0.100375pt}%
\definecolor{currentstroke}{rgb}{0.266667,0.266667,0.266667}%
\pgfsetstrokecolor{currentstroke}%
\pgfsetdash{}{0pt}%
\pgfpathmoveto{\pgfqpoint{4.128319in}{0.638889in}}%
\pgfpathlineto{\pgfqpoint{4.154070in}{0.638889in}}%
\pgfpathlineto{\pgfqpoint{4.154070in}{0.653784in}}%
\pgfpathlineto{\pgfqpoint{4.128319in}{0.653784in}}%
\pgfpathlineto{\pgfqpoint{4.128319in}{0.638889in}}%
\pgfpathclose%
\pgfusepath{stroke,fill}%
\end{pgfscope}%
\begin{pgfscope}%
\pgfpathrectangle{\pgfqpoint{0.781250in}{0.638889in}}{\pgfqpoint{4.843750in}{2.172222in}}%
\pgfusepath{clip}%
\pgfsetbuttcap%
\pgfsetmiterjoin%
\definecolor{currentfill}{rgb}{0.227451,0.192157,0.427451}%
\pgfsetfillcolor{currentfill}%
\pgfsetlinewidth{0.100375pt}%
\definecolor{currentstroke}{rgb}{0.266667,0.266667,0.266667}%
\pgfsetstrokecolor{currentstroke}%
\pgfsetdash{}{0pt}%
\pgfpathmoveto{\pgfqpoint{4.165106in}{0.638889in}}%
\pgfpathlineto{\pgfqpoint{4.190857in}{0.638889in}}%
\pgfpathlineto{\pgfqpoint{4.190857in}{0.700332in}}%
\pgfpathlineto{\pgfqpoint{4.165106in}{0.700332in}}%
\pgfpathlineto{\pgfqpoint{4.165106in}{0.638889in}}%
\pgfpathclose%
\pgfusepath{stroke,fill}%
\end{pgfscope}%
\begin{pgfscope}%
\pgfpathrectangle{\pgfqpoint{0.781250in}{0.638889in}}{\pgfqpoint{4.843750in}{2.172222in}}%
\pgfusepath{clip}%
\pgfsetbuttcap%
\pgfsetmiterjoin%
\definecolor{currentfill}{rgb}{0.227451,0.192157,0.427451}%
\pgfsetfillcolor{currentfill}%
\pgfsetlinewidth{0.100375pt}%
\definecolor{currentstroke}{rgb}{0.266667,0.266667,0.266667}%
\pgfsetstrokecolor{currentstroke}%
\pgfsetdash{}{0pt}%
\pgfpathmoveto{\pgfqpoint{4.201893in}{0.638889in}}%
\pgfpathlineto{\pgfqpoint{4.227644in}{0.638889in}}%
\pgfpathlineto{\pgfqpoint{4.227644in}{0.789703in}}%
\pgfpathlineto{\pgfqpoint{4.201893in}{0.789703in}}%
\pgfpathlineto{\pgfqpoint{4.201893in}{0.638889in}}%
\pgfpathclose%
\pgfusepath{stroke,fill}%
\end{pgfscope}%
\begin{pgfscope}%
\pgfpathrectangle{\pgfqpoint{0.781250in}{0.638889in}}{\pgfqpoint{4.843750in}{2.172222in}}%
\pgfusepath{clip}%
\pgfsetbuttcap%
\pgfsetmiterjoin%
\definecolor{currentfill}{rgb}{0.227451,0.192157,0.427451}%
\pgfsetfillcolor{currentfill}%
\pgfsetlinewidth{0.100375pt}%
\definecolor{currentstroke}{rgb}{0.266667,0.266667,0.266667}%
\pgfsetstrokecolor{currentstroke}%
\pgfsetdash{}{0pt}%
\pgfpathmoveto{\pgfqpoint{4.238680in}{0.638889in}}%
\pgfpathlineto{\pgfqpoint{4.264431in}{0.638889in}}%
\pgfpathlineto{\pgfqpoint{4.264431in}{1.373410in}}%
\pgfpathlineto{\pgfqpoint{4.238680in}{1.373410in}}%
\pgfpathlineto{\pgfqpoint{4.238680in}{0.638889in}}%
\pgfpathclose%
\pgfusepath{stroke,fill}%
\end{pgfscope}%
\begin{pgfscope}%
\pgfpathrectangle{\pgfqpoint{0.781250in}{0.638889in}}{\pgfqpoint{4.843750in}{2.172222in}}%
\pgfusepath{clip}%
\pgfsetbuttcap%
\pgfsetmiterjoin%
\definecolor{currentfill}{rgb}{0.164706,0.615686,0.560784}%
\pgfsetfillcolor{currentfill}%
\pgfsetlinewidth{0.100375pt}%
\definecolor{currentstroke}{rgb}{0.266667,0.266667,0.266667}%
\pgfsetstrokecolor{currentstroke}%
\pgfsetdash{}{0pt}%
\pgfpathmoveto{\pgfqpoint{4.275467in}{0.638889in}}%
\pgfpathlineto{\pgfqpoint{4.301218in}{0.638889in}}%
\pgfpathlineto{\pgfqpoint{4.301218in}{1.405994in}}%
\pgfpathlineto{\pgfqpoint{4.275467in}{1.405994in}}%
\pgfpathlineto{\pgfqpoint{4.275467in}{0.638889in}}%
\pgfpathclose%
\pgfusepath{stroke,fill}%
\end{pgfscope}%
\begin{pgfscope}%
\pgfpathrectangle{\pgfqpoint{0.781250in}{0.638889in}}{\pgfqpoint{4.843750in}{2.172222in}}%
\pgfusepath{clip}%
\pgfsetbuttcap%
\pgfsetmiterjoin%
\definecolor{currentfill}{rgb}{0.164706,0.615686,0.560784}%
\pgfsetfillcolor{currentfill}%
\pgfsetlinewidth{0.100375pt}%
\definecolor{currentstroke}{rgb}{0.266667,0.266667,0.266667}%
\pgfsetstrokecolor{currentstroke}%
\pgfsetdash{}{0pt}%
\pgfpathmoveto{\pgfqpoint{4.312254in}{0.638889in}}%
\pgfpathlineto{\pgfqpoint{4.338005in}{0.638889in}}%
\pgfpathlineto{\pgfqpoint{4.338005in}{2.647263in}}%
\pgfpathlineto{\pgfqpoint{4.312254in}{2.647263in}}%
\pgfpathlineto{\pgfqpoint{4.312254in}{0.638889in}}%
\pgfpathclose%
\pgfusepath{stroke,fill}%
\end{pgfscope}%
\begin{pgfscope}%
\pgfpathrectangle{\pgfqpoint{0.781250in}{0.638889in}}{\pgfqpoint{4.843750in}{2.172222in}}%
\pgfusepath{clip}%
\pgfsetbuttcap%
\pgfsetmiterjoin%
\definecolor{currentfill}{rgb}{0.164706,0.615686,0.560784}%
\pgfsetfillcolor{currentfill}%
\pgfsetlinewidth{0.100375pt}%
\definecolor{currentstroke}{rgb}{0.266667,0.266667,0.266667}%
\pgfsetstrokecolor{currentstroke}%
\pgfsetdash{}{0pt}%
\pgfpathmoveto{\pgfqpoint{4.349041in}{0.638889in}}%
\pgfpathlineto{\pgfqpoint{4.374792in}{0.638889in}}%
\pgfpathlineto{\pgfqpoint{4.374792in}{1.161774in}}%
\pgfpathlineto{\pgfqpoint{4.349041in}{1.161774in}}%
\pgfpathlineto{\pgfqpoint{4.349041in}{0.638889in}}%
\pgfpathclose%
\pgfusepath{stroke,fill}%
\end{pgfscope}%
\begin{pgfscope}%
\pgfpathrectangle{\pgfqpoint{0.781250in}{0.638889in}}{\pgfqpoint{4.843750in}{2.172222in}}%
\pgfusepath{clip}%
\pgfsetbuttcap%
\pgfsetmiterjoin%
\definecolor{currentfill}{rgb}{0.164706,0.615686,0.560784}%
\pgfsetfillcolor{currentfill}%
\pgfsetlinewidth{0.100375pt}%
\definecolor{currentstroke}{rgb}{0.266667,0.266667,0.266667}%
\pgfsetstrokecolor{currentstroke}%
\pgfsetdash{}{0pt}%
\pgfpathmoveto{\pgfqpoint{4.385828in}{0.638889in}}%
\pgfpathlineto{\pgfqpoint{4.411579in}{0.638889in}}%
\pgfpathlineto{\pgfqpoint{4.411579in}{1.232216in}}%
\pgfpathlineto{\pgfqpoint{4.385828in}{1.232216in}}%
\pgfpathlineto{\pgfqpoint{4.385828in}{0.638889in}}%
\pgfpathclose%
\pgfusepath{stroke,fill}%
\end{pgfscope}%
\begin{pgfscope}%
\pgfpathrectangle{\pgfqpoint{0.781250in}{0.638889in}}{\pgfqpoint{4.843750in}{2.172222in}}%
\pgfusepath{clip}%
\pgfsetbuttcap%
\pgfsetmiterjoin%
\definecolor{currentfill}{rgb}{0.227451,0.192157,0.427451}%
\pgfsetfillcolor{currentfill}%
\pgfsetlinewidth{0.100375pt}%
\definecolor{currentstroke}{rgb}{0.266667,0.266667,0.266667}%
\pgfsetstrokecolor{currentstroke}%
\pgfsetdash{}{0pt}%
\pgfpathmoveto{\pgfqpoint{4.422615in}{0.638889in}}%
\pgfpathlineto{\pgfqpoint{4.448366in}{0.638889in}}%
\pgfpathlineto{\pgfqpoint{4.448366in}{0.680161in}}%
\pgfpathlineto{\pgfqpoint{4.422615in}{0.680161in}}%
\pgfpathlineto{\pgfqpoint{4.422615in}{0.638889in}}%
\pgfpathclose%
\pgfusepath{stroke,fill}%
\end{pgfscope}%
\begin{pgfscope}%
\pgfpathrectangle{\pgfqpoint{0.781250in}{0.638889in}}{\pgfqpoint{4.843750in}{2.172222in}}%
\pgfusepath{clip}%
\pgfsetbuttcap%
\pgfsetmiterjoin%
\definecolor{currentfill}{rgb}{0.227451,0.192157,0.427451}%
\pgfsetfillcolor{currentfill}%
\pgfsetlinewidth{0.100375pt}%
\definecolor{currentstroke}{rgb}{0.266667,0.266667,0.266667}%
\pgfsetstrokecolor{currentstroke}%
\pgfsetdash{}{0pt}%
\pgfpathmoveto{\pgfqpoint{4.459403in}{0.638889in}}%
\pgfpathlineto{\pgfqpoint{4.485153in}{0.638889in}}%
\pgfpathlineto{\pgfqpoint{4.485153in}{0.963791in}}%
\pgfpathlineto{\pgfqpoint{4.459403in}{0.963791in}}%
\pgfpathlineto{\pgfqpoint{4.459403in}{0.638889in}}%
\pgfpathclose%
\pgfusepath{stroke,fill}%
\end{pgfscope}%
\begin{pgfscope}%
\pgfpathrectangle{\pgfqpoint{0.781250in}{0.638889in}}{\pgfqpoint{4.843750in}{2.172222in}}%
\pgfusepath{clip}%
\pgfsetbuttcap%
\pgfsetmiterjoin%
\definecolor{currentfill}{rgb}{0.227451,0.192157,0.427451}%
\pgfsetfillcolor{currentfill}%
\pgfsetlinewidth{0.100375pt}%
\definecolor{currentstroke}{rgb}{0.266667,0.266667,0.266667}%
\pgfsetstrokecolor{currentstroke}%
\pgfsetdash{}{0pt}%
\pgfpathmoveto{\pgfqpoint{4.496190in}{0.638889in}}%
\pgfpathlineto{\pgfqpoint{4.521941in}{0.638889in}}%
\pgfpathlineto{\pgfqpoint{4.521941in}{1.061541in}}%
\pgfpathlineto{\pgfqpoint{4.496190in}{1.061541in}}%
\pgfpathlineto{\pgfqpoint{4.496190in}{0.638889in}}%
\pgfpathclose%
\pgfusepath{stroke,fill}%
\end{pgfscope}%
\begin{pgfscope}%
\pgfpathrectangle{\pgfqpoint{0.781250in}{0.638889in}}{\pgfqpoint{4.843750in}{2.172222in}}%
\pgfusepath{clip}%
\pgfsetbuttcap%
\pgfsetmiterjoin%
\definecolor{currentfill}{rgb}{0.227451,0.192157,0.427451}%
\pgfsetfillcolor{currentfill}%
\pgfsetlinewidth{0.100375pt}%
\definecolor{currentstroke}{rgb}{0.266667,0.266667,0.266667}%
\pgfsetstrokecolor{currentstroke}%
\pgfsetdash{}{0pt}%
\pgfpathmoveto{\pgfqpoint{4.532977in}{0.638889in}}%
\pgfpathlineto{\pgfqpoint{4.558728in}{0.638889in}}%
\pgfpathlineto{\pgfqpoint{4.558728in}{0.808633in}}%
\pgfpathlineto{\pgfqpoint{4.532977in}{0.808633in}}%
\pgfpathlineto{\pgfqpoint{4.532977in}{0.638889in}}%
\pgfpathclose%
\pgfusepath{stroke,fill}%
\end{pgfscope}%
\begin{pgfscope}%
\pgfpathrectangle{\pgfqpoint{0.781250in}{0.638889in}}{\pgfqpoint{4.843750in}{2.172222in}}%
\pgfusepath{clip}%
\pgfsetbuttcap%
\pgfsetmiterjoin%
\definecolor{currentfill}{rgb}{0.227451,0.192157,0.427451}%
\pgfsetfillcolor{currentfill}%
\pgfsetlinewidth{0.100375pt}%
\definecolor{currentstroke}{rgb}{0.266667,0.266667,0.266667}%
\pgfsetstrokecolor{currentstroke}%
\pgfsetdash{}{0pt}%
\pgfpathmoveto{\pgfqpoint{4.569764in}{0.638889in}}%
\pgfpathlineto{\pgfqpoint{4.595515in}{0.638889in}}%
\pgfpathlineto{\pgfqpoint{4.595515in}{1.045094in}}%
\pgfpathlineto{\pgfqpoint{4.569764in}{1.045094in}}%
\pgfpathlineto{\pgfqpoint{4.569764in}{0.638889in}}%
\pgfpathclose%
\pgfusepath{stroke,fill}%
\end{pgfscope}%
\begin{pgfscope}%
\pgfpathrectangle{\pgfqpoint{0.781250in}{0.638889in}}{\pgfqpoint{4.843750in}{2.172222in}}%
\pgfusepath{clip}%
\pgfsetbuttcap%
\pgfsetmiterjoin%
\definecolor{currentfill}{rgb}{0.227451,0.192157,0.427451}%
\pgfsetfillcolor{currentfill}%
\pgfsetlinewidth{0.100375pt}%
\definecolor{currentstroke}{rgb}{0.266667,0.266667,0.266667}%
\pgfsetstrokecolor{currentstroke}%
\pgfsetdash{}{0pt}%
\pgfpathmoveto{\pgfqpoint{4.606551in}{0.638889in}}%
\pgfpathlineto{\pgfqpoint{4.632302in}{0.638889in}}%
\pgfpathlineto{\pgfqpoint{4.632302in}{0.872248in}}%
\pgfpathlineto{\pgfqpoint{4.606551in}{0.872248in}}%
\pgfpathlineto{\pgfqpoint{4.606551in}{0.638889in}}%
\pgfpathclose%
\pgfusepath{stroke,fill}%
\end{pgfscope}%
\begin{pgfscope}%
\pgfpathrectangle{\pgfqpoint{0.781250in}{0.638889in}}{\pgfqpoint{4.843750in}{2.172222in}}%
\pgfusepath{clip}%
\pgfsetbuttcap%
\pgfsetmiterjoin%
\definecolor{currentfill}{rgb}{0.227451,0.192157,0.427451}%
\pgfsetfillcolor{currentfill}%
\pgfsetlinewidth{0.100375pt}%
\definecolor{currentstroke}{rgb}{0.266667,0.266667,0.266667}%
\pgfsetstrokecolor{currentstroke}%
\pgfsetdash{}{0pt}%
\pgfpathmoveto{\pgfqpoint{4.643338in}{0.638889in}}%
\pgfpathlineto{\pgfqpoint{4.669089in}{0.638889in}}%
\pgfpathlineto{\pgfqpoint{4.669089in}{0.814529in}}%
\pgfpathlineto{\pgfqpoint{4.643338in}{0.814529in}}%
\pgfpathlineto{\pgfqpoint{4.643338in}{0.638889in}}%
\pgfpathclose%
\pgfusepath{stroke,fill}%
\end{pgfscope}%
\begin{pgfscope}%
\pgfpathrectangle{\pgfqpoint{0.781250in}{0.638889in}}{\pgfqpoint{4.843750in}{2.172222in}}%
\pgfusepath{clip}%
\pgfsetbuttcap%
\pgfsetmiterjoin%
\definecolor{currentfill}{rgb}{0.227451,0.192157,0.427451}%
\pgfsetfillcolor{currentfill}%
\pgfsetlinewidth{0.100375pt}%
\definecolor{currentstroke}{rgb}{0.266667,0.266667,0.266667}%
\pgfsetstrokecolor{currentstroke}%
\pgfsetdash{}{0pt}%
\pgfpathmoveto{\pgfqpoint{4.680125in}{0.638889in}}%
\pgfpathlineto{\pgfqpoint{4.705876in}{0.638889in}}%
\pgfpathlineto{\pgfqpoint{4.705876in}{0.912899in}}%
\pgfpathlineto{\pgfqpoint{4.680125in}{0.912899in}}%
\pgfpathlineto{\pgfqpoint{4.680125in}{0.638889in}}%
\pgfpathclose%
\pgfusepath{stroke,fill}%
\end{pgfscope}%
\begin{pgfscope}%
\pgfpathrectangle{\pgfqpoint{0.781250in}{0.638889in}}{\pgfqpoint{4.843750in}{2.172222in}}%
\pgfusepath{clip}%
\pgfsetbuttcap%
\pgfsetmiterjoin%
\definecolor{currentfill}{rgb}{0.164706,0.615686,0.560784}%
\pgfsetfillcolor{currentfill}%
\pgfsetlinewidth{0.100375pt}%
\definecolor{currentstroke}{rgb}{0.266667,0.266667,0.266667}%
\pgfsetstrokecolor{currentstroke}%
\pgfsetdash{}{0pt}%
\pgfpathmoveto{\pgfqpoint{4.716912in}{0.638889in}}%
\pgfpathlineto{\pgfqpoint{4.742663in}{0.638889in}}%
\pgfpathlineto{\pgfqpoint{4.742663in}{0.880006in}}%
\pgfpathlineto{\pgfqpoint{4.716912in}{0.880006in}}%
\pgfpathlineto{\pgfqpoint{4.716912in}{0.638889in}}%
\pgfpathclose%
\pgfusepath{stroke,fill}%
\end{pgfscope}%
\begin{pgfscope}%
\pgfpathrectangle{\pgfqpoint{0.781250in}{0.638889in}}{\pgfqpoint{4.843750in}{2.172222in}}%
\pgfusepath{clip}%
\pgfsetbuttcap%
\pgfsetmiterjoin%
\definecolor{currentfill}{rgb}{0.164706,0.615686,0.560784}%
\pgfsetfillcolor{currentfill}%
\pgfsetlinewidth{0.100375pt}%
\definecolor{currentstroke}{rgb}{0.266667,0.266667,0.266667}%
\pgfsetstrokecolor{currentstroke}%
\pgfsetdash{}{0pt}%
\pgfpathmoveto{\pgfqpoint{4.753699in}{0.638889in}}%
\pgfpathlineto{\pgfqpoint{4.779450in}{0.638889in}}%
\pgfpathlineto{\pgfqpoint{4.779450in}{1.648972in}}%
\pgfpathlineto{\pgfqpoint{4.753699in}{1.648972in}}%
\pgfpathlineto{\pgfqpoint{4.753699in}{0.638889in}}%
\pgfpathclose%
\pgfusepath{stroke,fill}%
\end{pgfscope}%
\begin{pgfscope}%
\pgfpathrectangle{\pgfqpoint{0.781250in}{0.638889in}}{\pgfqpoint{4.843750in}{2.172222in}}%
\pgfusepath{clip}%
\pgfsetbuttcap%
\pgfsetmiterjoin%
\definecolor{currentfill}{rgb}{0.164706,0.615686,0.560784}%
\pgfsetfillcolor{currentfill}%
\pgfsetlinewidth{0.100375pt}%
\definecolor{currentstroke}{rgb}{0.266667,0.266667,0.266667}%
\pgfsetstrokecolor{currentstroke}%
\pgfsetdash{}{0pt}%
\pgfpathmoveto{\pgfqpoint{4.790486in}{0.638889in}}%
\pgfpathlineto{\pgfqpoint{4.816237in}{0.638889in}}%
\pgfpathlineto{\pgfqpoint{4.816237in}{0.754327in}}%
\pgfpathlineto{\pgfqpoint{4.790486in}{0.754327in}}%
\pgfpathlineto{\pgfqpoint{4.790486in}{0.638889in}}%
\pgfpathclose%
\pgfusepath{stroke,fill}%
\end{pgfscope}%
\begin{pgfscope}%
\pgfpathrectangle{\pgfqpoint{0.781250in}{0.638889in}}{\pgfqpoint{4.843750in}{2.172222in}}%
\pgfusepath{clip}%
\pgfsetbuttcap%
\pgfsetmiterjoin%
\definecolor{currentfill}{rgb}{0.164706,0.615686,0.560784}%
\pgfsetfillcolor{currentfill}%
\pgfsetlinewidth{0.100375pt}%
\definecolor{currentstroke}{rgb}{0.266667,0.266667,0.266667}%
\pgfsetstrokecolor{currentstroke}%
\pgfsetdash{}{0pt}%
\pgfpathmoveto{\pgfqpoint{4.827273in}{0.638889in}}%
\pgfpathlineto{\pgfqpoint{4.853024in}{0.638889in}}%
\pgfpathlineto{\pgfqpoint{4.853024in}{1.151223in}}%
\pgfpathlineto{\pgfqpoint{4.827273in}{1.151223in}}%
\pgfpathlineto{\pgfqpoint{4.827273in}{0.638889in}}%
\pgfpathclose%
\pgfusepath{stroke,fill}%
\end{pgfscope}%
\begin{pgfscope}%
\pgfpathrectangle{\pgfqpoint{0.781250in}{0.638889in}}{\pgfqpoint{4.843750in}{2.172222in}}%
\pgfusepath{clip}%
\pgfsetbuttcap%
\pgfsetmiterjoin%
\definecolor{currentfill}{rgb}{0.227451,0.192157,0.427451}%
\pgfsetfillcolor{currentfill}%
\pgfsetlinewidth{0.100375pt}%
\definecolor{currentstroke}{rgb}{0.266667,0.266667,0.266667}%
\pgfsetstrokecolor{currentstroke}%
\pgfsetdash{}{0pt}%
\pgfpathmoveto{\pgfqpoint{4.864060in}{0.638889in}}%
\pgfpathlineto{\pgfqpoint{4.889811in}{0.638889in}}%
\pgfpathlineto{\pgfqpoint{4.889811in}{0.861076in}}%
\pgfpathlineto{\pgfqpoint{4.864060in}{0.861076in}}%
\pgfpathlineto{\pgfqpoint{4.864060in}{0.638889in}}%
\pgfpathclose%
\pgfusepath{stroke,fill}%
\end{pgfscope}%
\begin{pgfscope}%
\pgfpathrectangle{\pgfqpoint{0.781250in}{0.638889in}}{\pgfqpoint{4.843750in}{2.172222in}}%
\pgfusepath{clip}%
\pgfsetbuttcap%
\pgfsetmiterjoin%
\definecolor{currentfill}{rgb}{0.227451,0.192157,0.427451}%
\pgfsetfillcolor{currentfill}%
\pgfsetlinewidth{0.100375pt}%
\definecolor{currentstroke}{rgb}{0.266667,0.266667,0.266667}%
\pgfsetstrokecolor{currentstroke}%
\pgfsetdash{}{0pt}%
\pgfpathmoveto{\pgfqpoint{4.900847in}{0.638889in}}%
\pgfpathlineto{\pgfqpoint{4.926598in}{0.638889in}}%
\pgfpathlineto{\pgfqpoint{4.926598in}{0.795289in}}%
\pgfpathlineto{\pgfqpoint{4.900847in}{0.795289in}}%
\pgfpathlineto{\pgfqpoint{4.900847in}{0.638889in}}%
\pgfpathclose%
\pgfusepath{stroke,fill}%
\end{pgfscope}%
\begin{pgfscope}%
\pgfpathrectangle{\pgfqpoint{0.781250in}{0.638889in}}{\pgfqpoint{4.843750in}{2.172222in}}%
\pgfusepath{clip}%
\pgfsetbuttcap%
\pgfsetmiterjoin%
\definecolor{currentfill}{rgb}{0.227451,0.192157,0.427451}%
\pgfsetfillcolor{currentfill}%
\pgfsetlinewidth{0.100375pt}%
\definecolor{currentstroke}{rgb}{0.266667,0.266667,0.266667}%
\pgfsetstrokecolor{currentstroke}%
\pgfsetdash{}{0pt}%
\pgfpathmoveto{\pgfqpoint{4.937634in}{0.638889in}}%
\pgfpathlineto{\pgfqpoint{4.963385in}{0.638889in}}%
\pgfpathlineto{\pgfqpoint{4.963385in}{0.747500in}}%
\pgfpathlineto{\pgfqpoint{4.937634in}{0.747500in}}%
\pgfpathlineto{\pgfqpoint{4.937634in}{0.638889in}}%
\pgfpathclose%
\pgfusepath{stroke,fill}%
\end{pgfscope}%
\begin{pgfscope}%
\pgfpathrectangle{\pgfqpoint{0.781250in}{0.638889in}}{\pgfqpoint{4.843750in}{2.172222in}}%
\pgfusepath{clip}%
\pgfsetbuttcap%
\pgfsetmiterjoin%
\definecolor{currentfill}{rgb}{0.227451,0.192157,0.427451}%
\pgfsetfillcolor{currentfill}%
\pgfsetlinewidth{0.100375pt}%
\definecolor{currentstroke}{rgb}{0.266667,0.266667,0.266667}%
\pgfsetstrokecolor{currentstroke}%
\pgfsetdash{}{0pt}%
\pgfpathmoveto{\pgfqpoint{4.974421in}{0.638889in}}%
\pgfpathlineto{\pgfqpoint{5.000172in}{0.638889in}}%
\pgfpathlineto{\pgfqpoint{5.000172in}{0.777911in}}%
\pgfpathlineto{\pgfqpoint{4.974421in}{0.777911in}}%
\pgfpathlineto{\pgfqpoint{4.974421in}{0.638889in}}%
\pgfpathclose%
\pgfusepath{stroke,fill}%
\end{pgfscope}%
\begin{pgfscope}%
\pgfpathrectangle{\pgfqpoint{0.781250in}{0.638889in}}{\pgfqpoint{4.843750in}{2.172222in}}%
\pgfusepath{clip}%
\pgfsetbuttcap%
\pgfsetmiterjoin%
\definecolor{currentfill}{rgb}{0.227451,0.192157,0.427451}%
\pgfsetfillcolor{currentfill}%
\pgfsetlinewidth{0.100375pt}%
\definecolor{currentstroke}{rgb}{0.266667,0.266667,0.266667}%
\pgfsetstrokecolor{currentstroke}%
\pgfsetdash{}{0pt}%
\pgfpathmoveto{\pgfqpoint{5.011208in}{0.638889in}}%
\pgfpathlineto{\pgfqpoint{5.036959in}{0.638889in}}%
\pgfpathlineto{\pgfqpoint{5.036959in}{0.749362in}}%
\pgfpathlineto{\pgfqpoint{5.011208in}{0.749362in}}%
\pgfpathlineto{\pgfqpoint{5.011208in}{0.638889in}}%
\pgfpathclose%
\pgfusepath{stroke,fill}%
\end{pgfscope}%
\begin{pgfscope}%
\pgfpathrectangle{\pgfqpoint{0.781250in}{0.638889in}}{\pgfqpoint{4.843750in}{2.172222in}}%
\pgfusepath{clip}%
\pgfsetbuttcap%
\pgfsetmiterjoin%
\definecolor{currentfill}{rgb}{0.227451,0.192157,0.427451}%
\pgfsetfillcolor{currentfill}%
\pgfsetlinewidth{0.100375pt}%
\definecolor{currentstroke}{rgb}{0.266667,0.266667,0.266667}%
\pgfsetstrokecolor{currentstroke}%
\pgfsetdash{}{0pt}%
\pgfpathmoveto{\pgfqpoint{5.047995in}{0.638889in}}%
\pgfpathlineto{\pgfqpoint{5.073746in}{0.638889in}}%
\pgfpathlineto{\pgfqpoint{5.073746in}{0.772325in}}%
\pgfpathlineto{\pgfqpoint{5.047995in}{0.772325in}}%
\pgfpathlineto{\pgfqpoint{5.047995in}{0.638889in}}%
\pgfpathclose%
\pgfusepath{stroke,fill}%
\end{pgfscope}%
\begin{pgfscope}%
\pgfpathrectangle{\pgfqpoint{0.781250in}{0.638889in}}{\pgfqpoint{4.843750in}{2.172222in}}%
\pgfusepath{clip}%
\pgfsetbuttcap%
\pgfsetmiterjoin%
\definecolor{currentfill}{rgb}{0.227451,0.192157,0.427451}%
\pgfsetfillcolor{currentfill}%
\pgfsetlinewidth{0.100375pt}%
\definecolor{currentstroke}{rgb}{0.266667,0.266667,0.266667}%
\pgfsetstrokecolor{currentstroke}%
\pgfsetdash{}{0pt}%
\pgfpathmoveto{\pgfqpoint{5.084782in}{0.638889in}}%
\pgfpathlineto{\pgfqpoint{5.110533in}{0.638889in}}%
\pgfpathlineto{\pgfqpoint{5.110533in}{0.833148in}}%
\pgfpathlineto{\pgfqpoint{5.084782in}{0.833148in}}%
\pgfpathlineto{\pgfqpoint{5.084782in}{0.638889in}}%
\pgfpathclose%
\pgfusepath{stroke,fill}%
\end{pgfscope}%
\begin{pgfscope}%
\pgfpathrectangle{\pgfqpoint{0.781250in}{0.638889in}}{\pgfqpoint{4.843750in}{2.172222in}}%
\pgfusepath{clip}%
\pgfsetbuttcap%
\pgfsetmiterjoin%
\definecolor{currentfill}{rgb}{0.227451,0.192157,0.427451}%
\pgfsetfillcolor{currentfill}%
\pgfsetlinewidth{0.100375pt}%
\definecolor{currentstroke}{rgb}{0.266667,0.266667,0.266667}%
\pgfsetstrokecolor{currentstroke}%
\pgfsetdash{}{0pt}%
\pgfpathmoveto{\pgfqpoint{5.121569in}{0.638889in}}%
\pgfpathlineto{\pgfqpoint{5.147320in}{0.638889in}}%
\pgfpathlineto{\pgfqpoint{5.147320in}{0.896142in}}%
\pgfpathlineto{\pgfqpoint{5.121569in}{0.896142in}}%
\pgfpathlineto{\pgfqpoint{5.121569in}{0.638889in}}%
\pgfpathclose%
\pgfusepath{stroke,fill}%
\end{pgfscope}%
\begin{pgfscope}%
\pgfpathrectangle{\pgfqpoint{0.781250in}{0.638889in}}{\pgfqpoint{4.843750in}{2.172222in}}%
\pgfusepath{clip}%
\pgfsetbuttcap%
\pgfsetmiterjoin%
\definecolor{currentfill}{rgb}{0.164706,0.615686,0.560784}%
\pgfsetfillcolor{currentfill}%
\pgfsetlinewidth{0.100375pt}%
\definecolor{currentstroke}{rgb}{0.266667,0.266667,0.266667}%
\pgfsetstrokecolor{currentstroke}%
\pgfsetdash{}{0pt}%
\pgfpathmoveto{\pgfqpoint{5.158356in}{0.638889in}}%
\pgfpathlineto{\pgfqpoint{5.184107in}{0.638889in}}%
\pgfpathlineto{\pgfqpoint{5.184107in}{1.624457in}}%
\pgfpathlineto{\pgfqpoint{5.158356in}{1.624457in}}%
\pgfpathlineto{\pgfqpoint{5.158356in}{0.638889in}}%
\pgfpathclose%
\pgfusepath{stroke,fill}%
\end{pgfscope}%
\begin{pgfscope}%
\pgfpathrectangle{\pgfqpoint{0.781250in}{0.638889in}}{\pgfqpoint{4.843750in}{2.172222in}}%
\pgfusepath{clip}%
\pgfsetbuttcap%
\pgfsetmiterjoin%
\definecolor{currentfill}{rgb}{0.164706,0.615686,0.560784}%
\pgfsetfillcolor{currentfill}%
\pgfsetlinewidth{0.100375pt}%
\definecolor{currentstroke}{rgb}{0.266667,0.266667,0.266667}%
\pgfsetstrokecolor{currentstroke}%
\pgfsetdash{}{0pt}%
\pgfpathmoveto{\pgfqpoint{5.195143in}{0.638889in}}%
\pgfpathlineto{\pgfqpoint{5.220894in}{0.638889in}}%
\pgfpathlineto{\pgfqpoint{5.220894in}{1.287452in}}%
\pgfpathlineto{\pgfqpoint{5.195143in}{1.287452in}}%
\pgfpathlineto{\pgfqpoint{5.195143in}{0.638889in}}%
\pgfpathclose%
\pgfusepath{stroke,fill}%
\end{pgfscope}%
\begin{pgfscope}%
\pgfpathrectangle{\pgfqpoint{0.781250in}{0.638889in}}{\pgfqpoint{4.843750in}{2.172222in}}%
\pgfusepath{clip}%
\pgfsetbuttcap%
\pgfsetmiterjoin%
\definecolor{currentfill}{rgb}{0.164706,0.615686,0.560784}%
\pgfsetfillcolor{currentfill}%
\pgfsetlinewidth{0.100375pt}%
\definecolor{currentstroke}{rgb}{0.266667,0.266667,0.266667}%
\pgfsetstrokecolor{currentstroke}%
\pgfsetdash{}{0pt}%
\pgfpathmoveto{\pgfqpoint{5.231930in}{0.638889in}}%
\pgfpathlineto{\pgfqpoint{5.257681in}{0.638889in}}%
\pgfpathlineto{\pgfqpoint{5.257681in}{1.206770in}}%
\pgfpathlineto{\pgfqpoint{5.231930in}{1.206770in}}%
\pgfpathlineto{\pgfqpoint{5.231930in}{0.638889in}}%
\pgfpathclose%
\pgfusepath{stroke,fill}%
\end{pgfscope}%
\begin{pgfscope}%
\pgfpathrectangle{\pgfqpoint{0.781250in}{0.638889in}}{\pgfqpoint{4.843750in}{2.172222in}}%
\pgfusepath{clip}%
\pgfsetbuttcap%
\pgfsetmiterjoin%
\definecolor{currentfill}{rgb}{0.164706,0.615686,0.560784}%
\pgfsetfillcolor{currentfill}%
\pgfsetlinewidth{0.100375pt}%
\definecolor{currentstroke}{rgb}{0.266667,0.266667,0.266667}%
\pgfsetstrokecolor{currentstroke}%
\pgfsetdash{}{0pt}%
\pgfpathmoveto{\pgfqpoint{5.268717in}{0.638889in}}%
\pgfpathlineto{\pgfqpoint{5.294468in}{0.638889in}}%
\pgfpathlineto{\pgfqpoint{5.294468in}{1.719104in}}%
\pgfpathlineto{\pgfqpoint{5.268717in}{1.719104in}}%
\pgfpathlineto{\pgfqpoint{5.268717in}{0.638889in}}%
\pgfpathclose%
\pgfusepath{stroke,fill}%
\end{pgfscope}%
\begin{pgfscope}%
\pgfpathrectangle{\pgfqpoint{0.781250in}{0.638889in}}{\pgfqpoint{4.843750in}{2.172222in}}%
\pgfusepath{clip}%
\pgfsetbuttcap%
\pgfsetmiterjoin%
\definecolor{currentfill}{rgb}{0.227451,0.192157,0.427451}%
\pgfsetfillcolor{currentfill}%
\pgfsetlinewidth{0.100375pt}%
\definecolor{currentstroke}{rgb}{0.266667,0.266667,0.266667}%
\pgfsetstrokecolor{currentstroke}%
\pgfsetdash{}{0pt}%
\pgfpathmoveto{\pgfqpoint{5.305505in}{0.638889in}}%
\pgfpathlineto{\pgfqpoint{5.331255in}{0.638889in}}%
\pgfpathlineto{\pgfqpoint{5.331255in}{0.920347in}}%
\pgfpathlineto{\pgfqpoint{5.305505in}{0.920347in}}%
\pgfpathlineto{\pgfqpoint{5.305505in}{0.638889in}}%
\pgfpathclose%
\pgfusepath{stroke,fill}%
\end{pgfscope}%
\begin{pgfscope}%
\pgfpathrectangle{\pgfqpoint{0.781250in}{0.638889in}}{\pgfqpoint{4.843750in}{2.172222in}}%
\pgfusepath{clip}%
\pgfsetbuttcap%
\pgfsetmiterjoin%
\definecolor{currentfill}{rgb}{0.227451,0.192157,0.427451}%
\pgfsetfillcolor{currentfill}%
\pgfsetlinewidth{0.100375pt}%
\definecolor{currentstroke}{rgb}{0.266667,0.266667,0.266667}%
\pgfsetstrokecolor{currentstroke}%
\pgfsetdash{}{0pt}%
\pgfpathmoveto{\pgfqpoint{5.342292in}{0.638889in}}%
\pgfpathlineto{\pgfqpoint{5.368043in}{0.638889in}}%
\pgfpathlineto{\pgfqpoint{5.368043in}{0.705607in}}%
\pgfpathlineto{\pgfqpoint{5.342292in}{0.705607in}}%
\pgfpathlineto{\pgfqpoint{5.342292in}{0.638889in}}%
\pgfpathclose%
\pgfusepath{stroke,fill}%
\end{pgfscope}%
\begin{pgfscope}%
\pgfpathrectangle{\pgfqpoint{0.781250in}{0.638889in}}{\pgfqpoint{4.843750in}{2.172222in}}%
\pgfusepath{clip}%
\pgfsetbuttcap%
\pgfsetmiterjoin%
\definecolor{currentfill}{rgb}{0.227451,0.192157,0.427451}%
\pgfsetfillcolor{currentfill}%
\pgfsetlinewidth{0.100375pt}%
\definecolor{currentstroke}{rgb}{0.266667,0.266667,0.266667}%
\pgfsetstrokecolor{currentstroke}%
\pgfsetdash{}{0pt}%
\pgfpathmoveto{\pgfqpoint{5.379079in}{0.638889in}}%
\pgfpathlineto{\pgfqpoint{5.404830in}{0.638889in}}%
\pgfpathlineto{\pgfqpoint{5.404830in}{0.645406in}}%
\pgfpathlineto{\pgfqpoint{5.379079in}{0.645406in}}%
\pgfpathlineto{\pgfqpoint{5.379079in}{0.638889in}}%
\pgfpathclose%
\pgfusepath{stroke,fill}%
\end{pgfscope}%
\begin{pgfscope}%
\definecolor{textcolor}{rgb}{0.333333,0.333333,0.333333}%
\pgfsetstrokecolor{textcolor}%
\pgfsetfillcolor{textcolor}%
\pgftext[x=1.875000in,y=0.319444in,,top]{\color{textcolor}{\ifdefined\pdftexversion\else\setmainfont{NanumMyeongjo}\rmfamily\fi\fontsize{9.000000}{10.800000}\selectfont\catcode`\^=\active\def^{\ifmmode\sp\else\^{}\fi}\catcode`\%=\active\def%{\%}출처: 기상자료개방포털 자료 기반 저자 작성}}%
\end{pgfscope}%
\begin{pgfscope}%
\definecolor{textcolor}{rgb}{0.333333,0.333333,0.333333}%
\pgfsetstrokecolor{textcolor}%
\pgfsetfillcolor{textcolor}%
\pgftext[x=5.312500in,y=2.970833in,,top]{\color{textcolor}{\ifdefined\pdftexversion\else\setmainfont{NanumMyeongjo}\rmfamily\fi\fontsize{9.000000}{10.800000}\selectfont\catcode`\^=\active\def^{\ifmmode\sp\else\^{}\fi}\catcode`\%=\active\def%{\%}(단위: mm)}}%
\end{pgfscope}%
\end{pgfpicture}%
\makeatother%
\endgroup%
}
\end{center}
}

\slide
{\maintitle}
{\chapterthree}
{소결}{
\begin{tcolorbox}[colback=white, colframe=black, boxrule=0.8pt, rounded corners]
\begin{itemize}
    \item 전북은 2020년 이후 콩 생산량이 빠르게 증가였으며, \\특히, 전체 논콩 생산면적의 50\%를 차지고 있음
    \item "콩 자립형 융복합단지 조성 사업"을 통해 \\다른 지자체들도 콩 산업 지원 확대 중임
    \item 논콩 재배시에는 배수 관리에 각별히 신경써야 함
    \item 최근 발생한 초여름과 초가을 장마로 \\논콩 침수 피해가 큰 규모로 발생함
    \item 논콩 생산 농가는 기상이변에 철저히 대비해야하며,\\ 정부 또한 재해보장제도를 완비해야 함
\end{itemize}
\end{tcolorbox}
}

